The notion of {\em essential substitution} arises from the more
general issue of {\em capture-avoiding substitutions} in formal
languages with variable binding. This question is already dealt with
in the computer science literature of which by and large, we are
sadly unaware. It is therefore very difficult for us to give due
credit to the work that has already been done. However, we would
like to mention the papers of Murdoch J. Gabbay~\cite{GabbaySub} and
Murdoch J. Gabbay and Aad Mathijssen~\cite{GabbayCapture}, whose
existence came to our attention, and which deal with {\em
capture-avoiding substitutions}. Our own approach we think is
original, but of course we cannot be sure. The whole question of
{\em capture-avoiding substitutions} arises from the fact that most
substitutions $\sigma:\pv\to{\bf P}(W)$ associated with a map
$\sigma:V\to W$ are not {\em capture-avoiding}. We introduced the
notion of {\em valid substitutions} in
definition~(\ref{logic:def:FOPL:valid:substitution}) so as to
distinguish those cases when $\sigma(\phi)$ is {\em
capture-avoiding} and those when it is not. This was clearly a step
forward but we were still very short of defining a total map
$\sigma:\pv\to{\bf P}(W)$ which would somehow be associated to
$\sigma:V\to W$, while avoiding capture for every $\phi\in\pv$. Our
solution to the problem relies on the introduction of the {\em
minimal transform mapping} ${\cal M}:\pv\to\pvb$ where
$\bar{V}=V\oplus\N$ is the direct sum of $V$ and \N. In considering
the {\em minimal extension} $\bar{V}$ of the set $V$, we are
effectively adding a new set of variables specifically for the
purpose of variable binding, an idea which bears some resemblance
with the introduction of {\em atoms} in the {\em nominal set}
approach of Murdoch J. Gabbay. The {\em minimal transform mapping}
replaces all bound variables from the set $V$ to the set \N. The
benefit of doing so is immediate as given a map $\sigma:V\to W$, the
obvious extension $\bar{\sigma}:\bar{V}\to\bar{W}$ applied to the
minimal transform ${\cal M}(\phi)$ is now {\em capture-avoiding} for
all $\phi$. In other words, the formula $\bar{\sigma}\circ{\cal
M}(\phi)$ is always logically meaningful. However, this formula
belongs to the space ${\bf P}(\bar{W})$ and we are still short of
defining {\em an essential substitution} $\sigma:\pv\to{\bf P}(W)$.
The breakthrough consists in noting that, provided the set $W$ is
infinite or larger than the set $V$, it is always possible to
express the formula $\bar{\sigma}\circ{\cal M}(\phi)$ as the {\em
minimal transform} ${\cal M}(\psi)$ for some formula $\psi\in{\bf
P}(W)$. Using the axiom of choice, an {\em essential substitution}
$\sigma:\pv\to{\bf P}(W)$ associated with the map $\sigma:V\to W$ is
simply defined in terms of ${\cal
M}\circ\sigma=\bar{\sigma}\circ{\cal M}$, and is uniquely
characterized modulo $\alpha$-equivalence, i.e. the substitution
congruence. Of course, anyone interested in computer science and
computability in general will not look favorably upon the use of the
axiom of choice. We are dealing with arbitrary sets $V$ and $W$ of
arbitrary cardinality. In practice, a computer scientist will have
sets of variables which are well-ordered, and it is not difficult to
find some $\psi\in{\bf P}(W)$ such that $\bar{\sigma}\circ{\cal
M}(\phi)={\cal M}(\psi)$ in a way which is deterministic.
