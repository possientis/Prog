In this section we formalize the notion of sub-formula in a free
universal algebra. Specifically, given a free universal algebra $X$
of type $\alpha$, we define a relation $\preceq$ on $X$ such that
$\phi\preceq\psi$ expresses the fact that {\em the formula $\phi$ is
a sub-formula of the formula $\psi$}. For example, when studying our
{\em free universal algebra of first order logic} in a later part of
this document, we shall want to say that the formula $\phi=\forall x
(x\in y)$ is a sub-formula of the formula $\psi=\forall y\forall
x(x\in y)$. This notion will prove important when defining
substitutions of variables which are {\em valid} for a given formula
$\phi$. Loosely speaking, a substitution is valid for $\phi$ if no
free variable gets caught under the {\em scope of a quantifier}.
Making this idea precise requires the notion of sub-formula. In
order to add further motivation to this topic, we shall point out
that {\em valid substitutions} are themselves a key notion when
defining {\em specialization axioms}, which are axioms of the form
$\forall x\phi\to\phi[y/x]$ where the substitution $[y/x]$ of the
variable $y$ in place of $x$ is valid for $\phi$. It is very
important for us to properly define the idea of {\em valid
substitutions}: if we allow too much freedom in which a variable $y$
can be substituted for $x$, our axioms may prove too much including
statements which are not true. On the other hand, if we are too
restrictive our axioms may prove too little and G\"odel's
completeness theorem may fail to be true.

The notion of sub-formula can easily be defined on any free
universal algebra and not just in the context of first order logic.
So we shall write $x\preceq y$ rather than $\phi\preceq\psi$ and
carry out the analysis in its general setting. Given a free
universal algebra $X$ of type $\alpha$ with free generator
$X_{0}\subseteq X$, for all $y\in X$ we shall define the set
$\subf(y)$ of all sub-formulas of $y$, and simply define $x\preceq
y$ as equivalent to $x\in\subf(y)$. The key idea to remember is that
given $f\in\alpha$ and $x\in X^{\alpha(f)}$, the sub-formulas of
$f(x)$ are $f(x)$ itself, as well as the sub-formulas of every
$x(i)$, for all $i\in\alpha(f)$. As shall see, the relation
$\preceq$ is a partial order on $X$, i.e. a relation which is
reflexive, anti-symmetric and transitive on $X$. Recall that
$\preceq$ is said to be anti-symmetric \ifand\ for all $x,y\in X$ we
have:
    \[
    (x\preceq y)\land(y\preceq x)\ \Rightarrow\ (x=y)
    \]
\index{subformula@Sub-formula in free
algebra}\index{subformula@$\subf(x)\,$: set of sub-formulas of
$x$}\index{subformula@$x\preceq y\,$: $x$ is a sub-formula of $y$}
\begin{defin}\label{logic:def:subformula}
Let $X$ be a free universal algebra of type $\alpha$ with free
generator $X_{0}\subseteq X$. We call {\em sub-formula mapping} the
map $\subf:X\to{\cal P}(X)$ defined by:
 \begin{eqnarray*}
    (i)&& x\in X_{0}\ \ \Rightarrow\ \ \subf(x)=\{x\}\\
    (ii)&& x\in X^{\alpha(f)}\ \ \Rightarrow\ \ \subf(f(x)) =
    \{f(x)\}\cup\bigcup_{i\in\alpha(f)}\subf(x(i))
    \end{eqnarray*}
where $(ii)$ holds for all $f\in\alpha$. Given $x,y\in X$, we say
that {\em $x$ is a sub-formula of $y$}, denoted $x\preceq y$,
\ifand\ $x\in\subf(y)$.
\end{defin}

\begin{prop}\label{logic:prop:subformula:recursion}
The structural recursion of definition~(\ref{logic:def:subformula})
is legitimate.
\end{prop}
\begin{proof}
We need to show the existence of a unique map $\subf:X\to{\cal
P}(X)$ such that $(i)$ and $(ii)$ of
definition~(\ref{logic:def:subformula}) hold. We do so by applying
theorem~(\ref{logic:the:structural:recursion:2}) of
page~\pageref{logic:the:structural:recursion:2} with $A={\cal P}(X)$
and $g_{0}:X\to A$ defined by $g_{0}(x)=\{x\}$ for all $x\in X_{0}$.
Given $f\in\alpha$ we take $h(f):A^{\alpha(f)}\times
X^{\alpha(f)}\to A$ defined by:
    \[
    h(f)(a,x)=\{f(x)\}\cup\bigcup_{i\in\alpha(f)}a(i)
    \]
From theorem~(\ref{logic:the:structural:recursion:2}) of
page~\pageref{logic:the:structural:recursion:2} we obtain the
existence and uniqueness of a map $g:X\to A$ which satisfies
$g(x)=\{x\}$ for all $x\in X_{0}$, and for $f\in\alpha$, $x\in
X^{\alpha(f)}$:
    \[
    g(f(x))=h(f)(g(x),x)=\{f(x)\}\cup\bigcup_{i\in\alpha(f)}g(x)(i)
    =\{f(x)\}\cup\bigcup_{i\in\alpha(f)}g(x(i))
    \]
which is exactly $(ii)$ of definition~(\ref{logic:def:subformula})
with $g=\subf$.
\end{proof}
