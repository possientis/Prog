As already pointed out, the equivalence $\vals\circ{\cal
M}(\pi)\sim{\cal M}\circ\vals(\pi)$ is hugely important. The fact
that we are able to obtain it serves as a vindication of both
definition~(\ref{logic:def:FUAP:valuationmod:valuation:modulo}) of
the valuation modulo $\vals:\pvs\to\pv$ and of
definition~(\ref{logic:def:FUAP:mintransproof:transform}) of the
minimal transform ${\cal M}:\pvs\to\pvsb$. After much preliminary
work we are at last in a position to prove it:


\index{clean@Minimal transform of clean proof}\index{minimal@Minimal
transform of clean proof}
\begin{prop}\label{logic:prop:FUAP:mintransproof:valuation:commute}
Let $V$ be a set and $\pi\in\pvs$ be a clean proof. Then the minimal
transform ${\cal M}(\pi)$ is itself a clean proof and furthermore we
have:
    \begin{equation}\label{logic:eqn:FUAP:mintransproof:valuation:commute:1}
    \vals\circ{\cal M}(\pi)\sim{\cal M}\circ\vals(\pi)
    \end{equation}
where $\sim$ denotes the substitution congruence on \pvb.
\end{prop}
\begin{proof}
For every proof $\pi\in\pvs$, we need to show the following
implication:
    \[
    (\,\mbox{$\pi$ clean}\,)\ \Rightarrow(\,\mbox{${\cal M}(\pi)$ clean}\,)
    \land(\,\mbox{eq.~(\ref{logic:eqn:FUAP:mintransproof:valuation:commute:1})}\,)
    \]
We shall do so with a structural induction using
theorem~(\ref{logic:the:proof:induction}) of
page~\pageref{logic:the:proof:induction}. First we assume that
$\pi=\phi$ for some $\phi\in\pv$. From
definition~(\ref{logic:def:FUAP:almostclean:definition}), $\pi$ is
always a clean proof in this case. From
definition~(\ref{logic:def:FUAP:mintransproof:transform}) we have
${\cal M}(\pi)={\cal M}(\phi)$ which is also a clean proof.
Furthermore, from
definition~(\ref{logic:def:FUAP:valuationmod:valuation:modulo}) we
have the equality $\vals\circ{\cal M}(\pi)={\cal M}(\phi)={\cal
M}\circ\vals(\pi)$ which shows in particular that the substitution
equivalence~(\ref{logic:eqn:FUAP:mintransproof:valuation:commute:1})
is true. So we now assume that $\pi=\axi\phi$ for some $\phi\in\pv$.
We need to show that the implication is true for $\pi$. So we assume
that $\pi$ is a clean proof. From
definition~(\ref{logic:def:FUAP:almostclean:definition}) we have
$\phi\in\avs$, i.e. $\phi$ is an axiom modulo. We need to show that
${\cal M}(\pi)$ is a clean proof and the
equivalence~(\ref{logic:eqn:FUAP:mintransproof:valuation:commute:1})
holds. However from
proposition~(\ref{logic:prop:FUAP:valsubaxmodulo:min:transform}),
${\cal M}(\phi)\in{\bf A}^{+}(\bar{V})$ and consequently we see that
${\cal M}(\pi)=\axi{\cal M}(\phi)$ is a clean proof. Furthermore, we
have $\vals\circ{\cal M}(\pi)={\cal M}(\phi)={\cal
M}\circ\vals(\pi)$ and in particular the substitution
equivalence~(\ref{logic:eqn:FUAP:mintransproof:valuation:commute:1})
is true. So we now assume that $\pi=\pi_{1}\pon\pi_{2}$ where
$\pi_{1},\pi_{2}$ are proofs which satisfy our implication. We need
to show the same is true of $\pi$. So we assume that $\pi$ is a
clean proof. We need to show that ${\cal M}(\pi)$ is itself a clean
proof, and furthermore that the
equivalence~(\ref{logic:eqn:FUAP:mintransproof:valuation:commute:1})
holds. However, from
proposition~(\ref{logic:prop:FUAP:almostclean:modus:ponens}) we see
that both $\pi_{1}$ and $\pi_{2}$ are clean and:
    \begin{equation}\label{logic:eqn:FUAP:mintransproof:valuation:commute:2}
    \vals(\pi_{2})=\psi_{1}\to\psi_{2}
    \end{equation}
for some $\psi_{1},\psi_{2}\in\pv$ such that
$\psi_{1}\sim\vals(\pi_{1})$ and $\psi_{2}=\vals(\pi)$, where $\sim$
also denotes the substitution congruence on \pv. Having assumed our
implication is true for $\pi_{1}$ and $\pi_{2}$, it follows that
${\cal M}(\pi_{1})$ and ${\cal M}(\pi_{2})$ are clean and the
equivalence~(\ref{logic:eqn:FUAP:mintransproof:valuation:commute:1})
is true for $\pi_{1}$ and $\pi_{2}$. So let us prove that ${\cal
M}(\pi)$ is a clean proof: since ${\cal M}(\pi)={\cal
M}(\pi_{1})\pon\,{\cal M}(\pi_{2})$, from
proposition~(\ref{logic:prop:FUAP:almostclean:modus:ponens}) it is
sufficient to show that both ${\cal M}(\pi_{1})$ and ${\cal
M}(\pi_{2})$ are clean which we already know, and furthermore that
we have the equality:
    \begin{equation}\label{logic:eqn:FUAP:mintransproof:valuation:commute:3}
    \vals\circ{\cal M}(\pi_{2})=\chi_{1}\to\chi_{2}
    \end{equation}
for some $\chi_{1},\chi_{2}\in\pvb$ such that
$\chi_{1}\sim\vals\circ{\cal M}(\pi_{1})$, in which case we shall
have $\chi_{2}=\vals\circ{\cal M}(\pi)$. However, taking the minimal
transform on both sides of the
equality~(\ref{logic:eqn:FUAP:mintransproof:valuation:commute:2}) we
obtain ${\cal M}\circ\vals(\pi_{2})={\cal M}(\psi_{1})\to{\cal
M}(\psi_{2})$. Having established the fact that the
equivalence~(\ref{logic:eqn:FUAP:mintransproof:valuation:commute:1})
is true for $\pi_{2}$, it follows that $\vals\circ{\cal
M}(\pi_{2})\sim{\cal M}(\psi_{1})\to{\cal M}(\psi_{2})$. Using
theorem~(\ref{logic:the:sub:congruence:charac}) of
page~\pageref{logic:the:sub:congruence:charac} we see that
equation~(\ref{logic:eqn:FUAP:mintransproof:valuation:commute:3}) is
therefore true for some $\chi_{1},\chi_{2}\in\pvb$ such that
$\chi_{1}\sim{\cal M}(\psi_{1})$ and $\chi_{2}\sim{\cal
M}(\psi_{2})$. In order to show that ${\cal M}(\pi)$ is a clean
proof, it remains to prove that ${\cal
M}(\psi_{1})\sim\vals\circ{\cal M}(\pi_{1})$. However, from
$\psi_{1}\sim\vals(\pi_{1})$ and
theorem~(\ref{logic:the:FOPL:mintransfsubcong:kernel}) of
page~\pageref{logic:the:FOPL:mintransfsubcong:kernel} we obtain
${\cal M}(\psi_{1})={\cal M}\circ\vals(\pi_{1})$. So it is
sufficient to prove that ${\cal
M}\circ\vals(\pi_{1})\sim\vals\circ{\cal M}(\pi_{1})$ which follows
from the established fact that the
equivalence~(\ref{logic:eqn:FUAP:mintransproof:valuation:commute:1})
is true for $\pi_{1}$. So we have proved that ${\cal M}(\pi)$ is a
clean proof and it remains to prove the
equivalence~(\ref{logic:eqn:FUAP:mintransproof:valuation:commute:1})
for $\pi$\,:
    \[
    \vals\circ{\cal M}(\pi)=\chi_{2}
    \sim{\cal M}(\psi_{2})
    ={\cal M}\circ\vals(\pi)
    \]
This completes our structural induction in the case when
$\pi=\pi_{1}\pon\pi_{2}$. So we now assume that $\pi=\gen x\pi_{1}$
where $x\in V$ and $\pi_{1}\in\pvs$ is a proof which satisfies our
implication. We need to show the same is true of $\pi$. So we assume
that $\pi$ is a clean proof. We need to show that ${\cal M}(\pi)$ is
itself a clean proof, and furthermore that the
equivalence~(\ref{logic:eqn:FUAP:mintransproof:valuation:commute:1})
is true for $\pi$. However, from
proposition~(\ref{logic:prop:FUAP:almostclean:generalization}) we
see that $\pi_{1}$ is a clean proof and $x\not\in\spec(\pi_{1})$.
Having assumed $\pi_{1}$ satisfies our implication, it follows that
${\cal M}(\pi_{1})$ is clean, and the
equivalence~(\ref{logic:eqn:FUAP:mintransproof:valuation:commute:1})
is true for $\pi_{1}$. So let us prove that ${\cal M}(\pi)$ is a
clean proof: since ${\cal M}(\pi)=\gen n{\cal M}(\pi_{1})[n/x]$,
from proposition~(\ref{logic:prop:FUAP:almostclean:generalization})
it is sufficient to show that ${\cal M}(\pi_{1})[n/x]$ is a clean
proof and furthermore that $n\not\in\spec(\,{\cal
M}(\pi_{1})[n/x]\,)$. First we show that ${\cal M}(\pi_{1})[n/x]$ is
clean: this follows from
proposition~(\ref{logic:prop:FUAP:strongvalsubalmostclean:valuation:commute})
and the fact that ${\cal M}(\pi_{1})$ is a clean proof while $[n/x]$
is valid for ${\cal M}(\pi_{1})$ by virtue of
definition~(\ref{logic:def:FUAP:mintransproof:transform}). We now
show that $n\not\in\spec(\,{\cal M}(\pi_{1})[n/x]\,)$. So suppose to
the contrary that $n\in\spec(\,{\cal M}(\pi_{1})[n/x]\,)$. From
proposition~(\ref{logic:prop:FUAP:freevar:substitution}) we have
$\spec(\,{\cal M}(\pi_{1})[n/x]\,)\subseteq[n/x](\,\spec({\cal
M}(\pi_{1}))\,)$. Furthermore, since $\pi_{1}$ is a clean proof,
from proposition~(\ref{logic:prop:FUAP:mintransproof:variable}) we
have $ \spec({\cal M}(\pi_{1}))=\spec(\pi_{1})$. It follows that
$\spec(\,{\cal M}(\pi_{1})[n/x]\,)\subseteq[n/x](\spec(\pi_{1}))$,
and consequently $n=[n/x](u)$ for some $u\in\spec(\pi_{1})\subseteq
V$. Having established that $x\not\in\spec(\pi_{1})$ we obtain
$u\neq x$ and finally $n=u$. This contradicts the fact $n\in\N$
while $u\in V$ and $V\cap\N=\emptyset$. So we have proved that
${\cal M}(\pi)$ is indeed a clean proof. We now show the
equivalence~(\ref{logic:eqn:FUAP:mintransproof:valuation:commute:1})
is true for $\pi$, which goes as follows:
    \begin{eqnarray*}
    \vals\circ{\cal M}(\pi)&=&\vals\circ{\cal M}(\gen x\pi_{1})\\
    n=\min\{k:[k/x]\ \mbox{valid for}\ {\cal M}(\pi_{1})\}\ \rightarrow
    &=&\vals(\,\gen n{\cal M}(\pi_{1})[n/x]\,)\\
    n\not\in\spec(\,{\cal M}(\pi_{1})[n/x]\,)\ \rightarrow
    &=&\forall n\,\vals\circ[n/x]\circ{\cal M}(\pi_{1})\\
    \mbox{A: to be proved}\ \rightarrow
    &=&\forall n\,[n/x]\circ\vals\circ{\cal M}(\pi_{1})\\
    &=&[n/x](\,\forall x\,\vals\circ{\cal M}(\pi_{1})\,)\\
    \mbox{B: to be proved}\ \rightarrow&\sim&\forall x\,\vals\circ{\cal M}(\pi_{1})\\
    \mbox{(\ref{logic:eqn:FUAP:mintransproof:valuation:commute:1}) true for $\pi_{1}$}\ \rightarrow
    &\sim&\forall x\,{\cal M}\circ\vals(\pi_{1})\\
    \mbox{C: to be proved}\ \rightarrow
    &\sim&[m/x](\,\forall x\,{\cal M}\circ\vals(\pi_{1})\,)\\
    &=&\forall m\,{\cal M}(\vals(\pi_{1}))[m/x]\\
    m=\min\{k:[k/x]\ \mbox{valid for}\ {\cal M}(\vals(\pi_{1}))\}\ \rightarrow
    &=&{\cal M}(\,\forall x\,\vals(\pi_{1})\,)\\
    x\not\in\spec(\pi_{1})\ \rightarrow&=&{\cal M}\circ\vals(\gen x\pi_{1})\\
    &=&{\cal M}\circ\vals(\pi)\\
    \end{eqnarray*}
It remains to justify points A,B and C\,. First we start with point
A, which follows from
proposition~(\ref{logic:prop:FUAP:strongvalsubalmostclean:valuation:commute})
and the fact that ${\cal M}(\pi_{1})$ is a clean proof while $[n/x]$
is valid for ${\cal M}(\pi_{1})$. We shall now establish point B:
using proposition~(\ref{logic:prop:admissible:sub:congruence}), it
is sufficient to show that $[n/x]:\bar{V}\to\bar{V}$ is an
admissible substitution for $\forall x\,\vals\circ{\cal M}(\pi_{1})$
as per definition~(\ref{logic:def:admissible:substitution}). So we
need to show that $[n/x]$ is valid for $\forall x\,\vals\circ{\cal
M}(\pi_{1})$, and furthermore that $[n/x](u)=u$ for all
$u\in\free(\,\forall x\,\vals\circ{\cal M}(\pi_{1})\,)$. This last
equality is clear since any such $u$ would satisfy $u\neq x$. So we
can focus on the validity of $[n/x]$. Using
proposition~(\ref{logic:prop:FOPL:valid:recursion:quant}), it is
sufficient to show that $[n/x]$ is valid for $\vals\circ{\cal
M}(\pi_{1})$ and furthermore, given $u\in\free(\,\forall
x\,\vals\circ{\cal M}(\pi_{1})\,)$ we have $[n/x](u)\neq[n/x](x)$.
This last requirement can be expressed as $u\neq n$ which follows
from $V\cap\N=\emptyset$ provided we show that $u\in V$. However
since~(\ref{logic:eqn:FUAP:mintransproof:valuation:commute:1}) is
true for $\pi_{1}$, from
proposition~(\ref{logic:prop:sub:congruence:freevar}) we have
$\free(\,\vals\circ{\cal M}(\pi_{1})\,)=\free(\,{\cal
M}\circ\vals(\pi_{1})\,)\subseteq V$, where this last inclusion
follows from
proposition~(\ref{logic:prop:FOPL:mintransform:variables}). Hence we
have $u\in V$ as requested. So it remains to show that $[n/x]$ is
valid for $\vals\circ{\cal M}(\pi_{1})$ which follows from the
validity of $[n/x]$ for ${\cal M}(\pi_{1})$ and
proposition~(\ref{logic:prop:FUAP:valuationmod:valid:vals}). It
remains to justify point C\,: similarly to the previous point, it is
sufficient to prove that $[m/x]:\bar{V}\to\bar{V}$ is an admissible
substitution for $\forall x\,{\cal M}\circ\vals(\pi_{1})$. It is
clear that $[m/x](u)=u$ for all $u\in\free(\,\forall x\,{\cal
M}\circ\vals(\pi_{1})\,)$. So we simply need to show that $[m/x]$ is
valid for $\forall x\,{\cal M}\circ\vals(\pi_{1})$. As the integer
$m$ is chosen to be the smallest integer $k$ such that $[k/x]$ is
valid for ${\cal M}\circ\vals(\pi_{1})$, in particular $[m/x]$ is
valid for ${\cal M}\circ\vals(\pi_{1})$. Using
proposition~(\ref{logic:prop:FOPL:valid:recursion:quant}) it is
therefore sufficient to prove that $[m/x](u)\neq[m/x](x)$, which is
$u\neq m$ for all $u\in\free(\,\forall x\,{\cal
M}\circ\vals(\pi_{1})\,)$. This follows from $u\in V$, itself a
consequence of
proposition~(\ref{logic:prop:FOPL:mintransform:variables}).
\end{proof}


We have defined an notion of minimal transform for proofs which
extends the existing notion for formulas. From
proposition~(\ref{logic:prop:FUAP:mintransproof:valuation:commute})
we now know that the conclusion modulo of ${\cal M}(\pi)$ has the
correct shape modulo $\alpha$-equivalence. We also know from
proposition~(\ref{logic:prop:FUAP:mintransproof:hypothesis}) that
the hypothesis of ${\cal M}(\pi)$ does make sense. This is enough
for us to carry over sequents from $\Gamma\vdash\phi$ into ${\cal
M}(\Gamma)\vdash{\cal M}(\phi)$, just as we carried over sequents
from $\Gamma\vdash\phi$ into $\sigma(\Gamma)\vdash\sigma(\phi)$ for
injective maps in
proposition~(\ref{logic:prop:FUAP:validsubtotclean:sequent})
following
proposition~(\ref{logic:prop:FUAP:validsubtotclean:valuation:commute}).
However, the ability to carry over sequents $\Gamma\vdash\phi$ into
their minimal transform counterparts is not hugely interesting. What
would be interesting is to have the equivalence:
     \[
    \Gamma\vdash\phi\ \Leftrightarrow\ {\cal M}(\Gamma)\vdash{\cal
    M}(\phi)
    \]
Such equivalence would allow us to establish provability results
while focussing on minimal transforms, which is arguably a lot
easier. Unfortunately, this result (if true) is out of reach for the
time being. Take $\phi=\forall x\forall y(x \in y)\to\forall x(y\in
x)$ with $V=\{x,y\}$ and $x\neq y$. The only reason $\vdash\phi$ is
true is we made sure it was an axiom, thanks to an essential
substitution of $y$ in place of $x$. By contrast, the sequent
$\vdash{\cal M}(\phi)$ can be proved without {\em essential}
specialization axioms. so if $\vdash{\cal M}(\phi)\ \Rightarrow\
\vdash\phi$ happens to be true, it crucially has to do with our
larger set of axioms made possible by the use of essential
substitutions. It looks deeper.

\begin{prop}\label{logic:prop:FUAP:mintransproof:sequent}
Let $V$ be a set and $\Gamma\subseteq\pv$. Let $\phi\in\pv$. Then we
have:
    \[
    \Gamma\vdash\phi\ \Rightarrow\ {\cal M}(\Gamma)\vdash{\cal
    M}(\phi)
    \]
where ${\cal M}:\pv\to\pvb$ is the minimal transform mapping of {\em
definition~(\ref{logic:def:FOPL:mintransform:transform})}.
\end{prop}
\begin{proof}
We assume that $\Gamma\vdash\phi$. There exists a proof $\pi\in\pvs$
such that $\val(\pi)=\phi$ and $\hyp(\pi)\subseteq\Gamma$. Without
lost of generality, from
proposition~(\ref{logic:prop:FUAP:clean:counterpart}) we may assume
that $\pi$ is a totally clean proof. In particular, from
proposition~(\ref{logic:def:FUAP:almostclean:clean}) $\pi$ is clean.
Furthermore, from
proposition~(\ref{logic:prop:FUAP:valuationmod:clean:proof}) we have
$\vals(\pi)=\val(\pi)$. Applying
proposition~(\ref{logic:prop:FUAP:mintransproof:valuation:commute})
we see that $\vals\circ{\cal M}(\pi)\sim{\cal M}\circ\vals(\pi)$,
where $\sim$ is the substitution congruence on \pvb. It follows that
$\vals\circ{\cal M}(\pi)\sim{\cal M}(\phi)$. Using
theorem~(\ref{logic:the:FUAP:valuationmod:provability}) of
page~\pageref{logic:the:FUAP:valuationmod:provability}, in order to
prove ${\cal M}(\Gamma)\vdash{\cal M}(\phi)$ it is therefore
sufficient to show the inclusion $\hyp({\cal M}(\pi))\precsim{\cal
M}(\Gamma)$ modulo the substitution congruence. In particular, it is
sufficient to prove that $\hyp({\cal M}(\pi))\subseteq{\cal
M}(\Gamma)$. Since $\pi$ is a clean proof, from
proposition~(\ref{logic:prop:FUAP:mintransproof:hypothesis}) it
remains to show that ${\cal M}(\hyp(\pi))\subseteq{\cal M}(\Gamma)$
which follows from $\hyp(\pi)\subseteq\Gamma$ and which completes
our proof.
\end{proof}
