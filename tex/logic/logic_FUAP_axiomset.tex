In this section, we define and establish elementary properties of
the set $\ax(\pi)$ representing the set of all axioms being invoked
in a proof $\pi\in\pvs$. The set $\ax(\pi)$ is casually called the
{\em set of axioms of} $\pi$ just like $\hyp(\pi)$ is the set of
hypothesis of $\pi$. An element of $\ax(\pi)$ is casually called an
{\em axiom of} $\pi$. This terminology is slightly unfortunate since
an {\em axiom of} $\pi$ may not be an axiom at all. For convenience,
we allowed $\axi\phi$ to be meaningful for all $\phi\in\pv$. So the
inclusion $\ax(\pi)\subseteq\av$ does not hold in general, unless
$\pi$ is totally clean. \index{axiom@Axiom set
mapping}\index{a@$\ax(\pi)$ : set of axioms of
$\pi$}\index{axiom@Axioms of a proof}
\begin{defin}\label{logic:def:FUAP:axiomset:axiom:set}
Let $V$ be a set. The map $\ax:\pvs\to{\cal P}(\pv)$ defined by the
following structural recursion is called the {\em axiom set mapping}
on \pvs:
 \begin{equation}\label{logic:eqn:FUAP:axiomset:recursion}
    \forall\pi\in\pvs\ ,\ \ax(\pi)=\left\{
                    \begin{array}{lcl}
                    \ \emptyset&\mbox{\ if\ }&\pi=\phi\in\pv\\
                    \{\phi\}&\mbox{\ if\ }&\pi=\axi\phi\\
                    \ax(\pi_{1})\cup\ax(\pi_{2}) &\mbox{\ if\ }&\pi=\pi_{1}\pon\pi_{2}\\
                    \ax(\pi_{1})&\mbox{\ if\ }&\pi=\gen x\pi_{1}\\
                    \end{array}\right.
    \end{equation}
We say that $\phi\in\pv$ is an axiom of $\pi\in\pvs$ \ifand\
$\phi\in\ax(\pi)$.
\end{defin}
The recursive definition~(\ref{logic:def:FUAP:axiomset:axiom:set})
is easily seen to be legitimate and furthermore $\ax(\pi)$ is a
finite set, as a straightforward structural induction will show.

\begin{prop}\label{logic:prop:FUAP:axiomset:charac}
Let $V$ be a set and $\pi\in\pvs$. Then for all $\phi\in\pv$\,:
    \[
    \phi\in\ax(\pi)\ \Leftrightarrow\
    \axi\phi\preceq\pi
    \]
In other words, $\phi$ is an axiom $\pi$ \ifand\ $\axi\phi$ is a
sub-proof of $\pi$.
\end{prop}
\begin{proof}
Consider the map $\axi:\pv\to\pvs$ defined by $\axi(\phi)=\axi\phi$.
We need to show:
    \[
    \ax(\pi)=\axi^{-1}(\subf(\pi))=\{\phi\in\pv\ :\
    \axi(\phi)\in\subf(\pi)\}
    \]
We shall do so with an induction argument using
theorem~(\ref{logic:the:proof:induction}) of
page~\pageref{logic:the:proof:induction}.  First we assume that
$\pi=\phi$ for some $\phi\in\pv$. Then $\ax(\pi)=\emptyset$ while
$\subf(\pi)=\{\phi\}$. Using
theorem~(\ref{logic:the:unique:representation}) of
page~\pageref{logic:the:unique:representation} we obtain
$\axi^{-1}(\subf(\pi))=\emptyset$. So the equality is true. Next we
assume that $\pi=\axi\phi$ for some $\phi\in\pv$. Then we have
$\ax(\pi)=\{\phi\}$ and from definition~(\ref{logic:def:subformula})
$\subf(\pi)=\{\axi\phi\}$. Since from
proposition~(\ref{logic:prop:FUAP:proof:axi:injective}) $\axi$ is an
injective map, we have $\axi^{-1}(\subf(\pi))=\{\phi\}$ and the
equality is true. Next we assume that $\pi=\pi_{1}\pon\pi_{2}$ where
$\pi_{1},\pi_{2}\in\pvs$ satisfy the equality:
    \begin{eqnarray*}
    \ax(\pi)&=&\ax(\pi_{1}\pon\pi_{2})\\
    &=&\ax(\pi_{1})\cup\ax(\pi_{2})\\
    &=&\axi^{-1}(\subf(\pi_{1}))\,\cup\,\axi^{-1}(\subf(\pi_{2}))\\
    &=&\axi^{-1}(\,\subf(\pi_{1})\cup\subf(\pi_{2})\,)\\
    \mbox{theorem~(\ref{logic:the:unique:representation})
    of p.~\pageref{logic:the:unique:representation}}\ \rightarrow
    &=&\axi^{-1}(\,\{\pi_{1}\pon\pi_{2}\}\cup\subf(\pi_{1})\cup\subf(\pi_{2})\,)\\
    &=&\axi^{-1}(\subf(\pi_{1}\pon\pi_{2}))\\
    &=&\axi^{-1}(\subf(\pi))
    \end{eqnarray*}
Finally we assume that $\pi=\gen x\pi_{1}$ where $x\in V$ and
$\pi_{1}$ satisfies the equality:
    \begin{eqnarray*}
    \ax(\pi)&=&\ax(\gen x\pi_{1})\\
    &=&\ax(\pi_{1})\\
    &=&\axi^{-1}(\subf(\pi_{1}))\\
    \mbox{theorem~(\ref{logic:the:unique:representation})
    of p.~\pageref{logic:the:unique:representation}}\ \rightarrow
    &=&\axi^{-1}(\,\{\gen x\pi_{1}\}\cup\subf(\pi_{1})\,)\\
    &=&\axi^{-1}(\subf(\gen x\pi_{1}))\\
    &=&\axi^{-1}(\subf(\pi))
    \end{eqnarray*}
\end{proof}

The axioms of a totally clean proof are indeed axioms of first order
logic.
\begin{prop}\label{logic:prop:FUAP:axiomset:clean:proof}
Let $V$ be a set and $\pi\in\pvs$. Then we have the implication:
    \[
    (\mbox{$\pi$ totally clean})\ \Rightarrow\ \ax(\pi)\subseteq\av
    \]
\end{prop}
\begin{proof}
Let $\pi\in\pvs$ be totally clean. We need to show that
$\ax(\pi)\subseteq\av$. So let $\phi\in\ax(\pi)$. We need to show
that $\phi\in\av$. However from
proposition~(\ref{logic:prop:FUAP:axiomset:charac}) we obtain
$\axi\phi\preceq\pi$. Since $\pi$ is totally clean, it follows from
proposition~(\ref{logic:prop:FUAP:clean:sub:proof}) that $\axi\phi$
is also totally clean. From
definition~(\ref{logic:def:FUAP:clean:clean:proof}) we see that
$\phi\in\av$.
\end{proof}

The map $\ax:\pvs\to{\cal P}(\pv)$ defined on the free universal
algebra \pvs\ is increasing with respect to the inclusion partial
order on ${\cal P}(\pv)$.

\begin{prop}\label{logic:prop:FUAP:axiomset:subformula}
Let $V$ be a set and $\rho,\pi\in\pvs$. Then we have:
    \[
    \rho\preceq\pi\ \Rightarrow\ \ax(\rho)\subseteq\ax(\pi)
    \]
\end{prop}
\begin{proof}
This follows from an application of
proposition~(\ref{logic:prop:UA:subformula:non:decreasing}) to
$\ax:X\to A$ where $X=\pvs$ and $A={\cal P}(\pv)$ where the preorder
$\leq$ on $A$ is the usual inclusion $\subseteq$. We simply need to
check that given $\pi_{1},\pi_{2}\in\pvs$ and $x\in V$ we have the
inclusions $\ax(\pi_{1})\subseteq\ax(\pi_{1}\pon\pi_{2})$,
$\ax(\pi_{2})\subseteq\ax(\pi_{1}\pon\pi_{2})$ and
$\ax(\pi_{1})\subseteq\ax(\gen x\pi_{1})$ which follow from the
recursive definition~(\ref{logic:def:FUAP:axiomset:axiom:set}).
\end{proof}

Given a map $\sigma:V\to W$ and a proof $\pi\in\pvs$, the axioms of
the proof $\sigma(\pi)$ are the images by $\sigma:\pv\to{\bf P}(W)$
of the axioms of $\pi$. Note that the symbol '$\sigma$' is
overloaded and refers to three possible maps: apart from
$\sigma:V\to W$, there is $\sigma:\pvs\to{\bf\Pi}(W)$ when referring
to $\sigma(\pi)$. There is also $\sigma:\pv\to{\bf P}(W)$ whose
restriction to $\ax(\pi)$ has a range denoted $\sigma(\ax(\pi))$.
Hence we have:

\begin{prop}\label{logic:prop:FUAP:axiomset:substitution}
Let $V, W$ be sets and $\sigma:V\to W$ be a map. Let $\pi\in\pvs$\,:
    \[
    \ax(\sigma(\pi))=\sigma(\ax(\pi))
    \]
where $\sigma:\pvs\to{\bf\Pi}(W)$ also denotes the proof
substitution mapping.
\end{prop}
\begin{proof}
We shall prove this equality with an induction argument, using
theorem~(\ref{logic:the:proof:induction}) of
page~\pageref{logic:the:proof:induction}. First we assume that
$\pi=\phi$ for some $\phi\in\pv$. Then $\sigma(\pi)=\sigma(\phi)$
and consequently $\ax(\sigma(\pi))=\emptyset$. So the equality is
clear. Next we assume that $\pi=\axi\phi$ for some $\phi\in\pv$.
Then we have the equalities:
    \begin{eqnarray*}
    \ax(\sigma(\pi))&=&\ax(\sigma(\axi\phi))\\
    &=&\ax(\axi\sigma(\phi))\\
    &=&\{\sigma(\phi)\}\\
    &=&\sigma(\{\phi\})\\
    &=&\sigma(\ax(\axi\phi))\\
    &=&\sigma(\ax(\pi))
    \end{eqnarray*}
Next we assume that $\pi=\pi_{1}\pon\pi_{2}$ where
$\pi_{1},\pi_{2}\in\pvs$ are proofs for which the equality is true.
We need to show the same is true of $\pi$:
    \begin{eqnarray*}
    \ax(\sigma(\pi))&=&\ax(\sigma(\pi_{1}\pon\pi_{2}))\\
    &=&\ax(\,\sigma(\pi_{1})\pon\,\sigma(\pi_{2})\,)\\
    &=&\ax(\sigma(\pi_{1}))\cup\ax(\sigma(\pi_{2}))\\
    &=&\sigma(\ax(\pi_{1}))\cup\sigma(\ax(\pi_{2}))\\
    &=&\sigma(\,\ax(\pi_{1})\cup\ax(\pi_{2})\,)\\
    &=&\sigma(\ax(\pi_{1}\pon\pi_{2}))\\
    &=&\sigma(\ax(\pi))\\
    \end{eqnarray*}
Finally, we assume that $\pi=\gen x\pi_{1}$ where $x\in V$ and
$\pi_{1}\in\pvs$ is a proof for which the equality is true. We need
to show the same is true of $\pi$:
    \begin{eqnarray*}
    \ax(\sigma(\pi))&=&\ax(\sigma(\gen x\pi_{1}))\\
    &=&\ax(\,\gen\sigma(x)\sigma(\pi_{1})\,)\\
    &=&\ax(\sigma(\pi_{1}))\\
    &=&\sigma(\ax(\pi_{1}))\\
    &=&\sigma(\ax(\gen x\pi_{1}))\\
    &=&\sigma(\ax(\pi))
    \end{eqnarray*}
\end{proof}
