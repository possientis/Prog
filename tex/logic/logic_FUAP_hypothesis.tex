Given a proof $\pi\in\pvs$, the set of hypothesis $\hyp(\pi)$ was
defined in an earlier section as per
definition~(\ref{logic:def:FOPL:hypothesis}). At the time, a simple
definition was enough so we could define the set of specific
variables $\spec(\pi)$ which in turn allowed us to speak of the
valuation mapping $\val:\pvs\to\pv$ and correspondingly establish
the notion of provability and sequent $\Gamma\vdash\phi$, as per
definition~(\ref{logic:def:FOPL:proof:of:formula}). It is now time
to be a little bit more thorough and state some of the elementary
properties satisfied by the set of hypothesis $\hyp(\pi)$. We start
with the fact that $\hyp(\pi)$ is simply the set of formulas
$\phi\in\pv$ which are sub-proofs of $\pi$:
\index{hypothesis@Hypothesis of proof}
\begin{prop}\label{logic:prop:FUAP:hypothesis:charac}
Let $V$ be a set and $\pi\in\pvs$. Then for all $\phi\in\pv$:
    \[
    \phi\in\hyp(\pi)\ \Leftrightarrow\ \phi\preceq\pi
    \]
In other words, $\phi$ is a hypothesis of $\pi$ \ifand\ $\phi$ is a
sub-proof of $\pi$.
\end{prop}
\begin{proof}
We need to show the equality $\hyp(\pi)=\subf(\pi)\cap\pv$. We shall
do so with a structural induction argument using
theorem~(\ref{logic:the:proof:induction}) of
page~\pageref{logic:the:proof:induction}. First we assume that
$\pi=\phi$ for some $\phi\in\pv$. Then we have $\hyp(\pi)=\{\phi\}$
and from definition~(\ref{logic:def:subformula})
$\subf(\pi)=\{\phi\}$. So the equality is clear. Next we assume that
$\pi=\axi\phi$ for some $\phi\in\pv$. Then $\hyp(\pi)=\emptyset$
while $\subf(\pi)=\{\axi\phi\}$. Using
theorem~(\ref{logic:the:unique:representation}) of
page~\pageref{logic:the:unique:representation} we obtain
$\subf(\pi)\cap\pv=\emptyset$. So the equality is true. Next we
assume that $\pi=\pi_{1}\pon\pi_{2}$ where $\pi_{1},\pi_{2}\in\pvs$
satisfy the equality. Then:
    \begin{eqnarray*}
    \hyp(\pi)&=&\hyp(\pi_{1}\pon\pi_{2})\\
    &=&\hyp(\pi_{1})\cup\hyp(\pi_{2})\\
    &=&(\subf(\pi_{1})\cap\pv)\,\cup\,(\subf(\pi_{2})\cap\pv)\\
    &=&(\,\subf(\pi_{1})\cup\subf(\pi_{2})\,)\,\cap\,\pv\\
    \mbox{theorem~(\ref{logic:the:unique:representation})
    of p.~\pageref{logic:the:unique:representation}}\ \rightarrow
    &=&(\,\{\pi_{1}\pon\pi_{2}\}\cup\subf(\pi_{1})\cup\subf(\pi_{2})\,)\,\cap\,\pv\\
    &=&\subf(\pi_{1}\pon\pi_{2})\cap\pv\\
    &=&\subf(\pi)\cap\pv
    \end{eqnarray*}
Finally we assume that $\pi=\gen x\pi_{1}$ where $x\in V$ and
$\pi_{1}$ satisfies the equality:
    \begin{eqnarray*}
    \hyp(\pi)&=&\hyp(\gen x\pi_{1})\\
    &=&\hyp(\pi_{1})\\
    &=&\subf(\pi_{1})\cap\pv\\
    \mbox{theorem~(\ref{logic:the:unique:representation})
    of p.~\pageref{logic:the:unique:representation}}\ \rightarrow
    &=&(\,\{\gen x\pi_{1}\}\cup\subf(\pi_{1})\,)\cap\pv\\
    &=&\subf(\gen x\pi_{1})\cap\pv\\
    &=&\subf(\pi)\cap\pv
    \end{eqnarray*}
\end{proof}

The map $\hyp:\pvs\to{\cal P}(\pv)$ defined on the free universal
algebra \pvs\ is increasing with respect to the inclusion partial
order on ${\cal P}(\pv)$.
\begin{prop}\label{logic:prop:FUAP:hypothesis:subformula}
Let $V$ be a set and $\rho,\pi\in\pvs$. Then we have:
    \[
    \rho\preceq\pi\ \Rightarrow\ \hyp(\rho)\subseteq\hyp(\pi)
    \]
\end{prop}
\begin{proof}
This follows from an application of
proposition~(\ref{logic:prop:UA:subformula:non:decreasing}) to
$\hyp:X\to A$ where $X=\pvs$ and $A={\cal P}(\pv)$ where the
preorder $\leq$ on $A$ is the usual inclusion $\subseteq$. We simply
need to check that given $\pi_{1},\pi_{2}\in\pvs$ and $x\in V$ we
have the inclusions
$\hyp(\pi_{1})\subseteq\hyp(\pi_{1}\pon\pi_{2})$,
$\hyp(\pi_{2})\subseteq\hyp(\pi_{1}\pon\pi_{2})$ and
$\hyp(\pi_{1})\subseteq\hyp(\gen x\pi_{1})$ which follow from the
recursive definition~(\ref{logic:def:FOPL:hypothesis}).
\end{proof}

Given a map $\sigma:V\to W$ and a proof $\pi\in\pvs$, the hypothesis
of the proof $\sigma(\pi)$ are the images by $\sigma:\pv\to{\bf
P}(W)$ of the hypothesis of $\pi$. Note that the symbol '$\sigma$'
is overloaded and refers to three possible maps: apart from
$\sigma:V\to W$, there is $\sigma:\pvs\to{\bf\Pi}(W)$ when referring
to $\sigma(\pi)$. There is also $\sigma:\pv\to{\bf P}(W)$ whose
restriction to $\hyp(\pi)$ has a range denoted $\sigma(\hyp(\pi))$.
\begin{prop}\label{logic:prop:FUAP:substitution:hypothesis}
Let $V, W$ be sets and $\sigma:V\to W$ be a map. Let $\pi\in\pvs$\,:
    \[
    \hyp(\sigma(\pi))=\sigma(\hyp(\pi))
    \]
where $\sigma:\pvs\to{\bf\Pi}(W)$ also denotes the proof
substitution mapping.
\end{prop}
\begin{proof}
We shall proof this equality with an induction argument, using
theorem~(\ref{logic:the:proof:induction}) of
page~\pageref{logic:the:proof:induction}. First we assume that
$\pi=\phi$ for some $\phi\in\pv$. Then we have:
    \begin{eqnarray*}
    \hyp(\sigma(\pi))&=&\hyp(\sigma(\phi))\\
    \mbox{def.~(\ref{logic:def:FOPL:hypothesis})}\ \rightarrow
    &=&\{\sigma(\phi)\}\\
    &=&\sigma(\{\phi\})\\
    &=&\sigma(\hyp(\pi))
    \end{eqnarray*}
So we now assume that $\pi=\axi\phi$ for some $\phi\in\pv$. Then
$\hyp(\pi)=\emptyset$. However from
definition~(\ref{logic:def:FUAP:substitution:substitution}),
$\sigma(\pi)=\axi\sigma(\phi)$ and  we have
$\hyp(\sigma(\pi))=\emptyset$. So the equality
$\hyp(\sigma(\pi))=\sigma(\hyp(\pi))$ holds. So we now assume that
$\pi=\pi_{1}\pon\pi_{2}$ where $\pi_{1},\pi_{2}\in\pvs$ are proofs
for which the equality is true. Then:
    \begin{eqnarray*}
    \hyp(\sigma(\pi))&=&\hyp(\sigma(\pi_{1}\pon\pi_{2}))\\
    &=&\hyp(\,\sigma(\pi_{1})\pon\,\sigma(\pi_{2})\,)\\
    &=&\hyp(\sigma(\pi_{1}))\cup\hyp(\sigma(\pi_{2}))\\
    &=&\sigma(\hyp(\pi_{1}))\cup\sigma(\hyp(\pi_{2}))\\
    &=&\sigma(\,\hyp(\pi_{1})\cup\hyp(\pi_{2})\,)\\
    &=&\sigma(\hyp(\pi_{1}\pon\pi_{2}))\\
    &=&\sigma(\hyp(\pi))\\
    \end{eqnarray*}
Finally, we assume that $\pi=\gen x\pi_{1}$ where $x\in V$ and
$\pi_{1}\in\pvs$ is a proof for which the equality is true. We need
to show the same is true of $\pi$:
    \begin{eqnarray*}
    \hyp(\sigma(\pi))&=&\hyp(\sigma(\gen x\pi_{1}))\\
    &=&\hyp(\,\gen\sigma(x)\sigma(\pi_{1})\,)\\
    &=&\hyp(\sigma(\pi_{1}))\\
    &=&\sigma(\hyp(\pi_{1}))\\
    &=&\sigma(\hyp(\gen x\pi_{1}))\\
    &=&\sigma(\hyp(\pi))
    \end{eqnarray*}
\end{proof}
