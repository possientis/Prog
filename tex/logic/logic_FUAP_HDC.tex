Having created a sensible consequence relation $\vdash\,\subseteq
{\cal P}(\pv)\times\pv$ we can now formally define the Hilbert
deductive congruence on \pv. Given $\phi,\psi\in\pv$, we shall say
that $\phi$ and $\psi$ are equivalent \ifand\ both formulas
$\phi\to\psi$ and $\psi\to\phi$ are provable. So we naturally start
with a study of the relation $\leq$ defined by $\phi\leq\psi\
\Leftrightarrow\ \vdash(\phi\to\psi)$ which shall be seen to be a
preorder: \index{preorder@Hilbert deductive preorder}
\begin{defin}\label{logic:def:FOPL:HDC:quasiorder}
Let $V$ be a set. We call {\em Hilbert deductive preorder} on \pv\
the relation $\leq$ defined by $\phi\leq\psi$ \ifand\
$\vdash(\phi\to\psi)$, for all $\phi,\psi\in\pv$.
\end{defin}
Recall that a preorder is a binary relation which is reflexive and
transitive. The proof that $\leq$ is indeed a preorder follows. Note
that by virtue the deduction
theorem~(\ref{logic:the:FOPL:deduction}) of
page~\pageref{logic:the:FOPL:deduction}, the statement
$\vdash(\phi\to\psi)$ is equivalent to $\{\phi\}\vdash\psi$.
\begin{prop}\label{logic:prop:FOPL:quasiorder}
Let $V$ be a set. Then, the Hilbert deductive preorder $\leq$ on
\pv\ is a reflexive and transitive relation on \pv.
\end{prop}
\begin{proof}
First we show that $\leq$ is reflexive, namely that $\phi\leq\phi$
for all $\phi\in\pv$. We have to show that $\vdash (\phi\to\phi)$,
which follows from proposition~(\ref{logic:prop:FOPL:PimP}). Next we
show that $\leq$ is transitive. So let $\phi,\psi,\chi\in\pv$ such
that $\phi\leq\psi$ and $\psi\leq\chi$. We need to show that
$\phi\leq\chi$, that is $\vdash(\phi\to\chi)$. Using the deduction
theorem~(\ref{logic:the:FOPL:deduction}), it is sufficient to prove
that $\{\phi\}\vdash\chi$. However, from the assumption
$\phi\leq\psi$ we have $\vdash(\phi\to\psi)$ and consequently
$\{\phi\}\vdash\psi$. From the assumption $\psi\leq\chi$ we obtain
$\vdash(\psi\to\chi)$ and in particular
$\{\phi\}\vdash(\psi\to\chi)$. Using the modus ponens property of
proposition~(\ref{logic:prop:FOPL:modus:ponens}) we conclude that
$\{\phi\}\vdash\chi$ as requested.
\end{proof}

The Hilbert deductive preorder is not a congruent relation. If
$\phi_{1}\leq\psi_{1}$ and $\phi_{2}\leq\psi_{2}$, we cannot argue
that $\phi_{1}\to\phi_{2}\leq\psi_{1}\to\psi_{2}$. Although it would
be easy to provide a counterexample, we shall refrain from doing so
at this stage, as we are not able to prove anything: we do have some
tools allowing us to establish a given sequent
$\vdash(\phi\to\psi)$, but we are not yet equipped to refute such a
sequent. Proving that a formula is not provable is difficult. We
shall need some model theory to do this. For now, we shall focus on
the positive result:

\begin{prop}\label{logic:prop:FOPL:HDC:quasi:imp}
Let $V$ be a set and $\leq$ be the Hilbert deductive preorder on
\pv. Let $\phi_{1},\phi_{2}$ and $\psi_{1},\psi_{2}\in\pv$ such that
$\psi_{1}\leq\phi_{1}$ and $\phi_{2}\leq\psi_{2}$. Then:
    \[
    \phi_{1}\to\phi_{2}\leq\psi_{1}\to\psi_{2}
    \]
\end{prop}
\begin{proof}
We assume that $\psi_{1}\leq\phi_{1}$ and $\phi_{2}\leq\psi_{2}$,
i.e. $\vdash(\psi_{1}\to\phi_{1})$ and
$\vdash(\phi_{2}\to\psi_{2})$. We need to show that
$\phi_{1}\to\phi_{2}\leq\psi_{1}\to\psi_{2}$ i.e.
$\vdash(\phi_{1}\to\phi_{2})\to(\psi_{1}\to\psi_{2})$. Using the
deduction theorem~(\ref{logic:the:FOPL:deduction}) of
page~\pageref{logic:the:FOPL:deduction} it is sufficient to prove
that $\{\phi_{1}\to\phi_{2}\}\vdash(\psi_{1}\to\psi_{2})$. In fact,
using theorem~(\ref{logic:the:FOPL:deduction}) once more, it is
sufficient to prove that $\Gamma\vdash\psi_{2}$ where
$\Gamma=\{\,\phi_{1}\to\phi_{2}\,,\,\psi_{1}\,\}$. From the
assumption $\vdash(\psi_{1}\to\phi_{1})$ we obtain
$\{\psi_{1}\}\vdash\phi_{1}$ and consequently
$\Gamma\vdash\phi_{1}$. Furthermore, it is clear that
$\Gamma\vdash(\phi_{1}\to\phi_{2})$. Using the modus ponens property
of proposition~(\ref{logic:prop:FOPL:modus:ponens}) we obtain
$\Gamma\vdash\phi_{2}$. However, from the assumption
$\vdash(\phi_{2}\to\psi_{2})$ we have in particular
$\Gamma\vdash(\phi_{2}\to\psi_{2})$. Using the modus ponens property
of proposition~(\ref{logic:prop:FOPL:modus:ponens}) once more we
obtain $\Gamma\vdash\psi_{2}$ as requested.
\end{proof}

As we have just seen, the implication operator $\to$ formally
behaves like a {\em difference} $\phi_{2}-\phi_{1}$ when it comes to
the Hilbert deductive preorder. By contrast, the quantification
operator $\forall x$ respects the order:

\begin{prop}\label{logic:prop:FOPL:HDC:quasi:quant}
Let $V$ be a set and $\leq$ be the Hilbert deductive preorder on
\pv. Let $\phi_{1},\psi_{1}\in\pv$ such that $\phi_{1}\leq\psi_{1}$.
Then for all $x\in V$ we have:
    \[
    \forall x\phi_{1}\leq\forall x\psi_{1}
    \]
\end{prop}
\begin{proof}
Suppose $\phi_{1}\leq\psi_{1}$ and let $x\in V$. we need to show
that $\forall x\phi_{1}\leq\forall x\psi_{1}$ which is
$\vdash(\forall x\phi_{1}\to\forall x\psi_{1})$. Using the deduction
theorem~(\ref{logic:the:FOPL:deduction}) of
page~\pageref{logic:the:FOPL:deduction} it is sufficient to prove
that $\Gamma\vdash\forall x\psi_{1}$ with $\Gamma=\{\forall
x\phi_{1}\}$. It is clear that $\Gamma\vdash\forall x\phi_{1}$. From
proposition~(\ref{logic:prop:FOPL:esssubstprop:injective}) the
identity mapping $i:\pv\to\pv$ is an essential substitution
associated with the identity $i:V\to V$. In particular, it is an
{\em essential substitution of $y$ in place of $x$} in the case when
$y=x$. Using the specialization property of
proposition~(\ref{logic:prop:FOPL:specialization}) it follows that
$\Gamma\vdash\phi_{1}$. However, the assumption
$\phi_{1}\leq\psi_{1}$ leads to $\vdash(\phi_{1}\to\psi_{1})$ and in
particular $\Gamma\vdash(\phi_{1}\to\psi_{1})$. Using the modus
ponens property of proposition~(\ref{logic:prop:FOPL:modus:ponens})
we obtain $\Gamma\vdash\psi_{1}$. Since $x\not\in\free(\Gamma)$ we
conclude $\Gamma\vdash\forall x\psi_{1}$ from the generalization
property of proposition~(\ref{logic:prop:FOPL:generalization}).
\end{proof}

We shall now define the Hilbert deductive congruence in terms of the
Hilbert deductive preorder, and prove it is indeed a congruence on
\pv. \index{congruence@Hilbert deductive
congruence}\index{Lindenbaum@Lindenbaum-Tarski congruence}
\begin{defin}\label{logic:def:FOPL:HDC:HDC}
Let $V$ be a set and $\leq$ be the Hilbert deductive preorder on
\pv. We call {\em Hilbert deductive congruence} on \pv\ the relation
$\equiv$ defined~by:
    \[
    \phi\equiv\psi\ \ \Leftrightarrow\ \
    (\phi\leq\psi)\,\land\,(\psi\leq\phi)
    \]
\end{defin}
\begin{prop}\label{logic:prop:FOPL:HDC:congruence}
The Hilbert deductive congruence on \pv\ is a congruence.
\end{prop}
\begin{proof}
Let $\equiv$ denote the Hilbert deductive congruence on \pv. We need
to prove that $\equiv$ is an equivalence relation which is also a
congruent relation. First we show that it is indeed an equivalence
relation. Let $\leq$ denote the Hilbert deductive preorder on \pv.
From proposition~(\ref{logic:prop:FOPL:quasiorder}), $\leq$ is
reflexive. So it is clear that $\equiv$ is also reflexive. It is
also obvious that $\equiv$ is symmetric. So it remains to show that
$\equiv$ is transitive. So suppose $\phi,\psi,\chi\in\pv$ are such
that $\phi\equiv\psi$ and $\psi\equiv\chi$. In particular we have
$\phi\leq\psi$ and $\psi\leq\chi$ and it follows from the
transitivity of $\leq$ that $\phi\leq\chi$. We show similarly that
$\chi\leq\phi$ and we conclude that $\phi\equiv\chi$. So it remains
to show that $\equiv$ is a congruent relation on \pv. We have
already proved that $\bot\equiv\bot$ from reflexivity. Suppose
$\phi_{1}\equiv\psi_{1}$ and $\phi_{2}\equiv\psi_{2}$. We need to
show that $\phi_{1}\to\phi_{2}\equiv\psi_{1}\to\psi_{2}$. First we
show that $\phi_{1}\to\phi_{2}\leq\psi_{1}\to\psi_{2}$. This follows
from proposition~(\ref{logic:prop:FOPL:HDC:quasi:imp}) and the fact
that $\psi_{1}\leq\phi_{1}$ and $\phi_{2}\leq\psi_{2}$. Using
$\phi_{1}\leq\psi_{1}$ and $\psi_{2}\leq\phi_{2}$ we prove similarly
that $\psi_{1}\to\psi_{2}\leq\phi_{1}\to\phi_{2}$. Suppose now that
$\phi_{1}\equiv\psi_{1}$ and $x\in V$. We need to show that $\forall
x\phi_{1}\equiv\forall x\psi_{1}$. However, from
$\phi_{1}\leq\psi_{1}$ and
proposition~(\ref{logic:prop:FOPL:HDC:quasi:quant}) we obtain
$\forall x\phi_{1}\leq\forall x\psi_{1}$. Likewise from
$\psi_{1}\leq\phi_{1}$ we have $\forall x\psi_{1}\leq\forall
x\phi_{1}$. The equivalence $\forall x\phi_{1}\equiv\forall
x\psi_{1}$ follows.
\end{proof}

The Hilbert deductive congruence expresses the idea of {\em logical
equivalence} between formulas. We are already familiar with other
congruences which formalize the idea of {\em identical meaning}.
Obviously we should expect formulas which have {\em identical
meaning} to be {\em logically equivalent}. We shall now check this
is indeed the case, starting with the substitution congruence on
\pv\,:

\begin{prop}\label{logic:prop:FOPL:HDC:substitution:v:HDC}
Let $V$ be a set. Let $\sim$ and $\equiv$ denote the substitution
and Hilbert deductive congruence on \pv\ respectively. Then for all
$\phi,\psi\in\pv$:
    \[
    \phi\sim\psi\ \Rightarrow\ \phi\equiv\psi
    \]
The substitution congruence is stronger than the Hilbert deductive
congruence.
\end{prop}
\begin{proof}
We need to show the inclusion $\sim\,\subseteq\,\equiv$, for which
it is sufficient to show $R_{0}\subseteq\,\equiv$ where $R_{0}$ is a
generator of the substitution congruence. Using
definition~(\ref{logic:def:sub:congruence}) it is therefore
sufficient to prove that $\phi\equiv\psi$ where $\phi=\forall
x\phi_{1}$ for some $x\in V$ and $\phi_{1}\in\pv$, and $\psi=\forall
y\phi_{1}[y\!:\!x]$ with $x\neq y$ and $y\not\in\free(\phi_{1})$.
First we show that $\phi\leq\psi$ where $\leq$ is the Hilbert
preorder on \pv. So we need to show that $\vdash(\phi\to\psi)$.
Using the deduction theorem~(\ref{logic:the:FOPL:deduction}) of
page~\pageref{logic:the:FOPL:deduction} it is sufficient to prove
that $\{\phi\}\vdash\psi$, which is $\Gamma\vdash\forall
y\phi_{1}[y\!:\!x]$ where $\Gamma=\{\forall x\phi_{1}\}$. However,
having assumed that $y\not\in\free(\phi_{1})$, in particular we have
$y\not\in\free(\Gamma)$. Using the generalization property of
proposition~(\ref{logic:prop:FOPL:generalization}) it is therefore
sufficient to prove that $\Gamma\vdash\phi_{1}[y\!:\!x]$. Let us
accept for now that the formula $\chi=\forall
x\phi_{1}\to\phi_{1}[y\!:\!x]$ is a specialization axiom. Then from
proposition~(\ref{logic:prop:FOPL:axiom}) we have $\vdash\chi$ and
using the deduction theorem once more we obtain
$\Gamma\vdash\phi_{1}[y\!:\!x]$ as requested. So it remains to show
that $\chi$ is indeed a specialization axiom. From
proposition~(\ref{logic:prop:FOPL:specialization:axiom:2}) it is
sufficient to show that $\phi_{1}[y\!:\!x]\sim\phi_{1}[y/x]$ where
$[y/x]:\pv\to\pv$ is an essential substitution of $y$ in place of
$x$. Since the permutation $[y\!:\!x]:V\to V$ is injective, from
proposition~(\ref{logic:prop:FOPL:esssubstprop:injective}) its
associated map $[y\!:\!x]:\pv\to\pv$ is also an essential
substitution. Hence, in order to show the equivalence
$\phi_{1}[y\!:\!x]\sim\phi_{1}[y/x]$, from
proposition~(\ref{logic:prop:FOPL:esssubstprop:support}) it is
sufficient to show that $[y/x]$ and $[y\!:\!x]$ coincide on
$\free(\phi_{1})$. This is clearly the case since
$y\not\in\free(\phi_{1})$. So we have proved that $\phi\leq\psi$ and
it remains to show that $\psi\leq\phi$. However, define
$\phi_{1}^{*}=\phi_{1}[y\!:\!x]$. Then it is clear that
$\phi_{1}=\phi_{1}^{*}[x\!:\!y]$ and we need to show that $\forall
y\phi_{1}^{*}\leq\forall x \phi_{1}^{*}[x\!:\!y]$, for which we can
use an identical proof as previously, provided we show that
$x\not\in\free(\phi_{1}^{*})$. So suppose to the contrary that
$x\in\free(\phi_{1}^{*})$. Using
proposition~(\ref{logic:prop:freevar:of:substitution}), since
$[y\!:\!x]$ is injective we have
$\free(\phi_{1}^{*})=[x\!:\!y](\free(\phi_{1}))$. It follows that
there exists $u\in\free(\phi_{1})$ such that $x=[x\!:\!y](u)$. Hence
$u=y$ which contradicts $y\not\in\free(\phi_{1})$.
\end{proof}

Two formulas which are permutation equivalent are also logically
equivalent:

\begin{prop}\label{logic:prop:FOPL:HDC:permutation:v:HDC}
Let $V$ be a set. Let $\sim$ and $\equiv$ denote the permutation and
Hilbert deductive congruence on \pv\ respectively. Then for all
$\phi,\psi\in\pv$:
    \[
    \phi\sim\psi\ \Rightarrow\ \phi\equiv\psi
    \]
The permutation congruence is stronger than the Hilbert deductive
congruence.
\end{prop}
\begin{proof}
We need to show the inclusion $\sim\,\subseteq\,\equiv$, for which
it is sufficient to show $R_{0}\subseteq\,\equiv$ where $R_{0}$ is a
generator of the permutation congruence. Using
definition~(\ref{logic:def:perm:congruence}) it is therefore
sufficient to prove that $\phi\equiv\psi$ where $\phi=\forall
x\forall y\phi_{1}$ and $\psi=\forall y\forall x\phi_{1}$ where
$\phi_{1}\in\pv$ and $x,y\in V$. By symmetry, it is sufficient to
show that $\phi\leq\psi$ i.e. $\vdash(\phi\to\psi)$. Using the
deduction theorem~(\ref{logic:the:FOPL:deduction}) of
page~\pageref{logic:the:FOPL:deduction} it is therefore sufficient
to prove that $\{\phi\}\vdash\psi$ which is $\Gamma\vdash\forall
y\forall x\phi_{1}$ where $\Gamma=\{\forall x\forall y\phi_{1}\}$.
It is clear that $\Gamma\vdash\forall x\forall y\phi_{1}$. Using the
specialization property of
proposition~(\ref{logic:prop:FOPL:specialization}) we obtain
$\Gamma\vdash\forall y\phi_{1}$. Hence, using specialization once
more we obtain $\Gamma\vdash\phi_{1}$. However, it is clear that
$x\not\in\free(\Gamma)$. So we can use the generalization property
of proposition~(\ref{logic:prop:FOPL:generalization}) to obtain
$\Gamma\vdash\forall x\phi_{1}$. With one additional use of
generalization, since $y\not\in\free(\Gamma)$, we conclude that
$\Gamma\vdash\forall y\forall x\phi_{1}$ as requested.
\end{proof}

Two formulas which are absorption equivalent are also logically
equivalent:

\begin{prop}\label{logic:prop:FOPL:HDC:absorption:v:HDC}
Let $V$ be a set. Let $\sim$ and $\equiv$ denote the absorption and
Hilbert deductive congruence on \pv\ respectively. Then for all
$\phi,\psi\in\pv$:
    \[
    \phi\sim\psi\ \Rightarrow\ \phi\equiv\psi
    \]
The absorption congruence is stronger than the Hilbert deductive
congruence.
\end{prop}
\begin{proof}
We need to show the inclusion $\sim\,\subseteq\,\equiv$, for which
it is sufficient to show $R_{0}\subseteq\,\equiv$ where $R_{0}$ is a
generator of the absorption congruence. Using
definition~(\ref{logic:def:FOPL:abscong:absorption}) it is therefore
sufficient to prove that $\phi_{1}\equiv\forall x\phi_{1}$ where
$\phi_{1}\in\pv$ and $x\not\in\free(\phi_{1})$. First we show that
$\phi_{1}\leq\forall x\phi_{1}$. We need to show that
$\vdash(\phi_{1}\to\forall x\phi_{1})$ which is
$\{\phi_{1}\}\vdash\forall x\phi_{1}$ after application of the
deduction theorem~(\ref{logic:the:FOPL:deduction}) of
page~\pageref{logic:the:FOPL:deduction}. However, since
$x\not\in\free(\phi_{1})$, from $\{\phi_{1}\}\vdash\phi_{1}$ and the
generalization property of
proposition~(\ref{logic:prop:FOPL:generalization}) we obtain
$\{\phi_{1}\}\vdash\forall x\phi_{1}$ as requested. So it remains to
show that $\forall x\phi_{1}\leq\phi_{1}$, which is $\vdash(\forall
x\phi_{1}\to\phi_{1})$, which follows immediately from
proposition~(\ref{logic:prop:FOPL:axiom}) and the fact that $\forall
x\phi_{1}\to\phi_{1}$ is a specialization axiom.
\end{proof}

Propositional equivalence is also stronger than logical equivalence:

\begin{prop}\label{logic:prop:FOPL:HDC:propositional:v:HDC}
Let $V$ be a set. Let $\sim$ and $\equiv$ denote the propositional
and Hilbert deductive congruence on \pv\ respectively. Then for all
$\phi,\psi\in\pv$:
    \[
    \phi\sim\psi\ \Rightarrow\ \phi\equiv\psi
    \]
The propositional congruence is stronger than the Hilbert deductive
congruence.
\end{prop}
\begin{proof}
To be determined.
\end{proof}


So the substitution, permutation, absorption and propositional
congruences are all stronger than the Hilbert deductive congruence.
The following proposition is an easy consequence of this, and can be
added to our set of rules for sequent calculus: if a formula $\phi$
is provable while having the same {\em meaning} as a formula $\psi$,
then $\psi$ is also provable. In fact, if a formula $\phi$ is
provable while being logically equivalent to a formula $\psi$, then
$\psi$ is also provable.

\begin{prop}\label{logic:prop:FOPL:HDC:sequent:stronger:congruence}
Let $V$ be a set and $\sim$ be a congruence which is stronger than
the Hilbert deductive congruence. Then for all $\phi,\psi\in\pv$ and
$\Gamma\subseteq\pv$:
    \[
    (\Gamma\vdash\phi)\land(\phi\sim\psi)\ \Rightarrow\
    \Gamma\vdash\psi
    \]
\end{prop}
\begin{proof}
We assume that $\Gamma\vdash\phi$ and $\phi\sim\psi$. We need to
show that $\Gamma\vdash\psi$. However the congruence $\sim$ on $\pv$
is stronger than the Hilbert deductive congruence $\equiv$\,. Hence
from $\phi\sim\psi$ we obtain $\phi\equiv\psi$. It follows in
particular that $\phi\leq\psi$, i.e. $\vdash(\phi\to\psi)$ and
consequently we have $\Gamma\vdash(\phi\to\psi)$. From
$\Gamma\vdash\phi$ and the modus ponens property of
proposition~(\ref{logic:prop:FOPL:modus:ponens}) we conclude that
$\Gamma\vdash\psi$ as requested.
\end{proof}
