We start with a couple of definitions including that of {\em
elementary permutation} which formalizes the idea of 'adjacent
moves'.
\index{permutation@Permutation of order $n$}
\begin{defin}\label{logic:def:integer:permutation}
Let $n\in\N$ we call {\em permutation of order $n$} a bijection
$\sigma:n\to n$.
\end{defin}
\index{permutation@Elementary permutation}\index{i@$[i:i+1]$ :
permutation}
\begin{defin}\label{logic:def:elementary:permutation}
Let $n\in\N$, $n\geq 2$. We call {\em elementary permutation of
order $n$} any permutation $\sigma:n\to n$ of the form $\sigma=[i
:i+1]$ for some $i\in n-1$.
\end{defin}

For the sake of brevity, we allowed ourselves to use the notational
shortcut $\sigma = \tau_{k}\circ\ldots\circ\tau_{1}$ in the
following proof. This is fair enough. In a way, we are still doing
meta-mathematics with the established standards of usual
mathematics. If we are one day to provide a formal proof of the
following lemma, we shall either need to discard the '\ldots', or
create a high level language where its use is permitted, and
corresponding formulas correctly compiled as low level formulas of
first order logic with some form of recursive wording.
\index{permutation@Decomposition of permutation}
\begin{lemma}\label{logic:lemma:integer:permutation}
Let $n\in\N$, $n\geq 2$. Any permutation $\sigma:n\to n$ of order
$n$ is a composition of elementary permutations, i.e. there exist
$k\in\N^{*}$ and elementary permutations $\tau_{1},\ldots,\tau_{k}$
such that:
    \[
    \sigma = \tau_{k}\circ\ldots\circ\tau_{1}
    \]
\end{lemma}
\begin{proof}
We shall prove lemma~(\ref{logic:lemma:integer:permutation}), using
an induction argument on $n\geq 2$. First we show that the lemma is
true for $n=2$. So suppose $\sigma:2\to 2$ is a permutation of order
$2$. We shall distinguish two cases. First we assume that $\sigma =
[0\!:\!1]$. Then $\sigma$ is an elementary permutation and there is
nothing else to prove. We now assume that $\sigma$ is the identity.
Then $\sigma$ can be expressed as $\sigma =
[0\!:\!1]\circ[0\!:\!1]$. Having considered the only two possible
cases, we conclude that
lemma~(\ref{logic:lemma:integer:permutation}) is true when $n=2$. We
now assume that lemma~(\ref{logic:lemma:integer:permutation}) is
true for $n\geq 2$. We need to show that it is true for $n+1$. So we
assume that $\sigma: (n+1)\to(n+1)$ is a permutation of order $n+1$.
We need to show that $\sigma$ can be expressed as a composition of
elementary permutations. We shall distinguish two cases. First we
assume that $\sigma(n)=n$. Then the restriction $\sigma_{|n}$ is
easily seen to be a permutation of order $n$. Indeed, it is clearly
an injective map $\sigma_{|n}:n\to n$ which is also surjective.
Using the induction hypothesis, there exist $k\in\N^{*}$ and
elementary permutations $\tau_{1},\ldots,\tau_{k}$ of order $n$ such
that:
    \[
    \sigma_{|n}=\tau_{k}\circ\ldots\circ\tau_{1}
    \]
For all $j\in\{1,\ldots,k\}$ consider the extension $\tau^{*}_{j}$
of $\tau_{j}$ on $n+1$ defined by $(\tau^{*}_{j})_{|n}=\tau_{j}$ and
$\tau_{j}^{*}(n)=n$. Having assumed that $\sigma(n)=n$, we obtain:
     \[
    \sigma =\tau^{*}_{k}\circ\ldots\circ\tau^{*}_{1}
    \]
It is therefore sufficient to show that each $\tau^{*}_{j}$ is an
elementary permutation of order $n+1$. So let $j\in\{1,\ldots,k\}$.
Since $\tau_{j}$ is an elementary permutation of order $n$, there
exists $i\in n-1$ such that $\tau_{j}=[i:i+1]$ (of order $n$). It
follows immediately that $\tau^{*}_{j}=[i:i+1]$ (or order $n+1$). So
$\tau^{*}_{j}$ is indeed an elementary permutation of order $n+1$.
This completes our proof of
lemma~(\ref{logic:lemma:integer:permutation}) for $n+1$ in the case
when $\sigma(n)=n$. We now assume that $\sigma(n)\neq n$. Then
$\sigma(n)\in n$. In other words, there exists $j\in n$ such that
$\sigma(n)=n-1-j$. We shall prove by induction on $j\in n$ that
$\sigma$ can be expressed as a composition of elementary
permutations of order $n+1$. Note that in order to do so, it is
sufficient to prove the existence of $m\in\N^{*}$ and elementary
permutations $\tau_{1}',\ldots ,\tau_{m}'$ such that:
    \[
    \tau_{m}'\circ\ldots\circ\tau_{1}'\circ\sigma(n)=n
    \]
Indeed if that is the case, having proved
lemma~(\ref{logic:lemma:integer:permutation}) for $n+1$ in the case
when $\sigma(n)=n$, we deduce the existence of $k\in\N^{*}$ and
elementary permutations $\tau_{1},\ldots,\tau_{k}$ of order $n+1$
such that:
    \[
    \tau_{m}'\circ\ldots\circ\tau_{1}'\circ\sigma = \tau_{k}\circ\ldots\circ\tau_{1}
    \]
and since $\tau^{-1}=\tau$ for every elementary permutation $\tau$,
we conclude that:
    \[
    \sigma=\tau_{1}'\circ\ldots\tau_{m}'\circ\tau_{k}\circ\ldots\circ\tau_{1}
    \]
So we need to show the existence of $m\in\N^{*}$ and elementary
permutations $\tau_{1}',\ldots ,\tau_{m}'$ such that
$\tau_{m}'\circ\ldots\circ\tau_{1}'\circ\sigma(n)=n$, and we shall
do so by induction on $j\in n$ such that $\sigma(n)=n-1-j$. First we
assume that $j=0$. Then $\sigma(n)=n-1$ and it is clear that
$[\,n-1:n\,]\circ\sigma(n)=n$. So the property is true for $j=0$. We
now assume that the property is true for $j\in (n-1)$ and we need to
show that it is true for $j+1$. So we assume that $\sigma(n)= n-1
-(j+1)$. Then we have:
    \[
    [\,n-1 -(j+1)\,:\,n-1-j\,]\circ\sigma(n)=n-1-j
    \]
Having assumed the property is true for $j$, there exist
$m\in\N^{*}$ and elementary permutations $\tau_{1}',\ldots
,\tau_{m}'$ such that:
    \[
    \tau_{m}'\circ\ldots\circ\tau_{1}'\circ[\,n-1 -(j+1)\,:\,n-1-j\,]\circ\sigma(n)=n
    \]
Hence we see that the property is also true for $j+1$.
\end{proof}

Having established that every permutation $\sigma:n\to n$ is a
composition of elementary permutations for $n\geq 2$, our main
objective is to design a proof of\,:
    \begin{equation}\label{logic:eqn:FOPL:integerperm:eqn1}
    \forall x_{n-1}\ldots\forall x_{0}\,\phi_{1}\sim\forall
    x_{\sigma(n-1)}\ldots x_{\sigma(0)}\,\phi_{1}
    \end{equation}
However, we would also like to do so while at the same time creating
a more condensed formalism. In the next section, we shall define the
notion of {\em iterated quantification}, allowing us to replace the
expression $ \forall x_{n-1}\ldots\forall x_{0}\,\phi_{1}$ by a
simple $\forall x\phi_{1}$ with $x\in V^{n}$. Before we do so we
shall introduce a simple equivalence relation on $V^{n}$ so as to
identify $x\in V^{n}$ with $y=x\circ\sigma$. This will allow us to
replace the statement of the
equivalence~(\ref{logic:eqn:FOPL:integerperm:eqn1}) with the
following implication:
    \[
    x\sim y\ \Rightarrow\ \forall x\phi_{1}\sim\forall
    y\phi_{1}
    \]
\index{x@$x\sim y$, $x,y\in V^{n}$ : equivalence}
\begin{defin}\label{logic:def:permutation:equivalence:vn}
Let $V$ be a set, $n\in\N$ and $x,y\in V^{n}$. We say that $x$ is
{\em permutation equivalent to $y$} and we write $x\sim y$ \ifand\
there exists a permutation $\sigma:n\to n$ of order $n$ such that
$y=x\circ \sigma$.
\end{defin}
\begin{prop}\label{logic:prop:permutation:equivalence:vn}
Let $V$ be a set and $n\in\N$. Let $\sim$ denote the permutation
equivalence on $V^{n}$. Then $\sim$ is an equivalence relation on
$V^{n}$.
\end{prop}
\begin{proof}
We need to show that $\sim$ is a reflexive, symmetric and transitive
relation on $V^{n}$. First we show that $\sim$ is reflexive. So
suppose $x\in V^{n}$. We need to show that $x\sim x$. Let
$\sigma:n\to n$ be the identity permutation. Then we have $x=x\circ
\sigma$. Note that if $n=0$, then $x=0$ and $\sigma=0$ and the
equality $x=x\circ\sigma$ is still true. This shows that $x\sim x$.
We now show that $\sim$ is symmetric. So suppose $x,y\in V^{n}$ and
$x\sim y$. We need to show that $y\sim x$. By assumption, there
exists a permutation $\sigma:n\to n$ such that $y=x\circ \sigma$.
Composing to the right by the inverse permutation $\sigma^{-1}:n\to
n$ we obtain $y\circ \sigma^{-1} = x$ and it follows that $y\sim x$.
We now show that $\sim$ is transitive. So suppose $x,y,z\in V^{n}$
are such that $x\sim y$ and $y\sim z$. We need to show that $x\sim
z$. However, there exist permutations $\sigma:n\to n$ and $\tau:n\to
n$ such that $y=x\circ \sigma$ and $z=y\circ \tau$. It follows that
$z=x\circ(\sigma\circ\tau)$ and finally $x\sim z$.
\end{proof}
