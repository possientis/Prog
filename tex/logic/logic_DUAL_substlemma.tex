Let $M$ be a model and $\sigma:V\to W$ be a map. For each variables
assignment $a:W\to M$ we obtain map $a\circ \sigma:V\to M$ which is
also a variables assignment. Given $\phi\in\pv$, it is therefore
meaningful to ask whether the formula $\phi$ is true in the model
$M$ under the assignment $a\circ\sigma$. Denoting $\beta$ the model
valuation function of $M$ on \pv, this amounts to asking whether
$\beta(\phi)(a\circ\sigma)=1$. Now consider the formula
$\sigma(\phi)\in{\bf P}(W)$. It is also meaningful to ask wether the
formula $\sigma(\phi)$ is true in the model $M$ under the assignment
$a$. Again denoting $\beta$ the model valuation function of $M$ on
${\bf P}(W)$ this amounts to asking whether
$\beta(\sigma(\phi))(a)=1$. However, the formula $\sigma(\phi)$ is
roughly speaking identical to the formula $\phi$, after all
variables have been interpreted in $W$ according to the {\em
assignment} $\sigma:V\to W$. It follows that interpreting the
variables of $\sigma(\phi)$ in $M$ according to the assignment
$a:W\to M$ seemingly amounts to interpreting the variables of $\phi$
in $M$ according to the assignment $a\circ\sigma:V\to M$. So if the
formula $\sigma(\phi)$ is true under the assignment $a$, we should
expect the formula $\phi$ to be true under the assignment
$a\circ\sigma$ and conversely. In other words we expect:
    \begin{equation}\label{logic:eqn:FOPL:semsubthe:lemma}
    \beta(\sigma(\phi))(a)=\beta(\phi)(\,a\circ\sigma\,)
    \end{equation}
Of course things are slightly more complicated and we are now well
aware that things usually go wrong when $\sigma$ is not a valid
substitution for $\phi$. So let us find a simple counterexample:
Consider $V=\{x,y\}$ with $x\neq y$ and $W=\{u\}$. Let $\sigma$ be
the unique map $\sigma:V\to W$ defined by $\sigma(x)=\sigma(y)=u$.
Let $\phi=\forall y(x\in y)$. Then we have $\sigma(\phi)=\forall
u(u\in u)$ and $\sigma$ is clearly not valid for $\phi$. It should
not be difficult to find a model $(M,r)$ together with an assignment
$a:W\to M$ for which the
equation~(\ref{logic:eqn:FOPL:semsubthe:lemma}) fails. We cannot use
the empty model $M=\emptyset$ as there exists no assignment $a:W\to
M$ in this case. So let us try a model with a single element, namely
$M=1=\{0\}$. In this case, there exists a unique assignment $a:W\to
M$ defined by $a(u)=0$. It follows that $a\circ\sigma:V\to M$ is
given by $a\circ\sigma(x)=a\circ\sigma(y)=0$. So every variable of
$\phi=\forall y(x\in y)$ or $\sigma(\phi)=\forall u(u\in u)$ is
interpreted as $0$ in $M$. Now we need to choose a binary relation
$r$ on $M$. Since $M\times M=1\times 1=\{(0,0)\}$, we have two
possible choices, namely $r=\emptyset$ or $r=M\times M$. First we
assume that $r=\emptyset$. Then we have:
    \begin{eqnarray*}
    \beta(\sigma(\phi))(a)&=&\beta(\,\forall u(u\in u)\,)(a)\\
    &=&\min\left\{\beta(u\in u)(b)\ :\ b=a\mbox{\ on\ }W_{u}\right\}\\
    &=&\min\left\{1_{r}(b(u),b(u))\ :\ b=a\mbox{\ on\ }W_{u}\right\}\\
    r=\emptyset\ \rightarrow
    &=&\min\left\{0\ :\ b=a\mbox{\ on\ }W_{u}\right\}\\
    &=&0
    \end{eqnarray*}
So the formula $\sigma(\phi)=\forall u(u\in u)$ is false in $M$
under the assignment $a$ in the case when $r=\emptyset$, a fact
which we could have guessed from our intuition. Now:
    \begin{eqnarray*}
    \beta(\phi)(a\circ\sigma)&=&\beta(\,\forall y(x\in y)\,)(a\circ\sigma)\\
    &=&\min\left\{\beta(x\in y)(c)\ :\ c=a\circ\sigma\mbox{\ on\
    }V_{y}\right\}\\
    &=&\min\left\{1_{r}(c(x),c(y))\ :\ c=a\circ\sigma\mbox{\ on\
    }V_{y}\right\}\\
    r=\emptyset\ \rightarrow&=&\min\left\{0\ :\ c=a\circ\sigma\mbox{\ on\
    }V_{y}\right\}\\
    &=&0
    \end{eqnarray*}
So the formula $\phi=\forall y(x\in y)$ is also false in $M$ under
the assignment $a\circ\sigma$, and the case $r=\emptyset$ fails to
provide us with a counterexample. So we now assume that $r=M\times
M$. Then $1_{r}=1$ and going through the same calculations as above,
we obtain $\beta(\sigma(\phi))(a)=1=\beta(\phi)(a\circ\sigma)$. So
once again, we fail to obtain our desired counterexample. So we now
consider a more complicated model with two elements namely
$M=2=\{0,1\}$. There are two possible assignments $a:W\to M$
depending on whether $a(u)=0$ or $a(u)=1$. Since we have not yet
chosen any relation $r\subseteq M\times M$, it is easy to believe
that choosing the assignment $a(u)=0$ bears no loss of generality.
Now we want to find a relation $r$ on $M$ so that $\forall y(x\in
y)$ and $\forall u(u\in u)$ have different truth value in $(M,r)$.
For example, we want $\forall y(x\in y)$ to be true and $\forall
u(u\in u)$ to be false. Loosely speaking, since $a\circ\sigma(x)=0$
we want $\forall y(0\in y)$ to be true while $\forall u(u\in u)$ is
false. So let us pick $r=\{(0,0),(0,1)\}$. Then we have:
    \begin{eqnarray*}
    \beta(\sigma(\phi))(a)&=&\beta(\,\forall u(u\in u)\,)(a)\\
    &=&\min\left\{\beta(u\in u)(b)\ :\ b=a\mbox{\ on\ }W_{u}\right\}\\
    \mbox{for any\ }b:W\to M\ \rightarrow&\leq&\beta(u\in u)(b)\\
    &=&1_{r}(b(u),b(u))\\
    \mbox{choosing\ }b(u)=1\ \rightarrow&=&1_{r}(1,1)\\
    (1,1)\not\in r\ \rightarrow&=&0
    \end{eqnarray*}
and:
    \begin{eqnarray*}
    \beta(\phi)(a\circ\sigma)&=&\beta(\,\forall y(x\in y)\,)(a\circ\sigma)\\
    &=&\min\left\{\beta(x\in y)(c)\ :\ c=a\circ\sigma\mbox{\ on\
    }V_{y}\right\}\\
    c_{1}(y)=0\mbox{\ while\ }c_{2}(y)=1\ \rightarrow
    &=&\beta(x\in y)(c_{1})\land\beta(x\in y)(c_{2})\\
    &=&1_{r}(c_{1}(x),c_{1}(y))\land 1_{r}(c_{2}(x),c_{2}(y))\\
    &=&1_{r}(c_{1}(x),0)\land 1_{r}(c_{2}(x),1)\\
    x\in V_{y}\ \rightarrow
    &=&1_{r}(a\circ\sigma(x),0)\land 1_{r}(a\circ\sigma(x),1)\\
    &=&1_{r}(0,0)\land 1_{r}(0,1)\\
    &=& 1\land 1\\
    &=&1
    \end{eqnarray*}
So we have found a counterexample to
equation~(\ref{logic:eqn:FOPL:semsubthe:lemma}) in the case when
$\sigma$ is not a valid substitution for $\phi\in\pv$. We shall now
establish the positive result:

\index{substitution@Substitution lemma valid}
\begin{prop}\label{logic:prop:FOPL:model:valid:substitution}
Let $V,W$ be sets and $\sigma:V\to W$ be a map. Let $M$ be a model
of \pv\ and ${\bf P}(W)$ with associated model valuation functions
denoted $\beta$. Then for all $\phi\in\pv$ and assignment $a:W\to
M$, if $\sigma$ is valid for $\phi$ we have:
    \[
    \beta(\sigma(\phi))(a)=\beta(\phi)(\,a\circ\sigma\,)
    \]
where $\sigma:\pv\to{\bf P}(W)$ also denotes the associated
substitution mapping.
\end{prop}
\begin{proof}
Note that the '$\beta$' appearing on the left-hand-side is the model
valuation function $\beta:{\bf P}(W)\to{\cal P}(M^{W})$ while the
'$\beta$' appearing on the right-hand-side is the model valuation
function $\beta:\pv\to{\cal P}(M^{V})$. Recall that
$\sigma:\pv\to{\bf P}(W)$ is the associated substitution mapping as
per definition~(\ref{logic:def:substitution}) and consequently
$\sigma(\phi)\in{\bf P}(W)$ whenever $\phi\in\pv$. Finally, if
$a:W\to M$ is an assignment, then $a\circ\sigma:V\to M$ is also an
assignment. So everything makes sense. Given $M$ and $\sigma:V\to
W$, for all $\phi\in\pv$ we need to prove that for all $a:W\to M$:
    \[
    (\mbox{$\sigma$ valid for $\phi$})\ \Rightarrow\
    \beta(\sigma(\phi))(a)=\beta(\phi)(\,a\circ\sigma\,)
    \]
We shall do so by a structural induction argument, using
theorem~(\ref{logic:the:proof:induction}) of
page~\pageref{logic:the:proof:induction}. First we assume that
$\phi=(x\in y)$ where $x,y\in V$. Let $a:W\to M$. Then $\sigma$ is
always valid for $\phi$ and denoting $r\subseteq M\times M$ the
relation on $M$ we have:
    \begin{eqnarray*}
    \beta(\sigma(\phi))(a)&=&\beta(\sigma(x\in y))(a)\\
    &=&\beta(\,\sigma(x)\in\sigma(y)\,)(a)\\
    &=&1_{r}(\,a(\sigma(x))\,,\,a(\sigma(y))\,)\\
    &=&1_{r}(\,a\circ\sigma(x)\,,\,a\circ\sigma(y)\,)\\
    &=&\beta(x\in y)(a\circ\sigma)\\
    &=&\beta(\phi)(a\circ\sigma)\\
    \end{eqnarray*}
We now assume that $\phi=\bot$. Then given $a:W\to M$ we have:
    \[
    \beta(\sigma(\bot))(a)=\beta(\bot)(a)=0=\beta(\bot)(a\circ\sigma)
    \]
So we now assume that $\phi=\phi_{1}\to\phi_{2}$ where
$\phi_{1},\phi_{2}\in\pv$ satisfy our property. We need to show the
same is true of $\phi$. So let $a:W\to M$ be an assignment and
suppose $\sigma$ is valid for $\phi$. We need to show
$\beta(\sigma(\phi))(a)=\beta(\phi)(\,a\circ\sigma\,)$. However,
using proposition~(\ref{logic:prop:FOPL:valid:recursion:imp}),
$\sigma$ is valid for both $\phi_{1}$ and $\phi_{2}$. Hence the
equation is true both for $\phi_{1}$ and $\phi_{2}$ and consequently
we have:
    \begin{eqnarray*}
    \beta(\sigma(\phi))(a)&=&\beta(\sigma(\phi_{1}\to\phi_{2}))(a)\\
    &=&\beta(\,\sigma(\phi_{1})\to\sigma(\phi_{2})\,)(a)\\
    &=&\beta(\sigma(\phi_{1}))(a)\to\beta(\sigma(\phi_{2}))(a)\\
    &=&\beta(\phi_{1})(a\circ\sigma)\to\beta(\phi_{2})(a\circ\sigma)\\
    &=&\beta(\phi_{1}\to\phi_{2})(a\circ\sigma)\\
    &=&\beta(\phi)(a\circ\sigma)\\
    \end{eqnarray*}
So we now assume that $\phi=\forall x\phi_{1}$ where $x\in V$ and
$\phi_{1}\in\pv$ satisfy our property. We need to show the same is
true for $\phi$. So let $a:W\to M$ be an assignment and suppose
$\sigma$ is valid for $\phi$. We want
$\beta(\sigma(\phi))(a)=\beta(\phi)(\,a\circ\sigma\,)$. However,
from proposition~(\ref{logic:prop:FOPL:valid:recursion:quant}),
$\sigma$ is also valid for $\phi_{1}$. Hence, the equation is true
for $\phi_{1}$ and any assignment $b:W\to M$. It follows that:
    \begin{eqnarray*}
    \beta(\sigma(\phi))(a)&=&\beta(\sigma(\forall x\phi_{1}))(a)\\
    &=&\beta(\,\forall\sigma(x)\sigma(\phi_{1})\,)(a)\\
    &=&\min\left\{\,\beta(\sigma(\phi_{1}))(b)\ :\ b=a\mbox{\ on\ }W_{\sigma(x)}\,\right\}\\
    &=&\min\left\{\,\beta(\phi_{1})(b\circ\sigma)\ :\ b=a\mbox{\ on\ }W_{\sigma(x)}\,\right\}\\
    \mbox{A: to be proved}\ \rightarrow
    &=&\min\left\{\,\beta(\phi_{1})(k)\ :\ k=a\circ\sigma\mbox{\ on\ }V_{x}\,\right\}\\
    &=&\beta(\forall x\phi_{1})(a\circ\sigma)\\
    &=&\beta(\phi)(a\circ\sigma)\\
    \end{eqnarray*}
So it remains to prove point A, for which it is sufficient to show
the set equality:
    \[
    X=\left\{\,\beta(\phi_{1})(b\circ\sigma)\ :\ b=a\mbox{\ on\ }W_{\sigma(x)}\,\right\}
    = \left\{\,\beta(\phi_{1})(k)\ :\ k=a\circ\sigma\mbox{\ on\
    }V_{x}\,\right\}=Y
    \]
First we show that $X\subseteq Y$. So let $\epsilon\in X$. There
exists an assignment $b:W\to M$ which coincides with $a$ on
$W\setminus\{\sigma(x)\}$ such that
$\epsilon=\beta(\phi_{1})(b\circ\sigma)$. We need to show that
$\epsilon\in Y$. Define the assignment $k:V\to M$ by setting
$k=a\circ\sigma$ on $V\setminus\{x\}$ and $k(x)=b\circ\sigma(x)$. In
order to show that $\epsilon\in Y$ it is sufficient to show that
$\epsilon=\beta(\phi_{1})(k)$. So we need to show that
$\beta(\phi_{1})(b\circ\sigma)=\beta(\phi_{1})(k)$. Using
proposition~(\ref{logic:prop:FOPL:model:assignment:support}), it is
sufficient to prove that $b\circ\sigma$ and $k$ coincide on
$\free(\phi_{1})$. So let $u\in\free(\phi_{1})$. We need to show
that $b\circ\sigma(u)=k(u)$. We shall distinguish two cases: first
we assume that $u=x$. Then $b\circ\sigma(x)=k(x)$ is true by
definition of $k$. Next we assume that $u\neq x$. Then
$u\in\free(\phi_{1})\setminus\{x\}=\free(\phi)$. From the validity
of $\sigma$ for $\phi$ and
proposition~(\ref{logic:prop:FOPL:valid:recursion:quant}) we obtain
$\sigma(u)\neq\sigma(x)$. It follows that $a$ and $b$ coincide on
$\sigma(u)$ i.e. $a\circ\sigma(u)=b\circ\sigma(u)$. However by
definition, the assignment $k$ coincides with $a\circ\sigma$ on
$V\setminus\{x\}$. Since $u\neq x$ we obtain $k(u)=a\circ\sigma(u)$
and it follows that $b\circ\sigma(u)=k(u)$ as requested. So we now
show that $Y\subseteq X$: let $\epsilon\in Y$. There exists an
assignment $k:V\to M$ which coincides with $a\circ\sigma$ on
$V\setminus\{x\}$ such that $\epsilon=\beta(\phi_{1})(k)$. We need
to show that $\epsilon\in X$. Define the assignment $b:W\to M$ by
setting $b=a$ on $W\setminus\{\sigma(x)\}$ and $b(\sigma(x))=k(x)$.
In order to show that $\epsilon\in X$ it is sufficient to prove that
$\epsilon=\beta(\phi_{1})(b\circ\sigma)$. So we need to show that
$\beta(\phi_{1})(b\circ\sigma)=\beta(\phi_{1})(k)$, for which we
shall pretty much repeat our previous argument: using
proposition~(\ref{logic:prop:FOPL:model:assignment:support}), it is
sufficient to prove that $b\circ\sigma$ and $k$ coincide on
$\free(\phi_{1})$. So let $u\in\free(\phi_{1})$. We need to show
that $b\circ\sigma(u)=k(u)$. We shall distinguish two cases: first
we assume that $u=x$. Then $b\circ\sigma(x)=k(x)$ is true by
definition of $b$. Next we assume that $u\neq x$. Then
$u\in\free(\phi_{1})\setminus\{x\}=\free(\phi)$. From the validity
of $\sigma$ for $\phi$ and
proposition~(\ref{logic:prop:FOPL:valid:recursion:quant}) we obtain
$\sigma(u)\neq\sigma(x)$.  However by definition, the assignment $b$
coincides with $a$ on $W\setminus\{\sigma(x)\}$. Since
$\sigma(u)\neq\sigma(x)$ we obtain
$b\circ\sigma(u)=a\circ\sigma(u)$. Furthermore, since the assignment
$k$ coincide with $a\circ\sigma$ on $V\setminus\{x\}$ and $u\neq x$
we have $a\circ\sigma(u)=k(u)$ and it follows that
$b\circ\sigma(u)=k(u)$.
\end{proof}

So we know that
$\beta(\sigma(\phi))(a)=\beta(\phi)(\,a\circ\sigma\,)$ whenever
$\sigma:V\to W$ is valid for $\phi\in\pv$ and $a:W\to M$ is an
arbitrary assignment. As we have done on several occasions before,
we would like to extend this formula from valid substitutions to
essential substitutions of
definition~(\ref{logic:def:FOPL:esssubstprop:essential}). So we need
to say something about minimal transforms of
definition~(\ref{logic:def:FOPL:mintransform:transform}). So let $V$
be a set with minimal extension $\bar{V}$. Given $\phi\in\pv$ with
minimal transform the formula ${\cal M}(\phi)\in\pvb$, we know that
$\phi$ and ${\cal M}(\phi)$ are essentially the same formula where
the bound variables of $\phi$ have been replaced with elements of
\N. So we would expect the {\em truth} of the formula $\phi$ in a
model $M$ under an assignment $a:V\to M$ to be equivalent to the
{\em truth} of ${\cal M}(\phi)$ under the same assignment. More
precisely, whichever way we decide to extend the assignment $a:V\to
M$ into an assignment $a^{*}:\bar{V}\to M$, since every $n\in\N$
cannot be a free variable of the minimal transform ${\cal M}(\phi)$,
we know from the relevance lemma of
proposition~(\ref{logic:prop:FOPL:model:assignment:support}) that
$\beta({\cal M}(\phi))(a^{*})$ will essentially depend on $a$ and
not on the specifics of the extension $a^{*}$. In fact, we expect
$\beta({\cal M}(\phi))(a^{*})=\beta(\phi)(a)$\,:
\begin{prop}\label{logic:prop:FOPL:model:min:transform}
Let $V$ be a set with minimal extension $\bar{V}$. Let $M$ be a
model of\, \pv\ and \pvb\ with associated model valuation functions
denoted $\beta$. Let $a:V\to M$ and $a^{*}:\bar{V}\to M$ be an
arbitrary extension of $a$. For all $\phi\in\pv$\,:
    \begin{equation}\label{logic:eqn:FOPL:model:min:transform:1}
    \beta({\cal M}(\phi))(a^{*})=\beta(\phi)(a)
    \end{equation}
where ${\cal M}(\phi)\in\pvb$ is the minimal transform of $\phi$ as
per {\em definition~(\ref{logic:def:FOPL:mintransform:transform})}.
\end{prop}
\begin{proof}
Given $\phi\in\pv$ we need to show that for any assignment $a:V\to
M$ and any extension $a^{*}:\bar{V}\to M$, the
equality~(\ref{logic:eqn:FOPL:model:min:transform:1}) holds. We
shall do so by structural induction, using
theorem~(\ref{logic:the:proof:induction}) of
page~\pageref{logic:the:proof:induction}. First we assume that
$\phi=(x\in y)$ where $x,y\in V$. Let $a:V\to M$ and
$a^{*}:\bar{V}\to M$ be an extension of $a$. Then, denoting
$r\subseteq M\times M$ the relation on $M$, we have:
    \begin{eqnarray*}
    \beta({\cal M}(\phi))(a^{*})&=&\beta({\cal M}(x\in y))(a^{*})\\
    &=&\beta(x\in y)(a^{*})\\
    &=&1_{r}(a^{*}(x),a^{*}(y))\\
    a^{*}_{|V}=a\ \rightarrow&=&1_{r}(a(x),a(y))\\
    &=&\beta(x\in y)(a)\\
    &=&\beta(\phi)(a)\\
    \end{eqnarray*}
Next we assume that $\phi=\bot$. Given $a:V\to M$ and extension
$a^{*}:\bar{V}\to M$\,:
    \[
    \beta({\cal M}(\bot))(a^{*})=\beta(\bot)(a^{*})=0=\beta(\bot)(a)
    \]
So we now assume that $\phi=\phi_{1}\to\phi_{2}$ where
$\phi_{1},\phi_{2}\in\pv$ are such that for all $a:V\to M$ and
extension $a^{*}:\bar{V}\to M$, the
equality~(\ref{logic:eqn:FOPL:model:min:transform:1}) holds. We need
to show the same is true of $\phi$. So let $a:V\to M$ with extension
$a^{*}:\bar{V}\to M$\,:
    \begin{eqnarray*}
    \beta({\cal M}(\phi))(a^{*})&=&\beta({\cal M}(\phi_{1}\to\phi_{2}))(a^{*})\\
    &=&\beta(\,{\cal M}(\phi_{1})\to{\cal M}(\phi_{2})\,)(a^{*})\\
    &=&\beta({\cal M}(\phi_{1}))(a^{*})\to\beta({\cal M}(\phi_{2}))(a^{*})\\
    &=&\beta(\phi_{1})(a)\to\beta(\phi_{2})(a)\\
    &=&\beta(\phi_{1}\to\phi_{2})(a)\\
    &=&\beta(\phi)(a)\\
    \end{eqnarray*}
Finally, we assume that $\phi=\forall x\phi_{1}$ where $x\in V$ and
$\phi_{1}\in\pv$ is such that for all $a:V\to M$ and extension
$a^{*}:\bar{V}\to M$, the
equality~(\ref{logic:eqn:FOPL:model:min:transform:1}) holds. We need
to show the same is true of $\phi$. So let $a:V\to M$ with extension
$a^{*}:\bar{V}\to M$\,:
    \begin{eqnarray*}
    \beta({\cal M}(\phi))(a^{*})&=&\beta({\cal M}(\forall x\phi_{1}))(a^{*})\\
    \mbox{$[n/x]$\ valid for\ }{\cal M}(\phi_{1})\ \rightarrow
    &=&\beta(\,\forall n{\cal M}(\phi_{1})[n/x]\,)(a^{*})\\
    &=&\min\left\{\beta(\,[n/x]\circ{\cal M}(\phi_{1})\,)(b^{*}):
    b^{*}=a^{*}\mbox{\ on\ }\bar{V}_{n}\right\}\\
    \mbox{prop.~(\ref{logic:prop:FOPL:model:valid:substitution})}\ \rightarrow
    &=&\min\left\{\beta({\cal M}(\phi_{1}))(\,b^{*}\circ[n/x]\,):
    b^{*}=a^{*}\mbox{\ on\ }\bar{V}_{n}\right\}\\
    &=&\min\left\{\beta(\phi_{1})(\,(b^{*}\circ[n/x])_{|V}\,):
    b^{*}=a^{*}\mbox{\ on\ }\bar{V}_{n}\right\}\\
    \mbox{A: to be proved}\ \rightarrow&=&\min\left\{\beta(\phi_{1})(b)\ :\
    b=a\mbox{\ on\ }V_{x}\right\}\\
    &=&\beta(\forall x\phi_{1})(a)\\
    &=&\beta(\phi)(a)\\
    \end{eqnarray*}
So it remains to prove point A, for which it is sufficient to show
the set equality:
    \[
    X=\left\{\beta(\phi_{1})(\,(b^{*}\circ[n/x])_{|V}\,):
    b^{*}=a^{*}\mbox{\ on\ }\bar{V}_{n}\right\}=\left\{\beta(\phi_{1})(b)\ :\
    b=a\mbox{\ on\ }V_{x}\right\}=Y
    \]
First we show that $X\subseteq Y$: so let $\epsilon\in X$. There
exists an assignment $b^{*}:\bar{V}\to M$ such that $b^{*}=a^{*}$ on
$\bar{V}\setminus\{n\}$ and
$\epsilon=\beta(\phi_{1})(\,(b^{*}\circ[n/x])_{|V}\,)$. We need to
show that $\epsilon\in Y$. Define the assignment $b:V\to M$ by
setting $b=a$ on $V\setminus\{x\}$ and $b(x)= b^{*}(n)$. In order to
show that $\epsilon\in Y$, it is sufficient to prove that
$\epsilon=\beta(\phi_{1})(b)$. So we need to show that
$\beta(\phi_{1})(\,(b^{*}\circ[n/x])_{|V}\,)=\beta(\phi_{1})(b)$.
Using proposition~(\ref{logic:prop:FOPL:model:assignment:support}),
it is sufficient to prove that $(b^{*}\circ[n/x])_{|V}$ and $b$
coincide on $\free(\phi_{1})$. So let $u\in\free(\phi_{1})$. We need
to prove that $b^{*}\circ[n/x](u)=b(u)$. We shall distinguish two
cases: first we assume that $u=x$. Then we need to show that
$b^{*}(n)=b(x)$ which is true by definition of $b$. Next we assume
that $u\neq x$. Then we need to show that $b^{*}(u)=b(u)$. However,
since $b^{*}=a^{*}$ on $\bar{V}\setminus\{n\}$ and $u\in V$, we have
$u\neq n$ and consequently $b^{*}(u)=a^{*}(u)$. Furthermore, $a^{*}$
is an extension of $a$ so $a^{*}(u)=a(u)$. So we need to show that
$a(u)=b(u)$ which is true by definition of $b$ and $u\neq x$. We now
show that $Y\subseteq X$: so let $\epsilon\in Y$. There exists an
assignment $b:V\to M$ such that $b=a$ on $V\setminus\{x\}$ and
$\epsilon=\beta(\phi_{1})(b)$. We need to show that $\epsilon\in X$.
Define the assignment $b^{*}:\bar{V}\to M$ by setting $b^{*}=a^{*}$
on $\bar{V}\setminus\{n\}$ and $b^{*}(n)=b(x)$. In order to show
that $\epsilon\in X$ it is sufficient to prove that
$\epsilon=\beta(\phi_{1})(\,(b^{*}\circ[n/x])_{|V}\,)$. Hence we
need to show the equality
$\beta(\phi_{1})(\,(b^{*}\circ[n/x])_{|V}\,)=\beta(\phi_{1})(b)$.
Using proposition~(\ref{logic:prop:FOPL:model:assignment:support}),
it is sufficient to prove that $(b^{*}\circ[n/x])_{|V}$ and $b$
coincide on $\free(\phi_{1})$. So let $u\in\free(\phi_{1})$. We need
to prove that $b^{*}\circ[n/x](u)=b(u)$. We shall distinguish two
cases: first we assume that $u=x$. Then we need to show that
$b^{*}(n)=b(x)$ which is true by definition of $b^{*}$. Next we
assume that $u\neq x$. Then we need to show that $b^{*}(u)=b(u)$.
However, since $b^{*}=a^{*}$ on $\bar{V}\setminus\{n\}$ and $u\in
V$, we have $u\neq n$ and consequently $b^{*}(u)=a^{*}(u)$.
Furthermore, $a^{*}$ is an extension of $a$ so $a^{*}(u)=a(u)$. So
we need to show that $a(u)=b(u)$ which follows from $u\neq x$.
\end{proof}

Having established
proposition~(\ref{logic:prop:FOPL:model:min:transform}) we are able
to link the notions of minimal transform ${\cal M}(\phi)$ and that
of model valuation function $\beta$. So we can now hope to prove the
formula $\beta(\sigma(\phi))(a)=\beta(\phi)(\,a\circ\sigma\,)$ in
the general case when $\sigma:\pv\to{\bf P}(W)$ is an essential
substitution as per
definition~(\ref{logic:def:FOPL:esssubstprop:essential}). However
before we do so, we need to check the formula makes sense in
principle. So let $\sigma:\pv\to{\bf P}(W)$ be an essential
substitution. Then $\sigma$ is associated to a unique map
$\sigma:V\to W$ (also denoted $\sigma$) and for every map $a:W\to
M$, we have a meaningful assignment $a\circ\sigma:V\to M$. Hence the
expression $\beta(\phi)(\,a\circ\sigma\,)$ makes perfect sense for
all $\phi\in\pv$. However, we know from
proposition~(\ref{logic:prop:FOPL:esssubstprop:redefine}) that any
essential substitution $\sigma:\pv\to{\bf P}(W)$ can be redefined
arbitrarily modulo the substitution congruence, without changing its
associated map $\sigma:V\to W$. So the formula
$\beta(\sigma(\phi))(a)=\beta(\phi)(\,a\circ\sigma\,)$ cannot be
true unless $\beta(\,\cdot\,)(a)$ is invariant with respect to a
particular choice of formula modulo substitution. The following
proposition ensures this is the case. As we shall soon discover from
theorem~(\ref{logic:the:FOPL:soundness:soundness:2}) of
page~\pageref{logic:the:FOPL:soundness:soundness:2},
$\beta(\,\cdot\,)(a)$ is in fact a valuation and
proposition~(\ref{logic:prop:FOPL:model:subcong}) below is therefore
a particular case of
proposition~(\ref{logic:prop:FOPL:semantics:stronger:congruence}).
\begin{prop}\label{logic:prop:FOPL:model:subcong}
Let $V$ be a set and $M$ be a model of\, \pv\ with model valuation
function $\beta$. Then for all $\phi,\psi\in\pv$ and assignment
$a:V\to M$\,:
    \[
    \phi\sim\psi\ \Rightarrow\ \beta(\phi)(a)=\beta(\psi)(a)
    \]
where $\sim$ denotes the substitution congruence on \pv.
\end{prop}
\begin{proof}
Let $\phi,\psi\in\pv$ such that $\phi\sim\psi$ and let $a:V\to M$ be
an assignment. We need to show that $\beta(\phi)(a)=\beta(\psi)(a)$.
We shall distinguish two cases: first we assume that there exists an
assignment $a^{*}:\bar{V}\to M$ which is an extension of $a$. Then
using proposition~(\ref{logic:prop:FOPL:model:min:transform}) we
obtain the following equalities:
    \[
    \beta(\phi)(a)=\beta({\cal M}(\phi))(a^{*})=\beta({\cal
    M}(\psi))(a^{*})=\beta(\psi)(a)
    \]
where we have used the fact that ${\cal M}(\phi)={\cal M}(\psi)$,
itself a consequence of $\phi\sim\psi$ and
theorem~(\ref{logic:the:FOPL:mintransfsubcong:kernel}) of
page~\pageref{logic:the:FOPL:mintransfsubcong:kernel}. Next we
assume that there exists no extension $a^{*}:\bar{V}\to M$. This can
only be the case when $M=\emptyset$. Otherwise, pick an arbitrary
$m^{*}\in M$ and define $a^{*}(n)=m^{*}$ for all $n\in\N$. Now if
$M=\emptyset$, the only possible assignment $a:V\to M$ is the map
with empty domain, i.e. $a=\emptyset$. It follows that $V=\emptyset$
and consequently from definition~(\ref{logic:def:sub:congruence}),
the substitution congruence on \pv\ is generated by the empty set.
Hence from the equivalence $\phi\sim\psi$ we obtain $\phi=\psi$ and
the equality $\beta(\phi)(a)=\beta(\psi)(a)$ follows.
\end{proof}

We are now ready to prove our next theorem which extends
proposition~(\ref{logic:prop:FOPL:model:valid:substitution}) to
essential substitutions and is commonly known as the {\em
Substitution Lemma}.

\index{substitution@Substitution lemma essential}
\begin{theorem}\label{logic:the:FOPL:model:essential:substitution}
Let $V,W$ be sets and $\sigma:\pv\to{\bf P}(W)$ be an essential
substitution. Let $M$ be a model of \pv\ and ${\bf P}(W)$ with model
valuation functions~$\beta$. Then for all $\phi\in\pv$ and
assignment $a:W\to M$, we have the equality:
    \begin{equation}\label{logic:eqn:FOPL:model:essential:substitution:1}
    \beta(\sigma(\phi))(a)=\beta(\phi)(a\circ\sigma)
    \end{equation}
\end{theorem}
\begin{proof}
Note that the '$\sigma$' appearing on the right-hand-side of
equation~(\ref{logic:eqn:FOPL:model:essential:substitution:1}) is
the map $\sigma:V\to W$ associated with the essential substitution
$\sigma:\pv\to{\bf P}(W)$, i.e. the unique map $\sigma$ such that
${\cal M}\circ\sigma=\bar{\sigma}\circ{\cal M}$ as per
definition~(\ref{logic:def:FOPL:esssubstprop:essential}). So if
$a:W\to M$ is an assignment then $a\circ\sigma:V\to M$ is also an
assignment. Now given $\phi\in\pv$ and an assignment $a:W\to M$, we
need to prove
equation~(\ref{logic:eqn:FOPL:model:essential:substitution:1}). We
shall distinguish two cases: first we assume there exists an
extension $a^{*}:\bar{W}\to M$ of the assignment $a$. Then using
proposition~(\ref{logic:prop:FOPL:model:min:transform}) we obtain:
    \begin{eqnarray*}
    \beta(\sigma(\phi))(a)&=&\beta(\,{\cal M}\circ\sigma(\phi)\,)(a^{*})\\
    \mbox{def.~(\ref{logic:def:FOPL:esssubstprop:essential})}\ \rightarrow
    &=&\beta(\,\bar{\sigma}\circ{\cal M}(\phi)\,)(a^{*})\\
    \mbox{prop.~(\ref{logic:prop:FOPL:model:valid:substitution}),
    $\bar{\sigma}$ valid for ${\cal M}(\phi)$}\ \rightarrow
    &=&\beta({\cal M}(\phi))(\,a^{*}\circ\bar{\sigma}\,)\\
    \mbox{prop.~(\ref{logic:prop:FOPL:model:min:transform})}\ \rightarrow
    &=&\beta(\phi)(\,(a^{*}\circ\bar{\sigma})_{|V}\,)\\
    \bar{\sigma}_{|V}=\sigma,\ a^{*}_{|W}=a\ \rightarrow
    &=&\beta(\phi)(a\circ\sigma)\\
    \end{eqnarray*}
We now assume that there exists no extension $a^{*}:\bar{W}\to M$.
This can only be the case when $M=\emptyset$. Otherwise, pick an
arbitrary $m^{*}\in M$ and define $a^{*}(n)=m^{*}$ for all $n\in\N$.
Now if $M=\emptyset$, the only possible assignment $a:W\to M$ is the
map with empty domain, i.e. $a=\emptyset$. It follows that
$W=\emptyset$. However, since $\sigma:\pv\to{\bf P}(W)$ is an
essential substitution, using
theorem~(\ref{logic:the:FOPL:esssubst:existence}) of
page~\pageref{logic:the:FOPL:esssubst:existence} we see that
$V=\emptyset=W$. From definition~(\ref{logic:def:sub:congruence})
the substitution congruence on \pv\ is generated by the empty set
and therefore coincides with the equality. Using
proposition~(\ref{logic:prop:FOPL:esssubstprop:charac}) it follows
that
$\sigma(\phi_{1}\to\phi_{2})=\sigma(\phi_{1})\to\sigma(\phi_{2})$
for all $\phi_{1},\phi_{2}\in\pv$, and $\sigma(\bot)=\bot$. A simple
induction argument shows that $\sigma:\pv\to\pv$ coincides with the
identity mapping, which is associated to the empty mapping
$\sigma:V\to V$ as per
definition~(\ref{logic:def:FOPL:esssubstprop:essential}) since
${\cal M}\circ\sigma=\bar{\emptyset}\circ{\cal M}$, as the minimal
extension $\bar{\emptyset}:\bar{V}\to\bar{V}$ is simply the identity
on \N. So $a\circ\sigma$ is the empty assignment and the
equality~(\ref{logic:eqn:FOPL:model:essential:substitution:1})
follows.
\end{proof}
