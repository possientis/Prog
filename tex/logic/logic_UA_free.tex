\index{free@Free universal algebra}
\begin{defin}\label{logic:def:free:universal:algebra} A Universal Algebra is said
to be {\em free}  if it has a free generator. \end{defin} We have already seen a
few examples of free universal algebras. The set $\N$ with the successor operator
$s:\N^{1}\to\N$ , or the set $\N$ with both the successor operator and the
constant $0$. We have seen that $\emptyset$ is also a free universal algebra of
type $\alpha$, whenever $\alpha$ has no constant. Along the same lines, any set
$X$ is a free universal algebra of empty type. Indeed take $X_{0}=X$ and suppose
$Y$ is any set while $g_{0}:X_{0}\to Y$ is a map. There exists a unique map
$g:X\to Y$ such that $g_{|X_{0}}=g_{0}$, namely $g_{0}$ itself. Furthermore, the
conditions required for $g:X\to Y$ to be a morphism are vacuously satisfied when
$\alpha=\emptyset$. Hence we see that $X_{0}=X$ is a free generator of $X$, as a
universal algebra of empty type.

It would be wrong to think that most universal algebras are free. Before we
proceed, let us find a simple example of universal algebra which is not free. The
empty set $\emptyset$ is not a universal algebra of type $\alpha$ unless $\alpha$
has no constant. When this is the case the empty set is free. So let us try a
slightly bigger set, namely $X = \{0\}$. If $\alpha$ is the empty type, then $X$
is free, with itself as a free generator. Suppose $\alpha =\{(0,0)\}$. Then the
only possible structure on $X$ is given by $T(0,0):\{0\}\to X$ defined by
$T(0,0)(0)=0$. In other words, there is only one possible choice of constant
within $X$. Unfortunately $X$ is still free, with the empty set as a free
generator. For if $Y$ is another universal algebra of type $\alpha=\{(0,0)\}$ and
$g_{0}:\emptyset\to Y$ is a map, there exists a unique morphism $g:X\to Y$ which
extends $g_{0}$, namely the map $g$ defined by $g(0)=0$, where the r.h.s '$0$'
refer to the constant of $Y$. So we need to choose a slightly more complex
structure on $X$. Consider $\alpha=\{(0,1)\}$. Since $X=\{0\}$, there is only one
possible choice of operator $T(0,1):X^{1}\to X$, namely $T(0,1)(0)=0$. With this
particular structure, $X$ becomes a universal algebra of the same type as $\N$
with a single successor operator $s:\N^{1}\to \N$. We claim that $X$ is not free.
If it was free, there would exist $X_{0}\subseteq X$ which is a free generator of
$X$. The only possible values for $X_{0}$ are $X_{0}=\emptyset$ and $X_{0}=\{0\}$.
Now $X_{0}=\emptyset$ cannot be a free generator of $X$. As we have already seen,
since $\alpha$ has no constant, the empty set is the only universal algebra of
type $\alpha$ which can have an empty free generator. Suppose now that
$X_{0}=\{0\}$ is a free generator of $X$. We shall arrive at a contradiction.
Consider the map $g_{0}:X_{0}\to\N$ defined by $g_{0}(0)=0$. Since $X_{0}$ is a free
generator of $X$, there exists a unique morphism $g:X\to\N$ such that
$g_{|X_{0}}=g_{0}$. In fact, we must have $g=g_{0}$ since $X_{0}=X$. But $g=g_{0}$
is not a morphism. Indeed, If $g$ was a morphism, we would have:
    \[
    0=g(0) = g\circ T(0,1)(0) = s\circ g(0)= 0 + 1 = 1
    \]
So we have found a simple example of universal algebra which is not free, namely
$X=\{0\}$ of type $\alpha=\{(0,1)\}$ with the only possible operator
$T(0,1):X^{1}\to X$.

Of course, having one single example of universal algebra which fails to be free
is little evidence so far. The following proposition shows that free universal
algebras are in fact pretty rare. There is {\em essentially} at most {\em one}
free universal algebra of type $\alpha$, for a given cardinality of free
generator. \begin{prop}\label{logic:prop:isomorphic} Suppose $X$ and $Y$ are two
free universal algebras of type $\alpha$ with free generators $X_{0}$ and $Y_{0}$
respectively. If there exists a bijection $j:X_{0}\to Y_{0}$, then $X$ and $Y$ are
isomorphic. \end{prop} \begin{proof} Since $Y_{0}\subseteq Y$ we also have
$j:X_{0}\to Y$. Since $X_{0}$ is a free generator of $X$, there exists a unique
morphism $g:X\to Y$ such that $g_{|X_{0}}=j$. We claim that $g$ is in fact an
isomorphism. To show this, we only need to prove that $g:X\to Y$ is bijective. We
shall do so by finding a map $g':Y\to X$ such that $g'\circ g = id_{X}$ and
$g\circ g'=id_{Y}$ where $id_{X}:X\to X$ and $id_{Y}:Y\to Y$ are the identity
mappings. Since $j:X_{0}\to Y_{0}$ is a bijection, we have $j^{-1}:Y_{0}\to
X_{0}$. In particular, $j^{-1}:Y_{0}\to X$ and since $Y_{0}$ is a free generator
of $Y$ there exists a unique morphism $g':Y\to X$ such that $g'_{|Y_{0}}=j^{-1}$.
We will complete the proof by showing that $g'\circ g = id_{X}$ and $g\circ
g'=id_{Y}$. Since both $g:X\to Y$ and $g':Y\to X$ are morphisms, $g'\circ g:X\to
X$ is also a morphism. Furthermore:
    \[
    (g'\circ g)_{|X_{0}}=g'\circ (g_{|X_{0}})=g'\circ j = g'_{|Y_{0}}\circ j =
    j^{-1}\circ j=id_{X_{0}}
    \]
It follows that $g'\circ g:X\to X$ is a morphism which extends
$id_{X_{0}}:X_{0}\to X$. Since $X_{0}$ is a free generator of $X$, such morphism
is unique. As $id_{X}:X\to X$ is also a morphism such that
$(id_{X})_{|X_{0}}=id_{X_{0}}$ we conclude that $g'\circ g=id_{X}$. We prove
similarly that $g\circ g'=id_{Y}$. \end{proof}
