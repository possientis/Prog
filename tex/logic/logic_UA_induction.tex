We are all familiar with the notion of {\em proof by induction over
$\N$}. Let $\phi$ be a formula of first order logic and $u$ be a
variable. In order to prove that $\phi[n/u]$ is true for all
$n\in\N$, a {\em proof by induction over $\N$} consists in proving
first that $\phi[0/u]$ is true, and then proving the implication
$\phi[n/u]\to\phi[n+1/u]$ for all $n\in\N$. A {\em proof by
induction over $\N$} is legitimate as the following is a theorem:
    \begin{equation}\label{logic:eqn:induction:over:N}
    [\,\phi[0/u]\land(\forall n\in\N\ ,\
    \phi[n/u]\to\phi[n+1/u])\,]\to(\forall n\in\N\ ,\ \phi[n/u])
    \end{equation}
Hence for every formula $\phi$ of first order logic and every
variable $u$, we have a corresponding
theorem~(\ref{logic:eqn:induction:over:N}). This correspondence may
be called a {\em theorem schema}, in the same way that {\bf ZFC} has
a few {\em axiom schema}. If we denote $\texttt{Th}[\phi,u]$ the
theorem obtained in~(\ref{logic:eqn:induction:over:N}) from the
formula $\phi$ and the variable $u$, then the statement
$\forall\phi\forall u\texttt{Th}[\phi,u]$ cannot be represented as a
formula of first order logic, and is therefore not a {\em theorem}.
It may be called a {\em meta-theorem}, which is a part of {\em
meta-mathematics}. It is said that a meta-theorem can also become a
theorem, but only as part of a wider formal system. There is
something fundamentally disturbing about meta-theorems: most of us
do not understand them. There is always a high level of discomfort
when referring to {\em formulas of first order logic} and {\em
variables} without having defined any of those terms. It is also
problematic to use notations such as `$\phi[n/u]$' ($\phi$ with $n$
in place of $u$) which is most likely not always meaningful: some
form of restriction should probably be imposed on the variables $u$
and $n$, unless maybe $\phi$ is viewed as an equivalence class
rather than a string. The truth is we do not really know.

At the same time, there is something fascinating and magical about
meta-mathematics: one of our favorite meta-theorems states that
given a property $\phi[u]$, if there exists an ordinal $\beta$ such
that $\phi[\beta]$, then there exists a smallest ordinal with such
property. This is powerful. This is not something we want to give
up. And yet we hardly understand it, and it is not something we can
prove. Suppose we had a compiler which would transform a high level
mathematical statement into a formula of first order logic; suppose
we had defined the notion of {\em mathematical proof} as a finite
sequence of low level formulas satisfying certain properties. Then a
meta-theorem would not be compiled and would not be proved. But we
could use its {\em meta-proof} to {\em teach} our compiler to
automatically generate the appropriate fragment of code needed to
construct a proof of $\texttt{Th}[\phi,u]$ from the formula $\phi$.
There is light at the end of the tunnel.

In this section, we shall make little mention of the ordinals so as
to keep as wide an audience as possible. As it turns out, every
$n\in\N$ is an ordinal and the set $\N$ itself is an ordinal (also
denoted $\om$), but what really matters is the fact already used
before that every non-empty subset of $\N$ has a smallest element.
Now suppose we want to prove
theorem~(\ref{logic:eqn:induction:over:N}) given a formula $\phi$
and a variable $u$: we assume the left-hand-side
of~(\ref{logic:eqn:induction:over:N}) is true. We then consider the
set:
    \[
    A=\{n\in\N:\lnot\phi[n/u]\}
    \]
and we need to show that $A=\emptyset$. Note that the existence of
the set $A$ is a direct consequence of the Axiom Schema of
Comprehension which makes the formula:
    \[
    \forall\N\exists A[\forall n(n\in A\leftrightarrow (n\in\N)\land(\lnot\phi[n/u]))]
    \]
an axiom of {\bf ZFC}. Now suppose $A\neq\emptyset$. Then $A$ is a
non-empty subset of $\N$ which therefore has a smallest element, say
$n\in\N$. Since $\phi[0/u]$ is true, we must have $n\neq 0$. So
$n\geq 1$ and $n-1\in\N$. From the minimality of $n$, it follows
that $\phi[n-1/u]$ is true. From the implication
$\phi[n-1/u]\to\phi[n/u]$ we conclude that $\phi[n/u]$ is also true,
contradicting the fact that $n\in A$. So we have shown that
$A=\emptyset$, which completes the proof of
theorem~(\ref{logic:eqn:induction:over:N}).

Because our proof was somehow parameterized with the formula $\phi$,
we could call it a {\em meta-proof}. So we have just produced a
piece of meta-mathematics. However, this could have been avoided:
when dealing with mathematical induction over $\N$, there is no need
to consider a formula~$\phi$ of first order predicate logic.
Attempting to prove that a property is true for all $n\in\N$ is not
fundamentally different from proving that a particular subset of
$\N$, namely the subset on which the property is true coincide with
the whole of $\N$. So let $Y\subseteq\N$ be a subset of $\N$. In
order to prove that $Y=\N$, it is sufficient to show that $0\in Y$
and furthermore that $(n\in Y)\to(n+1\in Y)$ for all $n\in\N$.
Indeed, the following is a theorem:
    \begin{equation}\label{logic:eqn:induction:theorem}
    \forall Y\subseteq\N\ ,\ [\,(0\in Y)\land\forall n((n\in Y)
    \to (n+1\in Y))\,]\to (Y=\N)
    \end{equation}
No more meta-mathematics, no more guilt. We are back to the solid
grounds of standard mathematical arguments. We may be wrong and
deluded in our beliefs, but we certainly lose the awareness of it.
The proof of theorem~(\ref{logic:eqn:induction:theorem}) is
essentially the same but without the disturbing reference to the
formula $\phi$: suppose the complement of $Y$ in $\N$ is a non
empty-set. It has a smallest element $n$ which cannot be $0$. From
$n-1\in Y$ we obtain $n\in Y$ which is a contradiction.

We can now resume our study of universal algebras and deal with the
topic of {\em proof by structural induction}. Let $X$ be a universal
algebra of type $\alpha$ and suppose $X_{0}\subseteq X$ is a
generator of $X$. In order to prove that a property holds for all
$y\in X$, it is sufficient to prove that the property is true for
all $y\in X_{0}$ and furthermore given $f\in\alpha$ and $x\in
X^{\alpha(f)}$, that the property is also true for $f(x)$ whenever
it is true for all $x(i)$'s with $i\in\alpha(f)$. Just like in the
case of induction over $\N$, it would be possible to consider a
formula $\phi$ of first order predicate logic, and indulge in
meta-mathematics. Instead, we shall safely quote:
\index{induction@Proof by structural induction}
\begin{theorem}\label{logic:the:proof:induction}
Let $X$ be a universal algebra of type $\alpha$. Let $X_{0}\subseteq
X$ be a generator of $X$. Suppose $Y\subseteq X$ is a subset of $X$
with the following properties:
    \begin{eqnarray*}
    (i)&& x\in X_{0}\ \Rightarrow\ x\in Y\\
    (ii)&&(\forall i\in\alpha(f)\ ,\ x(i)\in Y)\ \Rightarrow\ f(x)\in Y
    \end{eqnarray*}
where $(ii)$ holds for all $f\in\alpha$ and $x\in X^{\alpha(f)}$.
Then $Y=X$.
\end{theorem}
\begin{proof}
We need to show that $Y=X$. From $(i)$ we see immediately that
$X_{0}\subseteq Y$. Since $X_{0}$ is a generator of $X$, we have
$\langle X_{0}\rangle =X$. Consequently, it is sufficient to prove
that $Y$ is a universal sub-algebra of $X$, as this would imply that
$\langle X_{0}\rangle\subseteq Y$, and finally $X\subseteq Y$. In
order to show that $Y$ is a universal sub-algebra of $X$, consider
$f\in\alpha$ and $x\in Y^{\alpha(f)}$. We need to prove that
$f(x)\in Y$. From $(ii)$, it is sufficient to show that $x(i)\in Y$
for all $i\in\alpha(f)$. This follows from $x\in Y^{\alpha(f)}$.
\end{proof}

Note that theorem~(\ref{logic:the:proof:induction}) does not require
$X$ to be a free universal algebra, but simply that it should have a
generator $X_{0}\subseteq X$. Furthermore, if $f\in\alpha$ is such
that $\alpha(f)=0$, then $(ii)$ reduces to $f(0)\in Y$. In other
words, in order to prove that a property holds for every element of
a universal algebra with constants, the least we can do is check
that every constant has the required property.
