In definition~(\ref{logic:def:strong:sub:congruence}) the strong
substitution congruence was defined in terms of a generator. This
makes it very convenient to show that two given formulas $\phi$ and
$\psi$ are equivalent. For instance, if $\phi=\forall x\forall
y(x\in y)$ and $\psi=\forall y\forall x(y\in x)$, if there is $u\in
V$ such that $u\not\in\{x,y\}$, the proof that $\phi\sim\psi$ would
go as follows:
    \begin{eqnarray*}
    \forall x\forall y(x\in y)&\sim&\forall u\forall y(u\in y)\\
    &\sim&\forall u\forall x(u\in x)\\
    &\sim&\forall y\forall x(y\in x)
    \end{eqnarray*}
The first and third equivalence stem directly from
definition~(\ref{logic:def:strong:sub:congruence}), as does the
equivalence $\forall y(u\in y)\sim\forall x(u\in x)$. Knowing that
$\sim$ is a congruent relation allows us to establish the second
equivalence, while knowing that $\sim$ is a transitive relation
allows us to conclude that $\phi\sim\psi$. More generally, the
knowledge of the generator $R_{0}$ of
definition~(\ref{logic:def:strong:sub:congruence}) gives us the
equivalence $\phi\sim\psi$ immediately, for many pairs
$(\phi,\psi)$. Using the congruent property, reflexivity, symmetry
and transitivity of the relation~$\sim$, we readily obtain this
equivalence for many more pairs. So showing that $\phi\sim\psi$ is
usually the easy part.

What is more difficult is proving that an equivalence $\phi\sim\psi$
does not hold. For instance, if $\phi=(x\in y)$ and $\psi=\bot$, we
would like to think that $\phi$ is not equivalent to $\psi$.
However, definition~(\ref{logic:def:strong:sub:congruence}) does not
give us an immediate tool to prove that $\phi\not\sim\psi$.
Fortunately, we showed in
proposition~(\ref{logic:prop:strong:freevar}) that if $\phi\sim\psi$
then we must have $\free(\phi)=\free(\psi)$. Since
$\free(\phi)=\{x,y\}$ while $\free(\psi)=\emptyset$ we are able in
this case to conclude that $\phi$ is not equivalent to $\psi$. But
if $\phi=(x\in y)$ while $\psi=(y\in x)$ with $x\neq y$, then the
implication $\phi\sim\psi\ \Rightarrow\ \free(\phi)=\free(\psi)$
does not help us to conclude that $\phi\not\sim\psi$. Somehow we
need something sharper.

So we need to prove an implication $\phi\sim\psi\ \Rightarrow\
\phi\simeq\psi$, having chosen a relation $\simeq$ which tells us a
bit more about the formulas $\phi$ and $\psi$, than the simple
$\free(\phi)=\free(\psi)$. In fact, we shall choose the relation
$\simeq$ so as to have the equivalence $\phi\sim\psi\
\Leftrightarrow\ \phi\simeq\psi$, as a way of ensuring that the
statement $\phi\simeq\psi$ tells us as much as possible about the
formulas $\phi$ and $\psi$. So before we prove anything, our first
task is to choose a sensible relation $\simeq$.

So let us assume that $\phi\sim\psi$. We know from
theorem~(\ref{logic:the:unique:representation}) of
page~\pageref{logic:the:unique:representation} that any formula of
first order predicate logic is either of the form $(x\in y)$, or is
the contradiction constant $\bot$, or is an implication
$\phi_{1}\to\phi_{2}$ or is a quantification $\forall x\phi_{1}$.
Let us review all these four cases in relation to $\phi$: so suppose
first that $\phi=(x\in y)$ with $\phi\sim\psi$. Clearly we would
expect the formula $\psi$ to be equal to the formula $\phi$. Suppose
now that $\phi=\bot$. Then we would expect the formula $\psi$ to be
equal to $\bot$. Suppose now that $\phi$ is an implication
$\phi=\phi_{1}\to\phi_{2}$. If $\phi\sim\psi$ we would also expect
the formula $\psi$ to be an implication $\psi=\psi_{1}\to\psi_{2}$.
Furthermore, we would expect to have $\phi_{1}\sim\psi_{1}$ and
$\phi_{2}\sim\psi_{2}$.

Suppose now that $\phi$ is a quantification $\phi=\forall
x\phi_{1}$. Then we would also expect the formula $\psi$ to be a
quantification  $\psi=\forall y\psi_{1}$. However, we would not
expect the variables $x$ and $y$ to be the same in general. If $x=y$
then we would expect to have $\phi_{1}\sim\psi_{1}$, but if $x\neq
y$ then the relationship between $\phi_{1}$ and $\psi_{1}$ should be
more complex. Informally, we would expect the formula $\psi_{1}$ to
be the {\em same mathematical statement} as the formula $\phi_{1}$,
but with the variable $x$ replaced by the variable $y$. Of course we
cannot hope to have $\psi_{1}=\phi_{1}[y/x]$ in general, as there
may have been other {\em changes of variable} within the formulas
$\phi_{1}$ and $\psi_{1}$. However, we would expect to have the
equivalence $\psi_{1}\sim\phi_{1}[y/x]$.

Unfortunately, this does not quite work. We already know from
definition~(\ref{logic:def:strong:sub:congruence}) that considering
the formula $\phi_{1}[y/x]$ is not very safe unless we have
$y\not\in\var(\phi_{1})$. We have made no mention of this fact so
far, so it is likely that the simple condition
$\psi_{1}\sim\phi_{1}[y/x]$ is doomed, without further
qualification. In fact, consider the case when $\phi=\forall
x\forall y (x\in y)$ and $\psi=\forall y\forall x(y\in x)$ with
$x\neq y$. Then $\phi_{1}=\forall y(x\in y)$ and $\psi_{1}=\forall
x(y\in x)$. So we obtain $\phi_{1}[y/x]=\forall y(y\in y)$ and we
certainly expect the equivalence $\psi_{1}\sim\phi_{1}[y/x]$ to be
false. So we need to look for something more complicated. Suppose
there exists $u\in V$ such that $u\not\in\{x,y\}$. Defining the
formula $\theta=\forall u (x\in u)$ the condition
$y\not\in\var(\theta)$ is now satisfied and we can safely consider
the formula $\theta[y/x]=\forall u(y\in u)$. From
definition~(\ref{logic:def:strong:sub:congruence}) we obtain
immediately $\phi_{1}\sim\theta$ and $\psi_{1}\sim\theta[y/x]$. So
this may be the relationship we are looking for between the formula
$\phi_{1}$ and the formula $\psi_{1}$: the existence of a third
formula $\theta$ such that $\phi_{1}\sim\theta$ and
$\psi_{1}\sim\theta[y/x]$, with the additional condition
$y\not\in\var(\theta)$. Note that we cannot hope to make this
relationship simpler in general. We have already established that
$\phi_{1}=\theta$ and $\psi_{1}\sim\theta[y/x]$, (or indeed
$\phi_{1}=\theta$ and $\psi_{1}=\theta[y/x]$) fails to work. The
condition $\phi_{1}\sim\theta$ and $\psi_{1}=\theta[y/x]$ does not
work in general either: from
proposition~(\ref{logic:prop:inplaceof:notvar}) we know that $x$ can
never be a variable of $\theta[y/x]$ when $x\neq y$, and
consequently we cannot hope to have the equality
$\psi_{1}=\theta[y/x]$ in the case when $\psi_{1}=\forall x(y\in
x)$. In the light of these comments, we can now venture a sensible
guess for our relation $\simeq$.

\begin{defin}\label{logic:def:almost:strong:equivalent}
Let $\sim$ be the strong substitution congruence on \pv\ where $V$ is a set. Let $\phi,\psi\in\pv$. We say that $\phi$ is {\em almost strongly equivalent to $\psi$} and we write $\phi\simeq\psi$, \ifand\ one of the following is the case:
    \begin{eqnarray*}
    (i)&&\phi\in\pvo\ ,\ \psi\in\pvo\ ,\ \mbox{and}\ \phi=\psi\\
    (ii)&&\phi=\bot\ \mbox{and}\ \psi=\bot\\
    (iii)&&\phi=\phi_{1}\to\phi_{2}\ ,\ \psi=\psi_{1}\to\psi_{2}\ ,\
    \phi_{1}\sim\psi_{1}\ \mbox{and}\ \phi_{2}\sim\psi_{2}\\
    (iv)&&\phi=\forall x\phi_{1}\ ,\ \psi=\forall x\psi_{1}\ \mbox{and}\ \phi_{1}\sim\psi_{1}\\
    (v)&&\phi=\forall x\phi_{1}\ ,\ \psi=\forall y\psi_{1}\ ,\ x\neq y\ ,\
    \phi_{1}\sim \theta\ ,\ \psi_{1}\sim \theta[y/x]\ ,\ y\not\in\var(\theta)
    \end{eqnarray*}
\end{defin}

It should be noted that $(v)$ of
definition~(\ref{logic:def:almost:strong:equivalent}) is a
notational shortcut for the more detailed statement $\phi=\forall
x\phi_{1}\ ,\ \psi=\forall y\psi_{1}\ ,\ x\neq y$ and {\em there
exists a formula} $\theta\in\pv$ such that $\phi_{1}\sim \theta\ ,\
\psi_{1}\sim \theta[y/x]\ ,\ y\not\in\var(\theta)$.

\begin{prop}
$(i),(ii),(iii),(iv), (v)$ of {\em
definition~(\ref{logic:def:almost:strong:equivalent})} are mutually
exclusive.
\end{prop}
\begin{proof}
This is an immediate consequence of
theorem~(\ref{logic:the:unique:representation}) of
page~\pageref{logic:the:unique:representation} applied to the free
universal algebra \pv\ with free generator \pvo, where a formula
$\phi\in\pv$ is either an element of \pvo, or the contradiction
constant $\phi=\bot$, or an implication $\phi=\phi_{1}\to\phi_{2}$,
or a quantification $\phi=\forall x\phi_{1}$, but cannot be equal to
any two of those things simultaneously. Since $(v)$ can only occur
with $x\neq y$, it also follows from
theorem~(\ref{logic:the:unique:representation}) that $(v)$ cannot
occur at the same time as $(iv)$.
\end{proof}

So we have defined our relation $\simeq$ and it remains to prove the
equivalence:
    \[
    \phi\sim\psi\ \Leftrightarrow\ \phi\simeq\psi
    \]
As we will soon discover, the difficult part is showing the
implication $\Rightarrow$. In order to do so, we shall need to show
that $\simeq$ is a congruence on \pv\ which contains the generator
$R_{0}$ of definition~(\ref{logic:def:strong:sub:congruence}). So we
start by proving $R_{0}\subseteq\,\simeq$\,.

\begin{prop}\label{logic:prop:almost:strong:contains:r0}
Let $\simeq$ be the almost strong equivalence relation on \pv\ where
$V$ is a set. Then $\simeq$ contains the generator $R_{0}$ of {\em
definition~(\ref{logic:def:strong:sub:congruence})}.
\end{prop}
\begin{proof}
Suppose $x,y\in V$ and $\phi_{1}\in\pv$ are such that $x\neq y$ and
$y\not\in\var(\phi_{1})$. Note that this cannot happen unless $V$
has at least two elements. We define $\phi=\forall x\phi_{1}$ and
$\psi=\forall y\,\phi_{1}[y/x]$. We need to show that
$\phi\simeq\psi$. We shall do so by proving that $(v)$ of
proposition~(\ref{logic:def:almost:strong:equivalent}) is the case.
Define  $\theta=\phi_{1}$ and $\psi_{1}=\phi_{1}[y/x]=\theta[y/x]$.
Since the strong substitution congruence $\sim$ is reflexive on \pv,
we have $\phi_{1}\sim\theta$ and $\psi_{1}\sim\theta[y/x]$. It
follows that $\phi=\forall x\phi_{1}$, $\psi=\forall y\psi_{1}$,
$x\neq y$, $\phi_{1}\sim\theta$, $\psi_{1}\sim\theta[y/x]$ and
$y\not\in\var(\theta)$.
\end{proof}

We now need to show that $\simeq$ is a congruence on \pv. In
particular, it is an equivalence relation, i.e. a relation on \pv\
which is reflexive, symmetric and transitive. We first deal with the
reflexivity.

\begin{prop}\label{logic:prop:almost:strong:reflexive}
Let $\simeq$ be the almost strong equivalence relation on \pv\ where
$V$ is a set. Then $\simeq$ is a reflexive relation on \pv.
\end{prop}
\begin{proof}
Let $\phi\in\pv$. We need to show that $\phi\simeq\phi$. From
theorem~(\ref{logic:the:unique:representation}) of
page~\pageref{logic:the:unique:representation} we know that $\phi$
is either an element of \pvo, or $\phi=\bot$ or
$\phi=\phi_{1}\to\phi_{2}$ or $\phi=\forall x\phi_{1}$ for some
$\phi_{1},\phi_{2}\in\pv$ and $x\in V$. We shall consider these four
mutually exclusive cases separately. Suppose first that
$\phi\in\pvo$: then from $\phi=\phi$ we obtain $\phi\simeq\phi$.
Suppose next that $\phi=\bot$: then it is clear that
$\phi\simeq\phi$. Suppose now that $\phi=\phi_{1}\to\phi_{2}$. Since
the strong substitution congruence $\sim$ is reflexive, we have
$\phi_{1}\sim\phi_{1}$ and $\phi_{2}\sim\phi_{2}$. It follows from
$(iii)$ of definition~(\ref{logic:def:almost:strong:equivalent})
that $\phi\simeq\phi$. Suppose finally that $\phi=\forall
x\phi_{1}$. From $\phi_{1}\sim\phi_{1}$ and $(iv)$ of
definition~(\ref{logic:def:almost:strong:equivalent}) we conclude
that $\phi\simeq\phi$. In all cases, we have proved that
$\phi\simeq\phi$.
\end{proof}


\begin{prop}\label{logic:prop:almost:strong:symmetric}
Let $\simeq$ be the almost strong equivalence relation on \pv\ where
$V$ is a set. Then $\simeq$ is a symmetric relation on \pv.
\end{prop}
\begin{proof}
Let $\phi,\psi\in\pv$ be such that $\phi\simeq\psi$. We need to show that $\psi\simeq\phi$. We shall consider the five possible cases of definition~(\ref{logic:def:almost:strong:equivalent}): suppose first that $\phi\in\pvo$, $\psi\in\pvo$ and $\phi=\psi$. Then it is clear that $\psi\simeq\phi$. Suppose next that $\phi=\bot$ and $\psi=\bot$. Then we also have $\psi\simeq\phi$. We now assume that $\phi=\phi_{1}\to\phi_{2}$ and $\psi=\psi_{1}\to\psi_{2}$ with $\phi_{1}\sim\psi_{1}$ and $\phi_{2}\sim\psi_{2}$. Since the strong substitution congruence on \pv\ is symmetric, we have $\psi_{1}\sim\phi_{1}$ and $\psi_{2}\sim\phi_{2}$. Hence we have $\psi\simeq\phi$. We now assume that $\phi=\forall x\phi_{1}$ and $\psi=\forall x\psi_{1}$ with $\phi_{1}\sim\psi_{1}$. Then once again by symmetry of the strong substitution congruence we have $\psi_{1}\sim\phi_{1}$ and consequently $\psi\simeq\phi$. We finally consider the last possible case of $\phi=\forall x\phi_{1}$, $\psi=\forall y\psi_{1}$ with $x\neq y$ and we assume the existence of $\theta\in\pv$ such that $\phi_{1}\sim\theta$, $\psi_{1}\sim\theta[y/x]$ and $y\not\in\var(\theta)$. Define $\theta^{*}=\theta[y/x]$. In order to show that $\psi\simeq\phi$ it is sufficient to prove that $\phi_{1}\sim\theta^{*}[x/y]$ and $x\not\in\var(\theta^{*})$. First we show that $x\not\in\var(\theta^{*})$. So we need to show that $x\not\in\var(\theta[y/x])$ which in fact follows immediately from proposition~(\ref{logic:prop:inplaceof:notvar}) and $x\neq y$. We now show that $\phi_{1}\sim\theta^{*}[x/y]=\theta[y/x][x/y]$. Since $\phi_{1}\sim\theta$ and the strong substitution congruence on \pv\ is a transitive relation, it is sufficient to prove that:
    \[
    \theta\sim\theta[y/x][x/y]
    \]
In fact, the strong substitution congruence being reflexive, it is
sufficient to prove that $\theta=\theta[y/x][x/y]$. From
proposition~(\ref{logic:prop:single:composition}), since
$y\not\in\var(\theta)$ we obtain $\theta[x/x]=\theta[y/x][x/y]$.
Since $[x/x]:V\to V$ is the identity mapping, we conclude from
proposition~(\ref{logic:prop:substitution:identity}) that
$\theta[x/x]=\theta$ and finally $\theta=\theta[y/x][x/y]$.
\end{proof}

Having shown that $\simeq$ is a reflexive and symmetric relation on
\pv, we now prove that it is also a transitive relation. This is by
far the most difficult result of this section as many technical
details need to be checked.

\begin{prop}\label{logic:prop:almost:strong:transitive}
Let $\simeq$ be the almost strong equivalence relation on \pv\ where
$V$ is a set. Then $\simeq$ is a transitive relation on \pv.
\end{prop}
\begin{proof}
Let $\phi,\psi$ and $\chi\in\pv$ be such that $\phi\simeq\psi$ and
$\psi\simeq\chi$. We need to show that $\phi\simeq\chi$. We shall
consider the five possible cases of
definition~(\ref{logic:def:almost:strong:equivalent}) in relation to
$\phi\simeq\psi$. Suppose first that $\phi,\psi\in\pvo$ and
$\phi=\psi$. Then from $\psi\simeq\chi$ we obtain $\psi,\chi\in\pvo$
and $\psi=\chi$. It follows that $\phi,\chi\in\pvo$ and $\phi=\chi$.
Hence we see that $\phi\simeq\chi$. We now assume that
$\phi=\psi=\bot$. Then from $\psi\simeq\chi$ we obtain
$\psi=\chi=\bot$. It follows that $\phi=\chi=\bot$ and consequently
$\phi\simeq\chi$. We now assume that $\phi=\phi_{1}\to\phi_{2}$ and
$\psi=\psi_{1}\to\psi_{2}$ with $\phi_{1}\sim\psi_{1}$ and
$\phi_{2}\sim\psi_{2}$. From $\psi\simeq\chi$ we obtain
$\chi=\chi_{1}\to\chi_{2}$ with $\psi_{1}\sim\chi_{1}$ and
$\psi_{2}\sim\chi_{2}$. The strong substitution congruence being
transitive, it follows that $\phi_{1}\sim\chi_{1}$ and
$\phi_{2}\sim\chi_{2}$. Hence we see that $\phi\simeq\chi$. We now
assume that $\phi=\forall x\phi_{1}$ and $\psi=\forall x\psi_{1}$
with $\phi_{1}\sim\psi_{1}$, for some $x\in V$. From
$\psi\simeq\chi$ only the cases $(iv)$ and $(v)$ of
definition~(\ref{logic:def:almost:strong:equivalent}) are possible.
First we assume that $(iv)$ is the case. Then $\chi=\forall
x\chi_{1}$ with $\psi_{1}\sim\chi_{1}$. The strong substitution
congruence being transitive, we obtain $\phi_{1}\sim\chi_{1}$ and
consequently $\phi\simeq\chi$. We now assume that $(v)$ is the case.
Then $\chi=\forall y\chi_{1}$ for some $y\in V$ with $x\neq y$, and
there exists $\theta\in\pv$ such that $\psi_{1}\sim\theta$,
$\chi_{1}\sim\theta[y/x]$ and $y\not\in\var(\theta)$. The strong
substitution congruence being transitive, we obtain
$\phi_{1}\sim\theta$ and consequently $\phi\simeq\chi$. It remains
to consider the last possible case of
definition~(\ref{logic:def:almost:strong:equivalent}). So we assume
that $\phi=\forall x\phi_{1}$ and $\psi=\forall y\psi_{1}$ with
$x\neq y$, and we assume the existence of $\theta\in\pv$ such that
$\phi_{1}\sim\theta$, $\psi_{1}\sim\theta[y/x]$ and
$y\not\in\var(\theta)$. From $\psi\simeq\chi$ only the cases $(iv)$
and $(v)$ of definition~(\ref{logic:def:almost:strong:equivalent})
are possible. First we assume that $(iv)$ is the case. Then
$\chi=\forall y\chi_{1}$ with $\psi_{1}\sim\chi_{1}$. The strong
substitution congruence being transitive, we obtain
$\chi_{1}\sim\theta[y/x]$ and consequently $\phi\simeq\chi$. We now
assume that $(v)$ is the case. Then $\chi=\forall z\chi_{1}$ for
some $z\in V$ with $y\neq z$, and there exists $\theta^{*}\in\pv$
such that $\psi_{1}\sim\theta^{*}$, $\chi_{1}\sim\theta^{*}[z/y]$
and $z\not\in\var(\theta^{*})$. We shall now distinguish two cases.
First we assume that $x=z$. Then $\phi=\forall x\phi_{1}$ and
$\chi=\forall x\chi_{1}$ and in order to show that $\phi\simeq\chi$
it is sufficient to prove that $\phi_{1}\sim\chi_{1}$. Since
$\phi_{1}\sim\theta$ and $\chi_{1}\sim\theta^{*}[z/y]$, using the
symmetry and transitivity of the strong substitution congruence, it
remains to show that $\theta\sim\theta^{*}[z/y]$. Having assumed
that $x=z$, we have to prove that $\theta\sim\theta^{*}[x/y]$. Since
$y\not\in\var(\theta)$, from
proposition~(\ref{logic:prop:single:composition}) we have
$\theta[y/x][x/y]=\theta[x/x]$ and from
proposition~(\ref{logic:prop:substitution:identity}) we obtain
$\theta[x/x]=\theta$. It follows that $\theta[y/x][x/y]=\theta$ and
it remains to prove that $\theta[y/x][x/y]\sim\theta^{*}[x/y]$.
Using proposition~(\ref{logic:prop:substitution:single:var}), it is
sufficient to prove that $\theta[y/x]\sim\theta^{*}$ and
$x\not\in\var(\theta[y/x])\cup\var(\theta^{*})$. The fact that
$\theta[y/x]\sim\theta^{*}$ follows from $\psi_{1}\sim\theta[y/x]$
and $\psi_{1}\sim\theta^{*}$, using the symmetry and transitivity of
the strong substitution congruence on \pv. The fact that
$x\not\in\var(\theta[y/x])$ follows from
proposition~(\ref{logic:prop:inplaceof:notvar}) and the assumption
$x\neq y$. The fact that $x\not\in\var(\theta^{*})$ follows from the
assumptions $x=z$ and $z\not\in\var(\theta^{*})$. This completes our
proof in the case when $x=z$. We now assume that $x\neq z$. So we
have $x\neq y$, $y\neq z$ and $x\neq z$, with $\phi=\forall
x\phi_{1}$, $\psi=\forall y\psi_{1}$ and $\chi=\forall z\chi_{1}$.
Furthermore, there exist $\theta$ and $\theta^{*}\in\pv$ such that
$\phi_{1}\sim\theta$, $\psi_{1}\sim\theta[y/x]$,
$\psi_{1}\sim\theta^{*}$ and $\chi_{1}\sim\theta^{*}[z/y]$. Finally,
we have $y\not\in\var(\theta)$ and $z\not\in\var(\theta^{*})$, and
we need to show that $\phi\simeq\chi$. Define $\eta=\theta[y/z]$. It
is sufficient to show that $\phi_{1}\sim\eta$ and
$\chi_{1}\sim\eta[z/x]$ with $z\not\in\var(\eta)$. The fact that
$z\not\in\var(\eta)$ is a direct consequence of
proposition~(\ref{logic:prop:inplaceof:notvar}) and $y\neq z$. So it
remains to show that $\phi_{1}\sim\eta$ and $\chi_{1}\sim\eta[z/x]$.
First we show that $\phi_{1}\sim\eta$. Since $\phi_{1}\sim\theta$
and $\eta=\theta[y/z]$, from the symmetry and transitivity of the
strong substitution congruence on \pv\, it is sufficient to prove
that $\theta[y/z]\sim\theta$, which itself follows from
proposition~(\ref{logic:prop:substitution:invariant}) provided we
show that $y\not\in\var(\theta)$ and $z\not\in\free(\theta)$. We
already know that $y\not\in\var(\theta)$, so it remains to prove
that $z\not\in\free(\theta)$. In order to do so, we shall first
prove the implication:
    \begin{equation}\label{logic:eqn:strong:transitivity:1}
    z\in\free(\theta)\ \Rightarrow\ z\in\free(\theta[y/x])
    \end{equation}
using proposition~(\ref{logic:prop:freevar:single:subst}) and the
assumption $y\not\in\var(\theta)$. In the case
$x\not\in\free(\theta)$, we obtain
$\free(\theta[y/x])=\free(\theta)$ and the
implication~(\ref{logic:eqn:strong:transitivity:1}) is clear. In the
case when $x\in\free(\theta)$, we obtain
$\free(\theta[y/x])=\free(\theta)\setminus\{x\}\cup\{y\}$, and the
implication~(\ref{logic:eqn:strong:transitivity:1}) follows
immediately from the fact that $z\neq x$. Having proved the
implication~(\ref{logic:eqn:strong:transitivity:1}), we can now show
that $z\not\in\free(\theta)$ by instead proving that
$z\not\in\free(\theta[y/x])$. Since $\psi_{1}\sim\theta[y/x]$ and
$\psi_{1}\sim\theta^{*}$ it follows from the symmetry and
transitivity of the strong substitution congruence on \pv\ that
$\theta[y/x]\sim\theta^{*}$. Thus, from
proposition~(\ref{logic:prop:strong:freevar}) we obtain
$\free(\theta[y/x])=\free(\theta^{*})$, and it is therefore
sufficient to prove that $z\not\in\free(\theta^{*})$, which follows
immediately from $z\not\in\var(\theta^{*})$ and
$\free(\theta^{*})\subseteq\var(\theta^{*})$, the latter being a
consequence of proposition~(\ref{logic:prop:FOPL:boundvar:free}).
This completes our proof of $\phi_{1}\sim\eta$. It remains to show
that $\chi_{1}\sim\eta[z/x]$. Since $\chi_{1}\sim\theta^{*}[z/y]$
and $\eta=\theta[y/z]$ we have to show:
    \begin{equation}\label{logic:eqn:strong:transitivity:2}
    \theta[y/z][z/x]\sim\theta^{*}[z/y]
    \end{equation}
We shall prove the
equivalence~(\ref{logic:eqn:strong:transitivity:2}) by proving the
following two results:
    \begin{equation}\label{logic:eqn:strong:transitivity:3}
    \theta[y/z][z/x]\sim\theta[y/z][z/x][x/y]
    \end{equation}
    \begin{equation}\label{logic:eqn:strong:transitivity:4}
    \theta^{*}[z/y]\sim\theta[y/x][x/z][z/y]
    \end{equation}
Suppose for now that (\ref{logic:eqn:strong:transitivity:3}) and
(\ref{logic:eqn:strong:transitivity:4}) have been proved. Comparing
with the equivalence~(\ref{logic:eqn:strong:transitivity:2}), it is
sufficient to show that:
    \begin{equation}\label{logic:eqn:strong:transitivity:5}
    \theta[y/z][z/x][x/y]=\theta[y/x][x/z][z/y]
    \end{equation}
So we shall now prove
equation~(\ref{logic:eqn:strong:transitivity:5}). From
proposition~(\ref{logic:prop:substitution:support}) it is sufficient
to show that the maps $[x/y]\circ[z/x]\circ[y/z]$ and
$[z/y]\circ[x/z]\circ[y/x]$ coincide on $\var(\theta)$. So let
$u\in\var(\theta)$. In particular $u\neq y$ and we have to show
that:
    \begin{equation}\label{logic:eqn:strong:transitivity:6}
    [x/y]\circ[z/x]\circ[y/z](u)=[z/y]\circ[x/z]\circ[y/x](u)
    \end{equation}
We shall distinguish three cases. First we assume that
$u\not\in\{x,y,z\}$. Then it is clear that the
equality~(\ref{logic:eqn:strong:transitivity:6}) holds. Next we
assume that $u=x$:
    \[
    [x/y]\circ[z/x]\circ[y/z](u)=[x/y]\circ[z/x](u)=[x/y](z)=z
    \]
and:
    \[
    [z/y]\circ[x/z]\circ[y/x](u) = [z/y]\circ[x/z](y)=[z/y](y)=z
    \]
So the equality~(\ref{logic:eqn:strong:transitivity:6}) holds.
Finally we assume that $u=z$:
    \[
    [x/y]\circ[z/x]\circ[y/z](u)=[x/y]\circ[z/x](y)=[x/y](y)=x
    \]
and:
    \[
    [z/y]\circ[x/z]\circ[y/x](u) = [z/y]\circ[x/z](u)=[z/y](x)=x
    \]
So the equality~(\ref{logic:eqn:strong:transitivity:6}) holds again.
This completes our proof of~(\ref{logic:eqn:strong:transitivity:6})
and~(\ref{logic:eqn:strong:transitivity:5}). It remains to show that
the equivalence~(\ref{logic:eqn:strong:transitivity:3})
and~(\ref{logic:eqn:strong:transitivity:4}) are true. First we show
the equivalence~(\ref{logic:eqn:strong:transitivity:3}). Using
proposition~(\ref{logic:prop:substitution:invariant}), it is
sufficient to show that we have both
$x\not\in\var(\theta[y/z][z/x])$ and
$y\not\in\free(\theta[y/z][z/x])$. The fact that
$x\not\in\var(\theta[y/z][z/x])$ is a consequence of
proposition~(\ref{logic:prop:inplaceof:notvar}) and $x\neq z$. So we
need to show that $y\not\in\free(\theta[y/z][z/x])$. We shall do so
by applying proposition~(\ref{logic:prop:freevar:single:subst}) from
which we obtain, provided we show $z\not\in\var(\theta[y/z])$:
    \begin{equation}\label{logic:eqn:strong:transitivity:7}
    \free(\theta[y/z][z/x])=\left\{\begin{array}{lcl}
    \free(\theta[y/z])\setminus\{x\}\cup\{z\}&\mbox{\ if\ }&x\in\free(\theta[y/z])\\
    \free(\theta[y/z])&\mbox{\ if\ }&x\not\in\free(\theta[y/z])
    \end{array}
    \right.
    \end{equation}
However, before we can use
equation~(\ref{logic:eqn:strong:transitivity:7}) we need to check
$z\not\in\var(\theta[y/z])$, which in fact immediately follows from
proposition~(\ref{logic:prop:inplaceof:notvar}) and $y\neq z$.
Having justified equation~(\ref{logic:eqn:strong:transitivity:7}),
it is now clear that in order to prove
$y\not\in\free(\theta[y/z][z/x])$, it is sufficient to prove that
$y\not\in\free(\theta[y/z])$. Since $y\not\in\var(\theta)$, we can
apply proposition~(\ref{logic:prop:freevar:single:subst}) once more,
and having already shown $z\not\in\free(\theta)$ while proving the
equivalence $\phi_{1}\sim\eta$, we obtain
$\free(\theta[y/z])=\free(\theta)$. Hence, it is sufficient to prove
that $y\not\in\free(\theta)$ which follows immediately from
$y\not\in\var(\theta)$ and $\free(\theta)\subseteq\var(\theta)$.
This completes our proof of the
equivalence~(\ref{logic:eqn:strong:transitivity:3}). It remains to
show that the equivalence~(\ref{logic:eqn:strong:transitivity:4}) is
true. Using proposition~(\ref{logic:prop:substitution:single:var}),
it is sufficient to prove that $\theta^{*}\sim\theta[y/x][x/z]$ and
$z\not\in\var(\theta^{*})\cup\var(\theta[y/x][x/z])$. We already
know that $z\not\in\var(\theta^{*})$, and
$z\not\in\var(\theta[y/x][x/z])$ is an immediate consequence of
proposition~(\ref{logic:prop:inplaceof:notvar}) and $x\neq z$. So we
need to show that $\theta^{*}\sim\theta[y/x][x/z]$. From
$\psi_{1}\sim\theta[y/x]$ and $\psi_{1}\sim\theta^{*}$ we obtain
$\theta[y/x]\sim\theta^{*}$ and it is therefore sufficient to prove
that $\theta[y/x]\sim\theta[y/x][x/z]$. Using
proposition~(\ref{logic:prop:substitution:invariant}), it is
sufficient to show that $x\not\in\var(\theta[y/x])$ and
$z\not\in\free(\theta[y/x])$. The fact that
$x\not\in\var(\theta[y/x])$ is an immediate consequence of
proposition~(\ref{logic:prop:inplaceof:notvar}) and $x\neq y$. So it
remains to show that $z\not\in\free(\theta[y/x])$, which in fact was
already proved in the course of proving the equivalence
$\phi_{1}\sim\eta$.
\end{proof}

Having shown that $\simeq$ is a reflexive, symmetric and transitive
relation on \pv, it remains to prove that it is also a congruent
relation on \pv. However, this cannot be done before we show the
implication $\phi\simeq\psi\ \Rightarrow\ \phi\sim\psi$.

\begin{prop}\label{logic:prop:almost:strong:implies:strong}
Let $\simeq$ be the almost strong equivalence and $\sim$ be the
strong substitution congruence on \pv, where $V$ is a set. For all
$\phi,\psi\in\pv$:
    \[
    \phi\simeq\psi\ \Rightarrow\ \phi\sim\psi
    \]
\end{prop}
\begin{proof}
Let $\phi,\psi\in\pv$ such that $\phi\simeq\psi$. We need to show
that $\phi\sim\psi$. We shall consider the five possible cases of
definition~(\ref{logic:def:almost:strong:equivalent}) in relation to
$\phi\simeq\psi$. Suppose first that $\phi=\psi\in\pvo$. From the
reflexivity of the strong substitution congruence, it is clear that
$\phi\sim\psi$. Suppose next that $\phi=\psi=\bot$. Then we also
have $\phi\sim\psi$. We now assume that $\phi=\phi_{1}\to\phi_{2}$
and $\psi=\psi_{1}\to\psi_{2}$ where $\phi_{1}\sim\psi_{1}$ and
$\phi_{2}\sim\psi_{2}$. The strong substitution congruence being a
congruent relation on \pv, we obtain $\phi\sim\psi$. Next we assume
that $\phi=\forall x\phi_{1}$ and $\psi=\forall x\psi_{1}$ where
$\phi_{1}\sim\psi_{1}$ and $x\in\ V$. Again, the strong substitution
congruence being a congruent relation we obtain $\phi\sim\psi$.
Finally we assume that $\phi=\forall x\phi_{1}$ and $\psi=\forall
y\psi_{1}$ where $x\neq y$, and there exists $\theta\in\pv$ such
that $\phi_{1}\sim\theta$, $\psi_{1}\sim\theta[y/x]$ and
$y\not\in\var(\theta)$. Using once again the fact that the strong
substitution congruence is a congruent relation, we see that
$\phi\sim\forall x\,\theta$ and $\psi\sim\forall y\,\theta[y/x]$. By
symmetry and transitivity of the strong substitution congruence, it
is therefore sufficient to show that $\forall x\,\theta\sim\forall
y\,\theta[y/x]$ which follows immediately from $x\neq y$,
$y\not\in\var(\theta)$ and
definition~(\ref{logic:def:strong:sub:congruence}).
\end{proof}

We are now in a position to show that $\simeq$ is a congruent
relation, which is the last part missing before we can conclude that
$\simeq$ is a congruence on \pv.

\begin{prop}\label{logic:prop:almost:strong:congruent}
Let $\simeq$ be the almost strong equivalence relation on \pv\ where
$V$ is a set. Then $\simeq$ is a congruent relation on \pv.
\end{prop}
\begin{proof}
From proposition~(\ref{logic:prop:almost:strong:reflexive}), the
almost strong equivalence $\simeq$ is reflexive and so
$\bot\simeq\bot$. We now assume that $\phi=\phi_{1}\to\phi_{2}$ and
$\psi=\psi_{1}\to\psi_{2}$ where $\phi_{1}\simeq\psi_{1}$ and
$\phi_{2}\simeq\psi_{2}$. We need to show that $\phi\simeq\psi$.
However from
proposition~(\ref{logic:prop:almost:strong:implies:strong}) we have
$\phi_{1}\sim\psi_{1}$ and $\phi_{2}\sim\psi_{2}$ and it follows
from definition~(\ref{logic:def:almost:strong:equivalent}) that
$\phi\simeq\psi$. We now assume that $\phi=\forall x\phi_{1}$ and
$\psi=\forall x\psi_{1}$ where $\phi_{1}\simeq\psi_{1}$ and $x\in
V$. We need to show that $\phi\simeq\psi$. Once again from
proposition~(\ref{logic:prop:almost:strong:implies:strong}) we have
$\phi_{1}\sim\psi_{1}$ and consequently from
definition~(\ref{logic:def:almost:strong:equivalent}) we obtain
$\phi\simeq\psi$.
\end{proof}


\begin{prop}\label{logic:prop:almost:strong:congruence}
Let $\simeq$ be the almost strong equivalence relation on \pv\ where
$V$ is a set. Then $\simeq$ is a congruence on \pv.
\end{prop}
\begin{proof}
We need to show that $\simeq$ is reflexive, symmetric, transitive
and that it is a congruent relation on \pv. From
proposition~(\ref{logic:prop:almost:strong:reflexive}), the
relation~$\simeq$ is reflexive. From
proposition~(\ref{logic:prop:almost:strong:symmetric}) it is
symmetric while from
proposition~(\ref{logic:prop:almost:strong:transitive}) it is
transitive. Finally from
proposition~(\ref{logic:prop:almost:strong:congruent}) the
relation~$\simeq$ is a congruent relation.
\end{proof}

So $\simeq$ is a congruence on \pv\ such that
$R_{0}\subseteq\,\simeq$. We conclude this section with the
equivalence $\phi\sim\psi\ \Leftrightarrow\ \phi\simeq\psi$ and
summarize with
theorem~(\ref{logic:the:strong:sub:congruence:charac}) below.

\begin{prop}\label{logic:prop:almost:strong:is:strong}
Let $\simeq$ be the almost strong equivalence and $\sim$ be the
strong substitution congruence on \pv, where $V$ is a set. For all
$\phi,\psi\in\pv$:
    \[
    \phi\simeq\psi\ \Leftrightarrow\ \phi\sim\psi
    \]
\end{prop}
\begin{proof}
From proposition~(\ref{logic:prop:almost:strong:implies:strong}) it
is sufficient to show the implication $\Leftarrow$ or equivalently
the inclusion $\sim\,\subseteq\,\simeq\,$. Since $\sim$ is the
strong substitution congruence on \pv, it is the smallest congruence
on \pv\ which contains the set $R_{0}$ of
definition~(\ref{logic:def:strong:sub:congruence}). In order to show
the inclusion $\sim\,\subseteq\,\simeq$ it is therefore sufficient
to show that $\simeq$ is a congruence on \pv\ such that
$R_{0}\subseteq\,\simeq$. The fact that it is a congruence stems
from proposition~(\ref{logic:prop:almost:strong:congruence}). The
fact that $R_{0}\subseteq\,\simeq$ follows from
proposition~(\ref{logic:prop:almost:strong:contains:r0}).
\end{proof}

We are now ready to provide a characterization of the strong
substitution congruence. It should be noted that $(v)$ of
theorem~(\ref{logic:the:strong:sub:congruence:charac}) below is a
notational shortcut for the more detailed statement $\phi=\forall
x\phi_{1}\ ,\ \psi=\forall y\psi_{1}\ ,\ x\neq y$ and {\em there
exists a formula} $\theta\in\pv$ such that $\phi_{1}\sim \theta\ ,\
\psi_{1}\sim \theta[y/x]\ ,\ y\not\in\var(\theta)$.
\index{congruence@Charact. of strong congruence}
\begin{theorem}\label{logic:the:strong:sub:congruence:charac}
Let $\sim$ be the strong substitution congruence on \pv\ where $V$
is a set. For all $\phi,\psi\in\pv$, $\phi\sim\psi$ \ifand\ one of
the following is the case:
    \begin{eqnarray*}
    (i)&&\phi\in\pvo\ ,\ \psi\in\pvo\ ,\ \mbox{and}\ \phi=\psi\\
    (ii)&&\phi=\bot\ \mbox{and}\ \psi=\bot\\
    (iii)&&\phi=\phi_{1}\to\phi_{2}\ ,\ \psi=\psi_{1}\to\psi_{2}\ ,\
    \phi_{1}\sim\psi_{1}\ \mbox{and}\ \phi_{2}\sim\psi_{2}\\
    (iv)&&\phi=\forall x\phi_{1}\ ,\ \psi=\forall x\psi_{1}\ \mbox{and}\ \phi_{1}\sim\psi_{1}\\
    (v)&&\phi=\forall x\phi_{1}\ ,\ \psi=\forall y\psi_{1}\ ,\ x\neq y\ ,\
    \phi_{1}\sim \theta\ ,\ \psi_{1}\sim \theta[y/x]\ ,\ y\not\in\var(\theta)
    \end{eqnarray*}
\end{theorem}
\begin{proof}
Immediately follows from
proposition~(\ref{logic:prop:almost:strong:is:strong}) and
definition~(\ref{logic:def:almost:strong:equivalent}).
\end{proof}
