The notion of {\em specific variable} of a proof should now be
natural to us . To every proof $\pi$ is associated a set of
hypothesis $\hyp(\pi)$ and a set of axioms $\ax(\pi)$. Every formula
$\phi\in\hyp(\pi)$ has a set of free variables $\free(\phi)$, and so
has every formula $\phi\in\ax(\pi)$. In
definition~(\ref{logic:def:FOPL:proof:free:variable}) we decided to
call a {\em specific variable of $\pi$} only those variables which
are free variables of some $\phi\in\hyp(\pi)$. The {\em specific}
variables are those which are not {\em general}. This allowed us to
conveniently express the idea of {\em legitimate use of
generalization} in $\gen x\pi_{1}$, with the simple formalism
$x\not\in\spec(\pi_{1})$. In this section, we provide a few
elementary properties of the set $\spec(\pi)$. We start with a
structural definition of $\spec(\pi)$\,:
\index{specific@Specific
variables of proof}
\begin{prop}\label{logic:prop:FUAP:freevar:recursive:def}
Let $V$ be a set. The specific variable map $\spec:\pvs\to {\cal
P}(V)$ of {\em
definition~(\ref{logic:def:FOPL:proof:free:variable})} satisfies the
following structural recursion equation:
 \[
    \forall\pi\in\pvs\ ,\ \spec(\pi)=\left\{
                    \begin{array}{lcl}
                    \free(\phi)&\mbox{\ if\ }&\pi=\phi\in\pv\\
                    \emptyset&\mbox{\ if\ }&\pi=\axi\phi\\
                    \spec(\pi_{1})\cup\spec(\pi_{2}) &\mbox{\ if\ }&\pi=\pi_{1}\pon\pi_{2}\\
                    \spec(\pi_{1})&\mbox{\ if\ }&\pi=\gen x\pi_{1}
                    \end{array}\right.
    \]
\end{prop}
\begin{proof}
Using definition~(\ref{logic:def:FOPL:proof:free:variable}) we have
$\spec(\pi)=\cup\{\free(\phi)\ :\ \phi\in\hyp(\pi)\}$. In the case
when $\pi=\phi$ for $\phi\in\pv$, since $\hyp(\pi)=\{\phi\}$ we
obtain immediately $\spec(\pi)=\free(\phi)$. In the case when
$\pi=\axi\phi$ for $\phi\in\pv$, since $\hyp(\pi)=\emptyset$ we
obtain $\spec(\pi)=\emptyset$. When $\pi=\pi_{1}\pon\pi_{2}$ for
some $\pi_{1},\pi_{2}\in\pvs$ we have:
    \begin{eqnarray*}
    \spec(\pi)&=&\spec(\pi_{1}\pon\pi_{2})\\
    &=&\cup\{\,\free(\phi)\ :\ \phi\in\hyp(\pi_{1}\pon\pi_{2})\,\}\\
    &=&\cup\{\,\free(\phi)\ :\ \phi\in\hyp(\pi_{1})\cup\hyp(\pi_{2})\,\}\\
    &=&\cup(\,\{\free(\phi):\phi\in\hyp(\pi_{1})\}\cup\{\free(\phi):\phi\in\hyp(\pi_{2})\}\,)\\
    &=&(\cup\{\free(\phi):\phi\in\hyp(\pi_{1})\})\,\cup\,(\cup\{\free(\phi):\phi\in\hyp(\pi_{2})\})\\
    &=&\spec(\pi_{1})\cup\spec(\pi_{2})
    \end{eqnarray*}
Finally if $\pi=\gen x\pi_{1}$ for some $x\in V$ and
$\pi_{1}\in\pvs$ we obtain:
    \begin{eqnarray*}
    \spec(\pi)&=&\spec(\gen x\pi_{1})\\
    &=&\cup\{\,\free(\phi)\ :\ \phi\in\hyp(\gen x\pi_{1})\,\}\\
    &=&\cup\{\,\free(\phi)\ :\ \phi\in\hyp(\pi_{1})\,\}\\
    &=&\spec(\pi_{1})
    \end{eqnarray*}
\end{proof}

The map $\spec:\pvs\to{\cal P}(V)$ is increasing with respect to the
inclusion on ${\cal P}(V)$. So the specific variables of a sub-proof
are specific variables of the proof.

\begin{prop}\label{logic:prop:FUAP:freevar:subformula}
Let $V$ be a set and $\rho,\pi\in\pvs$. Then we have:
    \[
    \rho\preceq\pi\ \Rightarrow\ \spec(\rho)\subseteq\spec(\pi)
    \]
\end{prop}
\begin{proof}
Suppose $\rho\preceq\pi$ is a sub-proof of $\pi\in\pvs$. Then we
have:
    \begin{eqnarray*}
    \spec(\rho)&=&\free(\hyp(\rho))\\
    &=&\cup\{\free(\phi)\ :\ \phi\in\hyp(\rho)\}\\
    \mbox{prop.~(\ref{logic:prop:FUAP:hypothesis:subformula})}\ \rightarrow
    &\subseteq&\cup\{\free(\phi)\ :\ \phi\in\hyp(\pi)\}\\
    &=&\free(\hyp(\pi))\\
    &=&\spec(\pi)
    \end{eqnarray*}
\end{proof}

Let $\sigma:V\to W$ be a map and $\pi\in\pvs$ be a proof. Then
$\sigma(\pi)$ is a proof whose specific variables are always images
of specific variables of $\pi$ by $\sigma$.

\begin{prop}\label{logic:prop:FUAP:freevar:substitution}
Let $V, W$ be sets and $\sigma:V\to W$ be a map. Let $\pi\in\pvs$\,:
    \[
    \spec(\sigma(\pi))\subseteq\sigma(\spec(\pi))
    \]
where $\sigma:\pvs\to{\bf\Pi}(W)$ also denotes the proof
substitution mapping.
\end{prop}
\begin{proof}
There is no induction required in this case. Using
definition~(\ref{logic:def:FOPL:proof:free:variable}) we have:
    \begin{eqnarray*}
    \spec(\sigma(\pi))&=&\free(\,\hyp(\sigma(\pi))\,)\\
    \mbox{prop.~(\ref{logic:prop:FUAP:substitution:hypothesis})}\ \rightarrow
    &=&\free(\,\sigma(\hyp(\pi))\,)\\
    &=&\cup\{\,\free(\psi)\ :\ \psi\in\sigma(\hyp(\pi))\,\}\\
    &=&\cup\{\,\free(\sigma(\phi))\ :\ \phi\in\hyp(\pi)\,\}\\
    \mbox{prop.~(\ref{logic:prop:freevar:of:substitution:inclusion})}\ \rightarrow
    &\subseteq&\cup\{\,\sigma(\free(\phi))\ :\ \phi\in\hyp(\pi)\,\}\\
    &=&\sigma(\,\cup\{\,\free(\phi)\ :\ \phi\in\hyp(\pi)\,\}\,)\\
    &=&\sigma(\,\free(\hyp(\pi))\,)\\
    &=&\sigma(\spec(\pi))
    \end{eqnarray*}
\end{proof}

A specific variable of $\pi$ is a variable of $\pi$. It had to be
said once:

\begin{prop}\label{logic:prop:FUAP:freevar:subset:variable}
Let $V$ be a set and $\pi\in\pvs$. Then we have:
    \[
    \spec(\pi)\subseteq\var(\pi)
    \]
\end{prop}
\begin{proof}
There is no induction required in this case. The proof goes as
follows:
    \begin{eqnarray*}
    \spec(\pi)&=&\cup\{\,\free(\phi)\ :\ \phi\in\hyp(\pi)\,\}\\
    \mbox{prop.~(\ref{logic:prop:FOPL:boundvar:free})}\ \rightarrow
    &\subseteq&\cup\{\,\var(\phi)\ :\ \phi\in\hyp(\pi)\,\}\\
    \val(\phi)=\phi\ \rightarrow
    &=&\cup\{\,\var(\val(\phi))\ :\ \phi\in\hyp(\pi)\,\}\\
    \mbox{prop.~(\ref{logic:prop:FUAP:hypothesis:charac})}\ \rightarrow
    &\subseteq&\cup\{\,\var(\val(\rho))\ :\ \rho\preceq\pi\,\}\\
    \mbox{prop.~(\ref{logic:prop:FUAP:variable:conclusions:sub:proof})}\ \rightarrow
    &\subseteq&\var(\pi)
    \end{eqnarray*}
\end{proof}
