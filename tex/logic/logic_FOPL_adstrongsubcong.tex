\begin{prop}\label{logic:prop:admissible:strong}
Let $\sim$ be the strong substitution congruence on \pv\ where $V$
is a set. Let $\phi\in\pv$ and $\sigma:V\to V$ be an admissible
substitution for $\phi$ such that $\var(\sigma(\phi))\neq V$. Then,
we have:
    \[
    \phi\sim\sigma(\phi)
    \]
\end{prop}
\begin{proof}
To be determined
\end{proof}

Proposition~(\ref{logic:prop:admissible:strong}) allows us to state
another characterization of the strong substitution congruence as
the following proposition shows. In
definition~(\ref{logic:def:strong:sub:congruence}), the strong
substitution congruence was defined in terms of a generator:
 \[
    R_{0}=\{(\phi,\sigma(\phi))\ :\ \phi=\forall x\phi_{1}\ ,\
    \sigma=[y/x]\ ,x\neq y\ , \ y\not\in\var(\phi_{1})\}
 \]
As we shall soon discover, this generator is in fact a subset of the
set $R_{1}$ of ordered pairs $(\phi,\sigma(\phi))$ where $\sigma$ is
admissible for $\phi$ and such that $\var(\sigma(\phi))\neq V$.
Since we now know from
proposition~(\ref{logic:prop:admissible:strong}) that $R_{1}$ is
itself a subset of the strong substitution congruence, it follows
that $R_{1}$ is also a generator of the strong substitution
congruence, which we shall now prove formally:

\begin{prop}\label{logic:prop:strong:quant:congruence:from:admissible}
Let $V$ be a set. Then the strong substitution congruence on \pv\ is
also generated by the following set $R_{1}\subseteq \pv\times\pv$:
    \[
    R_{1}=\left\{\,(\,\phi\,,\,\sigma(\phi)\,):\phi\in\pv\ ,\
    \mbox{$\sigma:V\to V$ admissible for $\phi$}\ ,\
    \var(\sigma(\phi))\neq V\ \right\}
    \]
\end{prop}
\begin{proof}
Let $\sim$ denote the strong substitution congruence on \pv\ and
$\equiv$ be the congruence on \pv\ generated by $R_{1}$. We need to
show that $\sim=\equiv$. First we show that $\sim\subseteq\equiv$.
Since $\sim$ is the smallest congruence on \pv\ which contains the
set $R_{0}$ of definition~(\ref{logic:def:strong:sub:congruence}),
in order to prove $\sim\subseteq\equiv$ it is sufficient to prove
that $R_{0}\subseteq\equiv$. So let $\phi_{1}\in\pv$ and $x,y\in V$
be such that $x\neq y$ and $y\not\in\var(\phi_{1})$. Define
$\phi=\forall x\phi_{1}$ and $\psi=\forall y\,\phi_{1}[y/x]$. We
need to show that $\phi\equiv\psi$. The congruence $\equiv$ being
generated by $R_{1}$ it is sufficient to prove that $(\phi,\psi)\in
R_{1}$. However, if we define $\sigma:V\to V$ by setting
$\sigma=[y/x]$ we have:
    \[
    \psi=\forall
    y\,\phi_{1}[y/x]=\forall\sigma(x)\,\sigma(\phi_{1})=\sigma(\forall
    x\phi_{1})=\sigma(\phi)
    \]
Hence, in order to show $(\phi,\psi)\in R_{1}$, it is sufficient to
prove that $\sigma$ is an admissible substitution for $\phi$ and
furthermore that $\var(\sigma(\phi))\neq V$. The fact that
$\var(\sigma(\phi))\neq V$ follows immediately from
$x\not\in\var(\phi[y/x])$, which is itself a consequence of $x\neq
y$ and proposition~(\ref{logic:prop:inplaceof:notvar}). So we need
to show that $\sigma$ is an admissible substitution for $\phi$.
First we show that $\sigma_{|\var(\phi)}$ is an injective map. So
let $u,v\in\var(\phi)$ be such that $\sigma(u)=\sigma(v)$. We need
to show that $u=v$. Since $\sigma=[y/x]$, this is clearly the case
when both $u=x$ and $v=x$ or when both $u\neq x$ and $v\neq x$. So
we assume that $u=x$ and $v\neq x$. Then we obtain $y=v$ and in
particular $y\in\var(\phi)=\{x\}\cup\var(\phi_{1})$. Since $x\neq y$
it follows that $y\in\var(\phi_{1})$ which is a contradiction. We
show similarly that the case $u\neq x$ and $v=x$ leads to a
contradiction, which completes our proof of the injectivity of
$\sigma_{|\var(\phi)}$. It remains to show that $\sigma(u)=u$ for
all $u\in\free(\phi)$. So let
$u\in\free(\phi)=\free(\phi_{1})\setminus\{x\}$. Then in particular
$u\neq x$ and consequently $\sigma(u)=[y/x](u)=u$. We now show that
$\equiv\subseteq\sim$. Since $\equiv$ is the smallest congruence on
\pv\ which contains the set $R_{1}$, it is sufficient to show that
$R_{1}\subseteq\sim$. So let $\phi\in\pv$ and $\sigma:V\to V$ be an
admissible substitution for $\phi$ such that $\var(\sigma(\phi))\neq
V$. We need to show that $\phi\sim\sigma(\phi)$. But this follows
immediately from proposition~(\ref{logic:prop:admissible:strong}).
\end{proof}
