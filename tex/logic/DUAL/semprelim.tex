One of the fascinating aspects of mathematical logic, at least for
us, is a sense of renewal: we are starting all over again,
rebuilding mathematics from scratch, from the solid ground of
axiomatic set theory. Most living mathematicians have studied set
theory from a naive point of view, where formal reasoning is muddled
up with intuition. The dream is to eliminate the old intuitive
knowledge and to replace it with a new set of completely formal
deductions. We are of course still relying on the old school:
everything we have done so far is tainted with the doubt and
suspicion which surrounds inappropriate foundations. But we console
ourselves by labeling the work as {\em meta-mathematics}. It is the
lesser kind of mathematics, the one for which it is ok to be unsure.
The {\em true} mathematics will come later. It will be almighty and
powerful, or so we hope.

In the pursuit of complete certainty, the purely syntactic approach
of spelling out a formal language \pv, a set of axioms \av\ and a
deductive system \pvs, appears to be the way forward. Why should we
need anything more? Why bother with semantics and in particular
model theory? Surely the last thing we want is to mix up axiomatic
set theory with semantic arguments. These cannot be trusted. So why?
There is an immediate answer to this: we simply love it. We may look
down on meta-mathematics for its lack of appropriate foundations,
but we are still compelled to push it further. Although we claim
otherwise, we believe in it. In fact, we are pretty sure our
heuristic meta-mathematical arguments can be formalized, once the
proper framework is in place. The sequent $\Gamma\vdash\phi$ will
become a formula $\mbox{\bf
Prf}\,(\,\Gamma\,,\ulcorner\phi\urcorner)$. We may even be able to
design our own proof of G\"odel's incompleteness theorem following
Peter Smith~\cite{SmithGodel}. There are many fascinating results of
mathematical logic, which we hope one day to understand: {\bf AC} is
independent of {\bf ZF} and {\bf GCH} is independent of {\bf ZFC}.
Clearly we cannot stop just yet.

However, there is an even greater purpose to semantics: it is all
very nice to spell out a set of axioms and a deductive system. But
how do we know it makes any sense? An axiomatization of first order
logic relies on many choices. What do we have to validate these
choices? Yes, the {\em modus ponens} rule of inference seems pretty
reasonable, and the deduction theorem holds in full generality. But
what about our choice of axioms? For all we know, these axioms could
be inconsistent. We have no way to tell. Or there could be too few
of them, which would prevent us from formalizing standard
mathematical arguments. Somehow we need a form of validation, and
this is where model theory comes in: a {\em good} axiomatization of
first order logic should be {\em sound} and {\em complete}. This is
the least we can ask for. So we shall put it to the test. We are not
claiming that {\em soundness} and {\em completeness} are enough. We
certainly believe the deduction theorem should also be true while
many authors disagree (e.g. \cite{Ferenczi}, \cite{Hoyois},
\cite{Johnstone}, \cite{Kunen}, \cite{Monk}, \cite{Metamath},
\cite{Tourlakis}). If we are completely honest, what constitutes a
{\em good} axiomatization of first order logic is still an open
question.

So we need semantics and model theory. For those who remain
unconvinced, here is the final blow: semantics is arguably about the
concept of {\em truth}. We give birth to special maps $v:\pv\to 2$
which we call {\em valuations}. Whenever $v(\phi)=1$ we say that a
formula $\phi$ is {\em true}. Otherwise if $v(\phi)=0$, we say that
a formula $\phi$ is false. After waiting for so long, we seem to
have reached an answer to one of the most fascinating questions of
all: {\em what is the nature of mathematical truth}\,? This is
magical. It feels like we are playing God, looking at mathematics
{\em from above}. Assuming we are right and \pv\ is rich enough to
code the Riemann hypothesis as a formula $\phi$,  here we are
wondering whether $v(\phi)=1$. For so many years, the concept of
{\em truth} has eluded us. We remember the words of Andr\'e
Lichnerowicz telling us ``{\em I do not know what a true proposition
is, which is not provable}''\footnote{Je ne sais pas ce qu'est une
proposition vraie non d\'emontrable.}, and somehow we agreed with
him. We still do, despite learning about G\"odel. But things have
become messy. As children, we believed in absolute truth:
mathematics was its guardian. Everything was either true or false.
Physicists could gloat as much as they want about the {\em real
world}. They had no handle on the {\em truth}. Everything they did
was mere {\em modeling}. How sad. Things are no longer so simple:
the pre-eminence of euclidian geometry is gone and the continuum
hypothesis is neither true nor false. Andr\'e Lichnerowicz is right
and so is Kurt G\"odel: we need to get smarter. The physicists were
right all along.
