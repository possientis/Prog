Given a set $X_{0}$, we are now in a position to prove the existence
of a free universal algebra of type $\alpha$ with $X_{0}$ as free
generator. As already hinted in the previous section, the proof
essentially runs in three stages. We start using
proposition~(\ref{logic:prop:construction}) to construct a free
universal algebra $Y$ with free generator $Y_{0}=\{(0,x):x\in
X_{0}\}$. We then use lemma~(\ref{logic:lemma:pullback}) to obtain a
copy $X$ of $Y$ and a bijection $g:X\to Y$ which preserves the
bijection $j:X_{0}\to Y_{0}$. We finally define on $X$ the unique
structure of universal algebra of type $\alpha$ turning $g:X\to Y$
into an isomorphism.
\index{free@Existence of free universal
algebra}
\begin{theorem}\label{logic:the:main:existence}
Let $\alpha$ be a type of universal algebra and $X_{0}$ be an
arbitrary set. There exists a free universal algebra of type
$\alpha$ with free generator $X_{0}$. Such universal algebra is
unique up to isomorphism.
\end{theorem}
\begin{proof}
Uniqueness up to isomorphism is a direct consequence of
proposition~(\ref{logic:prop:isomorphic}). So we only need to prove
the existence. If we define $Y_{0}=\{(0,x):x\in X_{0}\}$, from
proposition~(\ref{logic:prop:construction}) we have a free universal
algebra $Y$ with free generator $Y_{0}$. Furthermore, the map
$j:X_{0}\to Y$ defined by $j(x)=(0,x)$ for all $x\in X_{0}$ is an
injective map. Using lemma~(\ref{logic:lemma:pullback}) there exist
a set $X$ and a bijection $g:X\to Y$ such that $X_{0}\subseteq X$
and $g_{|X_{0}}=j$. At this stage, the set $X$ is only a set, not a
universal algebra of type $\alpha$. However, using the bijection
$g:X\to Y$ we can easily carry over the structure from $Y$ onto $X$.
Specifically, we define a map $T$ with domain $\alpha$ by setting
$T(f):X^{\alpha(f)}\to X$ to be defined by:
    \begin{equation}\label{logic:eqn:carry:over}
    \forall x\in X^{\alpha(f)}\ ,\ T(f)(x) = g^{-1}\circ f\circ g(x)
    \end{equation}
where it is understood that '$f$' on the r.h.s. of this equation is
a shortcut for the operator $f:Y^{\alpha(f)}\to Y$ and '$g$' refers
to $g^{\alpha(f)}:X^{\alpha(f)}\to Y^{\alpha(f)}$, while $g^{-1}$ is
simply the inverse $g^{-1}:Y\to X$. So $T(f)$ as defined by
equation~(\ref{logic:eqn:carry:over}) is indeed a map
$T(f):X^{\alpha(f)}\to X$. It follows that $(X,T)$ is now a
universal algebra of type $\alpha$ and $X_{0}\subseteq X$. We shall
complete the proof of this theorem by showing that $X_{0}$ is a free
generator of $X$. Note that this is hardly surprising since
$j:X_{0}\to Y_{0}$ is a bijection which coincide with the bijection
$g:X\to Y$ on $X_{0}$, and we know that $Y_{0}$ is a free generator
of $Y$. In order to prove this formally, let $Z$ be a universal
algebra of type $\alpha$ and $h_{0}:X_{0}\to Z$ be a map. We need to
show that $h_{0}$ can be uniquely extended into a morphism $h:X\to
Z$. Before we do so, note that from
equation~(\ref{logic:eqn:carry:over}) we have for all $x\in
X^{\alpha(f)}$:
    \[
    g\circ T(f)(x) = f\circ g(x)
    \]
which shows that $g:X\to Y$ is not only a bijection, but also a
morphism, i.e. $g$ is an isomorphism. Now from $h_{0}:X_{0}\to Z$ we
obtain a map $h_{0}\circ j^{-1}:Y_{0}\to Z$. Since $Y_{0}$ is a free
generator of $Y$, there exists a morphism $h':Y\to Z$ such that
$h'_{|Y_{0}}=h_{0}\circ j^{-1}$. It follows that $h:X\to Z$ defined
by $h=h'\circ g$ is a morphism, and furthermore, we have:
    \[
    h_{|X_{0}}=h'\circ g_{|X_{0}}=h'\circ j=h'_{|Y_{0}}\circ
    j=h_{0}\circ j^{-1}\circ j=h_{0}
    \]
So we have proved the existence of a morphism $h:X\to Z$ such that
$h_{|X_{0}}=h_{0}$. It remains to check that $h:X\to Z$ is in fact
unique. So suppose $h'$ now refers to a morphism $h':X\to Z$ such
that $h'_{|X_{0}}=h_{0}$. Then $h\circ g^{-1}:Y\to Z$ and $h'\circ
g^{-1}:Y\to Z$ are two morphisms which coincide on $Y_{0}$, since:
    \[
    (h\circ g^{-1})_{|Y_{0}} = h_{|X_{0}}\circ j^{-1} = h_{0}\circ
    j^{-1}=h'_{|X_{0}}\circ j^{-1}=(h'\circ g^{-1})_{|Y_{0}}
    \]
It follows that $h\circ g^{-1} = h'\circ g^{-1}$, from the
uniqueness property of $Y_{0}$ being a free generator of $Y$. We
conclude that $h=h'$.
\end{proof}
