Let $X$ be a universal algebra of type $\alpha$ and $R_{0}\subseteq
X\times X$. Then $R_{0}$ is a relation on $X$. However, $R_{0}$ may
not have the right properties and in particular may not be a
congruence on $X$. In this section, we prove the existence of the
{\em congruence generated by $R_{0}$}, namely the smallest
congruence on $X$ containing $R_{0}$ in the sense of inclusion. The
notion of {\em congruence generated} by a subset of $X\times X$ is a
very handy way of creating interesting congruences on $X$. As we
shall see, it is possible to view a congruence on $X$ as a universal
sub-algebra of $X\times X$, provided the latter has been embedded
with the right structure of universal algebra. The congruence
generated by $R_{0}$ will be seen to be the universal sub-algebra of
$X\times X$ generated by $R_{0}$, as per
definition~(\ref{logic:def:generated:sub:algebra}).

Before we proceed, we shall review a few simple lemmas which show
that common properties of a relation such as reflexivity, symmetry,
transitivity and being a congruent relation, are in fact akin to
closure properties.
\begin{lemma}\label{logic:lemma:reflexive}
Let $X$ be a set and $T_{x}:(X\times X)^{0}\to X\times X$ be defined by:
    \[
    T_{x}(0) = (x,x)\ ,\ (\forall x\in X)
    \]
A relation $\sim$ is reflexive on $X$ if and only if it is closed
under $T_{x}$ for all $x\in X$.
\end{lemma}
\begin{proof}
Suppose $\sim$ is closed under $T_{x}$ for all $x\in X$. We need to
show that $\sim$ is a reflexive relation on $X$. So let $x\in X$. We
need to show that $x\sim x$.  Since $0\in (\sim)^{0}$, we obtain
$T_{x}(0)\in\sim$. Hence we see that $(x,x)\in\sim$ or equivalently
$x\sim x$. Conversely, suppose $\sim$ is a reflexive relation on
$X$. We need to show that $\sim$ is closed under $T_{x}$ for all
$x\in X$. So let $x\in X$.  We need to show that $\sim$ is closed
under $T_{x}$. So let $u\in(\sim)^{0}$. We need to show that
$T_{x}(u)\in\sim$. But $u=0$ and $T_{x}(0)=(x,x)$. So we need to
show that $(x,x)\in\sim$ or equivalently $x\sim x$ which follows
immediately from the reflexivity of $\sim$.
\end{proof}

If $X$ is a set and $z\in(X\times X)^{1}$ then $z$ is a map
$z:\{0\}\to X\times X$. So $z(0)$ is an ordered pair $(x,y)$. Recall
that the notation '$(x,y)$' may be used as a notational shortcut to
refer to the map $z:\{0\}\to X\times X$ defined by $z(0)=(x,y)$.

\begin{lemma}\label{logic:lemma:symmetric}
Let $X$ be a set and $\sigma:(X\times X)^{1}\to X\times X$ be defined by:
    \[
    \forall (x,y)\in (X\times X)^{1}\ ,\ \sigma(x,y) = (y,x)
    \]
A relation $\sim$ is symmetric on $X$ if and only if it is closed
under $\sigma$.
\end{lemma}
\begin{proof}
Suppose $\sim$ is closed under $\sigma$. We need to show that $\sim$
is a symmetric relation on $X$. So let $x$, $y\in X$ such that
$x\sim y$. We need to show that $y\sim x$. Since
$(x,y)\in(\sim)^{1}$, we obtain $\sigma(x,y)\in\sim$. Hence we see
that $(y,x)\in\sim$ or equivalently $y\sim x$. Conversely, suppose
$\sim$ is a symmetric relation on $X$. We need to show that $\sim$
is closed under $\sigma$. So let $(x,y)\in(\sim)^{1}$. We need to
show that $\sigma(x,y)\in\sim$. So we need to show that
$(y,x)\in\sim$ or equivalently $y\sim x$ which follows immediately
from the symmetry of $\sim$ and $x\sim y$
\end{proof}

If $X$ is a set and $z\in(X\times X)^{2}$ then $z$ is a map
$z:\{0,1\}\to X\times X$. So $z(0)$ is an ordered pair $(x,y)$ and
$z(1)$ is an ordered pair $(y',z)$. Recall that the notation
'$[(x,y),(y',z)]$' may be used as a notational shortcut to refer to
the map $z:\{0,1\}\to X\times X$ defined by $z(0)=(x,y)$ and
$z(1)=(y',z)$.

\begin{lemma}\label{logic:lemma:transitive}
Let $X$ be a set and $\tau:(X\times X)^{2}\to X\times X$ be defined by:
    \[
    \tau[(x,y),(y',z)]=\left\{
                    \begin{array}{lcr}
                    (x,y)&\mbox{\ if\ }&y\neq y'\\
                    (x,z)&\mbox{\ if\ }&y=y'
                    \end{array}
                    \right.
    \]
  A relation $\sim$ is transitive on $X$ if and only if it is closed under $\tau$.
\end{lemma}
\begin{proof}
Suppose $\sim$ is closed under $\tau$. We need to show that $\sim$
is a transitive relation on $X$. So let $x$, $y$, $z\in X$ such that
$x\sim y$ and $y\sim z$. We need to show that $x\sim z$. Since
$[(x,y),(y,z)]\in(\sim)^{2}$ we obtain $\tau[(x,y),(y,z)]\in\sim$.
Hence we see that $(x,z)\in\sim$ or equivalently $x\sim z$.
Conversely, suppose $\sim$ is a transitive relation on $X$. We need
to show that $\sim$ is closed under $\tau$. So let
$[(x,y),(y',z)]\in(\sim)^{2}$. Then $x\sim y$ and $y'\sim z$ and we
need to show that $\tau[(x,y),(y',z)]\in\sim$. If $y\neq y'$ then we
need to show that $(x,y)\in\sim$ which is true by assumption. If
$y=y'$ then $y\sim z$ and we need to show that $(x,z)\in\sim$ or
equivalently  $x\sim z$ which follows immediately from the
transitivity of $\sim$ and $x\sim y$ and $y\sim z$.
\end{proof}

Let $X$ be a set and $\pi_{0},\pi_{1}:X\times X\to X$ be the
projection mappings, namely the maps defined by $\pi_{0}(x,y)=x$ and
$\pi_{1}(x,y)=y$ for all $x,y\in X$. Following our well established
convention, for all $n\in\N$ recall that '$\pi_{0}$' may be used as
a notational shortcut for $\pi_{0}^{n}:(X\times X)^{n}\to X^{n}$
defined by $\pi_{0}^{n}(z)(i)=\pi_{0}(z(i))$ for all $i\in n$. A
similar comment obviously applies to $\pi_{1}$.
\begin{lemma}\label{logic:lemma:congruent}
Let $X$ be a universal algebra of type $\alpha$. Consider the
structure of universal algebra of type $\alpha$ on $X\times X$
defined by:
    \[
    T(f)(z) = (f\circ \pi_{0}(z),f\circ\pi_{1}(z))\ ,\ \forall z\in (X\times X)^{\alpha(f)}\ ,\ \forall f\in\alpha
    \]
where $\pi_{0},\pi_{1}:X\times X\to X$ are the projection mappings.
Then, a relation $\sim$ is a congruent relation on $X$, if and only
if it is closed under $T(f)$ for all $f\in\alpha$.
\end{lemma}
\begin{proof}
Note from definition~(\ref{logic:def:universal:sub:algebra}) that
$\sim$ being closed under $T(f)$ for all $f\in\alpha$ could equally
have been phrased as $\sim$ being a universal sub-algebra of
$X\times X$ viewed as a universal algebra of type $\alpha$. Note
also that given $z\in(X\times X)^{\alpha(f)}$ for $f\in\alpha$,
$\pi_{0}(z)$ and $\pi_{1}(z)$ are well-defined elements of
$X^{\alpha(f)}$ (specifically we have  $\pi_{0}(z)(i)=\pi_{0}(z(i))$
for all $i\in\alpha(f)$ with similar equalities for $\pi_{1}(z)$).
It follows that $f\circ\pi_{0}(z)$ and $f\circ\pi_{1}(z)$ are
well-defined elements of $X$ and we see that $T(f)$ is indeed a
well-defined operator $T(f):(X\times X)^{\alpha(f)}\to X\times X$.
We now proceed with the proof: suppose $\sim$ is closed under $T(f)$
for all $f\in\alpha$. We need to show that $\sim$ is a congruent
relation on $X$. So let $f\in\alpha$ and $x,y\in X^{\alpha(f)}$ and
suppose that $x\sim y$. We need to show that $f(x)\sim f(y)$. Define
$z\in (X\times X)^{\alpha(f)}$ by $z(i) = (x(i),y(i))$ for all
$i\in\alpha(f)$, being understood that $z=0$ is $\alpha(f)=0$. From
$x\sim y$, we have $x(i)\sim y(i)$ for all $i\in\alpha(f)$. It
follows that $z(i)\in\sim$ for all $i\in\alpha(f)$ and consequently
$z\in(\sim)^{\alpha(f)}$. Since $\sim$ is closed under $T(f)$ we
obtain $T(f)(z)\in\sim$. However, for all $i\in\alpha(f)$ we have
$\pi_{0}(z)(i)=\pi_{0}(z(i))=x(i)$ and therefore $\pi_{0}(z)=x$.
Similarly we have $\pi_{1}(z)=y$. It follows that:
    \[
    T(f)(z)=(f\circ\pi_{0}(z),f\circ\pi_{1}(z)) = (f(x),f(y))
    \]
From $T(f)(z)\in\sim$ we conclude that $f(x)\sim f(y)$. Conversely,
suppose $\sim$ is a congruent relation on $X$. We need to show that
$\sim$ is closed under $T(f)$ for all $f\in\alpha$. So let
$f\in\alpha$ and $z\in(\sim)^{\alpha(f)}$. We need to show that
$T(f)(z)\in\sim$. Define $x,y\in X^{\alpha(f)}$ by setting
$x=\pi_{0}(z)$ and $y=\pi_{1}(z)$. Then $T(f)(z)=(f(x),f(y))$ and it
remains to show that $f(x)\sim f(y)$. Since $\sim$ is a congruent
relation on $X$, this will be achieved by showing that $x\sim y$, or
equivalently that $x(i)\sim y(i)$ for all $i\in\alpha(f)$. However,
for all $i\in\alpha(f)$ we have $x(i)=\pi_{0}(z)(i) = \pi_{0}(z(i))$
and similarly $y(i)=\pi_{1}(z(i))$. It follows that
$z(i)=(x(i),y(i))$. Consequently, we only need to show that
$z(i)\in\sim$ for all $i\in\alpha(f)$. But this is an immediate
consequence of $z\in(\sim)^{\alpha(f)}$.
\end{proof}

\begin{theorem}\label{logic:the:generated:congruence}
Let $X$ be a universal algebra of type $\alpha$ and $R_{0}\subseteq
X\times X$. Then, there exists a unique smallest congruence $\sim$
on $X$ such that $R_{0}\subseteq\sim$.
\end{theorem}
\begin{proof}
First we show the uniqueness. Suppose $\sim$ and $\simeq$ are two
congruence on $X$ which are both the smallest congruence on $X$
containing $R_{0}$. Since $\simeq$ is a congruence on $X$ containing
$R_{0}$, from the minimality of $\sim$ we obtain
$\sim\,\subseteq\,\simeq$. Likewise, since $\sim$ is a congruence on
$X$ containing $R_{0}$, from the minimality of $\simeq$ we obtain
$\simeq\,\subseteq\,\sim$. We now show the existence. For all $x\in
X$, let  $T_{x}:(X\times X)^{0}\to X\times X$ be defined as in
lemma~(\ref{logic:lemma:reflexive}). Let $\sigma:(X\times X)^{1}\to
X\times X$ be defined as in lemma~(\ref{logic:lemma:symmetric}) and
$\tau:(X\times X)^{2}\to X\times X$ be defined as in
lemma~(\ref{logic:lemma:transitive}). For all $\alpha\in f$, let
$T(f):(X\times X)^{\alpha(f)}\to X\times X$ be defined as in
lemma~(\ref{logic:lemma:congruent}). Consider the sets
$\alpha_{0}=\{((0,x),0):x\in X\}$, $\alpha_{1}=\{((1,0),1)\}$,
$\alpha_{2}=\{((2,0),2)\}$ and
$\alpha_{3}=\{((3,f),\alpha(f)):f\in\alpha\}$. Note that these sets
are all maps with range inside $\N$. Furthermore, their domains are
pairwise disjoint. It follows that
$\alpha^{*}=\alpha_{0}\cup\alpha_{1}\cup\alpha_{2}\cup\alpha_{3}$ is
a map with range inside $\N$, i.e. $\rng(\alpha^{*})\subseteq\N$. In
other words, $\alpha^{*}$ is a type of universal algebra. Let
$T^{*}$ be the map with domain $\alpha^{*}$, such that
$T^{*}(f):(X\times X)^{\alpha^{*}(f)}\to X\times X$ for all
$f\in\alpha^{*}$ and $T^{*}(f)$ is defined as:
    \begin{equation}\label{logic:eqn:congruence:structure}
    T^{*}(f)=\left\{\begin{array}{lcl}
        T_{x}&\mbox{\ if\ }&f\in\alpha_{0}\ ,\ f=((0,x),0)\ ,\ x\in X\\
        \sigma&\mbox{\ if\ }&f\in\alpha_{1}\ ,\ f=((1,0),1)\\
        \tau&\mbox{\ if\ }&f\in\alpha_{2}\ ,\ f=((2,0),2)\\
        T(f')&\mbox{\ if\ }&f\in\alpha_{3}\ ,\ f=((3,f'),\alpha(f'))\ ,\ f'\in\alpha
    \end{array}\right.
    \end{equation}
Note that equation~(\ref{logic:eqn:congruence:structure}) is indeed
such that $T^{*}(f):(X\times X)^{\alpha^{*}(f)}\to X\times X$ for
all $f\in\alpha^{*}$. It follows that the ordered pair $(X\times
X,\,T^{*})$ is a universal algebra of type $\alpha^{*}$. Using
definition~(\ref{logic:def:generated:sub:algebra}), let
$\sim=\langle R_{0}\rangle$ be the universal sub-algebra of $X\times
X$ generated by $R_{0}\subseteq X\times X$. We shall complete the
proof of the theorem by showing that the relation $\sim$ has all the
requested properties. First we show that $\sim$ is a congruence on
$X$. Since $\sim$ is a universal sub-algebra of $X\times X$, it is
closed under every operator $T^{*}(f)$ for all $f\in\alpha^{*}$. In
particular, it is closed under $T_{x}$ for all $x\in X$. It follows
from lemma~(\ref{logic:lemma:reflexive}) that $\sim$ is a reflexive
relation on $X$. Similarly $\sim$ is closed under $\sigma$ and
$\tau$, and we see from lemma~(\ref{logic:lemma:symmetric}) and
lemma~(\ref{logic:lemma:transitive}) that $\sim$ is a symmetric and
transitive relation on $X$. Furthermore, $\sim$ is closed under
every operator $T(f')$ for all $f'\in\alpha$ and it follows from
lemma~(\ref{logic:lemma:congruent}) that $\sim$ is a congruent
relation on $X$. So we have proved that $\sim$ is an equivalence
relation on $X$ which is a congruent relation, i.e. that $\sim$ is a
congruence on $X$. From $\sim=\langle R_{0}\rangle$ we obtain
immediately $R_{0}\subseteq\sim$. It remains to show that $\sim$ is
the smallest congruence on $X$ with $R_{0}\subseteq\sim$. So we
assume that $R$ is a congruence on $X$ such that $R_{0}\subseteq R$.
We need to show that $\sim\subseteq R$. Since $\sim=\langle
R_{0}\rangle$ and $R_{0}\subseteq R$ it is sufficient to show that
$R$ is a universal sub-algebra of $X\times X$, as this will imply
$\langle R_{0}\rangle\subseteq R$ by virtue of
proposition~(\ref{logic:prop:generated:smallest}), and finally
$\sim\subseteq R$. So suppose $f\in\alpha^{*}$. We need to show that
$R$ is closed under $T^{*}(f)$. If $f\in\alpha_{0}$ then
$f=((0,x),0)$ for some $x\in X$ and we need to show that $R$ is
closed under $T_{x}$ which follows immediately from
lemma~(\ref{logic:lemma:reflexive}) and the reflexivity of $R$. If
$f\in\alpha_{1}$ then we need to show that $R$ is closed under
$\sigma$ which follows immediately from
lemma~(\ref{logic:lemma:symmetric}) and the symmetry of $R$. If
$f\in\alpha_{2}$ then we need to show that $R$ is closed under
$\tau$ which follows immediately from
lemma~(\ref{logic:lemma:transitive}) and the transitivity of $R$.
Finally, if $f\in\alpha_{3}$ then $f=((3,f'),\alpha(f'))$ for some
$f'\in\alpha$ and we need to show that $R$ is closed under $T(f')$,
which follows immediately from lemma~(\ref{logic:lemma:congruent})
and the fact that $R$ is a congruent relation on $X$. In all four
possible cases, we have proved that $R$ is closed under $T^{*}(f)$.
\end{proof}
\index{congruence@Congruence generated by set}
\begin{defin}\label{logic:def:generated:congruence}
Let $X$ be a universal algebra of type $\alpha$ and $R_{0}\subseteq
X\times X$. We call {\em congruence on $X$ generated by $R_{0}$} the
smallest congruence on $X$ containing~$R_{0}$.
\end{defin}
