In this section, we prove that a free generator is also a generator.
This is one case where considering the order $\om:X\to\N$ on a free
universal algebra $X$ can be seen to be very useful.
\begin{defin}\label{logic:def:generator}
Let $X$ be a universal algebra of type $\alpha$ and $X_{0}\subseteq
X$. We say that $X_{0}$ is a {\em generator} of $X$, \ifand\
$\langle X_{0}\rangle =X$.
\end{defin}
\begin{prop}\label{logic:prop:free:generator:generator}
Let $X$ be a universal algebra of type $\alpha$ and $X_{0}\subseteq
X$. If $X_{0}$ is a free generator of $X$, then it is also a
generator of $X$.
\end{prop}
\begin{proof}
Suppose $X_{0}$ is a free generator of $X$. We want to show that
$X_{0}$ is a generator of $X$, i.e. that $\langle X_{0}\rangle =X$.
Suppose to the contrary that there exists $y\in X$ such that
$y\not\in\langle X_{0}\rangle$. Let $\om:X\to\N$ be the order on the
free universal algebra $X$ and consider the set:
    \[
    A=\{\om(y):y\in X\ ,\ y\not\in\langle X_{0}\rangle\}
    \]
Then $A$ is a non-empty subset of $\N$ which therefore has a
smallest element. So assume that $y\not\in\langle X_{0}\rangle$
corresponds to the smallest element of $A$. Since
$X_{0}\subseteq\langle X_{0}\rangle$, it is clear that $y\not\in
X_{0}$. Since $X$ is a free universal algebra, from
theorem~(\ref{logic:the:unique:representation}) of
page~\pageref{logic:the:unique:representation} there exist
$f\in\alpha$ and $x\in X^{\alpha(f)}$ such that $y=f(x)$. It follows
from proposition~(\ref{logic:prop:order}) that:
    \[
    \om(y)=\om(f(x))=1+\max\{\om(x(i)):i\in\alpha(f)\}
    \]
In particular, we see that $\om(x(i))<\om(y)$ for all
$i\in\alpha(f)$. From the minimality of $\om(y)$ it follows that
$x(i)\in\langle X_{0}\rangle$ for all $i\in\alpha(f)$. So
$x\in\langle X_{0}\rangle^{\alpha(f)}$. Note that this conclusion
holds even if $\alpha(f)=0$, and the argument presented is perfectly
valid in that case. Now since $\langle X_{0}\rangle$ is a universal
sub-algebra of $X$, we conclude from $x\in\langle
X_{0}\rangle^{\alpha(f)}$ that $f(x)\in\langle X_{0}\rangle$. So we
have proved that $y\in\langle X_{0}\rangle$ which contradicts the
initial assumption.
\end{proof}
