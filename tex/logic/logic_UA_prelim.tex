In the following, $\N=\{0,1,2,\ldots\}$ denotes the set of non-negative
integers, while $\N^{*}=\{1,2,\ldots\}$ is the set of positive
integers. The integer $0$ is also the empty set $\emptyset$, and for
all $n\in\N^{*}$:
    \[
    n=\{0,1,\ldots,n-1\}
    \]
Recall that a {\em map} or {\em function} is a (possibly empty) set
of ordered pairs $f$ such that $y=y'$ whenever $(x,y)\in f$ and
$(x,y')\in f$. The {\em domain} of $f$ is the set:
    \[
    \dom(f)=\{x:\exists y, (x,y)\in f\}
    \]
while the {\em range} of $f$ is the set:
    \[
    \rng(f)=\{y:\exists x, (x,y)\in f\}
    \]
The notation $f:A\to B$
indicates that $f$ is a map with domain $A$
and whose range is a subset of $B$. In particular, $f$ is a subset
of the cartesian product:
    \[
    A\times B=\{(x,y):x\in A, y\in B\}
    \]
The set of all maps $f:A\to B$ is denoted $B^{A}$. Note that
$B^{\emptyset}=\{\emptyset\}$ as the empty set $\emptyset$ is the
only map $f:\emptyset\to B$. The formula
$B^{\emptyset}=\{\emptyset\}$ can equally be written $B^{0} = 1$
which makes it easy to remember, and holds even if $B=0$. If
$A\neq\emptyset$ then $\emptyset^{A}=\emptyset$ as there is no map
$f:A\to\emptyset$. The formula $\emptyset^{A}=\emptyset$ can equally
be written $0^{A}=0$ which is also easy to remember, but does not
hold if $A=0$. There is nothing deep or interesting about the empty
map $\emptyset:\emptyset\to B$, $B^{\emptyset}$ or $\emptyset^{A}$.
But they are often confusing and we should confront them at an early
stage.

Regarding the cartesian product $A\times B$, it is very common to
refer to it as $A^{2}$ whenever $A=B$. This may seem confusing at
first since $A^{2}$ is the set of maps $f:\{0,1\}\to A$, which is
not the same thing as the set of ordered pairs $(x,y)$ with $x,y\in
A$. However, it is natural to represent a map $f:\{0,1\}\to A$ by
the ordered pair $(f(0),f(1))$. If $x=f(0)$ and $y=f(1)$, then we
are effectively representing $f=\{(0,x),(1,y)\}$ by the ordered pair
$(x,y)$. It makes communication easier. In summary, when we refer to
$(x,y)$ as the element of $A\times A$, things are exactly as they
are. But if we are referring to $(x,y)$ as an element of $A^{2}$,
what we really mean is the map $f=\{(0,x),(1,y)\}$, i.e the maps
$f:\{0,1\}\to A$ such that $f(0)=x$ and $f(1)=y$. Note that the sets
$A$ and $A^{1}$ are also different.
