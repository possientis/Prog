Given a map $\sigma:V\to W$ we would like to prove the existence of
an {\em essential substitution mapping}, namely a map as defined
below. Recall that the  minimal transform and minimal extension can
be found in
definitions~(\ref{logic:def:FOPL:mintransform:transform})
and~(\ref{logic:def:FOPL:commute:minextensioon:map}):
\index{essential@Essential substitution of map}
\begin{defin}\label{logic:def:FOPL:esssubst:esssubst}
Let $V,W$ be sets and $\sigma:V\to W$ be a map. We call {\em
essential substitution mapping} associated with $\sigma$, any map
$\sigma^{*}:\pv\to{\bf P}(W)$ such that:
    \begin{equation}\label{logic:eqn:FOPL:esssubst:key}
    {\cal M}\circ\sigma^{*}=\bar{\sigma}\circ{\cal M}
    \end{equation}
where $\bar{\sigma}:\bar{V}\to\bar{W}$ is the minimal extension and
${\cal M}$ is the minimal transform.
\end{defin}

We cannot expect an essential substitution mapping to be uniquely
defined. If $\sim$ denotes the substitution congruence on ${\bf
P}(W)$ then from
theorem~(\ref{logic:the:FOPL:mintransfsubcong:kernel}) of
page~\pageref{logic:the:FOPL:mintransfsubcong:kernel}, any other map
$\tau^{*}:\pv\to{\bf P}(W)$ satisfying
$\tau^{*}(\phi)\sim\sigma^{*}(\phi)$ for all $\phi\in\pv$ will also
satisfy equation~(\ref{logic:eqn:FOPL:esssubst:key}). However, if
$\sigma^{*}:\pv\to{\bf P}(W)$ does indeed satisfy
equation~(\ref{logic:eqn:FOPL:esssubst:key}), then for all
$\phi\in\pv$ the equivalence class of $\sigma^{*}(\phi)$ modulo the
substitution congruence is uniquely determined. This is the best we
can do. The notion of an {\em essential substitution mapping}
$\sigma^{*}$ associated with a map $\sigma$ we feel plays an
important role. If $\phi\in\pv$ is such that $\sigma$ is valid for
$\phi$, then it follows from
theorem~(\ref{logic:the:FOPL:commute:mintransform:validsub}) of
page~\pageref{logic:the:FOPL:commute:mintransform:validsub} that
$\sigma^{*}(\phi)\sim\sigma(\phi)$ where $\sigma:\pv\to{\bf P}(W)$
denotes the usual variable substitution mapping of
definition~(\ref{logic:def:substitution}). So $\sigma^{*}$ can be
viewed as a natural extension of $\sigma:\pv\to{\bf P}(W)$ which
thanks to equation~(\ref{logic:eqn:FOPL:esssubst:key}), assigns a
meaningful formula $\sigma^{*}(\phi)$ even in the case when $\sigma$
is not a valid substitution for $\phi$. This fundamentally allows us
to write $\forall x\phi\to\phi[y/x]^{*}$ as a valid instance of the
specialization axioms for all $y\in V$, without having to worry
whether the substitution $[y/x]$ is valid for $\phi$. For example,
if $\phi=\forall y(x\in y)$ with $x\neq y$, we shall now be able to
claim that $\forall x\phi\to\forall x(y\in x)$ is a legitimate
axiom. So our aim is to prove the existence of an essential
substitution mapping $\sigma^{*}:\pv\to{\bf P}(W)$ associated with
$\sigma$, at least when $W$ is a large enough set. Given
$\phi\in\pv$ the first step in our proof is to show the existence of
a formula $\psi\in{\bf P}(W)$ such that $\bar{\sigma}\circ{\cal
M}(\phi)={\cal M}(\psi)$. As already discussed in the previous
section, we cannot hope to obtain the existence of $\psi$ without
some condition on the set $W$. We conjectured the substitution rank
of $\bar{\sigma}\circ{\cal M}(\phi)$ had to be smaller than $|W|$,
i.e. $\rnk(\,\bar{\sigma}\circ{\cal M}(\phi)\,)\leq|W|$. So assuming
this condition holds, we are setting out to prove the existence of
$\psi$ which is the object of
theorem~(\ref{logic:the:FOPL:esssubst:key}) below. Until now, most
of our results have been obtained by structural induction arguments.
In this case, structural induction does not seem to work. It appears
the notion of {\em substitution rank} does not lend itself easily to
induction, and we shall need to find some other route. So let us go
back to the simple case when $\phi=\forall y(x\in y)$ with $x\neq
y$. Assuming $\sigma(x)=u$ we obtain $\bar{\sigma}\circ{\cal
M}(\phi)=\forall\,0(u\in 0)\in{\bf P}(\bar{W})$. Our first objective
will be to show the existence of some $\psi_{1}\in{\bf P}(\bar{W})$
such that $\var(\psi_{1})\subseteq W$ and $\bar{\sigma}\circ{\cal
M}(\phi)\sim\psi_{1}$ where $\sim$ denotes the substitution
congruence on ${\bf P}(\bar{W})$. Hopefully the condition
$\rnk(\,\bar{\sigma}\circ{\cal M}(\phi)\,)\leq|W|$ will allow us to
do that using
proposition~(\ref{logic:prop:FOPL:substrank:changeofvar}). Following
our example, the formula $\psi_{1}$ would look something like
$\psi_{1}=\forall v(u\in v)\in{\bf P}(\bar{W})$ for some $v\in W$
and $u\neq v$. Once we have the formula $\psi_{1}$ we can easily
project it onto ${\bf P}(W)$ and define $\psi=q(\psi_{1})$ where
$q:\bar{W}\to W$ is an appropriate substitution. Having defined a
formula $\psi\in{\bf P}(W)$ it will remain to show that:
    \begin{equation}\label{logic:eqn:FOPL:esssubst:eq1}
    \bar{\sigma}\circ{\cal M}(\phi)={\cal M}(\psi)
    \end{equation}
We shall proceed as follows: we first consider the operator ${\cal
N}:{\bf P}(\bar{W})\to{\bf P}(\bar{W})$ defined by ${\cal N}=p\circ
\bar{\cal M}$ where $p:\bar{\bar{W}}\to\bar{W}$ is an appropriate
substitution and $\bar{\cal M}:{\bf P}(\bar{W})\to{\bf
P}(\bar{\bar{W}})$ is the minimal transform mapping. From the
equivalence $\bar{\sigma}\circ{\cal M}(\phi)\sim\psi_{1}$ and
theorem~(\ref{logic:the:FOPL:mintransfsubcong:kernel}) of
page~\pageref{logic:the:FOPL:mintransfsubcong:kernel} we obtain
immediately:
    \begin{equation}\label{logic:eqn:FOPL:esssubst:eq2}
     {\cal N}\circ\bar{\sigma}\circ{\cal M}(\phi)={\cal N}(\psi_{1})
     \end{equation}
We then prove that ${\cal N}\circ\bar{\sigma}\circ{\cal
M}(\phi)=\bar{\sigma}\circ{\cal M}(\phi)$ so as to obtain:
    \begin{equation}\label{logic:eqn:FOPL:esssubst:eq3}
    \bar{\sigma}\circ{\cal M}(\phi)={\cal N}(\psi_{1})
    \end{equation}
Using the injection $i:W\to\bar{W}$ we finally prove that the
minimal transform ${\cal M}:{\bf P}(W)\to{\bf P}(\bar{W})$ is in
fact given by ${\cal M}={\cal N}\circ i$, and we conclude by arguing
that $\psi_{1}=i\circ q(\psi_{1})=i(\psi)$ which gives us
equation~(\ref{logic:eqn:FOPL:esssubst:eq1})
from~(\ref{logic:eqn:FOPL:esssubst:eq3}).

Before we formally define the operator ${\cal N}$, we shall say a
few words on the minimal extension $\bar{\bar{V}}$ of the set
$\bar{V}$ which is itself the minimal extension of~$V$. From
definition~(\ref{logic:def:FOPL:mintransform:minextension:set}) the
minimal extension $\bar{V}$ is defined as:
    \[
    \bar{V}=\{0\}\times V\cup\{1\}\times\N
    \]
Hence we have $\bar{\bar{V}}=\{0\}\times\bar{V}\cup\{1\}\times\N$.
So the set $\bar{\bar{V}}$ contains three types of elements: those
of the form $(0,(0,x))$ which we simply identify with $x\in V$,
those of the form $(0,(1,n))$ which we identify with $n\in\N$, and
those of the form $(1,n)$ which we denote $\bar{n}$ for all
$n\in\N$. Thus, in the interest of lighter notations, we regard
$\bar{\bar{V}}$ as the disjoint union
$\bar{\bar{V}}=V\uplus\N\uplus\bar{\N}$ where $\bar{\N}$ is the
image of $\N$ through the embedding $n\to\bar{n}$. So for example,
if $\phi=\forall y(x\in y)\in\pv$, we shall continue to denote the
minimal transform of $\phi$ as ${\cal M}(\phi)=\forall\,0(x\in 0)$,
while the minimal transform of ${\cal M}(\phi)$ is denoted
$\bar{\cal M}\circ{\cal
M}(\phi)=\forall\,\bar{0}(x\in\bar{0})\in{\bf P}(\bar{\bar{V}})$.
With these notational conventions in mind, we can now state:

\begin{defin}\label{logic:def:FOPL:esssubst:weak:transform}
Let $V$ be a set. We call {\em weak transform} on \pvb\ the map
${\cal N}:\pvb\to\pvb$ defined by ${\cal N}=p\circ\bar{\cal M}$
where $\bar{\cal M}:\pvb\to{\bf P}(\bar{\bar{V}})$ is the minimal
transform mapping and $p:\bar{\bar{V}}\to\bar{V}$ is defined by:
    \[
    \forall u\in\bar{\bar{V}}\ ,\ p\,(u)=\left\{
        \begin{array}{lcl}
        u&\mbox{\ if\ }&u\in\bar{V}\\
        n&\mbox{\ if\ }&u=\bar{n}\in\bar{\N}
        \end{array}
    \right.
    \]
\end{defin}
So for example, if $\phi=\forall\,0(x\in 0)\in\pvb$ we obtain
$\bar{\cal M}(\phi)=\forall\,\bar{0}(x\in \bar{0})$ and consequently
${\cal N}(\phi)=\forall\,0(x\in 0)=\phi$. However, if
$\phi=\forall\,1(0\in 1)$ then $\bar{\cal
M}(\phi)=\forall\,\bar{0}(0\in \bar{0})$ and ${\cal
N}(\phi)=\forall\,0(0\in 0)$. This last equality shows the
limitations of the operator ${\cal N}$. It is not a very interesting
operator, as the substitution equivalence ${\cal N}(\phi)\sim\phi$
is not necessarily preserved. On the positive side, the operator
${\cal N}$ has a simple definition which will make our forthcoming
proofs a lot easier. It also has the useful properties allowing us
to prove theorem~(\ref{logic:the:FOPL:esssubst:key})\,:

\begin{lemma}\label{logic:lemma:FOPL:esssubst:MNi}
Let $V$ be a set and ${\cal N}:\pvb\to\pvb$ be the weak transform on
\pvb. Let $i:V\to\bar{V}$ be the inclusion map. Then we have:
    \[
    {\cal M}={\cal N}\circ i
    \]
where ${\cal M}:\pv\to\pvb$ is the minimal transform mapping.
\end{lemma}
\begin{proof}
Let $\bar{\cal M}:\pvb\to{\bf P}(\bar{\bar{V}})$ be the minimal
transform mapping and $p:\bar{\bar{V}}\to\bar{V}$ be the map of
definition~(\ref{logic:def:FOPL:esssubst:weak:transform}). Given
$\phi\in\pv$, since $i:V\to\bar{V}$ is an injective map, in
particular it is valid for $\phi$. Hence we have:
    \begin{eqnarray*}
    {\cal N}\circ i(\phi)&=&p\circ\bar{\cal M}\circ i(\phi)\\
    \mbox{theorem~(\ref{logic:the:FOPL:commute:mintransform:validsub})}\
    \rightarrow&=&p\circ\bar{i}\circ{\cal M}(\phi)\\
    \mbox{A: to be proved}\ \rightarrow&=&{\cal M}(\phi)
    \end{eqnarray*}
So it remains to show that $p\circ\bar{i}(u)=u$ for all
$u\in\bar{V}$, where $\bar{i}:\bar{V}\to\bar{\bar{V}}$ is the
minimal extension of $i$. So let $u\in\bar{V}$. We shall distinguish
two cases: first we assume that $u\in V$. Then $\bar{i}(u)=i(u)=u$
and consequently $p\circ\bar{i}(u)=p\,(u)=u$ as requested. Next we
assume that $u\in\N$. Then $\bar{i}(u)=\bar{u}$ and it follows that
$p\circ\bar{i}(u)=p\,(\bar{u})=u$. So this appears to complete our
proof. However, we feel this proof is not completely convincing,
particularly with regards to the step $\bar{i}(u)=\bar{u}$. A blind
reading of
definition~(\ref{logic:def:FOPL:commute:minextensioon:map}) for the
minimal extension $\bar{i}:\bar{V}\to\bar{\bar{V}}$ seems to
indicate that $\bar{i}(u)=u$ whenever $u\in\N$. But of course, the
wording of
definition~(\ref{logic:def:FOPL:commute:minextensioon:map}) is the
result of identifications, which are now possibly confusing us. So
let $i_{2}:\N\to\bar{V}$ and $j_{2}:\N\to\bar{\bar{V}}$ be the
inclusion maps. What is really meant by
definition~(\ref{logic:def:FOPL:commute:minextensioon:map}) is that
$\bar{i}\circ i_{2}(u)=j_{2}(u)$ for all $u\in\N$. In our discussion
prior to stating
definition~(\ref{logic:def:FOPL:esssubst:weak:transform}) we agreed
to denote $j_{2}(u)$ as~$\bar{u}$. Hence if we identify $u$ and
$i_{2}(u)$ we obtain $\bar{i}(u)=\bar{u}$ as claimed. So let us
pursue this further and make all our identifications explicit. The
wording of definition~(\ref{logic:def:FOPL:esssubst:weak:transform})
is itself the result of identifications. Given $u\in\N$ the formula
$p(\bar{u})=u$ should be rigorously written as $p\circ
j_{2}(u)=i_{2}(u)$. So given $u\in\N$, if we identify $u$ and
$i_{2}(u)\in\bar{V}$ we have the following equalities:
    \begin{eqnarray*}
    p\circ\bar{i}(u)&=&p\circ\bar{i}\circ i_{2}(u)\\
    \mbox{def.~(\ref{logic:def:FOPL:commute:minextensioon:map})}\ \rightarrow
    &=&p\circ j_{2}(u)\\
    \mbox{def.~(\ref{logic:def:FOPL:esssubst:weak:transform})}\ \rightarrow
    &=&i_{2}(u)\\
    \mbox{identification}\ \rightarrow&=&u
    \end{eqnarray*}
Likewise, if we denote $i_{1}:V\to\bar{V}$ and
$j_{1}:\bar{V}\to\bar{\bar{V}}$ the inclusion mappings, and identify
$i_{1}(u)$ and $u$ for all $u\in V$, we obtain the following
equalities:
    \begin{eqnarray*}
    p\circ\bar{i}(u)&=&p\circ\bar{i}\circ i_{1}(u)\\
    \mbox{def.~(\ref{logic:def:FOPL:commute:minextensioon:map})}\ \rightarrow
    &=&p\circ j_{1}\circ i(u)\\
    \mbox{def.~(\ref{logic:def:FOPL:esssubst:weak:transform})}\ \rightarrow
    &=&i(u)\\
    i=i_{1}\ \rightarrow&=&i_{1}(u)\\
    \mbox{identification}\ \rightarrow&=&u
    \end{eqnarray*}
\end{proof}

\begin{lemma}\label{logic:lemma:FOPL:esssubst:NsM}
Let $V,W$ be sets and $\sigma:V\to W$ be a map. Then:
    \[
    {\cal N}\circ\bar{\sigma}\circ{\cal M}=\bar{\sigma}\circ{\cal M}
    \]
where ${\cal N}:{\bf P}(\bar{W})\to{\bf P}(\bar{W})$ is the weak
transform on ${\bf P}(\bar{W})$.
\end{lemma}
\begin{proof}
For once we shall be able to prove the formula without resorting to
a structural induction argument. In the interest of lighter
notations, we shall keep the same notations ${\cal M},\bar{\cal M},
{\cal N}$ and $p$ in relation to the sets $V$ and $W$. So let
$\phi\in\pv$:
    \begin{eqnarray*}
    {\cal N}\circ\bar{\sigma}\circ{\cal M}(\phi)&=&p\circ\bar{\cal
    M}\circ\bar{\sigma}\circ{\cal M}(\phi)\\
    \mbox{theorem~(\ref{logic:the:FOPL:commute:mintransform:validsub})
    of p.~\pageref{logic:the:FOPL:commute:mintransform:validsub}}\ \rightarrow
    &=&p\circ\bar{\bar{\sigma}}\circ\bar{\cal
    M}\circ{\cal M}(\phi)\\
    \mbox{prop.~(\ref{logic:prop:FOPL:mintransform:eqivalence})}\ \rightarrow
    &=&p\circ\bar{\bar{\sigma}}\circ\bar{\cal
    M}\circ i(\phi)\\
    \mbox{A: to be proved}\ \rightarrow&=&\bar{\sigma}\circ p\circ\bar{\cal
    M}\circ i(\phi)\\
    &=&\bar{\sigma}\circ{\cal
    N}\circ i(\phi)\\
    \mbox{lemma~(\ref{logic:lemma:FOPL:esssubst:MNi})}\ \rightarrow
    &=&\bar{\sigma}\circ{\cal
    M}(\phi)\\
    \end{eqnarray*}
So it remains to show that
$p\circ\bar{\bar{\sigma}}(u)=\bar{\sigma}\circ p\,(u)$ for all
$u\in\bar{\bar{V}}$. So let $u\in\bar{\bar{V}}$. Since
$\bar{\bar{V}}$ is the disjoint union of $\bar{V}$ and $\bar{\N}$,
we shall distinguish two cases: first we assume that $u\in\bar{V}$.
Then $\bar{\bar{\sigma}}(u)=\bar{\sigma}(u)\in\bar{W}$ and
consequently $p\circ\bar{\bar{\sigma}}(u)=\bar{\sigma}(u)$. So the
equality $p\circ\bar{\bar{\sigma}}(u)=\bar{\sigma}\circ p\,(u)$ is
satisfied since $p\,(u)=u$ for $u\in\bar{V}$. Next we assume that
$u\in\bar{\N}$ i.e. that $u=\bar{n}$ for some $n\in\N$. Then
$\bar{\bar{\sigma}}(u)=u=\bar{n}$ and consequently
$p\circ\bar{\bar{\sigma}}(u)=n=\bar{\sigma}\circ p\,(u)$.
\end{proof}

Before we move on to theorem~(\ref{logic:the:FOPL:esssubst:key})
which is our key result, we should pause a moment on an interesting
consequence of lemma~(\ref{logic:lemma:FOPL:esssubst:NsM}). In
general given $\phi$ and $\psi\in\pv$ such that $\phi\sim\psi$ where
$\sim$ is the substitution congruence, we cannot conclude that
$\phi=\psi$. Likewise if $\bar{\phi},\bar{\psi}\in\pvb$ are
substitution equivalent, we cannot conclude that that
$\bar{\phi}=\bar{\psi}$. However, if both $\bar{\phi}$ and
$\bar{\psi}$ are minimal transforms then things are different: it is
in fact possible to infer the equality ${\cal M}(\phi)={\cal
M}(\psi)$ simply from the substitution equivalence ${\cal
M}(\phi)\sim{\cal M}(\psi)$:

\begin{prop}\label{logic:prop:FOPL:esssubst:mintransform:equiv:imp:equal}
Let $V$ be a set and $\phi,\psi\in\pv$. Then we have:
    \[
    {\cal M}(\phi)\sim{\cal M}(\psi)\ \Rightarrow\ {\cal
    M}(\phi)={\cal M}(\psi)
    \]
where the relation $\sim$ denotes the substitution congruence on
\pvb.
\end{prop}
\begin{proof}
We assume that ${\cal M}(\phi)\sim{\cal M}(\psi)$. We need to show
that ${\cal M}(\phi)={\cal M}(\psi)$. However, from
theorem~(\ref{logic:the:FOPL:mintransfsubcong:kernel}) of
page~\pageref{logic:the:FOPL:mintransfsubcong:kernel} we obtain
$\bar{\cal M}\circ{\cal M}(\phi)=\bar{\cal M}\circ{\cal M}(\psi)$,
where $\bar{\cal M}:\pvb\to{\bf P}(\bar{\bar{V}})$ is the minimal
transform mapping. Hence, from
definition~(\ref{logic:def:FOPL:esssubst:weak:transform}), we see
that ${\cal N}\circ{\cal M}(\phi)={\cal N}\circ{\cal M}(\psi)$.
Applying lemma~(\ref{logic:lemma:FOPL:esssubst:NsM}) to $W=V$ and
the identity mapping $\sigma:V\to V$ we conclude that ${\cal
M}(\phi)={\cal M}(\psi)$.
\end{proof}

For future reference, we also quote the following lemma:
\begin{lemma}\label{logic:lemma:FOPL:esssubst:p:valid}
Let $V$ be a set and $p:\bar{\bar{V}}\to\bar{V}$ be the map of {\em
definition~(\ref{logic:def:FOPL:esssubst:weak:transform})}. Then for
all $\phi\in\pv$, the substitution $p$ is valid for the formula
$\bar{\cal M}\circ{\cal M}(\phi)$.
\end{lemma}
\begin{proof}
Using proposition~(\ref{logic:prop:FOPL:valid:injective}) it is
sufficient to show that $p$ is injective on the set $\var(\bar{\cal
M}\circ{\cal M}(\phi))$. First we shall show that $p$ is injective
on $V\cup\bar{\N}$: so suppose $u,v\in V\cup\bar{\N}$ are such that
$p\,(u)=p\,(v)$. We need to show that $u=v$. We shall distinguish
four cases: first we assume that $u,v\in V\subseteq\bar{V}$. Then
$u=v$ follows immediately from
definition~(\ref{logic:def:FOPL:esssubst:weak:transform}). Next we
assume that $u,v\in\bar{\N}$. Then $u=\bar{n}$ and $v=\bar{m}$ for
some $n,m\in\N$. From $p\,(u)=p\,(v)$ we obtain $n=m$ and
consequently $u=v$. Next we assume that $u\in V$ and $v\in\bar{\N}$.
Then $v=\bar{m}$ for some $m\in\N$ and from $p\,(u)=p\,(v)$ we
obtain $u=n$ which contradicts the fact that $V\cap\N=\emptyset$. So
this case is in fact impossible. The case $u\in\bar{\N}$ and $v\in
V$ is likewise impossible and we have proved that $p$ is injective
on $V\cup\bar{\N}$. So it remains to show that $\var(\bar{\cal
M}\circ{\cal M}(\phi))\subseteq V\cup\bar{\N}$. Let
$u\in\var(\bar{\cal M}\circ{\cal M}(\phi))$. We need to show that
$u\in V\cup\bar{\N}$. Since $\bar{\bar{V}}=\bar{V}\uplus\bar{\N}$,
we shall distinguish two cases: first we assume that $u\in\bar{V}$.
Then using
proposition~(\ref{logic:prop:FOPL:mintransform:variables}) we
obtain:
    \[
    u\in\var(\bar{\cal M}\circ{\cal
    M}(\phi))\cap\bar{V}=\free({\cal M}(\phi))=\free(\phi)\subseteq
    V\subseteq V\cup\bar{\N}
    \]
Next we assume that $u\in\bar{\N}$. Then it is clear that $u\in
V\cup\bar{\N}$.
\end{proof}
\begin{theorem}\label{logic:the:FOPL:esssubst:key}
Let $V,W$ be sets and $\sigma:V\to W$ be a map. Let $\phi\in\pv$
with:
    \[
    \rnk(\,\bar{\sigma}\circ{\cal M}(\phi)\,)\leq|W|
    \]
where $\bar{\sigma}:\bar{V}\to\bar{W}$ is the minimal extension of
$\sigma$ and ${\cal M}(\phi)$ is the minimal transform of $\phi$.
Then, there exists $\psi\in{\bf P}(W)$ such that
$\bar{\sigma}\circ{\cal M}(\phi)={\cal M}(\psi)$.
\end{theorem}
\begin{proof}
Our first step is to find $\psi_{1}\in{\bf P}(\bar{W})$ such that
$\bar{\sigma}\circ{\cal M}(\phi)\sim\psi_{1}$ and
$\var(\psi_{1})\subseteq W$, where $\sim$ denotes the substitution
congruence on ${\bf P}(\bar{W})$. We shall do so using
proposition~(\ref{logic:prop:FOPL:substrank:changeofvar}) applied to
the set $\bar{W}$ and the formula $\bar{\sigma}\circ{\cal
M}(\phi)\in{\bf P}(\bar{W})$. Suppose for now that we have proved
the inclusion $\free(\bar{\sigma}\circ{\cal M}(\phi))\subseteq W$.
Then from the assumption $\rnk(\,\bar{\sigma}\circ{\cal
M}(\phi)\,)\leq|W|$ and
proposition~(\ref{logic:prop:FOPL:substrank:changeofvar}) we obtain
the existence of $\psi_{1}\in{\bf P}(\bar{W})$ such that
$\bar{\sigma}\circ{\cal M}(\phi)\sim\psi_{1}$ and
$\var(\psi_{1})\subseteq W$. In fact,
proposition~(\ref{logic:prop:FOPL:substrank:changeofvar}) allows to
assume that $|\var(\psi_{1})|=\rnk(\,\bar{\sigma}\circ{\cal
M}(\phi)\,)$ but we shall not be using this property. So we need to
prove that $\free(\bar{\sigma}\circ{\cal M}(\phi))\subseteq W$:
    \begin{eqnarray*}
    \free(\bar{\sigma}\circ{\cal M}(\phi))&=&\free(\,\bar{\sigma}({\cal
    M}(\phi))\,)\\
    \mbox{prop.~(\ref{logic:prop:freevar:of:substitution:inclusion})}\ \rightarrow
    &\subseteq&\bar{\sigma}(\,\free({\cal M}(\phi))\,)\\
    \mbox{prop.~(\ref{logic:prop:FOPL:mintransform:variables})}\ \rightarrow
    &=&\bar{\sigma}(\free(\phi))\\
    \phi\in\pv\ \rightarrow&\subseteq&\bar{\sigma}(V)\\
    \mbox{def.~(\ref{logic:def:FOPL:commute:minextensioon:map})}\ \rightarrow
    &=&\sigma(V)\\
    &\subseteq&W
    \end{eqnarray*}
So the existence of $\psi_{1}\in{\bf P}(\bar{W})$ such that
$\bar{\sigma}\circ{\cal M}(\phi)\sim\psi_{1}$ and
$\var(\psi_{1})\subseteq W$ is now established. Next we want to
project $\psi_{1}$ onto ${\bf P}(W)$ by defining $\psi=q(\psi_{1})$
where $q:\bar{W}\to W$ is a substitution such that $q(u)=u$ for all
$u\in W$. To be rigorous, we need to define $q(n)$ for $n\in\N$
which we cannot do when $W=\emptyset$. So in the case when
$W\neq\emptyset$, let $u^{*}\in W$ and define $q:\bar{W}\to W$ as
follows:
    \[
    \forall u\in \bar{W}\ ,\ q(u)=\left\{
        \begin{array}{lcl}
        u&\mbox{\ if\ }&u\in W\\
        u^{*}&\mbox{\ if\ }&u\in\N
        \end{array}
    \right.
    \]
and consider the associated substitution mapping $q:{\bf
P}(\bar{W})\to{\bf P}(W)$. In the case when $W=\emptyset$, there
exists no substitution $q:\bar{W}\to W$ but we can define the
operator $q:{\bf P}(\bar{W})\to{\bf P}(W)$ with the following
structural recursion:
    \begin{equation}\label{logic:eqn:FOPL:esssubst:the:W:empty}
                    q(\chi)=\left\{
                    \begin{array}{lcl}
                    \bot&\mbox{\ if\ }&\chi=(u\in v)\\
                    \bot&\mbox{\ if\ }&\chi=\bot\\
                    q(\chi_{1})\to q(\chi_{2})
                    &\mbox{\ if\ }&\chi=\chi_{1}\to\chi_{2}\\
                    \bot&
                    \mbox{\ if\ }&\chi=\forall u\chi_{1}
                    \end{array}\right.
    \end{equation}
In both cases we set $\psi=q(\psi_{1})\in{\bf P}(W)$. We shall
complete the proof of the theorem by proving the equality
$\bar{\sigma}\circ{\cal M}(\phi)={\cal M}(\psi)$. Using
lemma~(\ref{logic:lemma:FOPL:esssubst:NsM})\,:
    \begin{eqnarray*}
    \bar{\sigma}\circ{\cal M}(\phi)&=&{\cal N}\circ\bar{\sigma}\circ{\cal
    M}(\phi)\\
    \mbox{def.~(\ref{logic:def:FOPL:esssubst:weak:transform})}\ \rightarrow
    &=&p\circ\bar{\cal M}\circ\bar{\sigma}\circ{\cal
    M}(\phi)\\
    \mbox{theorem~(\ref{logic:the:FOPL:mintransfsubcong:kernel})
    and $\bar{\sigma}\circ{\cal M}(\phi)\sim\psi_{1}$}\ \rightarrow
    &=&p\circ\bar{\cal M}(\psi_{1})\\
    &=&{\cal N}(\psi_{1})\\
    \mbox{A: to be proved}\ \rightarrow
    &=&{\cal N}\circ i\circ q(\psi_{1})\\
    \psi=q(\psi_{1})\ \rightarrow
    &=&{\cal N}\circ i(\psi)\\
    \mbox{lemma~(\ref{logic:lemma:FOPL:esssubst:MNi})}\ \rightarrow
    &=&{\cal M}(\psi)\\
    \end{eqnarray*}
So it remains to show that $i\circ q(\psi_{1})=\psi_{1}$ where
$i:W\to\bar{W}$ is the inclusion map. As before, we shall
distinguish two cases: first we assume that $W\neq\emptyset$. Then
$q$ arises from the substitution $q:\bar{W}\to W$ and from
proposition~(\ref{logic:prop:substitution:support}) it is sufficient
to prove that $i\circ q(u)=u$ for all $u\in\var(\psi_{1})$. Since
$\var(\psi_{1})\subseteq W$ the equality is clear. Next we assume
that $W=\emptyset$. From the inclusion $\var(\psi_{1})\subseteq W$
it follows that $\var(\psi_{1})=\emptyset$ and it is therefore
sufficient to prove the property:
    \[
    \var(\chi)=\emptyset\ \Rightarrow\ i\circ q(\chi)=\chi
    \]
for all $\chi\in{\bf P}(\bar{W})$. We shall do so by structural
induction. Note that when $W=\emptyset$, the inclusion map
$i:W\to\bar{W}$ is simply the map with empty domain, namely the
empty set. First we assume that $\chi=(u\in v)$ for some
$u,v\in\bar{W}$. Then the above implication is vacuously true. Next
we assume that $\chi=\bot$. Then $i\circ q(\chi)=\chi$ is clear. So
we now assume that $\chi=\chi_{1}\to\chi_{2}$ where
$\chi_{1},\chi_{2}$ satisfy the above implication. We need to show
the same is true of $\chi$. So we assume that
$\var(\chi)=\emptyset$. We need to show that $i\circ q(\chi)=\chi$.
However, from $\var(\chi)=\emptyset$ we obtain
$\var(\chi_{1})=\emptyset$ and $\var(\chi_{2})=\emptyset$. Having
assumed $\chi_{1}$ and $\chi_{2}$ satisfy the implication, we obtain
the equalities $i\circ q(\chi_{1})=\chi_{1}$ and $i\circ
q(\chi_{2})=\chi_{2}$ from which $i\circ q(\chi)=\chi$ follows
immediately. So it remains to check the case when $\chi=\forall
u\chi_{1}$ for which the above implication is also vacuously true.
\end{proof}

\index{essential@Existence of essential sub}
\begin{theorem}\label{logic:the:FOPL:esssubst:existence}
Let $V,W$ be sets and $\sigma:V\to W$ be a map. Then, there exists
an essential substitution mapping $\sigma^{*}:\pv\to{\bf P}(W)$
associated with $\sigma$, \ifand\  $|W|$ is an infinite cardinal or
the inequality  $|V|\leq|W|$ holds.
\end{theorem}
\begin{proof}
First we prove the 'if' part: so we assume that $|W|$ is an infinite
cardinal, or that it is finite with $|V|\leq |W|$. We need to prove
the existence of an essential substitution mapping
$\sigma^{*}:\pv\to{\bf P}(W)$ associated with $\sigma$. Let $c:{\cal
P}({\bf P}(W))\setminus\{\emptyset\}\to{\bf P}(W)$ be a choice
function whose existence follows from the axiom of choice. Let us
accept for now that for all $\phi\in\pv$, there exists some
$\psi\in{\bf P}(W)$ such that $\bar{\sigma}\circ{\cal M}(\phi)={\cal
M}(\psi)$. Given $\psi\in{\bf P}(W)$, let $[\psi]$ denote the
congruence class of $\psi$ modulo the substitution congruence, which
is a non-empty subset of ${\bf P}(W)$. Define $\sigma^{*}:\pv\to{\bf
P}(W)$ by setting $\sigma^{*}(\phi)=c([\psi])$, where $\psi$ is an
arbitrary formula of ${\bf P}(W)$ such that $\bar{\sigma}\circ{\cal
M}(\phi)={\cal M}(\psi)$. We need to check that $\sigma^{*}$ is well
defined, namely that $\sigma^{*}(\phi)$ is independent of the
particular choice of $\psi$. But if $\psi'$ is such that
$\bar{\sigma}\circ{\cal M}(\phi)={\cal M}(\psi')$ then ${\cal
M}(\psi)={\cal M}(\psi')$ and it follows from
theorem~(\ref{logic:the:FOPL:mintransfsubcong:kernel}) of
page~\pageref{logic:the:FOPL:mintransfsubcong:kernel} that
$[\psi]=[\psi']$. So it remains to show the existence of $\psi$ such
that $\bar{\sigma}\circ{\cal M}(\phi)={\cal M}(\psi)$ for all
$\phi\in\pv$. So let $\phi\in\pv$. Using
theorem~(\ref{logic:the:FOPL:esssubst:key}) of
page~\pageref{logic:the:FOPL:esssubst:key} it is sufficient to prove
that $\rnk(\,\bar{\sigma}\circ{\cal M}(\phi)\,)\leq|W|$. The
substitution rank of a formula being always finite, this is clearly
true if $|W|$ is infinite. Otherwise, using
proposition~(\ref{logic:prop:FOPL:subst:rank:substitution})\,:
    \begin{eqnarray*}
    \rnk(\,\bar{\sigma}\circ{\cal M}(\phi)\,)&\leq&\rnk({\cal
    M}(\phi))\\
    \mbox{prop.~(\ref{logic:prop:FOPL:substrank:minrank})}\ \rightarrow
    &=&\rnk(\phi)\\
    &\leq&|\var(\phi)|\\
    &\leq&|V|\\
    |V|\leq|W|\ \rightarrow&\leq&|W|
    \end{eqnarray*}
We now prove the 'only if' part: So we assume there exists
$\sigma^{*}:\pv\to{\bf P}(W)$ essential substitution associated with
$\sigma$. We need to show that $|W|$ is an infinite cardinal, or
that it is finite with $|V|\leq |W|$. So suppose $|W|$ is a finite
cardinal. There exists $n\in\N$ such that $|W|=n$. We need to show
that $|V|\leq n$ holds. So we assume to the contrary that
$n+1\leq|V|$ and we shall derive a contradiction. Using
proposition~(\ref{logic:prop:FOPL:substrank:closed:fullrank}) there
exists a formula $\phi\in\pv$ such that $\free(\phi)=\emptyset$ and
$\rnk(\phi)=n+1$. Hence we have the following:
    \begin{eqnarray*}
    n+1&=&\rnk(\phi)\\
    \mbox{prop.~(\ref{logic:prop:FOPL:substrank:minrank})}\ \rightarrow
    &=&\rnk({\cal M}(\phi))\\
    \mbox{A: to be proved}\ \rightarrow
    &=&\rnk(\bar{\sigma}\circ{\cal M}(\phi))\\
    \mbox{$\sigma^{*}$ associated with $\sigma$}\ \rightarrow
    &=&\rnk({\cal M}\circ\sigma^{*}(\phi))\\
    \mbox{prop.~(\ref{logic:prop:FOPL:substrank:minrank})}\ \rightarrow
    &=&\rnk(\sigma^{*}(\phi))\\
    &\leq&|\var(\sigma^{*}(\phi))|\\
    \sigma^{*}(\phi)\in{\bf P}(W)\ \rightarrow&\leq&|W|\\
    &=&n
    \end{eqnarray*}
which is our desired contradiction. So it remains to show
$\bar{\sigma}\circ{\cal M}(\phi)={\cal M}(\phi)$. From
proposition~(\ref{logic:prop:substitution:support}) it is enough to
prove that $\bar{\sigma}(u)=u$ for all $u\in\var({\cal M}(\phi))$.
Using definition~(\ref{logic:def:FOPL:commute:minextensioon:map}) of
the minimal extension $\bar{\sigma}$ it is sufficient to show that
$\var({\cal M}(\phi))\subseteq\N$. Since $\bar{V}=V\uplus\N$, it is
therefore sufficient to prove that $\var({\cal M}(\phi))\cap
V=\emptyset$. However from
proposition~(\ref{logic:prop:FOPL:mintransform:variables}) we have
the equality $\var({\cal M}(\phi))\cap V=\free(\phi)$. So the result
follows from $\free(\phi)=\emptyset$.
\end{proof}
