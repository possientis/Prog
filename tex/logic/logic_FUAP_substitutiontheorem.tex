We have now reached the end of a long journey, as we are about to
prove what appears to us as a delicate point of mathematical logic:
the substitution theorem. Everything we did in this long chapter,
all the details and the pedantry over the free algebra \pvs\ was
directed at this single result. We did what we could to arrive at it
with the shortest possible route, developing along the way some key
notions of valuation modulo, minimal transform, substitution
congruence, substitution rank and essential substitution. It is
ironic that most texts in mathematical logic will not take more than
half a page to prove the result. Some will simply use one line to
point out that substituting variables in a proof gives you another
proof etc\ldots So we must be doing something wrong. To be fair,
these texts always assume that $V$ is countably infinite. So it is
possible to carry out substitutions of variables which rely on
freshness and are injective. Proving a substitution theorem in the
injective case is of course a lot simpler, as can be seen from
proposition~(\ref{logic:prop:FUAP:validsubtotclean:sequent}). Our
substitution theorem is stated for essential substitutions, a notion
which does not exist in mathematical textbooks as far as we can
tell. We believe essential substitutions are a key notion in formal
languages with variable binding. If we knew enough about category
theory and decided to design a category whose objects are the spaces
\pv, the morphisms should be essential substitutions
$\sigma:\pv\to{\bf P}(W)$ or something approaching, depending on the
exact identification we wish to perform on \pv. Of course we are not
completely convinced about this, as we are troubled by the fact that
no essential substitution $\sigma:\pv\to{\bf P}(W)$ exists whenever
$W$ is a finite set which is smaller than $V$. This can be seen from
theorem~(\ref{logic:the:FOPL:esssubst:existence}) of
page~\pageref{logic:the:FOPL:esssubst:existence}.

Now what is the big fuss about the substitution theorem, why do we
care so much? A sequent $\Gamma\vdash\phi$ can be carried over into
the sequent $\sigma(\Gamma)\vdash\sigma(\phi)$\,: this is rather
obvious and nowhere near as interesting as determining for example
whether the sequent ${\bf ZF}\vdash\mathtt{Cons}({\bf ZF})$ is true
or not. Yes, G\"{o}del's second incompleteness theorem is far more
interesting, we cannot argue about that.  We are doing what we can.
We need to walk before we can run. In the big scheme of things, the
substitution theorem is just a small step. It has given us the
opportunity to derive some valuable insight on the free algebra
\pvs\ and more generally on Hilbert-style deduction systems. More
fundamentally, the substitution theorem is a prerequisite to a
milestone of mathematical logic: G\"{o}del's completeness theorem.
We have no hope of proving the completeness theorem unless we can
prove that consistency is preserved through embeddings.
Specifically, if $i:V\to W$ is an injective map and $\Gamma$ is a
consistent subset of $V$, then $i(\Gamma)$ should be a consistent
subset of $W$. As can be seen from
definition~(\ref{logic:def:FOPL:semantics:consistent:subset}),
proving the consistency of $i(\Gamma)$ from the consistency of
$\Gamma$ involves carrying over the sequent $i(\Gamma)\vdash\bot$
into $\Gamma\vdash\bot$. Without thinking very hard about this, it
is clear we need some form of substitution theorem before anything
else.

Unfortunately, despite the successful delivery of
theorem~(\ref{logic:the:FUAP:substitutiontheorem:main}) below, our
initial question remains open: in order to carry over the sequent
$i(\Gamma)\vdash\bot$ into $\Gamma\vdash\bot$, we need an essential
substitution $\sigma:{\bf P}(W)\to\pv$ associated with a left
inverse $\sigma:W\to V$ of the embedding $i:V\to W$. Such essential
substitution does not exist when $V$ is finite and strictly smaller
than $W$. So we still do not know whether consistency is preserved
through embedding in our axiomatization of first order logic with
finitely many variables.

\index{substitution@Substitution theorem essential}
\begin{theorem}[Substitution Theorem] \label{logic:the:FUAP:substitutiontheorem:main}
Let $\sigma:\pv\to{\bf P}(W)$ where $V,W$ are sets be an essential
substitution. Let $\Gamma\subseteq\pv$ and $\phi\in\pv$. Then:
    \[
    \Gamma\vdash\phi\ \ \ \Rightarrow\ \ \ \sigma(\Gamma)\vdash\sigma(\phi)
    \]
\end{theorem}
\begin{proof}
We assume the sequent $\Gamma\vdash\phi$ is true. There exists a
proof $\pi\in\pvs$ such that $\val(\pi)=\phi$ and
$\hyp(\pi)\subseteq\Gamma$. From
proposition~(\ref{logic:prop:FUAP:clean:counterpart}) we may assume
without loss of generality that $\pi$ is totally clean. In
particular from proposition~(\ref{logic:def:FUAP:almostclean:clean})
the proof $\pi$ is clean. Furthermore from
proposition~(\ref{logic:prop:FUAP:valuationmod:clean:proof}) we have
the equality $\vals(\pi)=\val(\pi)$. From
proposition~(\ref{logic:prop:FUAP:esssubstprop:extension}) there
exists an essential proof substitution $\sigma:\pvs\to{\bf\Pi}(W)$
which is an extension of $\sigma$. Consider the proof
$\sigma(\pi)\in{\bf\Pi}(W)$. In order to show that
$\sigma(\Gamma)\vdash\sigma(\phi)$ is true, from
theorem~(\ref{logic:the:FUAP:valuationmod:provability}) of
page~\pageref{logic:the:FUAP:valuationmod:provability} it is
sufficient to prove $\vals(\sigma(\pi))\sim\sigma(\phi)$ and
$\hyp(\sigma(\pi))\precsim\sigma(\Gamma)$, where $\sim$ is the
substitution congruence on ${\bf P}(W)$ and $\precsim$ is the
inclusion modulo. Since $\pi$ is a clean proof, from
proposition~(\ref{logic:prop:FUAP:esssubstcleanproof:main}) we have
the following:
    \[
    \vals(\sigma(\pi))\sim\sigma\circ\vals(\pi)=\sigma\circ\val(\pi)=\sigma(\phi)
    \]
Furthermore from
proposition~(\ref{logic:prop:substitutiontheorem:hypothesis}) since
$\pi$ is clean we obtain:
    \[
    \hyp(\sigma(\pi))\sim\sigma(\hyp(\pi))\subseteq\sigma(\Gamma)
    \]
From which we conclude $\hyp(\sigma(\pi))\precsim\sigma(\Gamma)$ as
requested.
\end{proof}

Most people are presumably not familiar with essential
substitutions. However, the idea of capture-avoiding substitutions
is quite common. We shall now provide a corollary of
theorem~(\ref{logic:the:FUAP:substitutiontheorem:main}) dealing with
naive substitutions in the sense of
definition~(\ref{logic:def:substitution}). Given a true sequent
$\Gamma\vdash\phi$, it is a natural question to ask whether
$\sigma(\Gamma)\vdash\sigma(\phi)$ whenever $\sigma:V\to W$ avoids
capture on $\phi$ and every element of $\Gamma$. We are able to
answer this question in the case when $|W|$ is an infinite cardinal
or $|V|\leq|W|$: we simply invoke the existence of an associated
essential substitution $\sigma:\pv\to{\bf P}(W)$ and use
theorem~(\ref{logic:the:FUAP:substitutiontheorem:main}) to conclude.
Needless to say that having to keep conditions on the cardinals
$|V|$ and $|W|$ is highly unsatisfactory. We would like to claim
that $\sigma(\Gamma)\vdash\sigma(\phi)$ in all cases when $\sigma$
avoids capture. We currently do not know whether this is true.

\index{substitution@Substitution theorem valid}
\begin{prop}\label{logic:prop:FUAP:substitutiontheorem:valid}
Let $V$, $W$ be sets such that $|W|$ is an infinite cardinal or
$|V|\leq|W|$. Let $\sigma:V\to W$ be a map. Let $\Gamma\subseteq\pv$
and $\phi\in\pv$. We assume that $\sigma$ is valid for $\phi$ and
every element of $\Gamma$. Then we have:
    \[
    \Gamma\vdash\phi\ \ \ \Rightarrow\ \ \ \sigma(\Gamma)\vdash\sigma(\phi)
    \]
where $\sigma:\pv\to{\bf P}(W)$ is the associated substitution as
per {\em definition~(\ref{logic:def:substitution})}.
\end{prop}
\begin{proof}
Since $|W|$ is an infinite cardinal or $|V|\leq|W|$, from
theorem~(\ref{logic:the:FOPL:esssubst:existence}) of
page~\pageref{logic:the:FOPL:esssubst:existence} there exists an
essential substitution $\sigma^{*}:\pv\to{\bf P}(W)$ associated with
$\sigma$. Hence if we assume that $\Gamma\vdash\phi$, from
theorem~(\ref{logic:the:FUAP:substitutiontheorem:main}) we obtain
$\sigma^{*}(\Gamma)\vdash\sigma^{*}(\phi)$. Using
theorem~(\ref{logic:the:FUAP:valuationmod:provability}) of
page~\pageref{logic:the:FUAP:valuationmod:provability} there exists
a proof $\pi^{*}\in{\bf\Pi}(W)$ such that
$\vals(\pi^{*})\sim\sigma^{*}(\phi)$ and
$\hyp(\pi^{*})\precsim\sigma^{*}(\Gamma)$, where $\sim$ is the
substitution congruence on ${\bf P}(W)$ and $\precsim$ is the
associated inclusion modulo. We need to show that
$\sigma(\Gamma)\vdash\sigma(\phi)$. In order to do so, applying
theorem~(\ref{logic:the:FUAP:valuationmod:provability}) once more it
is sufficient to prove that $\vals(\pi^{*})\sim\sigma(\phi)$ and
$\hyp(\pi^{*})\precsim\sigma(\Gamma)$. First we show that
$\vals(\pi^{*})\sim\sigma(\phi)$\,: it is sufficient to prove that
$\sigma^{*}(\phi)\sim\sigma(\phi)$ which follows immediately from
proposition~(\ref{logic:prop:FOPL:esssubstprop:validity}) and the
validity of $\sigma$ for $\phi$. Next we show that
$\hyp(\pi^{*})\precsim\sigma(\Gamma)$. So let
$\chi\in\hyp(\pi^{*})$. We need to show the existence of
$\psi\in\Gamma$ such that $\chi\sim\sigma(\psi)$. However, from the
inclusion modulo $\hyp(\pi^{*})\precsim\sigma^{*}(\Gamma)$ there
exists $\psi\in\Gamma$ such that $\chi\sim\sigma^{*}(\psi)$. It is
therefore sufficient to prove that
$\sigma^{*}(\psi)\sim\sigma(\psi)$. Having assumed that $\sigma:V\to
W$ is valid for every element of $\Gamma$, in particular it is valid
for $\psi$. Hence the equivalence $\sigma^{*}(\psi)\sim\sigma(\psi)$
follows once again from
proposition~(\ref{logic:prop:FOPL:esssubstprop:validity}), which
completes our proof.
\end{proof}
