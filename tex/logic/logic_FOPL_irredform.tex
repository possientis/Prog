Having proved proposition~(\ref{logic:prop:iterated:permutation})
our next objective is to establish some characterization theorem for
the permutation congruence, similar to
theorem~(\ref{logic:the:sub:congruence:charac}) of
page~\pageref{logic:the:sub:congruence:charac} which was established
for the substitution congruence. Suppose $\phi=\forall u\forall
v\forall w((u\in v)\to(v\in w))$. We know from
theorem~(\ref{logic:the:unique:representation}) of
page~\pageref{logic:the:unique:representation} that the
representation $\phi=\forall x\phi_{1}$ is unique: clearly
$\phi_{1}=\forall v\forall w((u\in v)\to(v\in w))$ and $x=u$.
However, if we introduce iterated quantification as we have done,
the representation $\phi=\forall x\phi_{1}$ with $x\in V^{n}$ and
$n\in\N$ is not unique. However, such representation must be unique
if we impose that $\phi_{1}$ {\em does not start} as a
quantification. For lack of a better word, we shall say that
$\phi_{1}$ is {\em irreducible} to indicate that no further
quantification can be removed at the top level. Although we are not
very happy with the chosen terminology ({\em irreducible} is a big
word in algebra), this is simply a temporary notion soon to be
forgotten. It will allow us to prove the uniqueness we require for
our characterization theorem.
\index{irreducible@Irreducible
formula}
\begin{defin}\label{logic:def:irreducible:formula}
Let $V$ be a set and $\phi\in\pv$. We say that $\phi$ is {\em
irreducible} \ifand\ one of the following is the case:
    \begin{eqnarray*}
    (i)&&\phi\in\pvo\\
    (ii)&&\phi=\bot\\
    (iii)&&\phi=\phi_{1}\to\phi_{2}\ ,\ \phi_{1}\in\pv\ ,\ \phi_{2}\in\pv
    \end{eqnarray*}
\end{defin}
\begin{prop}\label{logic:prop:irreducible:formula}
Let $V$ be a set and $\phi\in\pv$. Then $\phi$ can be uniquely
represented as $\phi=\forall x\phi_{1}$ where $x\in V^{n}$,
$n\in\N$, $\phi_{1}\in\pv$ and $\phi_{1}$ is irreducible.
\end{prop}
\begin{proof}
Given $\phi\in\pv$ we need to show the existence of $n\in\N$, $x\in
V^{n}$ and $\phi_{1}$ irreducible such that $\phi=\forall
x\phi_{1}$. Furthermore, we need to show that this representation is
unique, namely that if $m\in\N$, $y\in V^{m}$, $\psi_{1}$ is
irreducible and $\phi=\forall y\psi_{1}$, then we have $x=y$ and
$\phi_{1}=\psi_{1}$. Note that $x=y$ will automatically imply that
$n=m$, as identical maps have identical domains. We shall carry out
the proof by structural induction using
theorem~(\ref{logic:the:proof:induction}) of
page~\pageref{logic:the:proof:induction}. Since \pvo\ is a generator
of \pv, we first check that the property is true for $\phi\in\pvo$.
So suppose $\phi\in\pvo$. We take $n=0$, $x=0\in V^{n}$ and
$\phi_{1}=\phi$. Then $\phi_{1}$ is irreducible and we have
$\phi=\forall x\phi_{1}$ which shows the existence of the
representation. Suppose now that $\phi=\forall y\psi_{1}$ for some
$y\in V^{m}$, $m\in\N$ and $\psi_{1}$ irreducible. Since
$\phi\in\pvo$, from theorem~(\ref{logic:the:unique:representation})
of page~\pageref{logic:the:unique:representation}, $\phi$ cannot be
a quantification. It follows that $m=0$, $y=0$ and
$\phi_{1}=\phi=\forall 0\psi_{1}=\psi_{1}$. This proves the
uniqueness of the representation, and the property is proved for
$\phi\in\pvo$. We now assume that $\phi=\bot$ or indeed that
$\phi=\phi_{1}\to\phi_{2}$. Then $\phi$ is irreducible and taking
$n=0$, $x=0$ and $\phi_{1}=\phi$ we obtain $\phi=\forall x\phi_{1}$.
Once again from theorem~(\ref{logic:the:unique:representation}),
$\phi$ cannot be a quantification and it follows that this
representation is unique. It remains to show that the property is
true in the case when $\phi=\forall z\theta$ for some $z\in V$, and
the property is true for $\theta$. First we show the existence of
the representation. Having assumed that the property is true for
$\theta$, there exists $n\in\N$, $x\in V^{n}$ and $\theta_{1}$
irreducible such that $\theta=\forall x\theta_{1}$. Consider the map
$x^{*}:(n+1)\to V$ defined by $x^{*}_{|n} = x$ and $x^{*}(n)=z$.
Taking $\phi_{1}=\theta_{1}$ we obtain:
    \[
    \forall x^{*}\phi_{1}=\forall x^{*}(n)\forall
    x^{*}_{|n}\,\phi_{1}=\forall z\forall x\,\theta_{1}=\forall
    z\theta=\phi
    \]
which shows the existence of the representation. We now prove the
uniqueness. So suppose $m\in\N$, $y\in V^{m}$, $\psi_{1}$ is
irreducible and $\phi=\forall y\psi_{1}$. We need to show that
$y=x^{*}$ and $\psi_{1}=\phi_{1}$. From $\phi=\forall y\psi_{1}$ we
obtain $\forall z\theta=\forall y\psi_{1}$. It follows that $\forall
y\psi_{1}$ is a quantification and from
theorem~(\ref{logic:the:unique:representation}) it cannot be
irreducible. This shows that $m\geq 1$, and consequently $m=p+1$ for
some $p\in\N$. Hence:
    \[
    \forall z\theta = \phi=\forall y\psi_{1} =\forall y(p)\forall y_{|p}\psi_{1}
    \]
Applying theorem~(\ref{logic:the:unique:representation}) once more,
we obtain $z = y(p)$ and $\theta=\forall y_{|p}\psi_{1}$. Having
assumed the property is true for $\theta$, its representation
$\theta=\forall x\theta_{1}$ is unique. It follows that $y_{|p}=x$
and $\psi_{1}=\theta_{1}$. So we have proved that
$\psi_{1}=\phi_{1}$. Furthermore from $y_{|p}=x$ it follows that
$p=n$ and $m=n+1$. So $y$ is a map $y:(n+1)\to V$ such that
$y_{|n}=x$ and $y(n)=z$. We conclude that $y=x^{*}$.
\end{proof}
