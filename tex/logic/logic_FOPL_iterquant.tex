The following definition allows us to replace the formalism $\forall
x(n-1)\ldots\forall x(0)\,\phi$ with a more condensed expression
$\forall x\phi$. In doing so, we are overloading the symbol
$\forall$ by creating a new unary operator $\forall x:\pv\to\pv$ for
all $x\in V^{n}$ and $n\in\N$. This operator coincides with the
standard operator $\forall x$ when $x\in V$. Note that when $n=0$,
the only possible element $x\in V^{n}$ is $x=0$ and the operator
$\forall\,0:\pv\to \pv$ is simply the identity. There is obviously a
small risk of notational confusion when seeing $\forall\, 0$
especially in the context of minimal transforms. Overall, we do not
think this risk is high enough to warrant the introduction of a
different symbol for iterated quantification. Given $n\in\N$ and
$x\in V^{n}$ we define $\forall x\phi$ in terms of the expression
$\forall x(n-1)\ldots\forall x(0)\,\phi$. In case this is not clear,
this is simply a notational shortcut for the more formal {\em
definition by recursion} $\forall\, 0\phi=\phi$ and if $x\in
V^{n+1}$, $\forall x\phi=\forall x(n)\forall x_{|n}\phi$.
\index{iterated@Iterated quantification}\index{forall@$\forall x$,
$x\in V^{n}$ : iterated quant.}\index{forall@$\forall\,0$ : iterated
quantification}
\begin{defin}\label{logic:def:iterated:quant}
Let $V$ be a set, $n\in\N$ and $x\in V^{n}$. For all $\phi\in\pv$ we
call {\em iterated quantification of $\phi$ by $x$} the element
$\forall x\phi$ of \pv\ defined by:
    \[
    \forall x\phi=\forall x(n-1)\ldots\forall x(0)\,\phi
    \]
where it is understood that if $n=0$, we have
$\forall\,0\,\phi=\phi$.
\end{defin}
Given any congruence relation $\sim$ on \pv, we have $\forall
x\phi\sim\forall x\psi$ whenever $\phi\sim\psi$ and $x\in V$.
Obviously this property should extend to iterated quantification:
\begin{prop}\label{logic:prop:iterated:congruence}
Let $V$ be a set, $n\in\N$ and $x\in V^{n}$. Let $\sim$ be an
arbitrary congruent relation on \pv. Then for all $\phi,\psi\in\pv$
we have:
    \[
    \phi\sim\psi\ \Rightarrow\ \forall x\phi\sim\forall x\psi
    \]
\end{prop}
\begin{proof}
We shall prove this result by induction on $n\in\N$. First we assume
$n=0$. Then $V^{n}=\{0\}$ and $x=0$. By convention the corresponding
iterated quantifications are defined as $\forall 0\phi=\phi$ and
$\forall 0\psi=\psi$. So the property is clearly true. Next we
assume that the property is true for $n\in\N$. We need to show that
it is true for $n+1$. So we assume that $\phi\sim\psi$ and $x\in
V^{n+1}$. We need to show that $\forall x\phi\sim\forall x\psi$.
However, we have $x_{|n}\in V^{n}$ and from the induction hypothesis
we obtain $\forall x_{|n}\,\phi\sim\forall x_{|n}\,\psi$. Since
$\sim$ is a congruent relation on \pv\ we have:
    \[
    \forall x\phi=\forall x(n)\forall x_{|n}\,\phi\sim\forall
    x(n)\forall x_{|n}\,\psi=\forall x\psi
    \]
So we conclude that the property is true for $n+1$.
\end{proof}

We are now able to quote and prove what we set out from the
beginning. If two iterated quantification operators are equivalent
modulo some permutation, then the corresponding formulas are
permutation equivalent. This all seems pretty obvious, but it has to
be proved one way or another:
\begin{prop}\label{logic:prop:iterated:permutation}
Let $V$ be a set, $n\in\N$ and $x,y\in V^{n}$. Let $\sim$ denote the
permutation congruence on \pv. Then for all $\phi\in\pv$ we have:
    \[
    x\sim y\ \Rightarrow\ \forall x\phi\sim\forall y\phi
    \]
where $x\sim y$ refers to the permutation equivalence on $V^{n}$.
\end{prop}
\begin{proof}
We shall distinguish three cases. First we assume that $n=0$. Then
$x=y=0$ and the result is clear. Next we assume that $n=1$. Then
$x\sim y$ implies that $x=y$ and the result is also clear. We now
assume that $n\geq 2$ and $x\sim y$. We need to show that $\forall
x\phi\sim\forall y\phi$. From $x\sim y$ there exists a permutation
$\sigma:n\to n$ such that $y=x\circ \sigma$. We shall distinguish
two cases. First we assume that $\sigma$ is an elementary
permutation, namely that there exists $i\in n-1$ such that
$\sigma=[i:i+1]$ (of order $n$). We shall prove that $\forall
x\phi\sim\forall y\phi$ using an induction argument on $n\geq 2$.
First we assume that $n=2$. Then we must have $i=0$ and
$\sigma=[0\!:\!1]$ (of order $2$). Hence from $y=x\circ \sigma$ we
obtain:
    \[
    \forall y\phi=\forall y(1)\forall y(0)\,\phi=\forall x(0)\forall x(1)\,\phi
    \]
Comparing with $\forall x\phi=\forall x(1)\forall x(0)\phi$, it is
clear that the ordered pair $(\forall x\phi,\forall y\phi)$ belongs
to the generator $R_{0}$ of the permutation congruence as per
definition~(\ref{logic:def:perm:congruence}). In particular we have
$\forall x\phi\sim\forall y\phi$ which completes our proof in the
case when $\sigma$ is elementary and $n=2$. We now assume that the
property is true for $\sigma$ elementary and $n\geq 2$. We need to
show that it is also true for $\sigma$ elementary and $n+1$. So we
assume the existence of $i\in n$ such that $\sigma=[i:i+1]$ (of
order $n+1$). We need to show that $\forall x\phi\sim\forall y\phi$.
We shall distinguish two cases. First we assume that $i=n-1$ which
is the highest possible value when $i\in n$. Then for all $j\in n-1$
we have $j\neq i$ and $j\neq i+1$ and consequently
$y(j)=x\circ\sigma(j)=x\circ [i:i+1](j)=x(j)$. Hence we see that
$x_{|n-1}=y_{|n-1}$, and:
    \[
    \forall y\phi=\forall y(n)\,\forall y(n-1)\,\forall y_{|n-1}\,\phi =
    \forall x(n-1)\,\forall x(n)\,
    \forall x_{|n-1}\phi
    \]
while we have $\forall x\phi=\forall x(n)\,\forall x(n-1)\, \forall
x_{|n-1}\phi$. Setting $u=x(n)$ and $v=x(n-1)$ with
$\phi_{1}=\forall x_{|n-1}\phi$, it follows that $\forall x\phi =
\forall u\forall v\,\phi_{1}$ and $\forall y\phi=\forall v\forall
u\,\phi_{1}$. Hence we see that the ordered pair $(\forall
x\phi,\forall y\phi)$ belongs to the generator $R_{0}$ of the
permutation congruence as per
definition~(\ref{logic:def:perm:congruence}). In particular $\forall
x\phi\sim\forall y\phi$. We now assume that $i\in n-1$. In
particular we have $i+1\neq n$ and $i\neq n$ and:
    \[
    y(n)=x\circ\sigma(n)=x\circ[i:i+1](n)=x(n)
    \]
It follows that $\forall y\phi=\forall y(n)\forall
y_{|n}\,\phi=\forall x(n)\forall y_{|n}\,\phi$ while $\forall
x\phi=\forall x(n)\forall x_{|n}\,\phi$. Hence, we see that in order
to show that $\forall x\phi\sim\forall y\phi$, the permutation
congruence being a congruent relation on \pv, it is sufficient to
prove that $\forall x_{|n}\,\phi\sim\forall y_{|n}\,\phi$. This
follows immediately from our induction hypothesis and the fact that:
    \[
    y_{|n}=x\circ\sigma_{|n}=x\circ [i:i+1]_{|n}=x_{|n}\circ [i:i+1]
    \]
where it is understood that the second occurrence of '$[i:i+1]$' in
this equation refers to the elementary permutation of order $n$
(rather than $n+1$) defined for $i\in n-1$. This completes our
induction argument and we have proved that $\forall x\phi\sim\forall
y\phi$ in the case when $y=x\circ\sigma$ with $\sigma$ elementary.
We now assume that $\sigma:n\to n$ is an arbitrary permutation of
order $n$ and we need to show that $\forall x\phi\sim\forall y\phi$.
However since $n\geq 2$ it follows from
lemma~(\ref{logic:lemma:integer:permutation}) that $\sigma$ can be
expressed as a composition of elementary permutations of order $n$.
In other words, there exist $k\in\N^{*}$ and elementary permutations
$\tau_{1}, \ldots, \tau_{k}$ such that:
    \[
    \sigma=\tau_{k}\circ\ldots\circ\tau_{1}
    \]
We shall prove that $\forall x\phi\sim\forall y\phi$ by induction on
the integer $k\in\N^{*}$. If $k=1$ then $\sigma$ is an elementary
permutation and we have already proved that the result is true. We
now assume that the result is true for $k\geq 1$ and we need to show
it is also true for $k+1$. So we assume that:
    \[
    \sigma=\tau_{k+1}\circ\tau_{k}\circ\ldots\circ\tau_{1}
    \]
Defining $\sigma^{*}=\tau_{k+1}\circ\ldots\circ\tau_{2}$ we obtain
$\sigma=\sigma^{*}\circ\tau_{1}$. It follows that:
    \[
    y=x\circ\sigma=x\circ\sigma^{*}\circ\tau_{1}=y^{*}\circ\tau_{1}
    \]
where we have put $y^{*}=x\circ\sigma^{*}$. Now since $\tau_{1}$ is
an elementary permutation and $y=y^{*}\circ\tau_{1}$, we have
already proved that $\forall y\phi\sim\forall y^{*}\phi$.
Furthermore, since $y^{*}=x\circ\sigma^{*}$ and $\sigma^{*}$ is a
composition of $k$ elementary permutations, it follows from our
induction hypothesis that $\forall y^{*}\phi\sim\forall x\phi$. By
transitivity of the permutation congruence, we conclude that
$\forall x\phi\sim\forall y\phi$.
\end{proof}
