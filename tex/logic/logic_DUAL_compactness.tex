Lindenbaum's lemma~(\ref{logic:lemma:FOPL:semantics:lindenbaum}) has
many applications. We shall start by showing that the notions of
{\em satisfiable} and {\em consistent} subsets are in fact
equivalent. Recall from
definition~(\ref{logic:def:FOPL:semantics:valuation:truth}) that
being {\em satisfiable} is defined in terms of valuation and not
model. This is contrary to standard practice and the reader is
invited to refer to our discussion preceding
definition~(\ref{logic:def:FOPL:semantics:valuation}). As we shall
see, {\em every valuation has a model} so
definition~(\ref{logic:def:FOPL:semantics:valuation:truth}) is in
fact equivalent to the standard notion of {\em satisfiable} set. We
should also remember that $\pvd\neq\emptyset$ has not been proved at
this stage of the document. A consequence of the dual space being
empty would be that no subset of \pv\ is either satisfiable or
consistent, not even $\emptyset$.

\index{consistent@Consistency and
satisfiability}\index{satisfiable@Satisfiability and consistency}
\begin{theorem}\label{logic:the:FOPL:semantics:satis:equiv:consistent}
Let $V$ be a set and $\Gamma\subseteq\pv$. Then we have the
equivalence:
    \[
    \Gamma\mbox{\ is satisfiable}\ \Leftrightarrow\ \Gamma\mbox{\
    is consistent}
    \]
\end{theorem}
\begin{proof}
The implication $\Rightarrow$ follows from
proposition~(\ref{logic:prop:FOPL:semantics:satis:imp:consistent}).
So we now prove $\Leftarrow$\,: we assume that $\Gamma$ is
consistent. We need to show it is satisfiable. Using Lindenbaum's
lemma~(\ref{logic:lemma:FOPL:semantics:lindenbaum}), there exists
$\Delta$ maximal consistent subset such that
$\Gamma\subseteq\Delta$. From
proposition~(\ref{logic:prop:FOPL:semantics:bijection:max:cons:val})
the characteristic function $1_{\Delta}:\pv\to 2$ is a valuation.
Furthermore, for all $\phi\in\Gamma$ we have $\phi\in\Delta$ and
consequently $1_{\Delta}(\phi)=1$. It follows that $1_{\Delta}$
satisfies $\Gamma$ and we have proved that $\Gamma$ is satisfiable.
\end{proof}

The compactness theorem which follows is a consequence of
theorem~(\ref{logic:the:FOPL:semantics:satis:equiv:consistent}) and
therefore also a consequence of Lindenbaum's
lemma~(\ref{logic:lemma:FOPL:semantics:lindenbaum}). The same
warning applies as for
theorem~(\ref{logic:the:FOPL:semantics:satis:equiv:consistent})\,:
our version of the compactness theorem is not exactly the standard
version found in the literature, because we have defined a {\em
satisfiable} set in terms of valuation and not model. It is however
equivalent to the standard version, since {\em every valuation has a
model}, as we shall see.

\index{compactness@Compactness theorem}\index{satisfiable@Finitely
satisfiable}
\begin{theorem}[Compactness]\label{logic:the:FOPL:semantics:compactness}
Let $V$ be a set and $\Gamma\subseteq\pv$. Then $\Gamma$ is
satisfiable, \ifand\ every finite subset $\Gamma_{0}\subseteq\Gamma$
is satisfiable.
\end{theorem}
\begin{proof}
First we prove the 'only if' part: so we assume that $\Gamma$ is
satisfiable. Then clearly all subsets of $\Gamma$ are satisfiable,
including the finite subsets. So we now show the 'if' part: we
assume that every finite subset of $\Gamma$ is satisfiable. We need
to show that $\Gamma$ is itself satisfiable. Using
theorem~(\ref{logic:the:FOPL:semantics:satis:equiv:consistent}), it
is sufficient to show that $\Gamma$ is consistent. So suppose to the
contrary that $\Gamma$ is inconsistent. Then we have
$\Gamma\vdash\bot$ and consequently there exists $\Gamma_{0}$ finite
such that $\Gamma_{0}\subseteq\Gamma$ and $\Gamma_{0}\vdash\bot$.
Hence, $\Gamma_{0}$ is a finite subset of $\Gamma$ which is not
consistent. Using
theorem~(\ref{logic:the:FOPL:semantics:satis:equiv:consistent}),
$\Gamma_{0}$ is a finite subset of $\Gamma$ which is not
satisfiable, contradicting the hypothesis.
\end{proof}
