Before we start, we would like to say a few words to the more expert
readers. The notion of {\em substitution congruence} seems to be
commonly known as {\em $\alpha$-equivalence} in the computer science
literature, especially within the context of $\lambda$-calculus. We
decided to use the phrase {\em substitution congruence} partly out
of ignorance and partly because first order logic does not have the
notion of $\beta$-equivalence and $\eta$-equivalence as far as we
can  tell. Instead, we shall consider other congruences on \pv\
which are the {\em permutation}, the {\em absorption} and the {\em
propositional} congruence which have no obvious counterpart in
$\lambda$-calculus. So we were tempted to change our terminology out
of due respect for established traditions, but decided otherwise in
the end. The notion of $\alpha$-equivalence and the more general
issue of variable binding in formal languages, has already received
some academic attention. For example, the interested reader may wish
to refer to Murdoch J. Gabbay~\cite{GabbaySub} and Murdoch J. Gabbay
and Aad Mathijssen~\cite{GabbayCapture}. These references also focus
on {\em capture-avoiding substitutions} which we feel is an
important topic, intimately linked to $\alpha$-equivalence. Our own
treatment of {\em capture-avoiding substitutions} will give rise to
{\em essential substitutions}.
