In the following definition, we attempt to formalize the idea that a
$\beta$-substitution as defined in definition~(\ref{logic:def:LAM:beta:subst})
does not give rise to {\em variable capture}. Given a map $\sigma:V\to\tv$,
we define a map $\kappa_{\sigma}:\tv\to{\cal P}(V)\to\{0,1\}$, the
interpretation of which is as follows: given $\phi\in\tv$ and $U\subseteq V$, 
a value of $\kappa_{\sigma}(\phi)(U)=0$ indicates that {\em variable capture} 
did occur in the substitution $\sigma^{*}(\phi)(U)$ of
definition~(\ref{logic:def:LAM:beta:subst}), indicating that the substitution
is not valid (in relation to $U$) for the formula $\phi$. Otherwise,
if $\kappa_{\sigma}(\phi)(U)=1$ then no {\em variable capture} did occur,
and the ordered pair $(\sigma,U)$ is said to be {\em valid} for $\phi$.
In the case when $\phi=x$ for some $x\in V$, we set $\kappa_{\sigma}(\phi)
(U)=1$ since no {\em variable capture} can arise as a result of the variable
substitution. When $\phi=\phi_{1}\ \phi_{2}$ we set $\kappa_{\sigma}(\phi)(U)$
to be the minimum of $\kappa_{\sigma}(\phi_{1})(U)$ and $\kappa_{\sigma}
(\phi_{2})(U)$, taking the view that no {\em variable capture}
arises from the variable substitution, unless it arises for $\phi_{1}$ or 
$\phi_{2}$. The case when $\phi=\lambda x\phi_{1}$ is the most diffcult
one: since we have:
    \[
        \sigma^{*}(\lambda x \phi_{1})(U)
        =
        \lambda x \,\sigma^{*}(\phi_{1})(U\cup\{x\})
    \]
a necessary condition to avoid {\em variable capture} is that no {\em 
variable capture} occurs in $\sigma^{*}(\phi_{1})(U\cup\{x\})$. However,
we also expect any $u\in\free(\lambda x\phi_{1})\setminus U$ to be 
{\em replaced} by $\sigma(u)$ after the substitution. Hence we must have
$x\not\in\free(\sigma(u))$, as otherwise a free variable of $\sigma(u)$
would get {\em captured} by $x$. This motivate the following:
\begin{defin}\label{logic:def:LAM:beta:valid:substitution}
    Let $V$ be a set and $\sigma:V\to\tv$ be a map. We call {\em validity
    index mapping associated with $\sigma$} the map $\kappa_{\sigma}:\tv
    \to{\cal P}(V)\to\{0,1\}$ defined by the following structural induction,
    given $\phi\in\tv$ and $U\subseteq V$:
        \begin{equation}\label{logic:eqn:LAM:index}
            \kappa_{\sigma}(\phi)(U)=\left\{
                \begin{array}{lcl}
                    1&\mbox{\ if\ }&\phi=x\\
                    \kappa_{\sigma}(\phi_{1})(U)\,\land\,
                    \kappa_{\sigma}(\phi_{2})(U)
                    &\mbox{\ if\ }&\phi=\phi_{1}\ \phi_{2}\\
                    \epsilon\,\land\,\kappa_{\sigma}(\phi_{1})(U\cup\{x\})
                    &\mbox{\ if\ }&\phi=\lambda x\phi_{1}
            \end{array}\right.
        \end{equation} 
    where it is understood in the above equation that $\epsilon\in\{0,1\}$ 
    and $\epsilon=1$ holds \ifand\ the following implication is true 
    for all $u\in V$:
        \[
            u\in\free(\lambda x\phi_{1})\setminus U
                \ \Rightarrow\ 
            x\not\in\free(\sigma(u))
        \]
    We say that $(\sigma,U)$ is {\em $\beta$-valid for} $\phi$ \ifand\ 
    $\kappa_{\sigma}(\phi)(U)=1$. \newline
    We say that $\sigma$ is $\beta$-valid for $\phi$ \ifand\
    $(\sigma,\emptyset)$ is $\beta$-valid for $\phi$.
\end{defin}

\begin{prop}\label{logic:prop:LAM:beta:valid:recursion:x:gen}
    Let $V$ be a set and $\sigma:V\to\tv$ be a map. Let $\phi\in\tv$
    of the form $\phi=x$ with $x\in V$. Then for all $U\subseteq V$, 
    $(\sigma,U)$ is $\beta$-valid for $\phi$.
\end{prop}
\begin{proof}
    This follows immediately from 
    definition~(\ref{logic:def:LAM:beta:valid:substitution}) and $\kappa_{
        \sigma}(\phi)(U)=1$.
\end{proof}
\begin{prop}\label{logic:prop:LAM:beta:valid:recursion:x}
    Let $V$ be a set and $\sigma:V\to\tv$ be a map. Let $\phi\in\tv$
    of the form $\phi=x$ with $x\in V$. Then 
    $\sigma$ is $\beta$-valid for $\phi$.
\end{prop}
\begin{proof}
    This follows immediately from 
    proposition~(\ref{logic:prop:LAM:beta:valid:recursion:x:gen}) using 
    $U=\emptyset$.
\end{proof}

\begin{prop}\label{logic:prop:LAM:beta:valid:recursion:app:gen}
    Let $V$ be a set and $\sigma:V\to\tv$ be a map. Let $\phi\in\tv$ 
    of the form $\phi=\phi_{1}\ \phi_{2}$ with $\phi_{1},\phi_{2}\in\tv$. 
    Then for all $U\subseteq V$,  $(\sigma,U)$ is $\beta$-valid for $\phi$ 
    \ifand\ it is $\beta$-valid for both $\phi_{1}$ and $\phi_{2}$.
\end{prop}
\begin{proof}
    $(\sigma,U)$ is $\beta$-valid for $\phi$ \ifand\ $\kappa_{\sigma}(\phi)(U)
    =1$ which is the same as $\kappa_{\sigma}(\phi_{1})(U)\land\kappa_{\sigma}
    (\phi_{2})(U)=1$. This in turn is equivalent to $\kappa_{\sigma}
    (\phi_{1})(U)=1$ and $\kappa_{\sigma}(\phi_{2})(U)=1$. So it is 
    equivalent to $(\sigma,U)$ being $\beta$-valid for both $\phi_{1}$ and
    $\phi_{2}$.
\end{proof}

\begin{prop}\label{logic:prop:LAM:beta:valid:recursion:app}
    Let $V$ be a set and $\sigma:V\to\tv$ be a map. Let $\phi\in\tv$ 
    of the form $\phi=\phi_{1}\ \phi_{2}$ with $\phi_{1},\phi_{2}\in\tv$. 
    Then $\sigma$ is $\beta$-valid for $\phi$ 
    \ifand\ it is $\beta$-valid for both $\phi_{1}$ and $\phi_{2}$.
\end{prop}
\begin{proof}
    This follows immediately from 
    proposition~(\ref{logic:prop:LAM:beta:valid:recursion:app:gen}) using 
    $U=\emptyset$.
\end{proof}

\begin{prop}\label{logic:prop:LAM:beta:valid:recursion:lam:gen}
    Let $V$ be a set and $\sigma:V\to\tv$ be a map. Let $\phi\in\tv$ 
    of the form $\phi=\lambda x\phi_{1}$ with $\phi_{1}\in\tv$ and $x\in V$. 
    Then for all $U\subseteq V$, $(\sigma,U)$ is $\beta$-valid for $\phi$ 
    \ifand\ $(\sigma, U\cup\{x\})$ is $\beta$-valid for $\phi_{1}$ and for 
    all $u\in V$:
    \[
        u\in\free(\lambda x\phi_{1})\setminus U
        \ \Rightarrow\ 
        x\not\in\free(\sigma(u))
    \]
\end{prop}
\begin{proof}
    From definition~(\ref{logic:def:LAM:beta:valid:substitution}), $(\sigma,U)$
    is $\beta$-valid for $\phi$ \ifand\ $\kappa_{\sigma}(\phi)(U)=1$ which is
    $\epsilon\land\kappa_{\sigma}(\phi_{1})(U\cup\{x\})=1$. This is turn is
    equivalent to $\kappa_{\sigma}(\phi_{1})(U\cup\{x\})=1$ together with 
    $\epsilon=1$. So it is equivalent to $(\sigma,U\cup\{x\})$ being 
    $\beta$-valid for $\phi_{1}$ together with $\epsilon=1$. The proposition
    follows from the fact that $\epsilon=1$ is itself equivalent to the
    above implication being true for all $u\in V$.
\end{proof}
\begin{prop}\label{logic:prop:LAM:beta:valid:recursion:lam}
    Let $V$ be a set and $\sigma:V\to\tv$ be a map. Let $\phi\in\tv$ 
    of the form $\phi=\lambda x\phi_{1}$ with $\phi_{1}\in\tv$ and $x\in V$. 
    Then $\sigma$ is $\beta$-valid for $\phi$ 
    \ifand\ $(\sigma, \{x\})$ is $\beta$-valid for $\phi_{1}$ and for 
    all $u\in V$:
    \[
        u\in\free(\lambda x\phi_{1})
        \ \Rightarrow\ 
        x\not\in\free(\sigma(u))
    \]
\end{prop}
\begin{proof}
    This follows immediately from 
    proposition~(\ref{logic:prop:LAM:beta:valid:recursion:lam:gen}) using 
    $U=\emptyset$.
\end{proof}

\begin{prop}\label{logic:prop:LAM:beta:valid:monotone}
    Let $V$ be a set and $\sigma:V\to\tv$ be a map. Let $\phi\in\tv$ and
    $U,U'\subseteq V$. Then if $U\subseteq U'$, we have the implication:
        \[
            (\sigma,U)\mbox{\ $\beta$-valid for $\phi$}
            \ \Rightarrow\
            (\sigma,U')\mbox{\ $\beta$-valid for $\phi$}
        \]
\end{prop}
\begin{proof}
    We need to show the implication holds whenever $U\subseteq U'$. We shall 
    do so by structural induction on $\phi$, using 
    theorem~(\ref{logic:the:proof:induction}) of 
    page~\pageref{logic:the:proof:induction}. So first we assume that
    $\phi=x$ for some $x\in V$. Using
    proposition~(\ref{logic:prop:LAM:beta:valid:recursion:x:gen}), 
    $(\sigma,U')$ is always $\beta$-valid for $\phi$ and the implication
    is true. So we now assume that $\phi=\phi_{1}\ \phi_{2}$ for some
    $\phi_{1},\phi_{2}\in\tv$ for which the implication is true whenever
    $U\subseteq U'$. We assume that $U\subseteq U'$ and we need to show that 
    the implication is also true for $\phi$. So we assume that $(\sigma,U)$
    is $\beta$-valid for $\phi$, and we need to show that $(\sigma,U')$ is
    $\beta$-valid for $\phi$. Using
    proposition~(\ref{logic:prop:LAM:beta:valid:recursion:app:gen}), 
    this is equivalent to showing that $(\sigma,U')$ is $\beta$-valid
    for both $\phi_{1}$ and $\phi_{2}$. Having assumed the implication
    is true for $\phi_{1}$ and $\phi_{2}$ whenever $U\subseteq U'$, it is
    sufficient to show that $(\sigma,U)$ is $\beta$-valid for both
    $\phi_{1}$ and $\phi_{2}$ which follows again from
    proposition~(\ref{logic:prop:LAM:beta:valid:recursion:app:gen}) and
    the $\beta$-validity of $(\sigma,U)$ for~$\phi$.
    So we now assume that $\phi=\lambda x\phi_{1}$ for some $x\in V$, and 
    some $\phi_{1}\in\tv$ for which the implication is true whenever 
    $U\subseteq U'$. We assume that $U\subseteq U'$ and we need to show
    the implication is also true for $\phi$. So we assume that $(\sigma,U)$
    is $\beta$-valid for $\phi$ and we need to show that $(\sigma,U')$ is
    $\beta$-valid for $\phi$. Using
    proposition~(\ref{logic:prop:LAM:beta:valid:recursion:lam:gen}) it is
    sufficient to show that $(\sigma,U'\cup\{x\})$ is $\beta$-valid for 
    $\phi_{1}$, and furthermore that for all $u\in V$, the following 
    implication holds:
    \[
        u\in\free(\lambda x\phi_{1})\setminus U'
        \ \Rightarrow\ 
        x\not\in\free(\sigma(u))
    \]
    First we focus on the implication. Having assumed that $U\subseteq U'$
    it is sufficient to show a similar implication where $U'$ is replaced by $U$:
    \[
        u\in\free(\lambda x\phi_{1})\setminus U
        \ \Rightarrow\ 
        x\not\in\free(\sigma(u))
    \]
    This last implication follows from another use of
    proposition~(\ref{logic:prop:LAM:beta:valid:recursion:lam:gen}) and the
    fact that $(\sigma,U)$ is $\beta$-valid for $\phi$. So it remains to 
    show that $(\sigma,U'\cup\{x\})$ is $\beta$-valid for $\phi_{1}$.
    Having assumed the implication is true for $\phi_{1}$ and since 
    $U\cup\{x\}\subseteq U'\cup\{x\}$, it is sufficient to show that 
    $(\sigma, U\cup\{x\})$ is $\beta$-valid for $\phi_{1}$, which 
    follows once again from
    proposition~(\ref{logic:prop:LAM:beta:valid:recursion:lam:gen}) and the
    fact that $(\sigma,U)$ is $\beta$-valid for $\phi$.
\end{proof}

\begin{prop}\label{logic:prop:LAM:freevar:of:betasubst:valid:gen}
    Let $V$ be a set and $\phi\in\tv$. Let $\sigma:V\to\tv$ be a map and
    $U\subseteq V$ such that $(\sigma,U)$ is $\beta$-valid for $\phi$. Then: 
    \[
        \free(\sigma^{*}(\phi)(U)) 
            \ =\ 
        (\free(\phi)\cap U)
        \ \cup
        \!\!\!\!\!\!
        \bigcup_{x\in\free(\phi)\setminus U} 
        \!\!\!\!\!\!
        \free(\sigma(x))
    \]
    where $\sigma^{*}:\tv\to[{\cal P}(V)\to\tv]$ is defined as in 
    definition~(\ref{logic:def:LAM:beta:subst}).
\end{prop}
\begin{proof}
    We shall prove the equality is true whenever $(\sigma,U)$ is $\beta$-valid
    for $\phi$ by structural induction on $\phi$, using
    theorem~(\ref{logic:the:proof:induction}) of 
    page~\pageref{logic:the:proof:induction}. So first we assume that $\phi=x$ 
    for some $x\in V$. Let $U\subseteq V$. Note that from
    proposition~(\ref{logic:prop:LAM:beta:valid:recursion:x:gen}) it is 
    always the case that $(\sigma,U)$ is $\beta$-valid for $\phi$. So 
    we simply need to establish the equality and we shall distinguish
    two cases, either $x\in U$ or $x\not\in U$. If $x\in U$ we have:
    \begin{eqnarray*}
        \free(\sigma^{*}(\phi)(U))
        &=&\free(\sigma^{*}(x)(U))\\
        &=&\free(\sigma_{U}(x))\\
        \mbox{$x\in U\ \rightarrow\ $}&=&\free(x)\\
         &=&\{x\}\\
         &=&\free(\phi)\\
        \mbox{$x\in U\ \rightarrow\ $}&=&\free(\phi)\cap U\\
        \mbox{$\free(\phi)\setminus U = \emptyset\ \rightarrow\ $} &=& 
            (\free(\phi)\cap U)
            \ \cup
            \!\!\!\!\!\!
            \bigcup_{x\in\free(\phi)\setminus U} 
            \!\!\!\!\!\!
            \free(\sigma(x))
    \end{eqnarray*}
    If $x\not\in U$, then:
    \begin{eqnarray*}
        \free(\sigma^{*}(\phi)(U))
        &=&\free(\sigma^{*}(x)(U))\\
        &=&\free(\sigma_{U}(x))\\
        \mbox{$x\not\in U\ \rightarrow\ $}&=&\free(\sigma(x))\\
        &=&
        \!\!\!
        \bigcup_{u\in\{x\}} 
        \!\!
        \free(\sigma(u))\\
        &=&
        \!\!\!\!\!
        \bigcup_{u\in\free(\phi)} 
        \!\!\!\!
        \free(\sigma(u))\\
        \mbox{$x\not\in U\ \rightarrow\ $}
        &=&
        \!\!\!\!\!\!\!\!
        \bigcup_{u\in\free(\phi)\setminus U} 
        \!\!\!\!\!\!\!\!
        \free(\sigma(u))\\
        \mbox{$\free(\phi)\cap U = \emptyset\ \rightarrow\ $} 
        &=& 
        (\free(\phi)\cap U)
        \ \cup
        \!\!\!\!\!\!\!
        \bigcup_{x\in\free(\phi)\setminus U} 
        \!\!\!\!\!\!\!
        \free(\sigma(x))
    \end{eqnarray*}
    We now assume that $\phi=\phi_{1}\ \phi_{2}$ where $\phi_{1}, \phi_{2}
    \in\tv$ satisfy the equality whenever $(\sigma,U)$ is $\beta$-valid. 
    Let $U\subseteq V$ such that $(\sigma,U)$ is $\beta$-valid for $\phi$.
    Using proposition~(\ref{logic:prop:LAM:beta:valid:recursion:app:gen}),
    $(\sigma,U)$ is $\beta$-valid for both $\phi_{1}$ and $\phi_{2}$.
    Hence we have:
    \begin{eqnarray*}
        \free(\sigma^{*}(\phi)(U))
        &=&\free(\sigma^{*}(\phi_{1}\ \phi_{2})(U))\\
        &=&\free(\ \sigma^{*}(\phi_{1})(U)\ \ \sigma^{*}(\phi_{2})(U)\ )\\
        &=&\free(\sigma^{*}(\phi_{1})(U))\ \cup\ \free(\sigma^{*}(\phi_{2})(U))\\
        \mbox{$(\sigma,U)$ $\beta$-valid for $\phi_{1}\ \rightarrow\ $}
        &=&(\free(\phi_{1})\cap U)
        \ \cup\ 
        \!\!\!\!\!\!\!\!\!\!
        \bigcup_{x\in\free(\phi_{1})\setminus U} 
        \!\!\!\!\!\!\!\!
        \free(\sigma(x))\\
        \mbox{$(\sigma,U)$ $\beta$-valid for $\phi_{2}\ \rightarrow\ $}
        &\cup&(\free(\phi_{2})\cap U)
        \ \cup\ 
        \!\!\!\!\!\!\!\!\!\!
        \bigcup_{x\in\free(\phi_{2})\setminus U} 
        \!\!\!\!\!\!\!\!
        \free(\sigma(x))\\
        &=&(\free(\phi_{1})\cup\free(\phi_{2}))\cap U
        \ \cup\ 
        \!\!\!\!\!\!\!\!\!\!\!\!\!\!\!\!\!\!\!\!
        \bigcup_{x\in(\free(\phi_{1})\cup\free(\phi_{2}))\setminus U}
        \!\!\!\!\!\!\!\!\!\!\!\!\!\!\!\!\!\!\!
        \free(\sigma(x))\\
        &=&(\free(\phi)\cap U)\ \cup\ 
        \!\!\!\!\!\!\!\!
        \bigcup_{x\in\free(\phi)\setminus U}
        \!\!\!\!\!\!\!
        \free(\sigma(x))
    \end{eqnarray*}
    Finally, we assume that $\phi=\lambda x\phi_{1}$ where $\phi_{1}\in\tv$ 
    satisfies the equality for all $U\subseteq V$ such that $(\sigma,U)$ is
    $\beta$-valid for $\phi_{1}$. Let $U\subseteq V$ be such that $(\sigma,U)$
    is $\beta$-valid for $\phi$. Using
    proposition~(\ref{logic:prop:LAM:beta:valid:recursion:lam:gen}) we see 
    that $(\sigma,U\cup\{x\})$ is $\beta$-valid for $\phi_{1}$: 
    \begin{eqnarray*}\free(\ \sigma^{*}(\phi)(U)\ )
        &=&\free(\ \sigma^{*}(\lambda x\phi_{1})(U)\ )\\
        &=&\free(\ \lambda x\sigma^{*}(\phi_{1})(U\cup\{x\})\ )\\
        &=&\free(\ \sigma^{*}(\phi_{1})(U\cup\{x\})\ )\setminus\{x\}\\
        \mbox{$(\sigma,U\!\cup\!\{x\})$ $\beta$-valid for 
            $\phi_{1}\rightarrow$}
        &=& [(\free(\phi_{1})\cap(U\cup\{x\}))
        \ \cup\ 
        \!\!\!\!\!\!\!\!\!\!\!\!\!\!\!\!\!
        \bigcup_{u\in\free(\phi_{1})\setminus(U\cup\{x\})}
        \!\!\!\!\!\!\!\!\!\!\!\!\!\!\!
        \free(\sigma(u))
        ]\setminus\{x\}\\
        &=&(\free(\phi_{1})\cap U\cap \{x\}^{c})
        \ \cup\ 
        \!\!\!\!\!\!\!\!\!\!\!\!\!\!\!\!\!\!
        \bigcup_{u\in\free(\phi_{1})\cap U^{c}\cap\{x\}^{c}}
        \!\!\!\!\!\!\!\!\!\!\!\!\!\!\!\!
        \free(\sigma(u))\setminus\{x\}\\
        &=&(\free(\lambda x\phi_{1})\cap U)
        \ \cup\ 
        \!\!\!\!\!\!\!\!\!\!\!\!\!
        \bigcup_{u\in\free(\lambda x\phi_{1})\setminus U}
        \!\!\!\!\!\!\!\!\!\!\!
        \free(\sigma(u))\setminus\{x\}\\
        &=&(\free(\phi)\cap U)
        \ \cup\ 
        \!\!\!\!\!\!\!\!\!
        \bigcup_{u\in\free(\phi)\setminus U} 
        \!\!\!\!\!\!\!
        \free(\sigma(u))\setminus\{x\}\\
        \mbox{see below $\ \rightarrow\ $}
        &=&(\free(\phi)\cap U)
        \ \cup\ 
        \!\!\!\!\!\!\!\!\!
        \bigcup_{u\in\free(\phi)\setminus U}
        \!\!\!\!\!\!\!
        \free(\sigma(u))
    \end{eqnarray*}
    In order to jusfify the last equality, it is sufficient to prove that 
    $\free(\sigma(u))\subseteq\{x\}^{c}$ for all $u\in\free(\phi)\setminus U$.
    However, by assumption $(\sigma,U)$ is $\beta$-valid for $\phi=\lambda
    x\phi_{1}$. Hence using 
    proposition~(\ref{logic:prop:LAM:beta:valid:recursion:lam:gen}) once more,
    for all $u\in V$ we have the implication $u\in\free(\phi)\setminus 
    U\Rightarrow x\not\in \free(\sigma(u))$ which is exactly what we need.
\end{proof}

\begin{prop}\label{logic:prop:LAM:freevar:of:betasubst:valid}
    Let $V$ be a set and $\phi\in\tv$. Let $\sigma:V\to\tv$ be a map which is 
    $\beta$-valid for $\phi$. Then, we have: 
    \[
        \free(\sigma(\phi))
        =
        \bigcup_{x\in\free(\phi)}\free(\sigma(x))
    \]
    where $\sigma:\tv\to\tv$ also denotes the associated $\beta$-substitution
    mapping.
\end{prop}
\begin{proof}
    This is an immediate application of
    proposition~(\ref{logic:prop:LAM:freevar:of:betasubst:valid}) with
    $U=\emptyset$, bearing in mind from
    definition~(\ref{logic:def:LAM:beta:valid:substitution}) that the 
    $\beta$-validity of $\sigma$ for $\phi$ is nothing but the $\beta$-validity
    of $(\sigma,\emptyset)$ for $\phi$, and from
    definition~(\ref{logic:def:LAM:beta:subst}) that $\sigma(\phi)=
    \sigma^{*}(\phi)(\emptyset)$.
\end{proof}

\begin{prop}\label{logic:prop:LAM:free:support:beta:valid}
    Let $V$ be a set, $U\subseteq V$ and $\phi\in\tv$. Let $\sigma,\tau:V\to\tv$ be maps
    which coincide on $\free(\phi)\setminus U$. Then we have the equivalence:
        \[
            (\sigma,U)\ \mbox{$\beta$-valid for $\phi$}\ 
            \Leftrightarrow\
            (\tau,U)\ \mbox{$\beta$-valid for $\phi$}
        \]
\end{prop}
\begin{proof}
    We assume that $\sigma,\tau:V\to\tv$ are given and we consider the property 
    on $\phi$ stating that for all $U\subseteq V$ the equivalence is true whenever
    $\sigma$ and $\tau$ coincide on $\free(\phi)\setminus U$. We shall prove
    this property by a structural induction argument, using 
    theorem~(\ref{logic:the:proof:induction})
    of page~\pageref{logic:the:proof:induction}. So first we assume that $\phi=x$
    for some $x\in V$. We need to show the property is true for $\phi$. So
    let $U\subseteq V$ such that $\sigma$ and $\tau$ coincide on 
    $\free(\phi)\setminus U$. We need to show that the equivalence is true,
    which is obviously the case since both $(\sigma,U)$ and $(\tau,U)$ are
    always $\beta$-valid for $\phi$ by virtue of 
    proposition~(\ref{logic:prop:LAM:beta:valid:recursion:x:gen}). So next
    we assume that $\phi=\phi_{1}\ \phi_{2}$ for some $\phi_{1},\phi_{2}\in\tv$
    for which the property is true. We need to show the property is also true
    for $\phi$. So let $U\subseteq V$ such that $\sigma$ and $\tau$ coincide
    om $\free(\phi)\setminus U$. We need to show that the equivalence is true.
    By symmetry, it is sufficient to focus on one implication. So we assume
    that $(\sigma,U)$ is $\beta$-valid for $\phi$. We need to show that 
    $(\tau,U)$ is also $\beta$-valid for $\phi$. Using
    proposition~(\ref{logic:prop:LAM:beta:valid:recursion:app:gen}), it is
    sufficient to show that $(\tau,U)$ is $\beta$-valid for both $\phi_{1}$
    and $\phi_{2}$. However, having assumed the property is true for $\phi_{1}$
    and $\phi_{2}$ and since $\sigma,\tau$ coincide on $\free(\phi)\setminus U$,
    they also coincide on $\free(\phi_{1})\setminus U$ and $\free(\phi_{2})
    \setminus U$, and hence the equivalence is true for both $\phi_{1}$ and 
    $\phi_{2}$ in relation to $U$. It is therefore sufficient to prove that
    $(\sigma,U)$ is $\beta$-valid for both $\phi_{1}$ and $\phi_{2}$ which 
    follows from proposition~(\ref{logic:prop:LAM:beta:valid:recursion:app:gen})
    and our assumption that $(\sigma,U)$ is $\beta$-valid for $\phi$. Finally,
    we assume that $\phi=\lambda x\phi_{1}$ for some $x\in V$ and $\phi_{1}\in\tv$
    for which the property is true. We need to show the property is also true
    for $\phi$. So let $U\subseteq V$ such that $\sigma$ and $\tau$ coincide
    on $\free(\phi)\setminus U$. We need to show the equivalence is true for 
    $\phi$ is relation to $U$. By symmetry, it is sufficient to focus on one 
    implication. So we assume that $(\sigma,U)$ is $\beta$-valid for $\phi$.
    We need to show that $(\tau,U)$ is also $\beta$-valid for $\phi$. Using
    proposition~(\ref{logic:prop:LAM:beta:valid:recursion:lam:gen}), it is
    sufficient to establish two facts: on the one hand we need to prove
    that $(\tau, U\cup\{x\})$ is $\beta$-valid for $\phi_{1}$, and on the
    other hand we need to show that $x\not\in\free(\tau(u))$ whenever
    $u\in\free(\phi)\setminus U$. First we focus on the $\beta$-validity
    of $(\tau,U\cup\{x\})$ for $\phi_{1}$. Having assumed that $(\sigma,U)$
    is $\beta$-valid for $\phi$, using 
    proposition~(\ref{logic:prop:LAM:beta:valid:recursion:lam:gen}) once more
    we see that $(\sigma, U\cup\{x\})$ is $\beta$-valid for $\phi_{1}$.
    It is therefore sufficient to establish that the equivalence is true
    in relation to $\phi_{1}$ and $U\cup\{x\}$, and since the property is
    true for $\phi_{1}$ it is sufficient to prove that $\sigma$ and $\tau$
    coincide on $\free(\phi_{1})\setminus(U\cup\{x\})$. TODO
\end{proof}

\begin{prop}\label{logic:prop:LAM:var:inter:beta:valid:gen}
    Let $V$ be a set and $\phi\in\tv$. Let $\sigma:V\to\tv$ be a map. Then for
    all $U\subseteq V$, we have the following implication:
        \[
            \var(\phi)\ \cap
            \!\!\!\!\!\!\bigcup_{u\in\free(\phi)\setminus U}\!\!\!\!\!\!
            \free(\sigma(u))\setminus\{u\}
            =\emptyset
            \ \ \Rightarrow\ \ 
            \mbox{$(\sigma,U)$ is $\beta$-valid for $\phi$}
        \]
\end{prop}
\begin{proof}
    We shall prove this implication by structural induction on $\phi$, using 
    theorem~(\ref{logic:the:proof:induction}) of 
    page~\pageref{logic:the:proof:induction}. 
    So first we assume that $\phi=x$ for some $x\in V$. Then for all $U\subseteq 
    V$, from proposition~(\ref{logic:prop:LAM:beta:valid:recursion:x:gen}) it is
    always the case that $(\sigma, U)$ is $\beta$-valid for $\phi$ and the 
    implication is therefore satisfied. So we now assume that $\phi=\phi_{1}\ 
    \phi_{2}$ for some $\phi_{1},\phi_{2}\in\tv$ for which the implication
    is true for all $U\subseteq V$. Given $U\subseteq V$, we need to show that
    the implication is also true for $\phi$. Hence we assume:
        \begin{equation}\label{logic:eqn:LAM:var:inter:beta:valid:gen}
            \var(\phi)\ \cap
            \!\!\!\!\!\!\bigcup_{u\in\free(\phi)\setminus U}\!\!\!\!\!\!
            \free(\sigma(u))\setminus\{u\}
            =\emptyset
        \end{equation}
    and we shall show that $(\sigma,U)$ is $\beta$-valid for $\phi$.
    From $\var(\phi)=\var(\phi_{1})\cup\var(\phi_{2})$
    and $\free(\phi)=\free(\phi_{1})\cup\free(\phi_{2})$, in particular
    we have:
        \[
            \var(\phi_{1})\ \cap
            \!\!\!\!\!\!\bigcup_{u\in\free(\phi_{1})\setminus U}\!\!\!\!\!\!
            \free(\sigma(u))\setminus\{u\}
            =\emptyset
        \]
    and: 
        \[
            \var(\phi_{2})\ \cap
            \!\!\!\!\!\!\bigcup_{u\in\free(\phi_{2})\setminus U}\!\!\!\!\!\!
            \free(\sigma(u))\setminus\{u\}
            =\emptyset
        \]
    Having assumed the implication is true for both $\phi_{1}$ and $\phi_{2}$ 
    we see that $(\sigma,U)$ is therefore $\beta$-valid for $\phi_{1}$ and 
    $\phi_{2}$ and the $\beta$-validity of $(\sigma,U)$ for $\phi$ follows 
    from proposition~(\ref{logic:prop:LAM:beta:valid:recursion:app:gen}).
    So we now assume that $\phi=\lambda x\phi_{1}$ for some $x\in V$ and
    some $\phi_{1}\in\tv$ for which the implication is true for all
    $U\subseteq V$. Given $U\subseteq V$, we need to show the implication
    is also true for $\phi$. So once again we assume that
    equation~(\ref{logic:eqn:LAM:var:inter:beta:valid:gen}) holds and we
    need to show that $(\sigma,U)$ is $\beta$-valid for $\phi$. Using
    proposition~(\ref{logic:prop:LAM:beta:valid:recursion:lam:gen}), we 
    therefore need to show that $(\sigma, U\cup\{x\})$ is $\beta$-valid
    for $\phi_{1}$ and furthermore that for all $u\in V$ we have:
        \[
            u\in\free(\lambda x\phi_{1})\setminus U
            \ \Rightarrow\ 
            x\not\in\free(\sigma(u))
        \]
    We shall first prove this last implication. So we assume $u\in\free
    (\lambda x\phi_{1})\setminus U$ and we need to show that 
    $x\not\in\free(\sigma(u))$. From $u\in\free(\phi)\setminus U$, looking
    at equation~(\ref{logic:eqn:LAM:var:inter:beta:valid:gen}) we see that
    the following equality must hold:
        \[
            \var(\phi)\cap\free(\sigma(u))\cap\{u\}^{c}=\emptyset
        \]
    Hence, in order to show that $x\not\in\free(\sigma(u))$ it is 
    sufficient to show that $x\in\var(\phi)$ and $x\in\{u\}^{c}$.
    However $x\in\var(\phi)$ is clear since $\phi=\lambda x\phi_{1}$.
    As for $u\neq x$, this follows immediately from $u\in\free(\phi)=
    \free(\phi_{1})\setminus\{x\}$. So it now remains to show that
    $(\sigma,U\cup\{x\})$ is $\beta$-valid for $\phi_{1}$. Having
    assumed the implication is true for $\phi_{1}$ for all $U\subseteq V$,
    it is sufficient to prove:
        \[
            \var(\phi_{1})\ \cap
            \!\!\!\!\!\!\!\!
            \bigcup_{u\in\free(\phi_{1})\setminus (U\cup\{x\})}
            \!\!\!\!\!\!\!\!
            \free(\sigma(u))\setminus\{u\}
            =\emptyset
        \]
    which is:
        \[
            \var(\phi_{1})\ \cap
            \!\!\!\!\!\!\bigcup_{u\in\free(\phi)\setminus U}\!\!\!\!\!\!
            \free(\sigma(u))\setminus\{u\}
            =\emptyset
        \]
    and which follows immediately from 
    equation~(\ref{logic:eqn:LAM:var:inter:beta:valid:gen}) and $\var(\phi_{1})
    \subseteq\var(\phi)$.
 \end{proof}

\begin{prop}\label{logic:prop:LAM:var:inter:beta:valid}
    Let $V$ be a set, $\phi\in\tv$ and $\sigma\!:\!V\!\to\!\tv$ be a map. Then:
        \[
            \var(\phi)\ \cap
            \!\!\!\!\bigcup_{u\in\free(\phi)}\!\!\!\!
            \free(\sigma(u))\setminus\{u\}
            =\emptyset
            \ \ \Rightarrow\ \ 
            \mbox{$\sigma$ is $\beta$-valid for $\phi$}
        \]
\end{prop}
\begin{proof}
    This is an immediate application of 
    proposition~(\ref{logic:prop:LAM:var:inter:beta:valid:gen}) with $U=\emptyset$.
\end{proof}

\begin{prop}\label{logic:prop:LAM:beta:validsub:singlevar:gen}
    Let $V$ be a set, $x\in V$ and $\theta,\phi\in\tv$. Let $U\subseteq V$. Then:
        \[
            x\not\in\free(\phi)\setminus U
            \ \lor\
            \var(\phi)\cap\free(\theta)\subseteq\{x\}
            \ \Rightarrow\ 
            \mbox{$([\theta/x],U)$ $\beta$-valid for $\phi$}
        \]
\end{prop}
\begin{proof}
    We assume that $x\not\in\free(\phi)\setminus U$ or that $x$ is the only 
    possible element of the intersection $\var(\phi)\cap\free(\theta)$. We 
    need to show that $([\theta/x],U)$ is $\beta$-valid for $\phi$. Using
    proposition~(\ref{logic:prop:LAM:var:inter:beta:valid:gen}), we simply
    need to prove the quality:
        \[
            \var(\phi)\ \cap
            \!\!\!\!\!\!\bigcup_{u\in\free(\phi)\setminus U}\!\!\!\!\!\!
            \free([\theta/x](u))\setminus\{u\}
            =\emptyset
        \]
    So let $u\in\free(\phi)\setminus U$. We need to show the simpler equality:
        \begin{equation}\label{logic:eqn:LAM:beta:validsub:singlevar:gen}
            \var(\phi)\ \cap
            \free([\theta/x](u))\setminus\{u\}
            =\emptyset
        \end{equation}
    We shall distinguish two cases: first we assume that $u\neq x$. Then we have
    $[\theta/x](u)=u$ and consequently $\free([\theta/x](u))\setminus\{u\}
    =\emptyset$ and equation~(\ref{logic:eqn:LAM:beta:validsub:singlevar:gen})
    is clear. Next we assume that $u=x$. In this case we have $x\in\free(\phi)
    \setminus U$ which means that the condition $x\not\in\free(\phi)\setminus U$
    is not satisfied, so by assumption we know that $\var(\phi)\cap\free(\theta)
    \subseteq\{x\}$. Furthermore, we have $[\theta/x](u)=\theta$ and
    equation~(\ref{logic:eqn:LAM:beta:validsub:singlevar:gen}) reduces to
    $\var(\phi)\cap\free(\theta)\setminus\{x\}=\emptyset$, and this equation
    immediately follows from our assumption.
\end{proof}



\begin{prop}\label{logic:prop:LAM:beta:validsub:singlevar}
    Let $V$ be a set, $x\in V$ and $\theta,\phi\in\tv$. Then we have:
        \[
            x\not\in\free(\phi)
            \ \lor\
            \var(\phi)\cap\free(\theta)\subseteq\{x\}
            \ \Rightarrow\ 
            \mbox{$[\theta/x]$ $\beta$-valid for $\phi$}
        \]
\end{prop}
\begin{proof}
    This is an immediate application of 
    proposition~(\ref{logic:prop:LAM:beta:validsub:singlevar:gen}) with 
    $U=\emptyset$.
\end{proof}

\begin{prop}\label{logic:prop:LAM:beta:valid:composition:gen}
    Let $V$ be a set, $U\subseteq U' \subseteq V$ and $\sigma,\tau:V\to\tv$ 
    be maps. Let $\phi\in\tv$. Then, if $(\tau,U)$ is $\beta$-valid for $\phi$ 
    we have:
        \[
            (\sigma_{*}(U')\circ\tau)_{*}(U)(\phi)
            =(\,\sigma_{*}(U')\circ\tau_{*}(U)\,)(\phi)
        \]
    where for all $\chi:V\to\tv$, the map $\chi_{*}$ denotes the flipped version 
    of $\chi^{*}$ defined by $\chi_{*}(U)(\phi)=\chi^{*}(\phi)(U)$, and where 
    $\chi^{*}$ is defined as per definition~(\ref{logic:def:LAM:beta:subst}).
\end{prop}
\begin{proof}
    We assume $\sigma,\tau:V\to\tv$ given and we consider the property
    that for all $U\subseteq V$, if $(\tau, U)$ is $\beta$-valid for $\phi$
    then for all $U'\subseteq V$ with $U\subseteq U'$, the above equality
    holds. We shall prove this property by structural induction on $\phi$, 
    using theorem~(\ref{logic:the:proof:induction})
    of page~\pageref{logic:the:proof:induction}. So first we assume that 
    $\phi=x$ for some $x\in V$. Let $U\subseteq V$. Since from 
    proposition~(\ref{logic:prop:LAM:beta:valid:recursion:x:gen}) $(\tau,U)$ 
    is always $\beta$-valid for $\phi$, we simply need to prove that the 
    equality holds for all $U'\subseteq V$ with $U\subseteq U'$. We shall 
    distinguish two cases: first we assume that $x\in U$. Then we have:
        \begin{eqnarray*}(\sigma_{*}(U')\circ\tau)_{*}(U)(\phi)
            &=&(\sigma_{*}(U')\circ\tau)_{*}(U)(x)\\
            \mbox{flipping\ $\to$\ }
            &=&(\sigma_{*}(U')\circ\tau)^{*}(x)(U)\\
            \mbox{def.~(\ref{logic:def:LAM:beta:subst}), $x\in U\ \to\ $}
            &=&x\\
            \mbox{$U\subseteq U'$ so $x\in U'\ \to\ $}
            &=&\sigma^{*}(x)(U')\\
            \mbox{flipping\ $\to$\ }
            &=&\sigma_{*}(U')(x)\\
            \mbox{$x\in U\ \to\ $}
            &=&\sigma_{*}(U')(\,\tau^{*}(x)(U)\,)\\
            \mbox{flipping\ $\to$\ }
            &=&\sigma_{*}(U')(\,\tau_{*}(U)(x)\,)\\
            &=&(\,\sigma_{*}(U')\circ \tau_{*}(U)\,)(x)\\
            &=&(\,\sigma_{*}(U')\circ \tau_{*}(U)\,)(\phi)
        \end{eqnarray*}
    \noindent
    We now assume that $x\not\in U$. Then we have:
        \begin{eqnarray*}(\sigma_{*}(U')\circ\tau)_{*}(U)(\phi)
            &=&(\sigma_{*}(U')\circ\tau)_{*}(U)(x)\\
            \mbox{flipping\ $\to$\ }
            &=&(\sigma_{*}(U')\circ\tau)^{*}(x)(U)\\
            \mbox{def.~(\ref{logic:def:LAM:beta:subst}), $x\not\in U\ \to\ $}
            &=&(\sigma_{*}(U')\circ\tau)(x)\\
            &=&\sigma_{*}(U')(\tau(x))\\
            \mbox{$x\not\in U\ \to\ $}
            &=&\sigma_{*}(U')(\tau^{*}(x)(U))\\
            \mbox{flipping\ $\to$\ }
            &=&\sigma_{*}(U')(\tau_{*}(U)(x))\\
            &=&(\,\sigma_{*}(U')\circ\tau_{*}(U)\,)(x)\\
            &=&(\,\sigma_{*}(U')\circ\tau_{*}(U)\,)(\phi)
        \end{eqnarray*}
    Next, we assume that $\phi=\phi_{1}\ \phi_{2}$ for some $\phi_{1},\phi_{2}
    \in\tv$ for which the property is true. We need to show that the
    property is also true for $\phi$. So let $U\subseteq V$ such that 
    $(\tau, U)$ is $\beta$-valid for $\phi$. Given $U'\subseteq V$ with
    $U\subseteq U'$, we need to show the equality holds for $\phi$. Note that 
    from proposition~(\ref{logic:prop:LAM:beta:valid:recursion:app:gen}), 
    $(\tau,U)$ is $\beta$-valid for both $\phi_{1}$ and $\phi_{2}$ and 
    having assumed the property is true for these, so is the equality for
    all $U'\subseteq V$ with $U\subseteq U'$. Hence, we have:
        \begin{eqnarray*}(\sigma_{*}(U')\circ\tau)_{*}(U)(\phi)
            &=&(\sigma_{*}(U')\circ\tau)_{*}(U)(\phi_{1}\ \phi_{2})\\
            \mbox{flipping\ $\to$\ }
            &=&(\sigma_{*}(U')\circ\tau)^{*}(\phi_{1}\ \phi_{2})(U)\\
            \mbox{def.~(\ref{logic:def:LAM:beta:subst})\ $\to$\ }
            &=&(\sigma_{*}(U')\circ\tau)^{*}(\phi_{1})(U)\ \ 
               (\sigma_{*}(U')\circ\tau)^{*}(\phi_{2})(U)\\
            \mbox{flipping\ $\to$\ }
            &=&(\sigma_{*}(U')\circ\tau)_{*}(U)(\phi_{1})\ \ 
               (\sigma_{*}(U')\circ\tau)_{*}(U)(\phi_{2})\\
            \mbox{induction hypothesis$\ \to\ $}
            &=&(\,\sigma_{*}(U')\circ\tau_{*}(U)\,)(\phi_{1})\ \ 
               (\,\sigma_{*}(U')\circ\tau_{*}(U)\,)(\phi_{2})\\
            &=&\sigma_{*}(U')(\,\tau_{*}(U)(\phi_{1})\,)\ \ 
               \sigma_{*}(U')(\,\tau_{*}(U)(\phi_{2})\,)\\
            \mbox{flipping\ $\to$\ }
            &=&\sigma^{*}(\,\tau_{*}(U)(\phi_{1})\,)(U')\ \ 
               \sigma^{*}(\,\tau_{*}(U)(\phi_{2})\,)(U')\\
            \mbox{def.~(\ref{logic:def:LAM:beta:subst})\ $\to$\ }
            &=&\sigma^{*}(\,\tau_{*}(U)(\phi_{1})\ \ 
                            \tau_{*}(U)(\phi_{2})\,)(U')\\ 
            \mbox{flipping\ $\to$\ }
            &=&\sigma^{*}(\,\tau^{*}(\phi_{1})(U)\ \ 
                            \tau^{*}(\phi_{2})(U)\,)(U')\\
            \mbox{def.~(\ref{logic:def:LAM:beta:subst})\ $\to$\ }
            &=&\sigma^{*}(\,\tau^{*}(\phi_{1}\ \phi_{2})(U)\,)(U')\\
            &=&\sigma^{*}(\,\tau^{*}(\phi)(U)\,)(U')\\
            \mbox{flipping\ $\to$\ }
            &=&\sigma_{*}(U')(\,\tau_{*}(U)(\phi)\,)\\
            &=&(\,\sigma_{*}(U')\circ\tau_{*}(U)\,)(\phi)
        \end{eqnarray*}
    Finally, we assume that $\phi=\lambda x\phi_{1}$ for some $x\in V$, and
    some $\phi_{1}\in\tv$ for which the property is true. We need to show
    that the property is also true for $\phi$. So let $U\subseteq V$ such 
    that $(\tau,U)$ is $\beta$-valid for $\phi$. Given $U'\subseteq V$  with
    $U\subseteq U'$, we need to show that the equality holds for $\phi$. 
    Note that from 
    proposition~(\ref{logic:prop:LAM:beta:valid:recursion:lam:gen}), 
    $(\tau, U\cup\{x\})$ is $\beta$-valid for $\phi_{1}$ and having assumed 
    the property is true for $\phi_{1}$, so is the equality for all subsets 
    $U'\subseteq V$ with $U\cup\{x\}\subseteq U'$. Hence, we have:
        \begin{eqnarray*}(\sigma_{*}(U')\circ\tau)_{*}(U)(\phi)
            &=&(\sigma_{*}(U')\circ\tau)_{*}(U)(\lambda x\phi_{1})\\
            \mbox{flipping\ $\to$\ }
            &=&(\sigma_{*}(U')\circ\tau)^{*}(\lambda x\phi_{1})(U)\\
            \mbox{def.~(\ref{logic:def:LAM:beta:subst})\ $\to$\ }
            &=&\lambda x(\sigma_{*}(U')\circ\tau)^{*}(\phi_{1})(U\cup\{x\})\\
            \mbox{see below\ $\to$\ }
            &=&\lambda x(\sigma_{*}(U'\cup\{x\})\circ
                         \tau)^{*}(\phi_{1})(U\cup\{x\})\\
            \mbox{flipping\ $\to$\ }
            &=&\lambda x(\sigma_{*}(U'\cup\{x\})\circ
                         \tau)_{*}(U\cup\{x\})(\phi_{1})\\
            \mbox{induction, $U\cup\{x\}\subseteq U'\cup\{x\}\ \to\ $}
            &=&\lambda x(\ \sigma_{*}(U'\cup\{x\})\,\circ\,
                           \tau_{*}(U\cup\{x\})\ )(\phi_{1})\\
            &=&\lambda x \sigma_{*}(U'\cup\{x\})(\,
                         \tau_{*}(U\cup\{x\})(\phi_{1})\,)\\
            \mbox{flipping\ $\to$\ }
            &=&\lambda x \sigma^{*}(\,\tau_{*}(U\cup\{x\})(\phi_{1})\,)
                         (U'\cup\{x\})\\
            \mbox{def.~(\ref{logic:def:LAM:beta:subst})\ $\to$\ }
            &=&\sigma^{*}(\,\lambda x\tau_{*}(U\cup\{x\})(\phi_{1})\,)(U')\\
            \mbox{flipping\ $\to$\ }
            &=&\sigma_{*}(U')(\,\lambda x\tau_{*}(U\cup\{x\})(\phi_{1})\,)\\
            \mbox{flipping\ $\to$\ }
            &=&\sigma_{*}(U')(\,\lambda x\tau^{*}(\phi_{1})(U\cup\{x\})\,)\\
            \mbox{def.~(\ref{logic:def:LAM:beta:subst})\ $\to$\ }
            &=&\sigma_{*}(U')(\,\tau^{*}(\lambda x\phi_{1})(U)\,)\\
            &=&\sigma_{*}(U')(\,\tau^{*}(\phi)(U)\,)\\
            \mbox{flipping\ $\to$\ }
            &=&\sigma_{*}(U')(\,\tau_{*}(U)(\phi)\,)\\
            &=&(\,\sigma_{*}(U')\circ\tau_{*}(U)\,)(\phi)
        \end{eqnarray*}
    In order for us to complete the proof, we need to justify the quality:
        \[
            (\sigma_{*}(U')\circ\tau)^{*}(\phi_{1})(U\cup\{x\})
            =
            (\sigma_{*}(U'\cup\{x\})\circ\tau)^{*}(\phi_{1})(U\cup\{x\})
        \]
    This equality follows from 
    proposition~(\ref{logic:prop:LAM:freevar:beta:support:gen}) provided
    we show that the two functions $\sigma_{*}(U')\circ\tau$ and
    $\sigma_{*}(U'\cup\{x\})\circ\tau$ coincide on the set $\free(\phi_{1})
    \setminus(U\cup\{x\})$. So let $u\in\free(\phi_{1})\setminus(U\cup\{x\})$.
    It remains to show that:
        \[
            (\sigma_{*}(U')\circ\tau)(u) 
            =
            (\sigma_{*}(U'\cup\{x\})\circ\tau)(u) 
        \]
    or equivalently:
        \[
            \sigma^{*}(\tau(u))(U')
            =
            \sigma^{*}(\tau(u))(U'\cup\{x\})
        \]
    This last equality follows from
    proposition~(\ref{logic:prop:LAM:freevar:beta:intersect:gen}) provided
    we show that:
        \[
            U' \cap\free(\tau(u)) = (U'\cup\{x\})\cap\free(\tau(u))
        \]
    It is therefore sufficient to prove that $\{x\}\cap\free(\tau(u))=\emptyset$,
    or in other words that $x\not\in\free(\tau(u))$. However, we assumed that 
    $(\tau,U)$ is $\beta$-valid for $\phi=\lambda x\phi_{1}$. Using
    proposition~(\ref{logic:prop:LAM:beta:valid:recursion:lam:gen}) once more
    we have the implication:
        \[
            u\in\free(\lambda x\phi_{1})\setminus U\ 
            \Rightarrow\ 
            x\not\in\free(\tau(u))
        \]
    It is therefore sufficient to prove that $u\in\free(\lambda x\phi_{1})
    \setminus U$ which follows from the assumption $u\in\free(\phi_{1})
    \setminus(U\cup\{x\})=\free(\phi_{1})\cap\{x\}^{c}\cap U^{c}
    =\free(\lambda x\phi_{1})\setminus U$.
\end{proof}

\begin{prop}\label{logic:prop:LAM:beta:valid:composition}
    Let $V$ be a set and $\sigma,\tau:V\to\tv$ be maps. Let $\phi\in\tv$. Then
    if $\tau$ is $\beta$-valid for $\phi$, we have:
        \[
            (\sigma'\circ\tau)' (\phi) = (\sigma'\circ\tau')(\phi)
        \]
    where given $\chi:V\to\tv$, the map $\chi':\tv\to\tv$ denotes the 
    associated $\beta$-substitution mapping of 
    definition~(\ref{logic:def:LAM:beta:subst}).
\end{prop}
\begin{proof}
    This is an immediate application of 
    proposition~(\ref{logic:prop:LAM:beta:valid:composition:gen}) with
    $U=U'=\emptyset$: if we recall
    the notation $\chi_{*}(U)(\phi)=\chi^{*}(\phi)(U)$ where $\chi^{*}:\tv\to
    {\cal P}(V)\to\tv$ is defined as per 
    definition~(\ref{logic:def:LAM:beta:subst}) given $\chi:V\to\tv$, 
    if $\tau$ is $\beta$-valid for $\phi$:
        \begin{eqnarray*}(\sigma'\circ\tau)'(\phi)
            &=&(\sigma'\circ\tau)^{*}(\phi)(\emptyset)
            \mbox{\ \ \ \ \ $\leftarrow$\ def.~(\ref{logic:def:LAM:beta:subst})}\\
            \mbox{def.~(\ref{logic:def:LAM:beta:subst}) and flipping
            \ $\to$\ }
            &=&(\sigma_{*}(\emptyset)\circ\tau)^{*}(\phi)(\emptyset)\\
            \mbox{flipping\ $\to$\ }
            &=&(\sigma_{*}(\emptyset)\circ\tau)_{*}(\emptyset)(\phi)\\
            \mbox{prop.~(\ref{logic:prop:LAM:beta:valid:composition:gen})
            $(\tau,\emptyset)$ is $\beta$-valid for $\phi$\ $\to$\ }
            &=&(\,\sigma_{*}(\emptyset)\circ\tau_{*}(\emptyset)\,)(\phi)\\
            \mbox{def.~(\ref{logic:def:LAM:beta:subst}) and flipping\ $\to$\ }
            &=&(\sigma'\circ\tau')(\phi)
        \end{eqnarray*}
\end{proof}

\noindent
{\bf Remark}: A counter-example for
proposition~(\ref{logic:prop:LAM:beta:valid:composition}) when $\tau$ is not
$\beta$-valid for $\phi$ is as follows: consider $\phi=\lambda x y$ where
$x\neq y$. Assume that $\sigma,\tau:V\to\tv$ are such that
$\tau(y)=x$ and $\sigma(x)=y$. From
proposition~(\ref{logic:prop:LAM:beta:valid:recursion:lam}) we can see that
$\tau$ is not $\beta$-valid for $\phi$ since we have $y\in\free(\lambda x y)$
and yet $x\in\free(\tau(y))$. We have:
    \begin{eqnarray*}(\sigma'\circ\tau)' (\phi)
        &=&(\sigma'\circ\tau)' (\lambda x y)\\
        \mbox{def.~(\ref{logic:def:LAM:beta:subst})\ $\to$\ }
        &=&(\sigma'\circ\tau)^{*} (\lambda x y)(\emptyset)\\
        \mbox{def.~(\ref{logic:def:LAM:beta:subst})\ $\to$\ }
        &=&\lambda x(\sigma'\circ\tau)^{*} (y)(\{x\})\\
        y\not\in\{x\}\ \to\ 
        &=&\lambda x(\sigma'\circ\tau) (y)\\
        \tau(y)=x\ \to\ 
        &=&\lambda x\sigma'(x)\\
        \mbox{def.~(\ref{logic:def:LAM:beta:subst})\ $\to$\ }
        &=&\lambda x\sigma^{*}(x)(\emptyset)\\
        x\not\in\emptyset\ \to\ 
        &=&\lambda x\sigma(x)\\
        \sigma(x)=y\ \to\ 
        &=&\lambda x y\\
        x\neq y\ \to\ 
        &\neq&\lambda x x\\
        x\in\{x\}\ \to\ 
        &=&\lambda x \sigma^{*}(x)(\{x\})\\
        \mbox{def.~(\ref{logic:def:LAM:beta:subst})\ $\to$\ }
        &=&\sigma^{*}(\lambda x x)(\emptyset)\\
        \mbox{def.~(\ref{logic:def:LAM:beta:subst})\ $\to$\ }
        &=&\sigma'(\,\lambda x x\,)\\
        \tau(y)=x\ \to\ 
        &=&\sigma'(\,\lambda x\tau(y)\,)\\
        y\not\in\{x\}\ \to\ 
        &=&\sigma'(\,\lambda x\tau^{*}(y)(\{x\})\,)\\
        \mbox{def.~(\ref{logic:def:LAM:beta:subst})\ $\to$\ }
        &=&\sigma'(\,\tau^{*}(\lambda x y)(\emptyset)\,)\\
        \mbox{def.~(\ref{logic:def:LAM:beta:subst})\ $\to$\ }
        &=&\sigma'(\tau'(\lambda x y))\\
        &=&(\sigma'\circ\tau')(\phi)
    \end{eqnarray*}

