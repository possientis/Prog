\begin{defin}\label{logic:def:LAM:beta:valid:substitution}
Let $V$ be a set and $\sigma:V\to\tv$ be a map. Let
$\phi\in\tv$. We say that $\sigma$ is {\em $\beta$-valid for} $\phi$ \ifand\
for every sub-formula $\psi\preceq\phi$:
    \[
        x\in\free(\psi)\ \Rightarrow\ \free(\sigma(x))\subseteq\free(\sigma(\psi))
    \]
where $\sigma:\tv\to\tv$ also denotes the associated
$\beta$-substitution mapping.
\end{defin}

\begin{prop}\label{logic:prop:LAM:beta:valid:subformula}
Let $V$ be a set and $\sigma:V\to\tv$ be a map. Let
$\phi\in\tv$. Then $\sigma$ is $\beta$-valid for $\phi$ \ifand\ it is $\beta$-valid
for any sub-formula $\psi\preceq\phi$.
\end{prop}
\begin{proof}
Since $\phi\preceq\phi$, i.e. $\phi$ is a sub-formula of itself, the
'if' part of this proposition is clear. So we now prove the 'only
if' part. So suppose $\sigma$ is $\beta$-valid for $\phi$ and let
$\psi\preceq\phi$. We need to show that $\sigma$ is also $\beta$-valid for
$\psi$. So let $\chi\preceq\psi$ and let $x\in\free(\chi)$. We need
    to show that $\free(\sigma(x))\subseteq\free(\sigma(\chi))$, which follows
immediately from the $\beta$-validity of $\sigma$ for $\phi$ and the fact
(by transitivity) that $\chi\preceq\phi$, i.e. that $\chi$ is also a
sub-formula of $\phi$.
\end{proof}


