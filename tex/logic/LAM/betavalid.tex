In the following definition, we attempt to formalize the idea that a
$\beta$-substitution as defined in definition~(\ref{logic:def:LAM:beta:subst})
does not give rise to {\em variable capture}. Given a map $\sigma:V\to\tv$,
we define a map $\kappa_{\sigma}:\tv\to{\cal P}(V)\to\{0,1\}$, the
interpretation of which is as follows: given $\phi\in\tv$ and $U\subseteq V$, 
a value of $\kappa_{\sigma}(\phi)(U)=0$ indicates that variable capture did
occur in the substitution $\sigma^{*}(\phi)(U)$ of
definition~(\ref{logic:def:LAM:beta:subst}).
\begin{defin}\label{logic:def:LAM:beta:valid:index}
    Let $V$ be a set and $\sigma:V\to\tv$ be a map. We call {\em validity
    index mapping associated with $\sigma$} the map $\kappa_{\sigma}:\tv
    \to{\cal P}(V)\to\{0,1\}$ defined by the following structural induction,
    given $\phi\in\tv$ and $U\subseteq V$:
        \begin{equation}\label{logic:eqn:LAM:index}
            \kappa_{\sigma}(\phi)(U)=\left\{
                \begin{array}{lcl}
                    1&\mbox{\ if\ }&\phi=x\\
                    \kappa_{\sigma}(\phi_{1})(U)\,\land\,
                    \kappa_{\sigma}(\phi_{2})(U)
                    &\mbox{\ if\ }&\phi=\phi_{1}\ \phi_{2}\\
                    \epsilon\,\land\,\kappa_{\sigma}(\phi_{1})(U\cup\{x\})
                    &\mbox{\ if\ }&\phi=\lambda x\phi_{1}
            \end{array}\right.
        \end{equation} 
    where it is understood in the above equation that $\epsilon\in\{0,1\}$ 
    and $\epsilon=1$ holds \ifand\ the following implication is true 
    for all $u\in V$:
        \[
            u\in\free(\lambda x\phi_{1})\setminus U
                \ \Rightarrow\ 
            x\not\in\free(\sigma(u))
        \]
    We say that $(\sigma,U)$ is {\em $\beta$-valid for} $\phi$ \ifand\ 
    $\kappa_{\sigma}(\phi)(U)=1$. \newline
    We say that $\sigma$ is $\beta$-valid for $\phi$ \ifand\
    $\kappa_{\sigma}(\phi)(\emptyset)=1$.
\end{defin}
