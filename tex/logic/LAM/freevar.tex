\begin{defin}\label{logic:def:LAM:free:variable}
    Let $V$ be a set. The map $\free:\tv\to {\cal P}(V)$ defined by the
    following structural recursion is called {\em free variable mapping on \tv}:
    \begin{equation}\label{logic:eqn:LAM:free:var:recursion}
        \forall\phi\in\tv\ ,\ 
            \free(\phi) =
                \left\{\begin{array}{lcl}
                    \{x\}&\mbox{\ if\ }&\phi=x
                    \\
                    \free(\phi_{1})\cup\free(\phi_{2}) 
                        &\mbox{\ if\ }&
                    \phi=\phi_{1}\ \phi_{2}
                    \\
                    \free(\phi_{1})\setminus\{x\}
                        &\mbox{\ if\ }&
                    \phi=\lambda x\phi_{1}
                \end{array}\right.
    \end{equation}
    We say that $x\in V$ is a {\em free variable} of $\phi\in\tv$ 
    \ifand\ $x\in\free(\phi)$.
\end{defin}
\begin{prop}\label{logic:prop:LAM:free:variable}
    The structural recursion of 
    {\em definition~(\ref{logic:def:LAM:free:variable})} is legitimate.
\end{prop}
\begin{proof}
TODO
\end{proof}

\begin{prop}\label{logic:prop:LAM:freevar:of:substitution:inclusion}
    Let $V$, $W$ be sets and $\sigma:V\to W$ be a map. Let
    $\phi\in\tv$\,:
    \[
        \free(\sigma(\phi))\subseteq\sigma(\free(\phi))
    \]
    where $\sigma:\tv\to{\bf T}(W)$ also denotes the associated substitution 
    mapping.
\end{prop}
\begin{proof}
TODO
\end{proof}

\begin{prop}\label{logic:prop:LAM:freevar:of:betasubst:gen}
Let $V$ be a set and $\sigma:V\to\tv$ be a map. Let $\phi\in\tv$.
Then for all $U\subseteq V$ we have:
    \[
        \free(\sigma^{*}(\phi)(U)) 
            \subseteq 
        (\free(\phi)\cap U)
        \ \cup\ 
        \!\!\!\!\!\!\!\!
        \bigcup_{x\in\free(\phi)\setminus U}
        \!\!\!\!\!\!
        \free(\sigma(x))
    \]
where $\sigma^{*}:\tv\to[{\cal P}(V)\to\tv]$ is defined as in 
definition~(\ref{logic:def:LAM:subst}).
\end{prop}
\begin{proof}
We shall prove this inclusion by structural induction on $\phi$, using
theorem~(\ref{logic:the:proof:induction}) of 
page~\pageref{logic:the:proof:induction}. So first we assume that $\phi=x$ 
for some $x\in V$. Then for all $U\subseteq V$, there are two possible cases: 
either $x\in U$ or $x\not\in U$. If $x\in U$ we have:
    \begin{eqnarray*}
        \free(\sigma^{*}(\phi)(U))
        &=&\free(\sigma^{*}(x)(U))\\
        &=&\free(\sigma_{U}(x))\\
        \mbox{$x\in U\ \rightarrow\ $}&=&\free(x)\\
         &=&\{x\}\\
         &=&\free(\phi)\\
        \mbox{$x\in U\ \rightarrow\ $}&=&\free(\phi)\cap U
    \end{eqnarray*}
In particular, the inclusion holds. If $x\not\in U$, then:
     \begin{eqnarray*}
        \free(\sigma^{*}(\phi)(U))
        &=&\free(\sigma^{*}(x)(U))\\
        &=&\free(\sigma_{U}(x))\\
        \mbox{$x\not\in U\ \rightarrow\ $}&=&\free(\sigma(x))\\
        &=&
        \!\!\!
        \bigcup_{u\in\{x\}} 
        \!
        \free(\sigma(u))\\
        &=&
        \!\!\!\!\!
        \bigcup_{u\in\free(\phi)}
        \!\!\!\!
        \free(\sigma(u))\\
        \mbox{$x\not\in U\ \rightarrow\ $}
        &=&\!\!\!\!\!\!\!\!
        \bigcup_{u\in\free(\phi)\setminus U}
        \!\!\!\!\!\!\!
        \free(\sigma(u))
    \end{eqnarray*}
Hence once again, the inclusion holds. We now assume that 
$\phi=\phi_{1}\ \phi_{2}$ where $\phi_{1}, \phi_{2}\in\tv$ satisfy the 
inclusion for all $U\subseteq V$. Then for all $U\subseteq V$ we have:
    \begin{eqnarray*}
        \free(\sigma^{*}(\phi)(U))
        &=&\free(\ \sigma^{*}(\phi_{1}\ \phi_{2})(U)\ )\\
        &=&\free(\ \sigma^{*}(\phi_{1})(U)\ \ \sigma^{*}(\phi_{2})(U)\ )\\
        &=&\free(\sigma^{*}(\phi_{1})(U))\ \cup\ \free(\sigma^{*}(\phi_{2})(U))\\
        &\subseteq&(\free(\phi_{1})\cap U)
        \ \cup\ 
        \!\!\!\!\!\!\!\!\!\!
        \bigcup_{x\in\free(\phi_{1})\setminus U}
        \!\!\!\!\!\!\!\!
        \free(\sigma(x))\\
        &\cup&(\free(\phi_{2})\cap U)
        \ \cup\ 
        \!\!\!\!\!\!\!\!\!\!
        \bigcup_{x\in\free(\phi_{2})\setminus U}
        \!\!\!\!\!\!\!\!
        \free(\sigma(x))\\
        &=&(\free(\phi_{1})\cup\free(\phi_{2}))\cap U
        \ \cup\ 
        \!\!\!\!\!\!\!\!\!\!\!\!\!\!\!\!\!\!\!\!\!
        \bigcup_{x\in(\free(\phi_{1})\cup\free(\phi_{2}))\setminus U}
        \!\!\!\!\!\!\!\!\!\!\!\!\!\!\!\!\!\!\!
        \free(\sigma(x))\\
        &=&(\free(\phi)\cap U)
        \ \cup\ 
        \!\!\!\!\!\!\!\!
        \bigcup_{x\in\free(\phi)\setminus U}
        \!\!\!\!\!\!
        \free(\sigma(x))
    \end{eqnarray*}
Finally, we assume that $\phi=\lambda x\phi_{1}$ where $\phi_{1}\in\tv$ 
satisfies the inclusion for all $U\subseteq V$. Then for all $U\subseteq V$ 
we have:
    \begin{eqnarray*}
        \free(\sigma^{*}(\phi)(U))
        &=&\free(\sigma^{*}(\lambda x\phi_{1})(U))\\
        &=&\free(\lambda x\sigma^{*}(\phi_{1})(U\cup\{x\}))\\
        &=&\free(\sigma^{*}(\phi_{1})(U\cup\{x\}))\setminus\{x\}\\
        &\subseteq& [\ \ (\free(\phi_{1})\cap(U\cup\{x\}))
            \ \cup
            \!\!\!\!\!\!\!\!\!\!\!\!\!\!\!\!
            \bigcup_{u\in\free(\phi_{1})\setminus(U\cup\{x\})}
            \!\!\!\!\!\!\!\!\!\!\!\!\!\!\!\!
            \free(\sigma(u))
             \ \ ]\ \setminus\ \{x\}\\
        &=&(\free(\phi_{1})\cap U\cap \{x\}^{c})
            \ \cup
            \!\!\!\!\!\!\!\!\!\!\!\!\!\!\!\!
            \bigcup_{u\in\free(\phi_{1})\cap U^{c}\cap\{x\}^{c}}
            \!\!\!\!\!\!\!\!\!\!\!\!\!\!\!\!
            \free(\sigma(u))\setminus\{x\}\\
        &=&(\free(\lambda x\phi_{1})\cap U)
            \ \cup 
            \!\!\!\!\!\!\!\!\!\!
            \bigcup_{u\in\free(\lambda x\phi_{1})\setminus U}
            \!\!\!\!\!\!\!\!\!\!
            \free(\sigma(u))\setminus\{x\}\\
        &=&(\free(\phi)\cap U)
            \ \cup\ 
            \!\!\!\!\!\!\!\!
            \bigcup_{u\in\free(\phi)\setminus U} 
            \!\!\!\!\!\!\!
            \free(\sigma(u))\setminus\{x\}\\
        &\subseteq&(\free(\phi)\cap U)
            \ \cup\ 
            \!\!\!\!\!\!\!\!
            \bigcup_{u\in\free(\phi)\setminus U}
            \!\!\!\!\!\!
            \free(\sigma(u))
    \end{eqnarray*}
\end{proof}

\begin{prop}\label{logic:prop:LAM:freevar:of:betasubst:inclusion}
Let $V$ be a set and $\sigma:V\to\tv$ be a map. Let $\phi\in\tv$:
    \[
    \free(\sigma(\phi))\subseteq
    \!\!\!\!\bigcup_{x\in\free(\phi)}\!\!\!\!
    \free(\sigma(x))
    \]
where $\sigma:\tv\to\tv$ also denotes the associated $\beta$-subsitution mapping.
\end{prop}
\begin{proof}
Since for all $\phi\in\tv$ we have $\sigma(\phi)=\sigma^{*}(\phi)(\emptyset)$, 
the inclusion is an immediate consequence of 
proposition~(\ref{logic:prop:LAM:freevar:of:betasubst:gen}).
\end{proof}

\begin{prop}\label{logic:prop:LAM:freevar:of:substitution}
    Let $V$, $W$ be sets and $\sigma:V\to W$ be a map. Let $\phi\in\tv$ 
    be such that $\sigma_{|\var(\phi)}$ is an injective map. Then, we have:
    \[
        \free(\sigma(\phi))=\sigma(\free(\phi))
    \]
    where $\sigma:\tv\to{\bf T}(W)$ also denotes the associated substitution 
    mapping.
\end{prop}
\begin{proof}
TODO
\end{proof}

\begin{prop}\label{logic:prop:LAM:freevar:single:subst}
    Let $V$ be a set, $\phi\in\tv$, $x,y\in V$ with $y\not\in\var(\phi)$. Then:
    \[
        \free(\phi[y/x])=
            \left\{\begin{array}{lcl}
                \free(\phi)\setminus\{x\}\cup\{y\}
                    &\mbox{\ if\ }&
                x\in\free(\phi)
                \\
                \free(\phi)
                    &\mbox{\ if\ }&
                x\not\in\free(\phi)
            \end{array}\right.
    \]
\end{prop}
\begin{proof}
TODO
\end{proof}

\begin{prop}\label{logic:prop:LAM:congruence:freevar}
Let $V$ be a set and $\equiv$ be the relation on \tv\ defined by:
    \[
    \phi\equiv\psi\ \Leftrightarrow\ \free(\phi)=\free(\psi)
    \]
for all $\phi,\psi\in\tv$. Then $\equiv$ is a congruence on \tv.
\end{prop}
\begin{proof}
TODO
\end{proof}



