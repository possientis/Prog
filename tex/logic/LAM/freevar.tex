\begin{defin}\label{logic:def:LAM:free:variable}
    Let $V$ be a set. The map $\free:\tv\to {\cal P}(V)$ defined by the
    following structural recursion is called {\em free variable mapping on \tv}:
    \begin{equation}\label{logic:eqn:LAM:free:var:recursion}
        \forall\phi\in\tv\ ,\ 
            \free(\phi) =
                \left\{\begin{array}{lcl}
                    \{x\}&\mbox{\ if\ }&\phi=x
                    \\
                    \free(\phi_{1})\cup\free(\phi_{2}) 
                        &\mbox{\ if\ }&
                    \phi=\phi_{1}\ \phi_{2}
                    \\
                    \free(\phi_{1})\setminus\{x\}
                        &\mbox{\ if\ }&
                    \phi=\lambda x\phi_{1}
                \end{array}\right.
    \end{equation}
    We say that $x\in V$ is a {\em free variable} of $\phi\in\tv$ 
    \ifand\ $x\in\free(\phi)$.
\end{defin}
\begin{prop}\label{logic:prop:LAM:free:variable}
    The structural recursion of 
    {\em definition~(\ref{logic:def:LAM:free:variable})} is legitimate.
\end{prop}
\begin{proof}
TODO
\end{proof}

\begin{prop}\label{logic:prop:LAM:freevar:of:substitution:inclusion}
    Let $V$, $W$ be sets and $\sigma:V\to W$ be a map. Let
    $\phi\in\tv$\,:
    \[
        \free(\sigma(\phi))\subseteq\sigma(\free(\phi))
    \]
    where $\sigma:\tv\to{\bf T}(W)$ also denotes the associated substitution 
    mapping.
\end{prop}
\begin{proof}
TODO
\end{proof}

