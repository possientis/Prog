\begin{defin}\label{logic:def:LAM:free:variable}
    Let $V$ be a set. The map $\free:\tv\to {\cal P}(V)$ defined by the
    following structural recursion is called {\em free variable mapping on \tv}:
    \begin{equation}\label{logic:eqn:LAM:free:var:recursion}
        \forall\phi\in\tv\ ,\ 
            \free(\phi) =
                \left\{\begin{array}{lcl}
                    \{x\}&\mbox{\ if\ }&\phi=x
                    \\
                    \free(\phi_{1})\cup\free(\phi_{2}) 
                        &\mbox{\ if\ }&
                    \phi=\phi_{1}\ \phi_{2}
                    \\
                    \free(\phi_{1})\setminus\{x\}
                        &\mbox{\ if\ }&
                    \phi=\lambda x\phi_{1}
                \end{array}\right.
    \end{equation}
    We say that $x\in V$ is a {\em free variable} of $\phi\in\tv$ 
    \ifand\ $x\in\free(\phi)$.
\end{defin}
\begin{prop}\label{logic:prop:LAM:free:variable}
    The structural recursion of 
    {\em definition~(\ref{logic:def:LAM:free:variable})} is legitimate.
\end{prop}
\begin{proof}
TODO
\end{proof}

\begin{prop}\label{logic:prop:LAM:freevar:of:substitution:inclusion}
    Let $V$, $W$ be sets and $\sigma:V\to W$ be a map. Let
    $\phi\in\tv$\,:
    \[
        \free(\sigma(\phi))\subseteq\sigma(\free(\phi))
    \]
    where $\sigma:\tv\to{\bf T}(W)$ also denotes the associated substitution 
    mapping.
\end{prop}
\begin{proof}
TODO
\end{proof}

\begin{prop}\label{logic:prop:LAM:freevar:of:betasubst:gen}
Let $V$ be a set and $\sigma:V\to\tv$ be a map. Let $\phi\in\tv$.
Then for all $U\subseteq V$ we have:
    \[
        \free(\sigma^{*}(\phi)(U)) 
            \subseteq 
        (\free(\phi)\cap U)
        \ \cup\ 
        \!\!\!\!\!\!\!\!
        \bigcup_{x\in\free(\phi)\setminus U}
        \!\!\!\!\!\!
        \free(\sigma(x))
    \]
where $\sigma^{*}:\tv\to[{\cal P}(V)\to\tv]$ is defined as in 
definition~(\ref{logic:def:LAM:subst}).
\end{prop}
\begin{proof}
We shall prove this inclusion by structural induction on $\phi$, using
theorem~(\ref{logic:the:proof:induction}) of 
page~\pageref{logic:the:proof:induction}. So first we assume that $\phi=x$ 
for some $x\in V$. Then for all $U\subseteq V$, there are two possible cases: 
either $x\in U$ or $x\not\in U$. If $x\in U$ we have:
    \begin{eqnarray*}
        \free(\sigma^{*}(\phi)(U))
        &=&\free(\sigma^{*}(x)(U))\\
        &=&\free(\sigma_{U}(x))\\
        \mbox{$x\in U\ \rightarrow\ $}&=&\free(x)\\
         &=&\{x\}\\
         &=&\free(\phi)\\
        \mbox{$x\in U\ \rightarrow\ $}&=&\free(\phi)\cap U
    \end{eqnarray*}
In particular, the inclusion holds. If $x\not\in U$, then:
     \begin{eqnarray*}
        \free(\sigma^{*}(\phi)(U))
        &=&\free(\sigma^{*}(x)(U))\\
        &=&\free(\sigma_{U}(x))\\
        \mbox{$x\not\in U\ \rightarrow\ $}&=&\free(\sigma(x))\\
        &=&
        \!\!\!
        \bigcup_{u\in\{x\}} 
        \!
        \free(\sigma(u))\\
        &=&
        \!\!\!\!\!
        \bigcup_{u\in\free(\phi)}
        \!\!\!\!
        \free(\sigma(u))\\
        \mbox{$x\not\in U\ \rightarrow\ $}
        &=&\!\!\!\!\!\!\!\!
        \bigcup_{u\in\free(\phi)\setminus U}
        \!\!\!\!\!\!\!
        \free(\sigma(u))
    \end{eqnarray*}
Hence once again, the inclusion holds. We now assume that 
$\phi=\phi_{1}\ \phi_{2}$ where $\phi_{1}, \phi_{2}\in\tv$ satisfy the 
inclusion for all $U\subseteq V$. Then for all $U\subseteq V$ we have:
    \begin{eqnarray*}
        \free(\sigma^{*}(\phi)(U))
        &=&\free(\ \sigma^{*}(\phi_{1}\ \phi_{2})(U)\ )\\
        &=&\free(\ \sigma^{*}(\phi_{1})(U)\ \ \sigma^{*}(\phi_{2})(U)\ )\\
        &=&\free(\sigma^{*}(\phi_{1})(U))\ \cup\ \free(\sigma^{*}(\phi_{2})(U))\\
        &\subseteq&(\free(\phi_{1})\cap U)
        \ \cup\ 
        \!\!\!\!\!\!\!\!\!\!
        \bigcup_{x\in\free(\phi_{1})\setminus U}
        \!\!\!\!\!\!\!\!
        \free(\sigma(x))\\
        &\cup&(\free(\phi_{2})\cap U)
        \ \cup\ 
        \!\!\!\!\!\!\!\!\!\!
        \bigcup_{x\in\free(\phi_{2})\setminus U}
        \!\!\!\!\!\!\!\!
        \free(\sigma(x))\\
        &=&(\free(\phi_{1})\cup\free(\phi_{2}))\cap U
        \ \cup\ 
        \!\!\!\!\!\!\!\!\!\!\!\!\!\!\!\!\!\!\!\!\!
        \bigcup_{x\in(\free(\phi_{1})\cup\free(\phi_{2}))\setminus U}
        \!\!\!\!\!\!\!\!\!\!\!\!\!\!\!\!\!\!\!
        \free(\sigma(x))\\
        &=&(\free(\phi)\cap U)
        \ \cup\ 
        \!\!\!\!\!\!\!\!
        \bigcup_{x\in\free(\phi)\setminus U}
        \!\!\!\!\!\!
        \free(\sigma(x))
    \end{eqnarray*}
Finally, we assume that $\phi=\lambda x\phi_{1}$ where $\phi_{1}\in\tv$ 
satisfies the inclusion for all $U\subseteq V$. Then for all $U\subseteq V$ 
we have:
    \begin{eqnarray*}
        \free(\sigma^{*}(\phi)(U))
        &=&\free(\sigma^{*}(\lambda x\phi_{1})(U))\\
        &=&\free(\lambda x\sigma^{*}(\phi_{1})(U\cup\{x\}))\\
        &=&\free(\sigma^{*}(\phi_{1})(U\cup\{x\}))\setminus\{x\}\\
        &\subseteq& [\ \ (\free(\phi_{1})\cap(U\cup\{x\}))
            \ \cup
            \!\!\!\!\!\!\!\!\!\!\!\!\!\!\!\!
            \bigcup_{u\in\free(\phi_{1})\setminus(U\cup\{x\})}
            \!\!\!\!\!\!\!\!\!\!\!\!\!\!\!\!
            \free(\sigma(u))
             \ \ ]\ \setminus\ \{x\}\\
        &=&(\free(\phi_{1})\cap U\cap \{x\}^{c})
            \ \cup
            \!\!\!\!\!\!\!\!\!\!\!\!\!\!\!\!
            \bigcup_{u\in\free(\phi_{1})\cap U^{c}\cap\{x\}^{c}}
            \!\!\!\!\!\!\!\!\!\!\!\!\!\!\!\!
            \free(\sigma(u))\setminus\{x\}\\
        &=&(\free(\lambda x\phi_{1})\cap U)
            \ \cup 
            \!\!\!\!\!\!\!\!\!\!
            \bigcup_{u\in\free(\lambda x\phi_{1})\setminus U}
            \!\!\!\!\!\!\!\!\!\!
            \free(\sigma(u))\setminus\{x\}\\
        &=&(\free(\phi)\cap U)
            \ \cup\ 
            \!\!\!\!\!\!\!\!
            \bigcup_{u\in\free(\phi)\setminus U} 
            \!\!\!\!\!\!\!
            \free(\sigma(u))\setminus\{x\}\\
        &\subseteq&(\free(\phi)\cap U)
            \ \cup\ 
            \!\!\!\!\!\!\!\!
            \bigcup_{u\in\free(\phi)\setminus U}
            \!\!\!\!\!\!
            \free(\sigma(u))
    \end{eqnarray*}
\end{proof}

\begin{prop}\label{logic:prop:LAM:freevar:of:betasubst:inclusion}
Let $V$ be a set and $\sigma:V\to\tv$ be a map. Let $\phi\in\tv$:
    \[
    \free(\sigma(\phi))\subseteq
    \!\!\!\!\bigcup_{x\in\free(\phi)}\!\!\!\!
    \free(\sigma(x))
    \]
where $\sigma:\tv\to\tv$ also denotes the associated $\beta$-subsitution mapping.
\end{prop}
\begin{proof}
Since for all $\phi\in\tv$ we have $\sigma(\phi)=\sigma^{*}(\phi)(\emptyset)$, 
the inclusion is an immediate consequence of 
proposition~(\ref{logic:prop:LAM:freevar:of:betasubst:gen}).
\end{proof}

\begin{prop}\label{logic:prop:LAM:freevar:beta:intersect:gen}
    Let $V$ be a set and $\sigma:V\to\tv$ be a map. Let $\phi\in\tv$.
    Then for all $U,U'\subseteq V$ we have the implication:
        \[
            U\cap\free(\phi) = U'\cap\free(\phi)
            \ \Rightarrow\ 
            \sigma^{*}(\phi)(U) = \sigma^{*}(\phi)(U')
        \]
    where $\sigma^{*}:\tv\to[{\cal P}(V)\to\tv]$ is defined as in 
    definition~(\ref{logic:def:LAM:subst}).
\end{prop}
\begin{proof}
    We shall prove this implication for all $U,U'\subseteq V$  by structural 
    induction on $\phi$, using theorem~(\ref{logic:the:proof:induction}) of 
    page~\pageref{logic:the:proof:induction}. So first we assume that $\phi=x$
    for some $x\in V$ and furthermore that $U\cap\free(\phi) = U'\cap\free(\phi)$.
    We need to show that $\sigma^{*}(\phi)(U)=\sigma^{*}(\phi)(U')$ which 
    reduces to $\sigma_{U}(x)=\sigma_{U'}(x)$ using the notations of
    definition~(\ref{logic:def:LAM:subst}). However, when $\phi=x$ our assumption
    yields the equality $U\cap\{x\}=U'\cap\{x\}$ and we shall distinguish two
    cases: first we assume that $U\cap\{x\}=\emptyset$. Then equally 
    $U'\cap\{x\}=\emptyset$ and we have $x\not\in U$ together with $x\not\in U'$,
    so the equality $\sigma_{U}(x)=\sigma_{U'}(x)$ reduces to 
    $\sigma(x)=\sigma(x)$. Second we assume that $U\cap\{x\}=\{x\}$. Then
    equally $U'\cap\{x\}=\{x\}$ and we have $x\in U$ together with $x\in U'$
    so the equality $\sigma_{U}(x)=\sigma_{U'}(x)$ reduces to $x = x$.
    This completes our base case. So we now assume that $\phi=\phi_{1}\ \phi_{2}$
    for some $\phi_{1}, \phi_{2}\in\tv$ for which the implication is true for
    all $U,U'\subseteq V$. We also assume that 
    $U\cap\free(\phi)=U'\cap\free(\phi)$
    and we need to show that $\sigma^{*}(\phi)(U)=\sigma^{*}(\phi)(U')$. Now
    before we proceed, we claim that both equalities 
    $U\cap\free(\phi_{1})=U'\cap\free(\phi_{1})$ and 
    $U\cap\free(\phi_{2})=U'\cap\free(\phi_{2})$ hold. Indeed, suppose
    $x\in U\cap\free(\phi_{1})$. Since 
    $\free(\phi)=\free(\phi_{1})\cup\free(\phi_{2})$, we have 
    $x\in U\cap\free(\phi)$ and consequently $x\in U'\cap\free(\phi)$.
    So $x$ is an element of $U'$ and it is also an element of $\free(\phi_{1})$
    by assumption, and we see that $x\in U'\cap\free(\phi_{1})$. Hence we have
    proved the inclusion $U\cap\free(\phi_{1})\subseteq U'\cap\free(\phi_{1})$
    and similarly we have $U'\cap\free(\phi_{1})\subseteq U\cap\free(\phi_{1})$
    yielding the equality $U\cap\free(\phi_{1})=U'\cap\free(\phi_{1})$.
    The other equality $U\cap\free(\phi_{2})=U'\cap\free(\phi_{2})$ follows
    similarly. Hence:
        \begin{eqnarray*}\sigma^{*}(\phi)(U)
            &=&\sigma^{*}(\phi_{1}\ \phi_{2})(U)\\
            &=&\sigma^{*}(\phi_{1})(U)\ \ \sigma^{*}(\phi_{2})(U)\\
            \mbox{induction hypothesis $\ \rightarrow\ $}
            &=&\sigma^{*}(\phi_{1})(U')\ \ \sigma^{*}(\phi_{2})(U')\\
            &=&\sigma^{*}(\phi_{1}\ \phi_{2})(U')\\
            &=&\sigma^{*}(\phi)(U')
        \end{eqnarray*}
    So we now assume that $\phi=\lambda x\phi_{1}$ for some $x\in V$ and 
    $\phi_{1}\in\tv$ for which the implication is true for all $U,U'\subseteq V$.
    We also assume that $U\cap\free(\phi)=U'\cap\free(\phi)$ and we need to 
    show that $\sigma^{*}(\phi)(U)=\sigma^{*}(\phi)(U')$. We have:
        \begin{eqnarray*}\sigma^{*}(\phi)(U)
            &=&\sigma^{*}(\lambda x\phi_{1})(U)\\
            &=&\lambda x\sigma^{*}(\phi_{1})(U\cup\{x\})\\
            \mbox{induction, see below\ $\rightarrow$\ }
            &=&\lambda x\sigma^{*}(\phi_{1})(U'\cup\{x\})\\
            &=&\sigma^{*}(\lambda x\phi_{1})(U')\\
            &=&\sigma^{*}(\phi)(U')
        \end{eqnarray*}
    It remains to justify why we are able to apply the induction hypothesis, i.e.
    we need to prove the equality $(U\cup\{x\})\cap\free(\phi_{1})=
    (U'\cup\{x\})\cap\free(\phi_{1})$. So:
        \begin{eqnarray*}(U\cup\{x\})\cap\free(\phi_{1})
            &=&(\,U\setminus\{x\}\,\cup\,\{x\}\,)\cap\free(\phi_{1})\\
            &=&U\cap\free(\phi_{1})\setminus\{x\}\,
                \cup\,\{x\}\cap\free(\phi_{1})\\
            &=&U\cap\free(\phi)\,\cup\,\{x\}\cap\free(\phi_{1})\\
            \mbox{assumption\ $\rightarrow$\ }
            &=&U'\cap\free(\phi)\,\cup\,\{x\}\cap\free(\phi_{1})\\
            &=&(U'\cup\{x\})\cap\free(\phi_{1})
        \end{eqnarray*}
\end{proof}

\begin{prop}\label{logic:prop:LAM:freevar:beta:intersect}
    Let $V$ be a set and $\sigma:V\to\tv$ be a map. Let $\phi\in\tv$.
    Then for all $U\subseteq V$ we have the implication:
        \[
            U\cap\free(\phi) = \emptyset
            \ \Rightarrow\ 
            \sigma^{*}(\phi)(U) = \sigma(\phi)
        \]
    where $\sigma^{*}:\tv\to[{\cal P}(V)\to\tv]$ is defined as in 
    definition~(\ref{logic:def:LAM:subst}).
\end{prop}
\begin{proof}
    This is an immediate application of 
    proposition~(\ref{logic:prop:LAM:freevar:beta:intersect:gen}) 
    to $U'=\emptyset$: Given the assumption $U\cap\free(\phi) = \emptyset$ 
    we obtain $U\cap\free(\phi)=U'\cap\free(\phi)$ and consequently:
        \[
            \sigma^{*}(\phi)(U)=\sigma^{*}(\phi)(U')
            =\sigma^{*}(\phi)(\emptyset)=\sigma(\phi)
        \]
\end{proof}

\begin{prop}\label{logic:prop:LAM:freevar:beta:support}
    Let $V$ be a set and $\sigma,\tau:V\to\tv$ be two maps. Then:
        \[
            \sigma_{|\free(\phi)\setminus U}=\tau_{|\free(\phi)\setminus U}
            \ \Leftrightarrow\ \sigma^{*}(\phi)(U)=\tau^{*}(\phi)(U)
        \]
    for all $U\subseteq V$ and $\phi\in\tv$, where $\sigma^{*},\tau^{*}$ are 
    defined as in definition~(\ref{logic:def:LAM:subst}).
\end{prop}
\begin{proof}
    Given $U\subseteq V$ and $\phi\in\tv$, we need to show that if $\sigma$
    and $\tau$ coincide on the set $\free(\phi)\setminus U$, then 
    $\sigma^{*}(\phi)(U)=\tau^{*}(\phi)(U)$, and conversely that if 
    we have $\sigma^{*}(\phi)(U)=\tau^{*}(\phi)(U)$ then $\sigma$ and $\tau$ 
    coincide on $\free(\phi)\setminus U$. We shall do so by structural
    induction using theorem~(\ref{logic:the:proof:induction}) of
    page~\pageref{logic:the:proof:induction}. So first we assume that 
    $\phi=x$ for some $x\in V$. Given $U\subseteq V$, we shall distinguish 
    two cases: first we assume that $x\in U$. Then $\free(\phi)\setminus U=
    \emptyset$ and $\sigma, \tau$ always coincide on $\free(\phi)\setminus U$.
    However, the condition $\sigma^{*}(\phi)(U)=\tau^{*}(\phi)(U)$ reduces
    to $\sigma_{U}(x)=\tau_{U}(x)$ which is $x = x$ since $x\in U$. So this
    condition is always true and we see that the equivalence holds. Next
    we assume that $x\not\in U$. Then $\free(\phi)\setminus U$ reduces to 
    $\{x\}$ and $\sigma,\tau$ coinciding on this set simply means that 
    $\sigma(x) = \tau(x)$. However, the condition $\sigma_{U}(x)=\tau_{U}(x)$
    also reduces to $\sigma(x)=\tau(x)$ since $x\not\in U$ and we see that 
    the equivalence again holds, which complete the base case of our 
    induction.

    So we now assume that $\phi=\phi_{1}\ \phi_{2}$ for some 
    $\phi_{1},\phi_{2}\in\tv$ for which the equivalence holds for all
    $U\subseteq V$. Given $U\subseteq V$, we need to show that the 
    equivalence holds for $\phi$. So first we assume that $\sigma$ and
    $\tau$ coincide on $\free(\phi)\setminus U$. Since $\free(\phi)=
    \free(\phi_{1})\cup\free(\phi_{2})$, $\sigma$ and $\tau$ coincide
    both on $\free(\phi_{1})\setminus U$ and $\free(\phi_{2})\setminus U$.
    Hence, using our induction hypothesis we obtain:
        \begin{eqnarray*}\sigma^{*}(\phi)(U)
            &=&\sigma^{*}(\phi_{1}\ \phi_{2})(U)\\
            &=&\sigma^{*}(\phi_{1})(U)\ \ \sigma^{*}(\phi_{2})(U)\\
            \mbox{induction hypothesis\ $\rightarrow$\ }
            &=&\tau^{*}(\phi_{1})(U)\ \ \tau^{*}(\phi_{2})(U)\\
            &=&\tau^{*}(\phi_{1}\ \phi_{2})(U)\\
            &=&\tau^{*}(\phi)(U)
        \end{eqnarray*}
    Conversely if $\sigma^{*}(\phi)(U)=\tau^{*}(\phi)(U)$ then we obtain
    the equality: 
        \[
            \sigma^{*}(\phi_{1})(U)\ \ \sigma^{*}(\phi_{2})(U)
            =
            \tau^{*}(\phi_{1})(U)\ \ \tau^{*}(\phi_{2})(U)
        \]
    Using theorem~(\ref{logic:the:unique:representation}) of
    page~\pageref{logic:the:unique:representation} it follows that
    $\sigma^{*}(\phi_{1})(U)=\tau^{*}(\phi_{1})(U)$ and
    $\sigma^{*}(\phi_{2})(U)=\tau^{*}(\phi_{2})(U)$. Having assumed 
    the equivalence is true for both $\phi_{1}$ and $\phi_{2}$ and
    for all $U\subseteq V$, we see that $\sigma$ and $\tau$ coincide 
    on the sets $\free(\phi_{1})\setminus U$ and $\free(\phi_{2})\setminus U$
    and consequently they coincide on $\free(\phi)\setminus U$.

    So we now assume that $\phi=\lambda x\phi_{1}$ for some $x\in V$ and
    $\phi_{1}\in\tv$ for which the equivalence holds for all $U\subseteq V$.
    Given $U\subseteq V$, we need to show that the equivalence holds for 
    $\phi$. So first we assume that $\sigma$ and $\tau$ coincide on
    $\free(\phi)\setminus U$. Since $\free(\phi)=\free(\phi_{1})\setminus\{x\}$,
    $\sigma$ and $\tau$ coincide on $\free(\phi_{1})\setminus (U\cup\{x\})$.
    Hence, using our induction hypothesis we obtain:
        \begin{eqnarray*}\sigma^{*}(\phi)(U)
            &=&\sigma^{*}(\lambda x\phi_{1})(U)\\
            &=&\lambda x\sigma^{*}(\phi_{1})(U\cup\{x\})\\
            \mbox{induction hypothesis\ $\rightarrow$\ }
            &=&\lambda x\tau^{*}(\phi_{1})(U\cup\{x\})\\
            &=&\tau^{*}(\lambda x \phi_{1})(U)\\
            &=&\tau^{*}(\phi)(U)
        \end{eqnarray*}
    Conversely if $\sigma^{*}(\phi)(U)=\tau^{*}(\phi)(U)$ then we obtain
    the equality:
        \[
            \lambda x\sigma^{*}(\phi_{1})(U\cup\{x\})
            =
            \lambda x\tau^{*}(\phi_{1})(U\cup\{x\})
        \]
    Using theorem~(\ref{logic:the:unique:representation}) of
    page~\pageref{logic:the:unique:representation} it follows that
    $\sigma^{*}(\phi_{1})(U\cup\{x\})=\tau^{*}(\phi_{1})(U\cup\{x\})$.
    Having assumed the equivalence is true for $\phi_{1}$ and for all 
    $U\subseteq V$, we conclude that $\sigma$ and $\tau$ coincide on
    $\free(\phi_{1})\setminus(U\cup\{x\})$, i.e. they coincide on
    $\free(\phi)\setminus U$.
\end{proof}

\begin{prop}\label{logic:prop:LAM:freevar:of:substitution}
    Let $V$, $W$ be sets and $\sigma:V\to W$ be a map. Let $\phi\in\tv$ 
    be such that $\sigma_{|\var(\phi)}$ is an injective map. Then, we have:
    \[
        \free(\sigma(\phi))=\sigma(\free(\phi))
    \]
    where $\sigma:\tv\to{\bf T}(W)$ also denotes the associated substitution 
    mapping.
\end{prop}
\begin{proof}
TODO
\end{proof}

\begin{prop}\label{logic:prop:LAM:freevar:single:subst}
    Let $V$ be a set, $\phi\in\tv$, $x,y\in V$ with $y\not\in\var(\phi)$. Then:
    \[
        \free(\phi[y/x])=
            \left\{\begin{array}{lcl}
                \free(\phi)\setminus\{x\}\cup\{y\}
                    &\mbox{\ if\ }&
                x\in\free(\phi)
                \\
                \free(\phi)
                    &\mbox{\ if\ }&
                x\not\in\free(\phi)
            \end{array}\right.
    \]
\end{prop}
\begin{proof}
TODO
\end{proof}

\begin{prop}\label{logic:prop:LAM:congruence:freevar}
Let $V$ be a set and $\equiv$ be the relation on \tv\ defined by:
    \[
    \phi\equiv\psi\ \Leftrightarrow\ \free(\phi)=\free(\psi)
    \]
for all $\phi,\psi\in\tv$. Then $\equiv$ is a congruence on \tv.
\end{prop}
\begin{proof}
TODO
\end{proof}



