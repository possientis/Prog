There are many reasons why minimal transforms are so useful. One of
them is their ability to characterize $\alpha$-equivalence with a
simple equality ${\cal M}(\phi)={\cal M}(\psi)$ as illustrated by
theorem~(\ref{logic:the:FOPL:mintransfsubcong:kernel}) of
page~\pageref{logic:the:FOPL:mintransfsubcong:kernel}. Another
reason is the fact that ${\cal M}(\phi)$ is essentially a copy of
$\phi$ as can be seen from the equivalence ${\cal M}(\phi)\sim
i(\phi)$ of
proposition~(\ref{logic:prop:FOPL:mintransform:eqivalence}).
However, this copy ${\cal M}(\phi)$ is a lot more convenient to
handle, which leads us to our third and most fundamental reason: the
formula ${\cal M}(\phi)$ can be acted upon by any map $\sigma:V\to
W$ without caveats on variable capture. This is the whole point of
minimal transforms. It can be expressed formally by saying that the
minimal extension $\bar{\sigma}:\bar{V}\to\bar{W}$ is always valid
for the formula ${\cal M}(\phi)$, a fact which was proved in
proposition~(\ref{logic:def:FOPL:commute:minextension:valid}). So
why is this important? It is crucially important as we are able to
consider the formula $\bar{\sigma}\circ{\cal M}(\phi)$ which is
meaningful as the substitution avoids variable capture. This formula
is essentially what we mean by '$\sigma(\phi)$', even when $\sigma$
is not valid for $\phi$. In fact whenever $\sigma$ is valid for
$\phi$, from
theorem~(\ref{logic:the:FOPL:commute:mintransform:validsub}) of
page~\pageref{logic:the:FOPL:commute:mintransform:validsub} we have
the equality $\bar{\sigma}\circ{\cal M}(\phi)={\cal
M}\circ\sigma(\phi)$ which shows that our formula
$\bar{\sigma}\circ{\cal M}(\phi)$ essentially coincides with what we
intended to describe, namely $\sigma(\phi)$. So minimal transforms
allow us to give meaning to $\sigma(\phi)$ despite the fact $\sigma$
may not be capture-avoiding. Minimal transforms are the key to
essential substitutions.

We are now hoping to replicate these same ideas from formulas to
proofs. We have extended the notion of minimal transform to proofs
and established some of its important properties, among which is the
substitution equivalence $\vals\circ{\cal M}(\pi)\sim{\cal
M}\circ\vals(\pi)$ of
proposition~(\ref{logic:prop:FUAP:mintransproof:valuation:commute})
which is true for clean proofs. We shall now prove that the minimal
extension $\bar{\sigma}:\bar{V}\to\bar{W}$ is always valid for the
minimal transform ${\cal M}(\pi)$, followed by the commutation
property $\bar{\sigma}\circ{\cal M}(\pi)={\cal M}\circ\sigma(\pi)$
whenever $\sigma$ is valid for $\pi$. Note that these results are
true without qualification on $\pi$ being a clean proof. They are
purely syntactic and do not rely on the semantics
$\vals:\pvs\to\pv$. In fact, these results are immediate replication
of what was done for formulas. Their proofs are a case of
'cut-and-paste' which is somewhat embarrassing and symptomatic of
poor design. We know this should not be done, in computing or
mathematics alike. This is what abstraction and generalization are
meant to achieve: to avoid the duplication of code. On the positive
side, our duplicated approach has allowed us to offer plenty of
motivational background, making the material a lot more accessible.
Furthermore, when we do attempt to abstract and generalize on the
next occasion, we shall have a far greater insight of what needs to
be done. the following proposition is the counterpart of
proposition~(\ref{logic:def:FOPL:commute:minextension:valid})\,:

\index{minimal@Minimal extension valid}
\begin{prop}\label{logic:prop:FUAP:mintransformproof:minextension:valid}
Let $\sigma:V\to W$ be a map. Then for all $\pi\in\pvs$ the minimal
extension $\bar{\sigma}:\bar{V}\to\bar{W}$ is valid for the minimal
transform ${\cal M}(\pi)$.
\end{prop}
\begin{proof}
We need to check the three properties of
proposition~(\ref{logic:prop:FUAP:validsubproof:minimalextension})
are met in relation to $\bar{\sigma}:\bar{V}\to\bar{W}$ and ${\cal
M}(\pi)$ with $V_{0}=\N$. First we show property $(i)$\,: we need to
show that $\bound({\cal M}(\pi))\subseteq\N$ which follows from
proposition~(\ref{logic:prop:FUAP:mintransformproof:boundvar}). Next
we show property~$(ii)$\,: we need to show that $\bar{\sigma}_{|\N}$
is injective which is clear from
definition~(\ref{logic:def:FOPL:commute:minextensioon:map}). We
finally show property~$(iii)$\,: we need to show that
$\bar{\sigma}(\N)\cap\bar{\sigma}(\var({\cal
M}(\pi))\setminus\N)=\emptyset$. This follows from
$\bar{\sigma}(\N)\subseteq\N$ and $\bar{\sigma}(\var({\cal
M}(\pi))\setminus\N)\subseteq\bar{\sigma}(V)=\sigma(V)\subseteq W$,
while $W\cap\N = \emptyset$.
\end{proof}

Our next result is
theorem~(\ref{logic:the:FUAP:mintransvalidsub:commute}) below,
showing the commutation property $\bar{\sigma}\circ{\cal
M}(\pi)={\cal M}\circ\sigma(\pi)$ whenever $\sigma$ is valid for
$\pi$. However before we deal with this theorem, we shall need to
prove a couple of lemmas. The following is the counterpart of
lemma~(\ref{logic:lemma:FOPL:commute:mphi1}) which was established
for formulas:

\begin{lemma}\label{logic:lemma:FUAP:mintransvalidsub:mpi1}
Let $V,W$ be sets and $\sigma:V\to W$ be a map. Let $\pi=\gen
x\pi_{1}$ where $\pi_{1}\in\pvs$ and $x\in V$ such that
$\sigma(x)\not\in\sigma(\free(\pi))$. Then for all $n\in\N$ we have:
    \[
        \bar{\sigma}\circ [n/x]\circ{\cal
    M}(\pi_{1})= [n/\sigma(x)]\circ\bar{\sigma}\circ{\cal M}(\pi_{1})
    \]
where $\bar{\sigma}:\bar{V}\to\bar{W}$ is the minimal extension of
$\sigma: V\to W$.
\end{lemma}
\begin{proof}
Using proposition~(\ref{logic:prop:FUAP:variable:support}), we only
need to show that the mappings $\bar{\sigma}\circ [n/x]$ and
$[n/\sigma(x)]\circ\bar{\sigma}$ coincide on $\var({\cal
M}(\pi_{1}))$. So let $u\in\var({\cal M}(\pi_{1}))\subseteq\bar{V}$.
We want:
    \begin{equation}\label{logic:eqn:FUAP:mintransvalidsub:mpi1:1}
    \bar{\sigma}\circ [n/x](u)=[n/\sigma(x)]\circ\bar{\sigma}(u)
    \end{equation}
Since $\bar{V}$ is the disjoint union of $V$ and \N, we shall
distinguish two cases: first we assume that $u\in\N$. From
$V\cap\N=\emptyset$ we obtain $u\neq x$ and consequently
$\bar{\sigma}\circ[n/x](u)=\bar{\sigma}(u)=u$. From
$W\cap\N=\emptyset$ we obtain $u\neq\sigma(x)$ and consequently
$[n/\sigma(x)]\circ\bar{\sigma}(u)=[n/\sigma(x)](u)=u$. So
equation~(\ref{logic:eqn:FUAP:mintransvalidsub:mpi1:1}) is indeed
satisfied. Next we assume that $u\in V$. We shall distinguish two
further cases: first we assume that $u=x$. Then
$\bar{\sigma}\circ[n/x](u)=\bar{\sigma}(n)=n$ and
$[n/\sigma(x)]\circ\bar{\sigma}(u)=[n/\sigma(x)](\sigma(x))=n$ and
we see that equation~(\ref{logic:eqn:FUAP:mintransvalidsub:mpi1:1})
is again satisfied. Next we assume that $u\neq x$. Then
$\bar{\sigma}\circ[n/x](u)=\bar{\sigma}(u)=\sigma(u)$, and
furthermore
$[n/\sigma(x)]\circ\bar{\sigma}(u)=[n/\sigma(x)](\sigma(u))$. In
order to establish
equation~(\ref{logic:eqn:FUAP:mintransvalidsub:mpi1:1}), it is
therefore sufficient to prove that $\sigma(u)\neq\sigma(x)$.
However, since $u\in\var({\cal M}(\pi_{1}))$ and $u\in V$, it
follows from
proposition~(\ref{logic:prop:FUAP:mintransformproof:freevar}) that
$u\in\free(\pi_{1})$. Having assumed that $u\neq x$ we in fact have
$u\in\free(\gen x\pi_{1})=\free(\pi)$. Having assumed that
$\sigma(x)\not\in\sigma(\free(\pi))$ it follows that
$\sigma(u)\neq\sigma(x)$ as requested.
\end{proof}

The following lemma is the counterpart of
lemma~(\ref{logic:lemma:FOPL:commute:n:equivalence})\,:

\begin{lemma}\label{logic:lemma:FUAP:mintransvalidsub:n:equivalence}
Let $V,W$ be sets and $\sigma:V\to W$ be a map. Let $\pi=\gen
x\pi_{1}$ where $\pi_{1}\in\pvs$ and $x\in V$ such that
$\sigma(x)\not\in\sigma(\free(\pi))$. Then for all $k\in\N$ we have:
    \[
    \mbox{$[k/x]$ valid for ${\cal M}(\pi_{1})$}\ \Leftrightarrow
    \ \mbox{$[k/\sigma(x)]$ valid for $\bar{\sigma}\circ{\cal M}(\pi_{1})$}
    \]
where $\bar{\sigma}:\bar{V}\to\bar{W}$ is the minimal extension of
$\sigma: V\to W$.
\end{lemma}
\begin{proof}
First we show $\Leftarrow$\,: So we assume that $[k/\sigma(x)]$ is
valid for $\bar{\sigma}\circ{\cal M}(\pi_{1})$. We need to show that
$[k/x]$ is valid for ${\cal M}(\pi_{1})$. However, we know from
proposition~(\ref{logic:prop:FUAP:mintransformproof:minextension:valid})
that $\bar{\sigma}$ is valid for ${\cal M}(\pi_{1})$. Hence, from
the validity of $[k/\sigma(x)]$ for $\bar{\sigma}\circ{\cal
M}(\pi_{1})$ and
proposition~(\ref{logic:prop:FUAP:validsubproof:composition}) we see
that $[k/\sigma(x)]\circ\bar{\sigma}$ is valid for ${\cal
M}(\pi_{1})$. Furthermore, having assumed
$\sigma(x)\not\in\sigma(\free(\pi))$, we can use
lemma~(\ref{logic:lemma:FUAP:mintransvalidsub:mpi1}) to obtain:
    \begin{equation}\label{logic:eqn:FUAP:mintransvalidsub:n:equiv:1}
    \bar{\sigma}\circ [k/x]\circ{\cal
    M}(\pi_{1})= [k/\sigma(x)]\circ\bar{\sigma}\circ{\cal M}(\pi_{1})
    \end{equation}
It follows from
proposition~(\ref{logic:prop:FUAP:validsubproof:equal:image}) that
$\bar{\sigma}\circ[k/x]$ is valid for ${\cal M}(\pi_{1})$, and in
particular, using
proposition~(\ref{logic:prop:FUAP:validsubproof:composition}) once
more, we conclude that $[k/x]$ is valid for ${\cal M}(\pi_{1})$ as
requested. We now show $\Rightarrow$\,: so we assume that $[k/x]$ is
valid for ${\cal M}(\pi_{1})$. We need to show that $[k/\sigma(x)]$
is valid for $\bar{\sigma}\circ{\cal M}(\pi_{1})$. However, from
proposition~(\ref{logic:prop:FUAP:validsubproof:composition}) it is
sufficient to prove that $[k/\sigma(x)]\circ\bar{\sigma}$ is valid
for ${\cal M}(\pi_{1})$. Using
equation~(\ref{logic:eqn:FUAP:mintransvalidsub:n:equiv:1}) and
proposition~(\ref{logic:prop:FUAP:validsubproof:equal:image}) once
more, we simply need to show that $\bar{\sigma}\circ[k/x]$ is valid
for ${\cal M}(\pi_{1})$. Having assumed that $[k/x]$ is valid for
${\cal M}(\pi_{1})$, from
proposition~(\ref{logic:prop:FUAP:validsubproof:composition}) it is
sufficient to show that $\bar{\sigma}$ is valid for $[k/x]\circ{\cal
M}(\pi_{1})$. We shall do so with an application of
proposition~(\ref{logic:prop:FUAP:validsubproof:minimalextension})
to $V_{0}=\N$. First we need to check that $\bound(\,[k/x]\circ{\cal
M}(\pi_{1})\,)\subseteq\N$. However, from
proposition~(\ref{logic:prop:FUAP:boundvarproof:substitution}) we
have $\bound(\,[k/x]\circ{\cal M}(\pi_{1})\,) = [k/x](\,\bound({\cal
M}(\pi_{1}))\,)$ and from
proposition~(\ref{logic:prop:FUAP:mintransformproof:boundvar}) we
have $\bound({\cal M}(\pi_{1}))\subseteq\N$. So the desired
inclusion follows. Next we need to show that $\bar{\sigma}_{|\N}$ is
injective which is clear from
definition~(\ref{logic:def:FOPL:commute:minextensioon:map}).
Finally, we need to check that
$\bar{\sigma}(\N)\cap\bar{\sigma}(\,\var([k/x]\circ{\cal
M}(\pi_{1}))\setminus\N\,)=\emptyset$. This follows from
$\bar{\sigma}(\N)\subseteq\N$ and
$\bar{\sigma}(\,\var([k/x]\circ{\cal
M}(\pi_{1}))\setminus\N\,)\subseteq\bar{\sigma}(V)=\sigma(V)\subseteq
W$, while $W\cap\N = \emptyset$.
\end{proof}

We are now ready to prove
theorem~(\ref{logic:the:FUAP:mintransvalidsub:commute}) which his
the counterpart of
theorem~(\ref{logic:the:FOPL:commute:mintransform:validsub}) of
page~\pageref{logic:the:FOPL:commute:mintransform:validsub}. The
equality $\bar{\sigma}\circ{\cal M}(\pi)={\cal M}\circ\sigma(\pi)$
for $\sigma$ valid for $\pi$ will turn out to be very useful, just
as it was for formulas. More fundamentally, this equality vindicates
our belief that $\bar{\sigma}\circ{\cal M}(\pi)$ is {\em
essentially} the right definition for '$\sigma(\pi)$' when $\sigma$
is not valid for $\pi$, since the two coincide in the valid case.

\index{minimal@Minimal transform commutes}
\begin{theorem}\label{logic:the:FUAP:mintransvalidsub:commute}
Let $V$ and $W$ be sets. Let $\sigma:V\to W$ be a map. Let
$\pi\in\pvs$. If $\sigma$ is valid for $\pi$, then it commutes with
minimal transforms, specifically:
    \begin{equation}\label{logic:eqn:FUAP:mintransvalidsub:commute:1}
    \bar{\sigma}\circ{\cal M}(\pi)={\cal M}\circ\sigma(\pi)
    \end{equation}
where $\bar{\sigma}:\bar{V}\to\bar{W}$ is the minimal extension of
$\sigma: V\to W$.
\end{theorem}
\begin{proof}
Before we start, it should be noted that $\sigma$ in
equation~(\ref{logic:eqn:FUAP:mintransvalidsub:commute:1}) refers to
the substitution mapping $\sigma:\pvs\to{\bf\Pi}(W)$ while the
${\cal M}$ on the right-hand-side refers to the minimal transform
mapping ${\cal M}:{\bf\Pi}(W)\to{\bf\Pi}(\bar{W})$. The other ${\cal
M}$ which appears on the left-hand-side refers to the minimal
transform mapping ${\cal M}:\pvs\to{\bf\Pi}(\bar{V})$, and
$\bar{\sigma}$ is the substitution mapping $\bar{\sigma}:{\bf
\Pi}(\bar{V})\to{\bf\Pi}(\bar{W})$. So everything makes sense. For
all $\pi\in\pvs$, we need to show the property:
    \[
    (\mbox{$\sigma$ valid for $\pi$})\ \Rightarrow\
    \bar{\sigma}\circ{\cal M}(\pi)={\cal M}\circ\sigma(\pi)
    \]
We shall do so by structural induction using
theorem~(\ref{logic:the:proof:induction}) of
page~\pageref{logic:the:proof:induction}. First we assume that
$\pi=\phi$ for some $\phi\in\pv$. We need to show the property is
true for $\pi$. So we assume that $\sigma$ is valid for $\pi=\phi$.
We need to show the equality $\bar{\sigma}\circ{\cal M}(\pi)={\cal
M}\circ\sigma(\pi)$ which follows immediately from
theorem~(\ref{logic:the:FOPL:commute:mintransform:validsub}) of
page~\pageref{logic:the:FOPL:commute:mintransform:validsub}. Next we
assume that $\pi=\axi\phi$ for some $\phi\in\pv$. We need to show
that the property is true for $\pi$. So we assume that $\sigma$ is
valid for $\pi$. Using
proposition~(\ref{logic:prop:FUAP:validsubproof:recursion:axiom})
this means that $\sigma$ is valid for $\phi$. Hence we have:
    \begin{eqnarray*}
    \bar{\sigma}\circ{\cal M}(\pi)&=&\bar{\sigma}\circ{\cal
    M}(\axi\phi)\\
    &=&\bar{\sigma}(\axi{\cal M}(\phi))\\
    &=&\axi\bar{\sigma}\circ{\cal M}(\phi)\\
    \mbox{Th.~(\ref{logic:the:FOPL:commute:mintransform:validsub}), $\sigma$ valid for $\phi$}\ \rightarrow
    &=&\axi{\cal M}\circ\sigma(\phi)\\
    &=&{\cal M}(\axi\sigma(\phi))\\
    &=&{\cal M}\circ\sigma(\axi\phi)\\
    &=&{\cal M}\circ\sigma(\pi)\\
    \end{eqnarray*}
Next we assume that $\pi=\pi_{1}\pon\pi_{2}$ where the property is
true for $\pi_{1},\pi_{2}\in\pvs$. We need to show the property is
also true for $\pi$. So we assume that $\sigma$ is valid for $\pi$.
We need to show
equation~(\ref{logic:eqn:FUAP:mintransvalidsub:commute:1}) holds for
$\pi$. However, from
proposition~(\ref{logic:prop:FUAP:validsubproof:recursion:pon}), the
substitution $\sigma$ is valid for both $\pi_{1}$ and $\pi_{2}$.
Having assumed the property is true for $\pi_{1}$ and $\pi_{2}$, it
follows that
equation~(\ref{logic:eqn:FUAP:mintransvalidsub:commute:1}) holds for
$\pi_{1}$ and $\pi_{2}$. Hence, we have the following equalities:
    \begin{eqnarray*}
    \bar{\sigma}\circ{\cal M}(\pi)
    &=&\bar{\sigma}\circ{\cal M}(\pi_{1}\pon\pi_{2})\\
    &=&\bar{\sigma}(\,{\cal M}(\pi_{1})\pon\,{\cal M}(\pi_{2})\,)\\
    &=&\bar{\sigma}({\cal M}(\pi_{1}))\pon\,\bar{\sigma}({\cal
    M}(\pi_{2}))\\
    &=&{\cal M}(\sigma(\pi_{1}))\pon\,{\cal
    M}(\sigma(\pi_{2}))\\
    &=&{\cal M}(\,\sigma(\pi_{1})\pon\,\sigma(\pi_{2})\,)\\
    &=&{\cal M}(\sigma(\pi_{1}\pon\pi_{2}))\\
    &=&{\cal M}\circ\sigma(\pi)
    \end{eqnarray*}
Finally we assume that $\pi=\gen x\pi_{1}$ where $x\in V$ and the
property is true for $\pi_{1}\in\pvs$. We need to show the property
is also true for $\pi$. So we assume that $\sigma$ is valid for
$\pi$. We need to show
equation~(\ref{logic:eqn:FUAP:mintransvalidsub:commute:1}) holds for
$\pi$. However, from
proposition~(\ref{logic:prop:FUAP:validsubproof:recursion:gen}), the
substitution $\sigma$ is also valid for $\pi_{1}$. Having assumed
the property is true for $\pi_{1}$, it follows that
equation~(\ref{logic:eqn:FUAP:mintransvalidsub:commute:1}) holds for
$\pi_{1}$. Hence:
    \begin{eqnarray*}
    \bar{\sigma}\circ{\cal M}(\pi)&=&\bar{\sigma}\circ{\cal M}(\gen
    x\pi_{1})\\
    \mbox{$n=\min\{k:[k/x]\mbox{ valid for }{\cal M}(\pi_{1})\}$}\
    \rightarrow
    &=&\bar{\sigma}(\,\gen n{\cal M}(\pi_{1})[n/x]\,)\\
    &=&\gen \bar{\sigma}(n)\bar{\sigma}(\,{\cal
    M}(\pi_{1})[n/x]\,)\\
    &=&\gen n\,\bar{\sigma}\circ [n/x]\circ{\cal
    M}(\pi_{1})\\
    \mbox{A: to be proved}\ \rightarrow
    &=&\gen n\,[n/\sigma(x)]\circ\bar{\sigma}\circ{\cal M}(\pi_{1})\\
    &=&\gen n\,[n/\sigma(x)]\circ{\cal M}\circ\sigma(\pi_{1})\\
    &=&\gen n\,{\cal M}[\sigma(\pi_{1})][n/\sigma(x)]\\
    \mbox{B: to be proved}\ \rightarrow
    &=&\gen m\,{\cal M}[\sigma(\pi_{1})][m/\sigma(x)]\\
    \mbox{$m=\min\{k:[k/\sigma(x)]\mbox{ valid for }{\cal M}[\sigma(\pi_{1})]\}$}\
    \rightarrow
    &=&{\cal M}(\gen\sigma(x)\sigma(\pi_{1}))\\
    &=&{\cal M}(\sigma(\gen x\pi_{1}))\\
    &=&{\cal M}\circ\sigma(\pi)
    \end{eqnarray*}
So we have two more points A and B to justify. First we deal with
point A. It is sufficient for us to prove the following equality:
    \[
    \bar{\sigma}\circ [n/x]\circ{\cal
    M}(\pi_{1})= [n/\sigma(x)]\circ\bar{\sigma}\circ{\cal M}(\pi_{1})
    \]
Using lemma~(\ref{logic:lemma:FUAP:mintransvalidsub:mpi1}) it is
sufficient to show that $\sigma(x)\not\in\sigma(\free(\pi))$. So
suppose to the contrary that $\sigma(x)\in\sigma(\free(\pi))$. Then
there exists $u\in\free(\pi)$ such that $\sigma(u)=\sigma(x)$. This
contradicts
proposition~(\ref{logic:prop:FUAP:validsubproof:recursion:gen}) and
the fact that $\sigma$ is valid for $\pi$, which completes the proof
of point A. So we now turn to point B. We need to prove that $n=m$,
for which it is sufficient to show the equivalence:
    \[
    \mbox{$[k/x]$ valid for ${\cal M}(\pi_{1})$}\ \Leftrightarrow
    \ \mbox{$[k/\sigma(x)]$ valid for ${\cal M}[\sigma(\pi_{1})]$}
    \]
This follows from
lemma~(\ref{logic:lemma:FUAP:mintransvalidsub:n:equivalence}) and
the fact that  $\sigma(x)\not\in\sigma(\free(\pi))$, together with
the induction hypothesis $\bar{\sigma}\circ{\cal M}(\pi_{1})={\cal
M}\circ\sigma(\pi_{1})$.
\end{proof}
