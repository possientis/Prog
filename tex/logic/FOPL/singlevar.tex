In definition~(\ref{logic:def:substitution}) we introduced the
notion of substitution mapping associated with a given map. Our
motivation was to formalize the idea that $\forall x(x\in z)$ and
$\forall y(y\in z)$ were identical mathematical statements when
$z\not\in\{x,y\}$, or equivalently that replacing the variable $x$
by the variable $y$ in $\forall x(x\in z)$ did not change the {\em
meaning} of the formula. We are now in a position to formally define
the {\em substitution} of a variable by another.
\index{substitution@$[y/x]$ : $y$ in place of
$x$}\index{substitution@Substitution of single
variable}\index{substitution@$\phi[y/x]$ : image of $\phi$ by
$[y/x]$}
\begin{defin}\label{logic:def:single:var:substitution}
Let $V$ be a set and $x,y\in V$. We call {\em substitution of $y$ in
place of $x$}, the map $[y/x]:V\to V$ defined by:
    \[
    \forall u\in V\ ,\ [y/x](u)=\left\{
    \begin{array}{lcl}
    y&\mbox{\ if\ }&u=x\\
    u&\mbox{\ if\ }&u\neq x
    \end{array}
    \right.
    \]
If we denote $[y/x]:\pv\to\pv$ the associated substitution mapping,
given $\phi\in\pv$ the image $[y/x](\phi)$ is called {\em $\phi$
with $y$ in place of $x$} and denoted $\phi[y/x]$.
\end{defin}

The {\em substitution of $y$ in place of $x$} has a fundamental
flaw: it is not an injective map. In most introductory texts on
mathematical logic, this is hardly a problem as the set of variables
$V$ is always assumed to be infinite. So if we wish to argue that
$\forall x\forall y(x\in y)$ is the same mathematical statement as
$\forall y\forall x(y\in x)$, we can always choose a variable
$z\not\in\{x,y\}$ and prove the equivalence as:
    \begin{eqnarray*}
    \forall x\forall y(x\in y)&\sim&\forall z\forall y(z\in y)\\
    &\sim&\forall z\forall x(z\in x)\\
    &\sim&\forall y\forall x(y\in x)
    \end{eqnarray*}
Unfortunately, this will not be possible in the case when
$V=\{x,y\}$ and $x\neq y$, i.e. when $V$ has only two elements. As
we shall see in later parts of this document, a congruence defined
on \pv\ in terms of the substitution mapping $[y/x]$ will lead to
the paradoxical situation of $\forall x\forall y(x\in
y)\not\sim\forall y\forall x(y\in x)$ when $V=\{x,y\}$. To remedy
this fact, we shall introduce the {\em permutation} mapping
$[y\!:\!x]$ and use this as the basis of a new congruence which will
turn out to be simpler, equivalent to the traditional notion based
on $[y/x]$ for $V$ infinite, and working equally well for $V$ finite
or infinite. \index{substitution@$[y:x]$ : permutation of $x$ and
$y$}
\begin{defin}\label{logic:def:single:var:permutation}
Let $V$ be a set and $x,y\in V$. We call {\em permutation of $x$ and
$y$}, the map $[y\!:\!x]:V\to V$ defined by:
    \[
    \forall u\in V\ ,\ [y\!:\!x](u)=\left\{
    \begin{array}{lcl}
    y&\mbox{\ if\ }&u=x\\
    x&\mbox{\ if\ }&u=y\\
    u&\mbox{\ if\ }&u\not\in\{x,y\}
    \end{array}
    \right.
    \]
If we denote $[y\!:\!x]:\pv\to\pv$ the associated substitution
mapping, given $\phi\in\pv$ the image $[y\!:\!x](\phi)$ is denoted
$\phi[y\!:\!x]$.
\end{defin}

An immediate consequence of this definition is:
\begin{prop}\label{logic:prop:FUAP:singlevar:composition:injective}
Let $V$, $W$ be sets and $\sigma:V\to W$ be an injective map. Then
for all $x,y\in V$ we have the following equality:
    \[
    \sigma\circ[y\!:\!x]=[\sigma(y)\!:\!\sigma(x)]\circ\sigma
    \]
\end{prop}
\begin{proof}
Let $u\in V$. We shall distinguish three cases. First we assume
$u\not\in\{x,y\}$. Then:
    \[
    \sigma\circ[y\!:\!x](u)=\sigma(u)=[\sigma(y)\!:\!\sigma(x)]\circ\sigma(u)
    \]
where the last equality crucially depends on
$\sigma(u)\not\in\{\sigma(x),\sigma(y)\}$ which itself follows from
the injectivity of $\sigma$ and $u\not\in\{x,y\}$. We now assume
that $u=x$:
\begin{eqnarray*}
    \sigma\circ[y\!:\!x](u)&=&\sigma(y)\\
    &=&[\sigma(y)\!:\!\sigma(x)]\circ\sigma(x)\\
    &=&[\sigma(y)\!:\!\sigma(x)]\circ\sigma(u)
    \end{eqnarray*}
Finally, we assume that $u=y$ and obtain:
    \begin{eqnarray*}
    \sigma\circ[y\!:\!x](u)&=&\sigma(x)\\
    &=&[\sigma(y)\!:\!\sigma(x)]\circ\sigma(y)\\
    &=&[\sigma(y)\!:\!\sigma(x)]\circ\sigma(u)
    \end{eqnarray*}
In all cases we see that
$\sigma\circ[y\!:\!x](u)=[\sigma(y)\!:\!\sigma(x)]\circ\sigma(u)$ as
requested.
\end{proof}

As already noted, the permutation $[y\!:\!x]$ is injective while the
substitution $[y/x]$ is not. However, there will be cases when the
formulas $\phi[y/x]$ and $\phi[y\!:\!x]$ coincide, given a formula
$\phi$. Knowing when the equality holds is important and one simple
sufficient condition is $y\not\in\var(\phi)$, as will be seen from
the following:
\begin{prop}\label{logic:prop:FOPL:singlevar:support}
Let $V$ be a set and $U\subseteq V$. Then for all $x,y\in V$ we
have:
    \[
    y\not\in U\ \Rightarrow\ [y/x]_{|U}=[y\!:\!x]_{|U}
    \]
\end{prop}
\begin{proof}
We assume that $y\not\in U$. Let $u\in U$. We need to show that
$[y/x](u)=[y\!:\!x](u)$. We shall distinguish three cases: first we
assume that $u\not\in\{x,y\}$. Then the equality is clear. Next we
assume that $u=x$. Then the equality is also clear. Finally we
assume that $u=y$. In fact, this cannot occur since $y\not\in U$ and
$u\in U$.
\end{proof}
\begin{prop}\label{logic:prop:permutation:is:substitution}
Let $V$ be a set and $x,y\in V$. Let $\phi\in\pv$. Then, we have:
    \[
    y\not\in\var(\phi)\ \Rightarrow\ \phi[y/x]=\phi[y\!:\!x]
    \]
\end{prop}
\begin{proof}
We assume that $y\not\in\var(\phi)$. We need to show that
$\phi[y/x]=\phi[y\!:\!x]$. From
proposition~(\ref{logic:prop:substitution:support}) it is sufficient
to show that $[y/x]$ and $[y\!:\!x]$ coincide on $\var(\phi)$, which
follows immediately from $y\not\in\var(\phi)$ and
proposition~(\ref{logic:prop:FOPL:singlevar:support}).
\end{proof}

When replacing a variable $x$ by a variable $y$ in a formula $\phi$,
and subsequently replacing the variable $y$ by a variable $z$, we
would expect the outcome to be the same as replacing the variable
$x$ by the variable $z$ directly. This is in fact the case, provided
$y$ is not already a variable of $\phi$. Assuming $x$, $y$ and $z$
are distinct variables, a simple counterexample is $\phi=(x\in y)$.
\begin{prop}\label{logic:prop:single:composition}
Let $V$ be a set and $x,y,z\in V$. Then for all $\phi\in\pv$:
    \[
    y\not\in\var(\phi)\ \Rightarrow\ \phi[y/x][z/y]=\phi[z/x]
    \]
\end{prop}
\begin{proof}
Suppose $y\not\in\var(\phi)$. We need to show that
$[z/y]\circ[y/x](\phi)=[z/x](\phi)$. From
proposition~(\ref{logic:prop:substitution:support}) it is sufficient
to prove that $[z/y]\circ[y/x]$ and $[z/x]$ coincide on
$\var(\phi)$. So let $u\in\var(\phi)$. We need to show that
$[z/y]\circ[y/x](u)=[z/x](u)$. Since $y\not\in\var(\phi)$ we have
$u\neq y$. Suppose first that $u=x$. Then we have:
    \[
    [z/y]\circ[y/x](u)=[z/y](y)=z=[z/x](u)
    \]
Suppose now that $u\neq x$. Then $u\not\in\{x,y\}$ and furthermore:
    \[
    [z/y]\circ[y/x](u)=[z/y](u)=u=[z/x](u)
    \]
In any case, we have proved that $[z/y]\circ[y/x](u)=[z/x](u)$.
\end{proof}

When replacing a variable $x$ by a variable $y$ in a formula $\phi$,
we would expect all occurrences of the variable $x$ to have
disappeared in $\phi[y/x]$. In other words, we would expect the
variable $x$ {\em not} to be a variable of the formula  $\phi[y/x]$.
This is indeed the case when $x\neq y$, a condition which may easily
be forgotten.
\begin{prop}\label{logic:prop:inplaceof:notvar}
Let $V$ be a set, $x,y\in V$ with $x\neq y$. Then for all $\phi\in\pv$:
    \[
    x\not\in\var(\phi[y/x])
    \]
\end{prop}
\begin{proof}
Suppose to the contrary that $x\in\var(\phi[y/x])$. From
proposition~(\ref{logic:prop:var:of:substitution}), we have
$\var(\phi[y/x]) = [y/x](\var(\phi))$. So there exists
$u\in\var(\phi)$ such that $x=[y/x](u)$. If $x\neq u$ we obtain
$[y/x](u)=u$ and consequently $x=u$. If $x=u$ we obtain $[y/x](u)=y$
and consequently $x=y$. In both cases we obtain a contradiction.
\end{proof}
