We are still looking for the appropriate congruence which should be
defined on \pv\ so as to identify mathematical statements which are
deemed {\em identical}. Until now, we have highlighted the {\em
substitution congruence} defined in
page~\pageref{logic:def:sub:congruence} as a possible source of
identity. Unfortunately whatever final congruence we go for, it will
need to be larger (i.e. weaker) than the substitution congruence. As
mathematicians, we are familiar with the fact that the {\em order of
quantification} in a given mathematical statement is immaterial. For
example, the formulas $\phi=\forall x\forall y(x\in y)$ and
$\psi=\forall y\forall x(x\in y)$ should mean the same thing. Yet
the equivalence of $\phi$ and $\psi$ cannot be deduced from the
substitution congruence alone when $x\neq y$. Indeed, let $\sim$
denote the substitution congruence on \pv\ and suppose that
$\phi\sim\psi$. Define $\phi_{1}=\forall y(x\in y)$ and
$\psi_{1}=\forall x(x\in y)$. Then we have $\forall
x\phi_{1}\sim\forall y\psi_{1}$ with $x\neq y$. Using
theorem~(\ref{logic:the:sub:congruence:charac}) of
page~\pageref{logic:the:sub:congruence:charac} it follows that
$\psi_{1}\sim \phi_{1}[y\!:\!x]$ which is $\forall x(x\in
y)\sim\forall x(y\in x)$. Using
theorem~(\ref{logic:the:sub:congruence:charac}) once more we see
that $(x\in y)\sim(y\in x)$ which contradicts
theorem~(\ref{logic:the:sub:congruence:charac}). Permuting the order
of quantification cannot be achieved by substituting variables.
\index{congruence@Permutation congruence}
\begin{defin}\label{logic:def:perm:congruence}
Let $V$ be a set. We call {\em permutation congruence on \pv\ }the
congruence on \pv\ generated by the following set $R_{0}\subseteq
\pv\times\pv$:
    \[
    R_{0}=\left\{\,(\,\forall x\forall y\,\phi_{1}\,,\,\forall y\forall
    x\,\phi_{1}\,):\phi_{1}\in\pv\ ,\ x,y\in V\ \right\}
    \]
\end{defin}

We easily see that the permutation congruence preserves free variables\,:

\begin{prop}\label{logic:prop:FOPL:permcong:freevar}
Let $\sim$ denote the permutation congruence on \pv\ where $V$ is a
set. Then for all $\phi,\psi\in\pv$ we have the implication:
    \[
    \phi\sim\psi\ \Rightarrow\ \free(\phi)=\free(\psi)
    \]
\end{prop}
\begin{proof}
Let $\equiv$ denote the relation on \pv\ defined by $\phi\equiv\psi$
\ifand\ we have $\free(\phi)=\free(\psi)$. Then we need to show that
$\sim\,\subseteq\,\equiv\,$. However, we know from
proposition~(\ref{logic:prop:congruence:freevar}) that $\equiv$ is a
congruence on \pv. Since $\sim$ is defined as the smallest
congruence on \pv\ which contains the set $R_{0}$ of
definition~(\ref{logic:def:perm:congruence}), we simply need to show
that $R_{0}\subseteq\,\equiv\,$. So let $\phi_{1}\in\pv$ and $x,
y\in V$. We need to show that $\forall x\forall
y\phi_{1}\equiv\forall y\forall x\phi_{1}$ which is $\free(\forall
x\forall y\phi_{1})=\free(\forall y\forall x\phi_{1})$. This is
clearly the case since both sets are equal to
$\free(\phi_{1})\setminus\{x,y\}$.
\end{proof}


Suppose $\phi_{1}\in\pv$ and $x,y,z\in V$. If $\sim$ denotes the
permutation congruence on~\pv, then the following equivalence can
easily be proved:
    \begin{equation}\label{logic:eqn:FOPL:integerperm:example:1}
    \forall x\forall y\forall z\phi_{1}\sim\forall z\forall y\forall
    x\phi_{1}
    \end{equation}
The most direct proof probably involves permuting pairs of adjacent
variables until we reach the configuration $\forall z\forall
y\forall x$ having started from $\forall x\forall y\forall z$. More
generally, given $x\in V^{n}$ where $n\in\N$, given an arbitrary
permutation $\sigma:n\to n$, we would expect the following
equivalence to hold equally:
    \begin{equation}\label{logic:eqn:FOPL:integerperm:example:2}
    \forall x(n-1)\ldots\forall x (0)\,\phi_{1}\sim\forall
    x(\sigma(n-1))\ldots x(\sigma(0))\,\phi_{1}
    \end{equation}
There are several issues to be addressed with
equation~(\ref{logic:eqn:FOPL:integerperm:example:2}). One of our
concerns will be of course to establish its truth. There is also the
use of the loose notation '$\ldots$' which is arguably questionable
in a document dealing with formal logic. More fundamentally,
regardless of whether
formula~(\ref{logic:eqn:FOPL:integerperm:example:2}) is true or
false, it is an ugly formula. Granted we could improve things
slightly by writing:
    \begin{equation}\label{logic:eqn:FOPL:integerperm:example:3}
    \forall x_{n-1}\ldots\forall x_{0}\,\phi_{1}\sim\forall
    x_{\sigma(n-1)}\ldots x_{\sigma(0)}\,\phi_{1}
    \end{equation}
but this is hardly satisfactory. One of our objectives is therefore
to design the appropriate formalism to remove the ugliness
of~(\ref{logic:eqn:FOPL:integerperm:example:2}). Now when it comes
to establishing its proof, the key idea is the same as with $\forall
x\forall y\forall z$ and involves permuting adjacent variables until
we reach the appropriate configuration. So we shall need the fact
that an arbitrary permutation can always be expressed as a
composition of 'adjacent moves'. We provide a proof in the following
section.
