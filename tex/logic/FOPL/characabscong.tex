In this section, we shall provide a characterization theorem for the
absorption congruence, in similar fashion to what was done for
theorem~(\ref{logic:the:sub:congruence:charac}) of
page~\pageref{logic:the:sub:congruence:charac} and
theorem~(\ref{logic:the:perm:congruence:charac}) of
page~\pageref{logic:the:perm:congruence:charac} of the substitution
and permutation congruence respectively. As discussed in the
previous section, if $\sim$ denotes the absorption congruence on
\pv\ then the equivalence $\phi\sim\psi$ does not allow us to say
anything on the relative structures of $\phi$ and $\psi$. However,
we are able to infer something about the core formulas $\phi^{*}$
and $\psi^{*}$ of
definition~(\ref{logic:def:FOPL:abscong:core:decomposition}). Our
strategy to prove theorem~(\ref{logic:the:characabscong:charac})
below will mirror exactly that of
theorem~(\ref{logic:the:sub:congruence:charac}) and
theorem~(\ref{logic:the:perm:congruence:charac}). We start by
defining a binary relation of {\em almost equivalence} on \pv\ which
is our best guess of what a proper characterization of the
absorption congruence should look like. We then proceed to show that
the {\em almost equivalence} is indeed an equivalence relation on
\pv\ which is in fact a congruent relation, and we conclude by
showing that it coincides with the absorption congruence.

\begin{defin}\label{logic:def:characabscong:almost:equivalent}
Let $\sim$ be the absorption congruence on \pv\ where $V$ is a set.
Let $\phi,\psi\in\pv$. We say that $\phi$ is {\em almost equivalent
to $\psi$} and we write $\phi\simeq\psi$, \ifand\ one of the
following is the case:
    \begin{eqnarray*}
    (i)&&\phi^{*}\in\pvo\ ,\ \psi^{*}\in\pvo\ ,\ \mbox{and}\ \phi^{*}=\psi^{*}\\
    (ii)&&\phi^{*}=\bot\ \mbox{and}\ \psi^{*}=\bot\\
    (iii)&&\phi^{*}=\phi_{1}\to\phi_{2}\ ,\ \psi^{*}=\psi_{1}\to\psi_{2}\ ,\
    \phi_{1}\sim\psi_{1}\ \mbox{and}\ \phi_{2}\sim\psi_{2}\\
    (iv)&&\phi^{*}=\forall x\phi_{1}\ ,\ \psi^{*}=\forall x\psi_{1}\ ,\ x\in V\
    \mbox{and}\ \phi_{1}\sim\psi_{1}\\
    \end{eqnarray*}
where $\phi^{*}$ and $\psi^{*}$ are the core of $\phi$ and $\psi$
respectively as per {\em
definition~(\ref{logic:def:FOPL:abscong:core:decomposition})}.
\end{defin}
\begin{prop}
$(i),(ii),(iii),(iv)$ of {\em
def.~(\ref{logic:def:characabscong:almost:equivalent})} are mutually
exclusive.
\end{prop}
\begin{proof}
This follows immediately from
theorem~(\ref{logic:the:unique:representation}) of
page~\pageref{logic:the:unique:representation} applied to \pv.
\end{proof}

\begin{prop}\label{logic:prop:characabscong:almost:contains:r0}
Let $\simeq$ be the almost equivalence relation on \pv\ where $V$ is
a set. Then $\simeq$ contains the generator $R_{0}$ of {\em
definition~(\ref{logic:def:FOPL:abscong:absorption})}.
\end{prop}
\begin{proof}
Let $x\in V$ and $\phi_{1}\in\pv$ such that
$x\not\in\free(\phi_{1})$. We need to show that
$\phi_{1}\simeq\forall x \phi_{1}$. Let $\phi_{1}=\forall u\phi^{*}$
where $u\in V^{n}$ be the core decomposition of $\phi_{1}$. Since
$x\not\in\free(\phi_{1})$, from
lemma~(\ref{logic:lemma:FOPL:abscong:x:not:free}) we see that the
core decomposition of $\forall x\phi_{1}$ is $\forall v\phi^{*}$ for
some $v\in V^{n+1}$. In particular, $\phi_{1}$ and $\forall
x\phi_{1}$ have identical core $\phi^{*}$. Using
theorem~(\ref{logic:the:unique:representation}) of
page~\pageref{logic:the:unique:representation}, we must have
$\phi^{*}\in\pvo$ or $\phi^{*}=\bot$ or
$\phi^{*}=\phi_{1}\to\phi_{2}$ for some $\phi_{1},\phi_{2}\in\pv$ or
$\phi^{*}=\forall z\psi$ for some $z\in V$ and $\psi\in\pv$. So
$(i)$, $(ii)$, $(iii)$ or $(iv)$ of
definition~(\ref{logic:def:characabscong:almost:equivalent}) must be
the case.
\end{proof}

\begin{prop}\label{logic:prop:FOPL:characabscong:almost:reflexive}
The almost equivalence relation on \pv\ is reflexive.
\end{prop}
\begin{proof}
Let $\phi\in\pv$. We need to show that $\phi\simeq\phi$. Let
$\phi^{*}$ be the core of $\phi$. From
theorem~(\ref{logic:the:unique:representation}) of
page~\pageref{logic:the:unique:representation}, one of the four
following cases must occur: if $\phi^{*}=(x\in y)$ for some $x,y\in
V$, then $\phi\simeq\phi$ follows from $\phi^{*}=\phi^{*}$. If
$\phi^{*}=\bot$ then $\phi\simeq\phi$ follows again from
$\phi^{*}=\phi^{*}$. If $\phi^{*}$ is of the form
$\phi^{*}=\phi_{1}\to\phi_{2}$ then $\phi\simeq\phi$ follows from
the reflexivity of $\sim$\,. If $\phi^{*}$ is of the form
$\phi^{*}=\forall x\phi_{1}$,  then $\phi\simeq\phi$ follows again
from the reflexivity of $\sim$\,, which completes our proof.
\end{proof}

\begin{prop}\label{logic:prop:FOPL:characabscong:almost:symmetric}
The almost equivalence relation on \pv\ is symmetric.
\end{prop}
\begin{proof}
Follows immediately from the symmetry of the absorption congruence
$\sim$\,.
\end{proof}

\begin{prop}\label{logic:prop:FOPL:characabscong:almost:transitive}
The almost equivalence relation on \pv\ is transitive.
\end{prop}
\begin{proof}
Let $\phi,\psi,\chi\in\pv$ such that $\phi\simeq\psi$ and
$\psi\simeq\chi$. We need to show that $\phi\simeq\chi$. We shall
consider the four possible cases $(i)$, $(ii)$, $(iii)$ and $(iv)$
of definition~(\ref{logic:def:characabscong:almost:equivalent}) in
relation to $\phi\simeq\psi$. Let $\phi^{*},\psi^{*}$ and $\chi^{*}$
denote the core of $\phi,\psi$ and $\chi$ respectively. First we
assume that $\phi^{*}\in\pvo$, $\psi^{*}\in\pvo$ with
$\phi^{*}=\psi^{*}$. Then from $\psi\simeq\chi$ we must have
$\chi^{*}\in\pvo$ and $\psi^{*}=\chi^{*}$. It follows that
$\phi^{*}\in\pvo$, $\chi^{*}\in\pvo$ and $\phi^{*}=\chi^{*}$ and we
see that $\phi\simeq\chi$. Next we assume that
$\phi^{*}=\bot=\psi^{*}$. Then from $\psi\simeq\chi$ we must have
$\psi^{*}=\bot=\chi^{*}$ and we conclude once again that
$\phi\simeq\chi$. Next we assume that $\phi^{*}$ and $\psi^{*}$ are
of the form $\phi^{*}=\phi_{1}\to\phi_{2}$ and
$\psi^{*}=\psi_{1}\to\psi_{2}$ with $\phi_{1}\sim\psi_{1}$ and
$\phi_{2}\sim\psi_{2}$. Then from $\psi\simeq\chi$ we see that
$\chi^{*}$ must be of the form $\chi^{*}=\chi_{1}\to\chi_{2}$ with
$\psi_{1}\sim\chi_{1}$ and $\psi_{2}\sim\chi_{2}$. From the
transitivity of the absorption congruence, it follows that
$\phi_{1}\sim\chi_{1}$ and $\phi_{2}\sim\chi_{2}$ and consequently
$\phi\simeq\chi$. Finally, we assume that $\phi^{*}$ and $\psi^{*}$
are of the form $\phi^{*}=\forall x\phi_{1}$ and $\psi^{*}=\forall
x\psi_{1}$ with $\phi_{1}\sim\psi_{1}$. Then from $\psi\simeq\chi$
we see that $\chi^{*}$ must be of the form $\chi^{*}=\forall
x\chi_{1}$ with $\psi_{1}\sim\chi_{1}$. Once again by transitivity
we obtain $\phi_{1}\sim\chi_{1}$ and finally $\phi\simeq\chi$ as
requested.
\end{proof}

\begin{prop}\label{logic:prop:characabscong:almost:implies:abs:congruence}
Let $\simeq$ be the almost equivalence and $\sim$ be the absorption
congruence on \pv, where $V$ is a set. For all $\phi,\psi\in\pv$:
    \[
    \phi\simeq\psi\ \Rightarrow\ \phi\sim\psi
    \]
\end{prop}
\begin{proof}
Let $\phi,\psi\in\pv$ such that $\phi\simeq\psi$. We need to show
that $\phi\sim\psi$. We shall consider the four possible cases
$(i)$, $(ii)$, $(iii)$ and $(iv)$ of
definition~(\ref{logic:def:characabscong:almost:equivalent}) in
relation to $\phi\simeq\psi$. Let $\phi^{*}$ and $\psi^{*}$ denote
the core of $\phi$ and $\psi$ respectively. From
proposition~(\ref{logic:prop:FOPL:abscong:core:equivalent}) we have
$\phi\sim\phi^{*}$ and $\psi\sim\psi^{*}$. It is therefore
sufficient to prove that $\phi^{*}\sim\psi^{*}$. This equivalence is
clear in cases $(i)$ and $(ii)$ of
definition~(\ref{logic:def:characabscong:almost:equivalent}). So we
assume that $\phi^{*}$ and $\psi^{*}$ are of the form
$\phi^{*}=\phi_{1}\to\phi_{2}$ and $\psi^{*}=\psi_{1}\to\psi_{2}$
where $\phi_{1}\sim\psi_{1}$ and $\phi_{2}\sim\psi_{2}$. The
absorption congruence being a congruent relation we obtain
$\phi^{*}\sim\psi^{*}$ as requested. Finally, we assume that
$\phi^{*}$ and $\psi^{*}$ are of the form $\phi^{*}=\forall
x\phi_{1}$ and $\psi^{*}=\forall x\psi_{1}$ with
$\phi_{1}\sim\psi_{1}$. Once again, using the fact that the
absorption congruence is a congruent relation we obtain
$\phi^{*}\sim\psi^{*}$.
\end{proof}

\begin{prop}\label{logic:prop:characabscong:almost:congruent}
The almost equivalence relation on \pv\ is congruent.
\end{prop}
\begin{proof}
By reflexivity, we already know that $\bot\simeq\bot$. So we assume
that $\phi=\phi_{1}\to\phi_{2}$ and $\psi=\psi_{1}\to\psi_{2}$ where
$\phi_{1}\simeq\psi_{1}$ and $\phi_{2}\simeq\psi_{2}$. We need to
show that $\phi\simeq\psi$. However, using
proposition~(\ref{logic:prop:FOPL:abscong:representation}), it is
clear the core decomposition of $\phi$ and $\psi$ are $\phi=\forall
u\phi^{*}$ and $\psi=\forall v\psi^{*}$ with $u=v=0$,
$\phi^{*}=\phi$ and $\psi^{*}=\psi$. Furthermore, from
proposition~(\ref{logic:prop:characabscong:almost:implies:abs:congruence})
we have $\phi_{1}\sim\psi_{1}$ and $\phi_{2}\sim\psi_{2}$. It
follows that $\phi^{*}$ and $\psi^{*}$ are of the form
$\phi^{*}=\phi_{1}\to\phi_{2}$ and $\psi^{*}=\psi_{1}\to\psi_{2}$
with $\phi_{1}\sim\psi_{1}$ and $\phi_{2}\sim\psi_{2}$. Hence we see
that $\phi\simeq\psi$ as requested. We now assume that $\phi=\forall
x\phi_{1}$ and $\psi=\forall x\psi_{1}$ where $x\in V$ and
$\phi_{1}\simeq\psi_{1}$. We need to show that $\phi\simeq\psi$. We
shall distinguish two cases: first we assume that
$x\in\free(\phi_{1})$. From $\phi_{1}\simeq\psi_{1}$ and
proposition~(\ref{logic:prop:characabscong:almost:implies:abs:congruence})
we obtain $\phi_{1}\sim\psi_{1}$. It follows from
proposition~(\ref{logic:prop:FOPL:abscong:freevar}) that
$\free(\phi_{1})=\free(\psi_{1})$ and consequently
$x\in\free(\psi_{1})$. Using
proposition~(\ref{logic:prop:FOPL:abscong:representation}) it is
clear that the core decomposition of $\phi$ and $\psi$ are
$\phi=\forall u\phi^{*}$ and $\psi=\forall v\psi^{*}$ where $u=v=0$,
$\phi^{*}=\phi$ and $\psi^{*}=\psi$. Note that the conditions
$x\in\free(\phi_{1})$ and $x\in\free(\psi_{1})$ are crucially
required to ensure $(ii)$ of
proposition~(\ref{logic:prop:FOPL:abscong:representation}) is
satisfied. So we have proved that $\phi^{*}$ and $\psi^{*}$ are of
the form $\phi^{*}=\forall x\phi_{1}$ and $\psi^{*}=\forall
x\psi_{1}$ with $\phi_{1}\sim\psi_{1}$ and we conclude that
$\phi\simeq\psi$ as requested. We now assume that
$x\not\in\free(\phi_{1})$. Then we also have
$x\not\in\free(\psi_{1})$. Let $\phi_{1}=\forall u_{1}\phi^{*}$ and
$\psi_{1}=\forall v_{1}\psi^{*}$ be the core decompositions of
$\phi_{1}$ and $\psi_{1}$ respectively. Then from
lemma~(\ref{logic:lemma:FOPL:abscong:x:not:free}) the core
decompositions of $\phi$ and $\psi$ are of the form $\phi=\forall
u\phi^{*}$ and $\psi=\forall v\psi^{*}$. In particular, $\phi$ and
$\phi_{1}$ have identical core $\phi^{*}$ while $\psi$ and
$\psi_{1}$ have identical core $\psi^{*}$. From
$\phi_{1}\simeq\psi_{1}$ and
definition~(\ref{logic:def:characabscong:almost:equivalent}) we
conclude that $\phi\simeq\psi$ as requested.
\end{proof}

\begin{prop}\label{logic:prop:characabscong:almost:congruence}
The almost equivalence relation on \pv\ is a congruence.
\end{prop}
\begin{proof}
From
proposition~(\ref{logic:prop:FOPL:characabscong:almost:reflexive}),
the relation~$\simeq$ is reflexive. From
proposition~(\ref{logic:prop:FOPL:characabscong:almost:symmetric})
it is symmetric while from
proposition~(\ref{logic:prop:FOPL:characabscong:almost:transitive})
it is transitive. It is therefore an equivalence relation on \pv.
Furthermore, from
proposition~(\ref{logic:prop:characabscong:almost:congruent}), the
relation~$\simeq$ is congruent. We conclude that it is a congruence
relation on \pv.
\end{proof}

\begin{prop}\label{logic:prop:characabscong:almost:is:abs:congruence}
Let $\simeq$ be the almost equivalence and $\sim$ be the absorption
congruence on \pv, where $V$ is a set. For all $\phi,\psi\in\pv$:
    \[
    \phi\simeq\psi\ \Leftrightarrow\ \phi\sim\psi
    \]
\end{prop}
\begin{proof}
We need to show the equality $\simeq\,=\,\sim\,$. The inclusion
$\subseteq$ follows from
proposition~(\ref{logic:prop:characabscong:almost:implies:abs:congruence}).
So it remains to show $\supseteq$\,. However, since $\sim$ is the
smallest congruence on \pv\ which contains the set $R_{0}$ of
definition~(\ref{logic:def:FOPL:abscong:absorption}), it is
sufficient to show that $\simeq$ is a congruence which contains
$R_{0}$. The fact that $\simeq$ is a congruence follows from
proposition~(\ref{logic:prop:characabscong:almost:congruence}). The
fact that it contains $R_{0}$ follows
from~(\ref{logic:prop:characabscong:almost:contains:r0}).
\end{proof}

We can now forget about the {\em almost equivalence} and conclude
with:

\index{congruence@Charact. of absorption congruence}
\begin{theorem}\label{logic:the:characabscong:charac}
Let $\sim$ be the absorption congruence on \pv\ where $V$ is a set.
For all $\phi,\psi\in\pv$, $\phi\sim\psi$ \ifand\ one of the
following is the case:
    \begin{eqnarray*}
    (i)&&\phi^{*}\in\pvo\ ,\ \psi^{*}\in\pvo\ ,\ \mbox{and}\ \phi^{*}=\psi^{*}\\
    (ii)&&\phi^{*}=\bot\ \mbox{and}\ \psi^{*}=\bot\\
    (iii)&&\phi^{*}=\phi_{1}\to\phi_{2}\ ,\ \psi^{*}=\psi_{1}\to\psi_{2}\ ,\
    \phi_{1}\sim\psi_{1}\ \mbox{and}\ \phi_{2}\sim\psi_{2}\\
    (iv)&&\phi^{*}=\forall x\phi_{1}\ ,\ \psi^{*}=\forall x\psi_{1}\ ,\ x\in V\ \mbox{and}\ \phi_{1}\sim\psi_{1}\\
    \end{eqnarray*}
where $\phi^{*}$ and $\psi^{*}$ are the core of $\phi$ and $\psi$
respectively as per {\em
definition~(\ref{logic:def:FOPL:abscong:core:decomposition})}.
\end{theorem}
\begin{proof}
This follows immediately from
definition~(\ref{logic:def:characabscong:almost:equivalent}) and
proposition~(\ref{logic:prop:characabscong:almost:is:abs:congruence}).
\end{proof}
