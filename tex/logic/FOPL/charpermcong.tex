We are now able to confront our next objective, that of establishing
a characterization theorem for the permutation congruence, similar
to theorem~(\ref{logic:the:sub:congruence:charac}) of
page~\pageref{logic:the:sub:congruence:charac}. This type of theorem
has proved very useful in the case of the substitution congruence.
Whenever $\phi\sim\psi$ it is important to be able to say something
about the common structure of $\phi$ and $\psi$. For example, if
$\phi=\phi_{1}\to\phi_{2}$, we want the ability to claim that $\psi$
is also of the form $\psi=\psi_{1}\to\psi_{2}$. An example of this
is proposition~(\ref{logic:prop:FOPL:substrank:impl}) establishing
the formula between substitution ranks
$\rnk(\phi)=\max(\,|\free(\phi)|\,,\,\rnk(\phi_{1})\,,\,\rnk(\phi_{2})\,)$
when $\phi=\phi_{1}\to\phi_{2}$. The strategy to prove our
characterization theorem~(\ref{logic:the:perm:congruence:charac})
below will follow very closely that of
theorem~(\ref{logic:the:sub:congruence:charac}). We shall not try to
be clever: we define a new binary relation on \pv\ which constitutes
our best guess of what an appropriate characterization should look
like. We then painstakingly prove that this new relation has all the
desired properties: it contains the generator of the permutation
congruence, it is reflexive, symmetric and transitive. It is a
congruent relation, hence a congruence on \pv\ which is in fact
equivalent to the permutation congruence.


\begin{defin}\label{logic:def:perm:equivalent}
Let $\sim$ be the permutation congruence on \pv\ where $V$ is a set.
Let $\phi,\psi\in\pv$. We say that $\phi$ is {\em almost permutation
equivalent to $\psi$} and we write $\phi\simeq\psi$, \ifand\ one of
the following is the case:
    \begin{eqnarray*}
    (i)&&\phi\in\pvo\ ,\ \psi\in\pvo\ ,\ \mbox{and}\ \phi=\psi\\
    (ii)&&\phi=\bot\ \mbox{and}\ \psi=\bot\\
    (iii)&&\phi=\phi_{1}\to\phi_{2}\ ,\ \psi=\psi_{1}\to\psi_{2}\ ,\
    \phi_{1}\sim\psi_{1}\ \mbox{and}\ \phi_{2}\sim\psi_{2}\\
    (iv)&&\phi=\forall x\phi_{1}\ ,\ \psi=\forall y\psi_{1}\ ,\
    \phi_{1}\sim\psi_{1}\ ,\ x\sim y\in V^{n}\ ,\ n\geq 1
    \end{eqnarray*}
where it is understood that $\phi_{1}$ and $\psi_{1}$ are
irreducible in $(iv)$, and $x\sim y$ refers to the permutation
equivalence on $V^{n}$ as per {\em
definition~(\ref{logic:def:permutation:equivalence:vn})}.
\end{defin}
\begin{prop}
$(i),(ii),(iii),(iv)$ of {\em
definition~(\ref{logic:def:perm:equivalent})} are mutually
exclusive.
\end{prop}
\begin{proof}
This is an immediate consequence of
theorem~(\ref{logic:the:unique:representation}) of
page~\pageref{logic:the:unique:representation} applied to the free
universal algebra \pv\ with free generator \pvo, where a formula
$\phi\in\pv$ is either an element of \pvo, or the contradiction
constant $\phi=\bot$, or an implication $\phi=\phi_{1}\to\phi_{2}$,
or a quantification $\phi=\forall x\phi_{1}$, but cannot be equal to
any two of those things simultaneously. Note that we are requesting
that $n\geq 1$ in $(iv)$ of
definition~(\ref{logic:def:perm:equivalent}), so $\phi=\forall
x\phi_{1}$ and $\psi=\forall y\psi_{1}$ are indeed quantifications
when $(iv)$ is the case.
\end{proof}

\begin{prop}\label{logic:prop:perm:contains:r0}
Let $\simeq$ be the almost permutation equivalence on \pv\ where $V$
is a set. Then $\simeq$ contains the generator $R_{0}$ of {\em
definition~(\ref{logic:def:perm:congruence})}.
\end{prop}
\begin{proof}
Let $x,y\in V$ and $\phi_{1}\in\pv$. Let $\phi=\forall x\forall
y\phi_{1}$ and $\psi=\forall y\forall x\phi_{1}$. We need to show
that $\phi\simeq\psi$. From
proposition~(\ref{logic:prop:irreducible:formula}), there exist
$n\in\N$,  $z\in V^{n}$ and $\theta_{1}$ irreducible such that
$\phi_{1}=\forall z\theta_{1}$. Consider the map $u:(n+2)\to V$
defined by $u_{|n}=z$, $u(n)=y$ and $u(n+1)=x$. Then we have:
    \[
    \phi=\forall x\forall y\phi_{1}=\forall u(n+1)\,\forall
    u(n)\,\forall u_{|n}\,\theta_{1}
    =\forall u(n+1)\,\forall u_{|n+1}\,\theta_{1}=\forall u\theta_{1}
    \]
Consider the map $v:(n+2)\to V$ defined by $v=u\circ[n:n+1]$. Then
$v_{|n}=z$, $v(n)=x$ and $v(n+1)=y$ and it follows similarly that
$\psi=\forall y\forall x\phi_{1}=\forall v\theta_{1}$. Hence, we
have found $\theta_{1}$ irreducible, $u,v\in V^{n+2}$ with $u\sim v$
such that $\phi=\forall u\theta_{1}$ and $\psi=\forall v\theta_{1}$.
From $(iv)$ of definition~(\ref{logic:def:perm:equivalent}) we
conclude that $\phi\simeq\psi$.
\end{proof}

\begin{prop}\label{logic:prop:perm:reflexive}
Let $\simeq$ be the almost permutation equivalence on \pv\ where $V$
is a set. Then $\simeq$ is a reflexive relation on \pv.
\end{prop}
\begin{proof}
Let $\phi\in\pv$. We need to show that $\phi\simeq\phi$. From
theorem~(\ref{logic:the:unique:representation}) of
page~\pageref{logic:the:unique:representation} we know that $\phi$
is either an element of \pvo, or $\phi=\bot$ or
$\phi=\phi_{1}\to\phi_{2}$ or $\phi=\forall x\phi_{1}$ for some
$\phi_{1},\phi_{2}\in\pv$ and $x\in V$. We shall consider these four
mutually exclusive cases separately. Suppose first that
$\phi\in\pvo$: then from $\phi=\phi$ we obtain $\phi\simeq\phi$.
Suppose next that $\phi=\bot$: then it is clear that
$\phi\simeq\phi$. Suppose now that $\phi=\phi_{1}\to\phi_{2}$. Since
the permutation congruence $\sim$ is reflexive, we have
$\phi_{1}\sim\phi_{1}$ and $\phi_{2}\sim\phi_{2}$. It follows from
$(iii)$ of definition~(\ref{logic:def:perm:equivalent}) that
$\phi\simeq\phi$. Suppose finally that $\phi=\forall x\phi_{1}$.
From proposition~(\ref{logic:prop:irreducible:formula}), there exist
$n\in\N$, $z\in V^{n}$ and $\theta_{1}$ irreducible such that
$\phi_{1}=\forall z\theta_{1}$. Consider the map $u:(n+1)\to V$
defined by $u_{|n}=z$ and $u(n)=x$. Then we have:
    \[
    \phi=\forall x\phi_{1}=\forall x\forall z\,\theta_{1}=\forall
    u(n)\,\forall u_{|n}\,\theta_{1}=\forall u\theta_{1}
    \]
Since $\theta_{1}$ is irreducible, $\theta_{1}\sim\theta_{1}$ and
$u\sim u$, we conclude from $(iv)$ of
definition~(\ref{logic:def:perm:equivalent}) that $\phi\simeq\phi$.
\end{proof}
\begin{prop}\label{logic:prop:perm:symmetric}
Let $\simeq$ be the almost permutation equivalence on \pv\ where $V$
is a set. Then $\simeq$ is a symmetric relation on \pv.
\end{prop}
\begin{proof}
Let $\phi,\psi\in\pv$ be such that $\phi\simeq\psi$. We need to show
that $\psi\simeq\phi$. We shall consider the four possible cases of
definition~(\ref{logic:def:perm:equivalent}): suppose first that
$\phi\in\pvo$, $\psi\in\pvo$ and $\phi=\psi$. Then it is clear that
$\psi\simeq\phi$. Suppose next that $\phi=\bot$ and $\psi=\bot$.
Then we also have $\psi\simeq\phi$. We now assume that
$\phi=\phi_{1}\to\phi_{2}$ and $\psi=\psi_{1}\to\psi_{2}$ with
$\phi_{1}\sim\psi_{1}$ and $\phi_{2}\sim\psi_{2}$. Since the
permutation congruence on \pv\ is symmetric, we have
$\psi_{1}\sim\phi_{1}$ and $\psi_{2}\sim\phi_{2}$. Hence we have
$\psi\simeq\phi$. We now assume that $\phi=\forall x\phi_{1}$ and
$\psi=\forall y\psi_{1}$ with $\phi_{1}\sim\psi_{1}$ and
$\phi_{1},\psi_{1}$ irreducible, $x\sim y\in V^{n}$ and $n\geq 1$.
Then once again by symmetry of the permutation congruence we have
$\psi_{1}\sim\phi_{1}$ and furthermore $y\sim x$. It follows that
$\psi\simeq\phi$.
\end{proof}

\begin{prop}\label{logic:prop:perm:transitive}
Let $\simeq$ be the almost permutation equivalence on \pv\ where $V$
is a set. Then $\simeq$ is a transitive relation on \pv.
\end{prop}
\begin{proof}
Let $\phi,\psi$ and $\chi\in\pv$ be such that $\phi\simeq\psi$ and
$\psi\simeq\chi$. We need to show that $\phi\simeq\chi$. We shall
consider the four possible cases of
definition~(\ref{logic:def:perm:equivalent}) in relation to
$\phi\simeq\psi$. Suppose first that $\phi,\psi\in\pvo$ and
$\phi=\psi$. Then from $\psi\simeq\chi$ we obtain $\psi,\chi\in\pvo$
and $\psi=\chi$. It follows that $\phi,\chi\in\pvo$ and $\phi=\chi$.
Hence we see that $\phi\simeq\chi$. We now assume that
$\phi=\psi=\bot$. Then from $\psi\simeq\chi$ we obtain
$\psi=\chi=\bot$. It follows that $\phi=\chi=\bot$ and consequently
$\phi\simeq\chi$. We now assume that $\phi=\phi_{1}\to\phi_{2}$ and
$\psi=\psi_{1}\to\psi_{2}$ with $\phi_{1}\sim\psi_{1}$ and
$\phi_{2}\sim\psi_{2}$. From $\psi\simeq\chi$ we obtain
$\chi=\chi_{1}\to\chi_{2}$ with $\psi_{1}\sim\chi_{1}$ and
$\psi_{2}\sim\chi_{2}$. The permutation congruence being transitive,
it follows that $\phi_{1}\sim\chi_{1}$ and $\phi_{2}\sim\chi_{2}$.
Hence we see that $\phi\simeq\chi$. We now assume that $\phi=\forall
x\phi_{1}$ and $\psi=\forall y\psi_{1}$ with $\phi_{1}\sim\psi_{1}$,
and $\phi_{1},\psi_{1}$ irreducible, $x\sim y\in V^{n}$ and $n\geq
1$. Since $n\geq 1$, from $\psi\simeq\chi$ only the case $(iv)$ of
definition~(\ref{logic:def:perm:equivalent}) is possible.
Furthermore from proposition~(\ref{logic:prop:irreducible:formula}),
the representation $\psi=\forall y\psi_{1}$ with $\psi_{1}$
irreducible is unique. It follows that $\chi=\forall z\chi_{1}$ with
$\psi_{1}\sim\chi_{1}$ irreducible and $y\sim z\in V^{n}$. The
permutation congruence being transitive, we obtain
$\phi_{1}\sim\chi_{1}$. Furthermore, also by transitivity we obtain
$x\sim z$ and we conclude that $\phi\simeq\chi$.
\end{proof}

\begin{prop}\label{logic:prop:perm:implies:perm:congruence}
Let $\simeq$ be the almost permutation equivalence and $\sim$ be the
permutation congruence on \pv, where $V$ is a set. For all
$\phi,\psi\in\pv$:
    \[
    \phi\simeq\psi\ \Rightarrow\ \phi\sim\psi
    \]
\end{prop}
\begin{proof}
Let $\phi,\psi\in\pv$ such that $\phi\simeq\psi$. We need to show
that $\phi\sim\psi$. We shall consider the four possible cases of
definition~(\ref{logic:def:perm:equivalent}) in relation to
$\phi\simeq\psi$. Suppose first that $\phi=\psi\in\pvo$. From the
reflexivity of the permutation congruence, it is clear that
$\phi\sim\psi$. Suppose next that $\phi=\psi=\bot$. Then we also
have $\phi\sim\psi$. We now assume that $\phi=\phi_{1}\to\phi_{2}$
and $\psi=\psi_{1}\to\psi_{2}$ where $\phi_{1}\sim\psi_{1}$ and
$\phi_{2}\sim\psi_{2}$. The permutation congruence being a congruent
relation on \pv, we obtain $\phi\sim\psi$. Next we assume that
$\phi=\forall x\phi_{1}$ and $\psi=\forall y\psi_{1}$ where
$\phi_{1}\sim\psi_{1}$ and $\phi_{1},\psi_{1}$ are irreducible,
$x\sim y\in\ V^{n}$ and $n\geq 1$. The permutation congruence being
a congruent relation on \pv, from
proposition~(\ref{logic:prop:iterated:congruence}) and
$\phi_{1}\sim\psi_{1}$ we obtain $\forall y\phi_{1}\sim\forall
y\psi_{1}$. Hence by transitivity of the permutation congruence, it
remains to show that $\forall x\phi_{1}\sim \forall y\phi_{1}$ which
follows immediately from $x\sim y$ and
proposition~(\ref{logic:prop:iterated:permutation}).
\end{proof}

\begin{prop}\label{logic:prop:perm:congruent}
Let $\simeq$ be the almost permutation equivalence on \pv\ where $V$
is a set. Then $\simeq$ is a congruent relation on \pv.
\end{prop}
\begin{proof}
From proposition~(\ref{logic:prop:perm:reflexive}), the almost
permutation equivalence $\simeq$ is reflexive and so
$\bot\simeq\bot$. We now assume that $\phi=\phi_{1}\to\phi_{2}$ and
$\psi=\psi_{1}\to\psi_{2}$ where $\phi_{1}\simeq\psi_{1}$ and
$\phi_{2}\simeq\psi_{2}$. We need to show that $\phi\simeq\psi$.
However from
proposition~(\ref{logic:prop:perm:implies:perm:congruence}) we have
$\phi_{1}\sim\psi_{1}$ and $\phi_{2}\sim\psi_{2}$ and it follows
from definition~(\ref{logic:def:perm:equivalent}) that
$\phi\simeq\psi$. We now assume that $\phi=\forall x\phi_{1}$ and
$\psi=\forall x\psi_{1}$ where $\phi_{1}\simeq\psi_{1}$ and $x\in
V$. We need to show that $\phi\simeq\psi$. Once again from
proposition~(\ref{logic:prop:perm:implies:perm:congruence}) we have
$\phi_{1}\sim\psi_{1}$. We shall now distinguish two cases in
relation to $\phi_{1}\simeq\psi_{1}$. First we assume that $(i)$
$(ii)$ or $(iii)$ of definition~(\ref{logic:def:perm:equivalent}) is
the case. Then both $\phi_{1}$ and $\psi_{1}$ are irreducible.
Defining $x^{*}:1\to V$ by setting $x^{*}(0)=x$ we obtain:
    \[
    \phi=\forall x\phi_{1}=\forall x^{*}(0)\,\phi_{1}=\forall x^{*}\phi_{1}
    \]
and similarly $\psi=\forall x^{*}\psi_{1}$. From $\phi_{1},\psi_{1}$
irreducible and $\phi_{1}\sim\psi_{1}$ we conclude that
$\phi\simeq\psi$. We now assume that $(iv)$ of
definition~(\ref{logic:def:perm:equivalent}) is the case. Then
$\phi_{1}=\forall u\phi_{1}^{*}$ and $\psi_{1}=\forall
v\psi_{1}^{*}$ where $\phi_{1}^{*}\sim\psi_{1}^{*}$ and
$\phi_{1}^{*}$, $\psi_{1}^{*}$ are irreducible, $u\sim v\in V^{n}$
and $n\geq 1$. Defining $u^{*}:(n+1)\to V$ by setting $u^{*}_{|n}=u$
and $u^{*}(n)=x$ we obtain:
    \[
    \phi=\forall x\phi_{1}=\forall x\forall u\phi_{1}^{*}=\forall
    u^{*}(n)\forall u^{*}_{|n}\,\phi_{1}^{*}=\forall u^{*}\phi_{1}^{*}
    \]
and similarly $\psi=\forall v^{*}\psi_{1}^{*}$ where $v:(n+1)\to V$
is defined by $v^{*}_{|n}=v$ and $v^{*}(n)=x$. Since $\phi_{1}^{*},
\psi_{1}^{*}$ are irreducible and $\phi_{1}^{*}\sim\psi_{1}^{*}$, in
order to show that $\phi\simeq\psi$ it is sufficient to prove that
$u^{*}\sim v^{*}$. However since $u\sim v$, there exists a
permutation $\sigma:n\to n$ of order $n$ such that $v=u\circ\sigma$.
Defining $\sigma^{*}:(n+1)\to (n+1)$ by $\sigma^{*}_{|n}=\sigma$ and
$\sigma^{*}(n)=n$ we obtain a permutation of order $n+1$ such that
$v^{*}=u^{*}\circ\sigma^{*}$.
\end{proof}
\begin{prop}\label{logic:prop:perm:congruence}
Let $\simeq$ be the almost permutation equivalence on \pv\ where $V$
is a set. Then $\simeq$ is a congruence on \pv.
\end{prop}
\begin{proof}
We need to show that $\simeq$ is reflexive, symmetric, transitive
and that it is a congruent relation on \pv. From
proposition~(\ref{logic:prop:perm:reflexive}), the relation~$\simeq$
is reflexive. From proposition~(\ref{logic:prop:perm:symmetric}) it
is symmetric while from
proposition~(\ref{logic:prop:perm:transitive}) it is transitive.
Finally from proposition~(\ref{logic:prop:perm:congruent}) the
relation~$\simeq$ is a congruent relation.
\end{proof}


\begin{prop}\label{logic:prop:perm:is:perm:congruence}
Let $\simeq$ be the almost permutation equivalence and $\sim$ be the
permutation congruence on \pv, where $V$ is a set. For all
$\phi,\psi\in\pv$:
    \[
    \phi\simeq\psi\ \Leftrightarrow\ \phi\sim\psi
    \]
\end{prop}
\begin{proof}
From proposition~(\ref{logic:prop:perm:implies:perm:congruence}) it
is sufficient to show the implication $\Leftarrow$ or equivalently
the inclusion $\sim\,\subseteq\,\simeq\,$. Since $\sim$ is the
permutation congruence on \pv, it is the smallest congruence on \pv\
which contains the set $R_{0}$ of
definition~(\ref{logic:def:perm:congruence}). In order to show the
inclusion $\sim\,\subseteq\,\simeq$ it is therefore sufficient to
show that $\simeq$ is a congruence on \pv\ such that
$R_{0}\subseteq\,\simeq$. The fact that it is a congruence stems
from proposition~(\ref{logic:prop:perm:congruence}). The fact that
$R_{0}\subseteq\,\simeq$ follows from
proposition~(\ref{logic:prop:perm:contains:r0}).
\end{proof}

Recall that {\em irreducible} formulas are defined in
page~\pageref{logic:def:irreducible:formula}. We can now forget
about the {\em almost permutation equivalence} and simply remember
the following characterization of the permutation congruence on
\pv\,:
\index{congruence@Charact. of perm. congruence}
\begin{theorem}\label{logic:the:perm:congruence:charac}
Let $\sim$ be the permutation congruence on \pv\ where $V$ is a set.
For all $\phi,\psi\in\pv$, $\phi\sim\psi$ \ifand\ one of the
following is the case:
    \begin{eqnarray*}
    (i)&&\phi\in\pvo\ ,\ \psi\in\pvo\ ,\ \mbox{and}\ \phi=\psi\\
    (ii)&&\phi=\bot\ \mbox{and}\ \psi=\bot\\
    (iii)&&\phi=\phi_{1}\to\phi_{2}\ ,\ \psi=\psi_{1}\to\psi_{2}\ ,\
    \phi_{1}\sim\psi_{1}\ \mbox{and}\ \phi_{2}\sim\psi_{2}\\
    (iv)&&\phi=\forall x\phi_{1}\ ,\ \psi=\forall y\psi_{1}\ ,\
    \phi_{1}\sim\psi_{1}\ ,\ x\sim y\in V^{n}\ ,\ n\geq 1
    \end{eqnarray*}
where $x\sim y$ refers to the permutation equivalence on $V^{n}$ as
per {\em definition~(\ref{logic:def:permutation:equivalence:vn})}.
Furthermore, we may assume that $\phi_{1}$ and $\psi_{1}$ are
irreducible in $(iv)$.
\end{theorem}
\begin{proof}
First we show the implication $\Rightarrow$. So we assume that
$\phi\sim\psi$. We need to show that $(i)$, $(ii)$, $(iii)$ or
$(iv)$ is the case. However from
proposition~(\ref{logic:prop:perm:is:perm:congruence}) we have
$\phi\simeq\psi$. It follows from
definition~(\ref{logic:def:perm:equivalent}) that $(i)$, $(ii)$,
$(iii)$ or $(iv)$ is indeed the case with $\phi_{1}$ and $\psi_{1}$
irreducible in the case of $(iv)$. We now show the reverse
implication $\Leftarrow$. So we assume that $(i)$, $(ii)$, $(iii)$
or $(iv)$  is the case. We need to show that $\phi\sim\psi$. We
shall distinguish two cases. First we assume that $(i)$, $(ii)$ or
$(iii)$ is the case. Then from
definition~(\ref{logic:def:perm:equivalent}) we obtain
$\phi\simeq\psi$ and consequently from
proposition~(\ref{logic:prop:perm:is:perm:congruence}) we have
$\phi\sim\psi$. We now assume that $(iv)$ is the case. Then from
proposition~(\ref{logic:prop:iterated:congruence}) and
$\phi_{1}\sim\psi_{1}$ we obtain $\forall x\phi_{1}\sim\forall
x\psi_{1}$. Furthermore, from
proposition~(\ref{logic:prop:iterated:permutation}) and $x\sim y$ we
obtain $\forall x\psi_{1}\sim\forall y\psi_{1}$. By transitivity of
the permutation congruence, we conclude that:
    \[
    \phi=\forall x\phi_{1}\sim\forall y\psi_{1}=\psi
    \]
\end{proof}
