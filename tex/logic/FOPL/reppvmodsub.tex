One of our goals in defining the minimal transform mapping ${\cal
M}:\pv\to\pvb$ in
page~\pageref{logic:def:FOPL:mintransform:transform} was to design
an algebra \qv\ where the free and bound variables would be chosen
from different sets. We now pursue this idea further. Given a set
$V$, we defined the minimal extension $\bar{V}$ by adding to $V$ a
disjoint copy of $\N$, giving us plenty of bound variables to choose
from. Unfortunately in spite of this, the free algebra $\pvb$ is not
very interesting to us, as it contains too many formulas, including
those such as $\phi=\forall x(0\in x)$ which contravene our desired
convention of keeping free variables in $V$ and bound variables in
\N. So we need to restrict our attention to a smaller subset of
\pvb\ and the obvious choice is to pick $\qv={\cal M}(\pv)$ i.e. to
choose \qv\ as the range of the minimal transform mapping. From
proposition~(\ref{logic:prop:FOPL:mintransform:variables}) and
proposition~(\ref{logic:prop:FOPL:mintransform:variables:bound})
this guarantees all formulas have free variables in $V$ and bound
variables in \N. So we have the right set of formulas. However \qv\
is not yet an algebra, as it is not a universal sub-algebra of \pvb.
Indeed, given $x\in V$ and $\phi\in\qv$, the formula $\forall x\phi$
is not an element of \qv. Somehow we would like \qv\ to be an
algebra. But of which type? The free universal algebras \pv\ and
\pvb\ do not have the same type. The algebra \pvb\ has many more
quantification operators $\forall n$ for $n\in\N$ which \pv\ does
not have. So what do we want for \qv? All elements of \qv\ have free
variables in $V$. It does not add any value to quantify a formula
with respect to a variable which is not free. So it is clear the
unary operators $\forall n$ for $n\in\N$ should not be part of \qv,
and \qv\ should be of the same type of universal algebra as \pv. So
let's endow \qv\ with the appropriate structure of universal
algebra: First we need an operator $\bot:\qv^{0}\to\qv$. Since
$\bot={\cal M}(\bot)\in\qv$ we should keep the structure induced by
\pvb\ and define $\bot(0)=\bot\in\qv$. Next we need a binary
operator $\to:\qv^{2}\to\qv$. Since for all
$\phi_{1},\phi_{2}\in\pv$, we have:
    \[
    {\cal M}(\phi_{1})\to{\cal M}(\phi_{2})={\cal M}(\phi_{1}\to\phi_{2})
    \]
we see that \qv\ is closed under the operator $\to$ of \pvb. So we
should keep that too. Finally given $x\in V$, we need a unary
operator $\forall {\bf x}:\qv^{1}\to\qv$. In this case the operator
$\forall x$ of \pvb\ is not suitable for our purpose, as \qv\ is not
closed under $\forall x$. So we need to define a new operator which
we denote $\forall{\bf x}$ (with the letter $\bf x$ in bold) to
avoid any possible confusion. Let us pick:
    \begin{equation}\label{logic:eqn:FOPL:reppvmodsub:1}
    \forall\phi\in\qv\ ,\ \forall{\bf x}\phi = \forall n\phi[n/x]
    \end{equation}
where $n=\min\{k\in\N:\mbox{$[k/x]$ valid for $\phi$}\}$ and the
$\forall n$ on the right-hand-side of
equation~(\ref{logic:eqn:FOPL:reppvmodsub:1}) is the usual
quantification operator of \pvb. We need to check that this
definition does indeed yield an operator $\forall {\bf
x}:\qv^{1}\to\qv$, specifically that $\forall {\bf x}\phi\in\qv$ for
all $\phi\in\qv$. So let $\phi\in\qv$. There exists $\phi_{1}\in\pv$
such that $\phi={\cal M}(\phi_{1})$ and consequently we have:
    \[
    \forall{\bf x}\phi=\forall n\phi[n/x]=\forall n{\cal
    M}(\phi_{1})[n/x]={\cal M}(\forall x\phi_{1})\in\qv
    \]
where the last equality follows from
definition~(\ref{logic:def:FOPL:mintransform:transform}) and:
    \[
    n=\min\{k\in\N:\mbox{$[k/x]$ valid for ${\cal M}(\phi_{1})$}\}
    \]
So \qv\ is now a universal algebra of the same type as \pv\ which is
what we set out to achieve. Note that this algebra is not free in
general since:
    \[
    \forall{\bf x}\forall{\bf y}(x\in y)=\forall\,1\forall\,0\,(1\in
    0)=\forall{\bf y}\forall{\bf x}(y\in x)
    \]
whenever $x$ and $y$ are two distinct elements of $V$. This
contradicts the uniqueness principle of
theorem~(\ref{logic:the:unique:representation}) of
page~\pageref{logic:the:unique:representation} which prevails in
free universal algebras.
\index{minimal@Minimal transform
algebra}\index{q@$\qv$ : minimal transform algebra}
\begin{defin}\label{logic:def:FOPL:reppvmodsub:qv}
Let $V$ be a set with first order logic type $\alpha$. We call {\em
minimal transform algebra} associated with $V$, the universal
algebra \qv\ of type $\alpha$:
    \[
    \qv={\cal M}(\pv)\subseteq\pvb
    \]
where ${\cal M}:\pv\to\pvb$ is the minimal transform mapping, and
the operators $\bot, \to$ are induced from \pvb\ while for all $x\in
V$ and $\phi\in\qv$ we have:
   \[
   \forall{\bf x}\phi = \forall n\phi[n/x]
   \]
where $n=\min\{k\in\N:\mbox{$[k/x]$ valid for $\phi$}\}$.
\end{defin}

\begin{prop}\label{logic:prop:FOPL:reppvmodsub:morphism}
Let $V$ be a set with minimal transform algebra \qv. Then, the
minimal transform mapping ${\cal M}:\pv\to\qv$ is a surjective
morphism.
\end{prop}
\begin{proof}
The minimal transform mapping ${\cal M}:\pv\to\qv$ is clearly
surjective by definition~(\ref{logic:def:FOPL:reppvmodsub:qv}) of
\qv. So it remains to show it is a morphism. From
definition~(\ref{logic:def:FOPL:mintransform:transform}) of
page~\pageref{logic:def:FOPL:mintransform:transform} we have ${\cal
M}(\bot)=\bot$ and for all $\phi_{1},\phi_{2}\in\pv$:
    \[
     {\cal M}(\phi_{1}\to\phi_{2})={\cal M}(\phi_{1})
     \to{\cal M}(\phi_{2})
     \]
So it remains to show that ${\cal M}(\forall x\phi_{1})=\forall{\bf
x}{\cal M}(\phi_{1})$ for all $x\in V$ and $\phi_{1}\in\pv$.
However, defining $n=\min\{k\in\N:\mbox{$[k/x]$ valid for ${\cal
M}(\phi_{1})$}\}$ we have:
    \[
    {\cal M}(\forall x\phi_{1})=\forall n{\cal
    M}(\phi_{1})[n/x]=\forall{\bf x}{\cal M}(\phi_{1})
    \]
where the two equalities follow from
definitions~(\ref{logic:def:FOPL:mintransform:transform})
and~(\ref{logic:def:FOPL:reppvmodsub:qv}) respectively.
\end{proof}

So from proposition~(\ref{logic:prop:FOPL:reppvmodsub:morphism}) we
have a morphism ${\cal M}:\pv\to\qv$ which is surjective, while from
theorem~(\ref{logic:the:FOPL:mintransfsubcong:kernel}) we have the
equivalence:
     \[
     {\cal M}(\phi)={\cal M}(\psi)\ \Leftrightarrow\ \phi\sim\psi
     \]
So the kernel of this morphism is exactly the substitution
congruence on \pv.

\begin{theorem}\label{logic:the:FOPL:reppvmodsub:isomorphism}
Let $V$ be a set with minimal transform algebra \qv, and let the
quotient universal algebra of \pv\ modulo the substitution
congruence be denoted  $[\pv]$ as per {\em
theorem~(\ref{logic:the:quotient})}. Let ${\cal M}^{*}:[\pv]\to\qv$
be defined as:
    \[
    \forall\phi\in\pv\ ,\ {\cal M}^{*}([\phi])={\cal
    M}(\phi)
    \]
where ${\cal M}$ is the minimal transform mapping. Then ${\cal
M}^{*}$ is an isomorphism.
\end{theorem}
\begin{proof}
From proposition~(\ref{logic:prop:FOPL:reppvmodsub:morphism}) ${\cal
M}:\pv\to\qv$ is a surjective morphism and from
theorem~(\ref{logic:the:FOPL:mintransfsubcong:kernel}), the kernel
$\ker({\cal M})$ coincides with the substitution congruence on \pv.
Hence the fact that ${\cal M}^{*}:[\pv]\to\qv$ is an isomorphism
follows immediately from the first isomorphism
theorem~(\ref{logic:the:UA:firstiso}) of
page~\pageref{logic:the:UA:firstiso}.
\end{proof}
