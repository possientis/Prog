In the previous section, we hinted at the difficulty of defining the
{\em evaluation} $\phi[a]$ when $\phi=\forall y(x\in y)$ and $a=y$.
The issue is that the substitution $[y/x]$ is not valid for the
formula $\phi$ (when $x\neq y$), making it impossible to write:
    \[
    \forall a\in V\ ,\ \phi[a]=\phi[a/x]
    \]
We defined the minimal transform mapping ${\cal M}:\pv\to\pvb$ with
this problem in mind. Now from
definition~(\ref{logic:def:FOPL:mintransform:transform}), the
minimal transform of $\phi$ is ${\cal M}(\phi)=\forall\,0\,(x\in 0)$
and it is clear that $[a/x]$ is now valid for ${\cal M}(\phi)$ for
all $a\in V$. So it would be sensible to define:
    \[
    \forall a\in V\ ,\ \phi[a]={\cal M}(\phi)[a/x]
    \]
Of course such a definition would lead to $\phi[a]=\forall\,0\,(a\in
0)$ which is an element of \pvb\ and not an element of \pv. So it
may not be what we want, but we are certainly moving forward. More
generally suppose $\phi\in\pv$ is an arbitrary formula and
$\sigma:V\to W$ is an arbitrary map. We would like to define the
formula $\sigma(\phi)$ in ${\bf P}(W)$ but in a way which is
sensible, rather than following blindly the details of
definition~(\ref{logic:def:substitution}). The problem is that the
substitution $\sigma$ may not be valid for the formula $\phi$. This
doesn't mean we shouldn't be interested in $\sigma(\phi)$. In many
cases we may only care about what happens to the free variables.
Bound variables are not interesting. We do not mind having them
moved around, provided the basic structure of the formula remains.
So when $\sigma$ is valid for $\phi$, this is what happens when
using $\sigma(\phi)$ as per
definition~(\ref{logic:def:substitution}). However when $\sigma$ is
not valid for $\phi$, we need to find another way to move the free
variables of $\phi$ without making a mess of it. So consider the
minimal transform ${\cal M}(\phi)$. There is no longer any conflicts
between the free and he bound variables. If $\sigma:V\to W$ is an
arbitrary map, we can move the free variables of ${\cal M}(\phi)$
according to the substitution $\sigma$  without creating nonsense.
To make this idea precise, we define: \index{minimal@Minimal
extension of map}\index{sigma@$\bar{\sigma}$ : minimal extension of
$\sigma$}
\begin{defin}\label{logic:def:FOPL:commute:minextensioon:map}
Let $V$ and $W$ be sets. Let $\sigma:V\to W$ be a map. We call {\em
minimal extension} of $\sigma$ the map
$\bar{\sigma}:\bar{V}\to\bar{W}$ defined by:
    \[
    \forall u\in \bar{V}\ ,\ \bar{\sigma}(u)=\left\{
    \begin{array}{lcl}
    \sigma(u)&\mbox{\ if\ }&u\in V\\
    u&\mbox{\ if\ }&u\in\N
    \end{array}
    \right.
    \]
where $\bar{V}$ and $\bar{W}$ are the minimal extensions of $V$ and
$W$ respectively.
\end{defin}
So the map $\bar{\sigma}:\bar{V}\to\bar{W}$ is the map moving the
free variables of ${\cal M}(\phi)$ according to $\sigma$, without
touching the bound variables. As expected we have:

\index{minimal@Minimal extension valid}
\begin{prop}\label{logic:def:FOPL:commute:minextension:valid}
Let $\sigma:V\to W$ be a map. Then for all $\phi\in\pv$ the minimal
extension $\bar{\sigma}:\bar{V}\to\bar{W}$ is valid for the minimal
transform ${\cal M}(\phi)$.
\end{prop}
\begin{proof}
We need to check the three properties of
proposition~(\ref{logic:prop:FOPL:validsub:minimalextension}) are
met in relation to $\bar{\sigma}:\bar{V}\to\bar{W}$ and ${\cal
M}(\phi)$ with $V_{0}=\N$. First we show property $(i)$\,: we need
to show that $\bound({\cal M}(\phi))\subseteq\N$ which follows from
proposition~(\ref{logic:prop:FOPL:mintransform:variables:bound}).
Next we show property~$(ii)$\,: we need to show that
$\bar{\sigma}_{|\N}$ is injective which is clear from
definition~(\ref{logic:def:FOPL:commute:minextensioon:map}). We
finally show property~$(iii)$\,: we need to show that
$\bar{\sigma}(\N)\cap\bar{\sigma}(\var({\cal
M}(\phi))\setminus\N)=\emptyset$. This follows from
$\bar{\sigma}(\N)\subseteq\N$ and $\bar{\sigma}(\var({\cal
M}(\phi))\setminus\N)\subseteq\bar{\sigma}(V)=\sigma(V)\subseteq W$,
while $W\cap\N = \emptyset$.
\end{proof}

So here we are. Given $\phi\in\pv$ and an arbitrary map $\sigma:V\to
W$ we cannot safely consider $\sigma(\phi)$ unless $\sigma$ is valid
for $\phi$. However, there is no issues in looking at
$\bar{\sigma}\circ{\cal M}(\phi)$ instead, since $\bar{\sigma}$ is
always valid for ${\cal M}(\phi)$. Of course $\bar{\sigma}\circ{\cal
M}(\phi)$ is an element of ${\bf P}(\bar{W})$ and not ${\bf P}(W)$.
It is however a lot better to have a meaningful element of ${\bf
P}(\bar{W})$, rather than a nonsensical formula $\sigma(\phi)$ of
${\bf P}(W)$. But what if $\sigma$ was a valid substitution for
$\phi$? Then $\sigma(\phi)$ would be meaningful. There would not be
much point in considering $\bar{\sigma}\circ{\cal M}(\phi)$ as a
general scheme for substituting free variables, unless it coincided
with $\sigma(\phi)$ in the particular case when $\sigma$ is valid
for $\phi$. Since $\sigma(\phi)\in{\bf P}(W)$ while
$\bar{\sigma}\circ{\cal M}(\phi)\in{\bf P}(\bar{W})$, we cannot
expect these formulas to match. However, after we take the minimal
transform  ${\cal M}\circ\sigma(\phi)$ we should expect it to be
equal to $\bar{\sigma}\circ{\cal M}(\phi)$. This is indeed the case,
and is the object
theorem~(\ref{logic:the:FOPL:commute:mintransform:validsub}) below.
For now, we shall establish a couple of lemmas:

\begin{lemma}\label{logic:lemma:FOPL:commute:mphi1}
Let $V,W$ be sets and $\sigma:V\to W$ be a map. Let $\phi=\forall
x\phi_{1}$ where $\phi_{1}\in\pv$ and $x\in V$ such that
$\sigma(x)\not\in\sigma(\free(\phi))$. Then for all $n\in\N$ we
have:
    \[
        \bar{\sigma}\circ [n/x]\circ{\cal
    M}(\phi_{1})= [n/\sigma(x)]\circ\bar{\sigma}\circ{\cal M}(\phi_{1})
    \]
where $\bar{\sigma}:\bar{V}\to\bar{W}$ is the minimal extension of
$\sigma: V\to W$.
\end{lemma}
\begin{proof}
Using proposition~(\ref{logic:prop:substitution:support}), we only
need to show that the mappings $\bar{\sigma}\circ [n/x]$ and
$[n/\sigma(x)]\circ\bar{\sigma}$ coincide on $\var({\cal
M}(\phi_{1}))$. So let $u\in\var({\cal
M}(\phi_{1}))\subseteq\bar{V}$. We want:
    \begin{equation}\label{logic:eqn:FOPL:commute:mphi1:1}
    \bar{\sigma}\circ [n/x](u)=[n/\sigma(x)]\circ\bar{\sigma}(u)
    \end{equation}
Since $\bar{V}$ is the disjoint union of $V$ and \N, we shall
distinguish two cases: first we assume that $u\in\N$. From
$V\cap\N=\emptyset$ we obtain $u\neq x$ and consequently
$\bar{\sigma}\circ[n/x](u)=\bar{\sigma}(u)=u$. From
$W\cap\N=\emptyset$ we obtain $u\neq\sigma(x)$ and consequently
$[n/\sigma(x)]\circ\bar{\sigma}(u)=[n/\sigma(x)](u)=u$. So
equation~(\ref{logic:eqn:FOPL:commute:mphi1:1}) is indeed satisfied.
Next we assume that $u\in V$. We shall distinguish two further
cases: first we assume that $u=x$. Then
$\bar{\sigma}\circ[n/x](u)=\bar{\sigma}(n)=n$ and
$[n/\sigma(x)]\circ\bar{\sigma}(u)=[n/\sigma(x)](\sigma(x))=n$ and
we see that equation~(\ref{logic:eqn:FOPL:commute:mphi1:1}) is again
satisfied. Next we assume that $u\neq x$. Then
$\bar{\sigma}\circ[n/x](u)=\bar{\sigma}(u)=\sigma(u)$, and
furthermore
$[n/\sigma(x)]\circ\bar{\sigma}(u)=[n/\sigma(x)](\sigma(u))$. In
order to establish equation~(\ref{logic:eqn:FOPL:commute:mphi1:1}),
it is therefore sufficient to prove that $\sigma(u)\neq\sigma(x)$.
However, since $u\in\var({\cal M}(\phi_{1}))$ and $u\in V$, it
follows from
proposition~(\ref{logic:prop:FOPL:mintransform:variables}) that
$u\in\free(\phi_{1})$. Having assumed that $u\neq x$ we in fact have
$u\in\free(\forall x\phi_{1})=\free(\phi)$. Having assumed that
$\sigma(x)\not\in\sigma(\free(\phi))$ it follows that
$\sigma(u)\neq\sigma(x)$ as requested.
\end{proof}
\begin{lemma}\label{logic:lemma:FOPL:commute:n:equivalence}
Let $V,W$ be sets and $\sigma:V\to W$ be a map. Let $\phi=\forall
x\phi_{1}$ where $\phi_{1}\in\pv$ and $x\in V$ such that
$\sigma(x)\not\in\sigma(\free(\phi))$. Then for all $k\in\N$ we
have:
    \[
    \mbox{$[k/x]$ valid for ${\cal M}(\phi_{1})$}\ \Leftrightarrow
    \ \mbox{$[k/\sigma(x)]$ valid for $\bar{\sigma}\circ{\cal M}(\phi_{1})$}
    \]
where $\bar{\sigma}:\bar{V}\to\bar{W}$ is the minimal extension of
$\sigma: V\to W$.
\end{lemma}
\begin{proof}
First we show $\Leftarrow$\,: So we assume that $[k/\sigma(x)]$ is
valid for $\bar{\sigma}\circ{\cal M}(\phi_{1})$. We need to show
that $[k/x]$ is valid for ${\cal M}(\phi_{1})$. However, we know
from proposition~(\ref{logic:def:FOPL:commute:minextension:valid})
that $\bar{\sigma}$ is valid for ${\cal M}(\phi_{1})$. Hence, from
the validity of $[k/\sigma(x)]$ for $\bar{\sigma}\circ{\cal
M}(\phi_{1})$ and
proposition~(\ref{logic:prop:FOPL:valid:composition}) we see that
$[k/\sigma(x)]\circ\bar{\sigma}$ is valid for ${\cal M}(\phi_{1})$.
Furthermore, having assumed $\sigma(x)\not\in\sigma(\free(\phi))$,
we can use lemma~(\ref{logic:lemma:FOPL:commute:mphi1}) to obtain:
    \begin{equation}\label{logic:eqn:FOPL:commute:n:equiv:1}
    \bar{\sigma}\circ [k/x]\circ{\cal
    M}(\phi_{1})= [k/\sigma(x)]\circ\bar{\sigma}\circ{\cal M}(\phi_{1})
    \end{equation}
It follows from proposition~(\ref{logic:prop:FOPL:validsub:image})
that $\bar{\sigma}\circ[k/x]$ is valid for ${\cal M}(\phi_{1})$, and
in particular, using
proposition~(\ref{logic:prop:FOPL:valid:composition}) once more, we
conclude that $[k/x]$ is valid for ${\cal M}(\phi_{1})$ as
requested. We now show $\Rightarrow$\,: so we assume that $[k/x]$ is
valid for ${\cal M}(\phi_{1})$. We need to show that $[k/\sigma(x)]$
is valid for $\bar{\sigma}\circ{\cal M}(\phi_{1})$. However, from
proposition~(\ref{logic:prop:FOPL:valid:composition}) it is
sufficient to prove that $[k/\sigma(x)]\circ\bar{\sigma}$ is valid
for ${\cal M}(\phi_{1})$. Using
equation~(\ref{logic:eqn:FOPL:commute:n:equiv:1}) and
proposition~(\ref{logic:prop:FOPL:validsub:image}) once more, we
simply need to show that $\bar{\sigma}\circ[k/x]$ is valid for
${\cal M}(\phi_{1})$. Having assumed that $[k/x]$ is valid for
${\cal M}(\phi_{1})$, from
proposition~(\ref{logic:prop:FOPL:valid:composition}) it is
sufficient to show that $\bar{\sigma}$ is valid for $[k/x]\circ{\cal
M}(\phi_{1})$. We shall do so with an application of
proposition~(\ref{logic:prop:FOPL:validsub:minimalextension}) to
$V_{0}=\N$. First we need to check that $\bound(\,[k/x]\circ{\cal
M}(\phi_{1})\,)\subseteq\N$. However, from
proposition~(\ref{logic:prop:boundvar:of:substitution}) we have
$\bound(\,[k/x]\circ{\cal M}(\phi_{1})\,) = [k/x](\,\bound({\cal
M}(\phi_{1}))\,)$ and from
proposition~(\ref{logic:prop:FOPL:mintransform:variables:bound}) we
have $\bound({\cal M}(\phi_{1}))\subseteq\N$. So the desired
inclusion follows. Next we need to show that $\bar{\sigma}_{|\N}$ is
injective which is clear from
definition~(\ref{logic:def:FOPL:commute:minextensioon:map}).
Finally, we need to check that
$\bar{\sigma}(\N)\cap\bar{\sigma}(\,\var([k/x]\circ{\cal
M}(\phi_{1}))\setminus\N\,)=\emptyset$. This follows from
$\bar{\sigma}(\N)\subseteq\N$ and
$\bar{\sigma}(\,\var([k/x]\circ{\cal
M}(\phi_{1}))\setminus\N\,)\subseteq\bar{\sigma}(V)=\sigma(V)\subseteq
W$, while $W\cap\N = \emptyset$.
\end{proof}

\index{minimal@Minimal transform commutes}
\begin{theorem}\label{logic:the:FOPL:commute:mintransform:validsub}
Let $V$ and $W$ be sets. Let $\sigma:V\to W$ be a map. Let
$\phi\in\pv$. If $\sigma$ is valid for $\phi$, then it commutes with
minimal transforms, specifically:
    \begin{equation}\label{logic:eqn:FOPL:commute:mintransform:validsub}
    \bar{\sigma}\circ{\cal M}(\phi)={\cal M}\circ\sigma(\phi)
    \end{equation}
where $\bar{\sigma}:\bar{V}\to\bar{W}$ is the minimal extension of
$\sigma: V\to W$.
\end{theorem}
\begin{proof}
Before we start, it should be noted that $\sigma$ in
equation~(\ref{logic:eqn:FOPL:commute:mintransform:validsub}) refers
to the substitution mapping $\sigma:\pv\to{\bf P}(W)$ while the
${\cal M}$ on the right-hand-side refers to the minimal transform
mapping ${\cal M}:{\bf P}(W)\to{\bf P}(\bar{W})$. The other ${\cal
M}$ which appears on the left-hand-side refers to the minimal
transform mapping ${\cal M}:\pv\to\pvb$, and $\bar{\sigma}$ is the
substitution mapping $\bar{\sigma}:{\bf P}(\bar{V})\to{\bf
P}(\bar{W})$. So everything makes sense. For all $\phi\in\pv$, we
need to show the property:
    \[
    (\mbox{$\sigma$ valid for $\phi$})\ \Rightarrow\
    \bar{\sigma}\circ{\cal M}(\phi)={\cal M}\circ\sigma(\phi)
    \]
We shall do so by structural induction using
theorem~(\ref{logic:the:proof:induction}) of
page~\pageref{logic:the:proof:induction}. First we assume that
$\phi=(x\in y)$ where $x,y\in V$. Then any $\sigma$ is valid for
$\phi$ and we need to show
equation~(\ref{logic:eqn:FOPL:commute:mintransform:validsub}). This
goes as follows:
    \begin{eqnarray*}
    \bar{\sigma}\circ{\cal M}(\phi)
    &=&\bar{\sigma}\circ{\cal M}(x\in y)\\
    &=&\bar{\sigma}(x\in y)\\
    &=&\bar{\sigma}(x)\in\bar{\sigma}(y)\\
    &=&\sigma(x)\in\sigma(y)\\
    &=&{\cal M}(\sigma(x)\in\sigma(y))\\
    &=&{\cal M}(\sigma(x\in y))\\
    &=&{\cal M}\circ\sigma(x\in y)\\
    &=&{\cal M}\circ\sigma(\phi)
    \end{eqnarray*}
Next we assume that $\phi=\bot$. Then we have
$\bar{\sigma}\circ{\cal M}(\phi)=\bot={\cal M}\circ\sigma(\phi)$ and
the property is also true. Next we assume that
$\phi=\phi_{1}\to\phi_{2}$ where the property is true for
$\phi_{1},\phi_{2}\in\pv$. We need to show the property is also true
for $\phi$. So we assume that $\sigma$ is valid for $\phi$. We need
to show
equation~(\ref{logic:eqn:FOPL:commute:mintransform:validsub}) holds
for $\phi$. However, from
proposition~(\ref{logic:prop:FOPL:valid:recursion:imp}), the
substitution $\sigma$ is valid for both $\phi_{1}$ and $\phi_{2}$.
Having assumed the property is true for $\phi_{1}$ and $\phi_{2}$,
it follows that
equation~(\ref{logic:eqn:FOPL:commute:mintransform:validsub}) holds
for $\phi_{1}$ and $\phi_{2}$. Hence, we have:
    \begin{eqnarray*}
    \bar{\sigma}\circ{\cal M}(\phi)
    &=&\bar{\sigma}\circ{\cal M}(\phi_{1}\to\phi_{2})\\
    &=&\bar{\sigma}({\cal M}(\phi_{1})\to{\cal M}(\phi_{2}))\\
    &=&\bar{\sigma}({\cal M}(\phi_{1}))\to\bar{\sigma}({\cal
    M}(\phi_{2}))\\
    &=&{\cal M}(\sigma(\phi_{1}))\to{\cal
    M}(\sigma(\phi_{2}))\\
    &=&{\cal M}(\sigma(\phi_{1})\to\sigma(\phi_{2}))\\
    &=&{\cal M}(\sigma(\phi_{1}\to\phi_{2}))\\
    &=&{\cal M}\circ\sigma(\phi)
    \end{eqnarray*}
Finally we assume that $\phi=\forall x\phi_{1}$ where $x\in V$ and
the property is true for $\phi_{1}\in\pv$. We need to show the
property is also true for $\phi$. So we assume that $\sigma$ is
valid for $\phi$. We need to show
equation~(\ref{logic:eqn:FOPL:commute:mintransform:validsub}) holds
for $\phi$. However, from
proposition~(\ref{logic:prop:FOPL:valid:recursion:quant}), the
substitution $\sigma$ is also valid for $\phi_{1}$. Having assumed
the property is true for $\phi_{1}$, it follows that
equation~(\ref{logic:eqn:FOPL:commute:mintransform:validsub}) holds
for $\phi_{1}$. Hence:
    \begin{eqnarray*}
    \bar{\sigma}\circ{\cal M}(\phi)&=&\bar{\sigma}\circ{\cal M}(\forall
    x\phi_{1})\\
    \mbox{$n=\min\{k:[k/x]\mbox{ valid for }{\cal M}(\phi_{1})\}$}\
    \rightarrow
    &=&\bar{\sigma}(\,\forall n{\cal M}(\phi_{1})[n/x]\,)\\
    &=&\forall \bar{\sigma}(n)\bar{\sigma}(\,{\cal
    M}(\phi_{1})[n/x]\,)\\
    &=&\forall n\,\bar{\sigma}\circ [n/x]\circ{\cal
    M}(\phi_{1})\\
    \mbox{A: to be proved}\ \rightarrow
    &=&\forall n\,[n/\sigma(x)]\circ\bar{\sigma}\circ{\cal M}(\phi_{1})\\
    &=&\forall n\,[n/\sigma(x)]\circ{\cal M}\circ\sigma(\phi_{1})\\
    &=&\forall n\,{\cal M}[\sigma(\phi_{1})][n/\sigma(x)]\\
    \mbox{B: to be proved}\ \rightarrow
    &=&\forall m\,{\cal M}[\sigma(\phi_{1})][m/\sigma(x)]\\
    \mbox{$m=\min\{k:[k/\sigma(x)]\mbox{ valid for }{\cal M}[\sigma(\phi_{1})]\}$}\
    \rightarrow
    &=&{\cal M}(\forall\sigma(x)\sigma(\phi_{1}))\\
    &=&{\cal M}(\sigma(\forall x\phi_{1}))\\
    &=&{\cal M}\circ\sigma(\phi)
    \end{eqnarray*}
So we have two more points A and B to justify. First we deal with
point A. It is sufficient for us to prove the equality:
    \[
    \bar{\sigma}\circ [n/x]\circ{\cal
    M}(\phi_{1})= [n/\sigma(x)]\circ\bar{\sigma}\circ{\cal M}(\phi_{1})
    \]
Using lemma~(\ref{logic:lemma:FOPL:commute:mphi1}) it is sufficient
to show that $\sigma(x)\not\in\sigma(\free(\phi))$. So suppose to
the contrary that $\sigma(x)\in\sigma(\free(\phi))$. Then there
exists $u\in\free(\phi)$ such that $\sigma(u)=\sigma(x)$. This
contradicts
proposition~(\ref{logic:prop:FOPL:valid:recursion:quant}) and the
fact that $\sigma$ is valid for $\phi$, which completes the proof of
point A. So we now turn to point B. We need to prove that $n=m$, for
which it is sufficient to show the equivalence:
    \[
    \mbox{$[k/x]$ valid for ${\cal M}(\phi_{1})$}\ \Leftrightarrow
    \ \mbox{$[k/\sigma(x)]$ valid for ${\cal M}[\sigma(\phi_{1})]$}
    \]
This follows from
lemma~(\ref{logic:lemma:FOPL:commute:n:equivalence}) and the fact
that  $\sigma(x)\not\in\sigma(\free(\phi))$, together with the
induction hypothesis $\bar{\sigma}\circ{\cal M}(\phi_{1})={\cal
M}\circ\sigma(\phi_{1})$.
\end{proof}
