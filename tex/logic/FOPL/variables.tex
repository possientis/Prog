Let $V$ be a set and $x,y\in V$. Then $\phi=\forall x(x\in y)$ is an
element of \pv. In this section, we aim to formalize the idea that
the {\em set of variables} of $\phi$ is $\{x,y\}$. Hence we want to
define a map $\var:\pv\to{\cal P}(V)$ such that
$\var(\phi)=\{x,y\}$.
\index{variable@Variables of
formula}\index{variable@$\var(\phi)\,$: set of variables of $\phi$}
\begin{defin}\label{logic:def:variable}
Let $V$ be a set. The map $\var:\pv\to {\cal P}(V)$ defined by the
following structural recursion is called {\em variable mapping on
\pv}:
 \begin{equation}\label{logic:eqn:var:recursion}
    \forall\phi\in\pv\ ,\ \var(\phi)=\left\{
                    \begin{array}{lcl}
                    \{x,y\}&\mbox{\ if\ }&\phi=(x\in y)\\
                    \emptyset&\mbox{\ if\ }&\phi=\bot\\
                    \var(\phi_{1})\cup\var(\phi_{2}) &\mbox{\ if\ }&\phi=\phi_{1}\to\phi_{2}\\
                    \{x\}\cup\var(\phi_{1})&\mbox{\ if\ }&\phi=\forall x\phi_{1}
                    \end{array}\right.
    \end{equation}
We say that $x\in V$ is a {\em variable} of $\phi\in\pv$ \ifand\ $x\in\var(\phi)$.
\end{defin}
\begin{prop}\label{logic:prop:variable}
The structural recursion of {\em definition~(\ref{logic:def:variable})} is legitimate.
\end{prop}
\begin{proof}
We need to show the existence and uniqueness of $\var:\pv\to{\cal
P}(V)$ satisfying equation~(\ref{logic:eqn:var:recursion}). This
follows from an immediate application of
theorem~(\ref{logic:the:structural:recursion}) of
page~\pageref{logic:the:structural:recursion} to the free universal
algebra \pv\ and the set $A={\cal P}(V)$, using $g_{0}:\pvo\to A$
defined by $g_{0}(x\in y)=\{x,y\}$ for all $x,y\in V$, and the
operators $h(f):A^{\alpha(f)}\to A$ ($f\in\alpha$) defined for all
$V_{1},V_{2}\in A$ and $x\in V$ as:
    \begin{eqnarray*}
    (i)&&h(\bot)(0)=\emptyset\\
    (ii)&&h(\to)(V_{1},V_{2})=V_{1}\cup V_{2}\\
    (iii)&&h(\forall x)(V_{1})=\{x\}\cup V_{1}
    \end{eqnarray*}
\end{proof}

Recall that a set $A$ is said to be finite \ifand\ there exists a
bijection $j:n\to A$ for some $n\in\N$. The set $A$ is said to be
infinite if it is not finite. Note that if $A$ and $B$ are finite
sets, then $A\cup B$ is also finite. We shall accept this fact
without proof in this document. The following proposition shows that
a formula of first order predicate logic always has a finite number
of variables.
\begin{prop}\label{logic:prop:var:is:finite}
Let $V$ be a set and $\phi\in\pv$. Then $\var(\phi)$ is finite.
\end{prop}
\begin{proof}
Given $\phi\in\pv$, we need to show that $\var(\phi)$ is a finite
set. We shall do so by structural induction using
theorem~(\ref{logic:the:proof:induction}) of
page~\pageref{logic:the:proof:induction}. Since \pvo\ is a generator
of \pv, we first check that the property is true for $\phi\in\pvo$.
So suppose $\phi=(x\in y)$ for some $x,y\in V$. Since
$\var(\phi)=\{x,y\}$, it is indeed a finite set. Next we check that
the property is true for $\bot\in\pv$, which is clear since
$\var(\bot)=\emptyset$. Next we check that the property is true for
$\phi=\phi_{1}\to\phi_{2}$, if it is true for
$\phi_{1},\phi_{2}\in\pv$. This follows immediately from
$\var(\phi)=\var(\phi_{1})\cup\var(\phi_{2})$ which is clearly a
finite set if both $\var(\phi_{1})$ and $\var(\phi_{2})$ are finite.
Finally we check that the property is true for $\phi=\forall
x\phi_{1}$ if it is true for $\phi_{1}$. This is also immediate from
$\var(\phi)=\{x\}\cup\var(\phi_{1})$ and the fact that
$\var(\phi_{1})$ is finite.
\end{proof}

In the following proposition we show that if $\psi\preceq\phi$ is a
sub-formula of $\phi$, then the variables of $\psi$ are also
variables of $\phi$ as we should expect:
\index{subformula@$\phi\preceq\psi\,$: $\phi$ is a sub-formula of
$\psi$}
\begin{prop}\label{logic:prop:FOBL:variable:subformula}
Let $V$ be a set and $\phi,\psi\in\pv$. Then we have:
    \[
    \psi\preceq\phi\ \Rightarrow\ \var(\psi)\subseteq\var(\phi)
    \]
\end{prop}
\begin{proof}
This is a simple application of
proposition~(\ref{logic:prop:UA:subformula:non:decreasing}) to
$\var:X\to A$ where $X=\pv$ and $A={\cal P}(V)$ where the preorder
$\leq$ on $A$ is the usual inclusion $\subseteq$. We simply need to
check that given $\phi_{1},\phi_{2}\in\pv$ and $x\in V$ we have the
inclusions $\var(\phi_{1})\subseteq\var(\phi_{1}\to\phi_{2})$,
$\var(\phi_{2})\subseteq\var(\phi_{1}\to\phi_{2})$ and
$\var(\phi_{1})\subseteq\var(\forall x\phi_{1})$ which follow
immediately from the recursive
definition~(\ref{logic:def:variable}).
\end{proof}

Let $V,W$ be sets and $\sigma:V\to W$ be a map with associated
substitution mapping $\sigma:\pv\to{\bf P}(W)$. Given $\phi\in\pv$,
the variables of $\phi$ are the elements of the set $\var(\phi)$.
The following proposition allows us to determine which are the
variables of $\sigma(\phi)$. Specifically:
    \[
    \var(\sigma(\phi))=\{\sigma(x)\ :\ x\in\var(\phi)\}
    \]
In other words the variables of $\sigma(\phi)$ coincide with the
range of the restriction $\sigma_{|\var(\phi)}$. As discussed in
page~\pageref{logic:lemma:pullback}, this range is denoted
$\sigma(\var(\phi))$.
\begin{prop}\label{logic:prop:var:of:substitution}
Let $V$ and $W$ be sets and $\sigma:V\to W$ be a map. Then:
    \[
    \forall\phi\in\pv\ ,\ \var(\sigma(\phi))=\sigma(\var(\phi))
    \]
where $\sigma:\pv\to{\bf P}(W)$ also denotes the associated substitution mapping.
\end{prop}
\begin{proof}
Given $\phi\in\pv$, we need to show that
$\var(\sigma(\phi))=\{\sigma(x):x\in\var(\phi)\}$. We shall do so by
structural induction using theorem~(\ref{logic:the:proof:induction})
of page~\pageref{logic:the:proof:induction}. Since \pvo\ is a
generator of \pv, we first check that the property is true for
$\phi\in\pvo$. So suppose $\phi=(x\in y)\in\pvo$ for some $x,y\in
V$. Then we have:
    \begin{eqnarray*}
    \var(\sigma(\phi))&=&\var(\sigma(x\in y))\\
    &=&\var(\,\sigma(x)\in\sigma(y)\,)\\
    &=&\{\sigma(x),\sigma(y)\}\\
    &=&\{\sigma(u):u\in\{x,y\}\}\\
    &=&\{\sigma(u):u\in\var(x\in y)\}\\
    &=&\{\sigma(x):x\in\var(\phi)\}
    \end{eqnarray*}
Next we check that the property is true for $\bot\in\pv$:
    \[
    \var(\sigma(\bot))=\var(\bot)=\emptyset=\{\sigma(x):x\in\var(\bot)\}
    \]
Next we check that the property is true for
$\phi=\phi_{1}\to\phi_{2}$ if it is true for $\phi_{1},\phi_{2}$:
    \begin{eqnarray*}
    \var(\sigma(\phi))&=&\var(\sigma(\phi_{1}\to\phi_{2}))\\
    &=&\var(\sigma(\phi_{1})\to\sigma(\phi_{2}))\\
    &=&\var(\sigma(\phi_{1}))\cup\var(\sigma(\phi_{2}))\\
    &=&\{\,\sigma(x):x\in\var(\phi_{1})\,\}\cup\{\,\sigma(x):x\in\var(\phi_{2})\,\}\\
    &=&\{\,\sigma(x):x\in\var(\phi_{1})\cup\var(\phi_{2})\,\}\\
    &=&\{\,\sigma(x):x\in\var(\phi_{1}\to\phi_{2})\,\}\\
    &=&\{\,\sigma(x):x\in\var(\phi)\,\}
    \end{eqnarray*}
Finally we check that the property is true for $\phi=\forall
x\phi_{1}$ if it is true for $\phi_{1}$:
    \begin{eqnarray*}
    \var(\sigma(\phi))&=&\var(\sigma(\forall x\phi_{1}))\\
    &=&\var(\forall\sigma(x)\,\sigma(\phi_{1}))\\
    &=&\{\,\sigma(x)\,\}\cup\var(\sigma(\phi_{1}))\\
    &=&\{\,\sigma(x)\,\}\cup\{\,\sigma(u):u\in\var(\phi_{1})\,\}\\
    &=&\{\,\sigma(u):u\in\{x\}\cup\var(\phi_{1})\,\}\\
    &=&\{\,\sigma(u):u\in\var(\forall x\phi_{1})\,\}\\
    &=&\{\,\sigma(x):x\in\var(\phi)\,\}
    \end{eqnarray*}
\end{proof}

Let $V,W$ be sets and $\sigma:V\to W$ be a map with associated
substitution mapping $\sigma:\pv\to{\bf P}(W)$. It will soon be
apparent that whether or not $\sigma:V\to W$ is injective plays a
crucial role. For example if $\phi=\forall x\forall y(x\in y)$ with
$x\neq y$ and $\sigma(x)=u$ while $\sigma(y)=v$, we obtain
$\sigma(\phi)=\forall u\forall v(u\in v)$. When $\sigma$ is
injective we have $u\neq v$ and it is possible to argue in some
sense that the {\em meaning} of $\sigma(\phi)$ is the same as that
of $\phi$. If $\sigma$ is not an injective map, then we no longer
have the guarantee that $u\neq v$. In that case $\sigma(\phi)$ may
be equal to $\forall u\forall u(u\in u)$, which is a different
mathematical statement from that of $\phi$. Substitution mappings
which arise from an injective map are therefore important. However,
there will be cases when we shall need to consider a map $\tau:V\to
W$ which is not injective. One common example is the map $[y/x]:V\to
V$ defined in definition~(\ref{logic:def:single:var:substitution})
which replaces a variable $x$ by a variable $y$, but leaves the
variable $y$ as it is, if already present in the formula. Since
$[y/x](x)=y=[y/x](y)$, this map is not injective when $x\neq y$.
This map is not the same as the map $[y\!:\!x]:V\to V$ defined in
definition~(\ref{logic:def:single:var:permutation}) which permutes
the two variables $x$ and $y$. In contrast to $[y/x]$ the
permutation $[y:x]$ is an injective map. However if a formula
$\phi\in\pv$ does not contain the variable $y$, it is easy to
believe that the associated substitution mappings $[y/x]:\pv\to\pv$
and $[y\!:\!x]:\pv\to\pv$ have an identical effect on the formula
$\phi$, or in other words that $[y/x](\phi)=[y\!:\!x](\phi)$. If
this is the case, many things can be said about the formula
$[y/x](\phi)$ despite the fact that $[y/x]$ is not an injective map,
simply because of the equality $[y/x](\phi)=[y\!:\!x](\phi)$. More
generally, many things can be said about a formula $\tau(\phi)$
whenever it coincides with a formula $\sigma(\phi)$, and $\sigma$ is
an injective map. To achieve the equality $\sigma(\phi)=\tau(\phi)$
we only need $\sigma$ and $\tau$ to coincide on $\var(\phi)$ as we
shall now check in the following proposition.

\begin{prop}\label{logic:prop:substitution:support}
Let $V$ and $W$ be sets and $\sigma,\tau:V\to W$ be two maps. Then:
    \[
    \sigma_{|\var(\phi)}=\tau_{|\var(\phi)}\ \ \Leftrightarrow\
    \ \sigma(\phi)=\tau(\phi)
    \]
for all $\phi\in\pv$, where $\sigma,\tau:\pv\to{\bf P}(W)$ are also
the substitution mappings.
\end{prop}
\begin{proof}
Given $\phi\in\pv$, we need to show that if $\sigma$ and $\tau$
coincide on $\var(\phi)\subseteq V$, then $\sigma(\phi)=\tau(\phi)$,
and conversely that if $\sigma(\phi)=\tau(\phi)$, then $\sigma$ and
$\tau$ coincide on $\var(\phi)$. We shall do so by structural
induction using theorem~(\ref{logic:the:proof:induction}) of
page~\pageref{logic:the:proof:induction}. Since \pvo\ is a generator
of \pv, we first check that the equivalence is true for
$\phi\in\pvo$. So suppose $\phi=(x\in y)$ for some $x,y\in V$, and
suppose furthermore that $\sigma_{|\var(\phi)}=\tau_{|\var(\phi)}$.
Since $\var(\phi)=\{x,y\}$ it follows that $\sigma(x)=\tau(x)$ and
$\sigma(y)=\tau(y)$. Hence, we have the following equalities:
    \begin{eqnarray*}
    \sigma(\phi)&=&\sigma(x\in y)\\
    &=&(\,\sigma(x)\in\sigma(y)\,)\\
    &=&(\,\tau(x)\in\tau(y)\,)\\
    &=&\tau(x\in y)\\
    &=&\tau(\phi)
    \end{eqnarray*}
Conversely, if $\sigma(\phi)=\tau(\phi)$ then
$(\,\sigma(x)\in\sigma(y)\,)=(\,\tau(x)\in\tau(y)\,)$ and it follows
that $\sigma(x)=\tau(x)$ and $\sigma(y)=\tau(y)$. So $\sigma$ and
$\tau$ coincide on $\var(\phi)$. Next we check that the equivalence
is true for $\bot\in\pv$. Since $\var(\bot)=\emptyset$, we always
have the equality
$\sigma_{|\var(\bot)}=\emptyset=\tau_{|\var(\bot)}$. Furthermore
$\sigma(\bot)=\bot=\tau(\bot)$ is also always true. So the
equivalence does indeed hold. Next we check that the equivalence is
true for $\phi=\phi_{1}\to\phi_{2}$, if it is true for
$\phi_{1},\phi_{2}\in\pv$. So we assume that $\sigma$ and $\tau$
coincide on $\var(\phi)$. We need to prove that
$\sigma(\phi)=\tau(\phi)$. However, since
$\var(\phi)=\var(\phi_{1})\cup\var(\phi_{2})$, $\sigma$~and $\tau$
coincide both on $\var(\phi_{1})$ and $\var(\phi_{2})$. Having
assumed the property is true for $\phi_{1}$ and $\phi_{2}$ it
follows that $\sigma(\phi_{1})=\tau(\phi_{1})$ and
$\sigma(\phi_{2})=\tau(\phi_{2})$. Hence:
    \begin{eqnarray*}
    \sigma(\phi)&=&\sigma(\phi_{1}\to\phi_{2})\\
    &=&\sigma(\phi_{1})\to\sigma(\phi_{2})\\
    &=&\tau(\phi_{1})\to\tau(\phi_{2})\\
    &=&\tau(\phi_{1}\to\phi_{2})\\
    &=&\tau(\phi)
    \end{eqnarray*}
Conversely if $\sigma(\phi)=\tau(\phi)$ then we obtain
$\sigma(\phi_{1})\to\sigma(\phi_{2})=\tau(\phi_{1})\to\tau(\phi_{2})$.
Using theorem~(\ref{logic:the:unique:representation}) of
page~\pageref{logic:the:unique:representation} it follows that
$\sigma(\phi_{1})=\tau(\phi_{1})$ and
$\sigma(\phi_{2})=\tau(\phi_{2})$. Having assumed the equivalence is
true for $\phi_{1}$ and $\phi_{2}$ we conclude that $\sigma$ and
$\tau$ coincide on $\var(\phi_{1})$ and $\var(\phi_{2})$ and
consequently they coincide on $\var(\phi)$. Finally we check that
the equivalence is true for $\phi=\forall x\phi_{1}$ if it is true
for $\phi_{1}$. So we assume that $\sigma$ and $\tau$ coincide on
$\var(\phi)$. We need to show $\sigma(\phi)=~\tau(\phi)$. However,
since $\var(\phi)=\{x\}\cup\var(\phi_{1})$, $\sigma$ and $\tau$
coincide on $\var(\phi_{1})$ and we have $\sigma(x)=\tau(x)$. Having
assumed the property is true for $\phi_{1}$ it follows that
$\sigma(\phi_{1})=\tau(\phi_{1})$, and consequently:
\begin{eqnarray*}
    \sigma(\phi)&=&\sigma(\forall x\phi_{1})\\
    &=&\forall\sigma(x)\,\sigma(\phi_{1})\\
    &=&\forall\tau(x)\,\tau(\phi_{1})\\
    &=&\tau(\forall x\phi_{1})\\
    &=&\tau(\phi)
    \end{eqnarray*}
Conversely if $\sigma(\phi)=\tau(\phi)$ then we obtain
$\forall\sigma(x)\,\sigma(\phi_{1})=\forall\tau(x)\,\tau(\phi_{1})$.
Using theorem~(\ref{logic:the:unique:representation}) of
page~\pageref{logic:the:unique:representation} it follows that
$\sigma(\phi_{1})=\tau(\phi_{1})$ and $\sigma(x)=\tau(x)$. Having
assumed the equivalence is true for $\phi_{1}$ we conclude that
$\sigma$ and $\tau$ coincide on $\var(\phi_{1})$ and consequently
they coincide on $\var(\phi)=\{x\}\cup\var(\phi_{1})$.
\end{proof}
