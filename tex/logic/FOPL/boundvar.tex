As already noted, the variable $x$ in $\forall x(x\in y)$ or the
integral $\int\! f(x,y)\,dx$ is a dummy variable. In mathematical
logic, dummy variables are called {\em bound variables}. For the
sake of completeness, we now formally define the set of bound
variables of a formula $\phi$, i.e. the set of all variables which
appear in some quantifier $\forall x$ within the formula $\phi$. It
should be noted that a variable $x$ can be both a bound variable and
a free variable, as is the case for the formula:
    \[
    \phi=\forall x(x\in y)\to (x\in y)
    \]
\index{bound@Bound variables of formula}\index{bound@$\bound(\phi)$
: bound variables of $\phi$}
\begin{defin}\label{logic:def:bound:variable}
    Let $V$ be a set. The map $\bound:\pv\to {\cal P}(V)$ defined by the
    following structural recursion is called {\em bound variable mapping on \pv}:
    \begin{equation}\label{logic:eqn:bound:var:recursion}
        \forall\phi\in\pv\ ,\ \bound(\phi)=
            \left\{\begin{array}{lcl}
                \emptyset&\mbox{\ if\ }&\phi=(x\in y)
                \\
                \emptyset&\mbox{\ if\ }&\phi=\bot
                \\
                \bound(\phi_{1})\cup\bound(\phi_{2}) 
                    &\mbox{\ if\ }&
                \phi=\phi_{1}\to\phi_{2}
                \\
                \{x\}\cup\bound(\phi_{1})&\mbox{\ if\ }&\phi=\forall x\phi_{1}
            \end{array}\right.
    \end{equation}
    We say that $x\in V$ is a {\em bound variable} of $\phi\in\pv$ \ifand\ 
    $x\in\bound(\phi)$.
\end{defin}
\begin{prop}\label{logic:prop:bound:variable}
    The structural recursion of {\em definition~(\ref{logic:def:bound:variable})} 
    is legitimate.
\end{prop}
\begin{proof}
We need to show the existence and uniqueness of $\bound:\pv\to{\cal
P}(V)$ satisfying equation~(\ref{logic:eqn:bound:var:recursion}).
This follows from an immediate application of
theorem~(\ref{logic:the:structural:recursion}) of
page~\pageref{logic:the:structural:recursion} to the free universal
algebra \pv\ and the set $A={\cal P}(V)$, using $g_{0}:\pvo\to A$
defined by $g_{0}(x\in y)=\emptyset$ for all $x,y\in V$, and the
operators $h(f):A^{\alpha(f)}\to A$ ($f\in\alpha$) defined for all
$V_{1},V_{2}\in A$ and $x\in V$ as:
    \begin{eqnarray*}
    (i)&&h(\bot)(0)=\emptyset\\
    (ii)&&h(\to)(V_{1},V_{2})=V_{1}\cup V_{2}\\
    (iii)&&h(\forall x)(V_{1})=\{x\}\cup V_{1}
    \end{eqnarray*}
\end{proof}

Given a set $V$ and $\phi\in\pv$, the intersection
$\free(\phi)\cap\bound(\phi)$ is not empty in general. The formula
$\phi=(x\in y)\to\forall x(x\in y)$ is an example where such
intersection is not empty. However:

\begin{prop}\label{logic:prop:FOPL:boundvar:free}
Let $V$ be a set and $\phi\in\pv$. Then we have:
    \begin{equation}\label{logic:eqn:FOPL:boundvar:free:1}
    \var(\phi)=\free(\phi)\cup\bound(\phi)
    \end{equation}
\end{prop}
\begin{proof}
We shall prove the inclusion~(\ref{logic:eqn:FOPL:boundvar:free:1})
by structural induction using
theorem~(\ref{logic:the:proof:induction}) of
page~\pageref{logic:the:proof:induction}. Since \pvo\ is a generator
of \pv, we first check that the inclusion is true for $\phi\in\pvo$.
So suppose $\phi=(x\in y)\in\pvo$ for some $x,y\in V$. Then:
    \begin{eqnarray*}
    \var(\phi)&=&\var(x\in y)\\
    &=&\{x,y\}\\
    &=&\{x,y\}\cup\emptyset\\
    &=&\free(x\in y)\cup\bound(x\in y)\\
    &=&\free(\phi)\cup\bound(\phi)
    \end{eqnarray*}
Next we check that the property is true for $\bot\in\pv$:
    \[
    \var(\bot)=\emptyset=\free(\bot)\cup\bound(\bot)
    \]
Next we check that the property is true for
$\phi=\phi_{1}\to\phi_{2}$ if it is true for $\phi_{1},\phi_{2}$:
    \begin{eqnarray*}
    \var(\phi)&=&\var(\phi_{1}\to\phi_{2})\\
    &=&\var(\phi_{1})\cup\var(\phi_{2})\\
    &=&\free(\phi_{1})\cup\bound(\phi_{1})\cup\free(\phi_{2})\cup\bound(\phi_{2})\\
    &=&\free(\phi_{1}\to\phi_{2})\cup\bound(\phi_{1}\to\phi_{2})\\
    &=&\free(\phi)\cup\bound(\phi)
    \end{eqnarray*}
Finally we check that the property is true for $\phi=\forall
x\phi_{1}$ if it is true for $\phi_{1}$:
    \begin{eqnarray*}
    \var(\phi)&=&\var(\forall x\phi_{1})\\
    &=&\{x\}\cup\var(\phi_{1})\\
    &=&\{x\}\cup\free(\phi_{1})\cup\bound(\phi_{1})\\
    &=&\{x\}\cup(\,\free(\phi_{1})\setminus\{x\}\,)\cup\bound(\phi_{1})\\
    &=&\free(\forall x\phi_{1})\cup\bound(\forall x\phi_{1})\\
    &=&\free(\phi)\cup\bound(\phi)
    \end{eqnarray*}
\end{proof}

In the following proposition we show that if $\psi\preceq\phi$ is a
sub-formula of $\phi$, then the bound variables of $\psi$ are also
bound variables of $\phi$. Note that this property is not true of
free variables since $\free(\phi_{1})\not\subseteq\free(\forall
x\phi_{1}))$ in general.

\begin{prop}\label{logic:prop:FOPL:boundvar:subformula}
Let $V$ be a set and $\phi,\psi\in\pv$. Then we have:
    \[
    \psi\preceq\phi\ \Rightarrow\ \bound(\psi)\subseteq\bound(\phi)
    \]
\end{prop}
\begin{proof}
This is a simple application of
proposition~(\ref{logic:prop:UA:subformula:non:decreasing}) to the
map $\bound:X\to A$ where $X=\pv$ and $A={\cal P}(V)$ where the
preorder $\leq$ on $A$ is the usual inclusion $\subseteq$. We simply
need to check that given $\phi_{1},\phi_{2}\in\pv$ and $x\in V$ we
have the inclusions
$\bound(\phi_{1})\subseteq\bound(\phi_{1}\to\phi_{2})$,
$\bound(\phi_{2})\subseteq\bound(\phi_{1}\to\phi_{2})$ and
$\bound(\phi_{1})\subseteq\bound(\forall x\phi_{1})$ which follow
immediately from definition~(\ref{logic:def:bound:variable}).
\end{proof}

Let $V,W$ be sets and $\sigma:V\to W$ be a map with associated
substitution mapping $\sigma:\pv\to{\bf P}(W)$. Given $\phi\in\pv$,
the bound variables of $\phi$ are the elements of the set
$\bound(\phi)$. The following proposition allows us to determine
which are the bound variables of $\sigma(\phi)$. Specifically:
    \[
    \bound(\sigma(\phi))=\{\sigma(x)\ :\ x\in\bound(\phi)\}
    \]
In other words the bound variables of $\sigma(\phi)$ coincide with
the range of the restriction $\sigma_{|\bound(\phi)}$. As discussed
in page~\pageref{logic:lemma:pullback}, this range is denoted
$\sigma(\bound(\phi))$.
\begin{prop}\label{logic:prop:boundvar:of:substitution}
Let $V$ and $W$ be sets and $\sigma:V\to W$ be a map. Then:
    \[
    \forall\phi\in\pv\ ,\ \bound(\sigma(\phi))=\sigma(\bound(\phi))
    \]
where $\sigma:\pv\to{\bf P}(W)$ also denotes the associated
substitution mapping.
\end{prop}
\begin{proof}
Given $\phi\in\pv$, we need to show that
$\bound(\sigma(\phi))=\{\sigma(x):x\in\bound(\phi)\}$. We shall do
so by structural induction using
theorem~(\ref{logic:the:proof:induction}) of
page~\pageref{logic:the:proof:induction}. Since \pvo\ is a generator
of \pv, we first check that the property is true for $\phi\in\pvo$.
So suppose $\phi=(x\in y)\in\pvo$ for some $x,y\in V$. Then we have:
    \begin{eqnarray*}
    \bound(\sigma(\phi))&=&\bound(\sigma(x\in y))\\
    &=&\bound(\,\sigma(x)\in\sigma(y)\,)\\
    &=&\emptyset\\
    &=&\{\sigma(u):u\in\emptyset\}\\
    &=&\{\sigma(u):u\in\bound(x\in y)\}\\
    &=&\{\sigma(x):x\in\bound(\phi)\}
    \end{eqnarray*}
Next we check that the property is true for $\bot\in\pv$:
    \[
    \bound(\sigma(\bot))=\bound(\bot)=\emptyset=\{\sigma(x):x\in\bound(\bot)\}
    \]
Next we check that the property is true for
$\phi=\phi_{1}\to\phi_{2}$ if it is true for $\phi_{1},\phi_{2}$:
    \begin{eqnarray*}
    \bound(\sigma(\phi))&=&\bound(\sigma(\phi_{1}\to\phi_{2}))\\
    &=&\bound(\sigma(\phi_{1})\to\sigma(\phi_{2}))\\
    &=&\bound(\sigma(\phi_{1}))\cup\bound(\sigma(\phi_{2}))\\
    &=&\{\,\sigma(x):x\in\bound(\phi_{1})\,\}\cup\{\,\sigma(x):x\in\bound(\phi_{2})\,\}\\
    &=&\{\,\sigma(x):x\in\bound(\phi_{1})\cup\bound(\phi_{2})\,\}\\
    &=&\{\,\sigma(x):x\in\bound(\phi_{1}\to\phi_{2})\,\}\\
    &=&\{\,\sigma(x):x\in\bound(\phi)\,\}
    \end{eqnarray*}
Finally we check that the property is true for $\phi=\forall
x\phi_{1}$ if it is true for $\phi_{1}$:
    \begin{eqnarray*}
    \bound(\sigma(\phi))&=&\bound(\sigma(\forall x\phi_{1}))\\
    &=&\bound(\forall\sigma(x)\,\sigma(\phi_{1}))\\
    &=&\{\,\sigma(x)\,\}\cup\bound(\sigma(\phi_{1}))\\
    &=&\{\,\sigma(x)\,\}\cup\{\,\sigma(u):u\in\bound(\phi_{1})\,\}\\
    &=&\{\,\sigma(u):u\in\{x\}\cup\bound(\phi_{1})\,\}\\
    &=&\{\,\sigma(u):u\in\bound(\forall x\phi_{1})\,\}\\
    &=&\{\,\sigma(x):x\in\bound(\phi)\,\}
    \end{eqnarray*}
\end{proof}
