Let $V$ be a set and $x,y,z\in V$. Then $\phi=\forall x(x\in z)$ and
$\psi=\forall y(y\in z)$ are elements of \pv. In the case when
$x\neq z$ and $y\neq z$ most of us would regard $\phi$ and $\psi$ as
being the same {\em mathematical statement}. Yet from
theorem~(\ref{logic:the:unique:representation}) of
page~\pageref{logic:the:unique:representation}, $\phi$ and $\psi$
are distinct elements of \pv\ when $x\neq y$. We are still a long
way to have a clear opinion on what an appropriate congruence on
\pv\ should be. Yet we feel that whatever our choice of congruence
$\sim$, we should have $\phi\sim\psi$. So we need to formalize the
idea that {\em replacing the variable $x$ in $\phi=\forall x(x\in
z)$ by the variable $y$} does not change the {\em meaning} of the
formula $\phi$. The first step is for us to formally define the
notion of variable substitution.

Variable substitutions is a key idea in every textbook of
mathematical logic. However, the usual treatment attempts to define
substitutions where variables are replaced by {\em terms}. Since
quantification can only occur with respect to a variable, only free
occurrences of variables are substituted with terms, leaving bound
occurrences unchanged. This creates a major drawback: if
$\sigma:V\to W$ is a map and $\phi\in\pv$, we certainly want the
substituted formula $\sigma(\phi)$ to be an element of ${\bf P}(W)$,
which would be impossible to achieve if we were to leave bound
occurrences of variables unchanged. So contrary to standard
practice, we shall define a notion of variable substitution where
every variable is {\em carried over} by $\sigma:V\to W$. Defining
$\sigma:\pv\to{\bf P}(W)$ from the map $\sigma:V\to W$ we believe is
very fruitful. This type of substitution will not be
capture-avoiding in general, but this is no different from the
general treatment where we often require a term $t$ to be {\em
substitutable} for a variable $x$ in $\phi$, or to be {\em free for
$x$} in $\phi$. Note that our formal language \pv\ has no term
beyond the variables themselves. So focussing on variables only for
substitutions is no restriction of generality. In our view, terms or
constants should not exist in a low level mathematical statement.
The constant $\pi=3.14159\ldots$ should be a low level predicate
$\pi(x)$, and given a predicate $Q(x)$ the statement $Q(\pi)$
obtained by substituting the variable $x$ by the ground term $\pi$,
should in fact be the statement $\forall x[\pi(x)\to Q(x)]$ which is
syntactically a very different operation from a substitution of
variable. \index{substitution@$\sigma\,$: a map $\sigma:V\to
W$}\index{substitution@$\sigma\,$: associated $\sigma:\pv\to{\bf
P}(W)$}\index{substitution@Substitution of
variables}\index{substitution@$\sigma(\phi)\,$: image
$\sigma:\pv\to{\bf P}(W)$}
\begin{defin}\label{logic:def:substitution}
    Let $V$ and $W$ be two sets and $\sigma:V\to W$ be a map. We call
    {\em substitution mapping between \pv\ and ${\bf P}(W)$ associated
    with $\sigma:V\to W$} the map $\sigma^{*}:\pv\to{\bf P}(W)$ defined
    by the structural recursion:
    \begin{equation}\label{logic:eqn:substitution:recursion}
        \forall\phi\in\pv\ ,\ \sigma^{*}(\phi)=\left\{
            \begin{array}{lcl}
                (\sigma(x)\in\sigma(y))
                    &\mbox{\ if\ }&
                \phi=(x\in y)
                \\
                \bot
                    &\mbox{\ if\ }&
                \phi=\bot
                \\
                \sigma^{*}(\phi_{1})\to \sigma^{*}(\phi_{2}) 
                    &\mbox{\ if\ }&
                \phi=\phi_{1}\to\phi_{2}
                \\
                \forall\sigma(x)\,\sigma^{*}(\phi_{1})
                    &\mbox{\ if\ }&
                \phi=\forall x\phi_{1}
            \end{array}\right.
    \end{equation}
\end{defin}
This is one of our first definition by {\em structural recursion}.
The principle of such definition on a free universal algebra is
justified by virtue of
theorem~(\ref{logic:the:structural:recursion}) of
page~\pageref{logic:the:structural:recursion}. As this is one of our
first use of this theorem, we shall make sure its application is
done appropriately by checking all the relevant details.
\begin{prop}\label{logic:prop:substitution}
    The structural recursion of {\em definition~(\ref{logic:def:substitution})} 
    is legitimate.
\end{prop}
\begin{proof}
    We need to show the existence and uniqueness of the map
    $\sigma^{*}:\pv\to{\bf P}(W)$ satisfying
    equation~(\ref{logic:eqn:substitution:recursion}). We shall do so by
    applying theorem~(\ref{logic:the:structural:recursion}) of
    page~\pageref{logic:the:structural:recursion} to the free universal
    algebra \pv\ with free generator $X_{0}=\pvo$ and the set $A={\bf
    P}(W)$. First we define $g_{0}:X_{0}\to A$ by $g_{0}(x\in
    y)=(\sigma(x)\in\sigma(y))$ for all $x,y\in V$. Next, given
    $f\in\alpha$ we need to define an operator $h(f):A^{\alpha(f)}\to
    A$. For all $\phi_{1},\phi_{2}\in A$ and for all $x\in V$, we set:
    \begin{eqnarray*}
        (i)&&h(\bot)(0)=\bot
        \\
        (ii)&&h(\to)(\phi_{1},\phi_{2})=\phi_{1}\to\phi_{2}
        \\
        (iii)&&h(\forall x)(\phi_{1})=\forall \sigma(x)\phi_{1}
    \end{eqnarray*}
    Applying theorem~(\ref{logic:the:structural:recursion}) of
    page~\pageref{logic:the:structural:recursion}, there exists a unique
    map $\sigma^{*}:\pv\to A$ such that $\sigma^{*}_{|X_{0}}=g_{0}$ and
    for all $f\in\alpha$ and $y\in\pv^{\alpha(f)}$:
    \begin{equation}\label{logic:eqn:substitution:recursion:proof}
        \sigma^{*}(f(y))=h(f)(\sigma^{*}(y))
    \end{equation}
    In other words, there exists a unique map $\sigma^{*}:\pv\to{\bf
    P}(W)$ such that for all $x,y\in V$ we have  $\sigma^{*}(x\in
    y)=(\sigma(x)\in\sigma(y))$  and which satisfies
    equation~(\ref{logic:eqn:substitution:recursion:proof}). Taking
    $f=\bot$ and $y=0$ in
    equation~(\ref{logic:eqn:substitution:recursion:proof}) we obtain:
    \begin{equation}\label{logic:eqn:substitution:bot}
        \sigma^{*}(\bot)=\sigma^{*}(\bot(0))=h(\bot)(0)=\bot
    \end{equation}
    Taking $f=\to$ and $y=(\phi_{1},\phi_{2})\in\pv^{2}$ we obtain:
    \begin{equation}\label{logic:eqn:substitution:to}
        \sigma^{*}(\phi_{1}\to\phi_{2})
        =
        h(\to)(\sigma^{*}(\phi_{1}),\sigma^{*}(\phi_{2}))
        =
        \sigma^{*}(\phi_{1})\to\sigma^{*}(\phi_{2})
    \end{equation}
    Finally, taking $f=\forall x$ and $y=\phi_{1}\in\pv^{1}$ given $x\in V$:
    \begin{equation}\label{logic:eqn:substitution:forall}
        \sigma^{*}(\forall x\phi_{1})
        =
        h(\forall x)(\sigma^{*}(\phi_{1}))
        =
        \forall\sigma(x)\,\sigma^{*}(\phi_{1})
    \end{equation}
    Since equations~(\ref{logic:eqn:substitution:bot}),
    (\ref{logic:eqn:substitution:to}) and
    (\ref{logic:eqn:substitution:forall}) are exactly as
    equation~(\ref{logic:eqn:substitution:recursion}), we conclude there
    is a unique map $\sigma^{*}:\pv\to{\bf P}(W)$ satisfying
    equation~(\ref{logic:eqn:substitution:recursion}).
\end{proof}
Let $U,V,W$ be sets while $\tau:U\to V$ and $\sigma:V\to W$ are
maps. From definition~(\ref{logic:def:substitution}) we obtain the
substitution mappings $\tau^{*}:{\bf P}(U)\to\pv$ and
$\sigma^{*}:\pv\to{\bf P}(W)$. However, $\sigma\circ\tau:U\to W$ is
also a map and we therefore have a substitution mapping
$(\sigma\circ\tau)^{*}:{\bf P}(U)\to{\bf P}(W)$. One natural
question to ask is whether $(\sigma\circ\tau)^{*}$ coincide with the
composition $\sigma^{*}\circ\tau^{*}$. The following proposition
shows that it is indeed the case. As a consequence, we shall be able
to simplify our notations by referring to the substitution mappings
$\tau^{*}$, $\sigma^{*}$ and $(\sigma\circ\tau)^{*}$ simply as
$\tau$, $\sigma$ and $\sigma\circ\tau$, i.e.  we shall {\em drop}
the '$*$' when referring to a substitution mapping. Whether the
notation '$\sigma\circ\tau$' is understood to mean
$\sigma^{*}\circ\tau^{*}$ or $(\sigma\circ\tau)^{*}$ no longer
matters since the two mappings coincide anyway.
\begin{prop}\label{logic:prop:substitution:composition}
    Let $U$, $V$ and $W$ be sets. Let $\tau:U\to V$ and $\sigma:V\to W$
    be maps. Let $\tau^{*}:{\bf P}(U)\to\pv$ and $\sigma^{*}:\pv\to{\bf
    P}(W)$ be the substitution mappings associated with $\tau$ and
    $\sigma$ respectively. Then:
    \[
        (\sigma\circ\tau)^{*}
        =
        \sigma^{*}\circ\tau^{*}
    \]
    where $(\sigma\circ\tau)^{*}:{\bf P}(U)\to {\bf P}(W)$ is the
    substitution mapping associated with $\sigma\circ\tau$.
\end{prop}
\begin{proof}
We need to show that
$(\sigma\circ\tau)^{*}(\phi)=\sigma^{*}\circ\tau^{*}(\phi)$ for all
$\phi\in{\bf P}(U)$. We shall do so by structural induction, using
theorem~(\ref{logic:the:proof:induction}) of
page~\pageref{logic:the:proof:induction}. Since ${\bf P}_{0}(U)$ is
a generator of ${\bf P}(U)$, we show first that the property is true
on ${\bf P}_{0}(U)$. So let $\phi = (x\in y)\in{\bf P}_{0}(U)$,
where $x,y\in U$. We have:
    \[
        (\sigma\circ\tau)^{*}(\phi)
        =
        (\,\sigma(\tau(x))\,\in\,\sigma(\tau(y))\,)
        =
        \sigma^{*}(\,\tau(x)\in\tau(y)\,)
        =
        \sigma^{*}\circ\tau^{*}(\phi)
    \]
Next we check that the property is true for $\bot\in{\bf P}(U)$:
    \[
        (\sigma\circ\tau)^{*}(\bot)
        =
        \bot
        =
        \sigma^{*}(\bot)
        =
        \sigma^{*}\circ\tau^{*}(\bot)
    \]
Next we check that the property is true for
$\phi=\phi_{1}\to\phi_{2}$, if it is true for $\phi_{1},\phi_{2}$:
    \begin{eqnarray*}
        (\sigma\circ\tau)^{*}(\phi)
        &=&
        (\sigma\circ\tau)^{*}(\phi_{1})\to(\sigma\circ\tau)^{*}(\phi_{2})\\
        &=&
        \sigma^{*}(\tau^{*}(\phi_{1}))\to\sigma^{*}(\tau^{*}(\phi_{2}))\\
        &=&
        \sigma^{*}(\,\tau^{*}(\phi_{1})\to\tau^{*}(\phi_{2})\,)\\
        &=&
        \sigma^{*}(\tau^{*}(\phi_{1}\to\phi_{2}))\\
        &=&
        \sigma^{*}\circ\tau^{*}(\phi)
    \end{eqnarray*}
Finally we check that the property is true for $\phi=\forall
x\phi_{1}$, if it is true for $\phi_{1}$:
    \begin{eqnarray*}
    (\sigma\circ\tau)^{*}(\phi)&=&\forall \sigma(\tau(x))\,(\sigma\circ\tau)^{*}(\phi_{1})\\
    &=&\forall\sigma(\tau(x))\,\sigma^{*}(\tau^{*}(\phi_{1}))\\
    &=&\sigma^{*}(\forall\tau(x)\tau^{*}(\phi_{1}))\\
    &=&\sigma^{*}(\tau^{*}(\forall x\phi_{1}))\\
    &=&\sigma^{*}\circ\tau^{*}(\phi)
    \end{eqnarray*}
\end{proof}

In the case when $W=V$ and $\sigma:V\to W$ is the identity mapping
defined by $\sigma(x)=x$ for all $x\in V$,
equation~(\ref{logic:eqn:substitution:recursion}) of
definition~(\ref{logic:def:substitution}) becomes:
    \[
        \forall\phi\in\pv\ ,\ \sigma(\phi)
        =
        \left\{\begin{array}{lcl}
            (x\in y)&\mbox{\ if\ }&\phi=(x\in y)
            \\
            \bot&\mbox{\ if\ }&\phi=\bot
            \\
            \sigma(\phi_{1})\to \sigma(\phi_{2}) 
                &\mbox{\ if\ }&
            \phi=\phi_{1}\to\phi_{2}
            \\
            \forall x\,\sigma(\phi_{1})&\mbox{\ if\ }&\phi=\forall x\phi_{1}
        \end{array}\right.
    \]
It seems pretty obvious that $\sigma:\pv\to\pv$ is also the identity
mapping, i.e. that $\sigma(\phi)=\phi$ for all $\phi\in\pv$. Yet,
formally speaking, there seems to be no other way but to prove this
once and for all. So we shall do so now:
\begin{prop}\label{logic:prop:substitution:identity}
    Let $V$ be a set and $i:V\to V$ be the identity mapping. Then, the
    associated substitution mapping $i:\pv\to\pv$ is also the identity.
\end{prop}
\begin{proof}
We need to show that $i(\phi)=\phi$ for all $\phi\in\pv$. We shall
do so by structural induction, using
theorem~(\ref{logic:the:proof:induction}) of
page~\pageref{logic:the:proof:induction}. Since $\pvo$ is a
generator of \pv, we show first that the property is true on \pvo.
So let $\phi = (x\in y)\in\pvo$:
    \[
        i(\phi)
        =
        (\,i(x)\,\in\,i(y)\,)
        =
        (x\in y)
        =
        \phi
    \]
The property is clearly true for $\bot\in\pv$ since $i(\bot)=\bot$.
Next we check that the property is true for
$\phi=\phi_{1}\to\phi_{2}$, if it is true for $\phi_{1},\phi_{2}\in\pv$:
    \[
        i(\phi)
        =
        i(\phi_{1})\to i(\phi_{2})
        =
        \phi_{1}\to\phi_{2}
        =
        \phi
    \]
Finally we check that the property is true for $\phi=\forall
x\phi_{1}$, if it is true for $\phi_{1}$:
    \[
        i(\phi)
        =
        \forall i(x)i(\phi_{1})
        =
        \forall x\phi_{1}
        =
        \phi
    \]
\end{proof}

Since \pv\ is a free universal algebra, every formula $\phi\in\pv$
has a well defined set of sub-formulas $\subf(\phi)$ as per
definition~(\ref{logic:def:subformula}) of
page~\pageref{logic:def:subformula}. If $V$ and $W$ are sets and
$\sigma:V\to W$ is a map with associated substitution mapping
$\sigma:\pv\to{\bf P}(W)$, then given $\phi\in\pv$ and a sub-formula
$\psi\preceq\phi$ it seems pretty obvious that the image
$\sigma(\psi)$ is also a sub-formula of $\sigma(\phi)$. In fact, the
converse is also true and the sub-formulas of $\sigma(\phi)$ are
simply the images $\sigma(\psi)$ by $\sigma$ of the sub-formulas
$\psi\preceq\phi$. This property stems from the fact that
$\sigma:\pv\to{\bf P}(W)$ is a structural substitution, as per
definition~(\ref{logic:def:UA:structuralsub:structural:substitution}).

\begin{prop}\label{logic:prop:FOPL:substitution:subformula}
Let $V,W$ be sets and $\sigma:V\to W$ be a map. The associated
substitution $\sigma:\pv\to{\bf P}(W)$ is a structural substitution
and for all $\phi\in\pv$\,:
    \[
        \subf(\sigma(\phi))
        =
        \sigma(\subf(\phi))
    \]
\end{prop}
\begin{proof}
By virtue of
proposition~(\ref{logic:prop:UA:structuralsub:subformula}) it is
sufficient to prove that $\sigma:\pv\to{\bf P}(W)$ is a structural
substitution as per
definition~(\ref{logic:def:UA:structuralsub:structural:substitution}).
Let $\alpha(V)$ and $\alpha(W)$ denote the first order logic types
associated with $V$ and $W$ respectively as per
definition~(\ref{logic:def:FOPL:type}). Let
$q:\alpha(V)\to\alpha(W)$ be the map defined by:
    \[
        \forall f\in\alpha(V)\ ,\ q(f)
        =
        \left\{\begin{array}{lcl}
            \bot&\mbox{\ if\ }&f=\bot
            \\
            \to&\mbox{\ if\ }&f=\to
            \\
            \forall\sigma(x)&\mbox{\ if\ }&f=\forall x
        \end{array}\right.
    \]
Then $q$ is clearly arity preserving. In order to show that $\sigma$
is a structural substitution, we simply need to check that
properties $(i)$ and $(ii)$ of
definition~(\ref{logic:def:UA:structuralsub:structural:substitution})
are met. First we start with property $(i)$: so let $\phi\in\pvo$.
Then $\phi=(x\in y)$ for some $x,y\in V$ and we need to show that
$\sigma(\phi)\in{\bf P}_{0}(W)$ which follows immediately from
$\sigma(\phi)=\sigma(x)\in\sigma(y)$. So we now show property
$(ii)$. Given $f\in\alpha(V)$, given $\phi\in\pv^{\alpha(f)}$ we
need to show that $\sigma(f(\phi))=q(f)(\sigma(\phi))$. First we
assume that $f=\bot$. Then $\alpha(f)=0$, $\phi=0$ and consequently:
    \begin{eqnarray*}
        \sigma(f(\phi))&=&\sigma(\bot(0))
        \\
        \mbox{$\bot(0)$ denoted '$\bot$'}\ \rightarrow&=&\sigma(\bot)
        \\
        \mbox{def.~(\ref{logic:def:substitution})}\ \rightarrow&=&\bot
        \\
        \mbox{$\bot(0)$ denoted '$\bot$'}\ \rightarrow&=&\bot(0)
        \\
        \sigma:\{0\}\to\{0\}\ \rightarrow&=&q(\bot)(\sigma(0))
        \\
        &=&q(f)(\sigma(\phi))
    \end{eqnarray*}
Next we assume that $f=\to$. Then $\alpha(f)=2$ and given
$\phi=(\phi_{0},\phi_{1})$\,:
    \begin{eqnarray*}
        \sigma(f(\phi))&=&\sigma(\phi_{0}\to\phi_{1})
        \\
        \mbox{def.~(\ref{logic:def:substitution})}\ \rightarrow
            &=&\sigma(\phi_{0})\to\sigma(\phi_{1})
        \\
        \sigma:\pv^{2}\to{\bf P}(W)^{2}\ \rightarrow
            &=&q(\to)(\sigma(\phi_{0},\phi_{1}))
        \\
        &=&q(f)(\sigma(\phi))
    \end{eqnarray*}
Finally we assume that $f=\forall x$, $x\in V$. Then $\alpha(f)=1$
and given $\phi=(\phi_{0})$\,:
    \begin{eqnarray*}
        \sigma(f(\phi))&=&\sigma(\forall x\phi_{0})
        \\
        \mbox{def.~(\ref{logic:def:substitution})}\ \rightarrow
            &=&\forall\sigma(x)\sigma(\phi_{0})
        \\
        \sigma:\pv^{1}\to{\bf P}(W)^{1}\ \rightarrow
            &=&q(\forall x)(\sigma(\phi_{0}))
        \\
        &=&q(f)(\sigma(\phi))
    \end{eqnarray*}
\end{proof}
