This work is an attempt to provide an {\em algebraization of first
order logic with terms and without equality}. Due to personal
ignorance, we are not in a position to say much of the history of
this problem, or the existing literature. We shall nonetheless
endeavor to give here an honest account of our little understanding.
Any specialist who believes this account can be improved in any way
is welcome to contact the author. The idea of turning first order
logic into algebra is not new. Following wikipedia, it appears
Alfred Tarski initially worked on the algebraization of first order
logic without terms and with equality, which led to the emergence of
cylindric algebras. At a later stage, Paul Halmos developed a
version without equality, which led to polyadic algebras. There is
also a categorical approach initially due to William Lawvere,
leading to functorial semantics. The most recent treatment of
algebraization of first order logic without terms seems to be the
work of Ildik\'o Sain in~\cite{Sain}. There is also the categorical
work of George Voutsadakis, for example in~\cite{Voutsadakis}
following a recent trend of {\em abstract algebraic logic} which
seems to have been initiated by W.J. Blok and D.
Pigozzi~\cite{BlokPigozzi}. A new book~\cite{AndrekaNemeti} from
Hajnal Andr\'eka, Istv\'an N\'emeti and Ildik\'o Sain on algebraic
logic is about to be released. There are a few older monographs on
cylindric algebras such as~\cite{Cylindric} and Paul
Halmos~\cite{Halmos} on polyadic algebras. It is our understanding
that most of the work done so far has focussed on first order logic
{\em without terms}, an exception being the paper from Janis
Cirulis~\cite{Cirulis}, of which we are hoping to get a copy soon.
The algebraization of first order logic {\em with terms} seems to
have received far less attention than its counterpart {\em without
terms}, as it fails to fall under the scope of currently known
techniques of abstract algebraic logic. Before we complete our
history tour, we would like to mention the books of Donald W. Barnes
and John M. Mack~\cite{AlgLog} and P.T. Johnstone~\cite{Johnstone}
which are the initial motivation for the present work. For a
mathematician not trained in formal logic, these books have
something magical. It is hard for us to say if they succeeded in
presenting their subject. Many of the proofs are very short and
require a fair amount of maturity. This document is an attempt to
follow their trail while making the material accessible to less
sophisticated readers. We shall now explain our purpose in more
details.

Our aim is to study mathematical statements as mathematical objects
of their own. We believe universal algebras are likely to be the
right tool for this study. We would like to represent the set of all
possible mathematical statements as a universal algebra of some
type. Without knowing much of the details at this stage, we shall
call this universal algebra {\em the universal algebra of first
order logic}. The elements of this universal algebra will be called
{\em formulas of first order logic}. It may appear as somewhat naive
to refer to {\em the} universal algebra of first order logic as if
we had some form of uniqueness property. There are after all many
logical systems which may be classified as {\em first order}, and
many more algebras arising from these systems. However, we are
looking for an algebra which is minimal in some sense, and
encompasses {\em the language of {\bf ZF}} on which most of modern
mathematics can be built. When looking at a given mathematical
statement, we are often casually told that it could be (at least in
principle) coded into the language of {\bf ZF}. Oh really? We would
love to see that. There are so many issues surrounding this
question, it does not seem obvious at all. One cannot simply define
predicates of ever increasing complexity and hope to blindly
substitutes these definitions within formulas. Most mathematical
objects are not uniquely defined as sets, but are complex structures
which are known up to isomorphism. The list of difficulties is
endless. And there is of course category theory. Granted category
theory will never fit into the language of {\bf ZF}, so we shall
need another algebra for that. But who knows? it may be that {\em
the universal algebra of first order logic} will be applicable to
category theory: once we can formally define a predicate of a single
variable, we can formally define a class as a proper mathematical
object and dwell into the wonders of meta-mathematics without the
intuition and the hand waving.

So much for finding the names. The real challenge is to determine
the appropriate definitions. We saw in
theorem~(\ref{logic:the:quotient:free:algebra}) of
page~\pageref{logic:the:quotient:free:algebra}, that there was no
loss of generality in viewing our universal algebra as a quotient of
a free universal algebra, modulo the right congruence. One possible
way to define the universal algebra of first order logic is
therefore to provide specific answers to the following questions:
First, a free generator $X_{0}$ needs to be chosen. We then need to
agree on a specific type of universal algebra $\alpha$ in order to
derive a free universal algebra $X$ of type $\alpha$ with free
generator $X_{0}$. Finally, a decision needs to be made on what the
{\em appropriate} congruence $\sim$ on $X$ should be.

These we believe are the right questions. For those already familiar
with elements of formal logic, we may outline now a possible answer.
Having chosen a formal language with a deductive system on it, we
can consider the relation $\sim$ defined by $\phi\sim\psi$ \ifand\
the formulas $\phi\rightarrow\psi$ and $\psi\rightarrow\phi$ are
both provable with respect to this deductive system. This choice of
particular congruence gives rise to a Lindenbaum-Tarski algebra. So
it is possible our universal algebra of first order logic is just a
particular case of Lindenbaum-Tarski (and it would probably have
been sensible to call it that way). However at this point of the
document, we do not know whether the provability of
$\phi\rightarrow\psi$ is decidable. We may have heard a few things
about the undecidability of first order logic and suspect that
$\vdash(\phi\rightarrow\psi)$ is in fact undecidable. If this was
the case, a computer would not be able to tell in general whether
the equivalence $\phi\sim\psi$ holds. This would be highly
unsatisfactory. Whatever congruence we eventually choose, surely we
should want it to be decidable. The universal algebra of first order
logic is about mathematical statements, not theorems. Those
statements which happen to have the same {\em meaning} should be
regarded as identical, and a computer program should be able to
establish this identity. Those mathematical statements which happen
to be mathematically equivalent should be dealt with by proof
theory, model theory and set theory.
