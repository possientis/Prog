A substitution congruence for proofs will never be acceptable to us
unless it has the same connection to the minimal transform which it
has for formulas. So we need to prove the implication ${\cal
M}(\pi)={\cal M}(\rho)\ \Rightarrow\ \pi\sim\rho$. At some point we
shall be faced with an equality ${\cal M}(\pi_{1})[n/x]={\cal
M}(\rho_{1})[n/x]$ from which we shall want to argue that ${\cal
M}(\pi_{1})={\cal M}(\rho_{1})$. So we shall need to have a local
inversion theorem for proofs just as we have
theorem~(\ref{logic:the:FOPL:localinv:main}) of
page~\pageref{logic:the:FOPL:localinv:main} for formulas. So the
purpose of this section is to establish
theorem~(\ref{logic:the:FUAP:localinversion:inversion}) of
page~\pageref{logic:the:FUAP:localinversion:inversion} below. A
local inversion theorem allows us to derive $\pi=\rho$ from an
equality $\sigma(\pi)=\sigma(\rho)$ in certain conditions,
especially in cases when the proof substitution $\sigma:\pvs\to{\bf
\Pi}(W)$ associated with $\sigma:V\to W$ may not be an injective
map. We are pretty sure we could find ways to achieve our purpose
without a local inversion theorem. But the result is interesting in
its own right. So rather than redesigning old proofs based on the
equivalence ${\cal M}(\phi)\sim i(\phi)$ where $i:V\to\bar{V}$ is
the inclusion mapping, we have decided to keep a local inversion
theorem for proofs in our list of objectives. The material that
follows is relatively technical and dry. It is simply an extension
of the work previously done for formulas. A lot of motivational
background can be found in the discussions preceding
lemma~(\ref{logic:lemma:FOPL:localinv:lem}). We start with the
notion of {\em dual substitution} $\tau:\pvs\to{\bf\Pi}(W)$
associated with an ordered pair $(\tau_{0},\tau_{1})$ of maps
$\tau_{0},\tau_{1}:V\to W$. The underlying idea is that the proof
$\tau(\pi)$ should correspond to the proof $\pi$, where all free
variables have been substituted in accordance to the map $\tau_{0}$
and bound variables in accordance to the map $\tau_{1}$. The main
issue here is that $\tau(\pi)$ cannot be defined directly with a
simple recursion. A recursive formula would not be able to tell
whether a variable will remain free in the final proof being
considered. So the proof $\tau(\pi)$ needs to be defined in two
stages. We start by defining a map $\tau^{\circ}:\pvs\to[{\cal
P}(V)\to{\bf\Pi}(W)]$ so that $\tau^{\circ}(\pi)$ rather than being
a proof, is a map $\tau^{\circ}(\pi):{\cal P}(V)\to{\bf\Pi}(W)$. If
$U\subseteq V$ is a set of variables, then $\tau^{\circ}(\pi)(U)$ is
the proof obtained from $\pi$, $\tau_{0}$ and $\tau_{1}$ assuming
all variables in $U$ are free, except those being bound by the
recursive formula. We then define $\tau(\pi)$ as
$\tau(\pi)=\tau^{\circ}(\pi)(V)$. More explanations can be found in
the discussion preceding
definition~(\ref{logic:def:FOPL:dualsubst:dualsubst}). The following
definition is in fact the counterpart of
definition~(\ref{logic:def:FOPL:dualsubst:dualsubst}) which defines
a map $\tau^{*}:\pv\to[{\cal P}(V)\to{\bf P}(W)]$ for formulas which
corresponds to our $\tau^{\circ}:\pvs\to[{\cal P}(V)\to{\bf\Pi}(W)]$
for proofs. The key point of note in
definition~(\ref{logic:def:FUAP:localinversion:dualsubst}) below is
that $\tau^{\circ}(\pi)(U)$ is defined in terms of
$\tau^{\circ}(\pi_{1})(U\setminus\{x\})$ whenever $\pi$ is of the
form $\pi=\gen x\pi_{1}$. Loosely speaking, the recursive formula
has now become aware that $x$ is no longer a free variable.


\index{dual@Dual substitution in proof}
\begin{defin}\label{logic:def:FUAP:localinversion:dualsubst}
Let $V$, $W$ be sets and $\tau_{0}, \tau_{1}:V\to W$ be maps. We
call {\em dual variable substitution} associated with
$(\tau_{0},\tau_{1})$ the map $\tau:\pvs\to {\bf\Pi}(W)$ defined by
$\tau(\pi)=\tau^{\circ}(\pi)(V)$, where the map
$\tau^{\circ}:\pvs\to[{\cal P}(V)\to{\bf\Pi}(W)]$ is defined by the
following structural recursion, given $\pi\in\pvs$ and $U\in{\cal
P}(V)$:
    \begin{equation}\label{logic:eqn:FUAP:localinversion:dualsubst:1}
                    \tau^{\circ}(\pi)(U)=\left\{
                    \begin{array}{lcl}
                    \tau^{*}(\phi)(U)&\mbox{\ if\ }&\pi=\phi\in\pv\\
                    \axi\tau^{*}(\phi)(U)&\mbox{\ if\ }&\pi=\axi\phi\\
                    \tau^{\circ}(\pi_{1})(U)\pon\,\tau^{\circ}(\pi_{2})(U)
                    &\mbox{\ if\ }&\pi=\pi_{1}\pon\pi_{2}\\
                    \gen \tau_{1}(x)\tau^{\circ}(\pi_{1})(U\setminus\{x\})&\mbox{\ if\ }&\pi=\gen
                    x\pi_{1}\\
                    \end{array}\right.
    \end{equation}
where the map $\tau^{*}:\pv\to[{\cal P}(V)\to{\bf P}(W)]$ is as per
{\em definition~(\ref{logic:def:FOPL:dualsubst:dualsubst})}
\end{defin}

\begin{prop}
The structural recursion of {\em
definition~(\ref{logic:def:FUAP:localinversion:dualsubst})} is
legitimate.
\end{prop}
\begin{proof}
We need to prove the existence and uniqueness of
$\tau^{\circ}:\pvs\to[{\cal P}(V)\to{\bf\Pi}(W)]$ satisfying
equation~(\ref{logic:eqn:FUAP:localinversion:dualsubst:1}). We shall
do so using theorem~(\ref{logic:the:structural:recursion}) of
page~\pageref{logic:the:structural:recursion}. So we take $X=\pvs$,
$X_{0}=\pv$ and $A=[{\cal P}(V)\to{\bf\Pi}(W)]$. We consider the map
$g_{0}:X_{0}\to A$ defined by $g_{0}(\phi)(U)=\tau^{*}(\phi)(U)$
where it is understood that $\tau^{*}:\pv\to[{\cal P}(V)\to{\bf
P}(W)]$ is as per
definition~(\ref{logic:def:FOPL:dualsubst:dualsubst}). Given
$\phi\in\pv$ we define $h(\axi\phi):A^{0}\to A$ by setting
$h(\axi\phi)(0)(U)=\axi\tau^{*}(\phi)(U)$ and $h(\pon):A^{2}\to A$
by setting $h(\pon)(v)(U)=v(0)(U)\pon\,v(1)(U)$. Finally, we define
$h(\gen x):A^{1}\to A$\,:
    \[
    h(\gen x)(v)(U)=\gen\tau_{1}(x)v(0)(U\setminus\{x\})
    \]
From theorem~(\ref{logic:the:structural:recursion}) there exists a
unique map $\tau^{\circ}:X\to A$ with $\tau^{\circ}(\pi)=g_{0}(\pi)$
whenever $\pi\in\pv$ and
$\tau^{\circ}(f(\pi))=h(f)(\tau^{\circ}(\pi))$ for all
$f=\axi\phi,\pon,\gen x$ and $\pi\in X^{\alpha(f)}$. So let us check
that this works: from $\tau^{\circ}(\pi)=g_{0}(\pi)$ whenever
$\pi=\phi$ for some $\phi\in\pv$ we obtain
$\tau^{\circ}(\pi)(U)=g_{0}(\phi)(U)=\tau^{*}(\phi)(U)$ which is the
first line of
equation~(\ref{logic:eqn:FUAP:localinversion:dualsubst:1}). Take
$f=\axi\phi$ for some $\phi\in\pv$ and $\pi=\axi\phi$:
    \begin{eqnarray*}
    \tau^{\circ}(\pi)(U)
    &=&\tau^{\circ}(\axi\phi)(U)\\
    \mbox{proper notation}\ \rightarrow
    &=&\tau^{\circ}(\axi\phi(0))(U)\\
    &=&\tau^{\circ}(f(0))(U)\\
    \tau^{\circ}:X^{0}\to A^{0}\ \rightarrow
    &=&h(f)(\tau^{\circ}(0))(U)\\
    &=&h(f)(0)(U)\\
    &=&h(\axi\phi)(0)(U)\\
    &=&\axi\tau^{*}(\phi)(U)
    \end{eqnarray*}
which is the second line of
equation~(\ref{logic:eqn:FUAP:localinversion:dualsubst:1}). Recall
that the $\tau^{\circ}$ which appears in the fourth equality refers
to the unique mapping $\tau^{\circ}:X^{0}\to A^{0}$. We now take
$f=\pon$ and $\pi=\pi_{1}\pon\pi_{2}$ for some
$\pi_{1},\pi_{2}\in\pvs$. Then we have:
    \begin{eqnarray*}
    \tau^{\circ}(\pi)(U)&=&\tau^{\circ}(\pi_{1}\pon\pi_{2})(U)\\
    \pi^{*}(0)=\pi_{1},\ \pi^{*}(1)=\pi_{2}\ \rightarrow
    &=&\tau^{\circ}(f(\pi^{*}))(U)\\
    \tau^{\circ}:X^{2}\to A^{2}\ \rightarrow
    &=&h(f)(\tau^{\circ}(\pi^{*}))(U)\\
    &=&h(\pon)(\tau^{\circ}(\pi^{*}))(U)\\
    &=&\tau^{\circ}(\pi^{*})(0)(U)\pon\,\tau^{\circ}(\pi^{*})(1)(U)\\
    &=&\tau^{\circ}(\pi^{*}(0))(U)\pon\,\tau^{\circ}(\pi^{*}(1))(U)\\
    &=&\tau^{\circ}(\pi_{1})(U)\pon\,\tau^{\circ}(\pi_{2})(U)\\
    \end{eqnarray*}
This is the third line of
equation~(\ref{logic:eqn:FUAP:localinversion:dualsubst:1}). Finally
consider $f=\gen x$ and $\pi=\gen x\pi_{1}$:
    \begin{eqnarray*}
    \tau^{\circ}(\pi)(U)
    &=&\tau^{\circ}(\gen x\pi_{1})(U)\\
    \pi^{*}(0)=\pi_{1}\ \rightarrow
    &=&\tau^{\circ}(f(\pi^{*}))(U)\\
    \tau^{\circ}:X^{1}\to A^{1}\ \rightarrow
    &=&h(f)(\tau^{\circ}(\pi^{*}))(U)\\
    &=&h(\gen x)(\tau^{\circ}(\pi^{*}))(U)\\
    &=&\gen\tau_{1}(x)\tau^{\circ}(\pi^{*})(0)(U\setminus\{x\})\\
    &=&\gen\tau_{1}(x)\tau^{\circ}(\pi^{*}(0))(U\setminus\{x\})\\
    &=&\gen\tau_{1}(x)\tau^{\circ}(\pi_{1})(U\setminus\{x\})\\
    \end{eqnarray*}
and this is the last line of
equation~(\ref{logic:eqn:FUAP:localinversion:dualsubst:1}).
\end{proof}

The following lemma is the counterpart of
lemma~(\ref{logic:lemma:FOPL:localinv:lem}) prior to which some
motivational discussion may be found. It is the main lemma of this
section allowing us to prove the local inversion
theorem~(\ref{logic:the:FUAP:localinversion:inversion}) below. The
purpose of the theorem is to derive $\pi=\rho$ from the equality
$\sigma(\pi)=\sigma(\rho)$ in cases when $\sigma$ behaves nicely
enough on $\pi$ and $\rho$. Specifically, we require that $\sigma$
be an injective map when acting on the free variables of $\pi$ and
$\rho$, and also when acting on the bound variables of $\pi$ and
$\rho$. We also require that $\sigma$ be valid for $\pi$ and $\rho$
so that no confusion arises between the free and bound variables.
The strategy to obtain the equality $\pi=\rho$ is simply to
construct a {\em local inverse} of $\sigma$, namely a map
$\tau:{\bf\Pi}(W)\to\pvs$ for which the equality
$\tau\circ\sigma(\pi)=\pi$ holds in some {\em neighborhood} $\Pi$ of
$\pi$ and $\rho$. The map $\tau:{\bf\Pi}(W)\to\pvs$ is defined as a
{\em dual substitution} as per
definition~(\ref{logic:def:FUAP:localinversion:dualsubst})
associated with a pair $(\tau_{0},\tau_{1})$ which are left-inverses
of $\sigma$ when restricted to the free and bound variables
respectively. The main difficulty is to check the whole construction
works, namely that the equality $\tau\circ\sigma(\pi)=\pi$ holds.
The argument needs to be done by structural induction which is the
purpose of this lemma leading to
equation~(\ref{logic:eqn:FUAP:localinversion:Main:1}) below. The
lemma is carefully designed so that the induction hypothesis carries
through nicely at every step of the structural induction argument,
as it should.

\begin{lemma}\label{logic:lemma:FUAP:localinversion:Main}
Let $V$, $W$ be sets and $\sigma:V\to W$ be a map. Let $V_{0}$,
$V_{1}$ be subsets of $V$ and $\tau_{0},\tau_{1}:W\to V$ be maps
such that for all $x\in V$:
   \begin{eqnarray*}
    (i)&&x\in V_{0}\ \Rightarrow\ \tau_{0}\circ\sigma(x)=x\\
    (ii)&&x\in V_{1}\ \Rightarrow\ \tau_{1}\circ\sigma(x)=x
    \end{eqnarray*}
Let $\tau^{\circ}:{\bf\Pi}(W)\to[{\cal P}(W)\to\pvs]$ be the map
associated with $(\tau_{0},\tau_{1})$ as per {\em
definition~(\ref{logic:def:FUAP:localinversion:dualsubst})}. Then
for all $U\in{\cal P}(V)$ and $\pi\in\pvs$ we have:
    \begin{equation}\label{logic:eqn:FUAP:localinversion:Main:1}
    \tau^{\circ}(\sigma(\pi))(W\setminus\sigma(U))=\pi
    \end{equation}
provided $U$ and $\pi$ satisfy the following properties:
    \begin{eqnarray*}
    (iii)&&(\,\free(\pi)\setminus U\subseteq V_{0}\,)
    \land(\,\bound(\pi)\cup U\subseteq V_{1}\,)\\
    (iv)&&(\,\mbox{$\sigma$ valid for $\pi$}\,)\land
    (\,\sigma(U)\cap\sigma(\free(\pi)\setminus U)=\emptyset\,)\\
    \end{eqnarray*}
\end{lemma}
\begin{proof}
We assume $\sigma:V\to W$ is given together with the subsets
$V_{0}$, $V_{1}$ and the maps $\tau_{0},\tau_{1}:W\to V$ satisfying
$(i)$ and $(ii)$. Let $\tau^{\circ}:{\bf\Pi}(W)\to[{\cal
P}(W)\to\pvs]$ be the map associated with the ordered pair
$(\tau_{0},\tau_{1})$ as per
definition~(\ref{logic:def:FUAP:localinversion:dualsubst}) of
page~\pageref{logic:def:FUAP:localinversion:dualsubst}. Given
$U\subseteq V$ and $\pi\in\pvs$ consider the property $q(U,\pi)$
defined by $(iii)$ and $(iv)$. Then we need to show that for all
$\pi\in\pvs$ we have:
    \[
    \forall U\subseteq V\ ,\ [\,q(U,\pi)\ \Rightarrow\
    \tau^{\circ}(\sigma(\pi))(W\setminus\sigma(U))=\pi\,]
    \]
We shall do so by a structural induction argument, using
theorem~(\ref{logic:the:proof:induction}) of
page~\pageref{logic:the:proof:induction}. First we assume that
$\pi=\phi$ for some $\phi\in\pv$. Let $U\subseteq V$ and suppose
$q(U,\pi)$ is true. We need to show that
equation~(\ref{logic:eqn:FUAP:localinversion:Main:1}) holds. We
have:
    \begin{eqnarray*}
    \tau^{\circ}(\sigma(\pi))(W\setminus\sigma(U))&=&\tau^{\circ}(\sigma(\phi))(W\setminus\sigma(U))\\
    \mbox{def.~(\ref{logic:def:FUAP:localinversion:dualsubst})}\ \rightarrow
    &=&\tau^{*}(\sigma(\phi))(W\setminus\sigma(U))\\
    \mbox{lemma~(\ref{logic:lemma:FOPL:localinv:lem})}\ \rightarrow&=&\phi\\
    &=&\pi
    \end{eqnarray*}
Next we assume that $\pi=\axi\phi$ for some $\phi\in\pv$. Then we
have:
    \begin{eqnarray*}
    \tau^{\circ}(\sigma(\pi))(W\setminus\sigma(U))&=&\tau^{\circ}(\sigma(\axi\phi))(W\setminus\sigma(U))\\
    &=&\tau^{\circ}(\axi\sigma(\phi))(W\setminus\sigma(U))\\
    \mbox{def.~(\ref{logic:def:FUAP:localinversion:dualsubst})}\ \rightarrow
    &=&\axi\tau^{*}(\sigma(\phi))(W\setminus\sigma(U))\\
    \mbox{A: to be proved}\ \rightarrow&=&\axi\phi\\
    &=&\pi
    \end{eqnarray*}
It remains to justify point $A$ for which we need to show
$\tau^{*}(\sigma(\phi))(W\setminus\sigma(U))=\phi$. This follows
once again from an application of
lemma~(\ref{logic:lemma:FOPL:localinv:lem}), provided we can check
the assumptions of the lemma are indeed satisfied whenever
$q(U,\axi\phi)$ is itself satisfied. This follows from
$\free(\axi\phi)=\free(\phi)$, $\bound(\axi\phi)=\bound(\phi)$ and
the fact that the validity of $\sigma$ for $\axi\phi$ is equivalent
to the validity of $\sigma$ for $\phi$, as proven from
proposition~(\ref{logic:prop:FUAP:validsubproof:recursion:axiom}).
So we now assume that $\pi=\pi_{1}\pon\pi_{2}$ where
$\pi_{1},\pi_{2}\in\pvs$ satisfy our desired property. We need to
show the same is true of $\pi$. So let $U\subseteq V$ and suppose
$q(U,\pi)$ is true. We need to show that
equation~(\ref{logic:eqn:FUAP:localinversion:Main:1}) holds:
    \begin{eqnarray*}
    \tau^{\circ}(\sigma(\pi))(W\setminus\sigma(U))&=&
    \tau^{\circ}(\sigma(\pi_{1}\pon\pi_{2}))(W\setminus\sigma(U))\\
    \mbox{define $U^{*}=W\setminus\sigma(U)$}\ \rightarrow
    &=&\tau^{\circ}(\sigma(\pi_{1}\pon\pi_{2}))(U^{*})\\
    &=&\tau^{\circ}(\,\sigma(\pi_{1})\pon\,\sigma(\pi_{2})\,)(U^{*})\\
    \mbox{definition~(\ref{logic:def:FUAP:localinversion:dualsubst})}\
    \rightarrow
    &=&\tau^{\circ}(\sigma(\pi_{1}))(U^{*})\pon\,\tau^{\circ}(\sigma(\pi_{2}))(U^{*})\\
    \mbox{A: to be proved}\ \rightarrow&=&\pi_{1}\pon\pi_{2}\\
    &=&\pi
    \end{eqnarray*}
So it remains to show that
equation~(\ref{logic:eqn:FUAP:localinversion:Main:1}) holds for
$\pi_{1}$ and $\pi_{2}$. However, from our induction hypothesis, our
property is true for $\pi_{1}$ and $\pi_{2}$. Hence, it is
sufficient to prove that $q(U,\pi_{1})$ and $q(U,\pi_{2})$ are true.
First we show that $q(U,\pi_{1})$ is true. We need to prove that
$(iii)$ and $(iv)$ above are true for $\pi_{1}$\,:
    \[
    \free(\pi_{1})\setminus U\subseteq\free(\pi)\setminus U\subseteq V_{0}
    \]
where the second inclusion follows from our assumption of
$q(U,\pi)$. Furthermore:
    \[
    \bound(\pi_{1})\cup U\subseteq\bound(\pi)\cup U\subseteq V_{1}
    \]
So $(iii)$ is now established for $\pi_{1}$. Also from $q(U,\pi)$ we
obtain:
    \[
    \sigma(U)\cap\sigma(\free(\pi_{1})\setminus U)\subseteq
    \sigma(U)\cap\sigma(\free(\pi)\setminus U)=\emptyset
    \]
So it remains to show that $\sigma$ is valid for $\pi_{1}$, which
follows immediately from
proposition~(\ref{logic:prop:FUAP:validsubproof:recursion:pon}) and
the validity of $\sigma$ for $\pi=\pi_{1}\pon\pi_{2}$. This
completes our proof of $q(U,\pi_{1})$. The proof of $q(U,\pi_{2})$
being identical, we are now done with the case
$\pi=\pi_{1}\pon\pi_{2}$. So we now assume that $\pi=\gen x\pi_{1}$
where $x\in V$ and $\pi_{1}\in\pvs$ satisfy our induction
hypothesis. We need to show the same is true for $\pi$. So let
$U\subseteq V$ and suppose $q(U,\pi)$ is true. We need to show
equation~(\ref{logic:eqn:FUAP:localinversion:Main:1})\,:
    \begin{eqnarray*}
    \tau^{\circ}(\sigma(\pi))(W\setminus\sigma(U))&=&
    \tau^{\circ}(\sigma(\gen x\pi_{1}))(W\setminus\sigma(U))\\
    \mbox{define $U^{*}=W\setminus\sigma(U)$}\ \rightarrow
    &=&\tau^{\circ}(\sigma(\gen x\pi_{1}))(U^{*})\\
    &=&\tau^{\circ}(\,\gen\sigma(x)\sigma(\pi_{1})\,)(U^{*})\\
    \mbox{definition~(\ref{logic:def:FUAP:localinversion:dualsubst})}\
    \rightarrow
    &=&\gen \tau_{1}\circ\sigma(x)\,\tau^{\circ}(
    \sigma(\pi_{1}))(\,U^{*}\setminus\{\sigma(x)\}\,)\\
    \mbox{$(ii)$ and $x\in\bound(\pi)\subseteq V_{1}$}\ \rightarrow
    &=&\gen x\,\tau^{\circ}(
    \sigma(\pi_{1}))(\,U^{*}\setminus\{\sigma(x)\}\,)\\
    \mbox{define $U_{1}^{*}=U^{*}\setminus\{\sigma(x)\}$}\ \rightarrow
    &=&\gen x\,\tau^{\circ}(\sigma(\pi_{1}))(U_{1}^{*})\\
    \mbox{A: to be proved}\ \rightarrow
    &=&\gen x\pi_{1}\\
    &=&\pi
    \end{eqnarray*}
So it remains to show that
$\tau^{\circ}(\sigma(\pi_{1}))(U_{1}^{*})=\pi_{1}$. From
$U^{*}=W\setminus\sigma(U)$ and
$U_{1}^{*}=U^{*}\setminus\{\sigma(x)\}$ we obtain
$U_{1}^{*}=W\setminus\sigma(U_{1})$ where $U_{1}=U\cup\{x\}$. So it
remains to show that
$\tau^{\circ}(\sigma(\pi_{1}))(W\setminus\sigma(U_{1}))=\pi_{1}$, or
equivalently that
equation~(\ref{logic:eqn:FUAP:localinversion:Main:1}) holds for
$U_{1}$ and $\pi_{1}$. Having assumed $\pi_{1}$ satisfies our
induction hypothesis, it is therefore sufficient to prove that
$q(U_{1},\pi_{1})$ is true. So we need to show that $(iii)$ and
$(iv)$ above are true for $U_{1}$ and $\pi_{1}$, which goes as
follows:
    \begin{eqnarray*}
    \free(\pi_{1})\setminus
    U_{1}&=&\free(\pi_{1})\setminus\{x\}\setminus U\\
    &=&\free(\pi)\setminus U\\
    q(U,\pi)\ \rightarrow&\subseteq&V_{0}
    \end{eqnarray*}
Furthermore:
    \begin{eqnarray*}
    \bound(\pi_{1})\cup U_{1}&=&\bound(\pi_{1})\cup\{x\}\cup U\\
    &=&\bound(\pi)\cup U\\
    q(U,\pi)\ \rightarrow&\subseteq& V_{1}
    \end{eqnarray*}
and:
    \begin{eqnarray*}
    \sigma(U_{1})\cap\sigma(\free(\pi_{1})\setminus U_{1})
    &=&\sigma(U_{1})\cap\sigma(\,\free(\pi_{1})\setminus\{x\}\setminus
    U\,)\\
    &=&\sigma(U_{1})\cap\sigma(\free(\pi)\setminus U)\\
    &=&(\sigma(U)\cup\{\sigma(x)\})\cap\sigma(\free(\pi)\setminus
    U)\\
    q(U,\pi)\ \rightarrow&=&\{\sigma(x)\}\cap\sigma(\free(\pi)\setminus
    U)\\
    &\subseteq&\{\sigma(x)\}\cap\sigma(\free(\pi))\\
    \mbox{A: to be proved}\ \rightarrow&=&\emptyset
    \end{eqnarray*}
So we need to show that $\sigma(u)\neq\sigma(x)$ whenever
$u\in\free(\pi)$, which follows immediately from
proposition~(\ref{logic:prop:FUAP:validsubproof:recursion:gen}) and
the validity of $\sigma$ for $\pi=\gen x\pi_{1}$, itself a
consequence of our assumption $q(U,\pi)$. So we are almost done
proving $q(U_{1},\pi_{1})$ and it remains to show that $\sigma$ is
valid for $\pi_{1}$, which is another immediate consequence of
proposition~(\ref{logic:prop:FUAP:validsubproof:recursion:gen}).
This completes our induction argument.
\end{proof}

We are now ready to conclude with our local inversion theorem which
is the counterpart of theorem~(\ref{logic:the:FOPL:localinv:main})
of page~\pageref{logic:the:FOPL:localinv:main} and follows from
lemma~(\ref{logic:lemma:FUAP:localinversion:Main}).

\index{local@Local inversion for proof}
\begin{theorem}\label{logic:the:FUAP:localinversion:inversion}
Let $V$, $W$ be sets and $\sigma:V\to W$ be a map. Let $V_{0}$,
$V_{1}$ be subsets of $V$ such that $\sigma_{|V_{0}}$ and
$\sigma_{|V_{1}}$ are injective maps. Let $\Pi$ be the subset of
\pvs:
    \[
    \Pi=\{\pi\in\pvs:(\free(\pi)\subseteq
    V_{0})\land(\bound(\pi)\subseteq V_{1})\land(\mbox{$\sigma$
    valid for $\pi$})\}
    \]
Then, there exits $\tau:{\bf\Pi}(W)\to\pvs$ such that:
    \[
    \forall\pi\in\Pi\ ,\ \tau\circ\sigma(\pi)=\pi
    \]
where $\sigma:\pvs\to{\bf\Pi}(W)$ also denotes the associated proof
substitution mapping.
\end{theorem}
\begin{proof}
Let $\sigma:V\to W$ be a map and $V_{0},V_{1}\subseteq V$ be such
that $\sigma_{|V_{0}}$ and $\sigma_{|V_{1}}$ are injective maps. We
shall distinguish two cases: first we assume that $V=\emptyset$.
Then for all $\pi\in\pvs$ the inclusions $\free(\pi)\subseteq V_{0}$
and $\bound(\pi)\subseteq V_{1}$ are always true. Furthermore, from
definition~(\ref{logic:def:FUAP:validsubproof:validsub}) the
substitution $\sigma$ is always valid for~$\pi$. Hence we have
$\Pi=\pvs$ and we need to find $\tau:{\bf\Pi}(W)\to\pvs$ such that
$\tau\circ\sigma(\pi)=\pi$ for all $\pi\in\pvs$. Our first step is
to apply theorem~(\ref{logic:the:FOPL:localinv:main}) of
page~(\pageref{logic:the:FOPL:localinv:main}) to obtain a map
$\tau^{*}:{\bf P}(W)\to\pv$ such that
$\tau^{*}\circ\sigma(\phi)=\phi$ for all $\phi\in\Gamma$ where
$\Gamma=\{\phi\in\pv:(\free(\phi)\subseteq
    V_{0})\land(\bound(\phi)\subseteq V_{1})\land(\mbox{$\sigma$
    valid for $\phi$})\}$. However, since $V=\emptyset$ we have
$\Gamma=\pv$ and so $\tau^{*}\circ\sigma(\phi)=\phi$ for all
$\phi\in\pv$. We define $\tau$ with the following recursion on
${\bf\Pi}(W)$\,:
    \begin{equation}\label{logic:eqn:FUAP:localinversion:inversion:1}
                    \tau(\rho)=\left\{
                    \begin{array}{lcl}
                    \tau^{*}(\psi)&\mbox{\ if\ }&\rho=\psi\in{\bf P}(W)\\
                    \axi\tau^{*}(\psi)&\mbox{\ if\ }&\rho=\axi\psi\\
                    \tau(\rho_{1})\pon\,\tau(\rho_{2})
                    &\mbox{\ if\ }&\rho=\rho_{1}\pon\rho_{2}\\
                    \bot&\mbox{\ if\ }&\rho=\gen
                    u\rho_{1}\\
                    \end{array}\right.
    \end{equation}
We do not really care how $\tau(\rho)$ is defined in the case when
$\rho=\gen u\rho_{1}$ so we are setting $\tau(\rho)=\bot$ which is
the proof with single hypothesis $\bot$ and conclusion $\bot$. We
shall now prove that $\tau\circ\sigma(\pi)=\pi$ using a structural
induction argument. First we assume that $\pi=\phi$ for some
$\phi\in\pv$. Then we have:
    \begin{eqnarray*}
    \tau\circ\sigma(\pi)&=&\tau\circ\sigma(\phi)\\
    \mbox{eq.~(\ref{logic:eqn:FUAP:localinversion:inversion:1})}\ \rightarrow
    &=&\tau^{*}\circ\sigma(\phi)\\
    &=&\phi\\
    &=&\pi
    \end{eqnarray*}
Next we assume that $\pi=\axi\phi$ for some $\phi\in\pv$. Then we
have:
    \begin{eqnarray*}
    \tau\circ\sigma(\pi)&=&\tau\circ\sigma(\axi\phi)\\
    \tau\circ\sigma(\pi)&=&\tau(\axi\sigma(\phi))\\
    \mbox{eq.~(\ref{logic:eqn:FUAP:localinversion:inversion:1})}\ \rightarrow
    &=&\axi\tau^{*}\circ\sigma(\phi)\\
    &=&\axi\phi\\
    &=&\pi
    \end{eqnarray*}
Next we assume that $\pi=\pi_{1}\pon\pi_{2}$ with
$\tau\circ\sigma(\pi_{1})=\pi_{1}$ and
$\tau\circ\sigma(\pi_{2})=\pi_{2}$. Then:
    \begin{eqnarray*}
    \tau\circ\sigma(\pi)&=&\tau\circ\sigma(\pi_{1}\pon\pi_{2})\\
    &=&\tau(\,\sigma(\pi_{1})\pon\,\sigma(\pi_{2})\,)\\
    \mbox{eq.~(\ref{logic:eqn:FUAP:localinversion:inversion:1})}\ \rightarrow
    &=&\tau\circ\sigma(\pi_{1})\pon\,\tau\circ\sigma(\pi_{2})\\
    &=&\pi_{1}\pon\pi_{2}\\
    &=&\pi
    \end{eqnarray*}
Since $V=\emptyset$ there is nothing to check in the case when
$\pi=\gen x\pi_{1}$. So this completes our proof when $V=\emptyset$,
and we now assume that $V\neq\emptyset$. Let $x_{0}\in V$ and
consider $\tau_{0}:\sigma(V_{0})\to V$ and
$\tau_{1}:\sigma(V_{1})\to V$ to be the inverse mappings of
$\sigma_{|V_{0}}$ and $\sigma_{|V_{1}}$ respectively. Extend
$\tau_{0}$ and $\tau_{1}$ to the whole of $W$ by setting
$\tau_{0}(u)=x_{0}$ if $u\not\in\sigma(V_{0})$ and
$\tau_{1}(u)=x_{0}$ if $u\not\in\sigma(V_{1})$. Then $\tau_{0},
\tau_{1}:W\to V$ are left-inverse of $\sigma$ on $V_{0}$ and $V_{1}$
respectively, i.e. we have:
    \begin{eqnarray*}
    (i)&&x\in V_{0}\ \Rightarrow\ \tau_{0}\circ\sigma(x)=x\\
    (ii)&&x\in V_{1}\ \Rightarrow\ \tau_{1}\circ\sigma(x)=x
    \end{eqnarray*}
Let $\tau:{\bf\Pi}(W)\to\pvs$ be the dual variable substitution
associated with the ordered pair $(\tau_{0},\tau_{1})$ as per
definition~(\ref{logic:def:FUAP:localinversion:dualsubst}). We shall
complete the proof of the theorem by showing that
$\tau\circ\sigma(\pi)=\pi$ for all $\pi\in\Pi$ where:
    \[
    \Pi=\{\pi\in\pvs:(\free(\pi)\subseteq
    V_{0})\land(\bound(\pi)\subseteq V_{1})\land(\mbox{$\sigma$
    valid for $\pi$})\}
    \]
So let $\pi\in\Pi$. Then in particular $\pi$ satisfies property
$(iii)$ and $(iv)$ of
lemma~(\ref{logic:lemma:FUAP:localinversion:Main}) in the particular
case when $U=\emptyset$. So applying
lemma~(\ref{logic:lemma:FUAP:localinversion:Main}) for
$U=\emptyset$, we see that
$\tau^{\circ}(\sigma(\pi))(W\setminus\sigma(\emptyset))=\pi$ where
$\tau^{\circ}:{\bf\Pi}(W)\to[{\cal P}(W)\to\pvs]$ is the map
associated with $\tau$. Hence we conclude that
$\tau\circ\sigma(\pi)=\tau^{\circ}(\sigma(\pi))(W)=\pi$.
\end{proof}
