It all began with our desire to carry over sequents from
$\Gamma\vdash\phi$ to $\sigma(\Gamma)\vdash\sigma(\phi)$. We had no
control over the axioms potentially being used in a proof underlying
a sequent. Without validity, we could not say anything sensible
about proofs which arise from a blind substitution of variables. We
needed variable substitutions to avoid capture at all times. We knew
the minimal extension $\bar{\sigma}:\bar{V}\to\bar{W}$ was always
valid for the minimal transform ${\cal M}(\phi)$ of any formula
$\phi$. It was clear we needed minimal transforms to be extended
from formulas to proofs. This extension has now been done. We have a
minimal transform for proofs and sure enough $\bar{\sigma}$ is
always valid for ${\cal M}(\pi)$, as can be seen from
proposition~(\ref{logic:prop:FUAP:mintransformproof:minextension:valid}).
We also have the equality $\bar{\sigma}\circ{\cal M}(\pi)={\cal
M}\circ\sigma(\pi)$ whenever $\sigma$ is valid for $\pi$, as follows
from theorem~(\ref{logic:the:FUAP:mintransvalidsub:commute}) of
page~\pageref{logic:the:FUAP:mintransvalidsub:commute}. In fact, a
pattern has emerged: we have found ourselves shamelessly duplicating
results from formulas to proofs, while providing identical
arguments. There is an obvious parallel between the free algebras
\pv\ and \pvs. This should not be too surprising. After all, both
are formal languages with variable binding and similar signatures.
However, the realization of this close parallel between \pv\ and
\pvs\ is raising a new question: how far can it go? Can we define an
essential substitution $\sigma:\pvs\to{\bf \Pi}(W)$ for proofs just
as we did for formulas? Can we define a substitution congruence on
\pvs? In fact, this last question has never occurred to us. Of
course we know the argument: 'Let $x$ such that $p\,(x)$\ldots
therefore $q(x)$. QED' does not rely on what the variable $x$ really
is. There is clearly a notion of $\alpha$-equivalence for proofs.
But why should we care? We are interested in provability. We want to
establish sequents $\Gamma\vdash\phi$. We do not necessarily want to
spend too much time investigating little niceties of congruences on
\pvs. There needs to be a real purpose. So here it is: it all begins
with our desire to carry over sequents from $\Gamma\vdash\phi$ to
$\sigma(\Gamma)\vdash\sigma(\phi)$. It now occurs to us that
essential substitutions could also be defined for proofs. This would
be a powerful tool: an essential substitution $\sigma:\pvs\to{\bf
\Pi}(W)$ would allow us to generate a new proof $\sigma(\pi)$ from
an older proof $\pi$, and this without variable capture. This is
exactly what we need. However, we cannot speak of {\em essential}
substitutions for proofs without formalizing the notion of what {\em
essential} is. So we need to define a substitution congruence for
proofs, which is the purpose of this section.

Now we have a choice: unlike the situation we were facing when
dealing with formulas, we now have a few tools and some insight at
our disposal. In particular, it is very tempting to define a
substitution congruence $\equiv$ on \pvs\ simply with the equality
${\cal M}(\pi)={\cal M}(\rho)$ in light of
theorem~(\ref{logic:the:FOPL:mintransfsubcong:kernel}) of
page~\pageref{logic:the:FOPL:mintransfsubcong:kernel}. It is clear
that whatever choices we make, the end result should be that
$\alpha$-equivalence for proofs be characterized by minimal
transforms, just as it is for formulas. So one option is to start
directly from there. However, we can also define a substitution
congruence on \pvs\ from a generator, following
definition~(\ref{logic:def:sub:congruence}). We shall adopt the
latter which is the safe option, simply following a trail:

\index{congruence@Substitution congruence for
proofs}\index{alpha@$\alpha$-equivalence for proofs}
\begin{defin}\label{logic:def:FUAP:subcong:substitution:congruence}
Let $V$ be a set. We call substitution congruence on \pvs\ the
congruence on \pvs\ generated by $R_{0}\cup R_{1}\cup
R_{2}\subseteq\pvs\times\pvs$ where the set $R_{0}$ and $R_{1}$ are
defined by $R_{0}=\{(\phi,\psi):\phi\sim\psi\}$,
$R_{1}=\{(\axi\phi,\axi\psi):\phi\sim\psi\}$ and:
    \[
    R_{2}=\{\ (\gen x\pi_{1},\gen y\pi_{1}[y\!:\!x]):\pi_{1}\in\pvs\ ,\ x,y\in V\
    ,\ x\neq y\ ,\ y\not\in\free(\pi_{1})\ \}
    \]
where $\sim$ is the substitution congruence on \pv, and $[y\!:\!x]$
as per {\em definition~(\ref{logic:def:single:var:permutation}).}
\end{defin}
Let $\sim$ and $\equiv$ denote the substitution congruence on \pv\
and \pvs\ respectively. Let $\phi,\psi\in\pv$. If $\phi\sim\psi$
then the ordered pair $(\phi,\psi)$ is an element of $R_{0}$ and in
particular $\phi\equiv\psi$. However, if $\phi\equiv\psi$ it is not
obvious at this stage that $\phi\sim\psi$. The equivalence shall be
proved later in
proposition~(\ref{logic:prop:FUAP:charsubsong:equivalence:formula}).
In the meantime, we shall be very careful to avoid any possible
confusion between $\phi\sim\psi$ and $\phi\equiv\psi$.

Following definition~(\ref{logic:def:admissible:substitution}), we
shall now introduce the notion of {\em admissible substitutions} for
proofs. Admissible substitutions $\sigma:V\to W$ only apply to the
case when $W=V$. Admissibility is a stronger notion than validity,
as we also require that $\sigma(u)=u$ for all $u\in\free(\pi)$.
Admissible substitutions have proved useful as a tool to establish
cases of $\alpha$-equivalence $\phi\sim\psi$ in the case when
$\psi=\sigma(\phi)$ for some $\sigma:V\to V$. The proof simply
relies on showing that $\sigma$ is admissible for $\phi$. We shall
establish a similar criterion for proofs.


\index{admissible@Admissible substitution for proof}
\begin{defin}\label{logic:def:FUAP:subcong:admissible:substitution}
Let $V$ be a set and $\sigma:V\to V$ be a map. Let $\pi\in\pvs$. We
say that $\sigma$ is an {\em admissible substitution for $\pi$}
\ifand\ it satisfies:
    \begin{eqnarray*}
    (i)&&\mbox{$\sigma$ valid for $\pi$}\\
    (ii)&&\forall u\in\free(\pi)\ ,\ \sigma(u)=u
    \end{eqnarray*}
\end{defin}
Given a formula $\phi\in\pv$ and a map $\sigma:V\to V$, the
terminology $\sigma$ is {\em admissible for} $\phi$ is potentially
ambiguous. Since $\pv\subseteq\pvs$, it may refer to the usual
notion of admissibility as per
definition~(\ref{logic:def:admissible:substitution}), or to the new
notion of
definition~(\ref{logic:def:FUAP:subcong:admissible:substitution})
applied to the proof $\pi=\phi$ . Luckily, the two notions coincide.
The following proposition is the counterpart of
proposition~(\ref{logic:prop:admissible:sub:congruence}). Note
however that we shall not be able to provide a counterpart of
proposition~(\ref{logic:prop:sub:congruence:from:admissible}) which
shows that ordered pairs $(\phi,\sigma(\phi))$ with $\sigma$
admissible for $\phi$ form a generator of the substitution
congruence on \pv. This is one of the rare cases we shall encounter,
where the parallel between formulas and proofs is broken. However,
this will be of little significance and all other expected results
will hold.



\begin{prop}\label{logic:prop:FUAP:subcong:admissible:subcong}
Let $\equiv$ be the substitution congruence on \pvs\ where $V$ is a
set. Let $\pi\in\pvs$ and $\sigma:V\to V$ be an admissible
substitution for $\pi$. Then:
    \[
    \pi\equiv\sigma(\pi)
    \]
\end{prop}
\begin{proof}
We need to show the property $\forall\sigma[\,(\sigma\mbox{\
admissible for\ }\pi)\ \Rightarrow\ \pi\equiv\sigma(\pi)\,]$ for all
$\pi\in\pvs$. We shall do so by structural induction, using
theorem~(\ref{logic:the:proof:induction}) of
page~\pageref{logic:the:proof:induction}. First we assume that
$\pi=\phi$ for some $\phi\in\pv$. We need to show the property is
true for $\phi$. So we assume that $\sigma$ is admissible for
$\phi$. Then from
proposition~(\ref{logic:prop:admissible:sub:congruence}) we obtain
$\phi\sim\sigma(\phi)$, where $\sim$ denotes the substitution
congruence on \pv. In particular, the ordered pair
$(\phi,\sigma(\phi))$ is an element of the set $R_{0}$ of
definition~(\ref{logic:def:FUAP:subcong:substitution:congruence}).
It follows that $\phi\equiv\sigma(\phi)$ as requested. We now assume
that $\pi=\axi\phi$ for some $\phi\in\pv$. We need to show the
property is true for $\pi$. So we assume that $\sigma$ is admissible
for $\pi$. From
proposition~(\ref{logic:prop:FUAP:validsubproof:recursion:axiom}) it
follows that $\sigma$ is valid for $\phi$. Since
$\free(\pi)=\free(\phi)$ we see that $\sigma$ is in fact admissible
for $\phi$. Using
proposition~(\ref{logic:prop:admissible:sub:congruence}) once more,
we obtain $\phi\sim\sigma(\phi)$. Hence we see that the ordered pair
$(\axi\phi,\axi\sigma(\phi))=(\pi,\sigma(\pi))$ is an element of the
set $R_{1}$ of
definition~(\ref{logic:def:FUAP:subcong:substitution:congruence}).
In particular we obtain $\pi\equiv\sigma(\pi)$ as requested. We now
check that the property is true for $\pi=\pi_{1}\pon\pi_{2}$ if it
is true for $\pi_{1},\pi_{2}\in\pvs$. So we assume that $\sigma:V\to
V$ is an admissible substitution for $\pi$. We need to show that
$\pi\equiv\sigma(\pi)$. Since $\pi=\pi_{1}\pon\pi_{2}$ and
$\sigma(\pi)=\sigma(\pi_{1})\pon\,\sigma(\pi_{2})$, the substitution
congruence being a congruent relation on \pvs, it is sufficient to
show that $\pi_{1}\equiv\sigma(\pi_{1})$ and
$\pi_{2}\equiv\sigma(\pi_{2})$. First we show that
$\pi_{1}\equiv\sigma(\pi_{1})$. Having assumed the property is true
for $\pi_{1}$, it is sufficient to show that $\sigma$ is an
admissible substitution for $\pi_{1}$. Since $\sigma$ admissible for
$\pi$, in particular it is valid for $\pi$ and it follows from
proposition~(\ref{logic:prop:FUAP:validsubproof:recursion:pon}) that
it is also valid for $\pi_{1}$. So it remains to show that
$\sigma(u)=u$ for all $u\in\free(\pi_{1})$ which follows immediately
from $\free(\pi)=\free(\pi_{1})\cup\free(\pi_{2})$ and the fact that
$\sigma(u)=u$ for all $u\in\free(\pi)$. So we have proved that
$\pi_{1}\equiv\sigma(\pi_{1})$ and we show similarly that
$\pi_{2}\equiv\sigma(\pi_{2})$. We now need to check that the
property is true for $\pi=\gen x\pi_{1}$ if it is true for
$\pi_{1}\in\pvs$. So we assume that $\sigma:V\to V$ is an admissible
substitution for $\pi$. We need to show that $\pi\equiv\sigma(\pi)$.
We shall distinguish two cases: first we assume that $\sigma(x)=x$.
Then $\sigma(\pi)=\gen x\,\sigma(\pi_{1})$ and in order to show
$\pi\equiv\sigma(\pi)$, the substitution congruence being a
congruent relation on \pvs, it is sufficient to show that
$\pi_{1}\equiv\sigma(\pi_{1})$. Having assumed the property is true
for $\pi_{1}$, it is therefore sufficient to prove that $\sigma$ is
an admissible substitution for $\pi_{1}$. Since $\sigma$ admissible
for $\pi$, in particular it is valid for $\pi$ and it follows from
proposition~(\ref{logic:prop:FUAP:validsubproof:recursion:gen}) that
it is also valid for $\pi_{1}$. So it remains to show that
$\sigma(u)=u$ for all $u\in\free(\pi_{1})$. We shall distinguish two
further cases: first we assume that $u=x$. Then $\sigma(u)=u$ is
true from our assumption $\sigma(x)=x$. So we assume that $u\neq x$.
It follows that $u\in\free(\pi_{1})\setminus\{x\}=\free(\pi)$, and
since $\sigma$ is admissible for $\pi$, we conclude that
$\sigma(u)=u$. This completes our proof of $\pi\equiv\sigma(\pi)$ in
the case when $\sigma(x)=x$. We now assume that $\sigma(x)\neq x$.
Let $y=\sigma(x)$. Then $\sigma(\pi)=\gen y\,\sigma(\pi_{1})$ and we
need to show that $\gen x\pi_{1}\equiv\gen y\,\sigma(\pi_{1})$.
However, since $[y\!:\!x]\circ[y\!:\!x]$ is the identity mapping we
have $\sigma=[y\!:\!x]\circ\sigma^{*}$ where the map
$\sigma^{*}:V\to V$ is defined as $\sigma^{*}=[y\!:\!x]\circ\sigma$.
It follows that $\sigma(\pi_{1})=\sigma^{*}(\pi_{1})[y\!:\!x]$ and
we need to show that $\gen x\pi_{1}\equiv\gen
y\,\sigma^{*}(\pi_{1})[y\!:\!x]$. Let us accept for now that
$y\not\in\free(\sigma^{*}(\pi_{1}))$. Then from
definition~(\ref{logic:def:FUAP:subcong:substitution:congruence}) we
obtain $\gen x\sigma^{*}(\pi_{1})\equiv\gen
y\sigma^{*}(\pi_{1})[y\!:\!x]$, and it is therefore sufficient to
prove that $\gen x\pi_{1}\equiv\gen x\sigma^{*}(\pi_{1})$. So we see
that it is sufficient to prove $\pi_{1}\equiv\sigma^{*}(\pi_{1})$
provided we can justify the fact that
$y\not\in\free(\sigma^{*}(\pi_{1}))$. First we show that
$\pi_{1}\equiv\sigma^{*}(\pi_{1})$. Having assumed our property is
true for $\pi_{1}$ it is sufficient to prove that $\sigma^{*}$ is
admissible for $\pi_{1}$. However, we have already seen that
$\sigma$ is valid for $\pi_{1}$. Furthermore, since $[y\!:\!x]$ is
an injective map, from
proposition~(\ref{logic:prop:FUAP:validsubproof:injective}) it is a
valid substitution for $\sigma(\pi_{1})$. It follows from
proposition~(\ref{logic:prop:FUAP:validsubproof:composition}) that
$\sigma^{*}=[y\!:\!x]\circ\sigma$ is valid for $\pi_{1}$. So in
order to prove that $\sigma^{*}$ is admissible for $\pi_{1}$, it
remains to show that $\sigma^{*}(u)=u$ for all $u\in\free(\pi_{1})$.
So let $u\in\free(\pi_{1})$. We shall distinguish two cases: first
we assume that $u=x$. Then
$\sigma^{*}(u)=[y\!:\!x](\sigma(x))=[y\!:\!x](y)=x=u$. Next we
assume that $u\neq x$. Then $u$ is in fact an element of
$\free(\pi)$. Having assumed $\sigma$ is admissible for $\pi$ we
obtain $\sigma(u)=u$. We also obtain the fact that $\sigma$ is valid
for $\pi=\gen x\pi_{1}$ and consequently $\sigma(u)\neq\sigma(x)$,
i.e. $u\neq y$. Thus
$\sigma^{*}(u)=[y\!:\!x](\sigma(u))=[y\!:\!x](u)=u$. This completes
our proof that $\sigma^{*}$ is admissible for $\pi_{1}$ and
$\pi_{1}\equiv\sigma^{*}(\pi_{1})$. It remains to show that
$y\not\in\free(\sigma^{*}(\pi_{1}))$. So suppose to the contrary
that $y\in\free(\sigma^{*}(\pi_{1}))$. We shall obtain a
contradiction. Using
proposition~(\ref{logic:prop:FUAP:freevarproof:substitution:inclusion})
there exists $u\in\free(\pi_{1})$ such that $y=\sigma^{*}(u)$.
Having proven that $\sigma^{*}$ is admissible for $\pi_{1}$ we have
$\sigma^{*}(u)=u$ and consequently $y=u\in\free(\pi_{1})$. From the
assumption $y=\sigma(x)\neq x$ we in fact have $y\in\free(\pi)$. So
from the admissibility of $\sigma$ for $\pi$ we obtain $\sigma(y)=y$
and furthermore from the validity of $\sigma$ for $\pi=\gen
x\pi_{1}$ we obtain $\sigma(y)\neq\sigma(x)$. So we conclude that
$y\neq\sigma(x)$ which contradicts our very definition of $y$.
\end{proof}
