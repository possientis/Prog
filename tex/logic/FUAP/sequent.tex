Having constructed the free universal algebra of proofs \pvs, we are
now ready to define the notion of provability and syntactic
entailment. We shall introduce a binary {\em consequence} relation
$\vdash\,\subseteq {\cal P}(\pv)\times\pv$ as follows:

\index{proof@Proof of a formula}\index{gamma@$\Gamma, \Delta$ :
subsets of \pv}\index{sequent@Sequent and
provability}\index{provability@Provability from
hypothesis}\index{gamma@$\Gamma\vdash\phi$ : $\Gamma$ entails
$\phi$}\index{phi@$\vdash\phi$ : $\phi$ is
provable}\index{consequence@Consequence relation $\vdash$}
\begin{defin}\label{logic:def:FOPL:proof:of:formula}
Let $V$ be a set. Let $\Gamma\subseteq\pv$ and $\phi\in\pv$. We say
that $\pi\in\pvs$ is a {\em proof of the formula $\phi$ from
$\Gamma$} \ifand\ we have:
    \[
    \val(\pi)=\phi\mbox{\ and\ }\hyp(\pi)\subseteq\Gamma
    \]
Furthermore, we say that $\phi$ is {\em provable from $\Gamma$} or
that $\Gamma$ {\em entails} $\phi$, and we write:
    \[
    \Gamma\vdash\phi
    \]
\ifand\ there exists a proof $\pi\in\pvs$ of the formula $\phi$ from
$\Gamma$. We say that $\phi$ is {\em provable} and we write
$\vdash\phi$ \ifand\ it is provable from $\Gamma=\emptyset$.
\end{defin}

The relation $\vdash$ satisfies a few properties which are
highlighted in~\cite{Rivieccio}: the {\em identity property} is
satisfied as we clearly have $\{\phi\}\vdash\phi$ for all
$\phi\in\pv$. Indeed, the proof $\pi=\phi$ is a proof of $\phi$ from
the set $\{\phi\}$. The {\em monotonicity} property is also
satisfied as we have the implication
$(\Gamma\supseteq\Delta)\land(\Delta\vdash\phi)\ \Rightarrow\
\Gamma\vdash\phi$. Indeed, if $\pi\in\pvs$ is a proof with
$\val(\pi)=\phi$ and $\hyp(\pi)\subseteq\Delta$ then in particular
we have $\hyp(\pi)\subseteq\Gamma$ and consequently $\pi$ is also a
proof of $\phi$ from $\Gamma$. Another property satisfied by
$\vdash$ is the {\em finitary} property: if $\Gamma\vdash\phi$ then
there exists a finite subset $\Gamma_{0}\subseteq\Gamma$ such that
$\Gamma_{0}\vdash\phi$. Indeed, if $\pi$ is a proof of $\phi$ from
$\Gamma$, then it is also a proof of $\phi$ from
$\Gamma_{0}=\hyp(\pi)\subseteq\Gamma$. Other properties such as {\em
transitivity} and {\em structurality} are considered in Umberto
Rivieccio~\cite{Rivieccio}. We shall prove {\em transitivity} after
we have established the deduction
theorem~(\ref{logic:the:FOPL:deduction}) of
page~\pageref{logic:the:FOPL:deduction}. The property of {\em
structurality} is slightly more delicate in the case of first order
logic with terms. We would like to be able to claim that
$\sigma(\Gamma)\vdash\sigma(\phi)$ whenever we have
$\Gamma\vdash\phi$ and $\sigma:\pv\to\pv$ is a map of a certain
type. The existing literature on abstract algebraic logic e.g. W.J.
Blok and D. Pigozzi~\cite{BlokPigozzi} and the already
mentioned~\cite{Rivieccio} seem to consider logical systems without
terms, where the algebra of formulas is similar to that of
propositional logic. In our case, things are slightly more
complicated as it is not immediately obvious which type of
$\sigma:\pv\to\pv$ should be used to define the {\em structurality}
property. However, with a little bit of thought, it seems that a
pretty good candidate is to consider the class of essential
substitutions $\sigma:\pv\to\pv$ as per
definition~(\ref{logic:def:FOPL:esssubstprop:essential}). In fact,
we hope to be able to prove a {\em substitution theorem}\,:
$\Gamma\vdash\phi\ \Rightarrow\ \ \sigma(\Gamma)\vdash\sigma(\phi)$
in the more general case when $\sigma:\pv\to{\bf P}(W)$ is an
essential substitution and $W$ is a set of variables, which may not
be equal to $V$. Such theorem is already stated in~\cite{AlgLog} as
Theorem~4.3 page~33 using the notion of {\em semi-homomorphism}. For
now, we shall complete this section by proving other properties of
the consequence relation~$\vdash$. The idea behind this is to
develop a naive and heuristic form of sequent calculus allowing us
to prove a sequent $\Gamma\vdash\phi$ as easily as possible. The
claim that '\,$\Gamma\vdash\phi$\,' (i.e. '\,{\em $\Gamma$ entails
$\phi$}\,') is that there exists a proof $\pi$ of $\phi$ from
$\Gamma$. It is usually a painful exercise to exhibit an element
$\pi\in\pvs$ such that $\val(\pi)=\phi$ and
$\hyp(\pi)\subseteq\Gamma$. It is a lot easier to prove the
existence of such a $\pi$ without actually saying what it is. The
following propositions will allow us to do that. Of course, this
will not be very helpful if we are interested in proof searching
algorithms at a later stage. However, it would be wrong to think
that {\em proof searching} requires that we {\em find} a proof
$\pi\in\pvs$. Another approach consists in attempting to formally
prove or falsify the sequent $\Gamma\vdash\phi$, possibly along the
lines of the Gentzen system described in Jean H.
Gallier~\cite{Gallier} or Gilles Dowek~\cite{Dowek}, using another
algebra of proofs based on another language of sequents. The next
proposition establishes that the sequent $\Gamma\vdash\phi$ is
always true when $\phi$ is an axiom of first order logic\,:


\begin{prop}\label{logic:prop:FOPL:axiom}
Let $V$ be a set and $\Gamma\subseteq\pv$. Let $\phi\in\av$ be an
axiom of first order logic. Then $\phi$ is provable from $\Gamma$,
i.e. we have $\Gamma\vdash\phi$.
\end{prop}
\begin{proof}
Given an axiom of first order logic $\phi\in\av$, consider the proof
$\pi=\axi\phi$. Then $\val(\pi) = \phi$ and $\hyp(\pi)=\emptyset$.
In particular, we have $\val(\pi) = \phi$ and
$\hyp(\pi)\subseteq\Gamma$. It follows that $\pi\in\pvs$ is a proof
of $\phi$ from $\Gamma$, so $\Gamma\vdash\phi$.
\end{proof}

One way to look at proposition~(\ref{logic:prop:FOPL:axiom}) is to
say that to every axiom $\phi$ of first order logic is associated a
group of axioms $\Gamma\vdash\phi$ in the language of sequents. An
axiom of the form $\Gamma\vdash\phi$ can be viewed as an inference
rule of arity~$0$, that is a constant operator on a new algebra of
proofs by sequents. We are not claiming to be doing anything here,
but simply indicating an avenue for future development. In a similar
fashion, the modus ponens rule of the algebra \pvs\ gives rise to a
binary rule of inference in the language of sequents:

\begin{prop}\label{logic:prop:FOPL:modus:ponens}
Let $V$ be a set and $\Gamma\subseteq\pv$. For all $\phi,\psi\in\pv$
we have:
    \[
    (\Gamma\vdash\phi)\land(\Gamma\vdash (\phi\to\psi))\ \Rightarrow
    \Gamma\vdash\psi
    \]
\end{prop}
\begin{proof}
Let $V$ be a set and $\Gamma\subseteq\pv$. Let $\phi,\psi\in\pv$ be
formulas such that $\Gamma\vdash\phi$ and
$\Gamma\vdash(\phi\to\psi)$. From $\Gamma\vdash\phi$ we obtain the
existence of a proof $\pi_{1}\in\pvs$ such that $\val(\pi_{1})=\phi$
and $\hyp(\pi_{1})\subseteq\Gamma$. From $\Gamma\vdash(\phi\to\psi)$
we obtain the existence of a proof $\pi_{2}\in\pvs$ such that
$\val(\pi_{2})= \phi\to\psi$ and $\hyp(\pi_{2})\subseteq\Gamma$.
Define the proof $\pi=\pi_{1}\pon\pi_{2}$. Then, from
definition~(\ref{logic:def:FOPL:proof:valuation}) we have:
    \begin{eqnarray*}
    \val(\pi)&=&\val(\pi_{1}\pon\pi_{2})\\
        &=&M(\val(\pi_{1}),\val(\pi_{2}))\\
        &=&M(\phi,\phi\to\psi)\\
        &=&\psi
    \end{eqnarray*}
where $M:\pv^{2}\to\pv$ is the modus ponens mapping as per
definition~(\ref{logic:def:FOPL:modus:ponens}). Furthermore, using
definition~(\ref{logic:def:FOPL:hypothesis}) we obtain:
    \begin{eqnarray*}
    \hyp(\pi)&=&\hyp(\pi_{1}\pon\pi_{2})\\
    &=&\hyp(\pi_{1})\cup\hyp(\pi_{2})\\
    &\subseteq&\Gamma
    \end{eqnarray*}
It follows that $\pi\in\pvs$ is a proof of $\psi$ from $\Gamma$ and
we have proved that $\Gamma\vdash\psi$.
\end{proof}

Similarly, the generalization rule of the algebra \pvs\ gives rise
to a unary rule of inference in the language of sequents.

\begin{prop}\label{logic:prop:FOPL:generalization}
Let $V$ be a set and $\Gamma\subseteq\pv$. For all $\phi\in\pv$ and
$x\in V$:
    \[
    (\Gamma\vdash\phi)\land (x\not\in\free(\Gamma))\ \Rightarrow\
    \Gamma\vdash\forall x \phi
    \]
\end{prop}
\begin{proof}
Let $V$ be a set and $\Gamma\subseteq\pv$. Let $\phi\in\pv$ and
$x\in V$ be such that $\Gamma\vdash\phi$ and
$x\not\in\free(\Gamma)$. From $\Gamma\vdash\phi$ we obtain the
existence of a proof $\pi_{1}\in\pvs$ such that $\val(\pi_{1})=\phi$
and $\hyp(\pi_{1})\subseteq\Gamma$. From $x\not\in\free(\Gamma)$ and
$\hyp(\pi_{1})\subseteq\Gamma$ it follows in particular that
$x\not\in\spec(\pi_{1})$. Define the proof $\pi=\gen
x\pi_{1}\in\pvs$. Then:
    \begin{eqnarray*}
    \val(\pi)&=&\val(\gen x\pi_{1})\\
    x\not\in\spec(\pi_{1})\ \to\ &=&\forall x\val(\pi_{1})\\
    &=&\forall x\phi
    \end{eqnarray*}
Furthermore, we have $\hyp(\pi) = \hyp(\gen
x\pi_{1})=\hyp(\pi_{1})\subseteq\Gamma$. It follows that
$\pi\in\pvs$ is a proof of $\forall x\phi$ from $\Gamma$ and we
conclude that $\Gamma\vdash\forall x\phi$.
\end{proof}

Every axiom of first order logic is of the form
$\phi=\phi_{1}\to\phi_{2}$. Hence, given $\Gamma\subseteq\pv$ from
proposition~(\ref{logic:prop:FOPL:axiom}) we have
$\Gamma\vdash\phi_{1}\to\phi_{2}$. It follows that if the sequent
$\Gamma\vdash\phi_{1}$ has been proved, we obtain
$\Gamma\vdash\phi_{2}$ immediately by application of the modus
ponens property of proposition~(\ref{logic:prop:FOPL:modus:ponens}).
So every axiom of first order logic also gives rise to a unary rule
of inference in the language of sequents. We shall now make these
explicit by considering each of the five possible groups of axioms
separately. Of course, these unary rules of inference can be deduced
from other existing rules. So we may wish to regard them simply as
{\em theorems} of the form '$\Gamma\vdash\phi_{1}\ \Rightarrow\
\Gamma\vdash\phi_{2}$' in the language of sequents.

\begin{prop}\label{logic:prop:FOPL:simplification}
Let $V$ be a set and $\Gamma\subseteq\pv$. For all
$\phi_{1},\phi_{2}\in\pv$:
    \[
    \Gamma\vdash\phi_{1}\ \Rightarrow\ \Gamma\vdash (\phi_{2}\to\phi_{1})
    \]
\end{prop}
\begin{proof}
Let $V$ be a set and $\Gamma\subseteq\pv$. Let
$\phi_{1},\phi_{2}\in\pv$ such that $\Gamma\vdash\phi_{1}$. From
definition~(\ref{logic:def:FOPL:simplification:axiom}) the formula
$\phi_{1}\to(\phi_{2}\to\phi_{1})$ is a simplification axiom and it
follows from proposition~(\ref{logic:prop:FOPL:axiom}) that
$\Gamma\vdash\phi_{1}\to(\phi_{2}\to\phi_{1})$. Hence, using
$\Gamma\vdash\phi_{1}$ and the modus ponens property of
proposition~(\ref{logic:prop:FOPL:modus:ponens}) we conclude that
$\Gamma\vdash(\phi_{2}\to\phi_{1})$.
\end{proof}


\begin{prop}\label{logic:prop:FOPL:Frege}
Let $V$ be a set and $\Gamma\subseteq\pv$. For all
$\phi_{1},\phi_{2},\phi_{3}\in\pv$:
    \[
    \Gamma\vdash\phi_{1}\to(\phi_{2}\to\phi_{3})\ \Rightarrow\
    \Gamma\vdash (\phi_{1}\to\phi_{2})\to(\phi_{1}\to\phi_{3})
    \]
\end{prop}
\begin{proof}
Let $V$ be a set and $\Gamma\subseteq\pv$. Let
$\phi_{1},\phi_{2},\phi_{3}\in\pv$ with
$\Gamma\vdash\phi_{1}\to(\phi_{2}\to\phi_{3})$. From
definition~(\ref{logic:def:FOPL:frege:axiom}),
$[\phi_{1}\to(\phi_{2}\to\phi_{3})]\to[(\phi_{1}\to\phi_{2})\to(\phi_{1}\to\phi_{3})]$
is a Frege axiom and it follows from
proposition~(\ref{logic:prop:FOPL:axiom}) that:
    \[
    \Gamma\vdash [\phi_{1}\to(\phi_{2}\to\phi_{3})]\to[(\phi_{1}\to\phi_{2})
    \to(\phi_{1}\to\phi_{3})]
    \]
Hence, using $\Gamma\vdash\phi_{1}\to(\phi_{2}\to\phi_{3})$ and the
modus ponens property of
proposition~(\ref{logic:prop:FOPL:modus:ponens}) we conclude that
$\Gamma\vdash(\phi_{1}\to\phi_{2})\to(\phi_{1}\to\phi_{3})$.
\end{proof}

\begin{prop}\label{logic:prop:FOPL:transposition}
Let $V$ be a set and $\Gamma\subseteq\pv$. For all $\phi_{1}\in\pv$
we have:
    \[
    \Gamma\vdash(\phi_{1}\to\bot)\to\bot\ \Rightarrow\
    \Gamma\vdash\phi_{1}
    \]
\end{prop}
\begin{proof}
Let $V$ be a set and $\Gamma\subseteq\pv$. Let $\phi_{1}\in\pv$ such
that $\Gamma\vdash(\phi_{1}\to\bot)\to\bot$. From
definition~(\ref{logic:def:FOPL:transposition:axiom}) the formula
$[(\phi_{1}\to\bot)\to\bot]\to\phi_{1}$ is a transposition axiom and
it follows from proposition~(\ref{logic:prop:FOPL:axiom}) that
$\Gamma\vdash[(\phi_{1}\to\bot)\to\bot]\to\phi_{1}$. Hence, using
$\Gamma\vdash(\phi_{1}\to\bot)\to\bot$ and the modus ponens property
of proposition~(\ref{logic:prop:FOPL:modus:ponens}) we conclude that
$\Gamma\vdash\phi_{1}$.
\end{proof}

\begin{prop}\label{logic:prop:FOPL:quantification}
Let $V$ be a set and $\Gamma\subseteq\pv$. For all
$\phi_{1},\phi_{2}\in\pv$, $x\in V$:
    \[
    (x\not\in\free(\phi_{1}))\land (\Gamma\vdash\forall x(\phi_{1}\to\phi_{2}))\ \Rightarrow\
    \Gamma\vdash\phi_{1}\to\forall x\phi_{2}
    \]
\end{prop}
\begin{proof}
Let $V$ be a set and $\Gamma\subseteq\pv$. Let
$\phi_{1},\phi_{2}\in\pv$ and $x\in V$ such that
$x\not\in\free(\phi_{1})$ and $\Gamma\vdash\forall
x(\phi_{1}\to\phi_{2})$. From
definition~(\ref{logic:def:FOPL:quantification:axiom}) and
$x\not\in\free(\phi_{1})$ the formula $\forall
x(\phi_{1}\to\phi_{2})\to(\phi_{1}\to\forall x\phi_{2})$ is a
quantification axiom and it follows from
proposition~(\ref{logic:prop:FOPL:axiom}) that $\Gamma\vdash\forall
x(\phi_{1}\to\phi_{2})\to(\phi_{1}\to\forall x\phi_{2})$. Hence,
using $\Gamma\vdash\forall x(\phi_{1}\to\phi_{2})$ and the modus
ponens property of proposition~(\ref{logic:prop:FOPL:modus:ponens})
we conclude that $\Gamma\vdash\phi_{1}\to\forall x\phi_{2}$.
\end{proof}

\begin{prop}\label{logic:prop:FOPL:specialization}
Let $V$ be a set and $\Gamma\subseteq\pv$. For all $\phi_{1}\in\pv$,
$x,y\in V$:
    \[
    \Gamma\vdash\forall x\phi_{1}\ \Rightarrow\
    \Gamma\vdash\phi_{1}[y/x]
    \]
where $[y/x]:\pv\to\pv$ is an essential substitution of $y$ in place
of $x$.
\end{prop}
\begin{proof}
Let $V$ be a set and $\Gamma\subseteq\pv$. Let $\phi_{1}\in\pv$ and
$x,y\in V$ such that $\Gamma\vdash\forall x\phi_{1}$. Let
$[y/x]:\pv\to\pv$ be an essential substitution associated with
$[y/x]:V\to V$ as per
definition~(\ref{logic:def:FOPL:esssubst:esssubst}). From
definition~(\ref{logic:def:FOPL:specialization:axiom}) the formula
$\forall x\phi_{1}\to\phi_{1}[y/x]$ is a specialization axiom and it
follows from proposition~(\ref{logic:prop:FOPL:axiom}) that
$\Gamma\vdash\forall x\phi_{1}\to\phi_{1}[y/x]$. Hence, using
$\Gamma\vdash\forall x \phi_{1}$ and the modus ponens property of
proposition~(\ref{logic:prop:FOPL:modus:ponens}) we conclude that
$\Gamma\vdash\phi_{1}[y/x]$.
\end{proof}

The following proposition can also be viewed as an axiom in the
language of sequents. We shall need it when proving the deduction
theorem.

\begin{prop}\label{logic:prop:FOPL:PimP}
Let $V$ be a set and $\Gamma\subseteq\pv$. For all $\phi_{1}\in\pv$
we have:
    \[
    \Gamma\vdash(\phi_{1}\to\phi_{1})
    \]
\end{prop}
\begin{proof}
Let $V$ be a set and $\Gamma\subseteq\pv$. Let $\phi_{1}\in\pv$.
Since $\phi_{1}\to(\phi_{1}\to\phi_{1})$ is a simplification axiom,
from proposition~(\ref{logic:prop:FOPL:axiom}) we have
$\Gamma\vdash\phi_{1}\to(\phi_{1}\to\phi_{1})$. So in order to show
$\Gamma\vdash(\phi_{1}\to\phi_{1})$, by virtue of the modus ponens
property of proposition~(\ref{logic:prop:FOPL:modus:ponens}) it is
sufficient to prove that:
    \[
    \Gamma\vdash[\phi_{1}\to(\phi_{1}\to\phi_{1})]\to(\phi_{1}\to\phi_{1})
    \]
From proposition~(\ref{logic:prop:FOPL:Frege}) it is therefore
sufficient to prove that:
    \[
    \Gamma\vdash\phi_{1}\to[(\phi_{1}\to\phi_{1})\to\phi_{1}]
    \]
which follows immediately from the fact that
$\phi_{1}\to[(\phi_{1}\to\phi_{1})\to\phi_{1}]$ is also a
simplification axiom.
\end{proof}
