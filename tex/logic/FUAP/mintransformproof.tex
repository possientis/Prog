The following definition introduces a concept of minimal transform
for proofs. Comparing with the existing
definition~(\ref{logic:def:FOPL:mintransform:transform}) of minimal
transform for formulas, we see that our suggestion of ${\cal
M}(\pi)$ for $\pi\in\pvs$ is as natural as it can be. However we saw
that despite its simplicity, this definition failed to give us the
sensible outcome $\val\circ{\cal M}(\pi)\sim{\cal M}\circ\val(\pi)$
where $\sim$ is the substitution congruence on \pvb. Nevertheless,
we stood our ground and decided to review our deductive system
rather than contemplate an alternative definition of minimal
transform for proofs. We defined a new valuation $\vals:\pvs\to\pv$
which was flexible enough on $\alpha$-equivalence, while leading to
an equivalent set of sequents $\Gamma\vdash\phi$ as demonstrated by
proposition~(\ref{logic:prop:FUAP:valuationmod:equivalence}). This
new valuation will allow us to write $\vals\circ{\cal
M}(\pi)\sim{\cal M}\circ\vals(\pi)$ whenever $\pi$ is a clean proof,
a fact we shall prove in
proposition~(\ref{logic:prop:FUAP:mintransproof:valuation:commute}).
Needless to say that minimal transforms for proofs are likely to be
a key notion just as they were for formulas. We shall be able to
show the existence of {\em essential substitutions} for proofs of
definition~(\ref{logic:def:FUAP:esssubstprop:essential}) and
establish the substitution
theorem~(\ref{logic:the:FUAP:substitutiontheorem:main}) of
page~\pageref{logic:the:FUAP:substitutiontheorem:main}, allowing us
to carry over the sequent $\Gamma\vdash\phi$ into
$\sigma(\Gamma)\vdash\sigma(\phi)$ whenever $\sigma:\pv\to{\bf
P}(W)$ is an essential substitution. This is a beautiful result we
think, as it no longer contains the ugly caveats on variable capture
which are so common in mathematical textbooks. Note that the
following definition of minimal transform abides with a key
principle which we have adopted throughout these notes. Since
$\pv\subseteq\pvs$, we made sure this new definition of ${\cal
M}(\pi)$ is compatible with the existing
definition~(\ref{logic:def:FOPL:mintransform:transform}) whenever
$\pi=\phi$ for some $\phi\in\pv$. Thus, when confronted with the
notation ${\cal M}(\phi)$ we do not need to worry as to whether
$\phi$ is regarded as a formula or as a proof. We are now ready to
quote:

\index{minimal@Minimal transform of proof}\index{m@${\cal M}(\pi)$ :
minimal transform of $\pi$}
\begin{defin}\label{logic:def:FUAP:mintransproof:transform}
Let $V$ be a set with minimal extension $\bar{V}$. We call {\em
minimal transform mapping on \pvs\ }the map ${\cal
M}:\pvs\to{\bf\Pi}(\bar{V})$ defined by:
\begin{equation}\label{logic:eqn:FUAP:mintransproof:transform:1}
    \forall\pi\in\pvs\ ,\ {\cal M}(\pi)=\left\{
                    \begin{array}{lcl}
                    {\cal M}(\phi)&\mbox{\ if\ }&\pi=\phi\in\pv\\
                    \axi{\cal M}(\phi)&\mbox{\ if\ }&\pi=\axi\phi\\
                    {\cal M}(\pi_{1})\pon\,{\cal M}(\pi_{2})&\mbox{\ if\ }&\pi=\pi_{1}\pon\pi_{2}\\
                    \gen n{\cal M}(\pi_{1})[n/x]&\mbox{\ if\ }&\pi=\gen
                    x\pi_{1}
                    \end{array}\right.
    \end{equation}
where $n=\min\{k\in\N:\mbox{$[k/x]$ valid for ${\cal
M}(\pi_{1})$}\}$.
\end{defin}
\begin{prop}\label{logic:prop:FUAP:mintransproof:recursion}
The structural recursion of {\em
definition~(\ref{logic:def:FUAP:mintransproof:transform})} is
legitimate.
\end{prop}
\begin{proof}
We need to show the existence and uniqueness of the map ${\cal
M}:\pvs\to{\bf\Pi}(\bar{V})$ satisfying the four conditions
of~(\ref{logic:eqn:FUAP:mintransproof:transform:1}). We shall do so
using theorem~(\ref{logic:the:structural:recursion}) of
page~\pageref{logic:the:structural:recursion}. So we take $X=\pvs$
with free generator $X_{0}=\pv$ and we choose $A={\bf\Pi}(\bar{V})$.
We define $g_{0}:X_{0}\to A$ by setting $g_{0}(\phi)={\cal
M}(\phi)\in\pvb\subseteq A$, which takes care of the first
condition. For every $\phi\in\pv$, we define the map
$h(\axi\phi):A^{0}\to A$ by setting $h(\axi\phi)(0)=\axi{\cal
M}(\phi)$ which takes care of the second condition. We also define
$h(\pon):A^{2}\to A$ by setting
$h(\pon)(\pi_{1},\pi_{2})=\pi_{1}\pon\pi_{2}$ which takes care of
the third condition. Finally, for all $x\in V$ we define the map
$h(\gen x):A^{1}\to A$ by setting $h(\gen x)(\pi_{1})=\gen
n\pi_{1}[n/x]$, where:
    \[
    n=n(\gen x)(\pi_{1})=\min\{k\in\N\ :\ [k/x]\mbox{ valid for }\pi_{1}\}
    \]
\end{proof}

As we shall see, many properties of minimal transforms apply to both
proofs and formulas. The following property does not. By showing the
equality $\spec({\cal M}(\pi))=\spec(\pi)$ for clean proofs, we
shall be able to argue that minimal transforms of clean proofs are
clean. Note that the equality fails when $\pi$ is not clean. For
example, if $\pi=\gen x(x\in x)$ we have ${\cal M}(\pi)=\gen\,0(0\in
0)$. Thus, we see that $\spec(\pi)=\{x\}$ while $\spec({\cal
M}(\pi))=\{0\}$ and $\{x\}\neq\{0\}$ since $V\cap\N=\emptyset$.


\begin{prop}\label{logic:prop:FUAP:mintransproof:variable}
Let $V$ be a set and $\pi\in\pvs$ be a clean proof. Then:
    \[
    \spec({\cal M}(\pi))=\spec(\pi)
    \]
\end{prop}
\begin{proof}
For every proof $\pi\in\pvs$ we need to show the following
implication:
    \[
    (\mbox{$\pi$ clean})\ \Rightarrow\ \spec({\cal M}(\pi))=\spec(\pi)
    \]
We shall do so with a structural induction, using
theorem~(\ref{logic:the:proof:induction}) of
page~\pageref{logic:the:proof:induction}. First we assume that
$\pi=\phi$ for some $\phi\in\pv$. Then $\pi$ is always a clean
proof, and we need to show that $\free({\cal M}(\phi))=\free(\phi)$
which follows from
proposition~(\ref{logic:prop:FOPL:mintransform:variables}). Next we
assume that $\pi=\axi\phi$ for some $\phi\in\pv$. Then we have:
    \[
    \spec({\cal M}(\pi))=\spec(\,\axi{\cal
    M}(\phi)\,)=\emptyset=\spec(\pi)
    \]
Whether or not $\pi$ is a clean proof, i.e. whether or not $\phi$ is
an axiom modulo is irrelevant in this case. So we now assume that
$\pi=\pi_{1}\pon\pi_{2}$ where $\pi_{1},\pi_{2}\in\pvs$ are proofs
which satisfy our implication. We need to show the same is true of
$\pi$. So we assume that $\pi$ is a clean proof. From
proposition~(\ref{logic:prop:FUAP:almostclean:modus:ponens}) it
follows in particular that both $\pi_{1}$ and $\pi_{2}$ are clean
proofs. Consequently:
    \begin{eqnarray*}
    \spec({\cal M}(\pi))&=&\spec(\,{\cal M}(\pi_{1}\pon\pi_{2})\,)\\
    &=&\spec(\,{\cal M}(\pi_{1})\pon\,{\cal M}(\pi_{2})\,)\\
    \mbox{prop.~(\ref{logic:prop:FUAP:freevar:recursive:def})}\ \rightarrow
    &=&\spec({\cal M}(\pi_{1}))\cup\spec({\cal M}(\pi_{2}))\\
    \mbox{$\pi_{1},\pi_{2}$ clean}\ \rightarrow
    &=&\spec(\pi_{1})\cup\spec(\pi_{2})\\
    \mbox{prop.~(\ref{logic:prop:FUAP:freevar:recursive:def})}\ \rightarrow
    &=&\spec(\pi_{1}\pon\pi_{2})\\
    &=&\spec(\pi)
    \end{eqnarray*}
So we now assume that $\pi=\gen x\pi_{1}$ for some $x\in V$ and
$\pi_{1}\in\pvs$ which satisfies our implication. We need to show
the same is true of $\pi$. So we assume that $\pi$ is a clean proof.
From proposition~(\ref{logic:prop:FUAP:almostclean:generalization})
it follows that $\pi_{1}$ is itself a clean proof and furthermore
$x\not\in\spec(\pi_{1})$. Hence, we have the following equalities:
    \begin{eqnarray*}
    \spec({\cal M}(\pi))&=&\spec({\cal M}(\gen x\pi_{1}))\\
    n=\min\{k:[k/x]\mbox{\ valid for\ }{\cal M}(\pi_{1})\}\
    \rightarrow
    &=&\spec(\,\gen n{\cal M}(\pi_{1})[n/x]\,)\\
    \mbox{prop.~(\ref{logic:prop:FUAP:freevar:recursive:def})}\ \rightarrow
    &=&\spec(\,{\cal M}(\pi_{1})[n/x]\,)\\
    \mbox{prop.~(\ref{logic:prop:FUAP:validsubproof:specvar}),
    $[n/x]$ valid for ${\cal M}(\pi_{1})$}\ \rightarrow
    &=&[n/x](\,\spec({\cal M}(\pi_{1}))\,)\\
    \mbox{$\pi_{1}$ clean}\ \rightarrow
    &=&[n/x](\,\spec(\pi_{1})\,)\\
    x\not\in\spec(\pi_{1})\ \rightarrow
    &=&\spec(\pi_{1})\\
    \mbox{prop.~(\ref{logic:prop:FUAP:freevar:recursive:def})}\ \rightarrow
    &=&\spec(\gen x\pi_{1})\\
    &=&\spec(\pi)
    \end{eqnarray*}
\end{proof}

We have spent a lot of time discussing the importance of the
equivalence $\vals\circ{\cal M}(\pi)\sim{\cal M}\circ\vals(\pi)$,
where $\sim$ is the substitution congruence on \pvb. The notion of
minimal transform ${\cal M}(\pi)$ cannot be interesting unless we
can say something sensible about its conclusion or conclusion
modulo. However, we should not forget that controlling the
hypothesis of a proof is as important as controlling its conclusion.
So we need to make sure $\hyp({\cal M}(\pi))$ is also sensible.

\begin{prop}\label{logic:prop:FUAP:mintransproof:hypothesis}
Let $V$ be a set and $\pi\in\pvs$ be a clean proof. Then:
    \[
    \hyp({\cal M}(\pi))={\cal M}(\hyp(\pi))
    \]
\end{prop}
\begin{proof}
For every proof $\pi\in\pvs$ we need to show the following
implication:
    \[
    (\mbox{$\pi$ clean})\ \Rightarrow\ \hyp({\cal M}(\pi))={\cal M}(\hyp(\pi))
    \]
We shall do so with a structural induction, using
theorem~(\ref{logic:the:proof:induction}) of
page~\pageref{logic:the:proof:induction}. First we assume that
$\pi=\phi$ for some $\phi\in\pv$. Then $\pi$ is always clean in this
case and we simply need to prove the equality which goes as follows:
    \[
    \hyp({\cal M}(\phi))=\{{\cal M}(\phi)\}={\cal M}(\{\phi\})={\cal M}(\hyp(\phi))
    \]
We now assume that $\pi=\axi\phi$ for some $\phi\in\pv$. Then we
have:
    \begin{eqnarray*}
    \hyp({\cal M}(\pi))&=&\hyp({\cal M}(\axi\phi))\\
    &=&\hyp(\,\axi{\cal M}(\phi)\,)\\
    &=&\emptyset\\
    &=&{\cal M}(\emptyset)\\
    &=&{\cal M}(\hyp(\axi\phi))\\
    &=&{\cal M}(\hyp(\pi))\\
    \end{eqnarray*}
So the equality is always true and so is the implication. So we now
assume that $\pi=\pi_{1}\pon\pi_{2}$ where $\pi_{1},\pi_{2}\in\pvs$
are proofs satisfying our implication. We need to show the same is
true of $\pi$. So we assume that $\pi$ is clean. We need to show the
equality is true for $\pi$. However, from
proposition~(\ref{logic:prop:FUAP:almostclean:modus:ponens}) both
$\pi_{1}$ and $\pi_{2}$ are clean and the equality is therefore true
for $\pi_{1}$ and $\pi_{2}$. Hence:
    \begin{eqnarray*}
    \hyp({\cal M}(\pi))&=&\hyp({\cal M}(\pi_{1}\pon\pi_{2}))\\
    &=&\hyp(\,{\cal M}(\pi_{1})\pon\,{\cal M}(\pi_{2})\,)\\
    &=&\hyp({\cal M}(\pi_{1}))\cup\hyp({\cal M}(\pi_{2}))\\
    &=&{\cal M}(\hyp(\pi_{1}))\cup{\cal M}(\hyp(\pi_{2}))\\
    &=&{\cal M}(\,\hyp(\pi_{1})\cup\hyp(\pi_{2})\,)\\
    &=&{\cal M}(\hyp(\pi_{1}\pon\pi_{2}))\\
    &=&{\cal M}(\hyp(\pi))\\
    \end{eqnarray*}
We now assume that $\pi=\gen x\pi_{1}$ where $x\in V$ and
$\pi_{1}\in\pvs$ is a proof satisfying our implication. We need to
show the same is true of $\pi$. So we assume that $\pi$ is clean. We
need to show the equality is true for $\pi$. However, from
proposition~(\ref{logic:prop:FUAP:almostclean:generalization}) we
see that $\pi_{1}$ is clean and $x\not\in\spec(\pi_{1})$. Hence:
    \begin{eqnarray*}
    \hyp({\cal M}(\pi))&=&\hyp({\cal M}(\gen x\pi_{1}))\\
    &=&\hyp(\,\gen n{\cal M}(\pi_{1})[n/x]\,)\\
    &=&\hyp({\cal M}(\pi_{1})[n/x])\\
    \mbox{prop.~(\ref{logic:prop:FUAP:substitution:hypothesis})}\ \rightarrow
    &=&[n/x](\,\hyp({\cal M}(\pi_{1}))\,)\\
    &=&[n/x](\,{\cal M}(\hyp(\pi_{1}))\,)\\
    \mbox{A: to be proved}\ \rightarrow
    &=&{\cal M}(\hyp(\pi_{1}))\\
    &=&{\cal M}(\hyp(\gen x\pi_{1}))\\
    &=&{\cal M}(\hyp(\pi))\\
    \end{eqnarray*}
So it remains to prove point A, for which is it sufficient to show
the equality $[n/x](\psi)=\psi$ for all $\psi\in{\cal
M}(\hyp(\pi_{1}))$. So it is sufficient to show the equality
$[n/x]\circ{\cal M}(\phi)={\cal M}(\phi)$ for all
$\phi\in\hyp(\pi_{1})$. So let $\phi\in\hyp(\pi_{1})$. Using
proposition~(\ref{logic:prop:substitution:support}), we simply need
to show that $[n/x](u)=u$ for all $u\in\var({\cal M}(\phi))$. It is
therefore sufficient to prove that $x\not\in\var({\cal M}(\phi))$.
So suppose to the contrary that $x\in\var({\cal M}(\phi))$. Since
$x\in V$ we have $x\in\var({\cal M}(\phi))\cap V$ and it follows
from proposition~(\ref{logic:prop:FOPL:mintransform:variables}) that
$x\in\free(\phi)$ while $\phi\in\hyp(\pi_{1})$. Hence we see that
$x\in\spec(\pi_{1})$ which is a contradiction, and completes our
induction argument.
\end{proof}

The whole point of minimal transforms is to create logical copies of
formulas and proofs in which free and bound variables have been
segregated. The free variables remain in the set $V$ while the bound
variables belong to a copy of \N\ which is disjoint from $V$. The
following is the counterpart of
proposition~(\ref{logic:prop:FOPL:mintransform:variables}):

\begin{prop}\label{logic:prop:FUAP:mintransformproof:freevar}
Let $V$ be a set and $\pi\in\pvs$. Then we have the equality:
    \[
    \free({\cal M}(\pi))=\var({\cal M}(\pi))\cap V=\free(\pi)
    \]
where ${\cal M}(\pi)\in{\bf\Pi}(\bar{V})$ is the minimal transform
of $\pi\in\pvs$.
\end{prop}
\begin{proof}
We shall prove this equality with a structural induction, using
theorem~(\ref{logic:the:proof:induction}) of
page~\pageref{logic:the:proof:induction}. First we assume that
$\pi=\phi$ for some $\phi\in\pv$. Then the equality follows
immediately from
proposition~(\ref{logic:prop:FOPL:mintransform:variables}). So we
now assume that $\pi=\axi\phi$ for some $\phi\in\pv$. Then using
proposition~(\ref{logic:prop:FOPL:mintransform:variables}) once more
we obtain:
    \begin{eqnarray*}
    \var({\cal M}(\pi))\cap V&=&\var({\cal M}(\axi\phi))\cap V\\
    &=&\var(\axi{\cal M}(\phi))\cap V\\
    &=&\var({\cal M}(\phi))\cap V\\
    \mbox{prop.~(\ref{logic:prop:FOPL:mintransform:variables})}\ \rightarrow
    &=&\free(\phi)\\
    &=&\free(\axi\phi)\\
    &=&\free(\pi)\\
    \end{eqnarray*}
as well as:
    \begin{eqnarray*}
    \free({\cal M}(\pi))&=&\free({\cal M}(\axi\phi))\\
    &=&\free(\axi{\cal M}(\phi))\\
    &=&\free({\cal M}(\phi))\\
    \mbox{prop.~(\ref{logic:prop:FOPL:mintransform:variables})}\ \rightarrow
    &=&\free(\phi)\\
    &=&\free(\axi\phi)\\
    &=&\free(\pi)\\
    \end{eqnarray*}
So we now assume that $\pi=\pi_{1}\pon\pi_{2}$ where
$\pi_{1},\pi_{2}\in\pvs$ are proofs satisfying our equality. We need
to show the same is true of $\pi$. On the one hand we have:
    \begin{eqnarray*}
    \var({\cal M}(\pi))\cap V&=&\var({\cal M}(\pi_{1}\pon\pi_{2}))\cap V\\
    &=&\var(\,{\cal M}(\pi_{1})\pon\,{\cal M}(\pi_{2})\,)\cap V\\
    &=&(\,\var({\cal M}(\pi_{1}))\cup\var({\cal M}(\pi_{2}))\,)\cap V\\
    &=&\var({\cal M}(\pi_{1}))\cap V\,\cup\,\var({\cal M}(\pi_{2}))\cap V\\
    &=&\free(\pi_{1})\cup\free(\pi_{2})\\
    &=&\free(\pi_{1}\pon\pi_{2})\\
    &=&\free(\pi)\\
    \end{eqnarray*}
while on the other hand:
    \begin{eqnarray*}
    \free({\cal M}(\pi))&=&\free({\cal M}(\pi_{1}\pon\pi_{2}))\\
    &=&\free(\,{\cal M}(\pi_{1})\pon\,{\cal M}(\pi_{2})\,)\\
    &=&\free({\cal M}(\pi_{1}))\cup\free({\cal M}(\pi_{2}))\\
    &=&\free(\pi_{1})\cup\free(\pi_{2})\\
    &=&\free(\pi_{1}\pon\pi_{2})\\
    &=&\free(\pi)\\
    \end{eqnarray*}
We now assume that $\pi=\gen x\pi_{1}$ where $x\in V$ and
$\pi_{1}\in\pvs$ is a proof satisfying our equality. We need to show
the same is true of $\pi$. On the one hand we have:
    \begin{eqnarray*}
    \var({\cal M}(\pi))\cap V&=&\var({\cal M}(\gen x\pi_{1}))\cap V\\
    &=&\var(\,\gen n{\cal M}(\pi_{1})[n/x]\,)\cap V\\
    &=&(\,\{n\}\cup\var(\,{\cal M}(\pi_{1})[n/x]\,)\,)\cap V\\
    n\not\in V\ \rightarrow
    &=&\var(\,{\cal M}(\pi_{1})[n/x]\,)\cap V\\
    \mbox{prop.~(\ref{logic:prop:FUAP:variable:substitution})}\ \rightarrow
    &=&[n/x](\,\var({\cal M}(\pi_{1}))\,)\cap V\\
    n\not\in V\ \rightarrow
    &=&[n/x](\,\var({\cal M}(\pi_{1}))\setminus\{x\}\,)\cap V\\
    u\neq x\Rightarrow[n/x](u)=u\ \rightarrow
    &=&(\,\var({\cal M}(\pi_{1}))\setminus\{x\}\,)\cap V\\
    &=&(\,\var({\cal M}(\pi_{1}))\cap V\,)\setminus\{x\}\\
    &=&\free(\pi_{1})\setminus\{x\}\\
    &=&\free(\gen x\pi_{1})\\
    &=&\free(\pi)
    \end{eqnarray*}
while on the other hand:
    \begin{eqnarray*}
    \free({\cal M}(\pi))&=&\free({\cal M}(\gen x\pi_{1}))\\
    n=\min\{k:[k/x]\mbox{\ valid for\ }{\cal M}(\pi_{1})\}\
    \rightarrow&=&\free(\gen n{\cal M}(\pi_{1})[n/x])\\
    &=&\free({\cal M}(\pi_{1})[n/x])\setminus\{n\}\\
    \mbox{$[n/x]$ valid, prop.~(\ref{logic:prop:FUAP:validsubproof:freevar})}
    \ \rightarrow
    &=&[n/x](\,\free({\cal M}(\pi_{1}))\,)\setminus\{n\}\\
    \mbox{$[n/x](x)=n$}\ \rightarrow
    &=&[n/x](\,\free({\cal
    M}(\pi_{1}))\setminus\{x\}\,)\setminus\{n\}\\
    u\neq x\Rightarrow\ [n/x](u)=u\ \rightarrow
    &=&\free({\cal M}(\pi_{1}))\setminus\{x\}\setminus\{n\}\\
    &=&\free(\pi_{1})\setminus\{x\}\setminus\{n\}\\
    n\not\in V\ \rightarrow
    &=&\free(\pi_{1})\setminus\{x\}\\
    &=&\free(\gen x\pi_{1})\\
    &=&\free(\pi)
    \end{eqnarray*}
\end{proof}

The following is the counterpart of
proposition~(\ref{logic:prop:FOPL:mintransform:variables:bound})\,:
\begin{prop}\label{logic:prop:FUAP:mintransformproof:boundvar}
Let $V$ be a set and $\pi\in\pvs$. Then we have the equality:
    \[
    \bound({\cal M}(\pi))=\var({\cal M}(\pi))\cap\N
    \]
where ${\cal M}(\pi)\in{\bf\Pi}(\bar{V})$ is the minimal transform
of $\pi\in\pvs$.
\end{prop}
\begin{proof}
We shall prove this equality with a structural induction, using
theorem~(\ref{logic:the:proof:induction}) of
page~\pageref{logic:the:proof:induction}. First we assume that
$\pi=\phi$ for some $\phi\in\pv$. Then the equality follows
immediately from
proposition~(\ref{logic:prop:FOPL:mintransform:variables:bound}).
Next we assume that $\pi=\axi\phi$ for some $\phi\in\pv$. Then we
have the following equalities:
    \begin{eqnarray*}
    \bound({\cal M}(\pi))&=&\bound({\cal M}(\axi\phi))\\
    &=&\bound(\axi{\cal M}(\phi))\\
    &=&\bound({\cal M}(\phi))\\
    \mbox{prop.~(\ref{logic:prop:FOPL:mintransform:variables:bound})}\ \rightarrow
    &=&\var({\cal M}(\phi))\cap\N\\
    &=&\var(\axi{\cal M}(\phi))\cap\N\\
    &=&\var({\cal M}(\axi\phi))\cap\N\\
    &=&\var({\cal M}(\pi))\cap\N\\
    \end{eqnarray*}
So we now assume that $\pi=\pi_{1}\pon\pi_{2}$ where
$\pi_{1},\pi_{2}\in\pvs$ are proofs satisfying our equality. We need
to show the same is true of $\pi$, which goes as follows:
    \begin{eqnarray*}
    \bound({\cal M}(\pi))&=&\bound({\cal M}(\pi_{1}\pon\pi_{2}))\\
    &=&\bound(\,{\cal M}(\pi_{1})\pon\,{\cal M}(\pi_{2})\,)\\
    &=&\bound({\cal M}(\pi_{1}))\cup\bound({\cal M}(\pi_{2}))\\
    &=&(\var({\cal M}(\pi_{1}))\cap\N)\cup(\var({\cal M}(\pi_{2}))\cap\N)\\
    &=&(\,\var({\cal M}(\pi_{1}))\cup\var({\cal M}(\pi_{2}))\,)\cap\N\\
    &=&\var(\,{\cal M}(\pi_{1})\pon\,{\cal M}(\pi_{2})\,)\cap\N\\
    &=&\var({\cal M}(\pi_{1}\pon\pi_{2}))\cap\N\\
    &=&\var({\cal M}(\pi))\cap\N\\
    \end{eqnarray*}
We now assume that $\pi=\gen x\pi_{1}$ where $x\in V$ and
$\pi_{1}\in\pvs$ is a proof satisfying our equality. We need to show
the same is true for $\pi$ which goes as follows:
    \begin{eqnarray*}
    \bound({\cal M}(\pi))&=&\bound({\cal M}(\gen x\pi_{1}))\\
    &=&\bound(\,\gen n{\cal M}(\pi_{1})[n/x]\,)\\
    &=&\{n\}\cup\bound({\cal M}(\pi_{1})[n/x])\\
    \mbox{prop.~(\ref{logic:prop:FUAP:boundvarproof:substitution})}\ \rightarrow
    &=&\{n\}\cup[n/x](\,\bound({\cal M}(\pi_{1}))\,)\\
    \mbox{$[n/x](x)=n$}\ \rightarrow
    &=&\{n\}\cup[n/x](\,\bound({\cal M}(\pi_{1}))\setminus\{x\}\,)\\
    u\neq x\Rightarrow\ [n/x](u)=u\ \rightarrow
    &=&\{n\}\cup\bound({\cal M}(\pi_{1}))\setminus\{x\}\\
    &=&\{n\}\cup(\,\var({\cal M}(\pi_{1}))\cap\N\,)\setminus\{x\}\\
    &=&(\,\{n\}\cup\var({\cal M}(\pi_{1}))\setminus\{x\}\,)\cap\N\\
    u\neq x\Rightarrow\ [n/x](u)=u\ \rightarrow
    &=&(\,\{n\}\cup[n/x](\,\var({\cal M}(\pi_{1}))\setminus\{x\}\,)\,)\cap\N\\
    \mbox{$[n/x](x)=n$}\ \rightarrow
    &=&(\,\{n\}\cup[n/x](\,\var({\cal M}(\pi_{1}))\,)\,)\cap\N\\
    \mbox{prop.~(\ref{logic:prop:FUAP:variable:substitution})}\ \rightarrow
    &=&(\,\{n\}\cup\var({\cal M}(\pi_{1})[n/x])\,)\cap\N\\
    &=&\var(\,\gen n{\cal M}(\pi_{1})[n/x]\,)\cap\N\\
    &=&\var({\cal M}(\gen x\pi_{1}))\cap\N\\
    &=&\var({\cal M}(\pi))\cap\N\\
    \end{eqnarray*}
\end{proof}
