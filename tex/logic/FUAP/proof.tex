Most textbook references we know about define proofs as finite
sequences of formulas, and inference rules as ordered pairs
$(\Phi,\phi)$ where $\Phi$ is a finite sequence of formulas and
$\phi$ is a formula. A good illustration of this is Definition 1.6
of page 12 in~\cite{Ferenczi}. As far as we can tell, this approach
does not work. Having proofs as finite sequences of formulas means
that in order to establish any result, the only tool available to us
is induction on the length of the proof. This makes it
excruciatingly painful and messy to prove anything and is far
inferior to the seamless flow of a proof by structural induction.
Furthermore, denying the structure of a free algebra to the set of
proofs means we cannot use structural recursion to define anything.
There are many interesting functionals which could be defined on the
set of proofs. It is a shame to give that up. Granted, finite
sequences of formulas are easily understood. Everyone likes them for
this reason. Not everyone is familiar with the machinery of
universal algebras which require some effort to acquire. But it is
well worth it. Now when it comes to rules of inference, the
structure $(\Phi,\phi)$ is also inadequate. Indeed, it does not
allow {\em generalization} to be expressed in a sensible way. In
Definition 2.10 of page 22 in~\cite{Ferenczi}, the {\em
generalization} is introduced as $((\phi_{1}),\forall x\phi_{1})$.
In other words, if the formula $\phi_{1}$ has been proved, so has
the formula $\forall x\phi_{1}$. Surely this cannot be right. And
yet this is the accepted approach by many authors such as~Marc
Hoyois~\cite{Hoyois}, J. Donald Monk~\cite{Monk} and George
Tourlakis~\cite{Tourlakis}. Our favorite web site~\cite{Metamath}
also has it, as can be seen on
\texttt{http://us.metamath.org/mpegif/ax-gen.html}. Even if we are
willing to disregard the uncomfortable paradox of claiming $\forall
x\phi_{1}$ is proved whenever $\phi_{1}$ is, we know this approach
is failing because the deduction theorem is no longer true for
non-closed formulas. Granted there are other logical systems we are
told, where the deduction theorem does not work. Logicians are free
to do what they want. In this document we care about classical first
order logic, meta-mathematics, the language of {\bf ZF} and we are
looking for mathematical objects which offer a natural formalization
of standard mathematical arguments. It is clear the deduction
theorem has to hold. Suppose we need to prove a statement of the
form $\forall x(p(x)\to q(x))$. A standard mathematician will
proceed as follows: let $x$ be arbitrary. Suppose $p(x)$ holds\ldots
Then a long proof follows\ldots So we see that $q(x)$ is true. Hence
we conclude that $p(x)\to q(x)$ and since $x$ was arbitrary we
obtain $\forall x(p(x)\to q(x))$. This is a standard mathematical
argument, the essence of which is to construct a proof of $q(x)$
from the single hypothesis $p(x)$. So a standard mathematician will
devote most of his energy establishing the sequent $\{p(x)\}\vdash
q(x)$. He will then implicity use the deduction theorem to claim
that $\vdash (p(x)\to q(x))$ and conclude $\vdash\forall x(p(x)\to
q(x))$ from generalization, having rightly argued that $x$ is
arbitrary. So strictly speaking, a standard mathematician does not
write a complete proof $\forall x(p(x)\to q(x))$. Instead, he
resorts to some form of sequent calculus to argue that his
conclusion must be true if the sequent $\{p(x)\}\vdash q(x)$ is
itself true. This whole reasoning is what he regards as a valid
proof. Now we are free to formalize the notion of proof in anyway we
want. We are personally convinced that every theorem of
Metamath~\cite{Metamath} is correct. We are sure Don Monk is right.
However, as far as we are concerned, it is unthinkable to design a
concept of proof for which the arguments of the standard
mathematician fail to be legitimate. We must be able to infer the
sequent $\vdash (p(x)\to q(x))$ from $\{p(x)\}\vdash q(x)$. The
deduction theorem should be true, regardless of whether a formula is
closed or not.

In the light of these discussions, it should be clear by now that we
wish to construct our set of proofs as a free universal algebra $X$
of some type $\alpha$ generated by a set of formula
$X_{0}\subseteq\pv$. On this algebra should be defined two key
semantics $\val:X\to\pv$ and $\hyp:X\to{\cal P}(\pv)$ so that given
a proof $\pi\in X$, the valuation $\val(\pi)$  would represent the
conclusion being proved by $\pi$, while $\hyp(\pi)$ would be the set
of hypothesis. With this in mind, we could then define a consequence
relation $\vdash\,\subseteq{\cal P}(\pv)\times\pv$ as follows:
    \[
    \Gamma\vdash\phi\ \Leftrightarrow\
    (\val(\pi)=\phi)\land(\hyp(\pi)\subseteq\Gamma)\mbox{\ for some
    $\pi\in X$}
    \]
This is the outline of the plan. We now need to provide the
specifics. Our first task is to determine the exact type $\alpha$ of
the universal algebra $X$. In the case of classical first order
logic it would seem natural to have a binary operator $\pon$ for the
rule of inference known as {\em modus ponens}, and a unary operator
$\gen x$ for the {\em generalization} with respect to the variable
$x\in V$. This would work as follows: if $\pi_{1}$ was a proof of
$\phi_{1}$ and $\pi_{2}$ was a proof of $\phi_{1}\to\phi_{2}$, then
$\pi_{1}\pon\pi_{2}$ would be a proof of $\phi_{2}$. This can be
enforced by making sure the valuation $\val:X\to\pv$ satisfies
$\val(\pi_{1}\pon\pi_{2})=\phi_{2}$. Of course we need operators to
be defined everywhere, so the proof $\pi_{1}\pon\pi_{2}$ has to be
meaningful even in the case when the conclusion $\val(\pi_{2})$
cannot be expressed in the form of $\val(\pi_{1})\to\phi_{2}$ for
some formula $\phi_{2}$. But what is the conclusion of a proof
$\pi_{1}\pon\pi_{2}$ which stems from a flawed application of the
modus ponens rule of inference? We clearly haven't proved anything.
So the conclusion of $\pi_{1}\pon\pi_{2}$ should be the weakest
mathematical result of all. In other words, we should simply make
sure the valuation $\val:X\to\pv$ satisfies
$\val(\pi_{1}\pon\pi_{2})=\bot\to\bot$. In the case of
generalization, if $\pi_{1}$ was a proof of $\phi_{1}$, then $\gen
x\pi_{1}$ would be a proof of $\forall x\phi_{1}$, provided the
variable $x$ is truly {\em arbitrary}. This last condition can be
formalized by saying that $x$ is not a free variable of any formula
belonging to $\hyp(\pi_{1})$. We shall write this as
$x\not\in\spec(\pi_{1})$. So whenever this condition holds, we
simply need to make sure the valuation $\val:X\to\pv$ satisfies
$\val(\gen x\pi_{1})=\forall x\val(\pi_{1})$. In the case when
$x\in\spec(\pi_{1})$, then $\gen x\pi_{1}$ constitutes a flawed
application of the generalization rule of inference, and we should
simply set $\val(\gen x\pi_{1})=\bot\to\bot$.

So we have decided to include a binary operator $\pon$ and a unary
operator $\gen x$ for all $x\in V$, in defining the type $\alpha$ of
our free universal algebra of proofs. But what about axioms? If
$\phi\in\av$ is an axiom of first order logic and $\pi\in X$ is a
proof which relies on the axiom $\phi$, how should this axiom be
accounted for? One solution is not to treat axioms in any special
way and regard $\phi$ simply as yet another hypothesis of the proof
$\pi$. In that case we have $\phi\in\hyp(\pi)$. However, this is not
a good solution: for suppose we wish to design a proof of the
formula $\forall x((x\in x)\to(x\in x))$. Conceivably, the only
solution is to design a proof $\pi=\gen x\pi_{1}$ as the
generalization in $x$ of another proof $\pi_{1}$ whose conclusion is
$\val(\pi_{1})=(x\in x)\to(x\in x)$. We shall soon see that finding
a proof $\pi_{1}$ is not difficult. However such a proof relies on
axioms which have $x$ as a free variable. Consequently, if we were
to include axioms as part of the set of hypothesis $\hyp(\pi_{1})$
the condition $x\not\in\spec(\pi_{1})$ would not hold and $\pi=\gen
x\pi_{1}$ would constitute a case of flawed generalization, whose
conclusion is $\val(\pi)=\bot\to\bot$, and not what we set out to
prove. Of course it would be possible to have a set of axioms which
have no free variables, by considering universal closures. However,
this is not the solution we adopted. We chose a set of axioms which
may potentially have free variables, and it is important that we
exclude those axioms from the set of hypothesis of a given proof.
Hence, for every axiom $\phi\in\av$ we shall define a constant
operator $\axi \phi:X^{0}\to X$, so that $\pi=\axi\phi(0)$ is simply
a proof with conclusion $\val(\pi)=\phi$ and the crucial property
$\hyp(\pi)=\emptyset$. In fact, we shall introduce a unary operator
$\axi\phi:X^{0}\to X$ for every formula $\phi\in\pv$ and not simply
for $\phi\in\av$. If $\pi=\axi\phi(0)$ is a proof where $\phi$ is
not a legitimate axiom, we shall simply regard the proof $\pi$ as a
case of {\em flawed invocation of axiom} and set
$\val(\pi)=\bot\to\bot$.

The choice of creating an operator $\axi\phi:X^{0}\to X$ even in the
case when $\phi$ is not a legitimate axiom may seem odd at first
glance. What is the point of doing this? This choice will
effectively increase our free universal algebra of proofs by
allowing {\em flawed} and seemingly pointless proofs of the form
$\axi\phi$ where $\phi$ is not an axiom. Why? Well, firstly we
should realize that this isn't the end of the world. We have already
accepted the idea of having many proofs of the form
$\pi_{1}\pon\pi_{2}$ or $\gen x\pi_{1}$ which are flawed in some
sense, as they are illegitimate use of inference rules and whose
conclusion is $\bot\to\bot$. Allowing the possibility of wrongly
invoking an axiom makes little difference to the existing status.
More importantly, this flexibility brings considerable advantages as
we shall discover in later part of this document. One advantage is
to seamlessly define the notion of variable substitution in proofs.
Given a map $\sigma:V\to W$ we shall define a corresponding
substitution for proofs as per
definition~(\ref{logic:def:FUAP:substitution:substitution}).
Allowing $\axi\phi$ to be meaningful regardless of whether $\phi$ is
a legitimate axiom allows us to set
$\sigma(\axi\phi)=\axi\sigma(\phi)$ without having to worry whether
or not the formula $\sigma(\phi)$ is indeed an axiom. This makes a
lot of the formalism smoother and more elegant. Another advantage is
the ability to study other deductive systems based on an alternative
set of axioms without having to consider a different algebra of
proofs. We can keep the same free universal algebra of proofs and
simply change the semantics $\val:X\to\pv$ giving rise to an
alternative notion of provability.

\index{type@Hilbert deductive proof type}\index{proof@Hilbert
deductive proof type}\index{delta@$\axi\phi$ : the nullary operator
symbol}\index{delta@$\axi\phi$ : the operator $\{0\}\to\pvs$}
\index{delta@$\axi\phi$ : the proof $\axi\phi(0)$}\index{plus@$\pon$
: binary operator symbol}\index{plus@$\pon$ : operator
$\pon:\pvs^{2}\to\pvs$}\index{plus@$\pi_{1}\pon\pi_{2}$ : the proof
in \pvs}\index{axiom@Axiom invocation $\axi\phi$}\index{modus@Modus
ponens operator $\pon$}\index{gen@Generalization operator $\gen x$}
\index{forall@$\gen x$ : unary operator symbol}\index{forall@$\gen
x$ :  $\gen x:\pvs^{1}\to\pvs$}\index{forall@$\gen x\pi_{1}$ : the
proof in \pvs}
\begin{defin}\label{logic:def:FOPL:proof:type}
Let $V$ be a set. We call {\em Hilbert Deductive Proof Type}
associated with $V$, the type of universal algebra $\alpha$ defined
by:
    \[
    \alpha = \{\axi\phi\ :\ \phi\in\pv\}\cup \{\pon\}\cup\{\gen x\ :\ x\in V\}
    \]
where $\axi\phi=((0,\phi),0)$, $\phi\in\pv$, $\pon = ((1,0),2)$ and
$\gen x = ((2,x),1)$, $x\in V$.
\end{defin}

The specifics of how $\axi \phi$, $\pon$ and $\gen x$ are defined
are obviously not important. The coding is done in such a way that
the Hilbert deductive proof type $\alpha$ is indeed a set of ordered
pairs which is functional and with range in \N. So the set $\alpha$
is a map with range in \N, and is therefore a type of universal
algebra as per definition~(\ref{logic:def:type:universal:algebra}).
We have also set up the coding to ensure each operator has the
expected arity, namely $\alpha(\axi \phi)=0$, $\alpha(\pon)=2$ and
$\alpha(\gen x)=1$. We are now ready to define our free universal
algebra of proofs $X$ of type $\alpha$. We shall adopt the generator
$X_{0}=\pv$ and denote this algebra $\pvs$: \index{UA@Free universal
algebra of proofs}\index{free@Free universal algebra of
proofs}\index{pi@$\pvs$ : free algebra of
proofs}\index{pi@$\pi,\rho,\kappa$ : proofs in \pvs}
\begin{defin}\label{logic:def:FOPL:proof:algebra}
Let $V$ be  a set with associated Hilbert deductive proof type
$\alpha$. We call {\em Free Universal Algebra of Proofs} associated
with $V$, the free universal algebra \pvs\ of type $\alpha$ with
free generator \pv.
\end{defin}

Recall that the existence of \pvs\ is guaranteed by
theorem~(\ref{logic:the:main:existence}) of
page~\pageref{logic:the:main:existence}. It follows that we have
$\pv\subseteq\pvs$: any formula $\phi\in\pv$ is also a proof. It is
simply the proof which has itself as a premise, and itself as a
conclusion. In other words, we shall have $\hyp(\phi)=\{\phi\}$ and
$\val(\phi)=\phi$. Note that if $\phi\in\pv$ then both $\phi$ and
$\axi \phi$ are proofs, where we casually use the notation
'$\axi\phi$' as a shortcut for $\axi\phi(0)$. These proofs have the
same conclusion $\phi$, provided the formula $\phi$ is an axiom. The
difference between them is that an axiom has no premise, namely
$\hyp(\axi \phi)=\emptyset$. One other important point needs to be
made: both $\pv$ and $\pvs$ are free universal algebras for which is
defined a natural notion of {\em sub-formula} and {\em sub-proof} as
per definition~(\ref{logic:def:subformula}). So given
$\phi,\psi\in\pv$ and $\pi,\rho\in\pvs$ it is meaningful to write
$\phi\preceq\psi$ and $\pi\preceq\rho$ expressing the idea that
$\psi$ is a sub-formula of $\phi$ and $\pi$ is a sub-proof of
$\rho$. However, since $\pv\subseteq\pvs$ the statement
$\phi\preceq\psi$ is ambiguous. It is not clear whether we are
referring to the sub-formula partial order on \pv\ or \pvs. The
distinction is crucial: the only sub-proof of $\phi$ is $\phi$
itself, while $\phi$ may have many sub-formulas. So we have a
problem for which a solution is to use two different symbols, one
referring to sub-formulas and one referring to sub-proofs. In these
notes, we have decided to keep the same symbol $\preceq$ for both
notions, hoping the context will always make clear which of the two
notions is being considered. In general, we shall want to avoid this
sort of situation. Whenever extending one notion from formulas to
proofs, we shall always make sure the meaning remains the same
regardless of whether $\phi$ is viewed as an element of \pv\ or
\pvs. For example, the variables $\var(\pi)$, free variables
$\free(\pi)$ and bound variables $\bound(\pi)$ of a proof $\pi$ will
be defined with this in mind. The same will apply to the notions of
valid substitution, the image $\sigma(\pi)$ of a proof $\pi$ and its
minimal transform ${\cal M}(\pi)$. So the sub-formula partial order
$\preceq$ is an exception, where a confusion is possible. Recall
that another example of possibly confusing notation exists in this
note namely '$\forall 0$' which may refer to the standard
quantification $\forall x$ with $x=0$, or to the iterated
quantification as per definition~(\ref{logic:def:iterated:quant}).
Before we formally define the set of hypothesis $\hyp(\pi)$ for a
given proof $\pi\in\pvs$, we would like to stress:

\begin{prop}\label{logic:prop:FUAP:proof:axi:injective}
Let $V$ be a set and $\phi,\psi\in\pv$. Then we have:
    \[
    \axi\phi=\axi\psi\ \Rightarrow\ \phi=\psi
    \]
\end{prop}
\begin{proof}
We are now familiar with the free universal algebra of first order
logic \pv\ and a common abuse of notation regarding the symbol
'$\bot$'. This symbol is inherently ambiguous as it may refer to
three different things: firstly, it refers to an operator symbol
namely an element $\bot\in\alpha(V)$ of the first order logic type
of definition~(\ref{logic:def:FOPL:type}). Secondly, it refers to
the actual operator $\bot:\{0\}\to\pv$ of \pv\ with arity
$\alpha(\bot)=0$. Thirdly, it refers to the actual formula $\bot(0)$
which we commonly denote '$\bot$' in order to keep our notations
lighter. Given a formula $\phi\in\pv$, the same thing can be said of
the notation '$\axi\phi$' which is equally ambiguous: it refers to
the operator symbol of definition~(\ref{logic:def:FOPL:proof:type})
or the actual operator $\axi\phi:\{0\}\to\pvs$ or the proof
$\axi\phi(0)$ which we casually denote '$\axi\phi$'. So when it
comes to proposition~(\ref{logic:prop:FUAP:proof:axi:injective})
which we are now attempting to prove, there is potentially a
problem. Luckily the statement of this proposition is true
regardless of how we wish to interpret the notations '$\axi\phi$'
and '$\axi\psi$'. If the two symbols are equal, then the two
operators are equal and consequently the two proofs are equal. So it
is sufficient to prove the following implication:
    \[
    \axi\phi(0)=\axi\psi(0)\ \Rightarrow\ \phi=\psi
    \]
which is our initial statement where the abuse of notation has been
removed. So suppose $\axi\phi(0)=\axi\psi(0)$. Using
theorem~(\ref{logic:the:proof:induction}) of
page~\pageref{logic:the:proof:induction} we obtain the equality
between symbols $\axi\phi=\axi\psi$. However, from
definition~(\ref{logic:def:FOPL:proof:type}) we have
$\axi\phi=((0,\phi),0)$ and $\axi\psi=((0,\psi),0)$. So we conclude
that $\phi=\psi$ as requested.
\end{proof}
\index{hypothesis@Hypothesis of proof}\index{h@$\hyp(\pi)$ : set of
hypothesis of $\pi$}
\begin{defin}\label{logic:def:FOPL:hypothesis}
Let $V$ be a set. The map $\hyp:\pvs\to{\cal P}(\pv)$ defined by the
following structural recursion is called the {\em hypothesis
mapping} on \pvs:
 \begin{equation}\label{logic:eqn:FOPL:hypothesis:recursion}
    \forall\pi\in\pvs\ ,\ \hyp(\pi)=\left\{
                    \begin{array}{lcl}
                    \{\phi\}&\mbox{\ if\ }&\pi=\phi\in\pv\\
                    \ \ \emptyset&\mbox{\ if\ }&\pi=\axi\phi\\
                    \hyp(\pi_{1})\cup\hyp(\pi_{2}) &\mbox{\ if\ }&\pi=\pi_{1}\pon\pi_{2}\\
                    \hyp(\pi_{1})&\mbox{\ if\ }&\pi=\gen x\pi_{1}\\
                    \end{array}\right.
    \end{equation}
We say that $\phi$ is a {\em hypothesis} of the proof $\pi\in\pvs$
\ifand\ $\phi\in\hyp(\pi)$.
\end{defin}
Note that given a proof $\pi\in\pvs$, if we regard $\pi$ as some
formal expression involving variables of $\pv$, then the set of
hypothesis $\hyp(\pi)$ is simply the set of variables occurring in
$\pi$. We should also note that the recursive
definition~(\ref{logic:eqn:FOPL:hypothesis:recursion}) is easily
seen to be legitimate and furthermore that $\hyp(\pi)$ is a finite
set, as a straightforward structural induction will show. Our next
step is to define the valuation mapping $\val:\pvs\to\pv$ which
returns the conclusion $\val(\pi)$ of a proof $\pi\in\pvs$. Before
we can do so we need to define the set $\spec(\pi)$ representing the
free variables occurring in the premises of the proof $\pi$. This
set is important to formally decide whether a proof of the form
$\gen x\pi_{1}$ constitutes a legitimate application of the {\em
generalization} rule of inference. Obviously, having proved the
formula $\val(\pi_{1})$ using a proof $\pi_{1}$, we cannot claim to
have proved $\forall x\val(\pi_{1})$ unless the variable $x$ was
truly arbitrary, that is $x\not\in\spec(\pi_{1})$.

\index{arbitrary@Arbitrary variable
$x\not\in\spec(\pi)$}\index{specific@Specific variables of
proof}\index{s@$\spec(\pi)$ : specific variables of $\pi$}
\index{free@$\free(\Gamma)$ : free variables of $\phi\in\Gamma$}
\begin{defin}\label{logic:def:FOPL:proof:free:variable}
Let $V$ be a set. Given $\Gamma\subseteq\pv$, we say that $x\in V$
is a {\em free variable of $\Gamma$}, \ifand\ it belongs to the set
$\free(\Gamma)$ defined by:
    \[
    \free(\Gamma) = \cup\{\,\free(\phi)\ :\ \phi\in\Gamma\,\}
    \]
Given $\pi\in\pvs$, we say that $x\in V$ is a {\em specific variable
of the proof $\pi$}, \ifand\ it belongs to the set $\spec(\pi)$
defined by the equality:
    \[
    \spec(\pi) = \free(\hyp(\pi))= \cup\{\,\free(\phi)\ :\ \phi\in\hyp(\pi)\,\}
    \]
\end{defin}

We also need to formally decide whether a proof of the form
$\pi_{1}\pon\pi_{2}$ constitutes a legitimate application of the
rule of inference known as {\em modus ponens}. This is the case when
the conclusion $\val(\pi_{2})$ of the proof $\pi_{2}$ can be
expressed as $\val(\pi_{2})=\val(\pi_{1})\to\phi$ for some formula
$\phi\in\pv$. When this is the case, the proof $\pi_{1}\pon\pi_{2}$
is a proof of $\phi$. Otherwise, it simply becomes a proof of
$\bot\to\bot$ which is a very weak result. This motivates the
following: \index{modus@Modus ponens
mapping}\index{m@$M(\phi_{1},\phi_{2})$ : modus ponens map}
\begin{defin}\label{logic:def:FOPL:modus:ponens}
Let $V$ be a set. We call {\em modus ponens mapping} on \pv\ the map
$M:\pv^{2}\rightarrow \pv$ defined by:
    \[
    \forall \phi_{1},\phi_{2}\in\pv\ ,\
    M(\phi_{1},\phi_{2})=\left\{
        \begin{array}{lcl}
        \phi&\mbox{\ if\ }&\phi_{2}=\phi_{1}\to\phi\\
        \bot\to\bot&\mbox{\ otherwise\ }&\\
        \end{array}
    \right.
    \]
\end{defin}

Note that the modus ponens mapping $M:\pv^{2}\to\pv$ is well-defined
by virtue of theorem~(\ref{logic:the:unique:representation}) of
page~\pageref{logic:the:unique:representation} which guarantees the
uniqueness of any $\phi\in\pv$ such that $\phi_{2}=\phi_{1}\to\phi$,
assuming such $\phi$ does exist.
\index{valuation@Valuation mapping
on \pvs}\index{v@$\val$ : $\val:\pvs\to\pv$}\index{v@$\val(\pi)$ :
conclusion of proof $\pi$}\index{semantic@Semantics on \pvs}
\begin{defin}\label{logic:def:FOPL:proof:valuation}
Let $V$ be a set. We call {\em valuation mapping} on \pvs\ the map
$\val:\pvs\to\pv$ defined by the following structural recursion:
\[
    \forall\pi\in\pvs\ ,\ \val(\pi)=\left\{
                    \begin{array}{lcl}
                    \phi&\mbox{\ if\ }&\pi=\phi\in\pv\\
                    \phi&\mbox{\ if\ }&\pi=\axi\phi,\ \phi\in\av\\
                    \bot\to\bot&\mbox{\ if\ }&\pi=\axi\phi,\ \phi\not\in\av\\
                    M(\val(\pi_{1}), \val(\pi_{2})) &\mbox{\ if\ }&\pi=\pi_{1}\pon\pi_{2}\\
                    \forall x\val(\pi_{1})&\mbox{\ if\ }&\pi=\gen
                    x\pi_{1},\  x\not\in\spec(\pi_{1})\\
                    \bot\to\bot&\mbox{\ if\ }&\pi=\gen
                    x\pi_{1},\  x\in\spec(\pi_{1})\\
                    \end{array}\right.
\]
where $M:\pv^{2}\to\pv$ refers to the modus ponens mapping on \pv.
\end{defin}
\begin{prop}
The structural recursion of {\em
definition~(\ref{logic:def:FOPL:proof:valuation})} is legitimate.
\end{prop}
\begin{proof}
We need to prove the existence and uniqueness of the map
$\val:\pvs\to\pv$. We have to be slightly more careful than usual,
as we cannot apply theorem~(\ref{logic:the:structural:recursion}) of
page~\pageref{logic:the:structural:recursion} in this case. The
reason for this is that we do not wish $\val(\gen x\pi_{1})$ to be
simply a function of $\val(\pi_{1})$. Indeed, the conclusion of the
proof $\gen x\pi_{1}$ depends on whether $x\in\spec(\pi_{1})$ or
not. So we want $\val(\gen x\pi_{1})$ to be a function of both
$\val(\pi_{1})$ and $\pi_{1}$. So the main point is to define the
mapping $h(\gen x):\pv\times\pvs\to\pv$ by $h(\gen
x)(\phi_{1},\pi_{1}) =\forall x\phi_{1}$ if $x\not\in\spec(\pi_{1})$
and $h(\gen x)(\phi_{1},\pi_{1})=\bot\to\bot$ otherwise. We can then
apply theorem~(\ref{logic:the:structural:recursion:2}) of
page~\pageref{logic:the:structural:recursion:2}. Note that the
mapping $h(\axi\phi):\pv^{0}\times\pvs^{0}\to\pv$ should be defined
differently, depending on whether $\phi$ is an axiom of first order
logic or not. If $\phi\in\av$ we set $h(\axi\phi)(0,0)=\phi$ and
otherwise $h(\axi\phi)(0,0)=\bot\to\bot$.
\end{proof}
