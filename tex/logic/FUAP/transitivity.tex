We are now in a position to provide a proof of the transitivity
property of the consequence relation $\vdash$\,: if
$\Gamma\vdash\psi$ for all $\psi\in\Delta$ and $\Delta\vdash\phi$,
then we must have $\Gamma\vdash\phi$. In essence, we shall use the
deduction theorem~(\ref{logic:the:FOPL:deduction}) to argue that:
    \[
    \vdash(\psi_{n}\to(\psi_{n-1}\to\ldots(\psi_{1}\to\phi)
    \]
where $\{\psi_{1},\ldots,\psi_{n}\}\subseteq\Delta$ is the set of
hypothesis of a proof underlying the sequent $\Delta\vdash\phi$.
From $\Gamma\vdash\psi_{k}$ and a repeated use of modus ponens, we
conclude that $\Gamma\vdash\phi$. Note that if $\pi$ is a proof of
$\phi$ from $\Delta$, it would be tempting to design a new proof of
$\phi$ from $\Gamma$ by replacing every assumption $\psi_{k}$ of the
proof $\pi$ by a proof $\pi_{k}$ of $\psi_{k}$ from $\Gamma$. For
those who care, this idea does not work: the proof $\pi$ may contain
cases of generalization with respect to a variable $x$ which become
invalidated by the presence of $x$ as a free variable in some
$\hyp(\pi_{k})$. Despite our best efforts, we have not been able to
design a proof along those lines.

\index{transitivity@Transitivity of $\vdash$}
\begin{prop}\label{logic:prop:FOPL:deduction:transitivity}
Let $V$ be a set. Let $\Gamma,\Delta\subseteq\pv$ and $\phi\in\pv$.
Then:
    \begin{equation}\label{logic:eqn:FOPL:deduction:transitivity:1}
    (\Gamma\vdash\psi
    \mbox{\ for all\ }\psi\in\Delta)\land(\Delta\vdash\phi)\ \Rightarrow\
    \Gamma\vdash\phi
    \end{equation}
\end{prop}
\begin{proof}
Without loss of generality we may assume that $\Delta$ is a finite
set. Indeed, suppose the
implication~(\ref{logic:eqn:FOPL:deduction:transitivity:1}) has been
proved in this case. We shall show that it is then true in general.
So suppose $\Gamma\vdash\psi$ for all $\psi\in\Delta$ and
$\Delta\vdash\phi$. We need to show that $\Gamma\vdash\phi$.
However, there exists $\Delta_{0}$ finite such that
$\Delta_{0}\subseteq\Delta$ and $\Delta_{0}\vdash\phi$. Hence we
have $\Gamma\vdash\psi$ for all $\psi\in\Delta_{0}$ and
$\Delta_{0}\vdash\phi$. Having assumed the
implication~(\ref{logic:eqn:FOPL:deduction:transitivity:1}) is true
for $\Delta$ finite, it follows that $\Gamma\vdash\phi$ as
requested. So we assume without loss of generality that $\Delta$ is
a finite set. We shall show
that~(\ref{logic:eqn:FOPL:deduction:transitivity:1}) is true for all
$\Gamma\subseteq\pv$ and $\phi\in\pv$, using an induction argument
on the cardinal $|\Delta|$ of the set $\Delta$. First we assume that
$|\Delta|=0$. Let $\Gamma\subseteq\pv$ and $\phi\in\pv$. From
$\Delta=\emptyset$ and $\Delta\vdash\phi$ we obtain $\vdash\phi$ and
in particular $\Gamma\vdash\phi$. Hence the
implication~(\ref{logic:eqn:FOPL:deduction:transitivity:1}) is
necessarily true. We now assume
that~(\ref{logic:eqn:FOPL:deduction:transitivity:1}) is true for all
$\Gamma\subseteq\pv$ and $\phi\in\pv$ whenever $|\Delta|=n$ for some
$n\in\N$. We need to show the same is true when $|\Delta|=n+1$. So
let $\Gamma\subseteq\pv$ and $\phi\in\pv$. We assume that
$\Gamma\vdash\psi$ for all $\psi\in\Delta$ and $\Delta\vdash\phi$.
We need to show that $\Gamma\vdash\phi$. Having assumed
$|\Delta|=n+1$, in particular $\Delta\neq\emptyset$. Let
$\psi^{*}\in\Delta$ and define
$\Delta^{*}=\Delta\setminus\{\psi^{*}\}$. Then $|\Delta^{*}|=n$.
Furthermore we have $\Delta=\Delta^{*}\cup\{\psi^{*}\}$ and
consequently $\Delta^{*}\cup\{\psi^{*}\}\vdash\phi$. Using the
deduction theorem~(\ref{logic:the:FOPL:deduction}) we obtain
$\Delta^{*}\vdash(\psi^{*}\to\phi)$. However, since
$\Delta^{*}\subseteq\Delta$ we also have $\Gamma\vdash\psi$ for all
$\psi\in\Delta^{*}$. Having assumed the
implication~(\ref{logic:eqn:FOPL:deduction:transitivity:1}) is true
for all $\Gamma\subseteq\pv$, $\phi\in\pv$ and $|\Delta|=n$, in
particular it is true for $\Gamma$, $(\psi^{*}\to\phi)$ and
$\Delta^{*}$. Hence we see that $\Gamma\vdash(\psi^{*}\to\phi)$ is
true. Furthermore, since $\psi^{*}\in\Delta$ we have
$\Gamma\vdash\psi^{*}$. Using the modus ponens property of
proposition~(\ref{logic:prop:FOPL:modus:ponens}) we conclude that
$\Gamma\vdash\phi$.
\end{proof}

There is an interesting corollary to
proposition~(\ref{logic:prop:FOPL:deduction:transitivity}) which
looks like the {\em cut rule} for sequent calculus. In case this
turns out to be useful we quote:

\begin{prop}\label{logic:prop:FUAP:transitivity:cut:elimination}
Let $V$ be a set. Let $\Gamma\subseteq\pv$ and $\phi,\psi\in\pv$.
Then:
    \[
    (\,\Gamma\cup\{\psi\}\vdash\phi\,)\,\land\,(\,\Gamma\cup\{\psi\to\bot\}\vdash\phi\,)\
    \Rightarrow\ \Gamma\vdash\phi
    \]
\end{prop}
\begin{proof}
Using the deduction theorem~(\ref{logic:the:FOPL:deduction}) of
page~\pageref{logic:the:FOPL:deduction}, by assumption we have the
sequents $\Gamma\vdash(\psi\to\phi)$ and
$\Gamma\vdash((\psi\to\bot)\to\phi)$. Defining the set of formulas
$\Delta=\{\psi\to\phi,(\psi\to\bot)\to\phi\}$ it follows that
$\Gamma\vdash\chi$ for all $\chi\in\Delta$. In order to prove that
$\Gamma\vdash\phi$, from the transitivity of
proposition~(\ref{logic:prop:FOPL:deduction:transitivity}) it is
therefore sufficient to prove that $\Delta\vdash\phi$. Using the
transposition property of
proposition~(\ref{logic:prop:FOPL:transposition}), we simply need to
show that $\Delta\vdash(\phi\to\bot)\to\bot$. From the deduction
theorem~(\ref{logic:the:FOPL:deduction}), this amounts to showing
that $\Delta^{*}\vdash\bot$ where
$\Delta^{*}=\Delta\cup\{\phi\to\bot\}$. Let us accept for now that
$\Delta^{*}\vdash(\psi\to\bot)$ and
$\Delta^{*}\vdash(\psi\to\bot)\to\bot$. Then from the modus ponens
property of proposition~(\ref{logic:prop:FOPL:modus:ponens}) we
obtain $\Delta^{*}\vdash\bot$ as requested. So it remains to show
that $\Delta^{*}\vdash(\psi\to\bot)$ and
$\Delta^{*}\vdash(\psi\to\bot)\to\bot$. First we show the sequent
$\Delta^{*}\vdash(\psi\to\bot)$\,: from the deduction
theorem~(\ref{logic:the:FOPL:deduction}), we need to show that
$\Delta^{*}\cup\{\psi\}\vdash\bot$ which follows from
proposition~(\ref{logic:prop:FOPL:modus:ponens}) and the fact that
$\Delta^{*}\cup\{\psi\}$ contains the three formulas $\psi$,
$\psi\to\phi$ and $\phi\to\bot$. So we now show the sequent
$\Delta^{*}\vdash(\psi\to\bot)\to\bot$\,: from the deduction
theorem~(\ref{logic:the:FOPL:deduction}), we need to show that
$\Delta^{*}\cup\{\psi\to\bot\}\vdash\bot$ which follows from
proposition~(\ref{logic:prop:FOPL:modus:ponens}) and the fact that
$\Delta^{*}\cup\{\psi\to\bot\}$ contains the three formulas
$\psi\to\bot$, $(\psi\to\bot)\to\phi$ and $\phi\to\bot$.
\end{proof}
