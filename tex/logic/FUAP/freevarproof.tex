In this section we define and study the set of free variables
$\free(\pi)$ of a proof $\pi\in\pvs$. The notion of free variables
is crucial when attempting to formalize the idea of variable
substitutions which avoid capture. Free variables also play a key
role when considering minimal transforms which are in turn a key
step in constructing essential substitutions. Essential
substitutions for proofs will be shown to exist in
theorem~(\ref{logic:the:FUAP:esssubst:existence}) of
page~\pageref{logic:the:FUAP:esssubst:existence}. Given a proof
$\pi\in\pvs$, the free variables of $\pi$ are simply the free
variables of the hypothesis and axioms of $\pi$, with the exception
of generalization variables which are removed from $\free(\pi)$:

\index{free@Free variables of proof}\index{free@$\free(\pi)$ : set
of free variables of $\pi$}
\begin{defin}\label{logic:def:FUAP:freevarproof:free:var}
Let $V$ be a set. The map $\free:\pvs\to {\cal P}(V)$ defined by the
following structural recursion is called {\em free variable mapping
on \pvs}:
 \begin{equation}\label{logic:eqn:FUAP:freevarproof:free:var:1}
    \forall\pi\in\pvs\ ,\ \free(\pi)=\left\{
                    \begin{array}{lcl}
                    \free(\phi)&\mbox{\ if\ }&\pi=\phi\in\pv\\
                    \free(\phi)&\mbox{\ if\ }&\pi=\axi\phi\\
                    \free(\pi_{1})\cup\free(\pi_{2}) &\mbox{\ if\ }&\pi=\pi_{1}\pon\pi_{2}\\
                    \free(\pi_{1})\setminus\{x\}&\mbox{\ if\ }&\pi=\gen x\pi_{1}
                    \end{array}\right.
    \end{equation}
We say that $x\in V$ is a {\em free variable} of $\pi\in\pvs$
\ifand\ $x\in\free(\pi)$.
\end{defin}
Given a formula $\phi\in\pv$ the notation $\free(\phi)$ is
potentially ambiguous. Since $\pv\subseteq\pvs$, it may refer to the
usual $\free(\phi)$ of definition~(\ref{logic:def:free:variable}),
or to the set $\free(\pi)$ where $\pi=\phi$ of
definition~(\ref{logic:def:FUAP:freevarproof:free:var}). Luckily,
the two notions coincide.

\begin{prop}\label{logic:prop:FUAP:freevarproof:recursion}
The structural recursion of {\em
definition~(\ref{logic:def:FUAP:freevarproof:free:var})} is
legitimate.
\end{prop}
\begin{proof}
We need to show the existence and uniqueness of the map
$\free:\pvs\to{\cal P}(V)$ satisfying the four conditions of
equation~(\ref{logic:eqn:FUAP:freevarproof:free:var:1}). This
follows from an application of
theorem~(\ref{logic:the:structural:recursion}) of
page~\pageref{logic:the:structural:recursion} with $X=\pvs$,
$X_{0}=\pv$ and $A={\cal P}(V)$ where $g_{0}:X_{0}\to A$ is defined
as $g_{0}(\phi)=\free(\phi)$. Furthermore, given $\phi\in\pv$ we
take $h(\axi\phi):A^{0}\to A$ defined $h(\axi\phi)(0)=\free(\phi)$.
We take $h(\pon):A^{2}\to A$ defined by
$h(\pon)(A_{0},A_{1})=A_{0}\cup A_{1}$ and $h(\gen x):A^{1}\to A$
defined by $h(\gen x)(A_{0})=A_{0}\setminus\{x\}$.
\end{proof}

Given a map $\sigma:V\to W$ with associated
$\sigma:\pvs\to{\bf\Pi}(W)$, given a proof $\pi\in\pvs$ the free
variables of the proof $\sigma(\pi)$ must be images of free
variables of $\pi$ by $\sigma$. The following proposition is the
counterpart of
proposition~(\ref{logic:prop:freevar:of:substitution:inclusion})\,:

\begin{prop}\label{logic:prop:FUAP:freevarproof:substitution:inclusion}
Let $V$, $W$ be sets and $\sigma:V\to W$ be a map. Let
$\pi\in\pvs$\,:
    \[
    \free(\sigma(\pi))\subseteq\sigma(\free(\pi))
    \]
where $\sigma:\pvs\to{\bf\Pi}(W)$ also denotes the associated proof
substitution mapping.
\end{prop}
\begin{proof}
We shall prove the inclusion with a structural induction, using
theorem~(\ref{logic:the:proof:induction}) of
page~\pageref{logic:the:proof:induction}. First we assume that
$\pi=\phi$ for some $\phi\in\pv$. Then we need to show that
$\free(\sigma(\phi))\subseteq\sigma(\free(\phi))$ which follows
immediately from
proposition~(\ref{logic:prop:freevar:of:substitution:inclusion}). We
now assume that $\pi=\axi\phi$ for some $\phi\in\pv$. Then we have:
    \begin{eqnarray*}
    \free(\sigma(\pi))&=&\free(\sigma(\axi\phi))\\
    &=&\free(\axi\sigma(\phi))\\
    &=&\free(\sigma(\phi))\\
    \mbox{prop.~(\ref{logic:prop:freevar:of:substitution:inclusion})}\ \rightarrow
    &\subseteq&\sigma(\free(\phi))\\
    &=&\sigma(\free(\axi\phi))\\
    &=&\sigma(\free(\pi))\\
    \end{eqnarray*}
We now assume that $\pi=\pi_{1}\pon\pi_{2}$ where
$\pi_{1},\pi_{2}\in\pvs$ are proofs which satisfy our inclusion. We
need to show the same is true of $\pi$ which goes as follows:
    \begin{eqnarray*}
    \free(\sigma(\pi))&=&\free(\sigma(\pi_{1}\pon\pi_{2}))\\
    &=&\free(\sigma(\pi_{1})\pon\,\sigma(\pi_{2}))\\
    &=&\free(\sigma(\pi_{1}))\cup\free(\sigma(\pi_{2}))\\
    &\subseteq&\sigma(\free(\pi_{1}))\cup\sigma(\free(\pi_{2}))\\
    &=&\sigma(\,\free(\pi_{1})\cup\free(\pi_{2})\,)\\
    &=&\sigma(\free(\pi_{1}\pon\pi_{2}))\\
    &=&\sigma(\free(\pi))\\
    \end{eqnarray*}
Finally, we assume that $\pi=\gen x\pi_{1}$ where $x\in V$ and
$\pi_{1}\in\pvs$ is a proof satisfying our inclusion. We need to
show the same is true of $\pi$\,:
    \begin{eqnarray*}
    \free(\sigma(\pi))&=&\free(\sigma(\gen x\pi_{1}))\\
    &=&\free(\,\gen\sigma(x)\sigma(\pi_{1})\,)\\
    &=&\free(\sigma(\pi_{1}))\setminus\{\sigma(x)\}\\
    &\subseteq&\sigma(\free(\pi_{1}))\setminus\{\sigma(x)\}\\
    &=&\sigma(\,\free(\pi_{1})\setminus\{x\}\,)\setminus\{\sigma(x)\}\\
    &\subseteq&\sigma(\,\free(\pi_{1})\setminus\{x\}\,)\\
    &=&\sigma(\free(\gen x\pi_{1}))\\
    &=&\sigma(\free(\pi))\\
    \end{eqnarray*}
\end{proof}

The {\em specific variables} of a proof $\pi\in\pvs$ are those with
respect to which generalization is not permitted. So if a proof is
totally clean, it contains no flawed attempt at generalization and
the specific variables remain free variables:

\begin{prop}\label{logic:prop:FUAP:freevarproof:spec:free}
Let $V$ be a set and $\pi\in\pvs$ be totally clean. Then we have:
    \[
    \spec(\pi)\subseteq\free(\pi)
    \]
In other words, the specific variables of $\pi$ are also free
variables of $\pi$.
\end{prop}
\begin{proof}
For every proof $\pi\in\pvs$ we need to show the following
implication:
    \[
    (\mbox{$\pi$ totally clean})\ \Rightarrow\ \spec(\pi)\subseteq\free(\pi)
    \]
We shall do so with a structural induction, using
theorem~(\ref{logic:the:proof:induction}) of
page~\pageref{logic:the:proof:induction}. First we assume that
$\pi=\phi$ for some $\phi\in\pv$. Then $\pi$ is always totally clean
in this case and we have
$\spec(\phi)=\free(\hyp(\phi))=\free(\{\phi\})=\free(\phi)$. We now
assume that $\pi=\axi\phi$ for some $\phi\in\pv$. Then
$\spec(\pi)=\emptyset$ and the implication is clearly true. So we
now assume that $\pi=\pi_{1}\pon\pi_{2}$ where
$\pi_{1},\pi_{2}\in\pvs$ are proofs satisfying our implication. We
need to show the same is true of $\pi$. So we assume that $\pi$ is
totally clean. We need to show the inclusion
$\spec(\pi)\subseteq\free(\pi)$. However, from
proposition~(\ref{logic:prop:FUAP:clean:modus:ponens}) both
$\pi_{1}$ and $\pi_{2}$ are totally clean. Hence:
    \begin{eqnarray*}
    \spec(\pi)&=&\spec(\pi_{1}\pon\pi_{2})\\
    \mbox{prop.~(\ref{logic:prop:FUAP:freevar:recursive:def})}\ \rightarrow
    &=&\spec(\pi_{1})\cup\spec(\pi_{2})\\
    \mbox{$\pi_{1},\pi_{2}$ totally clean}\ \rightarrow
    &\subseteq&\free(\pi_{1})\cup\free(\pi_{2})\\
    &=&\free(\pi_{1}\pon\pi_{2})\\
    &=&\free(\pi)\\
    \end{eqnarray*}
Note that the equality $\val(\pi_{2})=\val(\pi_{1})\to\val(\pi)$
which follows from $\pi$ being totally clean, has not been used in
the argument. We now assume that $\pi=\gen x\pi_{1}$ where $x\in V$
and $\pi_{1}\in\pvs$ is a proof satisfying our implication. We need
to show the same is true for $\pi$. So we assume that $\pi$ is
totally clean. We need to show the inclusion
$\spec(\pi)\subseteq\free(\pi)$. However, from
proposition~(\ref{logic:prop:FUAP:clean:generalization}) we see that
$\pi_{1}$ is totally clean and $x\not\in\spec(\pi_{1})$. Hence, we
have:
    \begin{eqnarray*}
    \spec(\pi)&=&\spec(\gen x\pi_{1})\\
    \mbox{prop.~(\ref{logic:prop:FUAP:freevar:recursive:def})}\ \rightarrow
    &=&\spec(\pi_{1})\\
    x\not\in\spec(\pi_{1})\ \rightarrow
    &=&\spec(\pi_{1})\setminus\{x\}\\
    \mbox{$\pi_{1}$ totally clean}\ \rightarrow
    &\subseteq&\free(\pi_{1})\setminus\{x\}\\
    &=&\free(\gen x\pi_{1})\\
    &=&\free(\pi)\\
    \end{eqnarray*}
\end{proof}

Just as we did for formulas, we shall need to consider congruences
on \pvs. In particular, we shall introduce the {\em substitution
congruence} on \pvs\ as per
definition~(\ref{logic:def:FUAP:subcong:substitution:congruence}).
Whenever $\sim$ is a congruence on \pvs\ we may wish to prove the
implication $\pi\sim\rho\ \Rightarrow\ \free(\pi)=\free(\rho)$. One
of the key steps in the proof is to argue that equality between sets
of free variables is itself a congruence:

\begin{prop}\label{logic:prop:FUAP:freevarproof:congruence}
Let $V$ be a set and $\equiv$ be the relation on \pvs\ defined by:
    \[
    \pi\equiv\rho\ \Leftrightarrow\ \free(\pi)=\free(\rho)
    \]
for all $\pi,\rho\in\pvs$. Then $\equiv$ is a congruence on \pvs.
\end{prop}
\begin{proof}
The relation $\equiv$ is clearly reflexive, symmetric and transitive
on \pvs. So we simply need to show that $\equiv$ is a congruent
relation on \pvs. By reflexivity, we already have
$\axi\phi\equiv\axi\phi$ for all $\phi\in\pv$. So suppose
$\pi_{1},\pi_{2},\rho_{1}$ and $\rho_{2}\in\pvs$ are such that
$\pi_{1}\equiv\rho_{1}$ and $\pi_{2}\equiv\rho_{2}$. Define
$\pi=\pi_{1}\pon\pi_{2}$ and $\rho=\rho_{1}\pon\rho_{2}$. We need to
show that $\pi\equiv\rho$, or equivalently that
$\free(\pi)=\free(\rho)$. This follows from the fact that
$\free(\pi_{1})=\free(\rho_{1})$, $\free(\pi_{2})=\free(\rho_{2})$
and furthermore:
    \begin{eqnarray*}
    \free(\pi)&=&\free(\pi_{1}\pon\pi_{2})\\
    &=&\free(\pi_{1})\cup\free(\pi_{2})\\
    &=&\free(\rho_{1})\cup\free(\rho_{2})\\
    &=&\free(\rho_{1}\pon\rho_{2})\\
    &=&\free(\rho)
    \end{eqnarray*}
We now assume that $\pi_{1},\rho_{1}\in\pvs$ are such that
$\pi_{1}\equiv\rho_{1}$. Let $x\in V$ and define $\pi=\gen x\pi_{1}$
and $\rho=\gen x\rho_{1}$. We need to show that $\pi\equiv\rho$, or
equivalently that $\free(\pi)=\free(\rho)$. This follows from the
fact that $\free(\pi_{1})=\free(\rho_{1})$ and:
    \[
    \free(\pi)=\free(\gen x\pi_{1})
    =\free(\pi_{1})\setminus\{x\}
    =\free(\rho_{1})\setminus\{x\}
    =\free(\gen x\rho_{1})
    =\free(\rho)
    \]
\end{proof}
