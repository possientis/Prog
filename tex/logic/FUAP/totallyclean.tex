The universal algebra of proofs \pvs\ of
definition~(\ref{logic:def:FOPL:proof:algebra}) is not as natural as
it looks: a universal algebra has operators which must be defined
everywhere. However, the modus ponens operator $\pon$ should not be
defined in most cases. It makes little sense to infer anything from
two conclusions $\phi_{1}$ and $\phi_{2}$ using modus ponens, unless
the conclusion $\phi_{2}$ is of the form $\phi_{2}=\phi_{1}\to\phi$
in which case we can derive the new conclusion $\phi$. In other
words, the proof $\pi_{1}\pon\pi_{2}$ is not a valid application of
the modus ponens rule, unless $\val(\pi_{2})=\val(\pi_{1})\to\phi$
for some $\phi\in\pv$. In a similar way, the proof $\gen x\pi_{1}$
is not a valid application of the generalization rule, unless the
condition $x\not\in\spec(\pi_{1})$ is met, i.e. the variable $x$ is
truly general and not specific. So in some sense, it may be argued
that the structure of universal algebra is not suited to represent a
set of proofs. These proofs have operators which are naturally
partial functions, while universal algebras require operators which
are total functions. There is a prominent example for which this
issue arises: the algebraic structure of field has an inverse
operator which is not defined on $0$. So there is a problem: our
solution to it is to regard proofs simply as formal expressions
where the operators $\pon$ and $\gen x$ are defined everywhere, and
to introduce $\val:\pvs\to\pv$ as a semantics representing the
conclusion being proved by a given formal expression. So defying
common sense, we assign meaning to all proofs
$\pi=\pi_{1}\pon\pi_{2}$, $\pi=\gen x\pi_{1}$ and even
$\pi=\axi\phi$ when $\phi$ is not an axiom of first order logic,
while attempting to redeem ourselves by setting
$\val(\pi)=\bot\to\bot$ whenever a proof arises from a flawed
application of a rule of inference or flawed invocation of axiom.

In this section, we introduce the notion of {\em totally clean}
proof to emphasize the case when no such flawed inference occurs. We
shall use the phrase '{\em totally clean}' rather than simply '{\em
clean}' so as to reserve the latter for a weaker notion of {\em
flawlessness} which we shall introduce in
definition~(\ref{logic:def:FUAP:almostclean:definition}) when
dealing with proofs modulo. In this section, we shall also establish
proposition~(\ref{logic:prop:FUAP:clean:counterpart}) below showing
that any true sequent $\Gamma\vdash\phi$ can always be proved in a
{\em totally clean} way. This gives us some degree of reassurance:
although our universal algebra of proofs \pvs\ is arguably too large
and contains proofs which may be deemed to be flawed, every
conclusion reached by such a flawed proof is in fact provable in the
orthodox way. So it would be possible to reject \pvs\ and only
consider its subset of {\em totally clean} proofs, but this would
not change the notion of provability as introduced in
definition~(\ref{logic:def:FOPL:proof:of:formula}). However, beyond
our peace of mind there is another good reason to introduce the
notion of {\em totally clean} proof: given a map $\sigma:V\to W$ and
a proof $\pi\in\pvs$, we shall soon want to substitute the variables
of $\pi$ as specified by the map $\sigma$, thereby defining a map
$\sigma:\pvs\to{\bf\Pi}(W)$ which we shall do in
definition~(\ref{logic:def:FUAP:substitution:substitution}). This
map may give us a tool to create new proofs from old proofs,
allowing us to show some provability statements which would
otherwise be out of reach. For example, suppose we want to prove
$\vdash\sigma(\phi)$ knowing that $\vdash\phi$ is true. One possible
line of attack is to consider a proof $\pi\in\pvs$ underlying the
sequent $\vdash\phi$ and hope that $\sigma(\pi)$ is in fact a proof
with conclusion $\sigma(\phi)$. So we need the following equation to
hold:
    \begin{equation}\label{logic:eqn:FUAP:clean:clean:proof:intro:1}
    \val\circ\sigma(\pi)=\sigma\circ\val(\pi)
    \end{equation}
This equation is bound to play an important role. More generally, a
transformation $\sigma:\pvs\to{\bf\Pi}(W)$ is interesting if we
control the conclusion of $\sigma(\pi)$ in some way, given the
conclusion of $\pi$. Now consider the following proof:
    \[
    \pi=(x\in x)\pon\,((y\in y)\to(z\in z))
    \]
Whenever $x\neq y$, this proof is not totally clean as it contains a
flawed application of modus ponens. However, if
$\sigma(x)=\sigma(y)=u$ while $\sigma(z)=v$ we obtain:
    \[
    \sigma(\pi)=(u\in u)\pon\,((u\in u)\to(v\in v))
    \]
This is a totally clean proof with conclusion $v\in v$. It should be
clear from this example that we cannot hope to prove properties such
as~(\ref{logic:eqn:FUAP:clean:clean:proof:intro:1}) in full
generality, and must restrict our attention to the case of totally
clean proofs. Too many bizarre things may happen otherwise. It is
very hard to carry out a sensible analysis of proofs which are not
totally clean. Hence the following: \index{strength@Strength
mapping}\index{clean@Totally clean proof}
\begin{defin}\label{logic:def:FUAP:clean:clean:proof}
Let $V$ be a set. The map $s:\pvs\to 2=\{0,1\}$ defined by the
following structural recursion is called the {\em strength mapping}
on \pvs:
 \begin{equation}\label{logic:eqn:FUAP:clean:clean:proof:1}
    \forall\pi\in\pvs\ ,\ s(\pi)=\left\{
                    \begin{array}{lcl}
                    1&\mbox{\ if\ }&\pi=\phi\in\pv\\
                    1&\mbox{\ if\ }&\pi=\axi\phi,\ \phi\in\av\\
                    0&\mbox{\ if\ }&\pi=\axi\phi,\ \phi\not\in\av\\
                    s(\pi_{1})\land s(\pi_{2})\land\epsilon
                    &\mbox{\ if\ }&\pi=\pi_{1}\pon\pi_{2}\\
                    s(\pi_{1})\land\eta&\mbox{\ if\ }&\pi=\gen x\pi_{1}\\
                    \end{array}\right.
    \end{equation}
where $\epsilon\in 2=\{0,1\}$ is defined by $\epsilon=1$ \ifand\
 we have the equality:
    \[
    \val(\pi_{2})=\val(\pi_{1})\to\val(\pi)
    \]
and $\eta\in 2=\{0,1\}$ is defined by $\eta=1$ \ifand\
$x\not\in\spec(\pi_{1})$. We say that a proof $\pi\in\pvs$ is {\em
totally clean} \ifand\ it has full strength, i.e. $s(\pi)=1$.
\end{defin}

\begin{prop}
The structural recursion of {\em
definition~(\ref{logic:def:FUAP:clean:clean:proof})} is legitimate.
\end{prop}
\begin{proof}
We need to show the existence and uniqueness of $s:\pvs\to 2$
satisfying the five conditions of
equation~(\ref{logic:eqn:FUAP:clean:clean:proof:1}). We shall do so
using theorem~(\ref{logic:the:structural:recursion:2}) of
page~\pageref{logic:the:structural:recursion:2} with $X=\pvs$,
$X_{0}=\pv$ and $A=2$. Choosing the map $g_{0}:X_{0}\to A$ defined
by $g_{0}(\phi)=1$ we ensure the first condition is met. Next for
every formula $\phi\in\pv$ we define $h(\axi\phi):A^{0}\times
X^{0}\to A$ by setting $h(\axi\phi)(0,0)=1$ if $\phi\in\av$ and
$h(\axi\phi)(0,0)=0$ otherwise. This ensures the second and third
conditions are met. Next we define $h(\pon):A^{2}\times X^{2}\to A$
with the following formula:
    \[
        h(\pon)(\epsilon_{1},\epsilon_{2},\pi_{1},\pi_{2})=\left\{
        \begin{array}{lcl}
        \epsilon_{1}\land\epsilon_{2}&\mbox{\ if\ }&
        \val(\pi_{2})=\val(\pi_{1})\to\val(\pi_{1}\pon\pi_{2})\\
        0&\mbox{\ if\ }&\val(\pi_{2})\neq\val(\pi_{1})\to\val(\pi_{1}\pon\pi_{2})\\
        \end{array}
    \right.
    \]
This takes care of the fourth condition. Finally we define $h(\gen
x):A^{1}\times X^{1}\to A$:
        \[
        h(\gen x)(\epsilon_{1},\pi_{1})=\left\{
        \begin{array}{lcl}
        \epsilon_{1}&\mbox{\ if\ }&
        x\not\in\spec(\pi_{1})\\
        0&\mbox{\ if\ }&x\in\spec(\pi_{1})\\
        \end{array}
    \right.
    \]
which ensures the fifth condition is met and completes our proof.
\end{proof}

A proof which relies on a flawed sub-proof is itself flawed. Hence a
proof cannot be totally clean unless every one of its sub-proofs is
totally clean.

\begin{prop}\label{logic:prop:FUAP:clean:sub:proof}
Let $V$ be a set and $s:\pvs\to 2$ be the strength mapping. Then the
strength of a sub-proof is higher than the strength of a proof, i.e.
    \[
    \rho\preceq\pi\ \Rightarrow\ s(\pi)\leq
    s(\rho)
    \]
A proof $\pi\in\pvs$ is totally clean \ifand\ every $\rho\preceq\pi$
is totally clean.
\end{prop}
\begin{proof}
The implication is a simple application of
proposition~(\ref{logic:prop:UA:subformula:non:decreasing}) to
$s:X\to A$ where $X=\pvs$ and $A=2$ where the chosen preorder on $A$
is the usual order $\geq$ (reversed) on $\{0,1\}$. We simply need to
check that given $\pi_{1},\pi_{2}\in\pvs$ and $x\in V$ we have the
inequalities $s(\pi_{1})\geq s(\pi_{1}\pon\pi_{2})$, $s(\pi_{2})\geq
s(\pi_{1}\pon\pi_{2})$ and $s(\pi_{1})\geq s(\gen x\pi_{1})$ which
follow immediately from the recursive
definition~(\ref{logic:def:FUAP:clean:clean:proof}). It remains to
check that $\pi\in\pvs$ is totally clean \ifand\ every one of its
sub-proofs is totally clean. Since $\pi$ is a sub-proof of itself,
the 'if' part is clear. Conversely, let $\rho\preceq\pi$ be a
sub-proof of $\pi$ where $\pi$ is totally clean. Then $s(\pi)=1$ and
since $s(\pi)\leq s(\rho)$ we see that $s(\rho)=1$. So the proof
$\rho$ is also totally clean.
\end{proof}

A proof arising from modus ponens is of the form
$\pi=\pi_{1}\pon\pi_{2}$. Such a proof cannot be totally clean
unless the use of modus ponens is legitimate, i.e. the conclusion of
$\pi_{2}$ is of the form $\val(\pi_{2})=\val(\pi_{1})\to\val(\pi)$.
Of course we also need both $\pi_{1}$ and $\pi_{2}$ to be totally
clean. These conditions are in turn sufficient. The following
proposition is useful to carry out structural induction arguments.

\begin{prop}\label{logic:prop:FUAP:clean:modus:ponens}
Let $V$ be a set and $\pi$ be a proof of the form
$\pi=\pi_{1}\pon\pi_{2}$. Then $\pi$ is totally clean \ifand\ both
$\pi_{1},\pi_{2}$ are totally clean and:
    \[
    \val(\pi_{2})=\val(\pi_{1})\to\val(\pi)
    \]
\end{prop}
\begin{proof}
First we show the 'only if' part: so we assume that
$\pi=\pi_{1}\pon\pi_{2}$ is totally clean. Then $s(\pi)=1$ and from
definition~(\ref{logic:def:FUAP:clean:clean:proof}) we obtain
$s(\pi_{1})\land s(\pi_{2})\land\epsilon=1$. It follows that
$s(\pi_{1})=s(\pi_{2})=\epsilon=1$. So $\pi_{1}$ and $\pi_{2}$ are
totally clean and from $\epsilon=1$ and
definition~(\ref{logic:def:FUAP:clean:clean:proof}) we conclude that
$\val(\pi_{2})=\val(\pi_{1})\to\val(\pi)$. We now prove the 'if'
part: so we assume that $\pi=\pi_{1}\pon\pi_{2}$  where $\pi_{1}$,
$\pi_{2}$ are totally clean and
$\val(\pi_{2})=\val(\pi_{1})\to\val(\pi)$. Then from
definition~(\ref{logic:def:FUAP:clean:clean:proof}):
    \[
    s(\pi)=s(\pi_{1})\land s(\pi_{2})\land\epsilon=1\land 1\land 1
    =1
    \]
It follows that $\pi$ is itself a totally clean proof, as requested.
\end{proof}

A proof arising from generalization is of the form $\pi=\gen
x\pi_{1}$. Such a proof cannot be totally clean unless the use of
generalization is legitimate, that is $x\not\in\spec(\pi_{1})$. We
also need $\pi_{1}$ to be totally clean, and these conditions are in
turn sufficient. The following proposition is also useful for
structural induction.

\begin{prop}\label{logic:prop:FUAP:clean:generalization}
Let $V$ be a set and $\pi$ be a proof of the form $\pi=\gen
x\pi_{1}$. Then $\pi$ is totally clean \ifand\ $\pi_{1}$ is totally
clean and $x\not\in\spec(\pi_{1})$.
\end{prop}
\begin{proof}
First we show the 'only if' part: so we assume that $\pi=\gen
x\pi_{1}$ is totally clean. Then $s(\pi)=1$ and from
definition~(\ref{logic:def:FUAP:clean:clean:proof}) we obtain
$s(\pi_{1})\land\eta=1$. It follows that $s(\pi_{1})=\eta=1$. So
$\pi_{1}$ is totally clean and furthermore from $\eta=1$ and
definition~(\ref{logic:def:FUAP:clean:clean:proof}) we conclude that
$x\not\in\spec(\pi_{1})$. We now prove the 'if' part: so we assume
that $\pi=\gen x\pi_{1}$, where $\pi_{1}$ is totally clean and
$x\not\in\spec(\pi_{1})$. Then from
definition~(\ref{logic:def:FUAP:clean:clean:proof})
$s(\pi)=s(\pi_{1})\land\eta=1\land 1=1$. So $\pi$ is itself totally
clean.
\end{proof}

We decided to define a {\em totally clean} proof in terms of
strength $s:\pvs\to 2$ as per
definition~(\ref{logic:def:FUAP:clean:clean:proof}). This was not
the only way to go about it. A {\em totally clean} proof is one
which is not {\em flawed}. This means that every axiom invocation or
use of modus ponens or generalization is legitimate. The following
proposition confirms that a {\em totally clean} proof can be defined
as one without flawed steps:

\begin{prop}\label{logic:prop:FUAP:clean:characterization}
Let $V$ be a set. Then $\pi\in\pvs$ is totally clean \ifand\ for all
$\pi_{1},\pi_{2}\in\pvs$, $\phi\in\pv$ and $x\in V$, we have the
following:
    \begin{eqnarray*}
    (i)&&\axi\phi\preceq\pi\ \Rightarrow\ \phi\in\av\\
    (ii)&&\pi_{1}\pon\pi_{2}\preceq\pi\ \Rightarrow\
    \val(\pi_{2})=\val(\pi_{1})\to\val(\pi_{1}\pon\pi_{2})\\
    (iii)&&\gen x\pi_{1}\preceq\pi\ \Rightarrow\
    x\not\in\spec(\pi_{1})
    \end{eqnarray*}
\end{prop}
\begin{proof}
First we show the 'only if' part: so we assume that $\pi\in\pvs$ is
totally clean. We need to show that $(i)$, $(ii)$ and $(iii)$ hold.
First we show $(i)$\,: so we assume that $\rho\preceq\pi$ is a
sub-proof of $\pi$ of the form $\rho=\axi\phi$ for some
$\phi\in\pv$. We need to show that $\phi\in\av$, i.e. that $\phi$ is
an axiom. Having assumed that $\pi$ is totally clean, from
proposition~(\ref{logic:prop:FUAP:clean:sub:proof}) the proof
$\rho=\axi\phi$ is also totally clean. So $\phi\in\av$ follows
immediately from
definition~(\ref{logic:def:FUAP:clean:clean:proof}). Next we show
$(ii)$\,: so we assume that $\rho\preceq\pi$ is a sub-proof of $\pi$
of the form $\rho=\pi_{1}\pon\pi_{2}$. We need to show that
$\val(\pi_{2})=\val(\pi_{1})\to\val(\rho)$. Having assumed that
$\pi$ is totally clean, from
proposition~(\ref{logic:prop:FUAP:clean:sub:proof}) the proof $\rho$
is also totally clean. However, $\rho$ is of the form
$\rho=\pi_{1}\pon\pi_{2}$ and it follows from
proposition~(\ref{logic:prop:FUAP:clean:modus:ponens}) that we have
$\val(\pi_{2})=\val(\pi_{1})\to\val(\rho)$ as requested. We now show
$(iii)$\,: so we assume that $\rho\preceq\pi$ is a sub-proof of
$\pi$ of the form $\rho=\gen x\pi_{1}$. We need to show that
$x\not\in\spec(\pi_{1})$. Having assumed that $\pi$ is totally
clean, from proposition~(\ref{logic:prop:FUAP:clean:sub:proof}) the
proof $\rho$ is also totally clean. However, $\rho$ is of the form
$\rho=\gen x\pi_{1}$ and it follows from
proposition~(\ref{logic:prop:FUAP:clean:generalization}) that
$x\not\in\spec(\pi_{1})$ as requested. We now show this 'if' part:
we need to prove that every $\pi\in\pvs$ satisfies the property
$(i)+(ii)+(iii)\ \Rightarrow\ (\,\pi\mbox{\ is totally clean}\,)$.
We shall do so with a structural induction argument, using
theorem~(\ref{logic:the:proof:induction}) of
page~\pageref{logic:the:proof:induction}. So first we assume that
$\pi=\phi$ for some $\phi\in\pv$. From
definition~(\ref{logic:def:FUAP:clean:clean:proof}) we obtain
$s(\pi)=1$. So $\pi$ is always totally clean and the property is
satisfied. Next we assume that $\pi=\axi\phi$ where $\phi\in\pv$. We
need to show $\pi$ satisfies our property. So we assume that $(i)$,
$(ii)$ and $(iii)$ hold. We need to show that $\pi$ is totally
clean. However, since $\pi$ is a sub-proof of itself we have
$\axi\phi\preceq\pi$ and using $(i)$ we obtain $\phi\in\av$. It
follows from definition~(\ref{logic:def:FUAP:clean:clean:proof})
that $s(\pi)=s(\axi\phi)=1$ and we conclude that $\pi$ is totally
clean as requested. Next we assume that $\pi=\pi_{1}\pon\pi_{2}$
where $\pi_{1},\pi_{2}\in\pvs$ satisfy our induction property. We
need to show the same is true of $\pi$. So we assume that $(i)$,
$(ii)$ and $(iii)$ hold for $\pi$. We need to show that $\pi$ is
totally clean. Using
proposition~(\ref{logic:prop:FUAP:clean:modus:ponens}), it is
sufficient to prove that both $\pi_{1}$ and $\pi_{2}$ are totally
clean and furthermore that
$\val(\pi_{2})=\val(\pi_{1})\to\val(\pi)$. This last equality is an
immediate consequence of condition $(ii)$ and the fact that $\pi$ is
a sub-proof of itself i.e. $\pi=\pi_{1}\pon\pi_{2}\preceq\pi$. So it
remains to show that $\pi_{1}$ and $\pi_{2}$ are totally clean.
Having assumed that $\pi_{1}$ and $\pi_{2}$ satisfy our induction
property, it is therefore sufficient to prove that $(i)$, $(ii)$ and
$(iii)$ are satisfied by $\pi_{1}$ and $\pi_{2}$. First we show that
$(i)$ is satisfied by $\pi_{1}$\,: so we assume that
$\rho\preceq\pi_{1}$ is a sub-proof of $\pi_{1}$ of the form
$\rho=\axi\phi$. We need to show that $\phi\in\av$. However, having
assumed that $(i)$ holds for $\pi=\pi_{1}\pon\pi_{2}$, it is
sufficient to prove that $\rho\preceq\pi$ which follows immediately
from $\rho\preceq\pi_{1}$ and $\pi_{1}\preceq\pi$. We prove
similarly that $(i)$ holds for $\pi_{2}$. Next we show that $(ii)$
is satisfied by $\pi_{1}$\,: so we assume that $\rho\preceq\pi_{1}$
is a sub-proof of $\pi_{1}$ of the form $\rho=\rho_{1}\pon\rho_{2}$.
We need to show that $\val(\rho_{2})=\val(\rho_{1})\to\val(\rho)$.
However, having assumed that $(ii)$ holds for
$\pi=\pi_{1}\pon\pi_{2}$, it is sufficient to prove that
$\rho\preceq\pi$ which follows immediately from $\rho\preceq\pi_{1}$
and $\pi_{1}\preceq\pi$. We prove similarly that $(ii)$ holds for
$\pi_{2}$ and we now show that $(iii)$ holds for $\pi_{1}$\,: so we
assume that $\rho\preceq\pi_{1}$ is a sub-proof of $\pi_{1}$ of the
form $\rho=\gen x\rho_{1}$. We need to show that
$x\not\in\spec(\rho_{1})$. However, having assumed that $(iii)$
holds for $\pi=\pi_{1}\pon\pi_{2}$, it is sufficient to prove that
$\rho\preceq\pi$ which again follows from $\rho\preceq\pi_{1}$ and
$\pi_{1}\preceq\pi$. We prove similarly that $(iii)$ holds for
$\pi_{2}$ which completes our induction argument in the case when
$\pi=\pi_{1}\pon\pi_{2}$. So we finally assume that $\pi=\gen
x\pi_{1}$ where $x\in V$ and $\pi_{1}\in\pvs$ satisfies our
induction property. We need to show the same is true of $\pi$. So we
assume that $(i)$, $(ii)$ and $(iii)$ hold for $\pi$. We need to
show that $\pi$ is totally clean. Using
proposition~(\ref{logic:prop:FUAP:clean:generalization}), it is
sufficient to prove that $\pi_{1}$ is totally clean and furthermore
that $x\not\in\spec(\pi_{1})$. This last property is an immediate
consequence of condition $(iii)$ and the fact that $\pi$ is a
sub-proof of itself i.e. $\pi=\gen x\pi_{1}\preceq\pi$. So it
remains to show that $\pi_{1}$ is totally clean. Having assumed that
$\pi_{1}$ satisfies our induction property, it is therefore
sufficient to prove that $(i)$, $(ii)$ and $(iii)$ hold for
$\pi_{1}$. First we show that $(i)$ is true: so we assume that
$\rho\preceq\pi_{1}$ is a sub-proof of $\pi_{1}$ of the form
$\rho=\axi\phi$. We need to show that $\phi\in\av$. However, having
assumed that $(i)$ holds for $\pi=\gen x\pi_{1}$, it is sufficient
to prove that $\rho\preceq\pi$ which follows immediately from
$\rho\preceq\pi_{1}$ and $\pi_{1}\preceq\pi$. Next we show that
$(ii)$ is true: so we assume that $\rho\preceq\pi_{1}$ is a
sub-proof of $\pi_{1}$ of the form $\rho=\rho_{1}\pon\rho_{2}$. We
need to show that $\val(\rho_{2})=\val(\rho_{1})\to\val(\rho)$.
However, having assumed that $(ii)$ holds for $\pi=\gen x\pi_{1}$,
it is sufficient to prove that $\rho\preceq\pi$ which follows
immediately from $\rho\preceq\pi_{1}$ and $\pi_{1}\preceq\pi$. We
now show that $(iii)$ is satisfied: so we assume that
$\rho\preceq\pi_{1}$ is a sub-proof of $\pi_{1}$ of the form
$\rho=\gen u\rho_{1}$. We need to show that
$u\not\in\spec(\rho_{1})$. However, having assumed that $(iii)$
holds for $\pi=\gen x\pi_{1}$, it is sufficient to prove that
$\rho\preceq\pi$ which again follows from $\rho\preceq\pi_{1}$ and
$\pi_{1}\preceq\pi$.
\end{proof}

As we shall see in
proposition~(\ref{logic:prop:FUAP:clean:counterpart}) below, any
true sequent $\Gamma\vdash\phi$ can be established with a totally
clean proof. The key to the argument is that any flawed step
$\rho\preceq\pi$ of a proof with conclusion $\val(\rho)=\bot\to\bot$
can be discarded and replaced by a totally clean fragment as the
following lemma shows:

\begin{lemma}\label{logic:lemma:FUAP:clean:bot:bot}
Let $V$ be a set. There exists a totally clean proof $\pi\in\pvs$
with:
    \[
    (\,\hyp(\pi)=\emptyset\,)\land(\,\val(\pi)=\bot\to\bot\,)
    \]
\end{lemma}
\begin{proof}
Consider the formulas $\phi_{1}=((\bot\to\bot)\to\bot)\to\bot$ and
$\phi_{2}=\phi_{1}\to(\bot\to\bot)$. Let us accept for now that both
$\phi_{1}$ and $\phi_{2}$ are axioms and define $\pi=\axi
\phi_{1}\pon\,\axi\phi_{2}$. It is clear that $\hyp(\pi)=\emptyset$
and furthermore we have the equality:
    \begin{eqnarray*}
    \val(\pi)&=&\val(\axi\phi_{1}\pon\,\axi\phi_{2})\\
    \mbox{def.~(\ref{logic:def:FOPL:proof:valuation})}\ \rightarrow
    &=&M(\,\val(\axi\phi_{1}),\val(\axi\phi_{2})\,)\\
    \phi_{1},\phi_{2}\in\av\ \rightarrow&=&M(\phi_{1},\phi_{2})\\
    &=&M(\,\phi_{1}\,,\,\phi_{1}\to(\bot\to\bot)\,)\\
    &=&\bot\to\bot
    \end{eqnarray*}
So it remains to show that $\pi$ is totally clean. However, from
definition~(\ref{logic:def:FUAP:clean:clean:proof}) both
$\axi\phi_{1}$ and $\axi\phi_{2}$ are totally clean and it follows
from proposition~(\ref{logic:prop:FUAP:clean:modus:ponens}) and:
    \[
    \val(\axi\phi_{2})=\val(\axi\phi_{1})\to\val(\pi)
    \]
that $\pi$ is also totally clean. It remains to check that both
$\phi_{1}$ and $\phi_{2}$ are indeed axioms of first order logic.
However, we have $\phi_{1}=((\psi_{1}\to\bot)\to\bot)\to\psi_{1}$
where $\psi_{1}=\bot$. So $\phi_{1}$ is a transposition axiom as per
definition~(\ref{logic:def:FOPL:transposition:axiom}). Likewise, we
have $\phi_{2}=((\psi_{2}\to\bot)\to\bot)\to\psi_{2}$ where
$\psi_{2}=\bot\to\bot$ and $\phi_{2}$ is an axiom.
\end{proof}

We can now prove the main result of this section: there is no loss
of generality in assuming that a true sequent $\Gamma\vdash\phi$ has
an underlying proof which is totally clean. This shows that defining
our syntactic entailment solely in terms of totally clean proofs
would have made no difference to the final notion of provability.
However, not working on the free universal algebra of proofs \pvs\
would have made formal developments considerably more cumbersome.


\begin{prop}\label{logic:prop:FUAP:clean:counterpart}
Let $V$ be a set and $\Gamma\subseteq\pv$. Let $\phi\in\pv$ be such
that the sequent $\Gamma\vdash\phi$ is true. Then there exists a
totally clean proof of $\phi$ from $\Gamma$.
\end{prop}
\begin{proof}
In order to establish this proposition, we shall first prove it is
sufficient to show that every proof $\pi\in\pvs$ has a 'totally
clean counterpart', namely that:
    \[
    \exists\pi^{*}\in\pvs\ ,\ (\pi^{*}\mbox{ totally clean})
    \land(\,\val(\pi^{*})=\val(\pi)\,)\land(\,\hyp(\pi^{*})\subseteq\hyp(\pi)\,)
    \]
In other words, for every proof $\pi\in\pvs$, there exists a totally
clean proof $\pi^{*}$ with the same conclusion as $\pi$, and fewer
hypothesis. So let us assume this is the case for now. We need to
show the proposition is true. So let $\Gamma\subseteq\pv$ and
$\phi\in\pv$ such that $\Gamma\vdash\phi$. Then there exists a proof
$\pi$ of $\phi$ from $\Gamma$, i.e. $\pi\in\pvs$ such that
$\val(\pi)=\phi$ and $\hyp(\pi)\subseteq\Gamma$. Taking a totally
clean counterpart of $\pi$, we obtain a totally clean proof
$\pi^{*}\in\pvs$ such that $\val(\pi^{*})=\val(\pi)$ and
$\hyp(\pi^{*})\subseteq\hyp(\pi)$. It follows that $\pi^{*}$ is a
totally clean proof of $\phi$ from $\Gamma$ as requested. So we need
to show that every proof $\pi\in\pvs$ has a totally clean
counterpart. We shall do so by structural induction, using
theorem~(\ref{logic:the:proof:induction}) of
page~\pageref{logic:the:proof:induction}. First we assume that
$\pi=\phi$ for some $\phi\in\pv$. Then $\pi$ is totally clean and
there is nothing else to prove. So we now assume that $\pi=\axi\phi$
for some $\phi\in\pv$. We shall distinguish two cases: first we
assume that $\phi\in\av$. Then $\pi$ is totally clean and we are
done. We now assume that $\phi\not\in\av$. Then
$\val(\pi)=\bot\to\bot$. Using
lemma~(\ref{logic:lemma:FUAP:clean:bot:bot}) there exists a totally
clean proof $\pi^{*}\in\pvs$ such that $\hyp(\pi^{*})=\emptyset$ and
$\val(\pi^{*})=\bot\to\bot$. It follows that
$\hyp(\pi^{*})\subseteq\hyp(\pi)$ and $\val(\pi^{*})=\val(\pi)$ and
we have found a totally clean counterpart of $\pi$. Next we assume
that $\pi=\pi_{1}\pon\pi_{2}$ where $\pi_{1},\pi_{2}\in\pvs$ are
proofs with totally clean counterparts. We need to show that $\pi$
also has a totally clean counterpart. So let $\pi_{1}^{*}$ and
$\pi_{2}^{*}$ be totally clean counterparts of $\pi_{1}$ and
$\pi_{2}$ respectively. We shall distinguish two cases. First we
assume that:
    \begin{equation}\label{logic:eqn:FUAP:clean:counterpart:1}
    \val(\pi_{2}^{*})=\val(\pi_{1}^{*})\to\val(\pi_{1}^{*}\pon\pi_{2}^{*})
    \end{equation}
Then defining $\pi^{*}=\pi_{1}^{*}\pon\pi_{2}^{*}$, since
$\pi_{1}^{*}$ and $\pi_{2}^{*}$ are totally clean it follows from
proposition~(\ref{logic:prop:FUAP:clean:modus:ponens}) that
$\pi^{*}$ is totally clean. Furthermore, we have:
    \begin{eqnarray*}
    \hyp(\pi^{*})&=&\hyp(\pi_{1}^{*}\pon\pi_{2}^{*})\\
    &=&\hyp(\pi_{1}^{*})\cup\hyp(\pi_{2}^{*})\\
    &\subseteq&
    \hyp(\pi_{1})\cup\hyp(\pi_{2})\\
    &=&\hyp(\pi_{1}\pon\pi_{2})\\
    &=&\hyp(\pi)
    \end{eqnarray*}
and:
    \begin{eqnarray*}
    \val(\pi^{*})&=&\val(\pi_{1}^{*}\pon\pi_{2}^{*})\\
    \mbox{def.~(\ref{logic:def:FOPL:proof:valuation})}\ \rightarrow
    &=&M(\,\val(\pi_{1}^{*}),\val(\pi_{2}^{*})\,)\\
    \val(\pi_{i}^{*})=\val(\pi_{i})\ \rightarrow
    &=&M(\,\val(\pi_{1}),\val(\pi_{2})\,)\\
    &=&\val(\pi_{1}\pon\pi_{2})\\
    &=&\val(\pi)
    \end{eqnarray*}
So we have found a totally clean counterpart $\pi^{*}$ of $\pi$ as
requested. Next we assume that
equation~(\ref{logic:eqn:FUAP:clean:counterpart:1}) does not hold.
Then $\val(\pi_{2}^{*})=\val(\pi_{1}^{*})\to\phi$ is false for all
$\phi\in\pv$. It follows that $\val(\pi_{2})=\val(\pi_{1})\to\phi$
is also false for all $\phi$ and consequently from
definition~(\ref{logic:def:FOPL:proof:valuation})
$\val(\pi)=\bot\to\bot$. Using
lemma~(\ref{logic:lemma:FUAP:clean:bot:bot}) there exists a totally
clean proof $\pi^{*}\in\pvs$ such that $\hyp(\pi^{*})=\emptyset$ and
$\val(\pi^{*})=\bot\to\bot$. It follows that
$\hyp(\pi^{*})\subseteq\hyp(\pi)$ and furthermore
$\val(\pi^{*})=\val(\pi)$. So we have found a totally clean
counterpart of $\pi$. Finally, we assume that $\pi=\gen x\pi_{1}$
where $x\in V$ and $\pi_{1}\in\pvs$ is a proof which has a totally
clean counterpart. We need to show that $\pi$ also has a totally
clean counterpart. So let $\pi_{1}^{*}$ be a totally clean
counterpart of $\pi_{1}$. We shall distinguish two cases: first we
assume that $x\not\in\spec(\pi_{1})$. Since
$\hyp(\pi_{1}^{*})\subseteq\hyp(\pi_{1})$ we have
$\spec(\pi_{1}^{*})\subseteq\spec(\pi_{1})$ and consequently
$x\not\in\spec(\pi_{1}^{*})$. Hence, defining $\pi^{*}=\gen
x\pi_{1}^{*}$, since $\pi_{1}^{*}$ is totally clean it follows from
proposition~(\ref{logic:prop:FUAP:clean:generalization}) that
$\pi^{*}$ is itself totally clean. Furthermore we have the following
inclusion:
    \[
    \hyp(\pi^{*})=\hyp(\pi_{1}^{*})\subseteq\hyp(\pi_{1})=\hyp(\pi)
    \]
and the equalities:
    \begin{eqnarray*}
    \val(\pi^{*})&=&\val(\gen x\pi_{1}^{*})\\
    x\not\in\spec(\pi_{1}^{*})\ \rightarrow
    &=&\forall x\val(\pi_{1}^{*})\\
    &=&\forall x\val(\pi_{1})\\
    x\not\in\spec(\pi_{1})\ \rightarrow
    &=&\val(\gen x\pi_{1})\\
    &=&\val(\pi)
    \end{eqnarray*}
So we have found a totally clean counterpart $\pi^{*}$ of $\pi$ as
requested. Next we assume that $x\in\spec(\pi_{1})$. Then from
definition~(\ref{logic:def:FOPL:proof:valuation}) we have
$\val(\pi)=\bot\to\bot$, and using
lemma~(\ref{logic:lemma:FUAP:clean:bot:bot}) we see once again that
$\pi$ has a totally clean counterpart.
\end{proof}
