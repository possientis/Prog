We created the valuation $\vals:\pvs\to\pv$ in
definition~(\ref{logic:def:FUAP:valuationmod:valuation:modulo}) so
as to allow us to define a notion of minimal transform for proofs
which would behave sensibly, namely with the equivalence
$\vals\circ{\cal M}(\pi)\sim{\cal M}\circ\vals(\pi)$ for a {\em
reasonable} proof $\pi$, where $\sim$ is the substitution congruence
on \pvb. We now know that a {\em reasonable} proof is simply a clean
proof as per
definition~(\ref{logic:def:FUAP:almostclean:definition}).  However,
before we proceed with minimal transforms on \pvs, we need to check
that $\vals$ successfully passes its second test. Remember that some
degree of validation for $\vals$ was already obtained when we
established in
proposition~(\ref{logic:prop:FUAP:valuationmod:equivalence}) that
$\vals$ does lead to an equivalent notion of provability. However,
we now need to go further and make sure that the three key notions
of $\vals$, {\em clean} proof and valid substitution for proofs of
definition~(\ref{logic:def:FUAP:validsubproof:validsub}) all make
sense:

\index{clean@Image of clean proof}
\begin{prop}\label{logic:prop:FUAP:strongvalsubalmostclean:valuation:commute}
Let $V,W$ be sets and $\sigma:V\to W$ be a map. Let $\pi\in\pvs$ be
a clean proof  and let $\sigma$ be valid for $\pi$. Then
$\sigma(\pi)$ is also a clean proof and:
    \begin{equation}\label{logic:eqn:FUAP:strongvalsubalmostclean:valuation:commute:1}
    \vals\circ\sigma(\pi)=\sigma\circ\vals(\pi)
    \end{equation}
\end{prop}
\begin{proof}
For every proof $\pi\in\pvs$, we need to show the following
implication:
    \[
    (\mbox{$\pi$ clean})\land(\mbox{$\sigma$ valid for $\pi$})
    \ \Rightarrow\ (\mbox{$\sigma(\pi)$ clean})
    \land(\mbox{eq.~(\ref{logic:eqn:FUAP:strongvalsubalmostclean:valuation:commute:1})})
    \]
We shall do so with a structural induction using
theorem~(\ref{logic:the:proof:induction}) of
page~\pageref{logic:the:proof:induction}. First we assume that
$\pi=\phi$ for some $\phi\in\pv$. From
definition~(\ref{logic:def:FUAP:almostclean:definition}), $\pi$ is
always a clean proof in this case. So we assume that $\sigma$ is
valid for $\pi$, and we need to show that $\sigma(\pi)$ is a clean
proof together with
equation~(\ref{logic:eqn:FUAP:strongvalsubalmostclean:valuation:commute:1}).
From definition~(\ref{logic:def:FUAP:substitution:substitution}) we
have $\sigma(\pi)=\sigma(\phi)\in{\bf P}(W)$ and consequently
$\sigma(\pi)$ is a clean proof. Furthermore, we have
$\vals(\sigma(\pi))=\vals(\sigma(\phi))=\sigma(\phi)=\sigma(\vals(\pi))$
which shows that
equation~(\ref{logic:eqn:FUAP:strongvalsubalmostclean:valuation:commute:1})
is true. This completes the case when $\pi=\phi$ for which the
assumption of validity of $\sigma$ for $\pi$ was unnecessary. We now
assume $\pi=\axi\phi$ for some $\phi\in\pv$. We further assume that
$\pi$ is a clean proof and $\sigma$ is valid for $\pi$. We need to
show that $\sigma(\pi)$ is clean together with
equation~(\ref{logic:eqn:FUAP:strongvalsubalmostclean:valuation:commute:1}).
From definition~(\ref{logic:def:FUAP:almostclean:definition}),
having assumed that $\pi$ is a clean proof we obtain $\phi\in\avs$,
i.e. $\phi$ is an axiom modulo. Having assumed that $\sigma$ is
valid for $\pi$, from
proposition~(\ref{logic:prop:FUAP:validsubproof:recursion:axiom})
$\sigma$ is valid for $\phi$. It follows from
proposition~(\ref{logic:prop:FUAP:valsubaxmodulo:axiom:modulo}) that
$\sigma(\phi)\in{\bf A}^{+}(W)$. Hence, using
definition~(\ref{logic:def:FUAP:almostclean:definition}) once more
we see that $\sigma(\pi)=\axi\sigma(\phi)$ is a clean proof.
Furthermore from
definition~(\ref{logic:def:FUAP:valuationmod:valuation:modulo}) we
have $\vals(\sigma(\pi))=\sigma(\phi)=\sigma(\vals(\pi))$ and it
follows that
equation~(\ref{logic:eqn:FUAP:strongvalsubalmostclean:valuation:commute:1})
is true. So we now assume that $\pi=\pi_{1}\pon\pi_{2}$ where
$\pi_{1},\pi_{2}\in\pvs$ are proofs which satisfy our implication.
We need to show the same is true of $\pi$. So we assume that $\pi$
is a clean proof and furthermore that $\sigma$ is valid for $\pi$.
We need to show that $\sigma(\pi)$ is clean and
equation~(\ref{logic:eqn:FUAP:strongvalsubalmostclean:valuation:commute:1})
is true. However, using
proposition~(\ref{logic:prop:FUAP:almostclean:modus:ponens}) we see
that both $\pi_{1}$ and $\pi_{2}$ are clean proofs, and furthermore
that $\vals(\pi_{2})=\psi_{1}\to\psi_{2}$ for some
$\psi_{1},\psi_{2}\in\pv$ such that $\psi_{1}\sim\vals(\pi_{1})$ and
$\psi_{2}=\vals(\pi)$, where $\sim$ is the substitution congruence
on \pv. Moreover, using
proposition~(\ref{logic:prop:FUAP:validsubproof:recursion:pon}) we
see that $\sigma$ is valid for both $\pi_{1}$ and $\pi_{2}$. Having
assumed our induction hypothesis holds for $\pi_{1},\pi_{2}$ it
follows that $\sigma(\pi_{1})$ and $\sigma(\pi_{2})$ are both clean
proofs, and
equation~(\ref{logic:eqn:FUAP:strongvalsubalmostclean:valuation:commute:1})
is true for $\pi_{1}$ and $\pi_{2}$. So let us prove that
$\sigma(\pi)$ is a clean proof: since
$\sigma(\pi)=\sigma(\pi_{1})\pon\,\sigma(\pi_{2})$, from
proposition~(\ref{logic:prop:FUAP:almostclean:modus:ponens}) it is
sufficient to show that $\sigma(\pi_{1})$ and $\sigma(\pi_{2})$ are
clean which we already know, and furthermore that
$\vals(\sigma(\pi_{2}))=\chi_{1}\to\chi_{2}$ where
$\chi_{1}\sim\vals(\sigma(\pi_{1}))$. If this last property is true,
then $\chi_{2}=\vals(\sigma(\pi))$. So let us prove this is indeed
the case. Applying $\sigma:\pv\to{\bf P}(W)$ on both sides of
$\vals(\pi_{2})=\psi_{1}\to\psi_{2}$ we obtain
$\sigma\circ\vals(\pi_{2})=\sigma(\psi_{1})\to\sigma(\psi_{2})$.
Since
equation~(\ref{logic:eqn:FUAP:strongvalsubalmostclean:valuation:commute:1})
is true for $\pi_{2}$ it follows that
$\vals(\sigma(\pi_{2}))=\chi_{1}\to\chi_{2}$ where
$\chi_{1}=\sigma(\psi_{1})$ and $\chi_{2}=\sigma(\psi_{2})$. In
order to show that $\sigma(\pi)$ is a clean proof, it remains to
show that $\sigma(\psi_{1})\sim\vals(\sigma(\pi_{1}))$ where $\sim$
is the substitution congruence on ${\bf P}(W)$. Since
equation~(\ref{logic:eqn:FUAP:strongvalsubalmostclean:valuation:commute:1})
is true for $\pi_{1}$, this amounts to showing that
$\sigma(\psi_{1})\sim\sigma(\vals(\pi_{1}))$. Since
$\psi_{1}\sim\vals(\pi_{1})$, using
theorem~(\ref{logic:the:FOPL:mintransfsubcong:valid}) of
page~\pageref{logic:the:FOPL:mintransfsubcong:valid} it is
sufficient to show that $\sigma$ is valid for both $\psi_{1}$ and
$\vals(\pi_{1})$. The validity of $\sigma$ for $\vals(\pi_{1})$
follows from
proposition~(\ref{logic:prop:FUAP:valuationmod:valid:vals}) and the
validity of $\sigma$ for $\pi_{1}$. So it remains to show that
$\sigma$ is valid for $\psi_{1}$. Since
$\vals(\pi_{2})=\psi_{1}\to\psi_{2}$, from
proposition~(\ref{logic:prop:FOPL:valid:recursion:imp}) it is
sufficient to show that $\sigma$ is valid for $\vals(\pi_{2})$ which
follows from
proposition~(\ref{logic:prop:FUAP:valuationmod:valid:vals}) and the
validity of $\sigma$ for $\pi_{2}$. So we have proved that
$\sigma(\pi)$ is a clean proof, and it remains to show that
equation~(\ref{logic:eqn:FUAP:strongvalsubalmostclean:valuation:commute:1})
is true. However as already indicated, we have
$\vals(\sigma(\pi))=\chi_{2}=\sigma(\psi_{2})=\sigma(\vals(\pi))$
which completes our induction argument in the case when
$\pi=\pi_{1}\pon\pi_{2}$. So we now assume that $\pi=\gen x\pi_{1}$
where $x\in V$ and $\pi_{1}\in\pvs$ is a proof which satisfies our
implication. We need to show the same is true of $\pi$. So we assume
that $\pi$ is a clean proof and furthermore that $\sigma$ is valid
for $\pi$. We need to show that $\sigma(\pi)$ is clean and
equation~(\ref{logic:eqn:FUAP:strongvalsubalmostclean:valuation:commute:1})
is true. However from
proposition~(\ref{logic:prop:FUAP:almostclean:generalization}),
$\pi_{1}$ is a clean proof and $x\not\in\spec(\pi_{1})$.
Furthermore, using
proposition~(\ref{logic:prop:FUAP:validsubproof:recursion:gen}),
$\sigma$ is valid for $\pi_{1}$ and furthermore for all $u\in V$ we
have the implication:
    \begin{equation}\label{logic:eqn:FUAP:validsubcleanproof:1:1}
    u\in\free(\gen x\pi_{1})\ \Rightarrow\
    \sigma(u)\neq\sigma(x)
    \end{equation}
Having assumed our induction hypothesis holds for $\pi_{1}$ it
follows that $\sigma(\pi_{1})$ is a clean proof and
equation~(\ref{logic:eqn:FUAP:strongvalsubalmostclean:valuation:commute:1})
is true for $\pi_{1}$. So let us prove that $\sigma(\pi)$ is a clean
proof: since $\sigma(\pi)=\gen\sigma(x)\sigma(\pi_{1})$, from
proposition~(\ref{logic:prop:FUAP:almostclean:generalization}) it is
sufficient to prove that $\sigma(\pi_{1})$ is clean and
$\sigma(x)\not\in\spec(\sigma(\pi_{1}))$. We already know that
$\sigma(\pi_{1})$ is a clean proof so we only need to show that
$\sigma(x)\not\in\spec(\sigma(\pi_{1}))$. So suppose to the contrary
that $\sigma(x)\in\spec(\sigma(\pi_{1}))$. From
proposition~(\ref{logic:prop:FUAP:freevar:substitution}) we have
$\spec(\sigma(\pi_{1}))\subseteq\sigma(\spec(\pi_{1}))$ and it
follows that $\sigma(x)=\sigma(u)$ for some $u\in\spec(\pi_{1})$.
Having established that $x\not\in\spec(\pi_{1})$ we must have $u\neq
x$ and consequently $u\in\spec(\pi_{1})\setminus\{x\}$. However,
since $\pi_{1}$ is a clean proof, from
proposition~(\ref{logic:prop:FUAP:cleanproof:spec:free}) we have
$\spec(\pi_{1})\subseteq\free(\pi_{1})$. It follows that
$u\in\free(\gen x\pi_{1})$ while $\sigma(u)=\sigma(x)$. This
contradicts the
implication~(\ref{logic:eqn:FUAP:validsubcleanproof:1:1}). So we
have proved that $\sigma(\pi)$ is clean and it remains to prove
equation~(\ref{logic:eqn:FUAP:strongvalsubalmostclean:valuation:commute:1})
which goes as follows:
    \begin{eqnarray*}
    \vals(\sigma(\pi))&=&\vals(\sigma(\gen x\pi_{1}))\\
    \mbox{def.~(\ref{logic:def:FUAP:substitution:substitution})}\ \rightarrow
    &=&\vals(\gen\sigma(x)\sigma(\pi_{1}))\\
    \sigma(x)\not\in\spec(\sigma(\pi_{1}))\ \rightarrow
    &=&\forall\sigma(x)\vals(\sigma(\pi_{1}))\\
    \mbox{(\ref{logic:eqn:FUAP:strongvalsubalmostclean:valuation:commute:1})
     true for $\pi_{1}$}\ \rightarrow
    &=&\forall\sigma(x)\sigma(\vals(\pi_{1}))\\
    \mbox{def.~(\ref{logic:def:substitution})}\ \rightarrow
    &=&\sigma(\forall x\vals(\pi_{1}))\\
    x\not\in\spec(\pi_{1})\ \rightarrow
    &=&\sigma(\vals(\gen x\pi_{1}))\\
    &=&\sigma(\vals(\pi))\\
    \end{eqnarray*}
\end{proof}
