Our aim is to introduce the idea of {\em definition by structural
recursion} on a free universal algebra of type $\alpha$. We shall
deal with the subject in a later section. In this section, we shall
concentrate on a simpler notion, that of {\em definition by
recursion over $\N$}, also known as {\em definition by simple
recursion}. Consider the function $g:\N\to\N$ defined by $g(0)=0$
and:
    \[
    g(n+1)=(n+1).\,g(n)\ ,\ \forall n\in\N
    \]
The existence of the function $g$ is often taken for granted. Most
of us needed many years as mathematical students before it finally
occurred that the truth of the following statement was far from
obvious:
    \[
    \exists g[\,(g:\N\to\N)\land(g(0)=1)\land(\forall n\in\N\ ,\ g(n+1)=(n+1).\,g(n))\,]
    \]
Loosely speaking, defining a map $g$ by recursion consists in first
defining $g(0)$, and having defined $g(0)$, $g(1)$,\ldots $g(n-1)$,
to then specify what $g(n)$ is {\em as a function of} all the
preceding values $g(0)$, $g(1)$,\ldots, $g(n-1)$. More formally,
suppose $A$ is a set and we wish to define a map $g:\N\to A$ by
recursion over $\N$. We first define $g(0)\in A$, and then define
$g(n)$ as a function of $g_{|n}$. In other words, we set
$g(n)=h(g_{|n})$ for some function $h$. Incidentally, the function
$h$ is called an {\em oracle function}, as it enables us to predict
what the next value of $g$ is, given all the preceding values. Of
course, we want everything to make sense: so $h(g_{|n})$ should
always be meaningful, and since our objective is to obtain $g:\N\to
A$ in which case $g_{|n}\in A^{n}$ for all $n\in\N$, we should
probably request that the oracle function be a map
$h:\cup_{n\in\N}A^{n}\to A$. We could imagine an oracle function
with a wider domain and a wider range (as long as $h(g_{|n})$ is a
well defined element of $A$), but this is not likely to add much to
the generality of the present discussion. So we shall stick to
$h:\cup_{n\in\N}A^{n}\to A$. One thing to note is that the
expression $g(n)=h(g_{|n})$ is also meaningful in the case of $n=0$.
It is therefore unnecessary to define $g(0)$ separately: just set
$h(\emptyset)$ appropriately and request that $g(n)=h(g_{|n})$ for
all $n\in\N$ rather than just $n\in\N^{*}$. In the above example of
$g(n)=n!$ and $A=\N$, the oracle function
$h:\cup_{n\in\N}\N^{n}\to\N$ should be defined by setting
$h(\emptyset)=1$ and for all $n\in\N^{*}$ and $g:n\to\N$:
    \[
    h(g)=n.g(n-1)
    \]
Note that this definition is legitimate since
$\N^{n}\cap\N^{n'}=\emptyset$ for $n\neq n'$.

We are now faced with a crucial question, namely that of the truth
of the following statement, given a set $A$ and an oracle function
$h:\cup_{n\in\N}A^{n}\to A$.
    \[
    \exists g[\,(g:\N\to A)\land(\forall n\in\N\ ,\ g(n)=h(g_{|n})\,]
    \]
This will be proved in the next lemma, together with the uniqueness
of the function $g:\N\to A$. However before we proceed, a few
comments are in order. The proof of
lemma~(\ref{logic:lemma:recursion:over:N}) below may appear daunting
and tedious, as we systematically check a variety of small technical
details. However, there is not much to it. The key to remember is
that a {\em recursion over $\N$} is nothing but an {\em induction
over $\N$}. Specifically, for all $n\in\N$, there must exist a map
$g:n\to\N$ with the right property, for if there exists an $n\in\N$
for which this is not the case, choosing $n$ to be the smallest of
such integer, we can extend the map $g:(n-1)\to A$ using
$g(n)=h(g_{|n})$ and obtain a map $g':n\to A$ with the right
property, thereby reaching a contradiction. Once the induction
argument is complete, proving the existence of a map $g:\N\to A$ is
easily done by gathering together all the ordered pairs contained in
the various maps $g':n\to A$ for $n\in\N$, which incidentally are
extensions of one another.
\index{recursion@Definition by recursion
over \N}
\begin{lemma}\label{logic:lemma:recursion:over:N}
Let $A$ be a set and $h:\cup_{n\in\N}A^{n}\to A$ be a map. There
exists a unique map $g:\N\to A$ such that $g(n)=h(g_{|n})$ for all
$n\in\N$.
\end{lemma}
\begin{proof}
First we prove the uniqueness property. Suppose $g,g':\N\to A$ are
maps such that $g(n)=h(g_{|n})$ and $g'(n)=h(g'_{|n})$ for all
$n\in\N$. We need to show that $g=g'$, i.e. that $g(n)=g'(n)$ for
all $n\in\N$. One way to achieve this is to prove that
$g_{|n}=g'_{|n}$ for all $n\in\N$, for which we shall use an
induction argument. Since $g_{|0}=\emptyset=g'_{|0}$ the property is
true for $n=0$. Suppose it is true for $n\in\N$. Then
$g_{|n}=g'_{|n}$ and consequently $g(n)=h(g_{|n})=h(g'_{|n}) =
g'(n)$. Hence:
    \[
    g_{|(n+1)}=g_{|n}\cup\{(n,g(n))\}=g'_{|n}\cup\{(n,g'(n))\}=g'_{|(n+1)}
    \]
and we see that the property is true for $n+1$. This completes the
proof of the uniqueness property. In order to prove the existence of
$g:\N\to A$ with $g(n)=h(g_{|n})$ for all $n\in\N$, we shall first
restrict our attention to finite domains and prove by induction the
existence for all $n\in\N$ of a map $g:n\to A$ such that
$g(k)=h(g_{|k})$ for all $k\in n$. This is clearly true for $n=0$,
since the empty set is a map $g:0\to A$ for which the condition
$g(k)=h(g_{|k})$ for all $k\in 0$ is vacuously satisfied. So we
assume $n\in \N$, and the existence of a map $g:n\to A$ such that
$g(k)=h(g_{|k})$ for all $k\in n$. We need to show the existence of
a map $g':(n+1)\to A$ such that $g'(k)=h(g'_{|k})$ for all $k\in
n+1$. Consider the map $g':(n+1)\to A$ defined by $g'_{|n}=g$ and
$g'(n)=h(g)$, and suppose $k\in n+1$. We need to check that
$g'(k)=h(g'_{|k})$. If $k=n$, we need to check that
$g'(n)=h(g'_{|n})=h(g)$ which is true by definition of $g'$. If
$k\in n$, then:
    \[
    g'(k)=g'_{|n}(k)=g(k)=h(g_{|k})=h(g'_{|k})
    \]
This completes our induction argument and for all $n\in\N$ we have
proved the existence of a map $g:n\to A$ such that $g(k)=h(g_{|k})$
for all $k\in n$. In fact we claim that such a map is unique. For if
$g':n\to A$ is another such map, an identical induction argument to
the one already used shows that $g(k)=g'(k)$ for all $k\in n$.
Hence, for all $n\in\N$, we have proved the existence of a {\em
unique} map $g_{n}:n\to A$ such that $g_{n}(k)=h((g_{n})_{|k})$ for
all $k\in n$. Note that from the uniqueness property we must have
$(g_{n+1})_{|n}=g_{n}$ for all $n\in\N$. So every $g_{n+1}$ is an
extension of $g_{n}$. We are now in a position to prove the
existence of a map $g:\N\to A$ such that $g(n)=h(g_{|n})$ for all
$n\in\N$, by considering $g=\cup_{n\in\N}g_{n}$. First we show that
$g$ is indeed a map. It is clearly a set of ordered pairs. Suppose
$(x,y)\in g$ and $(x,y')\in g$. We need to show that $y=y'$.
However, there exist $n,n'\in\N$ such that $(x,y)\in g_{n}$ and
$(x,y')\in g_{n'}$. Without loss of generality, we may assume that
$n\leq n'$. From the above uniqueness property, we must have
$(g_{n'})_{|n}=g_{n}$. From $(x,y)\in g_{n}$ we obtain $x\in n$ and
consequently:
    \[
    y'=g_{n'}(x)=(g_{n'})_{|n}(x)=g_{n}(x)=y
    \]
So we have proved that $g$ is indeed a map. It is clear that
$\dom(g)=\N$ and $\rng(g)\subseteq A$. So $g$ is a map $g:\N\to A$.
It remains to check that $g(n)=h(g_{|n})$ for all $n\in\N$. So
suppose $n\in\N$. Let $k\in n+1$. Since $(k,g_{n+1}(k))\in
g_{n+1}\subseteq g$, we obtain $g_{n+1}(k)=g(k)$. It follows in
particular that $(g_{n+1})_{|n}=g_{|n}$ and $g_{n+1}(n)=g(n)$. Since
$g_{n+1}(k)=h((g_{n+1})_{|k})$ for all $k\in n+1$ we conclude that:
    \[
    g(n)=g_{n+1}(n)=h((g_{n+1})_{|n})=h(g_{|n})
    \]
\end{proof}

There is however one thing to note: the proof of
lemma~(\ref{logic:lemma:recursion:over:N}) tacitly makes use of the
Axiom Schema of Replacement without any explicit mention of the
fact. This is one other thing we usually take for granted. We have
avoided any mention of the Axiom Schema of Replacement to keep the
proof leaner, thereby following standard mathematical practice.
However, considering the low level nature of the result being
proved, we feel it is probably beneficial to say a few words about
it, or else the whole proof could be construed as a cheat. At some
point, we considered the set $g=\cup_{n\in\N} g_{n}$ which is really
a notational shortcut for $g=\cup B$ where $B=\{g':\exists n\in\N\
,\ g'=g_{n}\}$. Recall that given a set $z$, the set $\cup z$ is
defined as:
    \[
    \cup z = \{x:\exists y\ ,\ (x\in y)\land(y\in z)\}
    \]
The existence of this set is guaranteed by the Axiom of Union. So it
would seem that defining $g=\cup_{n\in\N}g_{n}$ is a simple
application of the Axiom of Union and of course it is. But there is
more to the story: we need to justify the fact that $B$ is indeed a
set and we have not done so. If we knew that
$G:\N\to\cup_{n\in\N}A^{n}$ defined by $G(n)=g_{n}$ was a map, then
$B$ would simply be the range of $G$, i.e. $B=\rng(G)$. Showing that
$B$ is a set would still require the Axiom Schema of Replacement (as
far as we can tell), but it would not be controversial to remain
silent about it. But why is $G$ not a map? Why can't we define
$G:\N\to\cup_{n\in\N}A^{n}$ by setting $G(n)=g_{n}$ for all
$n\in\N$? This is something we do all the time.

Consider $n\in\N$. We proved the existence and uniqueness of a map
$g:n\to A$ such that $g(k)=h(g_{|k})$ for all $k\in n$. We then {\em
labeled} this unique map as '$g_{n}$'. In fact, we proved that the
following formula of first order predicate logic:
    \[
    G[n,g]=(n\in\N)\land[\,(g:n\to A)\land(\forall k\in n\ ,\ g(k)=h(g_{|k}))\,]
    \]
is a functional class with domain $\N$. We have not formally defined
what a {\em functional class} is and we are not able to do so at
this stage. Informally, $G$ is a functional class as a formula of
first order predicate logic where two free variables $n$ and $g$ (in
that order) have been singled out, and which satisfies:
    \[
    \forall n\forall g\forall g'[\,G[n,g]\land G[n,g']\to (g=g')\,]
    \]
where $G[n,g']$ has the {\em obvious} meaning. Note that $G$ has
other variables such as $A$ and $h$, while $\N$ is a constant. The
functional class $G$ has a domain:
    \[
    \dom(G)[n]=\exists g\, G[n,g]
    \]
which is itself a formula of first order predicate logic with the
free variable $n$ singled out, something which we should call a {\em
class}. As it turns out, we have:
    \[
    \forall n[\, \dom(G)[n]\leftrightarrow (n\in\N)\,]
    \]
which allows us to say that the functional class $G$ has domain
$\N$. In general, the domain of a functional class need not be a
set. The functional class $G[n,g]$ also has a range, which is
defined as the class:
    \[
    \rng(G)[g]=\exists n\, G[n,g]
    \]
The Axiom Schema of Replacement asserts that if the domain of a
functional class is a set, then its range is also a set. More
formally:
    \[
    \exists B\forall g[\,(g\in B)\leftrightarrow \rng(G)[g]\,]
    \]
which shows as requested, that $B=\{g':\exists n\in\N\ ,\
g'=g_{n}\}$ is indeed a set, since '$g'=g_{n}$' is simply a
notational shortcut for '$G[n,g']$'. So much for proving that $B$ is
a set. We did not need to define a map $G:\N\to\cup_{n\in\N}A^{n}$
by setting $G(n)=g_{n}$ for all $n\in\N$. We considered a functional
class instead. But the question still stands: the practice of
casually defining a map $G:\N\to\cup_{n\in\N}A^{n}$ by setting
$G(n)=g_{n}$ is very common. Surely, it is a legitimate practice. To
avoid a confusing conflict of notation let us keep '$G$' as
referring to our functional class, and consider the map
$G^{*}:\N\to\cup_{n\in\N}A^{n}$ defined by $G^{*}(n)=g_{n}$ for all
$n\in\N$. What we are really saying is that $G^{*}$ is the set of
ordered pairs defined by $G^{*}=\{(n,g_{n}): n\in\N\}$ or
equivalently $G^{*}=\{(n,g):G[n,g]\}$. This definition is legitimate
provided $G^{*}$ is really a set. More formally, we need to prove:
    \begin{equation}\label{logic:eqn:functional:class:set}
    \exists G^{*}\forall z[\,(z\in G^{*})\leftrightarrow\exists n\exists g[(z=(n,g))\land G[n,g]]\,]
    \end{equation}
However, $G$ is a functional class whose domain is the set $\N$,
while its range is the set $B$. Hence, we have the implication
$G[n,g]\to((n,g)\in\N\times B)$, and it follows
that~(\ref{logic:eqn:functional:class:set}) can equivalently be
expressed as:
    \begin{equation}\label{logic:eqn:functional:class:set:2}
     \exists G^{*}\forall z[\,(z\in G^{*})\leftrightarrow(z\in \N\times B)\land \phi[z]\,]
    \end{equation}
where $\phi[z]$ stands for $\exists n\exists g[(z=(n,g))\land
G[n,g]]$. Since~(\ref{logic:eqn:functional:class:set:2}) is a direct
consequence of the Axiom Schema of Comprehension, this completes our
proof that $G^{*}$ is indeed a set. So much for
lemma~(\ref{logic:lemma:recursion:over:N}).

Unfortunately, we are not in a position to close  our discussion on
the subject of {\em recursion over $\N$}. Yes we have
lemma~(\ref{logic:lemma:recursion:over:N}), of which we gave a
rather detailed proof, including final meta-mathematical remarks
which hopefully were as convincing as one could have made them. The
truth is, lemma~(\ref{logic:lemma:recursion:over:N}) is rather
useless, as it does not do what we need: suppose we want to define
the map $g$ with domain $\N$, by setting $g(0)=\{0\}$ and
$g(n+1)={\cal P}(g(n))$ for all $n\in\N$. In order to apply
lemma~(\ref{logic:lemma:recursion:over:N}) we need to produce a set
$A$. Yet we have no idea what the set $A$ should be. So we are back
to the very beginning of this section, asking the very same question
as initially: does there exists a map $g$ with domain $\N$ such that
$g(0)=\{0\}$ and $g(n+1)={\cal P}(g(n))$ for all $n\in\N$? Of course
once we know that $g$ exists, it is possible for us to produce a set
$A$, namely $A=\rng(g)$ or anything larger. We can define an oracle
function $h:\cup_{n\in\N}A^{n}\to A$ by setting $h(\emptyset)=\{0\}$
and $h(g)={\cal P}(g(n-1))$ for all $n\in\N^{*}$ and $g:n\to A$. At
this stage, lemma~(\ref{logic:lemma:recursion:over:N}) works like a
dream. But it is clearly too late, as we needed to know the
existence of $g$ in the first place.

One may argue that defining $g(0)=\{0\}$ and $g(n+1)={\cal P}(g(n))$
is not terribly useful, and it may be that
lemma~(\ref{logic:lemma:recursion:over:N}) is pretty much all we
need in practice. But remember
proposition~(\ref{logic:prop:construction}) in which we claim to
have constructed a free universal algebra of type $\alpha$. We
crucially defined the sequence $(Y_{n})_{n\in\N}$ with a recursion
over $\N$, by setting $Y_{0}=\{(0,x):x\in X_{0}\}$,
$Y_{n+1}=Y_{n}\cup\bar{Y}_{n}$ and:
    \[
    \bar{Y}_{n}=\left\{(1,(f,x)):\ f\in\alpha\ ,\ x\in Y_{n}^{\alpha(f)}\right\}\ ,\ \forall n\in\N
    \]
Just as in the case of $g(n+1)={\cal P}(g(n))$, it is not clear what
set $A$ should be used in order to apply
lemma~(\ref{logic:lemma:recursion:over:N}). We need something more
powerful.

Let us go back to the initial problem: we would like to define a map
$g$ with domain $\N$ by first specifying $g(0)$, and having defined
$g(0)$, $g(1)$, \ldots,  $g(n-1)$, by specifying what $g(n)$ is {\em
as a function of} $g(0)$, $g(1)$,\ldots, $g(n-1)$. So we need an
oracle function $h:\cup_{n\in\N}A^{n}\to A$ so as to be able to set
$g(n)=h(g_{|n})$. Finding such an oracle function will not be an
easy task without a set $A$. But why should $A$ be a set? The only
reason we insisted on $A$ being a set was to keep clear of the
flakiness of meta-mathematics. We wanted a theorem proper, namely
lemma~(\ref{logic:lemma:recursion:over:N}), with a solid proof based
on a standard mathematical argument. In the case of induction, it
was possible for us to stay within the realm of standard
mathematics. When it comes to recursion, it would seem that we have
no choice but to step outside of it.

So let us assume that $A$ is a class, namely a formula of first
order predicate logic with a variable $x$ singled out. We could
denote this class $A[x]$ to emphasize the fact that $A$ is viewed as
a {\em predicate} over the variable $x$. Of course, the formula $A$
may have many other variables, and in fact $x$ itself need not be a
free variable of $A$. We could have $A[x]=\lnot\bot$, also denoted
$A[x]=\top$ or indeed:
    \[
    A[x]=(\forall x (x\in x))\lor\lnot(\forall x(x\in x))
    \]
The class $A$ is simply such that $\forall x A[x]$ is a true
statement. This class is the {\em largest} possible, since $A[x]$ is
true for all $x$. We could call $A[x]$ the {\em category of all
sets} also denoted {\bf Set}, but we have no need to do so. Note
that it is possible for this class to have $x$ as a free variable,
as in:
    \[
    A[x] = (y\in x)\lor\lnot(y\in x)
    \]
The good thing about having $A$ as a class is that it no longer
requires us to know what it is: if we have no idea as to which class
to pick, we can simply decide that $A$ is the class of all sets. Now
we need an oracle function $h:\cup_{n\in\N}A^{n}\to A$. Since $A$ is
not a set but a class, we should be looking for a functional class
$H:\cup_{n\in\N}A^{n}\to A$, rather than a function. But we hardly
know what this means, and a few clarifications are in order: first
of all, given a functional class $H[x,y]$, given two classes $A[x]$
and $B[x]$, we need to give meaning to the statement $H:B\to A$.
Informally, this statement should indicate that $H$ is a functional
class with domain $B$ and range {\em inside} $A$. The fact that
$H[x,y]$ is a functional class is formally translated as:
    \[
    \forall x\forall y\forall y'[\,H[x,y]\land H[x,y']\to(y=y')\,]
    \]
The fact that $H$ has domain $B$ can be written as:
    \[
    \forall x[\,B[x]\leftrightarrow\exists y(H[x,y])\,]
    \]
and finally expressing that the range of $H$ is {\em inside} $A$ can be written as:
    \[
    \forall y[\,\exists x(H[x,y])\to A[y]\,]
    \]
So we now know what the statement $H:B\to A$ formally represents. If
we intend to give meaning to $H:\cup_{n\in\N}A^{n}\to A$, we still
need to spell out what the class $\cup_{n\in\N}A^{n}$ is. First we
define the class $A^{n}[x]$ for $n\in\N$. Informally, $A^{n}[x]$
should indicate that $x$ is a map with domain $n$ and range {\em
inside} $A$, i.e.:
    \[
    A^{n}[x]=(x\mbox{\ is a map})\land(\dom(x)=n)\land\forall y[\,(y\in\rng(x)\to A[y])\,]
    \]
Finally, the class $\cup_{n\in\N}A^{n}$ should be defined as:
    \[
    \left(\cup_{n\in\N}A^{n}\right)[x]=\exists n[\,(n\in\N)\land A^{n}[x]\,]
    \]
So the statement $H:\cup_{n\in\N}A^{n}\to A$ is now meaningful. So
suppose it is true. We are looking for a map $g:\N\to A$ such that
$g(n)=H(g_{|n})$ for all $n\in\N$. Things are starting to be
clearer. The statement $g:\N\to A$ is clearly represented by:
    \[
    (g\mbox{\ is a map})\land(\dom(g)=\N)\land\forall y[\,(y\in\rng(g)\to A[g])\,]
    \]
while the statement $g(n)=H(g_{|n})$ is an intuitive way of saying
$H[g_{|n},g(n)]$. Furthermore, if we have $H:\cup_{n\in\N}A^{n}\to
A$ and $g:\N\to A$, then for all $n\in A$ it is easy to check that
$g_{|n}:n\to A$, by which we mean that $A^{n}[g_{|n}]$ is true. It
follows that $\dom(H)[g_{|n}]$ is true, and consequently since $H$
is functional, there exists a unique $y_{n}$ such that
$H[g_{|n},y_{n}]$, which furthermore is such that $A[y_{n}]$ is
true. We would like $g:\N\to A$ to be such that $g(n)$ is precisely
that unique $y_{n}$, for all $n\in\N$. So it all makes sense.

As an example, let us go back to the case when $g(n+1)={\cal
P}(g(n))$ for all $n\in\N$ with $g(0)=\{0\}$. Because we have no
idea what the class $A$ should be, let us pick $A$ to be the class
of all sets. We need to find an oracle functional class
$H:\cup_{n\in\N}A^{n}\to A$ which correctly sets up our recursion
objective. Informally speaking, we want $H(\emptyset)=\{0\}$ and
$H(g)={\cal P}(g(n-1))$ for all $n\in\N^{*}$ and $g:n\to A$. If we
want to formally code up the functional class $H[g,y]$ as a formula
of first order predicate logic, since $A^{0}[g]=(g=\emptyset)$, this
can be done as:
    \[
    H[g,y]=[A^{0}[g]\land(y=\{0\})]\lor\exists n[(n\in\N^{*})
    \land A^{n}[g]\land (y={\cal P}(g(n-1)))]
    \]
The fundamental question is this: does there exist a map $g$ with
domain $\N$ such that $g(n)=H(g_{|n})$ for all $n\in\N$? This is the
object of meta-theorem~(\ref{logic:meta:recursion:over:N}) below.

Before, we proceed with the proof of
meta-theorem~(\ref{logic:meta:recursion:over:N}), a few remarks may
be in order: we were compelled by the nature of the problem to
indulge in a fair amount of meta-mathematical discussions, i.e.
discussions involving formulas of first order logic, variables,
classes and functional classes, terms which have not been defined
anywhere. This of course falls very short of the usually accepted
standards of mathematical proof. It is however better than nothing
at all. We do not have the conceptual tools to do much more at this
stage. It is hoped that the study of universal algebras and the
introduction of formal languages as algebraic structures will
eventually allow us to handle meta-mathematics without the hand
waving.   For now, we did everything we could and were as careful as
possible in attempting to give legitimacy to the principle of
recursion over $\N$. So we shall accept the existence of a sequence
$(Y_{n})_{n\in\N}$ such that $Y_{n+1}=Y_{n}\cup\bar{Y}_{n}$ for all
$n\in\N$, where:
    \[
    \bar{Y}_{n}=\left\{(1,(f,x)):\ f\in\alpha\ ,\ x\in Y_{n}^{\alpha(f)}\right\}\ ,\
    \forall n\in\N
    \]
with the belief the existence of $(Y_{n})_{n\in\N}$ could be
formally proven within {\bf ZFC}, if we precisely knew what that
means. If it so happens that one of our conclusions is wrong, it is
very likely that the error will be caught at some point, once we
have clarified the ideas of formal languages and formal proofs.
\index{recursion@Definition by recursion over \N}
\begin{metath}\label{logic:meta:recursion:over:N}
Let $A$ be a class and $H:\cup_{n\in\N}A^{n}\to A$ be a functional
class. There exists a unique map $g:\N\to A$ such that
$g(n)=H(g_{|n})$ for all $n\in\N$.
\end{metath}
\begin{proof}
First we prove the uniqueness property. Suppose $g,g':\N\to A$ are
maps such that $g(n)=H(g_{|n})$ and $g'(n)=H(g'_{|n})$ for all
$n\in\N$. We need to show that $g=g'$, i.e. that $g(n)=g'(n)$ for
all $n\in\N$. One way to achieve this is to prove that
$g_{|n}=g'_{|n}$ for all $n\in\N$, for which we shall use an
induction argument. Since $g_{|0}=\emptyset=g'_{|0}$ the property is
true for $n=0$. Suppose it is true for $n\in\N$. Then
$g_{|n}=g'_{|n}$ and consequently $g(n)=H(g_{|n})=H(g'_{|n}) =
g'(n)$. Hence:
    \[
    g_{|(n+1)}=g_{|n}\cup\{(n,g(n))\}=g'_{|n}\cup\{(n,g'(n))\}=g'_{|(n+1)}
    \]
and we see that the property is true for $n+1$. This completes the
proof of the uniqueness property. In order to prove the existence of
$g:\N\to A$ with $g(n)=H(g_{|n})$ for all $n\in\N$, we shall first
restrict our attention to finite domains and prove by induction the
existence for all $n\in\N$ of a map $g:n\to A$ such that
$g(k)=H(g_{|k})$ for all $k\in n$. This is clearly true for $n=0$,
since the empty set is a map $g:0\to A$ for which the condition
$g(k)=H(g_{|k})$ for all $k\in 0$ is vacuously satisfied. So we
assume $n\in \N$, and the existence of a map $g:n\to A$ such that
$g(k)=H(g_{|k})$ for all $k\in n$. We need to show the existence of
a map $g':(n+1)\to A$ such that $g'(k)=H(g'_{|k})$ for all $k\in
n+1$. Consider the map $g':(n+1)\to A$ defined by $g'_{|n}=g$ and
$g'(n)=H(g)$, and suppose $k\in n+1$. We need to check that
$g'(k)=H(g'_{|k})$. If $k=n$, we need to check that
$g'(n)=H(g'_{|n})=H(g)$ which is true by definition of $g'$. If
$k\in n$, then:
    \[
    g'(k)=g'_{|n}(k)=g(k)=H(g_{|k})=H(g'_{|k})
    \]
This completes our induction argument and for all $n\in\N$ we have
proved the existence of a map $g:n\to A$ such that $g(k)=H(g_{|k})$
for all $k\in n$. In fact we claim that such a map is unique. For if
$g':n\to A$ is another such map, an identical induction argument to
the one already used shows that $g(k)=g'(k)$ for all $k\in n$.
Hence, for all $n\in\N$, we have proved the existence of a {\em
unique} map $g:n\to A$ such that $g(k)=H(g_{|k})$ for all $k\in n$.
It follows that:
    \[
    G[n,g]=(n\in\N)\land[\,(g:n\to A)\land(\forall k\in n\ ,\ g(k)=H(g_{|k}))\,]
    \]
is a functional class with domain $\N$. From the Axiom Schema of
Replacement:
    \[
    \rng(G)=\{g:\exists n\in\N\ ,\ G[n,g]\}
    \]
is therefore a set. We define $g=\cup\,\rng(G)$ and we shall
complete our proof by showing that $g:\N\to A$ is a map such that
$g(n)=H(g_{|n})$ for all $n\in\N$. First we show that $g$ is a set
of ordered pairs. Let $z\in g$. There exists $g'\in \rng(G)$ such
that $z\in g'$. Since $g'\in\ rng(G)$, we have $G[n,g']$ for some
$n\in\N$. It follows that $g':n\to A$ and in particular $g'$ is a
set of ordered pairs. From $z\in g'$ we see that $z$ is an ordered
pair. We then show that $g$ is functional. So we assume that
$(x,y)\in g$ and $(x,y')\in g$. We need to show that $y=y'$. Since
$g=\cup\,\rng(G)$, there exist $g_{1},g_{2}\in \rng(G)$ such that
$(x,y)\in g_{1}$ and $(x,y')\in g_{2}$. Furthermore, there exist
$n_{1},n_{2}\in\N$ such that $G[n_{1},g_{1}]$ and $G[n_{2},g_{2}]$.
Without loss of generality we may assume that $n_{1}\leq n_{2}$.
From $G[n_{2},g_{2}]$ it is not difficult to show that
$G[n_{1},(g_{2})_{|n_{1}}]$ is also true. Since $G$ is functional,
it follows that $g_{1}=(g_{2})_{|n_{1}}$. From $(x,y)\in g_{1}$ and
$g_{1}:n_{1}\to A$ we obtain $x\in n_{1}$. Hence:
    \[
    y'=g_{2}(x)=(g_{2})_{|n_{1}}(x)=g_{1}(x)=y
    \]
We now show that $\dom(g)=\N$. First we show that $\N\subseteq
\dom(g)$. So let $n\in\N$. We need to show that $n\in \dom(g)$.
However, the functional class $G$ has domain $\N$. In particular,
there exists $g'$ such that $G[n+1,g']$ is true. From $g':(n+1)\to
A$ we see that $(n,g'(n))\in g'$. Furthermore, from $G[n+1,g']$ we
see that $g'\in \rng(G)$. Since $g=\cup\,\rng(G)$, we conclude that
$(n,g'(n))\in g$, and in particular $n\in \dom(g)$. We now prove
that $\dom(g)\subseteq\N$. So let $x\in \dom(g)$. We need to show
that $x\in\N$. However, there exists $y$ such that $(x,y)\in g$.
Hence, there exists $g'\in \rng(G)$ such that $(x,y)\in g'$.
Furthermore, there exists $n\in\N$ such that $G[n,g']$ is true. In
particular, we have $g':n\to A$ and we conclude from $(x,y)\in g'$
that $x\in n\subseteq\N$. So $x\in\N$. In order to show that
$g:\N\to A$, it remains to check that $A[g(n)]$ is true for all
$n\in\N$. So let $n\in\N$. We need to show that $A[g(n)]$ is true.
However $n\in n+1$ and since $G$ has domain $\N$, there exists $g'$
such that $G[n+1,g']$. From $g':(n+1)\to A$ we obtain $(n,g'(n))\in
g'$. Furthermore, $A[g'(n)]$ is true. From $G[n+1,g']$ we obtain
$g'\in \rng(G)$, and it follows that $(n,g'(n))\in g$. Hence we see
that $g'(n)=g(n)$ and since $A[g'(n)]$ is true we conclude that
$A[g(n)]$ is also true. So we have proved that $g:\N\to A$ is a map.
It remains to check that $g(n)=H(g_{|n})$ for all $n\in\N$. So let
$n\in\N$. We need to show that $g(n)=H(g_{|n})$. Consider once more
the map $g'\in \rng(G)$ such that $G[n+1,g']$ is true. Since
$g':(n+1)\to A$, for all $k\in n+1$ we have $(k,g'(k))\in g'$ and
consequently $(k,g'(k))\in g$. It follows that $g'(k)=g(k)$ for all
$k\in n+1$, and we see that $g'_{|n}=g_{|n}$ as well as
$g'(n)=g(n)$. From $G[n+1,g']$ we also have $g'(k)=H(g'_{|k})$ for
all $k\in n+1$. In particular, we have:
    \[
    g(n)=g'(n)=H(g'_{|n})=H(g_{|n})
    \]
\end{proof}
