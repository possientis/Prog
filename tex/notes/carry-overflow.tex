\documentclass{article}
\usepackage{amssymb}
\usepackage{hyperref}
\newcommand{\N}{\mbox{$\bf N$}}
\newcommand{\Z}{\mbox{$\bf Z$}}
\newcommand{\Q}{\mbox{$\bf Q$}}
\newcommand{\R}{\mbox{$\bf R$}}
\newcommand{\C}{\mbox{$\bf C$}}
\newcommand{\om}{\mbox{$\omega$}}
\newcommand{\ifand}{if and only if}
\newcommand{\Cat}{\mbox{$\cal C$}}
\newcommand{\Cone}{\mbox{${\cal C}_{1}$}}
\newcommand{\Ctwo}{\mbox{${\cal C}_{2}$}}
\newcommand{\Dat}{\mbox{$\cal D$}}
\newcommand{\Eat}{\mbox{$\cal E$}}
\newcommand{\Set}{\mbox{\bf Set}}
\newcommand{\Cop}{\mbox{${\cal C}^{op}$}}
\newcommand{\Dop}{\mbox{${\cal D}^{op}$}}
\newcommand{\ob}{\mbox{\rm Ob}}
\newcommand{\arr}{\mbox{\rm Arr}}
\newcommand{\id}{\mbox{\rm id}}
\newcommand{\dom}{\mbox{\rm dom}}
\newcommand{\cod}{\mbox{\rm cod}}


\newtheorem{defin}{Definition}
\newtheorem{prop}{Proposition}
\newtheorem{lemma}{Lemma}
\newtheorem{theorem}{Theorem}
\newtheorem{metath}{Metatheorem}
\newtheorem{axiom}{Axiom}

\newenvironment{proof}{{\bf Proof}\\}{$\diamond$}

\newcommand{\zq}{\mbox{${\bf Z}_{q}$}}
\title{A Note on Carry and Overflow}
\author{Paul Ossientis}

\begin{document}
\maketitle

\section{Introduction}
In this note we assume given a natural number $q=2^{n}$ where 
$n\in\{8,16,32,64\}$. We denote \zq\ the ring of integers modulo $q$. 
The purpose of this note is to provide a formal presentation of the mathematics
underlying some operations of a computer's CPU and explain the notions of {\em
carry} and {\em overflow} within that formal framework. It all starts with the
realization that computer hardware is designed to perfectly perform the
operation of addition $+$ on the ring \zq. There is no approximation,
there is no error, no overflow, the result is always perfect and exact: 
computer hardware naturally operates on \zq. It is also very good at 
inverting bits: so given $x\in\zq$ it can easily compute $\lnot x$.
Furthermore, since $x+\lnot x = q-1$ we have the equality $-x = \lnot x + 1$.
It follows that a CPU can easily (and exactly) compute the opposit $-x$ of 
any $x\in\zq$, simply by incrementing $\lnot x$. Hence we see that not only 
is the addition $+$ a natural primitive of computer hardware, but so is
the subtraction $-$ defined by $x-y = x + (-y)$ for all $x,y\in\zq$.

\begin{defin}\label{carry:star} 
  For all $x\in\zq$, we define $x^{*}$ the unique integer with the property:
    \[
      x^{*} = x\ \mbox{mod}\ q\mbox{\ \ and\ \ }x^{*}\in[0,q[
    \]
\end{defin}

\section{Carry for Addition}
Users of computer hardware are not interested in \zq. One of their first interests
is to perform the operation of addition $+$ on the set of natural numbers \N, 
an operation commonly referred to as {\em unsigned addition} by computer 
scientists. While it is possible (and indeed is a built-in feature of many modern 
computer languages such as Python or Haskell) to handle every possible values 
of \N\ (as permitted by a computer's memory), for historical reasons and the 
purpose of this note on {\em carry} and {\em overflow}, it is important to 
restrict our attention to natural numbers which can be represented as 
{\em unsigned integers} within an $n$-bits register. These {\em representable}
natural numbers are exactly those belonging the interval $[0,q[$ and the
representation is exactly the associated integer modulo $q$. In other words,
the {\em representable} natural numbers are the range of the mapping 
$x\rightarrow x^{*}$ from \zq\ to \N, and for all $x\in\zq$, $x$ is the
representation (in computer hardware) of the natural number $x^{*}$. Now,
given two {\em representable} natural numbers $x^{*}$ and $y^{*}$, the users of 
computer hardware are interested in computing the sum $x^{*} + y^{*}$, while 
their machine only knows about $x+y$. Luckily, the following proposition shows that
computing $x+y$ allows us to infer the value of $x^{*}+y^{*}$, provided the 
latter is also {\em representable}:

\begin{prop}\label{carry:unsigned:add:morphism}
For all $x,y\in\zq$ the following are equivalent:
  \begin{eqnarray*}
    (i)&\ &x^{*} + y^{*}\in [0,q[\\
    (ii)&\ &(x + y)^{*} = x^{*} + y^{*} 
  \end{eqnarray*}
\end{prop}
\begin{proof}
  For all $x\in\zq$, from definition~(\ref{carry:star}) $x^{*}$ is an element 
  of $[0,q[$. It follows that $(i)$ is an immediate consequence of $(ii)$. 
  Conversely, if we assume that $(i)$ is true, then since it is clear that 
  $x^{*} + y^{*} = x + y$ modulo $q$, we conclude from 
  definition~(\ref{carry:star}) that $(x+y)^{*} = x^{*} + y^{*}$, 
  which completes our proof.
\end{proof}

Hence we see that as long as $x^{*}+y^{*}$ is a {\em representable} natural
number, it does not matter that our CPU should only know about addition in \zq:
adding the two representations $x$ and $y$ gives us a representation of 
$x^{*}+y^{*}$ which is as good as we can hope. However, there are cases 
when $x^{*} + y^{*}$ is not a {\em representable} natural number, in which 
case equality~$(ii)$ of proposition~(\ref{carry:unsigned:add:morphism}) 
does not hold. One way to think about this situation is saying that 
{\em the result of $x+y$ is wrong}. As it turns, there is nothing wrong 
with $x+y$ which is a perfectly correct answer to the question of adding 
two numbers in \zq. However, $x+y$ is not a representation of $x^{*}+y^{*}$, 
and in that sense, it is clearly wrong. This is where the notion of 
{\em carry for addition} naturally comes in:

\begin{defin}\label{carry:unsigned:add:carry}
We call {\em carry for addition} the map $c:\zq\times\zq\rightarrow2$ defined by:
  \[
    c(x,y) = 1  \ \ \Leftrightarrow\ \ x^{*} + y^{*} \not\in[0,q[
  \]
\end{defin}

By convention we denote $2=\{0,1\}$ which is our choice of boolean type, 
and the {\em carry for addition} is therefore a boolean function defined 
on the cartesian product $\zq\times\zq$, the purpose of which
is to flag any situation where the result of $x+y$ is {\em wrong} or (to
phrase it more accurately) any situation where $x+y$ is not a representation 
of the natural number $x^{*}+y^{*}$. 

So we now understand what the {\em carry} is, a notion which we have defined 
in the context of unsigned addition. As we shall see, there will be a notion 
of carry for (unsigned) subtraction, and similar notions for signed addition 
and subtraction (called {\em overflow} rather than {\em carry} for signed 
operations). What all these notions have in common is their purpose: 
to warn a user of computer hardware that the result of a primitive operation 
perfomed by the CPU on elements of \zq\ (which are representations of elements
of \N\ or \Z\ as the case may be), does not yield a representation of the result 
to the corresponding operation on natural numbers or signed integers. In short, 
{\em carry} and {\em overflow} are designed to tell us when the result of an
operation is {\em wrong}.

For those writing computer software, the ability to compute the carry flag 
$c(x,y)$ is crucial, as we need to know when $x+y$ ceases to be an accurate 
representation of the result we care about. For this reason, designers of
computer hardware have made the computation of the carry flag for addition, 
one of the fundamental primitives of a CPU. So whenever an addition $x+y$ is
perfomed, the carry flag will be set or cleared, depending on whether 
$c(x,y)=1$ or $c(x,y)=0$. The ability to test this carry flag and introduce
branching instructions in the code which depend on the outcome is also one
of the primitives of computer hardware. This in turn allows us to compare:


\begin{defin}\label{carry:order:zq}
  Given $x,y\in\zq$ we say that $x\leq y$ \ifand\ $x^{*}\leq y^{*}$.
\end{defin}

Mathematically speaking, the order $\leq$ thus defined on \zq\ may 
not be very interesting. However, it is interesting to us as it is 
a relation which can be tested by a CPU (As we shall see, it can
be tested using the {\em carry for subtraction}).
Note that having defined the relation $\leq$ on \zq, we have implicitely 
defined $<$, $>$ and $>=$, and since we can test these conditions, 
we are able to compute the corresponding binary $\min$ and $\max$ 
functions. So while definition~(\ref{carry:unsigned:add:carry}) 
relies on the quantity $x^{*}+y^{*}$ which we cannot compute,
the following proposition defines the {\em carry for addition} in
terms of conditions which can all be easily tested in assembly
language, thereby offering a possible validation.

\begin{prop}\label{carry:unsigned:add:criterium}
  For all $x,y\in\zq$, the following are equivalent:
    \begin{eqnarray*}
      (i)&\ &c(x,y) = 1\\
      (ii)&\ & x + y < \min(x,y)\\
      (iii)&\ & x + y < \max(x,y)\\
      (iv)&\ & x + y < x\\
      (v) &\ & x + y < y
    \end{eqnarray*}
where $c:\zq\times\zq\rightarrow 2$ is the carry for addition of 
  definition~(\ref{carry:unsigned:add:carry}).
\end{prop}
\begin{proof}
  Since for all $x,y\in\zq$, $\min(x,y)\leq x, y\leq\max(x,y)$ we immediately 
  have the implications $(ii)\Rightarrow(iv)$, $(ii)\Rightarrow(v)$, 
  $(iv)\Rightarrow(iii)$ and $(v)\Rightarrow(iii)$. In order to complete
  the proof, it remains to show that $(i)\Rightarrow(ii)$ and 
  $(iii)\Rightarrow(i)$. We first show that $(i)\Rightarrow(ii)$. So 
  we assume that $c(x,y) = 1$, and need to show that $x+y<\min(x,y)$.
  In other words we need to show that both inequalities $x+y<x$ and 
  $x+y<y$ hold. However from definition~(\ref{carry:unsigned:add:carry}),
  our assumption is equivalent to $x^{*}+y^{*}\not\in[0,q[$, and we know
  from definition~(\ref{carry:star}) that both $x^{*}$ and $y^{*}$ are elements
  of $[0,q[$. It follows that $x^{*}+y^{*}$ must be an element of $[0,2q[$
  without being an element of $[0,q[$. Hence it must be an element of
  $[q,2q[$, from which we see that $x^{*}+y^{*}-q$ is an element of $[0,q[$,
  while being equal to $x+y$ modulo $q$. From definition~(\ref{carry:star})
  it follows that $(x+y)^{*}=x^{*}+y^{*}-q$, and since both $y^{*}-q <0$ and 
  $x^{*}-q<0$ we obtain $(x+y)^{*}<x^{*}$ and $(x+y)^{*}<y^{*}$ which by
  virtue of definition~(\ref{carry:order:zq}) is equivalent to $(x+y)<x$ and 
  $(x+y)<y$ as requested. We now prove that $(iii)\Rightarrow(i)$. So we
  assume that $x+y <\max(x,y)$ and we need to show that $c(x,y)=1$, or
  equivalently that $x^{*}+y^{*}\not\in[0,q[$. However, our assumption
  implies that $x+y<x$ or $x+y<y$. So we shall distinguish two cases:
  first we assume that $x+y<x$. From definiton~(\ref{carry:order:zq}), 
  this means that $(x+y)^{*}<x^{*}$. Hence, it is impossible that 
  $x^{*}+y^{*}\in[0,q[$ as this would imply that $(x+y)^{*}=x^{*}+y^{*}$ 
  (being equal to $x+y$ modulo $q$) and consequently $x^{*}+y^{*}<x^{*}$, 
  yielding the contradiction $y^{*}<0$. Likewise, if we assume that $x+y<y$ 
  then $x^{*}+y^{*}\in[0,q[$ implies the contradiction $x^{*}<0$.
\end{proof}
\section{Carry for Subtraction}
Given two {\em representable} natural numbers $x^{*}$ and $y^{*}$, users of
computer hardware commonly want to compute their difference $x^{*}-y^{*}$. 
Unfortunately, the only result available to them is the difference $x-y\in\zq$
performed by the CPU on their representations $x$, $y$. In light of our
analysis for addition, the first question we should ask is whether $x-y$ 
is a representation of $x^{*}-y^{*}$. As before, as long as the sought out 
result is {\em representable}, the answer is 'yes':
\begin{prop}\label{carry:unsigned:sub:morphism}
  For all $x,y\in\zq$, the following are equivalent:
    \begin{eqnarray*}
      (i)&\ &x^{*}-y^{*}\in[0,q[\\
      (ii)&\ &x^{*}\geq y^{*}\\
      (iii)&\ &(x-y)^{*}=x^{*}-y^{*}
    \end{eqnarray*}
\end{prop}
\begin{proof}
Since both $x^{*}$ and $y^{*}$ are always elements of $[0,q[$, the difference
  $x^{*}-y^{*}$ is always an element of $]-q,q[$. Hence the equivalence between
  $(i)$ and $(ii)$ is clear. It remains to show the equivalence between $(ii)$
  and $(iii)$. First we show $(ii)\Rightarrow(iii)$. So suppose 
  $x^{*} \geq y^{*}$. Since $x^{*}-y^{*}$ is an element of $]-q,q[$, it is
  in fact an element of $[0,q[$ while being equal to $x-y$ modulo $q$. It
  follows from definition~(\ref{carry:star}) that $x^{*}-y^{*}=(x-y)^{*}$
  as requested. Conversely, if we assume that $x^{*}-y^{*}=(x-y)^{*}$ then
  in particular $x^{*}-y^{*}$is an element of $[0,q[$, so $x^{*}\geq y^{*}$.
\end{proof}

This in turn immediately motivates the following definition, whose purpose
it is to highlight any situation where the result of $x-y$ is {\em wrong}, that
is a situation where $x-y$ ceases to be a representation of the difference 
$x^{*}-y^{*}$.

\begin{defin}\label{carry:unsigned:sub:carry}
We call {\em carry for subtraction} the map $c:\zq\!\times\!\zq\rightarrow2$ defined by:
  \[
    c(x,y) = 1  \ \ \Leftrightarrow\ \ x^{*} < y^{*}
  \]
\end{defin}

In similar fashion to proposition~(\ref{carry:unsigned:add:criterium}) we state:

\begin{prop}\label{carry:unsigned:sub:criterium}
  For all $x,y\in\zq$, the following are equivalent:
    \begin{eqnarray*}
      (i)&\ &c(x,y) = 1\\
      (ii)&\ & x < y
    \end{eqnarray*}
where $c:\zq\times\zq\rightarrow 2$ is the carry for subtraction of 
  definition~(\ref{carry:unsigned:sub:carry}).
\end{prop}
\begin{proof}
Immediate from definition~(\ref{carry:unsigned:sub:carry}) and the fact 
that $x<y$ is equivalent $x^{*}<y^{*}$.
\end{proof}

The point of proposition~(\ref{carry:unsigned:sub:criterium}) is to be 
able to validate in software the consistency between testing for the 
carry flag after a subtraction $x-y$, and testing for the condition $x<y$.
Whenever a 'sub x y' instructtion is performed, the carry flag should be
set if and only if the condition $x<y$ holds.

\section{Overflow for Addition}
Users of computer hardware are not interested in \zq, and they may not be
interested in the set of natural numbers \N\ either. Given two integers 
$u,v\in\Z$ they often want to compute their sums $u+v$, an operation 
commonly referred to as {\em signed addition} by computer scientists.
Unfortunately, as we can guess by now, a CPU has no interstanding of
integers in \Z\ (it only knows about \zq) let alone of how to compute
their sum. We humans have to come up with a trick to obtain the answers
we want from a machine which is limited to operating on \zq. When faced
with operations on \N, we designed a successful strategy: first we 
restricted our domain to a subset of \N\ namely those natural numbers
belonging to $[0,q[$. Then we decided to give every natural number $x^{*}$
of this domain a representation $x$ in \zq. Then we let the CPU operate
on the representations $x$ and $y$ of our chosen natural numbers, 
hoping to obtain a representation of the sum $x^{*} + y^{*}$ or the 
difference $x^{*}-y^{*}$ as the case may be. This strategy works well,
but is not immune to failure, which led to the notions of {\em carry}
for addition and subtraction giving software designers the ability
to know if and when the CPU representations $x+y$ and $x-y$ in \zq\ 
are indeed what they want. We shall now spell out a similar strategy 
for \Z, that of {\em two's complement representation}:

\begin{defin}\label{carry:signed:representable}
We say that $n\in\Z$ is {\em representable} \ifand\ $n\in[-q/2,q/2[$.
\end{defin}
 
Definition~(\ref{carry:signed:representable}) describes our domain. We
have now defined the term {\em representable} in the context of signed 
addition (addition in \Z) which should not be confused with 
{\em representable} in the context of unsigned addition (addition in \N).
A {\em representable} natural number (aka unsigned integer) belongs to 
$[0,q[$, while a {\em representable} integer (aka signed integer) belongs
to $[-q/2,q/2[$. Now given a {\em representable} integer in $[-q/2,q/2[$,
we need to define its representation in \zq:

\begin{defin}\label{carry:signed:2:complement}
The bijection $r:\zq\rightarrow\Z\cap[-q/2,q/2[$ defined by:
  \[
    \forall x\in\zq\ ,\ r(x)=
      \left\{
          \begin{array}{lcr}
          x^{*}&\mbox{\ if\ }&x\in[0,q/2[\\
          x^{*}-q&\mbox{\ if\ }&x\in[q/2,q[
        \end{array}
      \right\}
  \]
is called {\em the two's complement representation mapping}. Given $x\in\zq$,
  we say that $x$ is the {\em two's complement representation} of the 
  integer $r(x)\in[-q/2,q/2[$.
\end{defin}

Now consider the simplest case when $q=2^8$. We have $256$ elements of \zq, 
from $0$ to $255$. According to definition~(\ref{carry:signed:2:complement}),
we have $r(0)=0$ and $r(1) = 1$ etc. up to $r(127)=127$. Then we see that 
$r(128)=-128$, $r(129)=-127$, etc. up to $r(255)=-1$. The two's complement
representation of $0\in\Z$ is $0\in\zq$. The two's complement representation
of $-128\in\Z$ is $128\in\zq$ and the two's complement representation of 
$-1\in\Z$ is $255\in\zq$.

\begin{prop}
The map $r:\zq\rightarrow\Z\cap[-q/2,q/2[$ is indeed a bijection.
\end{prop}
\begin{proof}
First we show that $r$ is injective. So suppose $r(x)=r(y)$ for some 
$x,y\in\zq$. We need to show that $x=y$ or equivalently that 
$x^{*}=y^{*}$. We consider $4$ cases:
  \begin{itemize}
    \item $x\in[0,q/2[$ and $y\in[0,q/2[$: then $r(x)=x^{*}$ and $r(y)=y^{*}$
      and the equality $r(x)=r(y)$ yield $x^{*}=y^{*}$ as requested.
    \item $x\in[0,q/2[$ and $y\in[q/2,q[$: then $r(x)=x^{*}$ and 
      $r(y)=y^{*}-q$ and the equality $r(x) = r(y)$ yields $x^{*}=y^{*}-q < 0$
      which is a contradiction. Hence this case cannot occur whenever 
      $r(x)=r(y)$.
    \item $x\in[q/2,q[$ and $y\in[0,q/2[$: a similar argument shows that this
      case cannot occur whenever $r(x)=r(y)$.
    \item $x\in[q/2,q[$ and $y\in[q/2,q[$: then $r(x)=x^{*}-q$ and 
      $r(y)=y^{*}-q$ and the equality $r(x)=r(y)$ yields $x^{*}=y^{*}$ as 
      requested.
  \end{itemize}
We now show that $r$ is surjective. So let $n$ be an integer in $[-q/2,q/2[$.
  We need to show the existence of $x\in\zq$ such that $n=r(x)$. We consider
  $2$ cases:
  \begin{itemize}
    \item $n\in[-q/2,0[$: then $n+q\in[q/2,q[$ and defining $x\in\zq$ to be
      the class of $n$ modulo
  \end{itemize}
 
\end{proof}





\end{document}
