\begin{defin}\label{Fun:def:hom:functor}
    Let \Cat\ be a locally small category. We call {\em hom-functor associated
    with} \Cat\ the functor $F:\Cop\times\Cat\to\Set$ defined by $F=(F_{0},F_{1})$
    with:
        \begin{eqnarray*}
            (1)&\ &F_{0}(a_{1},a_{2}) = \Cat(a_{1},a_{2})\\
            (2)&\ &F_{1}(f_{1},f_{2})(h) = f_{2}\circ h \circ f_{1}
        \end{eqnarray*}
    where $(2)$ hold for $a_{1},a_{2},b_{1},b_{2}\in\Cat$,
    $f_{1}:b_{1}\to a_{1}$, $h:a_{1}\to a_{2}$ and $f_{2}:a_{2}\to b_{2}$.
\end{defin}
\begin{notation}\label{Fun:notation:hom:functor}
    Given a locally small category \Cat, the hom-functor $F=(F_{0},F_{1})$ 
    associated with \Cat\ is denoted $\Cat=(\Cat_{0},\Cat_{1})$.
\end{notation}
\noindent
{\bf Remark}: Using the notation \Cat\ to denote both the locally small category
\Cat\ and its associated hom-functor may appear confusing, but the notation
actally makes sense since the equation $F_{0}(a_{1},a_{2})=
\Cat(a_{1},a_{2})$ simply becomes the tautology $\Cat(a_{1},a_{2})
=\Cat(a_{1},a_{2})$. In other words, using \Cat\ to denote the hom-functor 
makes it very easy to remember that when applied to the object $(a_{1},a_{2})$
of the category $\Cop\times\Cat$, we simply obtain the hom-set 
$\Cat(a_{1},a_{2})$ of the locally small category \Cat.

Given a locally small category \Cat, definition~(\ref{Fun:def:hom:functor}) 
defines a tuple $F=(F_{0},F_{1})$ where $F_{0}$ appears to be a map defined 
on $\ob\ \Cat\times\ob\ \Cat$ with values in \Set, and $F_{1}$ appears to be 
a map defined on $\arr\ \Cat\times\arr\ \Cat$ with values in some functional 
space (since it takes an $h$ as argument). Looking at this, it is far from 
obvious that definition~(\ref{Fun:def:hom:functor}) defines a functor 
$F:\Cop\times\Cat\to\Set$. Hence we state:
\begin{prop}\label{Fun:prop:hom:functor}
    Let \Cat\ be a locally small category. Then the hom-functor $F$ associated 
    with \Cat\ is indeed a functor $F:\Cop\times\Cat\to\Set$.
\end{prop}
\begin{proof}
    We need to check that properties~$(1)-(5)$ of 
    definition~(\ref{Fun:def:functor}) are satisfied:

    $(1)$: We need to show that $F_{0}$ is a map $F_{0}:\ob\ (\Cop\times\Cat)
    \to\ob\ \Set$. Having defined $F_{0}(a_{1},a_{2})=\Cat(a_{1},a_{2})$ and
    the category \Cat\ being locally small, we see that $F_{0}(a_{1},a_{2})$
    is a set for all $a_{1},a_{2}\in\Cat$. So $F_{0}$ is defined as a map
    $F_{0}:(\ob\ \Cat)\times(\ob\ \Cat)\to\ob\ \Set$, and it remains to check
    that the collections $(\ob\ \Cat)\times(\ob\ \Cat)$ and $\ob\ (\Cop\times
    \Cat)$ actually coincide, which goes as follows:
        \begin{eqnarray*}(\ob\ \Cat)\times(\ob\ \Cat)
            &=&\{\ (a_{1},a_{2})\ |\ a_{1}\in\ob\ \Cat\ ,\ a_{2}\in\ob\ \Cat\}\\
            \mbox{def.~(\ref{Cat:def:opposite})\ $\to$\ }
            &=&\{\ (a_{1},a_{2})\ |\ a_{1}\in\ob\ \Cop\ ,\ a_{2}\in\ob\ \Cat\}\\
            \mbox{$(1)$ of def.~(\ref{Cat:def:canonical:product})\ $\to$\ }
            &=&\ob\ (\Cop\times\Cat)
        \end{eqnarray*}

    $(2)$: We need to show that $F_{1}$ is a map $F_{1}:\arr\ (\Cop\times\Cat)
    \to\arr\ \Set$. Having defined $F_{1}(f_{1},f_{2})$ for any $f_{1}:b_{1}\to
    a_{1}$ and $f_{2}:a_{2}\to b_{2}$ where $a_{1},a_{2},b_{1},b_{2}$ are 
    arbitrary objects in \Cat, we see that $F_{1}$ is a map defined 
    on $(\arr\ \Cat)\times(\arr\ \Cat)$. So we need to check that the collections
    $(\arr\ \Cat)\times(\arr\ \Cat)$ and $\arr\ (\Cop\times\Cat)$ actually
    coincide, which goes as follows:
        \begin{eqnarray*}(\arr\ \Cat)\times(\arr\ \Cat)
            &=&\{\ (f_{1},f_{2})\ |\ f_{1}\in\arr\ \Cat\ ,\ f_{2}\in\arr\ \Cat\}\\
            \mbox{def.~(\ref{Cat:def:opposite})\ $\to$\ }
            &=&\{\ (f_{1},f_{2})\ |\ f_{1}\in\arr\ \Cop\ ,\ f_{2}\in\arr\ \Cat\}\\
            \mbox{$(2)$ of def.~(\ref{Cat:def:canonical:product})\ $\to$\ }
            &=&\arr\ (\Cop\times\Cat)
        \end{eqnarray*}
    However, given arrows $f_{1}, f_{2}$ in \Cat, we still need to check that
    $F_{1}(f_{1},f_{2})$ is a member of the collection $\arr\ \Set$. In other
    words, we need to check that $F_{1}(f_{1},f_{2})$ is a function. Introducing
    the notations $a_{1},a_{2},b_{1},b_{2}$ such that $f_{1}:b_{1}\to a_{1}$ and
    $f_{2}:a_{2}\to b_{2}$, our definition states that $F_{1}(f_{1},f_{2})(h)$ is
    defined for any $h:a_{1}\to a_{2}$. In other words, $F_{1}(f_{1},f_{2})(h)$
    is defined for any $h$ which belongs to the hom-set $\Cat(a_{1},a_{2})$.
    Having assumed that \Cat\ is a locally small category, the collection
    $\Cat(a_{1},a_{2})$ is in fact a set, and $F_{1}(f_{1},f_{2})$ is
    therefore a function with domain $\Cat(a_{1},a_{2})$, defined by
    $F_{1}(f_{1},f_{2})(h)=f_{2}\circ h\circ f_{1}$. Note that since
    $f_{1}:b_{1}\to a_{1}$ and $f_{2}:a_{2}\to b_{2}$, the composition
    $f_{2}\circ h\circ f_{1}$ is a well-defined arrow in \Cat\ whenever 
    $h\in\Cat(a_{1},a_{2})$. This arrow has domain $b_{1}$ and codomain 
    $b_{2}$ and we see that $F_{1}(f_{1},f_{2})(h)$ is in fact an element
    of the hom-set $\Cat(b_{1},b_{2})$. So $F_{1}(f_{1},f_{2})$ is actually
    a function $F_{1}(f_{1},f_{2}):\Cat(a_{1},a_{1})\to\Cat(b_{1},b_{2})$.
    Now looking at definition~(\ref{Cat:def:set}), an arrow of the category
    \Set\ is a triple $(a,b,f)$ where $a$ and $b$ are sets and $f$ is a
    function $f:a\to b$. Hence, strictly speaking our definition of 
    $F_{1}(f_{1},f_{2})$ is not a member of $\arr\ \Set$ but simply a 
    function $F_{1}(f_{1},f_{2}): \Cat(a_{1},a_{2})\to\Cat(b_{1},b_{2})$. 
    However, the triple 
    $(\Cat(a_{1},a_{2}),\Cat(b_{1},b_{2}),F_{1}(f_{1},f_{2}))$ is an arrow
    of the category \Set\ and we have agreed in 
    notation~(\ref{Cat:notation:set:arrow}) to simply refer to this arrow as
    $F_{1}(f_{1},f_{2})$, as the domain $\Cat(a_{1},a_{2})$ and intended
    codomain $\Cat(b_{1},b_{2})$ can easily be inferred from the formula
    $F_{1}(f_{1},f_{2})(h) = f_{2}\circ h\circ f_{1}$ for all $h:a_{1}\to 
    a_{2}$. This completes our proof that $F_{1}$ is a map $F_{1}:
    \arr\ (\Cop\times\Cat)\to\arr\ \Set$.

    $(3)$: We need to check that $F_{1}(f):F_{0}(a)\to F_{0}(b)$ whenever
    $a,b\in\Cop\times\Cat$ and $f:a\to b$. So let $a=(a_{1},a_{2})\in\Cop
    \times\Cat$, $b=(b_{1},b_{2})\in\Cop\times\Cat$ and let $f$ be an 
    arrow $f=(f_{1},f_{2}):(a_{1},a_{2})\to(b_{1},b_{2})\ @\ \Cop\times\Cat$. 
    From the equalities $F_{0}(a)=\Cat(a_{1},a_{2})$ and $F_{0}(b)=\Cat
    (b_{1},b_{2})$, it is clear that we simply need to check 
    $F_{1}(f_{1},f_{2}):\Cat(a_{1},a_{2})\to \Cat(b_{1},b_{2})$.
    However, we have already established this fact in part $(2)$ of this 
    proof, provided we show that $f_{1}:b_{1}\to a_{1}\ @\ \Cat$ together 
    with $f_{2}:a_{2}\to b_{2}\ @\ \Cat$. Hence we need
    $(f_{1},f_{2})\in\Cat(b_{1},a_{1})\times\Cat(a_{2},b_{2})$, knowing
    that $(f_{1},f_{2})\in\Cop\times\Cat[\,(a_{1},a_{2})\,,\,(b_{1},b_{2})
    \,]$ by assumption. It is therefore sufficient to prove that the 
    two collections $\Cat(b_{1},a_{1})\times\Cat(a_{2},b_{2})$ and
    $\Cop\times\Cat[\,(a_{1},a_{2})\,,\,(b_{1},b_{2})\,]$ coincide, which 
    goes as follows:
        \begin{eqnarray*}\Cop\times\Cat[\,(a_{1},a_{2})\,,\,(b_{1},b_{2})\,]
            &=&\Cop(a_{1},b_{1})\times\Cat(a_{2},b_{2})\ \leftarrow\ 
            \mbox{prop.~(\ref{Cat:prop:homset:product})}\\
            \mbox{prop.~(\ref{Cat:prop:homset:opposite})}\ \to\ 
            &=&\Cat(b_{1},a_{1})\times\Cat(a_{2},b_{2})
        \end{eqnarray*}

    $(4)$: We need to check that $F_{1}(\,\id(a)\,)=\id(\,F_{0}(a)\,)$ whenever
    $a\in\Cop\times\Cat$. So let $a=(a_{1},a_{2})\in\Cop\times\Cat$. Since
    $F_{0}(a)=\Cat(a_{1},a_{2})$, we need to check that $F_{1}(\,\id(a)\,)
    =\id(\,\Cat(a_{1},a_{2})\,)$. This is an equality between two arrows
    of the category \Set, with identical domain and codomain, namely the set 
    $\Cat(a_{1},a_{2})$. Given $h\in\Cat(a_{1},a_{2})$, 
    from proposition~(\ref{Cat:prop:set:arrow:equal})
    it is sufficient to 
    check that $F_{1}(\,\id(a)\,)(h)=h$:
        \begin{eqnarray*}F_{1}(\,\id(a)\,)(h)
            &=&F_{1}(\,\id\,(a_{1},a_{2})\,)(h)\\
            \mbox{$(5)$ of def.~(\ref{Cat:def:canonical:product})}\ \to\ 
            &=&F_{1}(\,\id(a_{1})\ @\ \Cop\,,\,\id(a_{2})\,)(h)\\
            \mbox{def.~(\ref{Cat:def:opposite})}\ \to\ 
            &=&F_{1}(\,\id(a_{1})\ @\ \Cat\,,\,\id(a_{2})\,)(h)\\
            \mbox{$(2)$ of def.~(\ref{Fun:def:hom:functor})}\ \to\ 
            &=&\id(a_{2})\circ h\circ \id(a_{1})\\
            \mbox{$(12)$ of def.~(\ref{Cat:def:category})}\ \to\ 
            &=&\id(a_{2})\circ h\\
            \mbox{$(13)$ of def.~(\ref{Cat:def:category})}\ \to\ 
            &=&h
        \end{eqnarray*}

    $(5)$: We need to check that $F_{1}(g\circ f)=F_{1}(g)\circ F_{1}(f)$ 
    whenever $f:a\to b$, $g:b\to c$ and $a,b,c\in\Cop\times\Cat$. So let
    $a=(a_{1},a_{2})$, $b=(b_{1},b_{2})$, $c=(c_{1},c_{2})$ be objects
    in $\Cop\times\Cat$, and $f=(f_{1},f_{1}):a\to b$ and $g=(g_{1},g_{2}):
    b\to c$. We have $g\circ f:a \to c$ and consequently $F_{1}(g\circ f):
    F_{0}(a)\to F_{0}(c)$. Hence the arrows $F_{1}(g\circ f)$ and
    $F_{1}(g)\circ F_{1}(f)$ are two arrows in \Set, with domain
    $F_{0}(a)=\Cat(a_{1},a_{2})$ and codomain $F_{0}(c)=\Cat(c_{1},c_{2})$.
    From proposition~(\ref{Cat:prop:set:arrow:equal}), in order to prove the 
    equality $F_{1}(g\circ f)=F_{1}(g)\circ F_{1}(f)$ it is therefore sufficient 
    to show that the underlying functions coincide for all $h\in\Cat(a_{1},a_{2})$
    which goes as follows:
        \begin{eqnarray*}F_{1}(g\circ f)(h)
            &=&F_{1}(\,(g_{1},g_{2})\,\circ\,(f_{1},f_{2})\,)(h)\\
            \mbox{$(6)$ of def.~(\ref{Cat:def:canonical:product})}\ \to\ 
            &=&F_{1}(\,g_{1} \circ f_{1}\ @\ \Cop,\,g_{2}\circ f_{2}\,)\,(h)\\
            \mbox{crucially, def.~(\ref{Cat:def:opposite})}\ \to\ 
            &=&F_{1}(\,f_{1} \circ g_{1}\,,\,g_{2}\circ f_{2}\,)\,(h)\\
            \mbox{$(2)$ of def.~(\ref{Fun:def:hom:functor})}\ \to\ 
            &=&(g_{2}\circ f_{2})\,\circ\, h\,\circ (f_{1}\circ g_{1})\\
            \mbox{associativity of $\circ$ in \Cat}\ \to\ 
            &=&g_{2}\,\circ\,(f_{2}\circ h\circ f_{1})\,\circ g_{1}\\
            \mbox{$(2)$ of def.~(\ref{Fun:def:hom:functor})}\ \to\ 
            &=&g_{2}\,\circ\,F_{1}(f_{1},f_{2})(h)\,\circ g_{1}\\
            \mbox{$(2)$ of def.~(\ref{Fun:def:hom:functor})}\ \to\ 
            &=&F_{1}(g_{1},g_{2})(\,F_{1}(f_{1},f_{2})(h)\,)\\
            &=&F_{1}(g)(\,F_{1}(f)(h)\,)\\
            &=&(F_{1}(g)\circ F_{1}(f))(h)
        \end{eqnarray*}
\end{proof}
