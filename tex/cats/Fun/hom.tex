\begin{defin}\label{Fun:def:hom:functor}
    Let \Cat\ be a locally small category. We call {\em hom-functor associated
    with} \Cat\ the functor $F:\Cop\times\Cat\to\Set$ defined by $F=(F_{0},F_{1})$
    with:
        \begin{eqnarray*}
            (1)&\ &F_{0}(a_{1},a_{2}) = \Cat(a_{1},a_{2})\\
            (2)&\ &F_{1}(f_{1},f_{2})(h) = f_{2}\circ h \circ f_{1}
        \end{eqnarray*}
    where $(2)$ hold for $a_{1},a_{2},b_{1},b_{2}\in\Cat$,
    $f_{1}:b_{1}\to a_{1}$, $h:a_{1}\to a_{2}$ and $f_{2}:a_{2}\to b_{2}$.
\end{defin}
\begin{notation}\label{Fun:notation:hom:functor}
    Given a locally small category \Cat, the hom-functor $F=(F_{0},F_{1})$ 
    associated with \Cat\ is denoted $\Cat=(\Cat_{0},\Cat_{1})$.
\end{notation}
\noindent
{\bf Remark}: Using the notation \Cat\ to denote both the small category
\Cat\ and its associated hom-functor may appear confusing, but the notation
actally makes sense since the equation $F_{0}(a_{1},a_{2})=
\Cat(a_{1},a_{2})$ simply becomes the tautology $\Cat(a_{1},a_{2})
=\Cat(a_{1},a_{2})$. In other words, using \Cat\ to denote the hom-functor 
makes it very easy to remember that when applied to the object $(a_{1},a_{2})$
of the category $\Cop\times\Cat$, we simply obtain the hom-set 
$\Cat(a_{1},a_{2})$ of the small category \Cat.

Given a locally small category \Cat, definition~(\ref{Fun:def:hom:functor}) 
defines a tuple $F=(F_{0},F_{1})$ where $F_{0}$ appears to be a map defined 
on $\ob\ \Cat\times\ob\ \Cat$ with values in \Set, and $F_{1}$ appears to be 
a map defined on $\arr\ \Cat\times\arr\ \Cat$ with values in some functional 
space (since it takes an $h$ as argument). Looking at this, it is far from 
obvious that definition~(\ref{Fun:def:hom:functor}) defines a functor 
$F:\Cop\times\Cat\to\Set$. Hence we state:
\begin{prop}\label{Fun:prop:hom:functor}
    Let \Cat\ be a locally-small category. Then the hom-functor $F$ associated 
    with \Cat\ is indeed a functor $F:\Cop\times\Cat\to\Set$.
\end{prop}
\begin{proof}
    We need to check that properties~$(1)-(5)$ of 
    definition~(\ref{Fun:def:functor}) are satisfied:

    $(1)$: We need to show that $F_{0}$ is a map $F_{0}:\ob\ (\Cop\times\Cat)
    \to\ob\ \Set$. Having defined $F_{0}(a_{1},a_{2})=\Cat(a_{1},a_{2})$ and
    the category \Cat\ being locally small, we see that $F_{0}(a_{1},a_{2})$
    is a set for all $a_{1},a_{2}\in\Cat$. So $F_{0}$ is defined as a map
    $F_{0}:(\ob\ \Cat)\times(\ob\ \Cat)\to\ob\ \Set$, and it remains to check
    that the collections $(\ob\ \Cat)\times(\ob\ \Cat)$ and $\ob\ (\Cop\times
    \Cat)$ actually coincide, which goes as follows:
        \begin{eqnarray*}(\ob\ \Cat)\times(\ob\ \Cat)
            &=&\{\ (a_{1},a_{2})\ |\ a_{1}\in\ob\ \Cat\ ,\ a_{2}\in\ob\ \Cat\}\\
            \mbox{def.~(\ref{Cat:def:opposite})\ $\to$\ }
            &=&\{\ (a_{1},a_{2})\ |\ a_{1}\in\ob\ \Cop\ ,\ a_{2}\in\ob\ \Cat\}\\
            \mbox{def.~(\ref{Cat:def:canonical:product})\ $\to$\ }
            &=&\ob\ (\Cop\times\Cat)
        \end{eqnarray*}

    $(2)$: TODO
\end{proof}
