\begin{prop}\label{Fun:prop:opposite}
    Let $F:\Cat\to\Dat$ be a functor where \Cat\ and \Dat\ are categories. Then
    $F$ is also a functor $F:\Cop\to\Dop$.
\end{prop}
\begin{proof}
    We need to check that properties $(1)-(5)$ of 
    definition~(\ref{Fun:def:functor}) are satisfied:

    $(1)$: $F_{0}$ is indeed a map $F_{0}:\ob\ \Cop\to\ob\ \Dop$ since $F$ is
    a functor and $F_{0}$ is therefore a map $F_{0}:\ob\ \Cat\to\ob\ \Dat$. 
    Furthermore, we have $\ob\ \Cop=\ob\ \Cat$ and $\ob\ \Dop=\ob\ \Dat$ by 
    virtue of definition~(\ref{Cat:def:opposite}).

    $(2)$: $F_{1}$ is indeed a map $F_{1}:\arr\ \Cop\to\arr\ \Dop$ since
    $\arr\ \Cop=\arr\ \Cat$ and $\arr\ \Dop=\arr\ \Dat$ from
    definition~(\ref{Cat:def:opposite}), and $F_{1}$ is a map
    $F_{1}:\arr\ \Cat\to\arr\ \Dat$.

    $(3)$: We have $F_{1}(f):F_{0}(a)\to F_{0}(b)\ @\ \Dop$ whenever
    $f:a\to b\ @\ \Cop$. Indeed, the assumption $f:a\to b\ @\ \Cop$ is equivalent
    to $f:b\to a\ @\ \Cat$. $F$ being a functor, this implies
    $F_{1}(f):F_{0}(b)\to F_{0}(a)\ @\ \Dat$ which is equivalent to 
    $F_{1}(f):F_{0}(a)\to F_{0}(b)\ @\ \Dop$.

    $(4)$: We have $F_{1}(\,\id(a)\ @\ \Cop)=\id(\,F_{0}(a)\,)\ @\ \Dop$ for
    all $a\in\Cop$. This follows from the equality $F_{1}(\,\id(a)\ @\ \Cat)
    =\id(\,F_{0}(a)\,)\ @\ \Dat$ and the fact that the notions of objects and
    identity coincide on a category and its opposite.

    $(5)$: We have $F_{1}(g\,\circ\,f\ @\ \Cop)=F_{1}(g)\circ F_{1}(f)\ @\ \Dop$
    when $f:a\to b\ @\ \Cop$ and $g:b\to c\ @\ \Cop$. Indeed, the assumptions
    $f:a\to b\ @\ \Cop$ and $g:b\to c\ @\ \Cop$ are equivalent to 
    $f:b\to a\ @\ \Cat$ and $g:c\to b\ @\ \Cat$. Hence we have:
        \begin{eqnarray*}F_{1}(g\circ f\ @\ \Cop)
            &=&F_{1}(f\circ g\ @\ \Cat)\ \leftarrow\ 
            \mbox{def.~(\ref{Cat:def:opposite})}\\
            \mbox{$(5)$ of def.~(\ref{Fun:def:functor})}\ \to\ 
            &=&F_{1}(f)\circ F_{1}(g)\ @\ \Dat\\
            \mbox{def.~(\ref{Cat:def:opposite})}\ \to\ 
            &=&F_{1}(g)\circ F_{1}(f)\ @\ \Dop
        \end{eqnarray*}
\end{proof}

Proposition~(\ref{Fun:prop:opposite}) illustrates the fact that the knowledge
of a functor $F$ by itself does not tell us what its domain and codomain are.
In fact, we have not even defined what the {\em domain} or {\em codomain} of 
a functor should be. By virtue of notation~(\ref{Fun:notation:functor:arrow}), 
writing $F:\Cat\to\Dat$ is simply a notational shortcut for the statement that 
$F$ is a functor between \Cat\ and \Dat. This is very similar to writing
$f:a \to b$ for an untyped function $f$. Now we cannot simply define 
the domain and codomain of a functor $F$ to be the categories \Cat\ and
\Dat\ respectively, whenever $F:\Cat\to\Dat$. This is because the categories
\Cat\ and \Dat\ are not unique, as whenever the statement $F:\Cat\to\Dat$
holds, the statement $F:\Cop\to\Dop$ also holds. To resolve this issue,
similarly to~(\ref{Cat:def:typed:untyped:function}), we define:

\begin{defin}\label{Fun:def:typed:untyped:functor}
    Given a functor $F:\Cat\to\Dat$ between categories \Cat, \Dat,
    we say that $F$ is the {\em untyped functor} while the triple 
    $(\Cat,\Dat,F)$ is called the {\em typed} functor.
\end{defin}
