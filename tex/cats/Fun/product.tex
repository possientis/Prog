\begin{defin}\label{Fun:def:canonical:product}
    We call {\em canonical product} of two functors $F_{1}:\Cone\to\Done$ and
    $F_{2}:\Ctwo\to\Dtwo$ the functor $G:\Cone\times\Ctwo\to\Done\times\Dtwo$
    defined by $G=(G_{0},G_{1})$ with:
        \begin{eqnarray*}
            (1)&\ &G_{0}(a_{1},a_{2}) = (\,F_{1}(a_{1})\,,\,F_{2}(a_{2})\,)\\
            (2)&\ &G_{1}(f_{1},f_{2}) = (\,F_{1}(f_{1})\,,\,F_{2}(f_{2})\,)
        \end{eqnarray*}
    where $\Cone,\Ctwo, \Done, \Dtwo$ are arbitrary categories, $(1)$ holds 
    for all $a_{1}\in\ob\ \Cone$ and $a_{2}\in\ob\ \Ctwo$, and $(2)$ holds 
    for all $f_{1}\in\arr\ \Cone$ and $f_{2}\in\arr\ \Ctwo$.
\end{defin}
\noindent
{\bf Remark}: In accordance with notation~(\ref{Fun:notation:functor:F0:F1}),
we are using the same notations $F_{1}$ and $F_{2}$ in 
definition~(\ref{Fun:def:canonical:product}) to describe the actions of these
functors both on objects and on arrows. The alternative would be to use the
notations $(F_{1})_{0}$, $(F_{1})_{1}$, $(F_{2})_{0}$ and $(F_{2})_{1}$ which 
would arguably be harder to read.

\begin{notation}\label{Fun:notation:canonical:product}
    The canonical product of functors $F_{1}$ and $F_{2}$ is denoted
    $F_{1}\times F_{2}$.
\end{notation}

\begin{prop}
    Let $F_{1}:\Cone\to\Done$ and $F_{2}:\Ctwo\to\Dtwo$ be two functors. Then 
    the product of $F_{1}$ and $F_{2}$ is indeed a functor $F_{1}\times F_{2}: 
    \Cone\times\Ctwo\to\Done\times\Dtwo$.
\end{prop}
\begin{proof}
    Let $G=(G_{0},G_{1})$ denote the product functor $F_{1}\times F_{2}$.
    We need to check that properties~$(1)-(5)$ of 
    definition~(\ref{Fun:def:functor}) are satisfied, which goes as follows:

    $(1)$: $G_{0}$ is indeed a map $G_{0}:\ob\ (\Cone\times\Ctwo)\to\ob\ (\Done
    \times\Dtwo)$: firstly, $G_{0}$ is defined on the 
    collection of all $(a_{1},a_{2})$ where $a_{1}\in\ob\ \Cone$
    and $a_{2}\in\ob\ \Ctwo$. According to 
    definition~(\ref{Cat:def:canonical:product}), this is precisely 
    the collection $\ob\ (\Cone\times\Ctwo)$. Furthermore $G_{0}(a_{1},a_{2})$
    is defined as $(\,F_{1}(a_{1})\,,\,F_{2}(a_{2})\,)$ and since 
    $F_{1}:\Cone\to\Done$ while $F_{2}:\Ctwo\to\Dtwo$, we have $F_{1}(a_{1})
    \in\ob\ \Done$ together with $F_{2}(a_{2})\in\ob\ \Dtwo$. Hence we see that
    $G_{0}(a_{1},a_{2})$ is indeed a member of the collection 
    $\ob\ (\Done\times\Dtwo)$.

    $(2)$: $G_{1}$ is indeed a map $G_{1}:\arr\ (\Cone\times\Ctwo)\to\arr\ (\Done
    \times\Dtwo)$: $G_{1}$ is defined on the 
    collection of all $(f_{1},f_{2})$ where $f_{1}\in\arr\ \Cone$
    and $f_{2}\in\arr\ \Ctwo$. According to 
    definition~(\ref{Cat:def:canonical:product}), this is precisely 
    the collection $\arr\ (\Cone\times\Ctwo)$. Furthermore $G_{1}(f_{1},f_{2})$
    is defined as $(\,F_{1}(f_{1})\,,\,F_{2}(f_{2})\,)$ and since 
    $F_{1}:\Cone\to\Done$ while we have $F_{2}:\Ctwo\to\Dtwo$, we see that
    $F_{1}(f_{1})\in\arr\ \Done$ and $F_{2}(f_{2})\in\arr\ \Dtwo$. Hence we 
    conclude that $G_{1}(f_{1},f_{2})$ is indeed a member of the collection 
    $\arr\ (\Done\times\Dtwo)$.

    $(3)$: We need to show that $G_{1}(f):G_{0}(a)\to G_{0}(b)$ whenever 
    $f:a\to b$: let $f\in\arr\ (\Cone\times\Ctwo)$ such that $\dom(f)=a$ and 
    $\cod(f)=b$. Then $f=(f_{1},f_{2})$ for some $f_{1}\in\arr\ \Cone$ and 
    $f_{2}\in\arr\ \Ctwo$. Furthermore, since $a,b\in\ob\ (\Cone\times\Ctwo)$, 
    we have $a=(a_{1},a_{2})$ and $b=(b_{1},b_{2})$ for some $a_{1},b_{1}\in\ob\ 
    \Cone$ and $a_{2},b_{2}\in\ob\ \Ctwo$:
        \begin{eqnarray*}(a_{1},a_{2})
            &=&a\\
            &=&\dom\,(f)\\
            &=&\dom\,(f_{1},f_{2})\\
            \mbox{$(3)$ of def.~(\ref{Cat:def:canonical:product})}\ \to\ 
            &=&(\,\dom\,(f_{1})\,,\,\dom\,(f_{2})\,)\\
        \end{eqnarray*}
    Hence we have $a_{1}=\dom(f_{1})$ and $a_{2}=\dom(f_{2})$ and similarly:
        \begin{eqnarray*}(b_{1},b_{2})
            &=&b\\
            &=&\cod\,(f)\\
            &=&\cod\,(f_{1},f_{2})\\
            \mbox{$(4)$ of def.~(\ref{Cat:def:canonical:product})}\ \to\ 
            &=&(\,\cod\,(f_{1})\,,\,\cod\,(f_{2})\,)\\
        \end{eqnarray*}
    from which we conclude that $b_{1}=\cod(f_{1})$ and $b_{2}=\cod(f_{2})$. 
    Hence we see that $f_{1}:a_{1}\to b_{1}\ @\ \Cone$ and $f_{2}:a_{2}\to 
    b_{2}\ @\ \Ctwo$. Since $F_{1}$ and $F_{2}$ are functors, using $(3)$ of
    definition~(\ref{Fun:def:functor}) we obtain $F_{1}(f_{1}): F_{1}(a_{1})
    \to F_{1}(b_{1})\ @\ \Done$ and likewise $F_{2}(f_{2}): F_{2}(a_{2})
    \to F_{2}(b_{2})\ @\ \Dtwo$. In order to show that $G_{1}(f):G_{0}(a)\to 
    G_{0}(b)$, since we already know that $G_{1}(f)\in\arr(
    \Done\times\Dtwo)$, it remains to show that $\dom\,(G_{1}(f))=G_{0}(a)$
    and $\cod\,(G_{1}(f))=G_{0}(b)$, which goes as follows:
        \begin{eqnarray*}\dom\,(G_{1}(f))
            &=& \dom\,(\,G_{1}(f_{1},f_{2})\,)\\
            \mbox{$(2)$ of def.~(\ref{Fun:def:canonical:product})}\ \to\ 
            &=&\dom\,(\,F_{1}(f_{1})\,,\,F_{2}(f_{2}))\\
            \mbox{$(3)$ of def.~(\ref{Cat:def:canonical:product})}\ \to\ 
            &=&(\,\dom\,(\,F_{1}(f_{1})\,)\,,\,\dom\,(\,F_{2}(f_{2})\,)\,)\\
            F_{1}(f_{1}):F_{1}(a_{1})\to F_{1}(b_{1})\ \to\ 
            &=&(\,F_{1}(a_{1})\,,\,\dom\,(\,F_{2}(f_{2})\,)\,)\\
            F_{2}(f_{2}):F_{2}(a_{2})\to F_{2}(b_{2})\ \to\ 
            &=&(\,F_{1}(a_{1})\,,\,F_{2}(a_{2})\,)\\
            \mbox{$(1)$ of def.~(\ref{Fun:def:canonical:product})}\ \to\ 
            &=&G_{0}(a_{1},a_{2})\\
            &=&G_{0}(a)
        \end{eqnarray*}
        \begin{eqnarray*}\cod\,(G_{1}(f))
            &=& \cod\,(\,G_{1}(f_{1},f_{2})\,)\\
            \mbox{$(2)$ of def.~(\ref{Fun:def:canonical:product})}\ \to\ 
            &=&\cod\,(\,F_{1}(f_{1})\,,\,F_{2}(f_{2}))\\
            \mbox{$(4)$ of def.~(\ref{Cat:def:canonical:product})}\ \to\ 
            &=&(\,\cod\,(\,F_{1}(f_{1})\,)\,,\,\cod\,(\,F_{2}(f_{2})\,)\,)\\
            F_{1}(f_{1}):F_{1}(a_{1})\to F_{1}(b_{1})\ \to\ 
            &=&(\,F_{1}(b_{1})\,,\,\cod\,(\,F_{2}(f_{2})\,)\,)\\
            F_{2}(f_{2}):F_{2}(a_{2})\to F_{2}(b_{2})\ \to\ 
            &=&(\,F_{1}(b_{1})\,,\,F_{2}(b_{2})\,)\\
            \mbox{$(1)$ of def.~(\ref{Fun:def:canonical:product})}\ \to\ 
            &=&G_{0}(b_{1},b_{2})\\
            &=&G_{0}(b)
        \end{eqnarray*}

    $(4)$: We have $G_{1}(\,\id(a)\,)=\id(\,G_{0}(a)\,)$ for all $a=(a_{1},a_{2})
    \in\Cone\times\Ctwo$:
        \begin{eqnarray*}G_{1}(\,\id(a)\,)
            &=&G_{1}(\,\id(\,a_{1},a_{2}\,)\,)\\
            \mbox{$(5)$ of def.~(\ref{Cat:def:canonical:product})}\ \to\ 
            &=&G_{1}(\,\id(a_{1})\,,\,\id(a_{2})\,)\\
            \mbox{$(2)$ of def.~(\ref{Fun:def:canonical:product})}\ \to\ 
            &=&(\,F_{1}(\,\id(a_{1})\,)\,,\,F_{2}(\,\id(a_{2})\,)\,)\\
            \mbox{$(4)$ of def.~(\ref{Fun:def:functor})}\ \to\ 
            &=&(\,\id(\,F_{1}(a_{1})\,)\,,\,\id(\,F_{2}(a_{2})\,)\,)\\
            \mbox{$(5)$ of def.~(\ref{Cat:def:canonical:product})}\ \to\ 
            &=&\id\,(\,F_{1}(a_{1})\,,\,F_{2}(a_{2})\,)\\
            \mbox{$(1)$ of def.~(\ref{Fun:def:canonical:product})}\ \to\ 
            &=&\id\,(\,G_{0}(a_{1},a_{2})\,)\\
            &=&\id\,(G_{0}(a))
        \end{eqnarray*}

    $(5)$: We need to show that $G_{1}(g\circ f)=G_{1}(g)\circ G_{1}(f)$ 
    for all $f:a\to b\ @\ \Cone\times\Ctwo$ and $g:b\to c\ @\ \Cone\times\Ctwo$.
    So let $f=(f_{1},f_{2})$ and $g=(g_{1},g_{2})$ with $a=(a_{1},a_{2})$, 
    $b=(b_{1},b_{2})$ and $c=(c_{1},c_{2})$. Following the same details as
    in $(3)$, we have $f_{1}:a_{1}\to b_{1}$ and $f_{2}:a_{2}\to b_{2}$ and
    similarly $g_{1}:b_{1}\to c_{1}$ and $g_{2}:b_{2}\to c_{2}$:
        \begin{eqnarray*}G_{1}(g\circ f)
            &=&G_{1}(\,(g_{1},g_{2})\,\circ\,(f_{1},f_{2})\,)\\
            \mbox{$(6)$ of def.~(\ref{Cat:def:canonical:product})}\ \to\ 
            &=&G_{1}(\,g_{1}\circ f_{1}\,,\,g_{2}\circ f_{2}\,)\\
            \mbox{$(2)$ of def.~(\ref{Fun:def:canonical:product})}\ \to\ 
            &=&(\,F_{1}(\,g_{1}\circ f_{1}\,)\,,\,F_{2}(\,g_{2}\circ f_{2}\,)\,)\\
            \mbox{$(5)$ of def.~(\ref{Fun:def:functor})}\ \to\ 
            &=&(\,F_{1}(g_{1})\,\circ\, F_{1}(f_{1})\,,\,F_{2}(g_{2})\,
                \circ\, F_{2}(f_{2})\,)\\
            \mbox{$(6)$ of def.~(\ref{Cat:def:canonical:product})}\ \to\ 
            &=&(\,F_{1}(g_{1})\,,\,F_{2}(g_{2})\,)\,\circ\,(\,F_{1}(f_{1})\,
                \,,\,F_{2}(f_{2})\,)\\
            \mbox{$(2)$ of def.~(\ref{Fun:def:canonical:product})}\ \to\ 
            &=&G_{1}(g_{1},g_{2})\,\circ\,G_{1}(f_{1},f_{2})\\
            &=&G_{1}(g)\,\circ\,G_{1}(f)
        \end{eqnarray*}
\end{proof}
