\begin{defin}\label{Fun:def:cartesian}
    The {\em cartesian functor} $F:\Set\times\Set\to\Set$ is defined by:
        \begin{eqnarray*}
            (1)&\ &F_{0}(a_{1},a_{2}) = a_{1}\times a_{2}\\
            (2)&\ &F_{1}(f_{1},f_{2})(x_{1},x_{2})
                =(\,f_{1}(x_{1})\,,\,f_{2}(x_{2})\,)
        \end{eqnarray*}
    where $(1)$ holds for all sets $a_{1},a_{2}$, and $(2)$ holds for all
    sets $a_{1},a_{2},b_{1},b_{2}$ as well as $f_{1}:a_{1}\to b_{1}$, 
    $f_{2}:a_{2}\to b_{2}$ together with $x_{1}\in a_{1}$ and $x_{2}\in a_{2}$.
\end{defin}

\noindent
{\bf Remark}: Recall that given two sets $a_{1}$ and $a_{2}$, the cartesian
product $a_{1}\times a_{2}$ of $a_{1}$ and $a_{2}$ is the set of all
ordered pairs $(x_{1},x_{2})$ where $x_{1}\in a_{1}$ and $x_{2}\in a_{2}$:
    \[
    a_{1}\times a_{2}=\{(x_{1},x_{2})\ |\ x_{1}\in a_{1},\ x_{2}\in a_{2}\}
    \]

\begin{notation}\label{Fun:notation:cartesian}
    The cartesian functor is denoted $(\times)$ as an infix operator.
\end{notation}

\noindent
{\bf Remark}: So if $F:\Set\times\Set\to\Set$ is the cartesian functor, we 
shall typicallly write $a_{1}\times a_{2}$ and $f_{1}\times f_{2}$ instead
of $F_{0}(a_{1},a_{2})$ and $F_{1}(f_{1},f_{2})$ respectively.

\noindent
{\bf Remark}: If $f_{1}$ and $f_{2}$ are arrows of the category \Set, strictly
speaking from definition~(\ref{Cat:def:set}) $f_{1}$ and $f_{2}$ are typed
functions $(a,b,f)$ where $f:a\to b$ is untyped. In particular 
$f_{1}$ and $f_{2}$ are sets and the cartesian product $f_{1}\times f_{2}$ is 
meaningful, so we have a notational conflict with $f_{1}\times f_{2}$, the 
cartesian functor evaluated at $(f_{1},f_{2})$. Similarly to the composition 
$\circ$, the symbol $\times$ is highly overloaded and may also be used to denote 
the canonical product of two categories as in 
definition~(\ref{Cat:def:canonical:product}), or the canonical product of
two functors as in notation~(\ref{Fun:notation:canonical:product}).

\begin{prop}\label{Fun:prop:cartesian}
    The cartesian functor $(\times)$ is a functor $(\times):\Set\times\Set
    \to\Set$. 
\end{prop}
\begin{proof}
    Let $F=(F_{0},F_{1})$ denote the cartesian functor $(\times)$. We need
    to check that properties $(1)-(5)$ of definition~(\ref{Fun:def:functor}) 
    are satisfied:

    $(1)$: $F_{0}$ is indeed a map $F_{0}:\ob\ (\Set\times\Set)\to\ob\ \Set$, 
    since $F_{0}$ is defined on the collection of all $(a_{1},a_{2})$ where
    $a_{1}$ and $a_{2}$ are sets and this collection is indeed the collection
    of all objects of $\Set\times\Set$. Furthemore $F_{0}(a_{1},a_{2})=
    a_{1}\times a_{2}\in\Set$.

    $(2)$: $F_{1}$ is indeed a map $F_{1}:\arr\ (\Set\times\Set)\to\arr\ \Set$:
    $F_{1}(f_{1},f_{2})$ is defined on the collection of all $(f_{1},f_{2})$
    where $f_{1}:a_{1}\to b_{1}$ and $f_{2}:a_{2}\to b_{2}$ for arbitrary
    $a_{1}, a_{2}, b_{1}, b_{2}$. So it is defined on the collection of all
    $(f_{1},f_{2})$ where $f_{1},f_{2}$ are arrows in \Set, and this collection
    is indeed $\arr\ (\Set\times\Set)$. Furthemore, if we have $f_{1}:a_{1}\to 
    b_{1}$ and $f_{2}:a_{2}\to b_{2}$, then $F_{1}(f_{1},f_{2})(x_{1},x_{2})$ is 
    defined for all $(x_{1},x_{2})$ where $x_{1}\in a_{1}$ and $x_{2}\in a_{2}$.
    So $F_{1}(f_{1},f_{2})$ is a function defined on the 
    cartesian product $a_{1}\times a_{2}$. Since $F_{1}(f_{1},f_{2})(x_{1},x_{2})$
    is defined as $(\,f_{1}(x_{1})\,,\,f_{2}(x_{2})\,)$ we see that $F_{1}
    (f_{1},f_{2})$ is in fact a function $F_{1}(f_{1},f_{2}):a_{1}\times a_{2}
    \to b_{1}\times b_{2}$. However, definition~(\ref{Fun:def:cartesian}) does
    not spell out the fact that $b_{1}\times b_{2}$ is the intended codomain
    of $F_{1}(f_{1},f_{2})$, but this is pretty clear from the context.
    Furthermore, definition~(\ref{Fun:def:cartesian}) does not explicitely 
    define a typed function $(a_{1}\times a_{2},b_{1}\times b_{2},F_{1}(f_{1},
    f_{2}))$ of \Set, but only an untyped function $F_{1}(f_{1},f_{2})$. This 
    is also fair enough given the context and in line with 
    notation~(\ref{Cat:notation:set:arrow}). We conlude that $F_{1}(f_{1},f_{2})
    \in\arr\ \Set$.

    $(3)$: We need to show that $F_{1}(f):F_{0}(a)\to F_{0}(b)$ whenever
    $f:a\to b$\,: let $f\in\arr\ (\Set\times\Set)$ such that $\dom(f)=a$ and
    $\cod(f)=b$. Then $a=(a_{1},a_{2})$ for some sets $a_{1},a_{2}$, $b=
    (b_{1},b_{2})$ for some sets $b_{1},b_{2}$ and $f=(f_{1},f_{2})$ for some
    arrows $f_{1}:a_{1}\to b_{1}$ and $f_{2}:a_{2}\to b_{2}$. We just established
    in $(2)$ the fact that $F_{1}(f_{1},f_{2}):a_{1}\times a_{2}\to b_{1}
    \times b_{2}$, which is $F_{1}(f):F_{0}(a)\to F_{0}(b)$ as requested.

    $(4)$: We need to show that $F_{1}(\,\id(a)\,)=\id(\,F_{0}(a)\,)$ for all 
    $a\in\Set\times\Set$. So let $a=(a_{1},a_{2})$. We need to show that
    $F_{1}(\,\id(a)\,)=\id(a_{1}\times a_{2})$. We already know that
    $F_{1}(\,\id(a)\,):a_{1}\times a_{2}\to a_{1}\times a_{2}$. So we simply
    need to check that the underlying function is the usual identity on the
    set $a_{1}\times a_{2}$. Let $(x_{1},x_{2})\in a_{1}\times a_{2}$:
        \begin{eqnarray*}F_{1}(\,\id(a)\,)(x_{1},x_{2})
            &=&F_{1}(\,\id(a_{1},a_{2})\,)(x_{1},x_{2})\\
            \mbox{$(5)$ of def.~(\ref{Cat:def:canonical:product})}\ \to\ 
            &=&F_{1}(\,\id(a_{1})\,,\,\id(a_{2})\,)(x_{1},x_{2})\\
            \mbox{$(2)$ of def.~(\ref{Fun:def:cartesian})}\ \to\ 
            &=&(\,\id(a_{1})(x_{1})\,,\,\id(a_{2})(x_{2})\,)\\
            \mbox{$(5)$ of def.~(\ref{Cat:def:set})}\ \to\ 
            &=&(\,(a_{1},a_{1},i(a_{1}))(x_{1})\,
               ,\,(a_{2},a_{2},i(a_{2}))(x_{2})\,)\\
            \mbox{notation~(\ref{Cat:notation:set:arrow:apply})}\ \to\ 
            &=&(\,i(a_{1})(x_{1})\,,\,i(a_{2})(x_{2})\,)\\
            i(a)(x)=x\ \to\ 
            &=&(x_{1},x_{2})
        \end{eqnarray*}

    $(5)$: We need to show that $F_{1}(g\circ f)=F_{1}(g)\circ F_{1}(f)$ for
    $f:a\to b\ @\ \Set\times\Set$ and $g:b\to c\ @\ \Set\times\Set$.
    So let $f=(f_{1},f_{2})$ and $g=(g_{1},g_{2})$ with $a=(a_{1},a_{2})$,
    $b=(b_{1},b_{2})$ and $c=(c_{1},c_{2})$. Then we have $f_{1}:a_{1}\to b_{1}$
    and $f_{2}:a_{2}\to b_{2}$ and similarly $g_{1}:b_{1}\to c_{1}$ and
    $g_{2}:b_{2}\to c_{2}$. We know that $F_{1}(f):a_{1}\times a_{2}
    \to b_{1}\times b_{2}$ and $F_{1}(g):b_{1}\times b_{2}\to c_{1}\times c_{2}$.
    Hence using proposition~(\ref{Cat:prop:set:arrow:equal}),
    in order to show that $F_{1}(g\circ f)=F_{1}(g)\circ F_{1}(f)$ it is
    sufficient to prove that the underlying functions coincide on the set
    $a_{1}\times a_{2}$. Given $(x_{1},x_{2})\in a_{1}\times a_{2}$, we have:
        \begin{eqnarray*}F_{1}(g\circ f)(x_{1},x_{2})
            &=&F_{1}(\,(g_{1},g_{2})\,\circ\,(f_{1},f_{2})\,)(x_{1},x_{2})\\
            \mbox{$(6)$ of def.~(\ref{Cat:def:canonical:product})}\ \to\ 
            &=&F_{1}(\,g_{1}\circ f_{1}\,,\,g_{2}\circ f_{2}\,)(x_{1},x_{2})\\
            \mbox{$(2)$ of def.~(\ref{Fun:def:cartesian})}\ \to\ 
            &=&(\,(g_{1}\circ f_{1})(x_{1})\,,\,(g_{2}\circ f_{2})(x_{2})\,)\\
            \mbox{$\circ$ in \Set}\ \to\ 
            &=&(\,g_{1}(f_{1}(x_{1}))\,,\,g_{2}(f_{2}(x_{2}))\,)\\
            \mbox{$(2)$ of def.~(\ref{Fun:def:cartesian})}\ \to\ 
            &=&F_{1}(g_{1},g_{2})(\,f_{1}(x_{1})\,,\,f_{2}(x_{2})\,)\\
            \mbox{$(2)$ of def.~(\ref{Fun:def:cartesian})}\ \to\ 
            &=&F_{1}(g_{1},g_{2})(F_{1}(f_{1},f_{2})(x_{1},x_{2}))\\
            \mbox{$\circ$ in \Set}\ \to\ 
            &=&(\,F_{1}(g_{1},g_{2})\,\circ\,F_{1}(f_{1},f_{2})\,)(x_{1},x_{2})\\
            &=&(\,F_{1}(g)\,\circ\,F_{1}(f)\,)(x_{1},x_{2})\\
        \end{eqnarray*}
\end{proof}

