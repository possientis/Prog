We are now familiar with the notion of {\em category} as defined
in~(\ref{Cat:def:category}) as well as that of functor as defined
in~(\ref{Fun:def:functor}). Thanks to definition~(\ref{Fun:def:composition}),
we know how to {\em compose functors}, and we also have a notion of
{\em identity functor} as defined in~(\ref{Fun:def:identity}). So it is
very tempting at this stage to wonder whether the {\em collection of all
categories} could be turned into a category itself, in which the objects
are categories and the arrows are functors. However, we know from set theory 
that assuming the existence of the {\em set of all sets} leads to a 
contradiction. Those familiar with proof assistants such that Coq, Agda
and Lean will also be used to the idea that {\em the type of all types}
does not exist, as we have {\em universes} and {\em type levels} instead.
So we shall not attempt to define {\em the category of all catgeories} here.
Instead, we shall focus on a collection which is a lot smaller by considering
only those categories which are {\em small}, as per 
definition~(\ref{Cat:def:category:small}).

\begin{defin}\label{Fun:def:category:small:cat}
    We call \cat\ the category $\cat=(\ob, \arr, \dom, \cod, \id, \circ)$ where
    \begin{eqnarray*}
        (1)&\ &\ob = \{\ \Cat\ |\ \mbox{\Cat\ is a small category}\ \}\\
        (2)&\ &\arr =\{\ (\Cat,\Dat,F)\ |\ \Cat,\Dat\in\ob\ 
        \mbox{and}\ F:\Cat\to\Dat\ \}\\
        (3)&\ &\dom\,(\Cat,\Dat,F)=\Cat\\
        (4)&\ &\cod\,(\Cat,\Dat,F)=\Dat\\
        (5)&\ &\id(\Cat)=(\Cat,\Cat,I_{\cal C})\\
        (6)&\ &(\Dat,\Eat,G) \circ (\Cat,\Dat,F) = (\Cat,\Eat,G\circ F)
    \end{eqnarray*}
    where $(3)-(6)$ hold for all small categories $\Cat,\Dat,\Eat$, and
    functors $F:\Cat\to\Dat$ and $G:\Dat\to\Eat$, $I_{\cal C}$ is the identity
    functor of definition~(\ref{Fun:def:identity}) and $G\circ F$ is the 
    composition of $G$ and $F$ as per definition~(\ref{Fun:def:composition}).
\end{defin}

\noindent
{\bf Remark}: Similarly to definition~(\ref{Cat:def:set}) where the arrows
of the category \Set\ are not functions but {\em typed functions}, the arrrows
of the category \cat\ are not functors but {\em typed functors} as per 
defintion~(\ref{Fun:def:typed:untyped:functor}).

The definition of \cat\ is formally very similar to that of the category \Set.
However, we still need to do our due diligence and checks it forms a category:

\begin{prop}\label{Fun:prop:category:small:cat}
    The category \cat\ of definition~(\ref{Fun:def:category:small:cat}) is 
    a category.
\end{prop}
\begin{proof}
    Given $(\ob, \arr, \dom, \cod, \id, \circ)$ of
    definition~(\ref{Fun:def:category:small:cat}), we need to check that this
    data satisfies condition~$(1)-(13)$ of definition~(\ref{Cat:def:category}).

    $(1)$: The collection $\ob=\{\ \Cat\ |\ \mbox{\Cat\ is a small category}\ \}$
    should be a collection with equality. The members of this collection are 
    small categories which according to definition~(\ref{Cat:def:category:small})
    are tuples $(\ob, \arr, \dom, \cod, \id, \circ)$ in which all the entries
    are sets. So such a tuple is a set and the collections of all small 
    categories is therefore a collection of sets for which set equality is 
    defined.

    $(2)$: The collection $\arr =\{\ (\Cat,\Dat,F)\ |\ \Cat,\Dat\in\ob\  
    \mbox{and}\ F:\Cat\to\Dat\ \}$ should be a collection with equality.
    It is sufficient for us to establish that all members of this collection
    are sets. We have already seen that a small category is a set
    (being a tuple with set entries) so each triple $(\Cat,\Dat,F)$ is also
    a set provided we can show that any functor $F:\Cat\to\Dat$ between two
    small categories is a set. However from definition~(\ref{Fun:def:functor}),
    such a functor is an ordered pair $F=(F_{0},F_{1})$ where $F_{0}$ and 
    $F_{1}$ are maps $F_{0}:\ob\ \Cat\to\ob\ \Dat$ and 
    $F_{1}:\arr\ \Cat\to\arr\ \Dat$. Now the categories \Cat\ and \Dat\ being
    small, all the collections $\ob\ \Cat$, $\ob\ \Dat$, $\arr\ \Cat$, 
    $\arr\ \Dat$ are in fact sets and $F_{0}$ and $F_{1}$ are therefore maps
    between sets. So $F_{0}$ and $F_{1}$ are themselves sets and so is 
    $F=(F_{0},F_{1})$.

    $(3)$: $\dom$ should be a map $\dom:\arr\to\ob$. The equation
    $\dom(\Cat,\Dat,F)=\Cat$ holds for all small categories \Cat,\Dat\ and 
    functor $F:\Cat\to\Dat$. So $\dom$ is indeed defined on the collection
    $\arr$ as requested. Furthermore $\dom f \in\ob$ for all $f$.

    $(4)$: $\cod$ should be a map $\cod:\arr\to\ob$ which is the case 
    as per $(3)$.

    $(5)$: $\id$ should be a map $\id:\ob\to\arr$. The equation $\id(\Cat)
    =(\Cat,\Cat,I_{\cal C})$ holds for all small category \Cat. So $\id$
    is indeed defined on \ob\ as requested. So it remains to show that
    $(\Cat,\Cat,I_{\cal C})\in\arr$ for all \Cat, which the case since
    the identity functor $I_{\cal C}$ is a functor $I_{\cal C}:\Cat\to\Cat$ as 
    per definition~(\ref{Fun:def:identity}).

    $(6)$: $\circ$ should be a partial map $\circ:\arr\times\arr\to\arr$.
    From definition~(\ref{Fun:def:category:small:cat}), $g\circ f$ is defined
    whenever $f$ and $g$ are of the form $f=(\Cat,\Dat,F)$ and $g=(\Dat,\Eat,G)$
    where $F:\Cat\to\Dat$ and $G:\Dat\to\Eat$. So $g\circ f$ is defined on
    a sub-collection of $\arr\times\arr$ as requested. So it remains to 
    show that $g\circ f\in\arr$ when defined. However, $g\circ f$ is defined
    as $(\Cat,\Eat,G \circ F)$ where $G\circ F$ is the functor composition,
    so it remains to show that $G\circ F:\Cat\to\Eat$ which follows from
    definition~(\ref{Fun:def:composition}).

    $(7)$: $g\circ f$ should be defined exactly when $\cod(f)=\dom(g)$. From
    definition~(\ref{Fun:def:category:small:cat}), $g\circ f$ is defined 
    exactly when $f$ is of the form $f=(\Cat,\Dat,F)$ and $g$ is of the form
    $(\Dat,\Eat,G)$. Since $\cod(f)=\Dat$ and $\dom(g)=\Dat$, we see that
    $g\circ f$ is defined for all arrows $f,g$ for which $\cod(f)=\dom(g)$ 
    as requested.

    $(8)$: We should have $\dom(g\circ f) = \dom(f)$ when $g\circ f$ is defined.
    So let $f=(\Cat,\Dat,F)$ and $g=(\Dat,\Eat,G)$. Then we have 
    $g\circ f=(\Cat,\Eat,G\circ F)$ and consequently $\dom(g\circ f)=\Cat$ which
    is $\dom(f)$ as requested.

    $(9)$: We should have $\cod(g\circ f) = \cod(g)$ when $g\circ f$ is defined.
    So let $f=(\Cat,\Dat,F)$ and $g=(\Dat,\Eat,G)$. Then we have 
    $g\circ f=(\Cat,\Eat,G\circ F)$ and consequently $\cod(g\circ f)=\Eat$ which
    is $\cod(g)$ as requested.

    $(10)$: We should have $(h\circ g)\circ f = h\circ(g\circ f)$ whenever
    $g\circ f$ and $h\circ g$ are well defined. So let $f=(\Cat,\Dat,F)$,
    $g=(\Dat,\Eat,G)$ and $h=(\Eat,{\cal F},H)$. We have:
        \begin{eqnarray*}(h\circ g)\circ f
            &=&(\,(\Eat,{\cal F},H)\,\circ\,(\Dat,\Eat,G)\,)\,
            \circ\,(\Cat,\Dat,F)\\
        \end{eqnarray*}
TODO
\end{proof}
