One question which we have not yet considered is that of equality between
functors. Given two categories \Cat\ and \Dat, and two functors $F,G:\Cat\to\Dat$, 
it will be convenient to argue when these two functors are equal. For one thing,
we would like to state that functor composition is associative, a statement which
is not meaningful without a notion of equality between functors. Such equality
is not obvious and may depend on the specifics of the logical context in 
which we are working. Equality between sets is given but anything else requires
careful consideration. Using the principles of section~(\ref{section:cat:equal}), 
we are able to state: 

\begin{prop}\label{Fun:prop:equal}
    Let $F,G:\Cat\to\Dat$ be two functors between categories \Cat,\Dat\ with:
        \begin{eqnarray*}
            (1)&\ &\forall a\in\ob\ \Cat\ \ ,\ F(a)=G(a)\\
            (2)&\ &\forall f\in\arr\ \Cat\ ,\ F(f)=G(f)
        \end{eqnarray*}
Then $F=G$, i.e. the two functors $F$ and $G$ are equal.
\end{prop}
\begin{proof}
    Let $F=(F_{0},F_{1})$ and $G=(G_{0},G_{1})$ be two functors between \Cat\
    and \Dat\ such that $F_{0}(a)=G_{0}(a)$ for all $a\in\ob\ \Cat$ and 
    $F_{1}(f)=G_{1}(f)$ for all $f\in\arr\ \Cat$. We need to show that 
    $F=G$, which is $(F_{0},F_{1})=(G_{0},G_{1})$. In order to show this
    equality, using axiom~(\ref{Cat:ax:tuple:extensional}) it is sufficient
    to prove that $F_{0}=G_{0}$ and $F_{1}=G_{1}$. However, by virtue of
    definition~(\ref{Fun:def:functor}), both $F_{0}$ and $G_{0}$ are
    maps $F_{0},G_{0}:\ob\ \Cat\to\ob\ \Dat$, and both $F_{1},G_{1}$
    are maps $F_{1},G_{1}:\arr\ \Cat\to\arr\ \Dat$. Hence, using
    axiom~(\ref{Cat:ax:map:extensional}), in order to show that $F_{0}=G_{0}$,
    it is sufficient to prove that $F_{0}(a)=G_{0}(a)$ for all $a\in\ob\ \Cat$,
    and this is true by assumption. Likewise, in order to show that
    $F_{1}=G_{1}$ it is sufficient to prove that $F_{1}(f)=G_{1}(f)$
    for all $f\in\arr\ \Cat$ which also true by assumption.
\end{proof}
