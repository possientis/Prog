One question which we have not yet considered is that of equality between
functors. Given two categories \Cat\ and \Dat, and two functors $F,G:\Cat\to\Dat$, 
it will sometimes be convenient to argue that these two functors are equal. For one 
thing, we would like to state that functor composition is associative, a statement 
which is not meaningful without a notion of equality between functors. Such equality
is not obvious and may depend on the specifics of the logical context in 
which we are working. Equality between sets is given but anything else requires
careful consideration. Using the principles of section~(\ref{section:cat:equal}), 
we are able to state: 

\begin{prop}\label{Fun:prop:equal}
    Let $F,G:\Cat\to\Dat$ be two functors between categories \Cat,\Dat\ with:
        \begin{eqnarray*}
            (1)&\ &\forall a\in\ob\ \Cat\ \ ,\ F(a)=G(a)\\
            (2)&\ &\forall f\in\arr\ \Cat\ ,\ F(f)=G(f)
        \end{eqnarray*}
    Then $F=G$, i.e. the two functors $F$ and $G$ are equal.
\end{prop}
\begin{proof}
    Let $F=(F_{0},F_{1})$ and $G=(G_{0},G_{1})$ be two functors between \Cat\
    and \Dat\ such that $F_{0}(a)=G_{0}(a)$ for all $a\in\ob\ \Cat$ and 
    $F_{1}(f)=G_{1}(f)$ for all $f\in\arr\ \Cat$. We need to show that 
    $F=G$, which is $(F_{0},F_{1})=(G_{0},G_{1})$. In order to show this
    equality, using axiom~(\ref{Cat:ax:tuple:extensional}) it is sufficient
    to prove that $F_{0}=G_{0}$ and $F_{1}=G_{1}$. However, by virtue of
    definition~(\ref{Fun:def:functor}), both $F_{0}$ and $G_{0}$ are
    maps $F_{0},G_{0}:\ob\ \Cat\to\ob\ \Dat$, and both $F_{1},G_{1}$
    are maps $F_{1},G_{1}:\arr\ \Cat\to\arr\ \Dat$. Hence, using
    axiom~(\ref{Cat:ax:map:extensional}), in order to show that $F_{0}=G_{0}$,
    it is sufficient to prove that $F_{0}(a)=G_{0}(a)$ for all $a\in\ob\ \Cat$,
    and this is true by assumption. Likewise, in order to show that
    $F_{1}=G_{1}$ it is sufficient to prove that $F_{1}(f)=G_{1}(f)$
    for all $f\in\arr\ \Cat$ which also true by assumption.
\end{proof}


\begin{prop}\label{Fun:prop:associative}
    Let $F:\Cat\to\Dat$, $G:\Dat\to\Eat$ and $H:\Eat\to{\cal F}$ be functors
    between categories \Cat,\Dat,\Eat\ and $\cal F$. Then we have the equality:
        \[
            (H\circ G)\circ F = H\circ (G\circ F)
        \]
    i.e. functor composition is associative.
\end{prop}
\begin{proof}
    Using proposition~(\ref{Fun:prop:equal}) we simply need to show that
    the two functors $(H\circ G)\circ F$ and $H\circ (G\circ F)$ act equally 
    on both objects and arrows of the category \Cat. For $a\in\ob\ \Cat$, 
    this goes as follows:
        \begin{eqnarray*}((H\circ G)\circ F)(a)
            &=&(H\circ G)(\,F(a)\,)\ \leftarrow\ 
            \mbox{def.~(\ref{Fun:def:composition})}\\
            \mbox{def.~(\ref{Fun:def:composition})}\ \to\ 
            &=&H(\,G(\,F(a)\,)\,)\\
            \mbox{def.~(\ref{Fun:def:composition})}\ \to\ 
            &=&H(\,(G\circ F)(a)\,)\\
            \mbox{def.~(\ref{Fun:def:composition})}\ \to\ 
            &=&(H\circ (G\circ F))(a)
        \end{eqnarray*}
    and likewise for $f\in\arr\ \Cat$:
        \begin{eqnarray*}((H\circ G)\circ F)(f)
            &=&(H\circ G)(\,F(f)\,)\ \leftarrow\ 
            \mbox{def.~(\ref{Fun:def:composition})}\\
            \mbox{def.~(\ref{Fun:def:composition})}\ \to\ 
            &=&H(\,G(\,F(f)\,)\,)\\
            \mbox{def.~(\ref{Fun:def:composition})}\ \to\ 
            &=&H(\,(G\circ F)(f)\,)\\
            \mbox{def.~(\ref{Fun:def:composition})}\ \to\ 
            &=&(H\circ (G\circ F))(f)
        \end{eqnarray*}
\end{proof}

\begin{prop}\label{Fun:prop:left:identity}
    Let $F:\Cat\to\Dat$ be a functor between categories \Cat, \Dat. Then:
        \[
            I_{\cal D}\circ F = F
        \]
    where $I_{\cal D}:\Dat\to\Dat$ is the identity functor on \Dat\ as per
    definition~(\ref{Fun:def:identity}).
\end{prop}
\begin{proof}
    Using proposition~(\ref{Fun:prop:equal}) we simply need to show that
    the two functors $I_{\cal D}\circ F$ and $F$ act equally on both objects 
    and arrows of the category \Cat. For $a\in\ob\ \Cat$:
        \begin{eqnarray*}(I_{\cal D}\circ F)(a)
            &=&I_{\cal D}(\,F(a)\,)
            \ \leftarrow\ \mbox{$(1)$ of def.~(\ref{Fun:def:composition})}\\
            \mbox{$(1)$ of def.~(\ref{Fun:def:identity})}\ \to\ 
            &=&F(a)
        \end{eqnarray*}
    and likewise for $f\in\arr\ \Cat$:
        \begin{eqnarray*}(I_{\cal D}\circ F)(f)
            &=&I_{\cal D}(\,F(f)\,)
            \ \leftarrow\ \mbox{$(2)$ of def.~(\ref{Fun:def:composition})}\\
            \mbox{$(2)$ of def.~(\ref{Fun:def:identity})}\ \to\ 
            &=&F(f)
        \end{eqnarray*}
\end{proof}

\begin{prop}\label{Fun:prop:right:identity}
    Let $F:\Cat\to\Dat$ be a functor between categories \Cat, \Dat. Then:
        \[
            F\circ I_{\cal C} = F
        \]
    where $I_{\cal C}:\Cat\to\Cat$ is the identity functor on \Cat\ as per
    definition~(\ref{Fun:def:identity}).
\end{prop}
\begin{proof}
    Using proposition~(\ref{Fun:prop:equal}) we simply need to show that
    the two functors $F\circ I_{\cal C}$ and $F$ act equally on both objects 
    and arrows of the category \Cat. For $a\in\ob\ \Cat$:
        \begin{eqnarray*}(F\circ I_{\cal C})(a)
            &=&F(\,I_{\cal C}(a)\,)
            \ \leftarrow\ \mbox{$(1)$ of def.~(\ref{Fun:def:composition})}\\
            \mbox{$(1)$ of def.~(\ref{Fun:def:identity})}\ \to\ 
            &=&F(a)
        \end{eqnarray*}
    and likewise for $f\in\arr\ \Cat$:
        \begin{eqnarray*}(F\circ I_{\cal C})(f)
            &=&F(\,I_{\cal C}(f)\,)
            \ \leftarrow\ \mbox{$(2)$ of def.~(\ref{Fun:def:composition})}\\
            \mbox{$(2)$ of def.~(\ref{Fun:def:identity})}\ \to\ 
            &=&F(f)
        \end{eqnarray*}
\end{proof}
