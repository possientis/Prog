\begin{defin}\label{Fun:def:pairing}
    We call {\em pairing} of two functors $F_{1}:\Cat\to\Done$ and
    $F_{2}:\Cat\to\Dtwo$ the functor $G:\Cat\to\Done\times\Dtwo$
    defined by $G=(G_{0},G_{1})$ with:
        \begin{eqnarray*}
            (1)&\ &G_{0}(a) = (\,F_{1}(a)\,,\,F_{2}(a)\,)\\
            (2)&\ &G_{1}(f) = (\,F_{1}(f)\,,\,F_{2}(f)\,)
        \end{eqnarray*}
    where $\Cat, \Done, \Dtwo$ are categories and $(1)-(2)$ hold
    for all $a\in\ob\ \Cat$ and $f\in\arr\ \Cat$.
\end{defin}
\noindent
{\bf Remark}: In accordance with notation~(\ref{Fun:notation:functor:F0:F1}),
we are using the same notations $F_{1}$ and $F_{2}$ in 
definition~(\ref{Fun:def:pairing}) to describe the actions of these
functors both on objects and on arrows. The alternative would be to use the
notations $(F_{1})_{0}$, $(F_{1})_{1}$, $(F_{2})_{0}$ and $(F_{2})_{1}$ which 
would arguably be harder to read.

\begin{notation}\label{Fun:notation:pairing}
    The pairing of two functors $F_{1}$ and $F_{2}$ is denoted
    $\langle F_{1}, F_{2}\rangle$.
\end{notation}
\begin{prop}\label{Fun:prop:pairing}
    Let $F_{1}:\Cat\to\Done$ and $F_{2}:\Cat\to\Dtwo$ be two functors. Then 
    the pairing of $F_{1}$ and $F_{2}$ is indeed a functor 
    $\langle F_{1}, F_{2}\rangle : \Cat\to\Done\times\Dtwo$.
\end{prop}
\begin{proof}
    Let $G=(G_{0},G_{1})$ denote the functor $\langle F_{1}, F_{2}\rangle$.
    We need to check that properties~$(1)-(5)$ of 
    definition~(\ref{Fun:def:functor}) are satisfied, which goes as follows:

    $(1)$: $G_{0}$ is indeed a map $G_{0}:\ob\ \Cat\to\ob\ (\Done
    \times\Dtwo)$: firstly, $G_{0}$ is defined on the 
    collection of all $a$'s where $a\in\ob\ \Cat$. Furthermore 
    $G_{0}(a)$ is defined as $(\,F_{1}(a)\,,\,F_{2}(a)\,)$ and since 
    $F_{1}:\Cat\to\Done$ while $F_{2}:\Cat\to\Dtwo$, $F_{1}(a)
    \in\ob\ \Done$ while $F_{2}(a)\in\ob\ \Dtwo$. Hence we see that
    $G_{0}(a)$ is indeed a member of the collection $\ob\ (\Done\times\Dtwo)$,
    by virtue of definition~(\ref{Cat:def:canonical:product}).
    
    $(2)$: $G_{1}$ is indeed a map $G_{1}:\arr\ \Cat\to\arr\ (\Done
    \times\Dtwo)$: $G_{1}$ is defined on the 
    collection of all $f$'s where $f\in\arr\ \Cat$. Furthermore
    $G_{1}(f)$ is defined as $(\,F_{1}(f)\,,\,F_{2}(f)\,)$ and since 
    $F_{1}:\Cat\to\Done$ while we have $F_{2}:\Cat\to\Dtwo$, we see that
    $F_{1}(f)\in\arr\ \Done$ while $F_{2}(f)\in\arr\ \Dtwo$. Hence we 
    conclude that $G_{1}(f)$ is indeed a member of the collection 
    $\arr\ (\Done\times\Dtwo)$.

    $(3)$: We need to show that $G_{1}(f):G_{0}(a)\to G_{0}(b)$ when
    $f:a\to b$: so consider $f\in\arr\ \Cat$ such that $\dom(f)=a$ and 
    $\cod(f)=b$. We need to show that $\dom\,(G_{1}(f))=G_{0}(a)$
    and $\cod\,(G_{1}(f))=G_{0}(b)$, which goes as follows:
        \begin{eqnarray*}\dom\,(G_{1}(f))
            &=&\dom\,(\,F_{1}(f)\,,\,F_{2}(f))\ \leftarrow \
            \mbox{$(2)$ of def.~(\ref{Fun:def:pairing})}\\
            \mbox{$(3)$ of def.~(\ref{Cat:def:canonical:product})}\ \to\ 
            &=&(\,\dom\,(\,F_{1}(f)\,)\,,\,\dom\,(\,F_{2}(f)\,)\,)\\
            F_{1}(f):F_{1}(a)\to F_{1}(b)\ \to\ 
            &=&(\,F_{1}(a)\,,\,\dom\,(\,F_{2}(f)\,)\,)\\
            F_{2}(f):F_{2}(a)\to F_{2}(b)\ \to\ 
            &=&(\,F_{1}(a)\,,\,F_{2}(a)\,)\\
            \mbox{$(1)$ of def.~(\ref{Fun:def:pairing})}\ \to\ 
            &=&G_{0}(a)
        \end{eqnarray*}
        \begin{eqnarray*}\cod\,(G_{1}(f))
            &=&\cod\,(\,F_{1}(f)\,,\,F_{2}(f))\ \leftarrow\ 
            \mbox{$(2)$ of def.~(\ref{Fun:def:pairing})}\\
            \mbox{$(4)$ of def.~(\ref{Cat:def:canonical:product})}\ \to\ 
            &=&(\,\cod\,(\,F_{1}(f)\,)\,,\,\cod\,(\,F_{2}(f)\,)\,)\\
            F_{1}(f):F_{1}(a)\to F_{1}(b)\ \to\ 
            &=&(\,F_{1}(b)\,,\,\cod\,(\,F_{2}(f)\,)\,)\\
            F_{2}(f):F_{2}(a)\to F_{2}(b)\ \to\ 
            &=&(\,F_{1}(b)\,,\,F_{2}(b)\,)\\
            \mbox{$(1)$ of def.~(\ref{Fun:def:pairing})}\ \to\ 
            &=&G_{0}(b)
        \end{eqnarray*}

    $(4)$: We have $G_{1}(\,\id(a)\,)=\id(\,G_{0}(a)\,)$ for all $a\in\Cat$:
        \begin{eqnarray*}G_{1}(\,\id(a)\,)
            &=&(\,F_{1}(\,\id(a)\,)\,,\,F_{2}(\,\id(a)\,)\,)\ \leftarrow\ 
            \mbox{$(2)$ of def.~(\ref{Fun:def:pairing})}\\
            \mbox{$(4)$ of def.~(\ref{Fun:def:functor})}\ \to\ 
            &=&(\,\id(\,F_{1}(a)\,)\,,\,\id(\,F_{2}(a)\,)\,)\\
            \mbox{$(5)$ of def.~(\ref{Cat:def:canonical:product})}\ \to\ 
            &=&\id\,(\,F_{1}(a)\,,\,F_{2}(a)\,)\\
            \mbox{$(1)$ of def.~(\ref{Fun:def:pairing})}\ \to\ 
            &=&\id\,(G_{0}(a))
        \end{eqnarray*}

    $(5)$: We need to show that $G_{1}(g\circ f)=G_{1}(g)\circ G_{1}(f)$ 
    for all $f:a\to b\ @\ \Cat$ and $g:b\to c\ @\ \Cat$. The proof goes
    as follows:
        \begin{eqnarray*}G_{1}(g\circ f)
            &=&(\,F_{1}(\,g\circ f\,)\,,\,F_{2}(\,g\circ f\,)\,)\ \leftarrow\ 
            \mbox{$(2)$ of def.~(\ref{Fun:def:pairing})}\\
            \mbox{$(5)$ of def.~(\ref{Fun:def:functor})}\ \to\ 
            &=&(\,F_{1}(g)\,\circ\, F_{1}(f)\,,\,F_{2}(g)\,
                \circ\, F_{2}(f)\,)\\
            \mbox{$(6)$ of def.~(\ref{Cat:def:canonical:product})}\ \to\ 
            &=&(\,F_{1}(g)\,,\,F_{2}(g)\,)\,\circ\,(\,F_{1}(f)\,
                \,,\,F_{2}(f)\,)\\
            \mbox{$(2)$ of def.~(\ref{Fun:def:pairing})}\ \to\ 
            &=&G_{1}(g)\,\circ\,G_{1}(f)
        \end{eqnarray*}
\end{proof}
