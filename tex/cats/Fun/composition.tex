\begin{defin}\label{Fun:def:composition}
    Let $F:\Cat\to\Dat$ and $G:\Dat\to\Eat$ be functors where \Cat,\Dat,\Eat\ 
    are categories. We call {\em composition of $G$ and $F$} the functor
    $H:\Cat\to\Eat$ defined by:
        \begin{eqnarray*}
            (1)&\ &H_{0}(a) = G_{0}(F_{0}(a))\\
            (2)&\ &H_{1}(f) = G_{1}(F_{1}(f))
        \end{eqnarray*}
    where $(1)$ holds for all $a\in\ob\ \Cat$ and $(2)$ holds for all
    $f\in\arr\ \Cat$.
\end{defin}

\begin{notation}\label{Fun:notation:composition}
    The composition of two functors $G$ and $F$ is denoted $GF$ or $G\circ F$.
\end{notation}

\noindent
{\bf Remark}: In view of notation~(\ref{Cat:ax:composition}), $(1)$ and $(2)$ of
definition~(\ref{Fun:def:composition}) could equally have been written
$H_{0}=G_{0}\circ F_{0}$ and $H_{1}=G_{1}\circ F_{1}$. Note that the overloaded
symbol '$\circ$' has many possible meanings: it is the usual symbol for
set-theoretic function composition, it also refers to composition of maps
between collections as per notation~(\ref{Cat:ax:composition}), it is the 
generic symbol for the composition operator in an arbitrary category as
per notation~(\ref{Cat:notation:composition}), and finally it is also used to
denote composition of functors as per notation~(\ref{Fun:notation:composition}).
\begin{prop}\label{Fun:prop:composition}
    Let $F:\Cat\to\Dat$ and $G:\Dat\to\Eat$ be functors where \Cat,\Dat,\Eat\ 
    are categories. Then $G\circ F$ is indeed a functor $G\circ F:\Cat\to\Eat$.
\end{prop}
\begin{proof}
    We need to check that properties~$(1)-(5)$ of 
    definition~(\ref{Fun:def:functor}) are satisfied:

    $(1)$: $(GF)_{0}=G_{0}\circ F_{0}$ is indeed a map $(GF)_{0}:\ob\ \Cat
    \to\ob\ \Eat$, since: 
        \[
            F_{0}:\ob\ \Cat\to\ob\ \Dat,\  G_{0}:\ob\ \Dat\to\ob\ \Eat
        \]

    $(2)$: $(GF)_{1}=G_{1}\circ F_{1}$ is indeed a map $(GF)_{1}:\arr\ \Cat
    \to\arr\ \Eat$, since: 
        \[
            F_{1}:\arr\ \Cat\to\arr\ \Dat,\  G_{1}:\arr\ \Dat\to\arr\ \Eat
        \]

    $(3)$: We have $(GF)_{1}(f) : (GF)_{0}(a)\to(GF)_{0}(b)$ whenever $f:a\to b$, 
    since:
        \begin{eqnarray*}f:a\to b
            &\Rightarrow& F_{1}(f):F_{0}(a)\to F_{0}(b) 
            \mbox{\ \ $\leftarrow$\ $(3)$ of def.~(\ref{Fun:def:functor})}\\ 
            \mbox{$(3)$ of def.~(\ref{Fun:def:functor})}\ \to\ 
            &\Rightarrow& G_{1}(F_{1}(f)):G_{0}(F_{0}(a))\to G_{0}(F_{0}(b))\\
            \mbox{$(1)$ and $(2)$ of def.~(\ref{Fun:def:composition})}\ \to\ 
            &\Leftrightarrow& (GF)_{1}(f):(GF)_{0}(a)\to (GF)_{0}(b)
        \end{eqnarray*}

    $(4)$: We have $(GF)_{1}(\,\id(a)\,)=\id(\,(GF)_{0}(a)\,)$ for all
    $a\in\Cat$, since:
        \begin{eqnarray*}(GF)_{1}(\,\id(a)\,)
            &=&G_{1}(F_{1}(\,\id(a)\,))
            \ \leftarrow\ \mbox{$(2)$ of def.~(\ref{Fun:def:composition})}\\
            \mbox{$(4)$ of def.~(\ref{Fun:def:functor})}\ \to\ 
            &=&G_{1}(\,\id(\,F_{0}(a)\,)\,)\\
            \mbox{$(4)$ of def.~(\ref{Fun:def:functor})}\ \to\ 
            &=&\id(\,G_{0}(\,F_{0}(a)\,)\,)\\
            \mbox{$(1)$ of def.~(\ref{Fun:def:composition})}\ \to\ 
            &=&\id(\,(GF)_{0}(a)\,)\\
        \end{eqnarray*}

    $(5)$: $(GF)_{1}(g\circ f)=(GF)_{1}(g)\circ (GF)_{1}(f)$ whenever $f:a\to b$
    and $g:b\to c$\,:
        \begin{eqnarray*}(GF)_{1}(g\circ f)
            &=&G_{1}(F_{1}(\,g\circ f\,))
            \ \leftarrow\ \mbox{$(2)$ of def.~(\ref{Fun:def:composition})}\\
            \mbox{$(5)$ of def.~(\ref{Fun:def:functor})}\ \to\ 
            &=&G_{1}(\,F_{1}(g)\circ F_{1}(f)\,)\\
            \mbox{$(5)$ of def.~(\ref{Fun:def:functor})}\ \to\ 
            &=&G_{1}(F_{1}(g))\circ G_{1}(F_{1}(f))\\
            \mbox{$(2)$ of def.~(\ref{Fun:def:composition})}\ \to\ 
            &=&(GF)_{1}(g)\circ (GF)_{1}(f)\\
        \end{eqnarray*}
\end{proof}

