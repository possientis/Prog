\begin{defin}\label{Fun:def:functor}
    We call {\em functor} from categories \Cat\ to \Dat\ any tuple 
    $(F_{0},F_{1})$ with:
        \begin{eqnarray*}
            (1)&\ &F_{0}:\ob\ \Cat\to\ob\ \Dat\mbox{\ is a map}\\
            (2)&\ &F_{1}:\arr\ \Cat\to\arr\ \Dat\mbox{\ is a map}\\
            (3)&\ &F_{1}(f):F_{0}(a)\to F_{0}(b)\\
            (4)&\ &F_{1}(\,\id(a)\,)=\id(\,F_{0}(a)\,)\\
            (5)&\ &F_{1}(g\circ f)=F_{1}(g)\circ F_{1}(f)
    \end{eqnarray*}
    where $(3)-(5)$ hold for all $a,b,c\in\Cat$, $f:a\to b$ and $g:b\to c$.
\end{defin}
\noindent
\begin{notation}\label{Fun:notation:functor:arrow}
    We shall use $F:\Cat\to\Dat$ as a notational shortcut for the statement 
    that {\em $F$ is a functor from the category \Cat\ to the category \Dat}.
\end{notation}

\begin{notation}\label{Fun:notation:functor:F0:F1}
If $F=(F_{0},F_{1})$ is a functor from \Cat\ to \Dat, we shall also commonly 
denote $F_{0}$ and $F_{1}$ simply by $F$.
\end{notation}


So if $F$ is a functor $F:\Cat\to\Dat$ we effectively have a map 
$F:\ob\ \Cat\to\ob\ \Dat$ acting on objects, and a map 
$F:\arr\ \Cat\to\arr\ \Dat$ acting on arrows. These two maps 
satisfy the consistency condition~$(3)$ of definition~(\ref{Fun:def:functor})
i.e. that if $f$ is an arrow $f:a\to b$ in \Cat, then $F(f)$ must be 
an arrow $F(f):F(a)\to F(b)$ in~\Dat. Furthermore, the functor
$F$ must preserve the identity operators on \Cat\ and \Dat\ which
is condition~$(4)$ of definition~(\ref{Fun:def:functor}): for all objects
$a\in\Cat$, we must have $F(\,\id(a)\,)=\id(\,F(a)\,)$. Note that since
$\id(a):a\to a$, by consistency we have $F(\,\id(a)\,):F(a)\to F(a)$,
and since $\id(\,F(a)\,):F(a)\to F(a)$ the equality makes sense.
Another way to express the preservation of identity operators by $F$
is simply $F\circ\id=\id\circ F$ or $F_{1}\circ\id=\id\circ F_{0}$ to 
be more explicit. However, we should remember that the notation '$\circ$'
in these equality does not refer to the composition operator $\circ$
of either \Cat\ or \Dat, nor does it in general refer to the usual
function composition since $\id$, $F_{0}$ and $F_{1}$ are maps between 
collections and not functions between sets. Now going back to our
functor $F$, it must also preserve the composition operators on
\Cat\ and \Dat, which is condition~$(5)$ of definition~(\ref{Fun:def:functor}):
For all objects $a,b,c\in\Cat$ and arrows $f:a\to b$ and $g:b\to c$, 
we must have $F(g\circ f)=F(g)\circ F(f)$. Note that given these 
assumptions, the composition arrow $g\circ f$ is well-defined,
and by consistency we have $F(f):F(a)\to F(b)$ and $F(g):F(b)\to F(c)$,
so $F(g)\circ F(f)$ is also well-defined. Furthermore, since
$g\circ f:a \to c$ by consistency we have $F(g\circ f):F(a)\to F(c)$
and since $F(g)\circ F(f):F(a)\to F(c)$, the equality 
$F(g\circ f)=F(g)\circ F(f)$ makes sense.
