\begin{defin}\label{Nat:def:natural}
    Let $F,G:\Cat\to\Dat$ be functors where \Cat\ and \Dat\ are categories.
    We call {\em natural transformation} from $F$ to $G$ any map $\alpha$
    defined on $\ob\ \Cat$ with:
        \begin{eqnarray*}
            (1)& &\alpha(a) : F(a) \to G(a)\\
            (2)& &G(f)\circ\alpha(a) = \alpha(b) \circ F(f)
        \end{eqnarray*}
    where $(1)$ holds for all $a\in\ob\ \Cat$ and $(2)$ holds for all 
    $a,b\in\ob\ \Cat$ and $f:a\to b$.
\end{defin}
\begin{notation}\label{Nat:notation:natural:arrow}
    We shall use $\alpha:F\Rightarrow G$ as a notational shortcut for the
    statement that {\em $\alpha$ is a natural transformation from the
    functor $F$ to the functor $G$}.
\end{notation}

\begin{defin}\label{Nat:def:natural:component}
    Given $F,G:\Cat\to\Dat$ and $\alpha:F\Rightarrow G$, given $a\in\Cat$
    we call {\em component at $a$ of the natural transformation $\alpha$},
    the arrow $\alpha(a):F(a)\to G(a)$.
\end{defin}

\noindent
{\bf Remark}: The component $\alpha(a)$ of $\alpha$ at $a\in\Cat$ is an arrow 
in the category \Dat.

\begin{notation}\label{Nat:notation:natural:component}
    The component $\alpha(a)$ of $\alpha$ at $a$ is denoted $\alpha_{a}$.
\end{notation}

\noindent
{\bf Remark}: Given $F,G:\Cat\to\Dat$ and $\alpha:F\Rightarrow G$, given
$a,b\in\ob\ \Cat$ and $f:a \to b$, we have $F(f):F(a) \to F(b)$ and 
$G(f):G(a)\to G(b)$. Furthemore, from $(1)$ of definition~(\ref{Nat:def:natural}), 
we have $\alpha_{a}:F(a)\to G(a)$ and $\alpha_{b}:F(b)\to G(b)$. It follows 
that both arrows $G(f)\circ\alpha_{a}$ and $\alpha_{b}\circ F(f)$ are well 
defined arrows in \Dat\ (from $F(a)$ to $G(b)$), and the equality $(2)$ of 
definition~(\ref{Nat:def:natural}) is always meaningful.
