\begin{defin}\label{Nat:def:natural}
    Let $F,G:\Cat\to\Dat$ be functors where \Cat\ and \Dat\ are categories.
    We call {\em natural transformation} from $F$ to $G$ any map 
    $\alpha : \ob\ \Cat\to\arr\ \Dat$ with:
        \begin{eqnarray*}
            (1)& &\alpha(a) : F(a) \to G(a)\\
            (2)& &G(f)\circ\alpha(a) = \alpha(b) \circ F(f)
        \end{eqnarray*}
    where $(1)$ holds for all $a\in\ob\ \Cat$ and $(2)$ holds for all 
    $a,b\in\ob\ \Cat$ and $f:a\to b$.
\end{defin}

\noindent
{\bf Remark}: the statement {\em $\alpha$ is a natural transformation
from $F$ to $G$} is strictly speaking ambiguous since $F$ and $G$ are
untyped functors which do not in themselves tell us which are the categories
\Cat\ and \Dat\ under consideration. These are normally clear from the context
however. It may sometimes be useful to say that {\em $\alpha$ is a natural 
transformation between the typed functors $(\Cat,\Dat,F)$ and $(\Cat,\Dat,G)$}.

\begin{notation}\label{Nat:notation:natural:arrow}
    We shall use $\alpha:F\Rightarrow G$ as a notational shortcut for the
    statement that {\em $\alpha$ is a natural transformation from the
    functor $F$ to the functor $G$}.
\end{notation}

\noindent
{\bf Remark}: A mental picture of a natural transformation $\alpha:F\Rightarrow 
G$ where $F$ and $G$ are two functors between categories \Cat\ and \Dat\ is as 
follows:
    \[
        \begin{tikzcd}
            \Cat \arrow[r, "F", bend left  = 50, ""{name=U, below}]
                 \arrow[r, swap, "G", bend right = 50, ""{name=D, above}]
              & \Dat
            \arrow[Rightarrow, "\,\alpha", from = U, to = D]
        \end{tikzcd}
    \]

\noindent
{\bf Remark}: Given $F,G:\Cat\to\Dat$ and $\alpha:F\Rightarrow G$, given
$a,b\in\ob\ \Cat$ and $f:a \to b$, since $F$ and $G$ are functors we have 
$F(f):F(a) \to F(b)$, $G(f):G(a)\to G(b)$ and from $(1)$ of 
definition~(\ref{Nat:def:natural}), $\alpha(a):F(a)\to G(a)$ and 
$\alpha(b):F(b)\to G(b)$. It follows that both arrows $G(f)\circ\alpha(a)$ 
and $\alpha(b)\circ F(f)$ are well defined arrows in \Dat\ (from $F(a)$ to 
$G(b)$), and the equality $(2)$ of definition~(\ref{Nat:def:natural}) is 
always meaningful.

\noindent
{\bf Remark}: Equality~$(2)$ of definition~(\ref{Nat:def:natural}) is commonly
visualized as: 
    \[
        \begin{tikzcd}
            a\arrow[d,swap, "f"]
            &F(a)\arrow[r, "\alpha(a)"]\arrow[d, swap,"F(f)"]
            &G(a)\arrow[d, "G(f)"]
            \\
            b
            &F(b)\arrow[r, swap, "\alpha(b)"]
            &G(b)
        \end{tikzcd}
    \]
This diagram is called the {\em naturality square} of the
natural transformation $\alpha$ relative to $f:a \to b$. Equality~$(2)$
is informally expressed by saying that {\em the naturality square commutes},
i.e. that both arrows obtained by composition along the two paths from $F(a)$ 
to $G(b)$ are equal. 

\begin{defin}\label{Nat:def:natural:component}
    Given $F,G:\Cat\to\Dat$ and $\alpha:F\Rightarrow G$, given $a\in\Cat$
    we call {\em component at $a$ of the natural transformation $\alpha$},
    the arrow $\alpha(a):F(a)\to G(a)$.
\end{defin}

\noindent
{\bf Remark}: The component $\alpha(a)$ of $\alpha$ at $a\in\Cat$ is an arrow 
in the category \Dat.

\begin{notation}\label{Nat:notation:natural:component}
    The component $\alpha(a)$ of $\alpha$ at $a$ is commonly denoted $\alpha_{a}$.
\end{notation}


