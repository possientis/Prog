\begin{defin}\label{Nat:def:natural}
    Let $F,G:\Cat\to\Dat$ be functors where \Cat\ and \Dat\ are categories.
    We call {\em natural transformation} from the typed functor $(\Cat,\Dat,F)$ 
    to the typed functor $(\Cat,\Dat,G)$ any map $\alpha : \ob\ \Cat\to\arr\ \Dat$ 
    with the following properties:
        \begin{eqnarray*}
            (1)& &\alpha(a) : F(a) \to G(a)\\
            (2)& &G(f)\circ\alpha(a) = \alpha(b) \circ F(f)
        \end{eqnarray*}
    where $(1)$ holds for all $a\in\ob\ \Cat$ and $(2)$ holds for all 
    $a,b\in\ob\ \Cat$ and $f:a\to b$.
\end{defin}

\noindent
{\bf Remark}: It is very common to casually say that {\em $\alpha$ is a natural
transformation between $F$ and $G$}, i.e. to refer only to the untyped functors
$F$ and $G$ rather than spell out the typed functors $(\Cat,\Dat,F)$ and
$(\Cat,\Dat,G)$. This is fine as long as the typed functors under consideration
are clear from the context. It should be remembered however that being a
natural transformation is a statement about typed functors, not untyped
functors, as the knowledge of which categories \Cat\ and \Dat\ are involved
crucially matters and these categories are not determined from $F$ and $G$
alone, which are also functors from $\Cop\to\Dop$ 
(proposition~(\ref{Fun:prop:opposite}).

\begin{notation}\label{Nat:notation:natural:arrow}
    We shall use $\alpha:(\Cat,\Dat,F)\Rightarrow (\Cat,\Dat,G)$ as a notational 
    shortcut for the statement that {\em $\alpha$ is a natural transformation 
    between the typed functors $(\Cat,\Dat,F)$ and $(\Cat,\Dat,G)$}. We shall
    write $\alpha:F\Rightarrow G$ when the context is clear.
\end{notation}

\noindent
{\bf Remark}: A mental picture of a natural transformation $\alpha:F\Rightarrow 
G$ where $F$ and $G$ are two functors between categories \Cat\ and \Dat\ is as 
follows:
    \[
        \begin{tikzcd}
            \Cat \arrow[r, "F", bend left  = 50, ""{name=U, below}]
                 \arrow[r, swap, "G", bend right = 50, ""{name=D, above}]
              & \Dat
            \arrow[Rightarrow, "\,\alpha", from = U, to = D]
        \end{tikzcd}
    \]

\noindent
{\bf Remark}: Given $F,G:\Cat\to\Dat$ and $\alpha:F\Rightarrow G$, given
$a,b\in\ob\ \Cat$ and $f:a \to b$, since $F$ and $G$ are functors we have 
$F(f):F(a) \to F(b)$, $G(f):G(a)\to G(b)$ and from $(1)$ of 
definition~(\ref{Nat:def:natural}), $\alpha(a):F(a)\to G(a)$ and 
$\alpha(b):F(b)\to G(b)$. It follows that both arrows $G(f)\circ\alpha(a)$ 
and $\alpha(b)\circ F(f)$ are well defined arrows in \Dat\ (from $F(a)$ to 
$G(b)$), and the equality $(2)$ of definition~(\ref{Nat:def:natural}) is 
always meaningful.

\noindent
{\bf Remark}: Equality~$(2)$ of definition~(\ref{Nat:def:natural}) is commonly
visualized as: 
    \[
        \begin{tikzcd}
            a\arrow[d,swap, "f"]
            &F(a)\arrow[r, "\alpha(a)"]\arrow[d, swap,"F(f)"]
            &G(a)\arrow[d, "G(f)"]
            \\
            b
            &F(b)\arrow[r, swap, "\alpha(b)"]
            &G(b)
        \end{tikzcd}
    \]
This diagram is called the {\em naturality square} of the
natural transformation $\alpha$ relative to $f:a \to b$. Equality~$(2)$
is informally expressed by saying that {\em the naturality square commutes},
i.e. that both arrows obtained by composition along the two paths from $F(a)$ 
to $G(b)$ are equal. 

\begin{defin}\label{Nat:def:natural:component}
    Given $F,G:\Cat\to\Dat$ and $\alpha:F\Rightarrow G$, given $a\in\Cat$
    we call {\em component at $a$ of the natural transformation $\alpha$},
    the arrow $\alpha(a):F(a)\to G(a)$.
\end{defin}

\noindent
{\bf Remark}: The component $\alpha(a)$ of $\alpha$ at $a\in\Cat$ is an arrow 
in the category \Dat.

\begin{notation}\label{Nat:notation:natural:component}
    The component $\alpha(a)$ of $\alpha$ at $a$ is commonly denoted $\alpha_{a}$.
\end{notation}


