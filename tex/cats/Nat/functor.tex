In this section, given two categories \Cat\ and \Dat, we define a new 
category denoted $[\Cat,\Dat]$ and called the {\em functor category between
\Cat\ and \Dat}. Heuristically, the functor category between \Cat\ and \Dat\
is the category in which the objects are the functors between \Cat\ and \Dat,
and the arrows are the natural transformations between them.
\begin{defin}\label{Nat:def:functor:category}
    We call {\em functor category between the categories \Cat\ and \Dat}, 
    the category denoted $[\Cat,\Dat]$ and  defined by 
    $[\Cat,\Dat]=(\ob,\arr,\dom,\cod,\id,\circ)$ where:
        \begin{eqnarray*}
            (1)&\ &\ob=\{\ F\ |\ F:\Cat\to\Dat\ \}\\
            (2)&\ &\arr=\{\ (F,G,\alpha)\ |\ F\,,\,G:\Cat\to\Dat\ 
            \mbox{and}\ \alpha:F\Rightarrow G\ \}\\
            (3)&\ &\dom\,(F,G,\alpha)=F\\
            (4)&\ &\cod\,(F,G,\alpha)=G\\
            (5)&\ &\id(F)=(F,F,\iota_{F})\\
            (6)&\ &(G,H,\beta)\circ(F,G,\alpha)=(F,H,\beta\circ\alpha)
        \end{eqnarray*}
    where $(3)-(6)$ hold for all functors $F,G,H:\Cat\to\Dat$ and natural
    transformations $\alpha:F\Rightarrow G$ and $\beta:G\Rightarrow H$,
    $\iota_{F}:F\Rightarrow F$ is the identify natural transformation 
    of definition~(\ref{Nat:def:identity}) and $\beta\circ\alpha$ is the
    composition of $\beta$ and $\alpha$ of definition~(\ref{Nat:def:composition}).
\end{defin}

\noindent
{\bf Remark}: The objects of the category $[\Cat,\Dat]$ are untyped functors,
not typed functors. This makes the notations lighter without changing
the essence of the category being defined: there is an obvious one-to-one
mapping between the collections $\{\ F\ |\ F:\Cat\to\Dat\ \}$
and $\{\ (\Cat,\Dat,F)\ |\ F:\Cat\to\Dat\ \}$, given \Cat\ and \Dat.

\noindent
{\bf Remark}: In light of notation~(\ref{Nat:notation:typed:natural}) we could
regard a triple $(F,G,\alpha)$ as a notational shortcut for the typed natural 
transformation $((\Cat,\Dat,F),(\Cat,\Dat,G),\alpha)$. 
