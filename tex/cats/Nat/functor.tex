In this section, given two categories \Cat\ and \Dat, we define a new 
category denoted $[\Cat,\Dat]$ and called the {\em functor category between
\Cat\ and \Dat}. Heuristically, the functor category between \Cat\ and \Dat\
is the category in which the objects are the functors between \Cat\ and \Dat,
and the arrows are the natural transformations between them.
\begin{defin}\label{Nat:def:functor:category}
    We call {\em functor category between the categories \Cat\ and \Dat}, 
    the category denoted $[\Cat,\Dat]$ and  defined by 
    $[\Cat,\Dat]=(\ob,\arr,\dom,\cod,\id,\circ)$ where:
        \begin{eqnarray*}
            (1)&\ &\ob=\{\ F\ |\ F:\Cat\to\Dat\ \}\\
            (2)&\ &\arr=\{\ (F,G,\alpha)\ |\ F\,,\,G:\Cat\to\Dat\ 
            \mbox{and}\ \alpha:F\Rightarrow G\ \}\\
            (3)&\ &\dom\,(F,G,\alpha)=F\\
            (4)&\ &\cod\,(F,G,\alpha)=G\\
            (5)&\ &\id(F)=(F,F,\iota_{F})\\
            (6)&\ &(G,H,\beta)\circ(F,G,\alpha)=(F,H,\beta\circ\alpha)
        \end{eqnarray*}
    where $(3)-(6)$ hold for all functors $F,G,H:\Cat\to\Dat$ and natural
    transformations $\alpha:F\Rightarrow G$ and $\beta:G\Rightarrow H$,
    $\iota_{F}:F\Rightarrow F$ is the identify natural transformation 
    of definition~(\ref{Nat:def:identity}) and $\beta\circ\alpha$ is the
    composition of $\beta$ and $\alpha$ of definition~(\ref{Nat:def:composition}).
\end{defin}

\noindent
{\bf Remark}: The objects of the category $[\Cat,\Dat]$ are untyped functors,
not typed functors. This makes the notations lighter without changing
the essence of the category being defined: there is an obvious one-to-one
mapping between the collections $\{\ F\ |\ F:\Cat\to\Dat\ \}$
and $\{\ (\Cat,\Dat,F)\ |\ F:\Cat\to\Dat\ \}$, given \Cat\ and \Dat.
There is also a one-to-one mapping between a collection containing 
triples of the form $(F,G,\alpha)$ and a collection containing triples
of the form $((\Cat,\Dat,F),(\Cat,\Dat,G),\alpha)$. This explains 
why we decided to define a {\em typed natural transformation} simply
as $(F,G,\alpha)$ rather than $((\Cat,\Dat,F),(\Cat,\Dat,G),\alpha)$
in definition~(\ref{Nat:def:typed:untyped:natural}).

\begin{prop}\label{Nat:prop:functor:category}
    The functor category $[\Cat,\Dat]$ for categories \Cat,\Dat\ is a 
    category.
\end{prop}
\begin{proof}
    Given $(\ob, \arr, \dom, \cod, \id, \circ)$ of
    definition~(\ref{Nat:def:functor:category}), we need to check that this
    data satisfies condition~$(1)-(13)$ of definition~(\ref{Cat:def:category}).

    $(1)$: The collection $\ob=\{\ F\ |\ F:\Cat\to\Dat\ \}$ should be a
    collection with equality. This is the case by virtue of 
    proposition~(\ref{Fun:prop:equal}), where two functors are equal if 
    and only if they coincide on all objects and all arrows.

    $(2)$: The collection $\arr=\{\ (F,G,\alpha)\ |\ F\,,\,G:\Cat\to\Dat\ 
    \mbox{and}\ \alpha:F\Rightarrow G\ \}$ should be a collection
    with equality. This is the case by virtue of 
    proposition~(\ref{Nat:prop:equal}), where two natural transformations
    $\alpha,\beta:F\Rightarrow G$ are equal if and only if they coincide on
    all objects of \Cat. Having equality for functors and natural
    transformations, we conclude from axiom~(\ref{Cat:ax:tuple:extensional})
    that we also have equality for triples $(F,G,\alpha)$.

    $(3)$: $\dom$ should be a map $\dom:\arr\to\ob$. The equation
    $\dom(F,G,\alpha)=F$ holds for all functors $F,G:\Cat\to\Dat$ and 
    natural transformations $\alpha:F\Rightarrow G$. So $\dom$ is 
    indeed defined on the collection $\arr$ as requested. Also
    $\dom\,(f)\in\ob$.

    $(4)$: $\cod$ should be a map $\cod:\arr\to\ob$ which is the case 
    as per $(3)$.

    $(5)$: $\id$ should be a map $\id:\ob\to\arr$. The equation $\id(F)
    =(F,F,\iota_{F})$ holds for all functors $F:\Cat\to\Dat$. So $\id$
    is indeed defined on \ob\ as requested. So it remains to show that
    $(F,F,\iota_{F})\in\arr$ for all $F$, which the case since
    the identity natural transformation $\iota_{F}$ of
    definition~(\ref{Nat:def:identity}) is a natural transformation 
    $\iota_{F}:F\Rightarrow F$ as per proposition~(\ref{Nat:prop:identity}).

    $(6)$: $\circ$ should be a partial map $\circ:\arr\times\arr\to\arr$.
    From definition~(\ref{Nat:def:functor:category}), $g\circ f$ is defined
    whenever $f$ and $g$ are of the form $f=(F,G,\alpha)$ and $g=(G,H,\beta)$
    where $F,G,H:\Cat\to\Dat$, $\alpha:F\Rightarrow G$ and $\beta:G\Rightarrow H$. 
    So $g\circ f$ is defined on a sub-collection of $\arr\times\arr$ as requested.     So it remains to show that $g\circ f\in\arr$ when defined. However, 
    $g\circ f$ is defined as $(F,H,\beta\circ\alpha)$ where $\beta\circ\alpha$ 
    is the composition of natural transformation, so it remains to show that 
    $\beta\circ\alpha:F\Rightarrow H$ which follows from
    proposition~(\ref{Nat:prop:composition}).

    $(7)$: $g\circ f$ should be defined exactly when $\cod(f)=\dom(g)$. From
    definition~(\ref{Nat:def:functor:category}), $g\circ f$ is defined 
    exactly when $f$ is of the form $f=(F,G,\alpha)$ and $g$ is of the form
    $(G,H,\beta)$. Since $\cod(f)=G$ and $\dom(g)=G$, we see that
    $g\circ f$ is defined for all arrows $f,g$ for which $\cod(f)=\dom(g)$ 
    as requested.

    $(8)$: We should have $\dom(g\circ f) = \dom(f)$ when $g\circ f$ is defined.
    So let $f=(F,G,\alpha)$ and $g=(G,H,\beta)$. Then we have 
    $g\circ f=(F,H,\beta\circ\alpha)$ and consequently 
    $\dom(g\circ f)=F$ which is $\dom(f)$ as requested.

    $(9)$: We should have $\cod(g\circ f) = \cod(g)$ when $g\circ f$ is defined.
    So let $f=(F,G,\alpha)$ and $g=(G,H,\beta)$. Then we have 
    $g\circ f=(F,H,\beta\circ\alpha)$ and consequently $\cod(g\circ f)=H$ which
    is $\cod(g)$ as requested.
\end{proof}
