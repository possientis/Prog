Given two categories \Cat\ and \Dat, and a functor $F:\Cat\to\Dat$, we saw
in proposition~(\ref{Fun:prop:opposite}) that $F$ is also a functor 
$F:\Cop\to\Dop$. Now consider two functors $F,G:\Cat\to\Dat$ and a natural 
transformation $\alpha:F \Rightarrow G$. By virtue of 
notation~(\ref{Nat:notation:natural:arrow}), this means that $\alpha$ is a 
natural transformation between the functor $F$ and the functor $G$. But 
which underlying categories are we referring to? Are we considering
the functor $F$ as a functor $F:\Cat\to\Dat$ or as a functor $F:\Cop\to\Dop$,
and likewise for $G$? In other words, do we mean 
$\alpha:(\Cat,\Dat,F)\Rightarrow(\Cat,\Dat,G)$ or 
$\alpha:(\Cop,\Dop,F)\Rightarrow(\Cop,\Dop,G)$? The distinction matters
as illustrated by:

\begin{prop}
    Let $F,G:\Cat\to\Dat$ be functors between categories \Cat,\Dat\ and let 
    $\alpha:F\Rightarrow G$ be a natural transformation from $F$ to $G$ 
    (w.r. to \Cat,\Dat). Then $\alpha$ is also a natural transformation
    $\alpha:G\Rightarrow F$ (w.r. to \Cop,\Dop).
\end{prop}
\begin{proof}
TODO
\end{proof}

