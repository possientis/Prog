Given categories \Cat\ and \Dat, and a functor $F:\Cat\to\Dat$, we saw
in proposition~(\ref{Fun:prop:opposite}) that $F$ is also a functor 
$F:\Cop\to\Dop$. Now consider two functors $F,G:\Cat\to\Dat$ and a natural 
transformation $\alpha:F \Rightarrow G$. By virtue of 
notation~(\ref{Nat:notation:natural:arrow}), $\alpha$ is a natural 
transformation between the typed functors $(\Cat,\Dat,F)$ and $(\Cat,\Dat,G)$, 
which are clearly understood from the context. But the notation is nonetheless 
ambiguous as it could also indicate that $\alpha$ is a natural
transformation between $(\Cop,\Dop,F)$ and $(\Cop,\Dop,G)$ among others. 
The two statements are not identical and in fact as it turns out, we
have the equivalence:

\begin{prop}\label{Nat:prop:opposite}
    Let $F,G:\Cat\to\Dat$ be functors between categories \Cat, \Dat. Then: 
    \[
        \alpha\,:\,(\Cat,\Dat,F)\,\Rightarrow\,(\Cat,\Dat,G)
    \]
    is equivalent to:
    \[
        \alpha\,:\,(\Cop,\Dop,G)\,\Rightarrow\,(\Cop,\Dop,F)
    \]
    In other words, being a natural transformation $\alpha:F\Rightarrow G$ (w.r.
    to \Cat\ and \Dat) is equivalent to being a natural transformation 
    $\alpha:G\Rightarrow F$ (w.r. to \Cop\ and \Dop):
    \begin{eqnarray}
        \begin{tikzcd}
            \Cat \arrow[r, "F", bend left  = 50, ""{name=U, below}]
                 \arrow[r, swap, "G", bend right = 50, ""{name=D, above}]
              & \Dat
            \arrow[Rightarrow, "\,\alpha", from = U, to = D]
        \end{tikzcd}
        &\Leftrightarrow&
        \begin{tikzcd}
            \Cop \arrow[r, "G", bend left  = 50, ""{name=U, below}]
                 \arrow[r, swap, "F", bend right = 50, ""{name=D, above}]
              & \Dop
            \arrow[Rightarrow, "\,\alpha", from = U, to = D]
        \end{tikzcd}
    \end{eqnarray}
\end{prop}
\begin{proof}
    from proposition~(\ref{Cat:prop:opposite:opposite}) we have
    $(\Cop)^{op}=\Cat$ and $(\Dop)^{op}=\Dat$ and it is therefore sufficient 
    to prove the implication $\Rightarrow$. So we assume that $\alpha$ is 
    a natural transformation $\alpha:(\Cat,\Dat,F)\Rightarrow(\Cat,\Dat,G)$,
    and we need to show that it is also a natural transformation
    $\alpha:(\Cop,\Dop,G)\Rightarrow(\Cop,\Dop,F)$. Looking at 
    definition~(\ref{Nat:def:natural}), we first need to establish that
    $\alpha$ is a map $\alpha:\ob\ \Cop\to\arr\ \Dop$. However, it is 
    certainly a map $\alpha:\ob\ \Cat\to\arr\ \Dat$ by assumption and from 
    definition~(\ref{Cat:def:opposite}) we have $\ob\ \Cat=\ob\ \Cop$ and
    $\arr\ \Dat=\arr\ \Dop$. Next we need to show property $(1)$ of
    definition~(\ref{Nat:def:natural}), namely that 
    $\alpha(a):G(a)\to F(a)\ @\ \Dop$ for all $a\in\ob\ \Cop$. This
    follows immediately from the fact that $\alpha(a):F(a)\to G(a)\ @\ \Dat$
    for all $a\in\ob\ \Cat$. So it remains to show property~$(2)$ of 
    definition~(\ref{Nat:def:natural}). So we assume that $a,b\in\ob\ \Cop
    =\ob\ \Cat$ and $f:a\to b\ @\ \Cop$. We need to show that 
    $F(f)\circ\alpha(a)\ @\ \Dop = \alpha(b)\circ G(f)\ @\ \Dop$, i.e.
    the following square commutes:
    \begin{eqnarray*}@\ \Cop
        &\begin{tikzcd}
            a\arrow[d,swap, "f"]
            &G(a)\arrow[r, "\alpha(a)"]\arrow[d, swap,"G(f)"]
            &F(a)\arrow[d, "F(f)"]
            \\
            b
            &G(b)\arrow[r, swap, "\alpha(b)"]
            &F(b)
        \end{tikzcd}&@\ \Dop
    \end{eqnarray*}
    A formal proof goes as follows:
    \begin{eqnarray*}F(f)\circ\alpha(a)\ @\ \Dop 
        &=&\alpha(a)\circ F(f)\ @\ \Dat
        \ \leftarrow\ \mbox{def.~(\ref{Cat:def:opposite})}\\
        \mbox{$(2)$ of def.~(\ref{Nat:def:natural})},\ 
        \alpha:F\Rightarrow G,\ f:b\to a\ @\ \Cat\ \to\ 
        &=&G(f)\circ\alpha(b)\ @\ \Dat\\
        \mbox{def.~(\ref{Cat:def:opposite})}\ \to\ 
        &=&\alpha(b)\circ G(f)\ @\ \Dop
    \end{eqnarray*}
\end{proof}

\noindent
{\bf Remark}: proving that the above naturality square in relation to \Cop\ and
\Dop\ commutes amounts to proving the same square commutes, in relation to \Cat\ 
and \Dat\ after all the arrows have been reversed, and this follows from 
$\alpha:F\Rightarrow G$.


Proposition~(\ref{Nat:prop:opposite}) illustrates the fact that the knowledge
of a natural transformation $\alpha$ by itself does not tell us what its 
intended domain and codomain are. This situation is similar to that encountered
with untyped functions and untyped functors. In line with
definitions~(\ref{Cat:def:typed:untyped:function}) 
and~(\ref{Fun:def:typed:untyped:functor}), we define:
\begin{defin}\label{Nat:def:typed:untyped:natural}
    Given a natural transformation $\alpha:(\Cat,\Dat,F)\Rightarrow(\Cat,\Dat,G)$
    where $(\Cat,\Dat,F)$ and $(\Cat,\Dat,G)$ are typed functors, $\alpha$ is 
    called the {\em untyped natural transformation} while
    $(F,G,\alpha)$ is called the {\em typed} natural transformation.
\end{defin}

\noindent
{\bf Remark}: It may appear surprising that a typed natural transformation is 
defined as a triple $(F,G,\alpha)$ which does not contain the data allowing 
us to recover the categories \Cat\ and \Dat. We are adopting this simpler
definition as opposed to the expected $((\Cat,\Dat,F),(\Cat,\Dat,G),\alpha)$
because it allows us to define the functor category of
definition~(\ref{Nat:def:functor:category}) while keeping notations light.


