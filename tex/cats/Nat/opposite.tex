Given categories \Cat\ and \Dat, and a functor $F:\Cat\to\Dat$, we saw
in proposition~(\ref{Fun:prop:opposite}) that $F$ is also a functor 
$F:\Cop\to\Dop$. Now consider two functors $F,G:\Cat\to\Dat$ and a natural 
transformation $\alpha:F \Rightarrow G$. By virtue of 
notation~(\ref{Nat:notation:natural:arrow}), $\alpha$ is a natural 
transformation between the typed functors $(\Cat,\Dat,F)$ and $(\Cat,\Dat,G)$, 
which are clearly understood from the context. But the notation is nonetheless 
ambiguous as it could also indicate that $\alpha$ is a natural
transformation between $(\Cop,\Dop,F)$ and $(\Cop,\Dop,G)$ among others. 
The two statements are not identical and in fact as it turns out, we
have the equivalence:

\begin{prop}
    Let $F,G:\Cat\to\Dat$ be functors between categories \Cat, \Dat. Then: 
    \[
        \alpha\,:\,(\Cat,\Dat,F)\,\Rightarrow\,(\Cat,\Dat,G)
    \]
    is equivalent to:
    \[
        \alpha\,:\,(\Cop,\Dop,G)\,\Rightarrow\,(\Cop,\Dop,F)
    \]
    In other words, being a natural transformation $\alpha:F\Rightarrow G$ (w.r.
    to \Cat\ and \Dat) is equivalent to being a natural transformation 
    $\alpha:G\Rightarrow F$ (w.r. to \Cop\ and \Dop):
    \begin{eqnarray}
        \begin{tikzcd}
            \Cat \arrow[r, "F", bend left  = 50, ""{name=U, below}]
                 \arrow[r, swap, "G", bend right = 50, ""{name=D, above}]
              & \Dat
            \arrow[Rightarrow, "\,\alpha", from = U, to = D]
        \end{tikzcd}
        &\Leftrightarrow&
        \begin{tikzcd}
            \Cop \arrow[r, "G", bend left  = 50, ""{name=U, below}]
                 \arrow[r, swap, "F", bend right = 50, ""{name=D, above}]
              & \Dop
            \arrow[Rightarrow, "\,\alpha", from = U, to = D]
        \end{tikzcd}
    \end{eqnarray}
\end{prop}
\begin{proof}
TODO
\end{proof}

