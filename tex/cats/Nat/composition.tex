\begin{defin}\label{Nat:def:composition}
    Let $\alpha:F\Rightarrow G$ and $\beta:G\Rightarrow H$ be natural
    transformations where $F,G$ and $H$ are functors $F,G,H:\Cat\to\Dat$
    and \Cat,\ \Dat\ are categories. We call {\em composition of $\beta$
    and $\alpha$} the natural transformation $\beta\circ\alpha:F\Rightarrow H$
    with\,:
        \begin{eqnarray*}
            (1)&\ &(\beta\circ\alpha)(a)=\beta(a)\,\circ\,\alpha(a)
        \end{eqnarray*}
    where $(1)$ holds for all $a\in\Cat$.
\end{defin}

{\bf Remark}: If $\alpha:F\Rightarrow G$ and $\beta:G\Rightarrow H$ are 
natural transformations then for all $a\in\Cat$ we have $\alpha(a):F(a)\to 
G(a)$ and $\beta(a):G(a)\to H(a)$, and $\beta(a)\circ\alpha(a)$ is therefore 
a well-defined arrow in \Dat.

\begin{prop}\label{Nat:prop:composition}
    Let $\alpha:F\Rightarrow G$ and $\beta:G\Rightarrow H$ be natural
    transformations where $F,G$ and $H$ are functors $F,G,H:\Cat\to\Dat$
    and \Cat,\ \Dat\ are categories. Then $\beta\circ\alpha$ is indeed
    a natural transformation $\beta\circ\alpha:F\Rightarrow H$.
\end{prop}
\begin{proof}
    We need to check that $\beta\circ\alpha$ is a map defined on
    $\ob\ \Cat$ which satisfies $(1)$ and $(2)$ of 
    definition~(\ref{Nat:def:natural}). As noted above, since
    $\alpha(a):F(a)\to G(a)$ and $\beta(a):G(a)\to H(a)$, the expression
    $\beta(a)\circ\alpha(a)$ is a well-defined arrow in \Dat\ and 
    $(\beta\circ\alpha)(a)$ is thus well-defined for all
    $a\in\Cat$. So $\beta\circ\alpha$ is a map defined on $\ob\ \Cat$.
    
    $(1)$: We need to check that $(\beta \circ \alpha)(a) : F(a) \to H(a)$
    for all $a\in\Cat$, which is clear since $(\beta\circ\alpha)(a)=
    \beta(a)\circ\alpha(a)$, $\alpha(a):F(a)\to G(a)$ and 
    $\beta(a):G(a)\to H(a)$.

    $(2)$: We need $H(f)\circ(\beta\circ\alpha)(a)=
    (\beta\circ\alpha)(b)\circ F(f)$ for all $a,b\in\Cat$ and $f:a\to b$,
    which goes as follows:
        \begin{eqnarray*}H(f)\circ(\beta\circ\alpha)(a)
            &=&H(f)\circ(\beta(a)\circ\alpha(a))
            \ \leftarrow\ \mbox{$(1)$ of def.~(\ref{Nat:def:composition})}\\
            \mbox{$\circ$ associative in \Dat}\ \to\ 
            &=&(H(f)\circ\beta(a))\circ\alpha(a)\\
            \mbox{$(2)$ of def.~(\ref{Nat:def:natural})}, \beta:G\Rightarrow H
            \ \to\
            &=&(\beta(b)\circ G(f))\circ\alpha(a)\\
            \mbox{$\circ$ associative in \Dat}\ \to\ 
            &=&\beta(b)\circ (G(f)\circ\alpha(a))\\
            \mbox{$(2)$ of def.~(\ref{Nat:def:natural})}, \alpha:F\Rightarrow G
            \ \to\
            &=&\beta(b)\circ (\alpha(b)\circ F(f))\\
            \mbox{$\circ$ associative in \Dat}\ \to\ 
            &=&(\beta(b)\circ\alpha(b))\circ F(f)\\
            \mbox{$(1)$ of def.~(\ref{Nat:def:composition})}\ \to\ 
            &=&(\beta\circ\alpha)(b)\circ F(f)\\
        \end{eqnarray*}
\end{proof}


\noindent
{\bf Remark}: Showing that $\beta\circ\alpha$ is a natural transformation is
essentially about proving the equality $H(f)\circ\beta(a)\circ\alpha(a)=
\beta(b)\circ\alpha(b)\circ F(f)$, where we no longer care about brackets
as composition is associative in \Dat. Informally, this equality amounts
to saying that the big rectangle below commutes:
    \[
        \begin{tikzcd}
            a\arrow[d,swap, "f"]
            &F(a)\arrow[r, "\alpha(a)"]\arrow[d, swap,"F(f)"]
            &G(a)\arrow[r,"\beta(a)"]\arrow[d, "G(f)"]
            &H(a)\arrow[d, "H(f)"]
            \\
            b
            &F(b)\arrow[r, swap, "\alpha(b)"]
            &G(b)\arrow[r, swap, "\beta(b)"]
            &H(b)
        \end{tikzcd}
    \]
We formally proved the commutativity of this rectangle using the
commutativity of the individual squares. Forgetting about associativity details:
    \[
        H(f)\circ\beta(a)\circ\alpha(a)=\beta(b)\circ G(f)\circ\alpha(a)
        =\beta(b)\circ \alpha(b)\circ F(f)
    \]
We obtain a proof which is a lot simpler.

