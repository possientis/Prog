Now that we have a notion of composition between natural transformations,
we would like to argue that this operation is associative. However, the 
statement is not meaningful unless we are very clear as to what it means
for two natural transformations to be equal. Now if $F,G:\Cat\to\Dat$ are
two functors between categories \Cat\ and \Dat, and if 
$\alpha,\beta:F\Rightarrow G$ are natural transformations between the 
typed functors $(\Cat,\Dat,F)$ and $(\Cat,\Dat,G)$, then
$\alpha,\beta:\ob\ \Cat\to\arr\ \Dat$, i.e. both $\alpha$ and $\beta$
are maps from the collection $\ob\ \Cat$ to the collection $\arr\ \Dat$.
Furthermore since \Dat\ is a category, from $(2)$ of 
definition~(\ref{Cat:def:category}) there is a clear notion of equality 
defined on $\arr\ \Dat$. Hence given $a\in\ob\ \Cat$, it is perfectly 
meaningful to ask whether $\alpha(a)=\beta(a)$. In fact, using the
extentionality axiom~(\ref{Cat:ax:map:extensional}) we have: 
\begin{prop}\label{Nat:prop:equal}
    Let $\alpha,\beta:F\Rightarrow G$ be two natural transformations between
    functors $F,G:\Cat\to\Dat$ where \Cat\ and \Dat\ are categories, such that:
        \[
            \forall a\in\ob\ \Cat\ ,\ \alpha(a)=\beta(a)
        \]
    Then $\alpha=\beta$, i.e. the two natural transformations $\alpha$
    and $\beta$ are equal.
\end{prop}
\begin{proof}
TODO
\end{proof}
