Let $F,G:\Cat\to\Dat$ and $T:\Dat\to\Eat$ be functors where \Cat, \Dat\ and \Eat\ 
are categories, and let $\alpha:F\Rightarrow G$ be a natural 
transformation. Then the situation is as follows: 
    \[
        \begin{tikzcd}
            & \Cat \arrow[r, "F", bend left  = 50, ""{name=U, below}]
                 \arrow[r, swap, "G", bend right = 50, ""{name=D, above}]
            & \Dat 
            \arrow[Rightarrow, "\,\alpha", from = U, to = D]
            \arrow[r,"T"]
            & \Eat
        \end{tikzcd}
    \] 
In particular the functors $T\circ F:\Cat\to\Eat$ and $T\circ G:\Cat\to\Eat$
are well-defined by virtue of definition~(\ref{Fun:def:composition}). Furthermore, 
since $\alpha$ is a natural transformation $\alpha:F\Rightarrow G$,
using definition~(\ref{Nat:def:natural}), for all $a\in\Cat$ we have 
$\alpha(a):F(a)\to G(a)$. $T$ being a functor, it follows that 
$T(\,\alpha(a)\,) : T(\,F(a)\,)\to T(\,G(a)\,)$. So the question arises 
as to whether the expression $T(\,\alpha(a)\,)$ for all $a\in\Cat$ defines 
a natural transformation between $T\circ F$ and $T\circ G$. As we shall see the 
answer is 'yes' so we shall define a new natural transformation denoted 
$T\alpha$:
    \[
        \begin{tikzcd}
            & \Cat \arrow[r, "F", bend left  = 50, ""{name=U, below}]
                 \arrow[r, swap, "G", bend right = 50, ""{name=D, above}]
            & \Dat
            \arrow[Rightarrow, "\,\alpha", from = U, to = D]
            \arrow[r,"T"]
            & \Eat
        \end{tikzcd}
        \ \to\ 
        \begin{tikzcd}
            \Cat \arrow[r, "T\circ F", bend left  = 50, ""{name=U, below}]
                 \arrow[r, swap, "T\circ G", bend right = 50, ""{name=D, above}]
            & \Eat
            \arrow[Rightarrow, "\,T\alpha", from = U, to = D]
        \end{tikzcd}
    \]
\begin{defin}\label{Nat:def:leftmul}
    Let $F,G:\Cat\to\Dat$ and $T:\Dat\to\Eat$ be functors where \Cat, \Dat\ 
    and \Eat\ are categories, and let $\alpha:F\Rightarrow G$ be a natural 
    transformation. We denote $T\alpha: T\circ F\Rightarrow T\circ G$ the 
    natural transformation defined by, given $a\in\ob\ \Cat$:
        \begin{eqnarray*}
            (1)&\ &(T\alpha)(a) = T(\,\alpha(a)\,)
        \end{eqnarray*}
\end{defin}  

\begin{prop}\label{Nat:prop:leftmul}
    $T\alpha$ of~(\ref{Nat:def:leftmul}) is a natural transformation 
    $T\alpha:T\circ F\Rightarrow T\circ G$.
\end{prop}
\begin{proof}
    We need to show that $T\alpha$ is a map $T\alpha : \ob\ \Cat\to\arr\ \Eat$
    which satisfies properties~$(1)$ and~$(2)$ of 
    definition~(\ref{Nat:def:natural}). If we explicit the functor $T$ as the 
    ordered pair $T=(T_{0},T_{1})$ (definition~(\ref{Fun:def:functor})), it is
    implicit in definition~(\ref{Nat:def:leftmul}) that given $a\in\ob\ \Cat$,
    $(T\alpha)(a)$ is defined as $T_{1}(\,\alpha(a)\,)$. Since 
    $\alpha:F\Rightarrow G$, in particular $\alpha$ is a map $\alpha:\ob\ \Cat
    \to\arr\ \Dat$ while $T_{1}$ is a map $T_{1}:\arr\ \Dat\to\arr\ \Eat$. It 
    follows that $T_{1}(\,\alpha(a)\,)$ is a well defined arrow in \Eat\ and we
    see that $T\alpha:\ob\ \Cat\to\arr\ \Eat$.

    $(1)$: We need to show that $(T\alpha)(a): (T\circ F)(a) \to (T\circ G)(a)$
    for all $a\in\Cat$. This is the same as showing that $T(\,\alpha(a)\,) :
    T(\,F(a)\,)\to T(\,G(a)\,)$, which is the case since $\alpha(a):F(a)\to G(a)$ 
    for all $a\in\Cat$ and $T$ is a functor.

    $(2)$: We need to show that the naturality square commutes, namely that
    given $a,b\in\Cat$ and $f:a\to b$, we have $(T\circ G)(f)\circ(T\alpha)(a)
    =(T\alpha)(b)\circ(T\circ F)(f)$.
    \[
        \begin{tikzcd}
            a\arrow[d,swap, "f"]
            &(T\circ F)(a)\arrow[r, "(T\alpha)(a)"]\arrow[d, swap,"(T\circ F)(f)"]
            &(T\circ G)(a)\arrow[d, "(T\circ G)(f)"]
            \\
            b
            &(T\circ F)(b)\arrow[r, swap, "(T\alpha)(b)"]
            &(T\circ G)(b)
        \end{tikzcd}
    \]
    However, this equality is the same as $T(\,G(f)\,)\circ T(\,\alpha(a)\,)
    =T(\,\alpha(b)\,)\circ T(\,F(f)\,)$, and is simply a lifting by the 
    functor $T$ of the equality expressing the naturality of $\alpha$, 
    associated with $a,b\in\Cat$ and $f:a\to b$.
    \[
        \begin{tikzcd}
            a\arrow[d,swap, "f"]
            &T(\,F(a)\,)\arrow[r, "T(\,\alpha(a)\,)"]\arrow[d, swap,"T(\,F(f)\,)"]
            &T(\,G(a)\,)\arrow[d, "T(\,G(f)\,)"]
            \\
            b
            &T(\,F(b)\,)\arrow[r, swap, "T(\,\alpha(b)\,)"]
            &T(\,G(b)\,).
        \end{tikzcd}
    \]
    However, we crucially need $T$ to be a functor for this square to commute:
    \begin{eqnarray*}T(\,G(f)\,)\circ T(\,\alpha(a)\,)
        &=&T(\,G(f)\circ\alpha(a)\,)
        \ \leftarrow\ \mbox{$(5)$ of def.~(\ref{Fun:def:functor})}\\
        \mbox{$(2)$ of def.~(\ref{Nat:def:natural})}\ \to\ 
        &=&T(\,\alpha(b)\circ F(f)\,)\\
        \mbox{$(5)$ of def.~(\ref{Fun:def:functor})}\ \to\ 
        &=&T(\,\alpha(b)\,)\circ T(\,F(f)\,)
    \end{eqnarray*}
\end{proof}
