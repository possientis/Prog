\begin{defin}\label{Nat:def:isomorphsim}
    Let $F,G:\Cat\to\Dat$ be functors where \Cat\ and \Dat\ are categories.
    We say that $\alpha:F\Rightarrow G$ is a {\em natural isomorphism} 
    \ifand\ the arrow $(F,G,\alpha)$ of the functor category
    $[\Cat,\Dat]$ of definition~(\ref{Nat:def:functor:category}) 
    is an isomorphism.
\end{defin}

\begin{prop}\label{Nat:prop:isomorphism}
    Let $F,G:\Cat\to\Dat$ be functors where \Cat\ and \Dat\ are categories.
    Then $\alpha:F\Rightarrow G$ is a natural isomorphism \ifand\ there
    exists a natural transformation $\beta:G\Rightarrow F$ such that
    $\beta\circ\alpha=\iota_{F}$ and $\alpha\circ\beta=\iota_{G}$.
\end{prop}
\begin{proof}
    We first show the {\em only if} part. So we assume that 
    $\alpha:F\Rightarrow G$ is a natural isomorphism. From 
    definition~(\ref{Nat:def:isomorphsim}), it follows that
    $f=(F,G,\alpha)$ is an isomorphism of the functor category
    $[\Cat,\Dat]$. Thus, from definition~(\ref{Cat:def:isomorphism}), 
    the arrow $f$ has an inverse. Hence there exists some
    $g=(F',G',\beta)$ arrow in $[\Cat,\Dat]$ which is both a left and
    right-inverse of $f$. Since $f:F\to G$ we must have $g:G\to F$ and
    consequently $g=(G,F,\beta)$ with $\beta:G\Rightarrow F$. So we
    have found a natural transformation $\beta:G\Rightarrow F$ and 
    it remains to show that $\beta\circ\alpha=\iota_{F}$ and 
    $\alpha\circ\beta=\iota_{G}$. Since $g$ is a left-inverse of $f$,
    we have: 
        \begin{eqnarray*}(F,F,\iota_{F})
            &=&\id(F)\ \leftarrow\ 
            \mbox{$(5)$ of def.~(\ref{Nat:def:functor:category})}\\
            \mbox{$g$ left-inverse of $f$}\ \to\ 
            &=&g\circ f\\
            &=&(G,F,\beta)\,\circ\,(F,G,\alpha)\\
            \mbox{$(6)$ of def.~(\ref{Nat:def:functor:category})}\ \to\ 
            &=&(F,F,\beta\circ\alpha)
        \end{eqnarray*}
    and it follows that $\beta\circ\alpha=\iota_{F}$ as requested. Since $g$ is
    a right-inverse of $f$:
        \begin{eqnarray*}(G,G,\iota_{G})
            &=&\id(G)\ \leftarrow\ 
            \mbox{$(5)$ of def.~(\ref{Nat:def:functor:category})}\\
            \mbox{$g$ right-inverse of $f$}\ \to\ 
            &=&f\circ g\\
            &=&(F,G,\alpha)\,\circ\,(G,F,\beta)\\
            \mbox{$(6)$ of def.~(\ref{Nat:def:functor:category})}\ \to\ 
            &=&(G,G,\alpha\circ\beta)
        \end{eqnarray*}
    and it follows that $\alpha\circ\beta=\iota_{G}$ which completes
    the proof of the {\em only if} part.
    
    We now show the {\em if} part. So we assume the existence of a 
    natural transformation $\beta:G\Rightarrow F$ such that 
    $\beta\circ\alpha=\iota_{F}$ and $\alpha\circ\beta=\iota_{G}$.
    We need to show that $\alpha$ is a natural isomorphim, i.e.
    that $f=(F,G,\alpha)$ has an inverse in the functor category
    $[\Cat,\Dat]$. Consider the arrow $g=(G,F,\beta)$. From
    definition~(\ref{Nat:def:functor:category}), this is an
    arrow of $[\Cat,\Dat]$ with domain $G$ and codomain $F$.
    We can complete the proof by showing that $g$ is an inverse of
    $f$, i.e. that $g\circ f=\id(F)$ and $f\circ g=\id(G)$. 
    This follows easily reversing the above derivations with
    $\beta\circ\alpha=\iota_{F}$ and $\alpha\circ\beta=\iota_{G}$.
\end{proof}

\begin{prop}\label{Nat:prop:isomorphism:component}
    Let $F,G:\Cat\to\Dat$ be functors where \Cat\ and \Dat\ are categories.
    Then $\alpha:F\Rightarrow G$ is a natural isomorphism \ifand\ for
    all $a\in\Cat$, the component $\alpha(a):F(a)\to G(a)$ is an isomorphism.
\end{prop}
\begin{proof}
    We first show the {\em only if} part. So we assume that $\alpha$
    is a natural isomorphism and we need to show that $\alpha(a):F(a)\to G(a)$ 
    is an isomorphism for all $a\in\Cat$. Using
    proposition~(\ref{Nat:prop:isomorphism}), there exists a natural
    transformation $\beta:G\Rightarrow F$ such that
    $\beta\circ\alpha=\iota_{F}$ and $\alpha\circ\beta=\iota_{G}$.
    It follows that $\beta(a)$ is the inverse of $\alpha(a)$, since:
        \begin{eqnarray*}\beta(a)\circ\alpha(a)
            &=&(\beta\circ\alpha)(a)\ \leftarrow\ 
            \mbox{$(1)$ of def.~(\ref{Nat:def:composition})}\\
            &=&(\iota_{F})(a)\\
            \mbox{$(1)$ of def.~(\ref{Nat:def:identity})}\ \to\ 
            &=&\id(\,F(a)\,)
        \end{eqnarray*}
    and:
        \begin{eqnarray*}\alpha(a)\circ\beta(a)
            &=&(\alpha\circ\beta)(a)\ \leftarrow\ 
            \mbox{$(1)$ of def.~(\ref{Nat:def:composition})}\\
            &=&(\iota_{G})(a)\\
            \mbox{$(1)$ of def.~(\ref{Nat:def:identity})}\ \to\ 
            &=&\id(\,G(a)\,)
        \end{eqnarray*}
    Since $\alpha(a):F(a)\to G(a)$ has an inverse, it is an isomophism
    as requested.

    We now show the {\em if} part. So we assume that $\alpha:F\Rightarrow G$
    is a natural transformation such that $\alpha(a):F(a)\to G(a)$ is an
    isomorphism for all $a\in\Cat$. We need to show that $\alpha$ is in
    fact a natural isomorphism. From proposition~(\ref{Nat:prop:isomorphism})
    it is sufficient to prove the existence of a natural transformation
    $\beta:G\Rightarrow F$ such that $\beta\circ\alpha=\iota_{F}$ and 
    $\alpha\circ\beta=\iota_{G}$. By assumption, given $a\in\Cat$, there 
    exists $\beta(a):G(a)\to F(a)$ which is an inverse of $\alpha(a)$. In 
    fact using proposition~(\ref{Cat:prop:inverse:unique}) this inverse is 
    unique and collecting all $\beta(a)$'s for $a\in\ob\ \Cat$ we obtain a map
    $\beta:\ob\ \Cat\to\arr\ \Dat$. We do not know at this stage whether
    $\beta$ is a natural transformation $\beta:G\Rightarrow F$ but let us
    assume that it is. Then we have:
        \begin{eqnarray*}(\beta\circ\alpha)(a)
            &=&\beta(a)\circ\alpha(a)\ \leftarrow\ 
            \mbox{$(1)$ of def.~(\ref{Nat:def:composition})}\\
            \mbox{$\beta(a)$ left-inverse of $\alpha(a)$}\ \to\ 
            &=&\id(\,F(a)\,)\\
            \mbox{$(1)$ of def.~(\ref{Nat:def:identity})}\ \to\ 
            &=&(\iota_{F})(a)
        \end{eqnarray*}
    and:
        \begin{eqnarray*}(\alpha\circ\beta)(a)
            &=&\alpha(a)\circ\beta(a)\ \leftarrow\ 
            \mbox{$(1)$ of def.~(\ref{Nat:def:composition})}\\
            \mbox{$\beta(a)$ right-inverse of $\alpha(a)$}\ \to\ 
            &=&\id(\,G(a)\,) \\
            \mbox{$(1)$ of def.~(\ref{Nat:def:identity})}\ \to\ 
            &=&(\iota_{G})(a)
        \end{eqnarray*}
    These equalities being true for all $a\in\ob\ \Cat$, from 
    proposition~(\ref{Nat:prop:equal}) we obtain the equalities
    of natural transformations $\beta\circ\alpha=\iota_{F}$ and
    $\alpha\circ\beta=\iota_{G}$ as requested.

    It remains to show that $\beta:\ob\ \Cat\to\arr\ \Dat$ is indeed
    a natural transformation $\beta:G\Rightarrow F$, i.e. that properies 
    $(1)$ and~$(2)$ of definition~(\ref{Nat:def:natural}) are satisfied. 
    We already know that $\beta(a):G(a)\to F(a)$, being the inverse
    of $\alpha(a):F(a)\to G(a)$. So $(1)$ is done, and it remains to
    show property $(2)$. So let $f:a\to b$ where $a,b\in\Cat$.
    We need to show the equality $F(f)\circ\beta(a)=\beta(b)\circ G(f)$:
    \[
        \begin{tikzcd}
            a\arrow[d,swap, "f"]
            &F(a)\arrow[r, "\alpha(a)"]\arrow[d, swap,"F(f)"]
            &G(a)\arrow[r,"\beta(a)"]\arrow[d, "G(f)"]
            &H(a)\arrow[d, "H(f)"]
            \\
            b
            &F(b)\arrow[r, swap, "\alpha(b)"]
            &G(b)\arrow[r, swap, "\beta(b)"]
            &H(b)
        \end{tikzcd}
    \]
    TODO
\end{proof}

