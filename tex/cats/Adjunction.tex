\documentclass[]{article}
\usepackage{tikz-cd}
\usepackage{graphicx}
\usepackage{amsmath, amssymb}
\usepackage{amsthm,thmtools}
\usepackage{hyperref, cleveref}
\newcommand{\dotarrow}{% to be used in math mode...
	\mathrel{\ooalign{\hss\raise.65ex\hbox{\scalebox{1.25}{\normalfont .}}%
	\kern0.35ex\hss\cr$\longrightarrow$}}}
		
\declaretheoremstyle[postheadspace=\newline]{def}
\declaretheorem[style=def,name=Definition]{definition}
\declaretheorem[name=Lemma]{lemma}

%opening
\title{Adjunction definitions}
\author{Cristóvão Gomes Ferreira}

\begin{document}

\newcommand{\blackqed}{
	\renewcommand{\qedsymbol}{$\blacksquare$}
	\qed
	\renewcommand{\qedsymbol}{$\square$}
}

\maketitle

\begin{abstract}
We are going to prove that three definitions of adjunction are equivalent.
\end{abstract}

\section{Definitions}

\subsection{Prelude}

For this lecture, we assume the following:

\begin{itemize}
	\item $\mathcal{C}$, $\mathcal{D}$ or actually any calligraphed letter denotes a category.
	\item $Hom(x,y)$ is the set of arrows between objects $x$ and $y$.
	\item $I$ is the identity functor ($Ic=c$, $If=f$).
	\item Basic category theory identities are trivial. \footnote{Namely, if F is a functor, $F(g \circ h) = Fg \circ Fh$, and $F id_a = id_{Fa}$}
\end{itemize}

\subsection{Adjunction}

Let $F$ and $G$ such that:

\begin{tikzcd}
  \mathcal{D} \arrow[r, shift left, "F"] & \mathcal{C} \arrow[l, shift left, "G"]
\end{tikzcd}

We say that $F$ is left adjoint\footnote{or $F$ has a right adjoint $G$, or $G$ is right adjoint to $F$, or other trivially equivalent expression.} to $G$, written $F \dashv G$, when some rules apply.

There are several equivalent definitions of the rules in this relation, we will prove the equivalence between two of them.

\begin{definition}[Isomorphism collection]\label{def:isocollection}

$F \dashv G$ if there is a collection of functions $\phi_{x,y} \colon Hom(F x, y) \to Hom(x, G y)$ that forms a natural isomorphism, i.e.:
	
\begin{itemize}
	\item $\phi_{x,y}$ has an inverse $\phi^{-1}_{x,y}$;
	\item The following diagrams commute:
\end{itemize}
\[
\begin{tikzcd}
    Hom(Fa, b) \arrow[r, "\phi_{a,b}"] \arrow[d, "{Hom(Ff,b)}"] & Hom(a, Gb) \arrow[d, "{Hom(f, Gb)}"] & Hom(Fa, b) \arrow[r, "\phi_{a,b}"] \arrow[d, "{Hom(Fa,f)}"] & Hom(a, Gb) \arrow[d, "{Hom(a, Gf)}"] 
    \\ Hom(Fa', b) \arrow[r, "\phi_{a',b}"] & Hom(a', Gb) & Hom(Fa, b') \arrow[r, "\phi_{a,b'}"] & Hom(a, Gb')
\end{tikzcd}
\]
\end{definition}
\begin{definition}[Universal morphism]\label{def:universal}
	
	$F \dashv G$ if there is a natural transformation $\epsilon \colon FG \dotarrow I$ such that each $epsilon_c \colon FGc \to c$ is universal from $Fc$ to $c$, i.e., for all morphisms $f$ there is a unique $g$ such that:
\end{definition}
\[
\begin{tikzcd}
	d \arrow[d, dotted, "\exists! g"] & F d \arrow[d, dotted, "F g"] \arrow[rd, "f"] \\ Gc & FGc \arrow[r, "\epsilon_c"] & c
\end{tikzcd}
\]
\begin{definition}[Natural units]\label{def:natural}
	
$F \dashv G$ if there are two natural transformations
$\eta \colon I \dotarrow GF$ and $\epsilon \colon FG \dotarrow I$ such that $G\epsilon \circ \eta_G = I$, $\epsilon_F \circ F\eta = I$.
\end{definition}

\section{Main section}

We are going to prove that all these definitions are equivalent, by proving $ \cref{def:isocollection} \Rightarrow \cref{def:universal} \Rightarrow \cref{def:natural} \Rightarrow \cref{def:isocollection}$

\subsection{Proofs}
\subsubsection{$\ref{def:isocollection} \Rightarrow \ref{def:universal}$}
We have a collection of isomorphisms $\phi_{x,y}$, such that the naturality squares in \ref{def:isocollection} commute.

We want to define a natural transformation $\epsilon \colon FG \dotarrow I$, such that for each arrow $f \colon Fd \to c$ there exists only one $g \colon d \to G c$ such that $\epsilon_c \circ F g = f$.

\[
\begin{tikzcd}
d \arrow[d, dotted, "\exists! g"] & F d \arrow[d, dotted, "F g"] \arrow[rd, "f"] \\ Gc & FGc \arrow[r, "\epsilon_c"] & c
\end{tikzcd}
\]

We start by taking the arrow $f \colon Fd \to c$

\[
\begin{tikzcd}
d & F d \arrow[rd, "f"] \\ Gc & FGc & c
\end{tikzcd}
\]

A motivation for defining $\epsilon$, is recognizing that $\epsilon_c \in Hom(FGc, c)$ and $\phi_{Gc,c}$ defines an isomorphism between $Hom(Gc,Gc)$ and $Hom(FGc,c)$. So $\phi_{Gc,c}(\epsilon_c) \in Hom(Gc,Gc)$.

The canonical arrow to take from $Hom(Gc,Gc)$ is $id_{Gc}$, so we'll define $\epsilon_c = \phi_{Gc,c}^{-1}(id_{Gc})$, and see how that goes (the proof that $\epsilon$ is a natural transformation is \cref{lem:phi_inv_nat}).

\[
\begin{tikzcd}
d & F d \arrow[rd, "f"] \\ Gc & FGc \arrow[r, swap, "\phi_{Gc,c}^{-1}(id_{Gc})"] & c
\end{tikzcd}
\]

The motivation for defining the unique $g$ for each $f$ is actually quite intuitive: $g$ is just $\phi_{d,c}(f)$.

\[
\begin{tikzcd}
d \arrow[d, swap, "\phi_{d,c}(f)"] & F d  \arrow[d, swap, "F (\phi_{d,c}(f))"] \arrow[rd, "f"] \\ Gc & FGc \arrow[r, swap, "\phi_{Gc,c}^{-1}(id_{Gc})"] & c
\end{tikzcd}
\]


We actually have to prove that this diagram commutes, but note that, by \cref{lem:phi_inv}, $\phi_{Gc,c}^{-1}(id_{Gc}) \circ F (\phi_{d,c}(f)) = \phi_{d,c}^{-1}(\phi_{d,c}(f)) = f$, so it does!

At last, we have to prove the uniqueness of our $g$, but note that, if $\phi_{d,c}^{-1}(g') = \phi_{Gc,c}^{-1}(id_{Gc}) \circ F g' = f = \phi_{d,c}^{-1}(\phi_{d,c}(f))$ then $g' = \phi_{d,c}(f)$ by injectiveness of $\phi_{d,c}^{-1}$.

And with that we're done. \qed

\subsubsection{$\ref{def:universal} \Rightarrow \ref{def:natural}$}
We have a natural transformation $\epsilon \colon FG \dotarrow I$, such that for each arrow $f \colon Fd \to c$ there exists only one $g \colon d \to G c$ such that $\epsilon_c \circ F g = f$.

\[
\begin{tikzcd}
d \arrow[d, dotted, "\exists! g"] & F d \arrow[d, dotted, "F g"] \arrow[rd, "f"] \\ Gc & FGc \arrow[r, "\epsilon_c"] & c
\end{tikzcd}
\]

We want to get two natural transformations $\epsilon' \colon FG \dotarrow I$, $\eta \colon I \dotarrow GF$, such that $G\epsilon' \circ \eta_G = I$, $\epsilon'_F \circ F\eta = I$.

Let us define $\epsilon' = \epsilon$ (and stop using the $'$ notation) and $\eta_d$ the unique $g$ for $f = id_{Fd}$. We now have to prove that:
\begin{enumerate}
	\item \label{epsilon_id} $\epsilon_F \circ F\eta = I$
	\item \label{eta_id} $G\epsilon \circ \eta_G = I$
	\item \label{eta_natural} $\eta$ is a natural transformation
\end{enumerate}

\Cref{epsilon_id} is immediate, by definition of $\eta$.

\blackqed

We prove \cref{eta_id} by noting that

\[
\begin{aligned}
  \epsilon_a \circ F(G \epsilon_a \circ \eta_{Ga}) =
  \\ \epsilon_a \circ FG \epsilon_a \circ F\eta_{Ga} = && \text{(by naturality of $\epsilon$)}
  \\ \epsilon_a \circ \epsilon_{FGa} \circ F \eta_{Ga} = && \text{(by \cref{epsilon_id})}
  \\ \epsilon_a \circ id_{FG a} = 
  \\ \epsilon_a = 
  \\ \epsilon_{a} \circ id_{FGa} = 
  \\ \epsilon_{a} \circ F id_{Ga}
\end{aligned}
\]
 
 hence, by \cref{lem:epsilon_unique} on $G\epsilon_a \circ \eta_{Ga}$ and $id_{Ga}$, we get $G\epsilon_a \circ \eta_{Ga} = id_{Ga}$. 

 \blackqed
 
 
To prove \cref{eta_natural} we need to prove that
\[
\begin{tikzcd}
a \arrow[r, "\eta_a"] \arrow[d, "{f}"] 
& GF a \arrow[d,"GF f"] \\
b \arrow[r, "\eta_b"] & GF b
\end{tikzcd}
\]
commutes, i.e., $GF f \circ \eta_a = \eta_b \circ f$. We prove this by resorting to \Cref{lem:epsilon_unique} again, but this time on
\[
\begin{aligned}
\epsilon_{Fb} \circ F(GF f \circ \eta_a) =
\\ \epsilon_{Fb} \circ FGF f \circ F \eta_a = && \text{(by naturality of $\epsilon$)}
\\ F f \circ \epsilon_{Fa} \circ F \eta_a = && \text{(by \cref{epsilon_id})}
\\ F f \circ id_{Fa} =
\\ F f = 
\\ id_{Fb} \circ F f = && \text{(by \cref{epsilon_id})}
\\ \epsilon_{Fb} \circ F \eta_b \circ F f =
\\ \epsilon_{Fb} \circ F (\eta_b \circ f)
\end{aligned}
\]

This proves that $GF f \circ \eta_a = \eta_b \circ f$ which is exactly the equation defined by the naturality square.

\qed

\subsubsection{$\ref{def:natural} \Rightarrow \ref{def:isocollection}$}
We have two natural transformations $\epsilon \colon FG \dotarrow I$, $\eta \colon I \dotarrow GF$, such that $G\epsilon \circ \eta_G = I$, $\epsilon_F \circ F\eta = I$.

We want to get a collection of isomorphisms $\phi_{x,y} \colon Hom(Fx, y) \stackrel{\simeq}{\longrightarrow} Hom(x, Gy)$ that is natural in $x$ and in $y$.

We do this by defining $\phi_{x,y}$, $\phi_{x,y}^{-1}$ as follows.
\[
\begin{tikzcd}
Hom(Fx, y) \arrow[r, shift left, "\phi_{x,y}"] & \arrow[l, shift left, "\phi_{x,y}^{-1}"] Hom(x,Gy) \\[-10pt]
f \arrow[r, maps to] & Gf \circ \eta_x \\[-20pt]
\epsilon_y \circ F g & g \arrow[l, maps to]
\end{tikzcd}
\]

We now have to prove:
\begin{enumerate}
	\item \label{iso} $\phi_{x,y}$ is an isomorphism
	\item \label{square1} The first naturality square in \cref{def:isocollection} commutes.
	\item \label{square2} The second naturality square in \cref{def:isocollection} commutes.
\end{enumerate}

We prove \cref{iso} by showing that our definitions of $\phi_{x,y}$ and $\phi_{x,y}^{-1}$ are actually inverses. This happens because:

\[
\begin{aligned}
\phi_{x,y} (\phi_{x,y}^{-1}(g)) =
\\ G(\phi_{x,y}^{-1}(g)) \circ \eta_x =
\\ G(\epsilon_y \circ F g) \circ \eta_x =
\\ G\epsilon_y \circ GF g \circ \eta_x = && \text{(by naturality of $\eta$)}
\\ G\epsilon_y \circ \eta_{Gy} \circ g =
\\ id_{Gy} \circ g = 
\\ g
\end{aligned}
\]

and

\[
\begin{aligned}
\phi_{x,y}^{-1}(\phi_{x,y}(f)) =
\\ \epsilon_y \circ F(\phi_{x,y}(f)) =
\\ \epsilon_y \circ F(Gf \circ \eta_x) =
\\ \epsilon_y \circ FGf \circ F\eta_x = && \text{(by naturality of $\epsilon$)}
\\ f \circ \epsilon_{Fx} \circ F\eta_x =
\\ f \circ id_{Fx} =
\\ f
\end{aligned}
\]

\blackqed

To prove \cref{square1}, let us take a look at the first naturality square for $\phi$:

\[
\begin{tikzcd}
Hom(Fa, b) \arrow[r, "\phi_{a,b}"] \arrow[d, "{Hom(Ff,b)}"] & Hom(a, Gb) \arrow[d, "{Hom(f, Gb)}"]
\\ Hom(Fa', b) \arrow[r, "\phi_{a',b}"] & Hom(a', Gb)
\end{tikzcd}
\begin{tikzcd}
h \arrow[r, maps to] \arrow[d, maps to] & \phi_{a,b}(h) \arrow[d, maps to]
\\ h \circ Ff \arrow[r, maps to] & \phi_{a',b}(h \circ Ff) = \phi_{a,b}(h) \circ f
\end{tikzcd}
\]

Replacing with our definition for $\phi$, we get that we have to prove:

\[
\begin{tikzcd}
h \arrow[r, maps to] \arrow[d, maps to] & Gh \circ \eta_a \arrow[d, maps to]
\\ h \circ Ff \arrow[r, maps to] & G(h \circ Ff) \circ \eta_{a'} = Gh \circ \eta_a \circ f
\end{tikzcd}
\]

But this is true, for:

\[
\begin{aligned}
G(h \circ Ff) \circ \eta_{a'} =
\\ Gh \circ GFf \circ \eta_{a'} = && \text{(by naturality of $\eta$)}
\\ Gh \circ \eta_{a} \circ f
\end{aligned}
\]

\blackqed

To prove \cref{square2}, let us take a look at the second naturality square for $\phi$:

\[
\begin{tikzcd}
Hom(Fa, b) \arrow[r, "\phi_{a,b}"] \arrow[d, "{Hom(Fa,f)}"] & Hom(a, Gb) \arrow[d, "{Hom(a, Gf)}"]
\\ Hom(Fa, b') \arrow[r, "\phi_{a,b'}"] & Hom(a, Gb')
\end{tikzcd}
\begin{tikzcd}
h \arrow[r, maps to] \arrow[d, maps to] & \phi_{a,b}(h) \arrow[d, maps to]
\\ f \circ h \arrow[r, maps to] & \phi_{a,b'}(f \circ h) = Gf \circ \phi_{a,b}(h)
\end{tikzcd}
\]

Replacing with our definition for $\phi$, we get that we have to prove:

\[
\begin{tikzcd}
h \arrow[r, maps to] \arrow[d, maps to] & Gh \circ \eta_a \arrow[d, maps to]
\\ f \circ h \arrow[r, maps to] & G(f \circ h) \circ \eta_{a} = Gf \circ Gh \circ \eta_a
\end{tikzcd}
\]

This is trivial. \qed

\subsection{Conclusions}

So\textellipsis Why did we go through all of this? What do we have now that we didn't before?

The answer is two-fold: We now have more understanding of what an adjunction is, and we have more tools to work with adjunctions in the future.

We understand adjunctions better because we are able to see them through different angles now.

We no longer think an adjunction is just a relation between Hom-sets, we know that it also defines a family of universal morphisms (and we also know \textit{in what way} it defines such a family).

And that also gives us more power: We can now use any of the lenses we have to see adjunctions to attack any problem related to them.

That will be essential, when we start attacking the relation between monads and adjunctions next class.

\pagebreak

\section{Appendix}

\subsection{Lemmas}

Some lemmas were used above and are defined here to keep the flow of the proofs steady.

\begin{lemma}{For any arrow $g \colon x \to Gy$, $\phi_{x,y}^{-1}(g) = \phi_{Gy,y}^{-1}(id_{Gy}) \circ Fg$}\label{lem:phi_inv}
	
\end{lemma}

The proof for this lemma can be easily obtained by chasing the first naturality diagram for $\phi^{-1}$ (which is just the first naturality diagram for $\phi$ reversed at the x-axis) at $a = Gy, a' = x, b = y$.

\[
\begin{tikzcd}
Hom(G y, G y) \arrow[r, "\phi_{Gy,y}^{-1}"] \arrow[d, "{Hom(g,Gy)}"] 
& Hom(FG y, y) \arrow[d,"{Hom(Fg,y)}"] \\[25pt]
Hom(x, Gy) \arrow[r, swap, "\phi_{x,y}^{-1}"] & Hom(F x, y)
\end{tikzcd}
\begin{tikzcd}
id_{Gy} \arrow[r,maps to] \arrow[d,maps to] &[-15pt] \phi^{-1}_{Gy,y} (id_{Gy}) \arrow[d, maps to] \\[25pt]
id_{Gy} \circ g = g \arrow[r, maps to] & \phi_{x,y}^{-1} (g) = \phi_{Gy,y}^{-1}(id_{Gy}) \circ F g
\end{tikzcd}
\]

\qed

\begin{lemma}{$\epsilon_c = \phi_{Gc,c}^{-1}(id_{Gc}) \colon FG \dotarrow I$ is a natural transformation}\label{lem:phi_inv_nat}
\end{lemma}
	
In this case, the definition of natural transformation is that the following square commutes for all $c, c', f \colon c \to c'$:
\[
\begin{tikzcd}
FG c \arrow[r, "\epsilon_c"] \arrow[d, "{FG f}"] 
&[50pt] c \arrow[d,"f"] \\[25pt]
FG c' \arrow[r, "\epsilon_{c'}"] & c'
\end{tikzcd}
\]

Or, in algebraic terms, $\epsilon_{c'} \circ FG f = f \circ \epsilon_c$.

We get to that point by applying \cref{lem:phi_inv} at $x = G c, y = c', g = G f$ and get $\epsilon_{c'} \circ FG f = \phi_{Gc',c'}^{-1}(id_{Gc'}) \circ FG f = \phi_{Gc,c'}^{-1}(G f)$, and then chasing the second naturality diagram for $\phi^{-1}$ at $a = Gc, b = c, b' = c'$ to get:

\[
\begin{tikzcd}
Hom(G c, G c) \arrow[r, "\phi_{Gc,c}^{-1}"] \arrow[d, "{Hom(Gc,Gf)}"] 
& Hom(FG c, c) \arrow[d,"{Hom(FG c,f)}"] \\[25pt]
Hom(G c, G c') \arrow[r, swap, "\phi_{Gc,c'}^{-1}"] & Hom(FG c, c')
\end{tikzcd}
\begin{tikzcd}
id_{Gc} \arrow[r,maps to] \arrow[d,maps to] &[-15pt] \phi^{-1}_{Gc,c} (id_{Gc}) \arrow[d, maps to] \\[25pt]
Gf \arrow[r, maps to] & \phi_{Gc,c'}^{-1} (Gf) = f \circ \phi_{Gc,c}^{-1}(id_{Gc})
\end{tikzcd}
\]

Hence, $\epsilon_{c'} \circ FG f = \phi_{Gc',c'}^{-1}(id_{Gc'}) \circ FG f = \phi_{Gc,c'}^{-1}(G f) = f \circ \phi_{Gc,c}^{-1}(id_{Gc}) =  f \circ \epsilon_c$. \qed

\begin{lemma}\label{lem:epsilon_unique}
	If two arrows $f_1,f_2 \colon a \to Gb$ are such that $\epsilon_b \circ Ff_1 = \epsilon_b \circ Ff_2$, then $f_1 = f_2$.
\end{lemma}
We prove this lemma by observing that we can put both $f_1$ and $f_2$ in the universal diagram:
\[
\begin{tikzcd}
a \arrow[d, "f_2"] \arrow[d, swap, "f_1"] &[25pt] F a \arrow[d, "Ff_2"] \arrow[d, swap, "Ff_1"] \arrow[rd, "\epsilon_b \circ Ff_1 = \epsilon_b \circ Ff_2"] \\[35pt] Gb & FGb \arrow[r, "\epsilon_{b}"] &[50pt] b
\end{tikzcd}
\]

Due to the uniqueness of $g$ for any $f$ in the universal diagram, both arrows must be the same.

\end{document}

