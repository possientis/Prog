The notion of {\em small category} defined in 
definition~(\ref{Cat:def:category:small}) is similar to that of any other 
algebraic structure the reader may be familiar with. It can safely be encoded 
in set theory as a tuple (which is a set) containing data (which are other 
sets) which satisfies certain properties. In set theory, everything is a 
set. A small category \Cat\ is a set, its collection of objects $\ob\ \Cat$ is 
a set, its arrows $\arr\ \Cat$ form a set, the functions $\dom$, $\cod$, $\id$ 
and the composition operator $\circ$ are all sets (functions are typically 
encoded as sets of ordered pairs). 

Category theory falls outside of set theory. While the definition of a
{\em category} we provide below is formally identical to that of a small
category, the object we are defining can no longer be encoded in general 
as an object of set theory. For example, say we want to speak about the 
{\em universe of all sets} or the {\em universe of all monoids}. These 
{\em universes} which are known as {\em classes} cannot be represented 
as sets. They are not objects of set theory. Or say we are working within
the formal framework of a proof assistant such as {\rm Coq}, {\rm Agda} or
{\rm Lean}. These tools are based on type theory and do not fall within
the scope of set theory. When defining a {\em category}, we assume some
form of meta-theoretic context, some form of logic, some way of reasoning
about objects which may not be sets, where some meaning is attached to the
words {\em tuple}, {\em collection}, {\em equality} and {\em map}. 
This may sound all very fuzzy, yet we cannot be more formal at this stage.

\begin{defin}\label{Cat:def:category}
    We call {\em category} any tuple $(\ob, \arr, \dom, \cod, \id, \circ)$ 
    such that:
        \begin{eqnarray*}
            (1)&\ &\ob\mbox{\ is a collection with equality}\\
            (2)&\ &\arr\mbox{\ is a collection with equality}\\
            (3)&\ &\dom:\arr\to\ob\mbox{\ is a map}\\
            (4)&\ &\cod:\arr\to\ob\mbox{\ is a map}\\
            (5)&\ &\id:\ob\to\arr\mbox{\ is a map}\\
            (6)&\ &\circ:\arr\times\arr\to\arr\mbox{\ is a partial map}\\
            (7)&\ &g\circ f\mbox{\ is defined}
                \ \Leftrightarrow\ \cod(f)=\dom(g) \\
            (8)&\ &\cod(f)=\dom(g)\ \Rightarrow\ \dom(g\circ f) = \dom(f)\\
            (9)&\ &\cod(f)=\dom(g)\ \Rightarrow\ \cod(g\circ f) = \cod(g)\\
            (10)&\ &\cod(f)=\dom(g)\,\land\,\cod(g)=\dom(h)
               \ \Rightarrow\ (h\circ g)\circ f = h\circ(g\circ f)\\
            (11)&\ &\dom\,(\,\id(a)\,) = a = \cod\,(\,\id(a)\,)\\
            (12)&\ &\dom(f)=a\ \Rightarrow\ f\circ\id(a) = f\\
            (13)&\ &\cod(f)=a\ \Rightarrow\ \id(a)\circ f = f
       \end{eqnarray*} 
    where $(7)-(13)$ hold for all $f,g,h\in\arr$ and $a\in\ob$: 
\end{defin}

\newpage
So let $\Cat=(\ob, \arr, \dom, \cod, \id, \circ)$ be a category: then \Cat\ is 
a {\em tuple} but it is no longer a tuple in a set theoretic sense. We assume
given some logical framework where the notion of {\em tuple} is clear, even
if not formally defined. Furthermore, We are no longer imposing that \ob\ 
should be a set, but are instead using the phrase {\em collection with 
equality}, whatever this may mean in our given logical context. So we shall 
still make use of the notation $\ob\ \Cat$ but this will now refer to the 
{\em collection} of all {\em objects} of the category \Cat. In fact, if
$a$ is an object of the category \Cat, we shall abuse notations somewhat
by writing $a\in\ob\ \Cat$ or even simply $a\in\Cat$ to express the fact 
that $a$ is an object of \Cat, being understood that this use of the
set membership symbol '$\in$' does not mean anything is a set. Since we 
are stepping out of set theory, the objects of the category \Cat\ may not
be sets themselves. They are simply members of the {\em collection} 
$\ob\ \Cat$. However, properties $(7)-(13)$ of 
definition~(\ref{Cat:def:category}) are all refering to equalities
between objects such that $\cod(f)=\dom(g)$. So it must be the case that 
the notion of {\em equality} be meaningful on the collection $\ob\ \Cat$. 
This explains our use of the phrase {\em collection with equality}: given
$a,b\in\Cat$, the statement $a=b$ while not a set-theoretic equality is 
nonetheless assumed to be defined.
