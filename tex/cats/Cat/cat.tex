The notion of {\em small category} defined in 
definition~(\ref{Cat:def:category:small}) is similar to that of any other 
algebraic structure the reader may be familiar with. It can safely be encoded 
in set theory as a tuple (which is a set) containing data (which are other 
sets) which satisfies certain properties. In set theory, everything is a 
set. A small category \Cat\ is a set, its collection of objects $\ob\ \Cat$ is 
a set, its arrows $\arr\ \Cat$ form a set, the functions $\dom$, $\cod$, $\id$ 
and the composition operator $\circ$ are all sets (functions are typically 
encoded as sets of ordered pairs). 

Category theory falls outside of set theory. While the definition of a
{\em category} we provide below is formally identical to that of a small
category, the object we are defining can no longer be encoded in general 
as an object of set theory. For example, say we want to speak about the 
{\em universe of all sets} or the {\em universe of all monoids}. These 
{\em universes} which are known as {\em classes} cannot be represented 
as sets. They are not objects of set theory. Or say we are working within
the formal framework of a proof assistant such as {\rm Coq}, {\rm Agda} or
{\rm Lean}. These tools are based on type theory and do not fall within
the scope of set theory. When defining a {\em category}, we assume some
form of meta-theoretic context, some form of logic, some way of reasoning
about objects which may not be sets, where some meaning is attached to the
words {\em tuple}, {\em collection}, {\em equality} and {\em map}. 
This may sound all very fuzzy, yet we cannot be more formal at this stage.

\begin{defin}\label{Cat:def:category}
    We call {\em category} any tuple $(\ob, \arr, \dom, \cod, \id, \circ)$ 
    such that:
        \begin{eqnarray*}
            (1)&\ &\ob\mbox{\ is a collection with equality}\\
            (2)&\ &\arr\mbox{\ is a collection with equality}\\
            (3)&\ &\dom:\arr\to\ob\mbox{\ is a map}\\
            (4)&\ &\cod:\arr\to\ob\mbox{\ is a map}\\
            (5)&\ &\id:\ob\to\arr\mbox{\ is a map}\\
            (6)&\ &\circ:\arr\times\arr\to\arr\mbox{\ is a partial map}\\
            (7)&\ &g\circ f\mbox{\ is defined}
                \ \Leftrightarrow\ \cod(f)=\dom(g) \\
            (8)&\ &\cod(f)=\dom(g)\ \Rightarrow\ \dom(g\circ f) = \dom(f)\\
            (9)&\ &\cod(f)=\dom(g)\ \Rightarrow\ \cod(g\circ f) = \cod(g)\\
            (10)&\ &\cod(f)=\dom(g)\,\land\,\cod(g)=\dom(h)
               \ \Rightarrow\ (h\circ g)\circ f = h\circ(g\circ f)\\
            (11)&\ &\dom\,(\,\id(a)\,) = a = \cod\,(\,\id(a)\,)\\
            (12)&\ &\dom(f)=a\ \Rightarrow\ f\circ\id(a) = f\\
            (13)&\ &\cod(f)=a\ \Rightarrow\ \id(a)\circ f = f
       \end{eqnarray*} 
    where $(7)-(13)$ hold for all $f,g,h\in\arr$ and $a\in\ob$: 
\end{defin}

\newpage
So let $\Cat=(\ob, \arr, \dom, \cod, \id, \circ)$ be a category: then \Cat\ is 
a {\em tuple} but it is no longer a tuple in a set-theoretic sense. We assume
given some logical framework where the notion of {\em tuple} is clear, even
if not formally defined. Furthermore, We are no longer imposing that \ob\ 
should be a set, but are instead using the phrase {\em collection with 
equality}, whatever this may mean in our given logical context. So we shall 
still make use of the notation $\ob\ \Cat$ but this will now refer to the 
{\em collection} of all {\em objects} of the category \Cat. 
\begin{notation}\label{Cat:notation:objects}
    Let \Cat\ be a category. Its collection of {\em objects} is denoted $\ob\ 
    \Cat$.
\end{notation}

In fact, if
$a$ is an object of the category \Cat, we shall abuse notations somewhat
by writing '$a\in\ob\ \Cat$' or even simply '$a\in\Cat$' to express the fact 
that $a$ is an object of \Cat, being understood that this use of the
set membership symbol '$\in$' does not mean anything is a set. Since we 
are stepping out of set theory, the objects of the category \Cat\ may not
be sets themselves. They are simply members of the {\em collection} 
$\ob\ \Cat$. 
\begin{notation}\label{Cat:notation:membership}
    Let \Cat\ be a category. We write $a\in\Cat$ as a shortcut for 
    $a\in\ob\ \Cat$.
\end{notation}

However, properties $(7)-(13)$ of 
definition~(\ref{Cat:def:category}) are all referring to equalities
between objects such that $\cod(f)=\dom(g)$. So it must be the case that 
the notion of {\em equality} be meaningful on the collection $\ob\ \Cat$. 
This explains our use of the phrase {\em collection with equality}: given
$a,b\in\Cat$, the statement $a=b$ while not a set-theoretic equality is 
nonetheless assumed to be defined. 

Similarly, the {\em collection} of
{\em arrows} of the category \Cat\ shall still be denoted $\arr\ \Cat$,
but is no longer required to be a set. If $f$ is an arrow of the category
\Cat\ then $f$ itself may not be a set and we may still write '$f\in\arr\ \Cat$'
simply to indicate that $f$ is a {\em member} of the {\em collection}
$\arr\ \Cat$. Properties $(10)$, $(12)$ and $(13)$ of 
definition~(\ref{Cat:def:category}) are all referring to equalities 
between arrows so the {\em collection} $\arr\ \Cat$ must have some notion
of {\em equality} defined on it. 

\begin{notation}\label{Cat:notation:arrows}
    Let \Cat\ be a category. Its collection of {\em arrows} is denoted $\arr\ 
    \Cat$.
\end{notation}

Since \ob\ and \arr\ are no longer sets in general, the {\em maps} 
$\dom:\arr\to\ob$, $\cod:\arr\to\ob$, $\id:\ob\to\arr$ and the partial map 
$\circ:\arr\times\arr\to\arr$ cannot possibly be {\em functions} in the 
set-theoretic sense. So there must be some meaning to the word {\em map}
(from one {\em collection} to another) in whatever logical framework
we are working in. The {\em collection} $\arr\times\arr$ is not a set, 
and is simply the {\em collection} of all $2$-dimensional {\em tuples}
made from \arr. Our using the word {\em map} rather than {\em function}
in definition~(\ref{Cat:def:category}) is simply an attempt at reminding
ourselves of the fact these are not set-theoretic functions, eventhough
the words {\em map} and {\em function} are perfectly interchangeable in 
standard (set-theoretic) mathematics. Given $f\in\arr\ \Cat$, we shall 
still call the object $\dom(f)$ the {\em domain} of $f$ and the object 
$\cod(f)$ the {\em codomain} of $f$.

\begin{notation}\label{Cat:notation:domain}
    Let \Cat\ be a category. The domain of an arrow $f\in\arr\ \Cat$ is 
    denoted $\dom(f)$, while its codomain is denoted $\cod(f)$.
\end{notation}


\noindent
{\bf Remark}: Notation~(\ref{Cat:notation:domain}) is potentially ambiguous 
as a mathematical object $f$ could in principle be an arrow in several 
categories, and the designations $\dom(f)$ and $\cod(f)$ do not specfify 
which category is being referred to.

Given $a,b\in\Cat$, we shall still use the notation $f:a\to b$ as a
notational shortcut for $\dom(f)=a$ and $\cod(f)=b$. Hence we state:

\begin{notation}\label{Cat:notation:arrow:a:b}
    Let \Cat\ be a category. We write $f:a\to b$ or $f:a\to b\ @\ \Cat$ 
    as a shortcut for $f\in\arr\ \Cat$ together with $\dom(f)=a$ and 
    $\cod(f)=b$.
\end{notation}
\noindent
{\bf Remark}: The qualification $@\ \Cat$ in 
notation~(\ref{Cat:notation:arrow:a:b}) may be useful to disambiguate
between several categories in a given context.

The partial map $\circ$ is still the {\em composition operator} and the 
arrow $g\circ f$ shall still be called the {\em composition} of $g$ and $f$, 
provided it is defined. 
\begin{notation}\label{Cat:notation:composition}
    Let \Cat\ be a category. The composition
    operator on \Cat\ is denoted $\circ$, and the composition of 
    two arrows $g$ and $f$ is denoted $g\circ f$.
\end{notation}

\noindent
{\bf Remark}: $\circ$ may also be ambiguous in a context with several 
categories. It is also the common symbol to refer to standard function
composition.

The map $\id:\ob\to\arr$ is still the {\em identity operator}
on the category \Cat, and for all $a\in\Cat$, the arrow $id(a):a\to a$ is 
known as the {\em identity at} $a$. 

\begin{notation}\label{Cat:notation:identity}
    Let \Cat\ be a category. We write $\id(a)$ or $\id(a)\ @\ \Cat$ 
    to denote the {\em identity at $a$} in the category \Cat.
\end{notation}

For all arrows $f:a\to b$, it is still
the case that the arrows $\id(b)\circ f$ and $f\circ\id(a)$ are well-defined
and both equal to $f$. Just as in the case of a small category, whenever 
$f:a\to b$, $g:b\to c$ and $h:c\to d$, all the terms involved in the 
equation $(h\circ g)\circ f=h\circ(g\circ f)$ of
definition~(\ref{Cat:def:category}) are well defined.


\begin{prop}\label{Cat:prop:smallcat:is:cat}
    A small category is a category.
\end{prop}
\begin{proof}
    When considering a small category \Cat, we are implicitely working
    within a set theoretic framework in which equality between any two
    sets is always meaningful and elements of sets are themselves sets.
    So any set can be viewed as a {\em collection with equality} and 
    hence a small category satisfies definition~(\ref{Cat:def:category}).
\end{proof}
