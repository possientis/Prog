Before we define a category in full generality, we shall focus our attention
on the notion of {\em small category}. This notion is interesting to us 
because while it essentially describes the notion of {\em category} itself,
it remains simple enough to be compared with various other algebraic structures.
So let us look at a monoid: a monoid is essentially a set $M$ together with 
a binary relation $\circ$ defined on $M$ which is associative, and an 
element $e$ of $M$ which acts as an identity element for $\circ$. In short 
a monoid is a tuple $(M,\circ,e)$ containing some data, and  which satisfy 
certain properties. The same is true of a {\em small category}: it is also
a tuple containing some data, and which satisfy certain properties:

\begin{defin}\label{Cat:def:category:small}
    We call {\em small category} any tuple $(\ob, \arr, \dom, \cod, \id, \circ)$ 
    with:
        \begin{eqnarray*}
            (1)&\ &\ob\mbox{\ is a set}\\
            (2)&\ &\arr\mbox{\ is a set}\\
            (3)&\ &\dom:\arr\to\ob\mbox{\ is a map}\\
            (4)&\ &\cod:\arr\to\ob\mbox{\ is a map}\\
            (5)&\ &\id:\ob\to\arr\mbox{\ is a map}\\
            (6)&\ &\circ:\arr\times\arr\to\arr\mbox{\ is a partial map}\\
            (7)&\ &g\circ f\mbox{\ is defined}
                \ \Leftrightarrow\ \cod(f)=\dom(g) \\
            (8)&\ &\cod(f)=\dom(g)\ \Rightarrow\ \dom(g\circ f) = \dom(f)\\
            (9)&\ &\cod(f)=\dom(g)\ \Rightarrow\ \cod(g\circ f) = \cod(g)\\
            (10)&\ &\cod(f)=\dom(g)\,\land\,\cod(g)=\dom(h)
               \ \Rightarrow\ (h\circ g)\circ f = h\circ(g\circ f)\\
            (11)&\ &\dom\,(\,\id(a)\,) = a = \cod\,(\,\id(a)\,)\\
            (12)&\ &\dom(f)=a\ \Rightarrow\ f\circ\id(a) = f\\
            (13)&\ &\cod(f)=a\ \Rightarrow\ \id(a)\circ f = f
       \end{eqnarray*} 
    where $(7)-(13)$ hold for all $f,g,h\in\arr$ and $a\in\ob$: 
\end{defin}

\begin{defin}\label{Cat:def:category}
    We call {\em category} any tuple $(\ob, \arr, \dom, \cod, \id, \circ)$ 
    such that:
        \begin{eqnarray*}
            (1)&\ &\ob\mbox{\ is a collection with equality}\\
            (2)&\ &\arr\mbox{\ is a collection with equality}\\
            (3)&\ &\dom:\arr\to\ob\mbox{\ is a map}\\
            (4)&\ &\cod:\arr\to\ob\mbox{\ is a map}\\
            (5)&\ &\id:\ob\to\arr\mbox{\ is a map}\\
            (6)&\ &\circ:\arr\times\arr\to\arr\mbox{\ is a partial map}\\
            (7)&\ &g\circ f\mbox{\ is defined}
                \ \Leftrightarrow\ \cod(f)=\dom(g) \\
            (8)&\ &\cod(f)=\dom(g)\ \Rightarrow\ \dom(g\circ f) = \dom(f)\\
            (9)&\ &\cod(f)=\dom(g)\ \Rightarrow\ \cod(g\circ f) = \cod(g)\\
            (10)&\ &\cod(f)=\dom(g)\,\land\,\cod(g)=\dom(h)
               \ \Rightarrow\ (h\circ g)\circ f = h\circ(g\circ f)\\
            (11)&\ &\dom\,(\,\id(a)\,) = a = \cod\,(\,\id(a)\,)\\
            (12)&\ &\dom(f)=a\ \Rightarrow\ f\circ\id(a) = f\\
            (13)&\ &\cod(f)=a\ \Rightarrow\ \id(a)\circ f = f
       \end{eqnarray*} 
    where $(7)-(13)$ hold for all $f,g,h\in\arr$ and $a\in\ob$: 
\end{defin}

