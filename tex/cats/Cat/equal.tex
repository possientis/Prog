Whichever logical framework we are working from, we saw that when defining
a category $\Cat=(\ob, \arr, \dom, \cod, \id, \circ)$, some notion of
equality had to be defined on the collections \ob\ and \arr. Now if
$\Cat'=(\ob', \arr', \dom', \cod', \id', \circ')$ is another category,
the question may arise as to whether $\Cat=\Cat'$. Or indeed, we may 
simply be asking whether the collections \ob\ and \ob'\ are the same,
or whether $\dom=\dom'$ etc. It is very difficult for us to carry out
any sort of formal reasoning on things without equality. So having
equality defined on \ob\ and \arr\ is neccessary for
definition~(\ref{Cat:def:category}) to even make sense, but it is not
enough for us to formally prove anything about categories. Hence we shall
assume:

\begin{axiom}\label{Cat:ax:collection}
    A notion of equality exists for {\em collections}.
\end{axiom}

\noindent
The equality on {\em collections} should of course be reflexive, symmetric
and also transitive.




