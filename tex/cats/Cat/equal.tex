Whichever logical framework we are working from, we saw that when defining
a category $\Cat=(\ob, \arr, \dom, \cod, \id, \circ)$, some notion of
equality had to be defined on the collections \ob\ and \arr. Now if
$\Cat'=(\ob', \arr', \dom', \cod', \id', \circ')$ is another category,
the question may arise as to whether $\Cat=\Cat'$. Or indeed, we may 
simply be asking whether the collections \ob\ and \ob'\ are the same,
or whether $\dom=\dom'$ etc. It is very difficult for us to carry out
any sort of formal reasoning on things without equality. So having
equality defined on \ob\ and \arr\ is neccessary for
definition~(\ref{Cat:def:category}) to even make sense, but it is not
enough for us to formally prove anything about categories. Hence we shall
assume:

\begin{axiom}\label{Cat:ax:collection}
    A notion of equality exists for {\em collections}.
\end{axiom}

\noindent
It is implicit in the statement of axiom~(\ref{Cat:ax:collection})
that the notion of equality between {\em collections} should 
be reflexive, symmetric and transitive. Furthermore:

\begin{axiom}\label{Cat:ax:collection:extensional}
    Two collections with identical members are equal and conversely.
\end{axiom}
In particular, if $\Cat=(\ob, \arr, \dom, \cod, \id, \circ)$ is
a category and we have another partial map $\circ':\arr\times\arr\to\arr$ 
such that $g\circ' f$ is defined \ifand\ $g\circ f$ is defined,
then axiom~(\ref{Cat:ax:collection:extensional}) allows us to argue 
that the domain of $\circ'$ is the same collection as the domain
of $\circ$.

\begin{axiom}\label{Cat:ax:map:extensional}
    Let $A$ be a collection and $B$ be a collection with equality.
    Then two maps $F,G:A\to B$ are equal \ifand\ $F(x)=G(x)$ for all x in $A$.
\end{axiom}
In particular, if $\Cat=(\ob, \arr, \dom, \cod, \id, \circ)$ is a 
category and $\dom':\arr\to\ob$ is another map such that $\dom'(x)=\dom(x)$
for every object $x\in\Cat$, then $\dom'=\dom$. Or if $\circ':\arr\times\arr\to\arr$ 
is another partial map  with the same domain as that of $\circ$ and such that 
$g\circ' f = g\circ f$ when defined, then $\circ'=\circ$.

\begin{axiom}\label{Cat:ax:tuple:extensional}
    Two tuples with identical entries are equal.
\end{axiom}
So if $\Cat=(\ob, \arr, \dom, \cod, \id, \circ)$ and
$\Cat'=(\ob', \arr', \dom', \cod', \id', \circ')$ are two catgeories
such that $\ob=\ob'$, $\arr=\arr'$, $\dom=\dom'$, $\cod=\cod'$,
$\id=\id'$ and $\circ=\circ'$ then we have the equality $\Cat=\Cat'$.

\begin{notation}\label{Cat:ax:composition}
    If $f:A\to B$ and $g:B\to C$ are maps between collections,
    we denote $g\circ f$ the map $g\circ f:A\to C$ defined by 
    $(g\circ f)(a) = g(f(a))$ for all $a\in A$.
\end{notation}
