\begin{defin}\label{Cat:def:left:inverse}
    Let \Cat\ be a category and $f:a\to b$ for some $a,b\in\Cat$. 
    We say that $g:b \to a$ is a {\em left-inverse of} $f$, \ifand\ 
    $g\circ f=\id(a)$.
\end{defin}

\begin{defin}\label{Cat:def:right:inverse}
    Let \Cat\ be a category and $f:a\to b$ for some $a,b\in\Cat$. 
    We say that $g:b \to a$ is a {\em right-inverse of} $f$, \ifand\ 
    $f\circ g=\id(b)$.
\end{defin}

\noindent
{\bf Remark}: Comparing definitions~(\ref{Cat:def:left:inverse})
and~(\ref{Cat:def:right:inverse}), given a category \Cat\ and given 
$f,g\in\arr\ \Cat$, $g$ is a left-inverse of $f$ \ifand\ $f$ is a 
right-inverse of $g$.

\begin{defin}\label{Cat:def:inverse}
    Let \Cat\ be a category and $f:a\to b$ for some $a,b\in\Cat$. 
    We say that $g:b \to a$ is an {\em inverse of} $f$, \ifand\ 
    it is a left and right-inverse of $f$.
\end{defin}

\noindent 
{\bf Remark} : Given a category \Cat\ and $f,g\in\arr\ \Cat$, $g$ is
an inverse of $f$ \ifand\ $f$ is an inverse of $g$.

\begin{defin}\label{Cat:def:isomorphism}
    Let \Cat\ be a category and $f\in\arr\ \Cat$. We say that $f$ is
    an {\em isomorphism} \ifand\ $f$ has an inverse.
\end{defin}

\begin{prop}\label{Cat:prop:inverse:left:right:inverse:equal}
    Let \Cat\ be a category and $f\in\arr\ \Cat$. If $f$
    has a right-inverse and a left-inverse, then these are equal.
\end{prop}
\begin{proof}
    Let $a=\dom(f)$ and $b=\cod(f)$. Suppose $g:b\to a$ is a left-inverse of 
    $f$ and $h:b \to a$ is a right-inverse of $f$. We need to show that $g=h$:
    \begin{eqnarray*}g
        &=&g\circ\id(b)\ \leftarrow\ 
        \mbox{$(12)$ of def~(\ref{Cat:def:category})}\\
        \mbox{$h$ is right inverse}\ \to\ 
        &=&g\circ(f\circ h)\\
        \mbox{$\circ$ assoc}\ \to\ 
        &=&(g\circ f)\circ h\\
        \mbox{$g$ is left inverse}\ \to\ 
        &=&\id(a)\circ h\\
        \mbox{$(12)$ of def~(\ref{Cat:def:category})}\ \to\ 
        &=&h
    \end{eqnarray*}
\end{proof}

\begin{prop}\label{Cat:prop:inverse:unique}
    Let \Cat\ be a category. If $f\in\arr\ \Cat$ has 
    an inverse, it is unique.
\end{prop}
\begin{proof}
    Suppose $g,h\in\arr\ \Cat$ are both inverses of $f$. Then in particular
    $g$ is a left-inverse of $f$ and $h$ is a right-inverse of $f$. From
    proposition~(\ref{Cat:prop:inverse:left:right:inverse:equal}), we have $g=h$.
\end{proof}

\begin{notation}\label{Cat:notation:inverse}
    Given a category \Cat, an inverse of $f\in\arr\ \Cat$ is denoted $f^{-1}$.
\end{notation}

\begin{prop}\label{Cat:prop:isomorphism:left:right}
    Let \Cat\ be a category and $f\in\arr\ \Cat$. Then $f$ is an isomorphism
    \ifand\ $f$ has a left-inverse and a right-inverse.
\end{prop}
\begin{proof}
    Suppose $f$ is an isomorphism. From definition~(\ref{Cat:def:isomorphism}),
    $f$ has an inverse. Hence from definition~(\ref{Cat:def:inverse}), there
    is some $g\in\arr\ \Cat$ which is both a left-inverse and a right-inverse
    of $f$. So in particular, $f$ has a left-inverse and a right-inverse.
    Conversely, suppose $f$ has a left-inverse $g$ and a right-inverse $h$.
    from proposition~(\ref{Cat:prop:inverse:left:right:inverse:equal}) we 
    have $g=h$. So $g$ is in fact both a left-inverse and a right-inverse 
    of $f$. From definition~(\ref{Cat:def:inverse}), $g$ is in fact an
    inverse of $f$ and from definition~(\ref{Cat:def:isomorphism}) we conclude
    that $f$ is an isomorphism.
\end{proof}

\begin{prop}\label{Cat:prop:isomorphism:composition}
    Let \Cat\ be a category, $f:a\to b$ and $g:b\to c$ where $a,b,c\in\Cat$.
    If $f$ and $g$ are isomorphisms then  $g\circ f$ is an isomophism and:
        \[
            (g\circ f)^{-1} = f^{-1}\circ g^{-1}
        \]
\end{prop}
\begin{proof}
    We assume that $f$ and $g$ are isomophisms with inverses denoted 
    $f^{-1}$ and $g^{-1}$ respectively. We need to show that $g\circ f$
    is an isomorphism, i.e. that it has an inverse and that the inverse
    is actually $f^{-1}\circ g^{-1}$. So we need to show that 
    $f^{-1}\circ g^{-1}$ is both a left-inverse and a right-inverse
    of $g\circ f$. Note that since $f:a\to b$ and $g:b\to c$, we have
    $f^{-1}:b\to a$ and $g^{-1}:c\to b$ and consequently
    $f^{-1}\circ g^{-1}:c \to a$. The proof of left-identity goes as follows:
        \begin{eqnarray*}(f^{-1}\circ g^{-1})\circ(g\circ f)
            &=&f^{-1}\circ(g^{-1}\circ g)\circ f
            \ \leftarrow\ \circ\ \mbox{assoc}\\
            \mbox{$g^{-1}$ left-identity of $g$}\ \to\ 
            &=&f^{-1}\circ\id\,(b)\circ f\\
            \mbox{$(12)$ or $(13)$ of def.~(\ref{Cat:def:category})}\ \to\ 
            &=&f^{-1}\circ f\\
            \mbox{$f^{-1}$ left-identity of $f$}\ \to\ 
            &=&\id\,(a)
        \end{eqnarray*}
The proof of right-identity goes as follows:
        \begin{eqnarray*}(g\circ f)\circ(f^{-1}\circ g^{-1})
            &=&g\circ (f\circ f^{-1})\circ g^{-1}
            \ \leftarrow\ \circ\ \mbox{assoc}\\
            \mbox{$f^{-1}$ right-identity of $f$}\ \to\ 
            &=&g\circ\id\,(b)\circ g^{-1}\\
            \mbox{$(12)$ or $(13)$ of def.~(\ref{Cat:def:category})}\ \to\ 
            &=&g\circ g^{-1}\\
            \mbox{$g^{-1}$ right-identity of $g$}\ \to\ 
            &=&\id\,(c)
        \end{eqnarray*}
\end{proof}


