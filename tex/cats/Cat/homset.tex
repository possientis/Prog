\begin{defin}\label{Cat:def:homset}
    Let \Cat\ be a category and $a,b\in\Cat$. We call {\em hom-set of \Cat\ 
    associated with the ordered pair $(a,b)$} the collection denoted $\Cat(a,b)$ 
    and defined as:
        \[
            \Cat(a,b) = \{\ f\in\arr\ \Cat\ |\ f : a \to b\ \}
        \]
\end{defin}

\noindent
In other words the collection $\Cat(a,b)$ is the collection of all arrows $f$ in
\Cat\ such that $\dom(f)=a$ and $\cod(f)=b$. Note that despite being called a
'hom-set', the collection $\Cat(a,b)$ is generally not a set but an 
arbitary collection.

\begin{prop}\label{Cat:prop:homset:opposite}
    Let \Cat\ be a category and $a,b\in\Cat$. Then $\Cop(a,b)=\Cat(b,a)$.
\end{prop}
\begin{proof}
    When working in the context of two categories \Cat\ and \Cop, the notation
    $f:a\to b$ is ambiguous as the underlying category is unclear. Let us
    write $f:a\to b\ @\Cat$ and $f:a\to b\ @\Cop$ to disambiguate.
    Then if $\dom'=\cod$ and $\cod'=\dom$:
        \begin{eqnarray*}\Cop(a,b)
            &=&\{\ f\in\arr\ \Cop\ |\ f : a \to b\ @\Cop\ \}\\
            \mbox{def.~(\ref{Cat:def:opposite})\ $\to$\ }
            &=&\{\ f\in\arr\ \Cat\ |\ f : a \to b\ @\Cop\ \}\\
            \mbox{def.~(\ref{Cat:def:opposite})\ $\to$\ }
            &=&\{\ f\in\arr\ \Cat\ |\ \dom'(f)=a\ ,\ \cod'(f)=b\ \}\\
            &=&\{\ f\in\arr\ \Cat\ |\ \cod(f)=a\ ,\ \dom(f)=b\ \}\\
            &=&\{\ f\in\arr\ \Cat\ |\ f : b \to a\ @\Cat\ \}\\
            &=&\Cat(b,a)
        \end{eqnarray*}
\end{proof}

\begin{prop}\label{Cat:prop:homset:product}
    Let \Cone, \Ctwo\ be two categories and $a,b\in\Cone\times\Ctwo$. Then:
        \[
            \Cone\times\Ctwo\,(a,b) 
            = 
            \Cone(a_{1},b_{1})\times\Ctwo(a_{2},b_{2})
        \]
    where it is understood that $a=(a_{1},a_{2})$ and $b=(b_{1},b_{2})$.
\end{prop}
\begin{proof}
    Let $a=(a_{1},a_{2})$ and $b=(b_{1},b_{2})$ be objects in the
    product category $\Cone\times\Ctwo$:
    \begin{eqnarray*}\Cone\times\Ctwo\,(a,b) 
        &=&\Cone\times\Ctwo\,[\,(a_{1},a_{2})\,,\,(b_{1},b_{2})\,]\\
        \mbox{def.~(\ref{Cat:def:homset})\ $\to$\ }
        &=&\{\ f\in\arr\ (\Cone\times\Ctwo)\ \ |\ \ 
               f : (a_{1},a_{2})\to(b_{1},b_{2})\ \}\\
        \mbox{def.~(\ref{Cat:def:canonical:product})\ $\to$\ }
        &=&\{\ (f_{1},f_{2})\ |\ f_{1}\in\arr\ \Cone\ ,\ f_{2}\in\arr\ \Ctwo\\
        &\ &\,,\ (f_{1},f_{2}) : (a_{1},a_{2})\to(b_{1},b_{2})\ \}
    \end{eqnarray*}
\end{proof}

\begin{defin}\label{Cat:def:locally:small}
    A category \Cat\ is said to be {\em locally small} \ifand\ the hom-set
    $\Cat(a,b)$ associated with every ordered pair of objects $(a,b)$ is 
    actually a set.
\end{defin}

\begin{prop}\label{Cat:def:locally:small:opposite}
    A category \Cat\ is locally small \ifand\ \Cop is locally small.
\end{prop}
\begin{proof}
    The category \Cat\ being locally small is equivalent to $\Cat(a,b)$
    being a set for all $a,b\in\ob\ \Cat$. Since $\ob\ \Cat=\ob\ \Cop$
    and $\Cop(a,b)=\Cat(b,a)$ from 
    proposition~(\ref{Cat:prop:homset:opposite}), this is in turn equivalent 
    to $\Cop(a,b)$ being a set for all $a,b\in\ob\ \Cop$. Hence, it is
    equivalent to \Cop\ being locally small.
\end{proof}
