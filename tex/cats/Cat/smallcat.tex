Before we define a category in full generality, we shall focus our attention
on the notion of {\em small category}. This notion is interesting to us 
because while it essentially describes the notion of {\em category} itself,
it remains simple enough to be compared with various other algebraic structures.
For example, consider the case of a monoid: a monoid is essentially a set $M$ 
together with 
a binary relation $\circ$ defined on $M$ which is associative, and an 
element $e$ of $M$ which acts as an identity element for $\circ$. In short 
a monoid is a tuple $(M,\circ,e)$ containing some data, and  which satisfies
certain properties. The same is true of a {\em small category}: it is also
a tuple containing some data, and which satisfies certain properties:
\begin{defin}\label{Cat:def:category:small}
    We call {\em small category} any tuple $(\ob, \arr, \dom, \cod, \id, \circ)$ 
    with:
        \begin{eqnarray*}
            (1)&\ &\ob\mbox{\ is a set}\\
            (2)&\ &\arr\mbox{\ is a set}\\
            (3)&\ &\dom:\arr\to\ob\mbox{\ is a function}\\
            (4)&\ &\cod:\arr\to\ob\mbox{\ is a function}\\
            (5)&\ &\id:\ob\to\arr\mbox{\ is a function}\\
            (6)&\ &\circ:\arr\times\arr\to\arr\mbox{\ is a partial function}\\
            (7)&\ &g\circ f\mbox{\ is defined}
                \ \Leftrightarrow\ \cod(f)=\dom(g) \\
            (8)&\ &\cod(f)=\dom(g)\ \Rightarrow\ \dom(g\circ f) = \dom(f)\\
            (9)&\ &\cod(f)=\dom(g)\ \Rightarrow\ \cod(g\circ f) = \cod(g)\\
            (10)&\ &\cod(f)=\dom(g)\,\land\,\cod(g)=\dom(h)
               \ \Rightarrow\ (h\circ g)\circ f = h\circ(g\circ f)\\
            (11)&\ &\dom\,(\,\id(a)\,) = a = \cod\,(\,\id(a)\,)\\
            (12)&\ &\dom(f)=a\ \Rightarrow\ f\circ\id(a) = f\\
            (13)&\ &\cod(f)=a\ \Rightarrow\ \id(a)\circ f = f
       \end{eqnarray*} 
    where $(7)-(13)$ hold for all $f,g,h\in\arr$ and $a\in\ob$: 
\end{defin}

So if $\Cat=(\ob, \arr, \dom, \cod, \id, \circ)$ is a small category, we have
two sets $\ob$ and $\arr$ together with some structure defined on these sets.
This feels very much like a monoid, except that we have two sets instead of one
and it all looks more complicated. The set \ob\ is called the {\em set of 
objects} of the small category \Cat\ and is denoted $\ob\ \Cat$, while the set 
\arr\ is called the {\em set of arrows} of the small category \Cat\ and is 
denoted $\arr\ \Cat$. An element $x\in\ob\ \Cat$ is called an {\em object} 
of \Cat, while an element $f\in\arr\ \Cat$ is called an {\em arrow} of \Cat.

As part of the structure defined on the small category \Cat, we have two
functions $\dom:\arr\to\ob$ and $\cod:\arr\to\ob$. Hence, given an arrow $f$
of the small category \Cat, we have two objects $\dom(f)$ and $\cod(f)$ of the
small category \Cat. The object $\dom(f)$ is called the {\em domain} of $f$. 
The object $\cod(f)$ is called the {\em codomain} of $f$. Note that an arrow $f$ 
of the small category \Cat\ is simply an element of the set $\arr\ \Cat$. So it 
is itself a set but it may not be a function. The words {\em domain} and 
{\em codomain} are therefore overloaded as we are using them in relation to a 
set $f$ which is possibly not a function. Whenever $f$ is an arrow of the small 
category  \Cat\ and $a,b$ are objets, it is common to use the notation $f:a\to b$ 
as a notational shortcut for the equations $\dom(f)=a$ and $\cod(f)=b$. Once
again, it is important to guard against the possible confusion induced 
by the notation $f:a\to b$ which does not mean that $f$ is function. It 
simply means that $f$ is an arrow with domain $a$ and codomain $b$ in the 
small category \Cat. 

One of the main ingredients of the structure defining
a small category \Cat\ is the partial function $\circ:\arr\times\arr\to\arr$,
called the {\em composition operator} in the small category \Cat. The domain
of this partial function is made of all ordered pairs $(g,f)$ of arrows in
\Cat\ for which $\cod(f)=\dom(g)$. As already indicated in 
definition~(\ref{Cat:def:category:small}), we use the infix notation $g\circ f$
rather than $\circ(g,f)$ and the arrow $g\circ f$ is called the {\em composition}
of $g$ and $f$. Once again, we should remember that the notation $g\circ f$ does
not mean that $g$ or $f$ are functions. They are simply arrows in the small
category \Cat. One key property of the composition operator $\circ$ is the 
associativity postulated by $(10)$ of definition~(\ref{Cat:def:category:small}).
Note that if $f:a\to b$ and $g:b\to c$, then from properties $(8)$ and $(9)$ of
definition~(\ref{Cat:def:category:small}) we obtain $g\circ f:a\to c$. 
Furthermore, if $h:c\to d$ we have $h\circ g:b\to d$ and the arrows 
$(h\circ g)\circ f$ and $h\circ(g\circ f)$ are therefore well-defined arrows
with domain $a$ and codomain $d$. This shows that the expression involved
in the associativity condition~$(10)$ of definition~(\ref{Cat:def:category:small})
is always meaningful, involving terms which are well-defined provided
$g\circ f$ and $h\circ g$ are themselves well-defined, i.e. provided
$\cod(f)=\dom(g)$ and $\cod(g)=\dom(h)$. 

Finally, as part of the structure defining the small category \Cat, we have
a function $\id:\ob\to\arr$ called the {\em identity operator} on the small
category \Cat. Hence, for every object $a$ of \Cat\ we have an arrow 
$\id(a)$ called the {\em identity at} $a$. Looking at property~$(11)$ of 
definition~(\ref{Cat:def:category:small}) we see that $\id(a):a\to a$. 
In other words, the arrow $\id(a)$ has domain $a$ and codomain $a$. 
Furthermore, looking at properties~$(12)$ and~$(13)$ of 
definition~(\ref{Cat:def:category:small}), for every arrow $f:a\to b$, the
composition arrows $\id(b)\circ f$ and $f\circ \id(a)$ are well-defined and both
equal to $f$.


