\begin{defin}\label{Cat:def:set}
    We call \Set\ the category $\Set=(\ob, \arr, \dom, \cod, \circ)$ where
    \begin{eqnarray*}
        (1)&\ &\ob = \{\ x\ |\  \mbox{x is a set}\ \}\\
        (2)&\ &\arr = \{\ (a,b,f)\ |\  \mbox{f is a function $f : a\to b$}\ \}\\
        (3)&\ &\dom\,(a,b,f) = a\\
        (4)&\ &\cod\,(a,b,f) = b\\
        (5)&\ &(b,c,g)\circ(a,b,f) = (a,c,g\circ f)
    \end{eqnarray*}
    where $(3)$, $(4)$, $(5)$ hold for all sets $a,b,c$ and functions $f:a\to b$,
    $g:b\to c$, and $g\circ f$ denotes the usual function composition with 
    $(g\circ f)(x)=g(f(x))$.
\end{defin}
The collection of objects of the category \Set\ is defined to be the 
class of all sets. We are using the set comprehension notation 
$\{\ x\ |\ \mbox{$x$ is a set}\ \}$ to denote this class, but this is an
abuse of notation as \ob\ is not a set but a proper class. One could 
think of a class as a precicate $P(x)$ of first order logic with one
free variable. From this point of view $\ob$ becomes the predicate
$\ob(x)=\top$, i.e. the predicate which returns true for all $x$.
Every set satisfies the predicate \ob, so every set is a member of 
the class \ob. The class \ob\ is not a set because the set-theoretic
statement $\exists y, \forall z, z\in y\ \Leftrightarrow\ \ob(z)$
can be proven false. In other words, there exists no set $y$ whose
elements $z$ are exactly the sets which satisfy the predicate \ob.
There exists no set which contains all sets.

The collection of arrows of the category \Set\ is defined to be the 
class of triples $(a,b,f)$ where $a,b$ are sets and $f$ is a function
$f:a\to b$. This last notation is a common set-theoretic shortcut to
express that fact that $f$ is a {\em function} with {\em domain} $a$ and 
{\em range} {\bf which is a subset of} $b$. A {\em function} is any set $f$
whose elements are ordered pairs $(x,y)$ and which is functional, i.e.
for which the following implication holds for all sets $x,y,y'$:
    \[
        (x,y)\in f\ \land\ (x,y')\in f\ \Rightarrow\ y = y'
    \]
The {\em domain} of a function $f$ is the set of all sets $x$ for which there
exists a set $y$ with $(x,y)\in f$. The {\em range} of a function $f$ is the 
set of all sets $y$ for which there exists a set $x$ with $(x,y)\in f$.
If $x$ belongs to the domain of a function $f$, the notation '$f(x)$' commonly
refers to the unique set $y$ with $(x,y)\in f$. 

Now, as already pointed
out the notation $f:a\to b$ only requires that the range of $f$ should be
a subset of $b$. There is no requirement that the range of $f$ should be
equal to $b$. So if $f:a\to b$ and $b\subseteq c$ then $f:a\to c$.
This explains why the collection of arrows \arr\ is defined as a class of 
triples $(a,b,f)$ rather than just functions $f$. Knowing the function
$f$ does not tell you which {\em codomain} it should have. Any set $b$
which is a superset of its range is a possible codomain. 



