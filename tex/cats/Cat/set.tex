\begin{defin}\label{Cat:def:set}
    We call \Set\ the category $\Set=(\ob, \arr, \dom, \cod, \id, \circ)$ where
    \begin{eqnarray*}
        (1)&\ &\ob = \{\ x\ |\  \mbox{x is a set}\ \}\\
        (2)&\ &\arr = \{\ (a,b,f)\ |\  \mbox{f is a function $f : a\to b$}\ \}\\
        (3)&\ &\dom\,(a,b,f) = a\\
        (4)&\ &\cod\,(a,b,f) = b\\
        (5)&\ &\id(a) = (a,a,i(a))\\
        (6)&\ &(b,c,g)\circ(a,b,f) = (a,c,g\circ f)
    \end{eqnarray*}
    where $(3)-(6)$ hold for all sets $a,b,c$ and functions $f:a\to b$,
    $g:b\to c$, $i(a):a\to a$ denotes the usual identity function on $a$,
    and $g\circ f$ denotes the usual function composition defined by 
    $(g\circ f)(x)=g(f(x))$, for all $x\in a$.
\end{defin}
The collection of objects of the category \Set\ is defined to be the 
class of all sets. We are using the set comprehension notation 
$\{\ x\ |\ \mbox{$x$ is a set}\ \}$ to denote this class, but this is an
abuse of notation as \ob\ is not a set but a proper class. One could 
think of a class as a precicate $P(x)$ of first order logic with one
free variable. From this point of view $\ob$ becomes the predicate
$\ob(x)=\top$, i.e. the predicate which returns true for all $x$.
Every set satisfies the predicate \ob, so every set is a member of 
the class \ob. The class \ob\ is not a set because the set-theoretic
statement $\exists y, \forall z, z\in y\ \Leftrightarrow\ \ob(z)$
can be proven false. In other words, there exists no set $y$ whose
elements $z$ are exactly the sets which satisfy the predicate \ob.
There exists no set which contains all sets.

The collection of arrows of the category \Set\ is defined to be the 
class of triples $(a,b,f)$ where $a,b$ are sets and $f$ is a function
$f:a\to b$. This last notation is a common set-theoretic shortcut to
express the fact that $f$ is a {\em function} with {\em domain} $a$ and 
{\em range} {\bf which is a subset of} $b$. A {\em function} is any set $f$
whose elements are ordered pairs $(x,y)$ and which is functional, i.e.
for which the following implication holds for all sets $x,y,y'$:
    \[
        (x,y)\in f\ \land\ (x,y')\in f\ \Rightarrow\ y = y'
    \]
The {\em domain} of a function $f$ is the set of all sets $x$ for which there
exists a set $y$ with $(x,y)\in f$. The {\em range} of a function $f$ is the 
set of all sets $y$ for which there exists a set $x$ with $(x,y)\in f$.
If $x$ belongs to the domain of a function $f$, the notation '$f(x)$' commonly
refers to the unique set $y$ with $(x,y)\in f$. 

Now, as already pointed
out the notation $f:a\to b$ only requires that the range of $f$ should be
a subset of $b$. There is no requirement that the range of $f$ should be
equal to $b$. So if $f:a\to b$ and $b\subseteq c$ then $f:a\to c$.
This explains why the collection of arrows \arr\ is defined as a class of 
triples $(a,b,f)$ rather than a class of functions $f$. Knowing the 
function $f$ does not tell you which {\em codomain} it should have. Any 
set $b$ which is a superset of its range is a possible codomain. So we 
keep the set $b$ together with the function $f$ in the triple $(a,b,f)$
so as to remember which codomain is intended for this particular arrow
of the category \Set. Incidentally, we also keep the range $a$ of the 
function $f$ in the triple $(a,b,f)$ but this is not necessary, as the
knowledge of $f$ does allow us to recover its domain $a$. However,
the triple $(a,b,f)$ is convenient, allowing us to treat {\em domain} 
and {\em codomain} uniformly. In fact, it is worth pausing for a second
and emphasize the difference between $f$ and $(a,b,f)$:
\begin{defin}\label{Cat:def:typed:untyped:function}
    Given a function $f:a\to b$ between two sets $a$ and $b$ we say that
    $f$ is the {\em untyped function} while the triple $(a,b,f)$ is called
    the {\em typed} function.
\end{defin}

Once again, it should be remembered that
the collection of arrows \arr\ is not a set but a proper class, 
corresponding to the predicate $\arr(x)$:
    \[
        \arr(x)=\exists a\,\exists b\,\exists f\,,\  
        x = (a,b,f)\ \land\ f : a \to b
    \]
Informally, this predicates expresses the fact that $x$ is a typed function.
The maps $\dom:\arr\to\ob$ and $\cod:\arr\to\ob$ for the category \Set\
are defined respectively by $\dom(a,b,f)=a$ and $\cod(a,b,f)=b$. This 
looks simple enough, but for those who worry about foundational issues,
we should just note that these are also proper classes which can be
encoded as predicates. For example:
    \[
    \dom(x)=\exists u\,\exists v\,,\ x = (u,v)\ \land\ \arr(u)\ \land 
    \ (\ \exists a\,\exists b\,\exists f\,,\  u = (a,b,f)\ \land\ 
    v = a\ ) 
    \]
In other words, any set $x$ satisfies the predicate $\dom(x)$ \ifand\ 
it is an ordered pair $(u,v)$ where $u$ satisfies the predicate $\arr(u)$
and for which there exist sets $a,b,f$ with $u=(a,b,f)$ and $v=a$.
In short, $(u,v)$ satisfies the predicate $\dom$ \ifand\ $u$ is an
arrow $u=(a,b,f)$ and $v=a$.

We defined the identity operator $\id$ by $\id(a)=(a,a,i(a))$ and the 
composition operator $\circ$ by $(b,c,g)\circ(a,b,f)= (a,c,g\circ f)$ 
where $g\circ f$ is the usual function composition and $i(a):a\to a$ is
the usual identity function. As before, these defined maps are not 
functional sets of ordered pairs but rather proper classes which we 
could also encode as precicates of first order logic. One important
point to note is the fact that $(6)$ of definition~(\ref{Cat:def:set})
only defines the composition arrow $(b,c,g)\circ(a,b,f)$ where
$f:a\to b$ and $g:b\to c$. In other words, the composition 
$(d,c,g)\circ(a,b,f)$ with $f:a\to b$ and $g:d\to c$ is only 
defined when $b=d$. Furthermore, the usual function composition
$g\circ f$ is a function $g\circ f:a \to c$ which from~$(2)$ of
definition~(\ref{Cat:def:set}) means that the composed arrow
$(b,c,g)\circ(a,b,f)=(a,c,g\circ f)$ is indeed a member of the
collection \arr, and the partial map $\circ$ thus defined is 
indeed a partial map $\circ:\arr\times\arr\to\arr$.


\begin{prop}\label{Cat:prop:set:is:category}
    The category \Set\ of definition~(\ref{Cat:def:set}) is a category.
\end{prop}
\begin{proof}
    Now that we have defined 
    the data $(\ob, \arr, \dom, \cod, \id, \circ)$ of the category \Set, 
    it is time to check this data actually forms a category. We need to
    check that conditions $(7)-(13)$ of definition~(\ref{Cat:def:category})
    are satisfied. 
    
    $(7)$: suppose $f^{*}$ and $g^{*}$ are two members of 
    the collection \arr. We need to check that $g^{*}\circ f^{*}$ is defined 
    \ifand\ $\cod(f^{*})=\dom(g^{*})$. Using~$(2)$ of 
    definition~(\ref{Cat:def:set}), $f^{*}$ can be written $f^{*}=(a,b,f)$
    for some function $f:a\to b$ and $g^{*}$ can be written $g^{*}=(d,c,g)$
    for some function $g:d\to c$. However, from $(6)$ of 
    definition~(\ref{Cat:def:set}), the arrrow $(d,c,g)\circ(a,b,f)$ is only 
    defined in the case when $b=d$. Furthermore, from~$(4)$ of 
    definition~(\ref{Cat:def:set}) we have $\cod(f^{*})=b$ and from~$(3)$ 
    of definition~(\ref{Cat:def:set}) we have $\dom(g^{*})=d$. We conclude
    that $g^{*}\circ f^{*}$ is defined \ifand\ $\cod(f^{*})=\dom(g^{*})$
    as required.

    $(8)$: Let $f^{*},g^{*}\in\arr$ such that $\cod(f^{*})=\dom(g^{*})$. 
    We need to show that $\dom(g^{*}\circ f^{*})=\dom(f^{*})$. As before,
    $f^{*}$ and $g^{*}$ can be written as $f^{*}=(a,b,f)$ and
    $g^{*}=(b,c,g)$ where $f:a\to b$ and $g:b \to c$. We have
    $g^{*}\circ f^{*}=(a,c,g\circ f)$. Using~$(3)$ of 
    definition~(\ref{Cat:def:set}) we obtain $\dom(g^{*}\circ f^{*})=a=
    \dom(f^{*})$.

    $(9)$: Let $f^{*},g^{*}\in\arr$ such that $\cod(f^{*})=\dom(g^{*})$. 
    We need to show that $\cod(g^{*}\circ f^{*})=\cod(g^{*})$. As before,
    we have $g^{*}\circ f^{*}=(a,c,g\circ f)$ and $g^{*}=(b,c,g)$. Using~$(4)$ 
    of definition~(\ref{Cat:def:set}) we obtain $\cod(g^{*}\circ f^{*})=c=
    \cod(g^{*})$.

    $(10)$: Let $f^{*},g^{*},h^{*}\in\arr$ with $\cod(f^{*})=\dom(g^{*})$ 
    and $\cod(g^{*})=\dom(h^{*})$. We need to show the equality:
    $(h^{*}\circ g^{*})\circ f^{*} = h^{*}\circ(g^{*}\circ f^{*})$.
    However, $f^{*},g^{*},h^{*}$ can be decomposed as 
    $f^{*}=(a,b,f)$, $g^{*}=(b,c,g)$ and $h^{*}=(c,d,h)$ with
    $f:a\to b$, $g:b \to c$, and $h:c\to d$. We have:
        \begin{eqnarray*}(h^{*}\circ g^{*})\circ f^{*}
            &=&((c,d,h)\circ(b,c,g))\circ (a,b,f)\\
            \mbox{$(6)$ of Def~(\ref{Cat:def:set})\ $\rightarrow$\ } 
            &=&(b,d,h\circ g)\circ (a,b,f)\\
            \mbox{$(6)$ of Def~(\ref{Cat:def:set})\ $\rightarrow$\ } 
            &=&(a,d,(h\circ g)\circ f)\\
            \mbox{assoc of usual composition\ $\rightarrow$\ }
            &=&(a,d,h\circ (g\circ f))\\
            \mbox{$(6)$ of Def~(\ref{Cat:def:set})\ $\rightarrow$\ } 
            &=&(c,d,h)\circ (a,c,g\circ f)\\
            \mbox{$(6)$ of Def~(\ref{Cat:def:set})\ $\rightarrow$\ } 
            &=&(c,d,h)\circ ((b,c,g)\circ(a,b,f))\\
            &=&h^{*}\circ(g^{*}\circ f^{*})
        \end{eqnarray*}
    
    $(11)$: Let $a$ be a set. We need to show that $\dom(\id(a))=a=\cod(\id(a))$.  
    This follows immediately from $\id(a)=(a,a,i(a))$ which is $(5)$ of
    definition~(\ref{Cat:def:set}).

    $(12)$: Let $f^{*}=(a,b,f)$ be an arrow with $\dom(f^{*})=a$. We need to 
    show that $f^{*}\circ\id(a)=f^{*}$, which follows from:
        \begin{eqnarray*}f^{*}\circ\id(a)
            &=&(a,b,f)\circ\id(a)\\
            \mbox{$(5)$ of Def~(\ref{Cat:def:set})\ $\rightarrow$\ } 
            &=&(a,b,f)\circ(a,a,i(a))\\
            \mbox{$(6)$ of Def~(\ref{Cat:def:set})\ $\rightarrow$\ } 
            &=&(a,b,f\circ i(a))\\
            \mbox{usual right-identity\ $\rightarrow$\ }
            &=&(a,b,f)\\
            &=&f^{*}
        \end{eqnarray*}
    $(13)$: Let $f^{*}=(b,a,f)$ be an arrow with $\cod(f^{*})=a$. We need to 
    show that $\id(a)\circ f^{*}=f^{*}$, which follows from:
        \begin{eqnarray*}\id(a)\circ f^{*}
            &=&\id(a)\circ(b,a,f)\\
            \mbox{$(5)$ of Def~(\ref{Cat:def:set})\ $\rightarrow$\ } 
            &=&(a,a,i(a))\circ(b,a,f)\\
            \mbox{$(6)$ of Def~(\ref{Cat:def:set})\ $\rightarrow$\ } 
            &=&(b,a,i(a)\circ f)\\
            \mbox{usual left-identity\ $\rightarrow$\ }
            &=&(b,a,f)\\
            &=&f^{*}
        \end{eqnarray*}
This completes our proof of properties~$(7)-(13)$.
\end{proof}

\begin{notation}\label{Cat:notation:set:arrow}
    We shall often refer to an arrow $(a,b,f)$ of the category \Set\
    i.e. a typed function, simply as its untyped counterpart~$f$. 
    The context should make it clear that $f$ actually refers to a 
    typed function.
\end{notation}
\noindent
{\bf Remark} So on top of its usual set-theoretic meaning for untyped
functions, the notation $f:a\to b$ may also have its categorical meaning 
for typed functions, expressing the fact that $f$ is an arrow of the 
category \Set\ with domain $a$ and codomain $b$, i.e. that $f$ is really 
the typed function $(a,b,f)$. 

Whenever $f:a\to b$ is an arrow of the category \Set\ and $x\in a$, the
notation $f(x)$ is not strictly speaking meaningful since $f$ is not
a function but a typed function, i.e a tuple $(a,b,f)$. However, it is
natural enough to set:

\begin{notation}\label{Cat:notation:set:arrow:apply}
    If $f:a \to b$ is an arrow of the category \Set, 
    for all $x\in a$ we shall write $f(x)$ as a shortcut 
    for $f(x)$ where $f$ is the underlying untyped function.
\end{notation}

Functions in set theory are just sets, and equality between functions is 
simply the standard equality between sets. As it turns out, if $f$ and $g$
are two functions with the same domain $a$, then the set equality $f = g$
is equivalent to the {\em extensional} equality $\forall x\in a\ ,\ f(x)=g(x)$:

\begin{prop}\label{Cat:prop:functions:extensionality}
    Let $f,g$ be two functions with identical domain $a$. We have:
        \[
        f = g\ \Leftrightarrow\ \forall x\in a\ ,\ f(x)=g(x)
        \]
\end{prop}
\begin{proof}
    $(\Rightarrow)$: We assume that $f=g$ and $x\in a$. We need to show that 
    $f(x)=g(x)$. However $f(x)$ is defined as the unique set $y$ such that
    $(x,y)\in f$, while $g(x)$ is the unique set $y$ such that $(x,y)\in g$.
    Having assumed that $f=g$, both $f(x)$ and $g(x)$ are the unique set $y$
    such that $(x,y)\in f$. So we must have $f(x)=g(x)$ by uniqueness.

    \noindent
    $(\Leftarrow)$: We assume that $f(x)=g(x)$ for all $x\in a$. We need to show
    that $f=g$. Hence we need to show that $f\subseteq g$ and $g\subseteq f$.
    By symmetry, we can focus on proving $f\subseteq g$ as the same proof will
    carry over for $g\subseteq f$. So suppose $z\in f$. We need to show that 
    $z\in g$. Having assumed that $f$ is a function, the element $z$ must be
    an ordered pair $z=(x,y)$. Hence we have $(x,y)\in f$. Having assumed
    that the domain of $f$ is $a$, this shows that $x\in a$. Furthermore,
    from $(x,y)\in f$ we obtain $f(x)=y$. Hence, by assumption we obtain
    $g(x)=y$. However $g(x)$ is the unique set $y'$ with $(x,y')\in g$.
    Hence, we have $(x,y)\in g$ and finally $z\in g$.
\end{proof}

An important question which will invariably arise is deciding when two
arrows of the category \Set\ are equal. Although we have defined an
arrow to be a typed function $(a,b,f)$, as indicated in 
notation~(\ref{Cat:notation:set:arrow}) we will often refer to such
arrow simply as $f$. However, we should not forget that the equality
$f=g$ between underlying untyped functions is not enough for two arrows
$(a,b,f)$ and $(c,d,g)$ to be equal. The equality between untyped functions
will ensure that they have the same domain and the same
range, but it does not tell us anything
about the intended codomains of their respective typed functions.

\begin{prop}\label{Cat:prop:functions:equal:domain}
    If two functions $f,g$ are equal, they have the same domain.
\end{prop}
\begin{proof}
    We assume $f,g$ are functions and $f=g$. We need to show $f$ and
    $g$ have the same domain. By symmetry, it is sufficient to show
    that the domain of $f$ is a subset of that of $g$. So let $x$
    be an element of the domain of $f$. There exists some $y$ with
    $(x,y)\in f$. From $f=g$ we obtain $(x,y)\in g$ and consequently
    $x$ is also an element of the domain of $g$.
\end{proof}


\begin{prop}\label{Cat:prop:functions:equal:range}
    If two functions $f,g$ are equal, they have the same range.
\end{prop}
\begin{proof}
    We assume $f,g$ are functions and $f=g$. We need to show $f$ and
    $g$ have the same range. By symmetry, it is sufficient to show
    that the range of $f$ is a subset of that of $g$. So let $y$
    be an element of the range of $f$. There exists some $x$ with
    $(x,y)\in f$. From $f=g$ we obtain $(x,y)\in g$ and consequently
    $y$ is also an element of the range of $g$.
\end{proof}

\begin{prop}\label{Cat:prop:set:arrow:equal}
    Let $f:a\to b$ and $g:c\to d$ be two arrows of the category \Set. Then 
    $f=g$ \ifand\ $a=c$, $b=d$ and $f(x)=g(x)$ for all $x\in a$.
\end{prop}
\begin{proof}
    Let $f^{*}:a \to b$ and $g^{*}:c \to d$, that is $f^{*}=(a,b,f)$ 
    (for some $f$) and $g^{*}=(c,d,g)$ (for some $g$) be two arrows
    of the category \Set. We call these arrows $f^{*}$ and $g^{*}$ in
    this proof so as to be very precise on the distinction between
    an arrow (a typed function) and its underlying function (an
    untyped function). Looking at definition~(\ref{Cat:def:set}), 
    $f$ and $g$ are functions $f:a\to b$ and $g:c\to d$. So the 
    domain of $f$ is the set $a$ while the domain of $g$ 
    is the set $c$, and the range of $f$ is a subset of $b$ while the range of 
    $g$ is a subset of $d$. First we assume that $f^{*}=g^{*}$. Then we have 
    the equality between triples $(a,b,f)=(c,d,g)$. Hence we obtain immediately
    $a=c$ and $b=d$ as requested. However, we also obtain the equality
    between underlying functions $f=g$. Using 
    proposition~(\ref{Cat:prop:functions:extensionality}), we see
    that $f(x)=g(x)$ for all $x\in a$. By virtue of
    notation~(\ref{Cat:notation:set:arrow:apply}), $f^{*}(x)$ and 
    $g^{*}(x)$ are notational shortcuts for $f(x)$ and $g(x)$ 
    respectively. Hence we have $f^{*}(x)=g^{*}(x)$ for all $x\in a$
    as requested. We now assume that $a=c$ , $b=d$ and $f^{*}(x)=g^{*}(x)$
    for all $x\in a$. We need to show that $f^{*}=g^{*}$, or in other
    words $(a,b,f)=(c,d,g)$. Hence it remains to show that $f=g$. This
    follows from proposition~(\ref{Cat:prop:functions:extensionality}) and
    the fact that $f(x)=g(x)$ for all $x\in a$.
\end{proof}

\begin{prop}\label{Cat:prop:set:not:small}
    The category Set is not small.
\end{prop}
\begin{proof}
    Assume \Set\ is a small category. Then $\Set=(\ob, \arr, \dom, 
    \cod, \id, \circ)$ where the entries of the tuple satify 
    conditions $(1)-(13)$ of definition~(\ref{Cat:def:category:small}).
    In particular $\ob$ is a set. Using
    axiom~(\ref{Cat:ax:tuple:extensional}), from the equality 
    $\Set=(\ob, \arr, \dom, \cod, \id, \circ)$ we obtain in 
    particular $\ob\ \Set=\ob$. This is not a standard equality
    between sets since $\ob\ \Set$ was defined 
    in~(\ref{Cat:def:set}) as the collection 
    $\{\ x\ |\ \mbox{x is a set}\ \}$ and not as a set.
    However, it is an equality between collections and 
    using axiom~(\ref{Cat:ax:collection:extensional}), it follows
    that the set $\ob$ and the collection 
    $\{\ x\ |\ \mbox{x is a set}\ \}$ have identical members.
    So we have found a set \ob\ whose members are all possible sets.
    This is a contradiction as no such set exists.
\end{proof}

\noindent
{\bf Remark}: These notes do not assume any specific formal 
foundations and the proof of 
proposition~(\ref{Cat:prop:set:not:small}) is therefore standing
on shaky grounds. However, we feel some form of formal reasoning
is better than nothing at all.
