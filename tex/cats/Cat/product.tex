\begin{defin}\label{Cat:def:canonical:product}
    We call {\em canonical product} of categories \Cone\ and \Ctwo\ the category 
    denoted $\Cone\times\Ctwo$ and defined by $\Cone\times\Ctwo=
    (\ob, \arr, \dom, \cod, \id, \circ)$ where:
        \begin{eqnarray*}
            (1)&\ &\ob=\{\ (x_{1},x_{2})\ |\ x_{1}\in\ob\ \Cone\ 
                                          ,\ x_{2}\in\ob\ \Ctwo\ \}\\
            (2)&\ &\arr=\{\ (f_{1},f_{2})\ |\ f_{1}\in\arr\ \Cone\ 
                                           ,\ f_{2}\in\arr\ \Ctwo\ \}\\
            (3)&\ &\dom(f_{1},f_{2})=(\,\dom(f_{1})\,,\,\dom(f_{2})\,)\\
            (4)&\ &\cod(f_{1},f_{2})=(\,\cod(f_{1})\,,\,\cod(f_{2})\,)\\
            (5)&\ &\id(x_{1},x_{2}) =(\,\id(x_{1})\,,\,\id(x_{2})\,)\\
            (6)&\ &(g_{1},g_{2})\circ(f_{1},f_{2})=(\,g_{1}\circ f_{1}\,
                                                   ,\,g_{2}\circ f_{2}\,)
        \end{eqnarray*}
    where $(3)$ and $(4)$ hold for all $f_{1}\in\arr\ \Cone$ and $f_{2}
    \in\arr\ \Ctwo$, $(5)$ holds for all $x_{1}\in\ob\ \Cone$ and $x_{2}
    \in\ob\ \Ctwo$, and $(6)$ holds for all $f_{1},g_{1}\in\arr\ \Cone$
    and $f_{2},g_{2}\in\arr\ \Ctwo$ for which $g_{1}\circ f_{1}$ and
    $g_{2}\circ f_{2}$ are defined.
\end{defin}

So if \Cone\ and \Ctwo\ are two categories, the objects of $\Cone\times\Ctwo$
are the collection of all tuples $(x_{1}, x_{2})$ where $x_{1}$ is an
object of \Cone\ and $x_{2}$ is an object of \Ctwo. The set comprehension 
notation $\{\ (x_{1},x_{2})\ |\ x_{1}\in\ob\ \Cone\ ,\ x_{2}\in\ob\ \Ctwo\ \}$
if of course an abuse of notation as it does not in general represent a set
but a collection. We could also have denoted this collection 
$\ob\ \Cone\times\ob\ \Ctwo$ using a cartesian product notation, keeping 
in mind that this is a product of two collections.

Similarly, the arrows of $\Cone\times\Ctwo$ are the collection of all tuples
$(f_{1},f_{2})$ where $f_{1}$ is an arrow of \Cone\ and $f_{2}$ is an arrow
of \Ctwo, a collection which could reasonably be denoted $\arr\ \Cone\times
\arr\ \Ctwo$ instead of the set-comprehension notation.

It should be clear from definition~(\ref{Cat:def:canonical:product}) that
the notations '$\dom$', '$\cod$', '$\id$' and '$\circ$' are overloaded, 
referring either to \Cone, \Ctwo\ or $\Cone\times\Ctwo$. Given our definitions 
of $\ob\ (\Cone\times\Ctwo)$ and $\arr\ (\Cone\times\Ctwo)$, given that we 
have $\dom:\arr\ \Cone\to\ob\ \Cone$ and $\dom:\arr\ \Ctwo\to\ob\ \Ctwo$ it 
should be clear that $(3)$ of definition~(\ref{Cat:def:canonical:product})
defines a map $\dom:\arr\ (\Cone\times\Ctwo)\to\ob\ (\Cone\times\Ctwo)$,
and $\cod:\arr\ (\Cone\times\Ctwo)\to\ob\ (\Cone\times\Ctwo)$
follows from $(4)$. Furthermore from $\id:\ob\ \Cone\to\arr\ \Cone$
and $\id:\ob\ \Ctwo\to\arr\ \Ctwo$ we obtain $\id:\ob\ (\Cone\times\Ctwo)
\to\arr\ (\Cone\times\Ctwo)$ using $(5)$. Finally using $(6)$, given 
the partial maps $\circ:\arr\ \Cone\times\arr\ \Cone\to\arr\ \Cone$
and $\circ:\arr\ \Ctwo\times\arr\ \Ctwo\to\arr\ \Ctwo$ we obtain
a partial map $\circ:\arr\ (\Cone\times\Ctwo)\times\arr
\ (\Cone\times\Ctwo)\to\arr\ \Cone\times\Ctwo$.

\begin{prop}\label{Cat:prop:canonical:product:is:category}
    The canonical product $\Cone\times\Ctwo$ of 
    definition~(\ref{Cat:def:canonical:product}) is a category.
\end{prop}
\begin{proof}
    We need to check that the data $\Cone\times\Ctwo=(\ob, \arr, \dom, 
    \cod, \id, \circ)$ of definition~(\ref{Cat:def:canonical:product})
    forms a category, having assumed \Cone\ and \Ctwo\ are categories.
\end{proof}
