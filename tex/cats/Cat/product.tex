\begin{defin}\label{Cat:def:canonical:product}
    We call {\em canonical product} of categories \Cone\ and \Ctwo\ the category 
    denoted $\Cone\times\Ctwo$ and defined by $\Cone\times\Ctwo=
    (\ob, \arr, \dom, \cod, \id, \circ)$ where:
        \begin{eqnarray*}
            (1)&\ &\ob=\{\ (x_{1},x_{2})\ |\ x_{1}\in\ob\ \Cone\ 
                                          ,\ x_{2}\in\ob\ \Ctwo\ \}\\
            (2)&\ &\arr=\{\ (f_{1},f_{2})\ |\ f_{1}\in\arr\ \Cone\ 
                                           ,\ f_{2}\in\arr\ \Ctwo\ \}\\
            (3)&\ &\dom(f_{1},f_{2})=(\,\dom(f_{1})\,,\,\dom(f_{2})\,)\\
            (4)&\ &\cod(f_{1},f_{2})=(\,\cod(f_{1})\,,\,\cod(f_{2})\,)\\
            (5)&\ &\id(x_{1},x_{2}) =(\,\id(x_{1})\,,\,\id(x_{2})\,)\\
            (6)&\ &(g_{1},g_{2})\circ(f_{1},f_{2})=(\,g_{1}\circ f_{1}\,
                                                   ,\,g_{2}\circ f_{2}\,)
        \end{eqnarray*}
    where $(3)$ and $(4)$ hold for all $f_{1}\in\arr\ \Cone$ and $f_{2}
    \in\arr\ \Ctwo$, $(5)$ holds for all $x_{1}\in\ob\ \Cone$ and $x_{2}
    \in\ob\ \Ctwo$, and $(6)$ holds for all $f_{1},g_{1}\in\arr\ \Cone$
    and $f_{2},g_{2}\in\arr\ \Ctwo$ for which $g_{1}\circ f_{1}$ and
    $g_{2}\circ f_{2}$ are defined.
\end{defin}

So if \Cone\ and \Ctwo\ are two categories, the objects of $\Cone\times\Ctwo$
are the collection of all tuples $(x_{1}, x_{2})$ where $x_{1}$ is an
object of \Cone\ and $x_{2}$ is an object of \Ctwo. The set comprehension 
notation $\{\ (x_{1},x_{2})\ |\ x_{1}\in\ob\ \Cone\ ,\ x_{2}\in\ob\ \Ctwo\ \}$
is of course an abuse of notation as it does not in general represent a set
but a collection. We could also have denoted this collection 
$\ob\ \Cone\times\ob\ \Ctwo$ using a cartesian product notation, keeping 
in mind that this is a product of two collections.

Similarly, the arrows of $\Cone\times\Ctwo$ are the collection of all tuples
$(f_{1},f_{2})$ where $f_{1}$ is an arrow of \Cone\ and $f_{2}$ is an arrow
of \Ctwo, a collection which could reasonably be denoted $\arr\ \Cone\times
\arr\ \Ctwo$ instead of the set-comprehension notation.

It should be clear from definition~(\ref{Cat:def:canonical:product}) that
the notations $\dom$, $\cod$, $\id$ and $\circ$ are overloaded, 
referring either to \Cone, \Ctwo\ or $\Cone\times\Ctwo$. Given our definitions 
of $\ob\ (\Cone\times\Ctwo)$ and $\arr\ (\Cone\times\Ctwo)$, given that we 
have $\dom:\arr\ \Cone\to\ob\ \Cone$ and $\dom:\arr\ \Ctwo\to\ob\ \Ctwo$ it 
should be clear that $(3)$ of definition~(\ref{Cat:def:canonical:product})
defines a map $\dom:\arr\ (\Cone\times\Ctwo)\to\ob\ (\Cone\times\Ctwo)$,
and $\cod:\arr\ (\Cone\times\Ctwo)\to\ob\ (\Cone\times\Ctwo)$
follows from $(4)$. Furthermore from $\id:\ob\ \Cone\to\arr\ \Cone$
and $\id:\ob\ \Ctwo\to\arr\ \Ctwo$ we obtain $\id:\ob\ (\Cone\times\Ctwo)
\to\arr\ (\Cone\times\Ctwo)$ using $(5)$. Finally using $(6)$, given 
the partial maps $\circ:\arr\ \Cone\times\arr\ \Cone\to\arr\ \Cone$
and $\circ:\arr\ \Ctwo\times\arr\ \Ctwo\to\arr\ \Ctwo$ we obtain
a partial map $\circ:\arr\ (\Cone\times\Ctwo)\times\arr
\ (\Cone\times\Ctwo)\to\arr\ \Cone\times\Ctwo$.

\begin{prop}\label{Cat:prop:canonical:product:is:category}
    The canonical product $\Cone\times\Ctwo$ of 
    definition~(\ref{Cat:def:canonical:product}) is a category.
\end{prop}
\begin{proof}
    We need to check that the data $\Cone\times\Ctwo=(\ob, \arr, \dom, 
    \cod, \id, \circ)$ of definition~(\ref{Cat:def:canonical:product})
    forms a category, having assumed \Cone\ and \Ctwo\ are categories.
    We have established that \ob\ and \arr\ are collections, that 
    $\dom,\ \cod,\ \id$ are maps with the appropriate signatures and
    $\circ$ is a partial map with the appropriate signature.
    It remains to check properties $(7)-(13)$ of 
    definition~(\ref{Cat:def:category}).

    $(7)$: Let $f,g\in\arr\ (\Cone\times\Ctwo)$. We need to show that $g\circ f$
    is defined \ifand\ $\cod(f)=\dom(g)$. Let $f_{1},g_{1}\in\arr\ \Cone$ and
    $f_{2},g_{2}\in\arr\ \Ctwo$ such that $f=(f_{1},f_{2})$ and $g=(g_{1},g_{2})$.
    Then $\cod(f)=(\cod(f_{1}),\cod(f_{2}))$ and 
    $\dom(g)=(\dom(g_{1}),\dom(g_{2}))$. So we need to show that $g\circ f$ is
    defined \ifand\ $\cod(f_{1})=\dom(g_{1})$ and $\cod(f_{2})=\dom(g_{2})$.
    However since \Cone\ and \Ctwo\ are categories, $\cod(f_{1})=\dom(g_{1})$
    is equivalent to $g_{1}\circ f_{1}$ being defined, and 
    $\cod(f_{2})=\dom(g_{2})$ is equivalent to $g_{2}\circ f_{2}$ being defined.
    So we need to show that $g\circ f$ is defined \ifand\ both $g_{1}\circ f_{1}$
    and $g_{2}\circ f_{2}$ are defined which follows exactly from 
    definition~(\ref{Cat:def:canonical:product}).

    $(8)$: Let $f=(f_{1},f_{2})$ and $g=(g_{1},g_{2})$ be arrows in $\Cone\times
    \Ctwo$ with the equality $\cod(f)=\dom(g)$, i.e. for which $g_{1}\circ f_{1}$
    and $g_{2}\circ f_{2}$ are defined. We need to check the equality
    $\dom(g\circ f) = \dom(f)$ which goes as follows:
        \begin{eqnarray*}\dom(g\circ f)
            &=&\dom(\,(g_{1},g_{2})\,\circ\,(f_{1},f_{2})\,)\\
            \mbox{$(6)$ of def.~(\ref{Cat:def:canonical:product})\ $\to$\ }
            &=&\dom(\,(g_{1}\circ f_{1}\,,\,g_{2}\circ f_{2})\,)\\
            \mbox{$(3)$ of def.~(\ref{Cat:def:canonical:product})\ $\to$\ }
            &=&(\,\dom(g_{1}\circ f_{1})\,,\,\dom(g_{2}\circ f_{2})\,)\\
            \mbox{\Cone,\ \Ctwo\ categories, 
                $(8)$ of def.~(\ref{Cat:def:category})\ $\to$\ }
            &=&(\,\dom(f_{1})\,,\,\dom(f_{2})\,)\\
            \mbox{$(3)$ of def.~(\ref{Cat:def:canonical:product})\ $\to$\ }
            &=&\dom(f_{1},f_{2})\\
            &=&\dom(f)
        \end{eqnarray*}

    $(9)$: Let $f=(f_{1},f_{2})$ and $g=(g_{1},g_{2})$ be arrows in $\Cone\times
    \Ctwo$ with the equality $\cod(f)=\dom(g)$, i.e. for which $g_{1}\circ f_{1}$
    and $g_{2}\circ f_{2}$ are defined. We need to check the equality
    $\cod(g\circ f) = \cod(g)$ which goes as follows:
        \begin{eqnarray*}\cod(g\circ f)
            &=&\cod(\,(g_{1},g_{2})\,\circ\,(f_{1},f_{2})\,)\\
            \mbox{$(6)$ of def.~(\ref{Cat:def:canonical:product})\ $\to$\ }
            &=&\cod(\,(g_{1}\circ f_{1}\,,\,g_{2}\circ f_{2})\,)\\
            \mbox{$(4)$ of def.~(\ref{Cat:def:canonical:product})\ $\to$\ }
            &=&(\,\cod(g_{1}\circ f_{1})\,,\,\cod(g_{2}\circ f_{2})\,)\\
            \mbox{\Cone,\ \Ctwo\ categories, 
                $(9)$ of def.~(\ref{Cat:def:category})\ $\to$\ }
            &=&(\,\cod(g_{1})\,,\,\cod(g_{2})\,)\\
            \mbox{$(4)$ of def.~(\ref{Cat:def:canonical:product})\ $\to$\ }
            &=&\cod(g_{1},g_{2})\\
            &=&\cod(g)
        \end{eqnarray*}

    $(10)$: Let $f=(f_{1},f_{2})$, $g=(g_{1},g_{2})$ and $h=(h_{1},h_{2})$ be
    arrows in $\Cone\times\Ctwo$ with the equalities $\cod(f)=\dom(g)$ and
    $\cod(g)=\dom(h)$, i.e. for which the composition arrows $g_{1}\circ f_{1}$,
    $g_{2}\circ f_{2}$, $h_{1}\circ g_{1}$ and $h_{2}\circ g_{2}$ are defined.
    We need to check the equality $(h\circ g)\circ f=h\circ(g\circ f)$ 
    which goes as follows:
        \begin{eqnarray*}(h\circ g)\circ f
            &=&(\,(h_{1},h_{2})\,\circ\,(g_{1},g_{2})\,)\,\circ\,(f_{1},f_{2})\\ 
            \mbox{$(6)$ of def.~(\ref{Cat:def:canonical:product})\ $\to$\ }
            &=&(\,h_{1}\circ g_{1}\,,\,h_{2}\circ g_{2}\,)\,\circ\,(f_{1},f_{2})\\
            \mbox{$(6)$ of def.~(\ref{Cat:def:canonical:product})\ $\to$\ }
            &=&(\,(h_{1}\circ g_{1})\circ f_{1}\,,\,
                   (h_{2}\circ g_{2})\circ f_{2}\,)\\
            \mbox{\Cone,\ \Ctwo\ categories, 
                $(10)$ of def.~(\ref{Cat:def:category})\ $\to$\ }
            &=&(\,h_{1}\circ (g_{1}\circ f_{1})\,,\,
                   h_{2}\circ(g_{2}\circ f_{2})\,)\\
            \mbox{$(6)$ of def.~(\ref{Cat:def:canonical:product})\ $\to$\ }
            &=&(h_{1},h_{2})\circ(\,(g_{1}\circ f_{1})\,,\,(g_{2}\circ f_{2})\,)\\
            \mbox{$(6)$ of def.~(\ref{Cat:def:canonical:product})\ $\to$\ }
            &=&(h_{1},h_{2})\circ(\,(g_{1},g_{2})\,\circ(f_{1},f_{2})\,)\\
            &=&h\circ(g\circ f)
        \end{eqnarray*}

    $(11)$: Let $a=(a_{1},a_{2})$ be an object in $\Cone\times\Ctwo$. We need
    to check that $\dom\,(\,\id(a)\,)=a=\cod\,(\,\id(a)\,)$ which goes as
    follows:
        \begin{eqnarray*}\dom\,(\,\id(a)\,)
            &=&\dom\,(\,\id(a_{1},a_{2})\,)\\
            \mbox{$(5)$ of def.~(\ref{Cat:def:canonical:product})\ $\to$\ }
            &=&\dom\,(\,\id(a_{1})\,,\,\id(a_{2})\,)\\
            \mbox{$(3)$ of def.~(\ref{Cat:def:canonical:product})\ $\to$\ }
            &=&(\,\dom\,(\,\id(a_{1})\,)\,,\,\dom\,(\,\id(a_{2})\,)\,)\\
            \mbox{\Cone,\ \Ctwo\ categories, 
                $(11)$ of def.~(\ref{Cat:def:category})\ $\to$\ }
            &=&(a_{1},a_{2})\\
            &=&a
        \end{eqnarray*}
        \begin{eqnarray*}\cod\,(\,\id(a)\,)
            &=&\cod\,(\,\id(a_{1},a_{2})\,)\\
            \mbox{$(5)$ of def.~(\ref{Cat:def:canonical:product})\ $\to$\ }
            &=&\cod\,(\,\id(a_{1})\,,\,\id(a_{2})\,)\\
            \mbox{$(4)$ of def.~(\ref{Cat:def:canonical:product})\ $\to$\ }
            &=&(\,\cod\,(\,\id(a_{1})\,)\,,\,\cod\,(\,\id(a_{2})\,)\,)\\
            \mbox{\Cone,\ \Ctwo\ categories, 
                $(11)$ of def.~(\ref{Cat:def:category})\ $\to$\ }
            &=&(a_{1},a_{2})\\
            &=&a
        \end{eqnarray*}

    $(12)$: Let $f=(f_{1},f_{2})$ be an arrow and $a=(a_{1},a_{2})$ be an 
    object in $\Cone\times\Ctwo$ such that $\dom(f)=a$. We need to show that
    $f\circ\id(a)=f$ which goes as follows: Using $(3)$ of 
    definition~(\ref{Cat:def:canonical:product}) and the condition $\dom(f)=a$ 
    we obtain the equation $(\,\dom(f_{1})\,,\,\dom(f_{2})\,)=(a_{1},a_{2})$.
    Hence, we have:
        \begin{eqnarray*}f\circ\id(a)
            &=&(f_{1},f_{2})\circ\id\,(a_{1},a_{2})\\
            \mbox{$(5)$ of def.~(\ref{Cat:def:canonical:product})\ $\to$\ }
            &=&(f_{1},f_{2})\,\circ\,(\,\id(a_{1})\,,\,\id(a_{2})\,)\\
            \mbox{$(6)$ of def.~(\ref{Cat:def:canonical:product})\ $\to$\ }
            &=&(\,f_{1}\circ\id(a_{1})\,,\,f_{2}\circ\id(a_{2})\,)\\
            \mbox{\Cone\ category, $\dom(f_{1})=a_{1}$\ $\to$\ }
            &=&(f_{1},\,f_{2}\circ\id(a_{2})\,)\\
            \mbox{\Ctwo\ category, $\dom(f_{2})=a_{2}$\ $\to$\ }
            &=&(f_{1},f_{2})\\
            &=&f
        \end{eqnarray*}
    $(13)$: Let $f=(f_{1},f_{2})$ be an arrow and $a=(a_{1},a_{2})$ be an 
    object in $\Cone\times\Ctwo$ such that $\cod(f)=a$. We need to show that
    $\id(a)\circ f=f$ which goes as follows: Using $(4)$ of 
    definition~(\ref{Cat:def:canonical:product}) and the condition $\cod(f)=a$ 
    we obtain the equation $(\,\cod(f_{1})\,,\,\cod(f_{2})\,)=(a_{1},a_{2})$.
    Hence, we have:
        \begin{eqnarray*}\id(a)\circ f
            &=&\id\,(a_{1},a_{2})\,\circ\,(f_{1},f_{2})\\
            \mbox{$(5)$ of def.~(\ref{Cat:def:canonical:product})\ $\to$\ }
            &=&(\,\id(a_{1})\,,\,\id(a_{2})\,)\,\circ\,(f_{1},f_{2})\\
            \mbox{$(6)$ of def.~(\ref{Cat:def:canonical:product})\ $\to$\ }
            &=&(\,\id(a_{1})\circ f_{1}\,,\,\id(a_{2})\circ f_{2}\,)\\
            \mbox{\Cone\ category, $\cod(f_{1})=a_{1}$\ $\to$\ }
            &=&(f_{1},\,\id(a_{2})\circ f_{2}\,)\\
            \mbox{\Ctwo\ category, $\cod(f_{2})=a_{2}$\ $\to$\ }
            &=&(f_{1},f_{2})\\
            &=&f
        \end{eqnarray*}
This completes our proof of properties~$(7)-(13)$.
\end{proof}
