\begin{defin}\label{Adj:def:unit}
    Let $F:\Cat\to\Dat$ and $G:\Dat\to\Cat$ be functors where \Cat\ and \Dat\ 
    are categories. We call {\em unit} of the typed functors $(\Cat,\Dat,F)$
    and $(\Dat,\Cat,G)$, a natural transformation 
    $\eta:I_{\cal C}\Rightarrow G\circ F$ such that for all 
    $c\in\Cat$, $d\in\Dat$ and $f:c\to G(d)$, there exists 
    unique $g:F(c) \to d$ such that $f = G(g) \circ \eta_{c}$.
\end{defin}

\noindent
{\bf Remark}: The equality $f = G(g) \circ \eta_{c}$ can be visualized with
the diagram:
    \[
        \begin{tikzcd}
            c \arrow[rd,swap, "f"]\arrow[r,"\eta_{c}"] 
            &G(\,F(c)\,)\arrow[d,"G(\,g\,)"]\\
            & G(\,d\,)
        \end{tikzcd}
    \]

\noindent
{\bf Remark}: Recall from definition~(\ref{Nat:def:natural}) that $\eta$ being
a natural transformation is a statement about typed functors, not just functors. 
The same is true of a unit. Being a unit of $F$ and $G$ viewed as functors 
$F:\Cat\to\Dat$ and $G:\Dat\to\Cat$ is not the same as being a unit of $F$ 
and $G$ viewed as functor $F:\Cop\to\Dop$ and $G:\Dop\to\Cop$. It is important
to remember which categories are being considered, hence typed functors. The
order also matters. Being a unit of $F$ and $G$ is not the same as being a unit
of $G$ and $F$.

\begin{notation}\label{Adj:notation:unit}
    Whenever the categories \Cat\ and \Dat\ are clearly undertood from the 
    context, we shall simply say that $\eta$ {\em is a unit of} the
    ordered pair $(F,G)$.
\end{notation}

