In this section, we are interested in pairs of functors as follows:
    \[
        \begin{tikzcd}
              \Cat \arrow[r, "F", bend left  = 50, ""{name=U, below}]
            & \Dat \arrow[l, "G", bend left = 50, ""{name=D, above}]
        \end{tikzcd}
    \]

\begin{defin}\label{Adj:def:unit}
    Let $F:\Cat\to\Dat$ and $G:\Dat\to\Cat$ be functors where \Cat\ and \Dat\ 
    are categories. We call {\em unit} of the typed functors $(\Cat,\Dat,F)$
    and $(\Dat,\Cat,G)$, a natural transformation 
    $\eta:I_{\cal C}\Rightarrow G\circ F$ such that for all 
    $c\in\Cat$, $d\in\Dat$ and $f:c\to G(d)$, there exists a
    unique $g:F(c) \to d$ such that $f = G(g) \circ \eta_{c}$.
\end{defin}

\noindent
{\bf Remark}: The equality $f = G(g) \circ \eta_{c}$ can be visualized with
the diagram:
    \[
        \begin{tikzcd}
            c \arrow[rd,swap, "f"]\arrow[r,"\eta_{c}"] 
            &G(\,F(c)\,)\arrow[d,"G(\,g\,)"]\\
            & G(\,d\,)
        \end{tikzcd}
    \]

\noindent
{\bf Remark}: Recall from definition~(\ref{Nat:def:natural}) that $\eta$ being
a natural transformation is a statement about typed functors, not just functors. 
The same is true of a unit. Being a unit of $F$ and $G$ viewed as functors 
$F:\Cat\to\Dat$ and $G:\Dat\to\Cat$ is not the same as being a unit of $F$ 
and $G$ viewed as functor $F:\Cop\to\Dop$ and $G:\Dop\to\Cop$. It is important
to remember which categories are being considered, hence typed functors. The
order also matters. Being a unit of $F$ and $G$ is not the same as being a unit
of $G$ and $F$.

\begin{notation}\label{Adj:notation:unit}
    Whenever the categories \Cat\ and \Dat\ are clearly undertood from the 
    context, we shall simply say that $\eta$ {\em is a unit of} the
    ordered pair $(F,G)$.
\end{notation}

\begin{prop}\label{Adj:prop:unit:proving:equality}
    Let $F:\Cat\to\Dat$ and $G:\Dat\to\Cat$ be functors where \Cat\ and \Dat\ 
    are categories, and let $\eta:I_{\cal C}\Rightarrow G\circ F$ be a unit 
    of $(F,G)$. Then, for all $c\in\Cat$, $d\in\Dat$ and $g_{1},g_{2}:F(c)\to d$,
    we have the implication:
        \[
            G(g_{1})\circ\eta_{c} = G(g_{2})\circ\eta_{c}\ 
            \Rightarrow\ 
            g_{1}=g_{2}
        \]
\end{prop}
\begin{proof}
    This is an immediate consequence of the uniqueness property of
    definition~(\ref{Adj:def:unit}). Consider the arrow 
    $f=G(g_{1})\circ\eta_{c}$. Since $g_{1}:F(c)\to d$ and $G$ is a 
    functor, we have 
    $G(g_{1}):G(F(c))\to G(d)$ and since $\eta_{c}:c\to G(F(c))$, $f$
    is a well-defined arrow $f:c\to G(d)$ in the category \Cat. Having
    assumed that $\eta$ is a unit of $(F,G)$, from 
    definition~(\ref{Adj:def:unit}) there exists a unique arrow 
    $g:F(c)\to d$ in the category \Dat\ such that $f=G(g)\circ\eta_{c}$.
    However the equality $f=G(g)\circ\eta_{c}$ is satified by the 
    arrow $g=g_{1}$ since we defined $f$ as $f=G(g_{1})\circ\eta_{c}$.
    Now if we assume that $G(g_{1})\circ\eta_{c}=G(g_{2})\circ\eta_{c}$,
    then the equality $f=G(g)\circ\eta_{c}$ is also satisfied by $g_{2}$.
    By uniquenes, it follows that $g_{1}=g_{2}$.
\end{proof}

\begin{prop}\label{Adj:prop:unit:natural}
    Let $F:\Cat\to\Dat$ and $G:\Dat\to\Cat$ be functors where \Cat\ and \Dat\ 
    are categories, and let $\eta:I_{\cal C}\Rightarrow G\circ F$ be a unit 
    of $(F,G)$. Then we have:
        \[
            (G\circ F)(f)\circ\eta_{a}=\eta_{b}\circ f
        \]
    for all objects $a,b\in\Cat$ and arrow $f:a\to b$.
\end{prop}
\begin{proof}
    This is an immediate consequence of definition~(\ref{Nat:def:natural}) 
    and the fact that $\eta$ is a natural transformation 
    $\eta:I_{\cal C}\Rightarrow G\circ F$, so the following square commutes:
    \[
        \begin{tikzcd}
            a\arrow[r, "\eta_{a}"]\arrow[d, swap,"f"]
            &G(F(a))\arrow[d, "\ (G\circ F)(f)"]
            \\
            b\arrow[r, swap, "\eta_{b}"]
            &G(F(b))
        \end{tikzcd}
    \]
\end{proof}

Given functors $F:\Cat\to\Dat$ and $G:\Dat\to\Cat$ where \Cat, \Dat\ are 
categories, if $\eta:I_{\cal C}\Rightarrow G\circ F$ is a unit of $(F,G)$,
then a question arises as to whether this unit is unique. As the following
proposition shows, the answer is 'no' in general as every natural 
isomorphism $\alpha : F\Rightarrow F$ potentially gives rise to a 
new unit.

\begin{prop}\label{Adj:prop:unit:not:unique}
    Let $F:\Cat\to\Dat$ and $G:\Dat\to\Cat$ be functors where \Cat\ and \Dat\ 
    are categories, and let $\eta:I_{\cal C}\Rightarrow G\circ F$ be a unit
    of $(F,G)$. Then, a natural transformation $\eta':I_{\cal C}\Rightarrow 
    G\circ F$ is a unit of $(F,G)$ \ifand\ there exists a natural isomorphism 
    $\alpha:F\Rightarrow F$ such that:
        \[
            \eta'=(G\alpha)\circ\eta
        \]
\end{prop}

\noindent
{\bf Remark}: If $\alpha: F\Rightarrow F$ and $G:\Dat\to\Cat$, then $G\alpha$
is the natural transformation $G\alpha:G\circ F\Rightarrow G\circ F$ as per
definition~(\ref{Nat:def:leftmul}), and if $\eta:I_{\cal C}\Rightarrow G\circ F$
then the composition $(G\alpha)\circ\eta$ is a well-defined natural 
transformation from the identity functor $I_{\cal C}$ to the functor $G\circ F$ 
as per definition~(\ref{Nat:def:composition}).

\noindent
\begin{proof}
    First we show the {\em if} part. So we assume that $\eta'=(G\alpha)\circ\eta$
    for some natural isomorphism $\alpha:F\Rightarrow F$. We need to show that
    $\eta'$ is a unit of $(F,G)$. As already indicated $\eta'$ is a well-defined
    natural transformation $\eta':I_{\cal C}\Rightarrow G\circ F$. Hence we
    simply need to show that it satisfies the universal property of 
    definition~(\ref{Adj:def:unit}). So let $c\in\Cat$, $d\in\Dat$ and 
    $f:c\to G(d)$. We need to show the existence of a unique $g:F(c)\to d$
    such that $f = G(g)\circ\eta'_{c}$. However, by assumption we have:
        \begin{eqnarray*}\eta'_{c}
            &=&((G\alpha)\circ\eta)_{c}\\
            \mbox{$(1)$ of def.~(\ref{Nat:def:composition})}\ \to\ 
            &=&(G\alpha)_{c}\circ\eta_{c}\\
            \mbox{$(1)$ of def.~(\ref{Nat:def:leftmul})}\ \to\ 
            &=&G(\,\alpha_{c}\,)\circ\eta_{c}\\
        \end{eqnarray*}
    So we need to prove the existence of a unique arrow $g:F(c)\to d$
    such that $f=G(g)\circ G(\alpha_{c})\circ\eta_{c}$, which is
    $f=G(g\circ\alpha_{c})\circ\eta_{c}$ since $G$ is a functor:

    Existence: having assumed $\eta$ is a unit of $(F,G)$, since 
    $f:c\to G(d)$ using definition~(\ref{Adj:def:unit}) there exists
    a (unique) $h:F(c)\to d$ with $f = G(h)\circ\eta_{c}$.
    However, by assumption $\alpha:F\Rightarrow F$ is a natural
    isomorphism. From proposition~(\ref{Nat:prop:isomorphism:component}),
    every component of $\alpha$ is an isomorphism and in particular
    $\alpha_{c}:F(c)\to F(c)$ has an inverse $\alpha_{c}^{-1}:F(c)\to F(c)$.
    Defining $g=h\circ\alpha_{c}^{-1}$ we obtain an arrow $g:F(c)\to d$ such
    that $f=G(g\circ\alpha_{c})\circ\eta_{c}$ as requested.

    Uniqueness: Suppose $g_{1},g_{2}:F(c)\to d$ are two arrows in \Dat\ 
    such that $G(g_{1}\circ\alpha_{c})\circ\eta_{c}=f=G(g_{2}\circ\alpha_{c})
    \circ\eta_{c}$. Having assumed $\eta$ is a unit of $(F,G)$ using
    proposition~(\ref{Adj:prop:unit:proving:equality}) we obtain 
    $g_{1}\circ\alpha_{c}=g_{2}\circ\alpha_{c}$ and composing both sides
    to the right by $\alpha_{c}^{-1}$ we conclude that $g_{1}=g_{2}$
    as requested.

    We now prove the {\em only if} part. So we assume that 
    $\eta':I_{\cal C}\Rightarrow G\circ F$ is a unit of $(F,G)$ and we need
    to show the existence of a natural isomorphism $\alpha:F\Rightarrow F$
    such that $\eta'=(G\alpha)\circ\eta$. So let $c\in\Cat$. First we need
    to define $\alpha_{c}:F(c)\to F(c)$. Define $d=F(c)\in\Dat$. Then we 
    have $\eta'_{c}:c\to G(d)$. Since $\eta$ is a unit of $(F,G)$ there exists
    a unique arrow $g:F(c)\to d$ such that $\eta'_{c}=G(g)\circ\eta_{c}$.
    Define $\alpha_{c}$ to be precisely this unique arrow $g$. Then we 
    have $\alpha_{c}:F(c)\to F(c)$ and:
        \begin{equation}\label{Adj:eqn:unit:not:unique}
            \eta'_{c}=G(\alpha_{c})\circ\eta_{c}
        \end{equation}
    Collecting all these arrows $\alpha_{c}:F(c)\to F(c)$ for $c\in\Cat$,
    we obtain a map $\alpha:\ob\ \Cat\to\arr\ \Dat$. Next we need to show 
    that $\alpha$ is a natural transformation, so it remains to show that
    condition~$(2)$ of definition~(\ref{Nat:def:natural}) satisfied. So
    let $a,b\in\Cat$ and $f:a\to b$. We need to show the equality
    $F(f)\circ\alpha_{a}=\alpha_{b}\circ F(f)$:
    \[
        \begin{tikzcd}
            a\arrow[d,swap, "f"]
            &F(a)\arrow[r, "\alpha_{a}"]\arrow[d, swap,"F(f)"]
            &F(a)\arrow[d, "F(f)"]
            \\
            b
            &F(b)\arrow[r, swap, "\alpha_{b}"]
            &F(b)
        \end{tikzcd}
    \]
    Define $c=a\in\Cat$ and $d=F(b)\in\Dat$. Then both arrows 
    $F(f)\circ\alpha_{a}$ and $\alpha_{b}\circ F(f)$ have domain $F(c)$ and
    codomain $d$ in \Dat. Appplying 
    proposition~(\ref{Adj:prop:unit:proving:equality}) 
    since $\eta$ is a unit of $(F,G)$, in order to show that these are equal
    it is sufficient to prove:
        \[
            G(\,F(f)\circ\alpha_{a}\,)\circ\eta_{c}=
            G(\,\alpha_{b}\circ F(f)\,)\circ\eta_{c}
        \]
    The proof goes as follows:
        \begin{eqnarray*}G(\,F(f)\circ\alpha_{a}\,)\circ\eta_{c}
            &=&G(\,F(f)\circ\alpha_{a}\,)\circ\eta_{a}\\
            \mbox{$G$ functor}\ \to\ 
            &=&G(F(f))\circ G(\alpha_{a})\circ\eta_{a}\\
            \mbox{eqn.~(\ref{Adj:eqn:unit:not:unique})}\ \to\ 
            &=&G(F(f))\circ\eta'_{a}\\
            \mbox{prop.~(\ref{Adj:prop:unit:natural}), $\eta'$ unit}\ \to\ 
            &=&\eta'_{b}\circ f\\
            \mbox{eqn.~(\ref{Adj:eqn:unit:not:unique})}\ \to\ 
            &=&G(\alpha_{b})\circ\eta_{b}\circ f\\
            \mbox{prop.~(\ref{Adj:prop:unit:natural}), $\eta$ unit}\ \to\ 
            &=&G(\alpha_{b})\circ G(F(f))\circ\eta_{a}\\
            \mbox{$G$ functor}\ \to\ 
            &=&G(\,\alpha_{b}\circ F(f)\,)\circ\eta_{a}\\
            &=&G(\,\alpha_{b}\circ F(f)\,)\circ\eta_{c}\\
        \end{eqnarray*}
    So we have now proved that $\alpha$ is a natural transformation
    $\alpha:F\Rightarrow F$. However, we aim to show that it is actually
    a natural isomorphism. 
    Using proposition~(\ref{Nat:prop:isomorphism:component}), it is sufficient
    to prove that each component $\alpha_{c}:F(c)\to F(c)$ for $c\in\Cat$ is 
    an isomorphisn in \Dat.
\end{proof}
