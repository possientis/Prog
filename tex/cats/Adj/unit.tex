In this section, we are interested in pairs of functors as follows:
    \[
        \begin{tikzcd}
              \Cat \arrow[r, "F", bend left  = 50, ""{name=U, below}]
            & \Dat \arrow[l, "G", bend left = 50, ""{name=D, above}]
        \end{tikzcd}
    \]

\begin{defin}\label{Adj:def:unit}
    Let $F:\Cat\to\Dat$ and $G:\Dat\to\Cat$ be functors where \Cat\ and \Dat\ 
    are categories. We call {\em unit} of the typed functors $(\Cat,\Dat,F)$
    and $(\Dat,\Cat,G)$, a natural transformation 
    $\eta:I_{\cal C}\Rightarrow G\circ F$ such that for all 
    $c\in\Cat$, $d\in\Dat$ and $f:c\to G(d)$, there exists a
    unique $g:F(c) \to d$ such that $f = G(g) \circ \eta_{c}$.
\end{defin}

\noindent
{\bf Remark}: The equality $f = G(g) \circ \eta_{c}$ can be visualized with
the diagram:
    \[
        \begin{tikzcd}
            c \arrow[rd,swap, "f"]\arrow[r,"\eta_{c}"] 
            &G(\,F(c)\,)\arrow[d,"G(\,g\,)"]\\
            & G(\,d\,)
        \end{tikzcd}
    \]

\noindent
{\bf Remark}: Recall from definition~(\ref{Nat:def:natural}) that $\eta$ being
a natural transformation is a statement about typed functors, not just functors. 
The same is true of a unit. Being a unit of $F$ and $G$ viewed as functors 
$F:\Cat\to\Dat$ and $G:\Dat\to\Cat$ is not the same as being a unit of $F$ 
and $G$ viewed as functor $F:\Cop\to\Dop$ and $G:\Dop\to\Cop$. It is important
to remember which categories are being considered, hence typed functors. The
order also matters. Being a unit of $F$ and $G$ is not the same as being a unit
of $G$ and $F$.

\begin{notation}\label{Adj:notation:unit}
    Whenever the categories \Cat\ and \Dat\ are clearly undertood from the 
    context, we shall simply say that $\eta$ {\em is a unit of} the
    ordered pair $(F,G)$.
\end{notation}

\begin{prop}\label{Adj:prop:unit:proving:equality}
    Let $F:\Cat\to\Dat$ and $G:\Dat\to\Cat$ be functors where \Cat\ and \Dat\ 
    are categories, and let $\eta:I_{\cal C}\Rightarrow G\circ F$ be a unit 
    of $(F,G)$. Then, for all $c\in\Cat$, $d\in\Dat$ and $g_{1},g_{2}:F(c)\to d$,
    we have the implication:
        \[
            G(g_{1})\circ\eta_{c} = G(g_{2})\circ\eta_{c}\ 
            \Rightarrow\ 
            g_{1}=g_{2}
        \]
\end{prop}
\begin{proof}
    This is an immediate consequence of the uniqueness property of
    definition~(\ref{Adj:def:unit}). Consider the arrow 
    $f=G(g_{1})\circ\eta_{c}$. Since $g_{1}:F(c)\to d$ and $G$ is a 
    functor, we have 
    $G(g_{1}):G(F(c))\to G(d)$ and since $\eta_{c}:c\to G(F(c))$, $f$
    is a well-defined arrow $f:c\to G(d)$ in the category \Cat. Having
    assumed that $\eta$ is a unit of $(F,G)$, from 
    definition~(\ref{Adj:def:unit}) there exists a unique arrow 
    $g:F(c)\to d$ in the category \Dat\ such that $f=G(g)\circ\eta_{c}$.
    However the equality $f=G(g)\circ\eta_{c}$ is satified by the 
    arrow $g=g_{1}$ since we defined $f$ as $f=G(g_{1})\circ\eta_{c}$.
    Now if we assume that $G(g_{1})\circ\eta_{c}=G(g_{2})\circ\eta_{c}$,
    then the equality $f=G(g)\circ\eta_{c}$ is also satisfied by $g_{2}$.
    By uniquenes, it follows that $g_{1}=g_{2}$.
\end{proof}

Given functors $F:\Cat\to\Dat$ and $G:\Dat\to\Cat$ where \Cat, \Dat\ are 
categories, if $\eta:I_{\cal C}\Rightarrow G\circ F$ is a unit of $(F,G)$,
then a question arises as to whether this unit is unique. As the following
proposition shows, the answer is 'no' in general as every natural 
isomorphism $\alpha : F\Rightarrow F$ potentially gives rise to a 
new unit.

\begin{prop}\label{Adj:prop:unit:not:unique}
    Let $F:\Cat\to\Dat$ and $G:\Dat\to\Cat$ be functors where \Cat\ and \Dat\ 
    are categories, and let $\eta:I_{\cal C}\Rightarrow G\circ F$ be a unit
    of $(F,G)$. Then, a natural transformation $\eta':I_{\cal C}\Rightarrow 
    G\circ F$ is a unit of $(F,G)$ \ifand\ there exists a natural isomorphism 
    $\alpha:F\Rightarrow F$ such that:
        \[
            \eta'=(G\alpha)\circ\eta
        \]
\end{prop}

\noindent
{\bf Remark}: If $\alpha: F\Rightarrow F$ and $G:\Dat\to\Cat$, then $G\alpha$
is the natural transformation $G\alpha:G\circ F\Rightarrow G\circ F$ as per
definition~(\ref{Nat:def:leftmul}), and if $\eta:I_{\cal C}\Rightarrow G\circ F$
then the composition $(G\alpha)\circ\eta$ is a well-defined natural 
transformation from the identity functor $I_{\cal C}$ to the functor $G\circ F$ 
as per definition~(\ref{Nat:def:composition}).

\noindent
\begin{proof}
    First we show the {\em if} part. So we assume that $\eta'=(G\alpha)\circ\eta$
    for some natural isomorphism $\alpha:F\Rightarrow F$. We need to show that
    $\eta'$ is a unit of $(F,G)$. As already indicated $\eta'$ is a well-defined
    natural transformation $\eta':I_{\cal C}\Rightarrow G\circ F$. Hence we
    simply need to show that it satisfies the universal property of 
    definition~(\ref{Adj:def:unit}). So let $c\in\Cat$, $d\in\Dat$ and 
    $f:c\to G(d)$. We need to show the existence of a unique $g:F(c)\to d$
    such that $f = G(g)\circ\eta'_{c}$. However, by assumption we have:
        \begin{eqnarray*}\eta'_{c}
            &=&((G\alpha)\circ\eta)_{c}\\
            \mbox{$(1)$ of def.~(\ref{Nat:def:composition})}\ \to\ 
            &=&(G\alpha)_{c}\circ\eta_{c}\\
            \mbox{$(1)$ of def.~(\ref{Nat:def:leftmul})}\ \to\ 
            &=&G(\,\alpha_{c}\,)\circ\eta_{c}\\
        \end{eqnarray*}
    So we need to prove the existence of a unique arrow $g:F(c)\to d$
    such that $f=G(g)\circ G(\alpha_{c})\circ\eta_{c}$, which is
    $f=G(g\circ\alpha_{c})\circ\eta_{c}$ since $G$ is a functor.
    TODO
\end{proof}
