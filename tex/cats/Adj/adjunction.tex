\begin{defin}\label{Adj:def:adjunction}
    We call {\em adjunction} between two categories \Cat\ and \Dat\ any
    pair of functors $(F,G)$ where $F:\Cat\to\Dat$ and $G:\Dat\to\Cat$ which 
    has a unit.
\end{defin}

\begin{defin}\label{Adj:def:left:right:adjoint}
    Let $F:\Cat\to\Dat$ and $G:\Dat\to\Cat$ be functors where \Cat\ and \Dat\ 
    are categories. We say that $F$ and $G$ are {\em adjoint functors}, 
    or that $F$ is {\em left adjoint} to $G$, or that $G$ is {\em right-adjoint}
    to $F$, \ifand\ $(F,G)$ is an adjunction.
\end{defin}

\begin{notation}\label{Adj:notation:adjunction}
    Let $F:\Cat\to\Dat$ and $G:\Dat\to\Cat$ be functors where \Cat\ and \Dat\ 
    are categories. We write $F\dashv G$ to express the fact that $(F,G)$
    is an adjunction.
\end{notation}

\begin{prop}\label{Adj:prop:adjunction:TFAE}
    Let $F:\Cat\to\Dat$ and $G:\Dat\to\Cat$ be functors where \Cat\ and \Dat\ 
    are categories. Then the following are equivalent:
        \begin{eqnarray*}
            (i)&\ &\mbox{$(F,G)$ is an adjunction between \Cat\ and \Dat}\\
            (ii)&\ &F\dashv G\\
            (iii)&\ &\mbox{$F$ and $G$ are adjoint functors}\\
            (iv)&\ &\mbox{$F$ is left adjoint to $G$}\\
            (v)&\ &\mbox{$G$ is right adjoint to $F$}\\
            (vi)&\ &\mbox{$(F,G)$ has a unit}\\
            (vii)&\ &\mbox{$(F,G)$ has a counit}\\
        \end{eqnarray*}
\end{prop}
\begin{proof}
    $(i)\Leftrightarrow (ii)$: is a restatement of notation~(\ref{Adj:notation:adjunction}).
    \newline\noindent
    $(i)\Leftrightarrow (iii)\Leftrightarrow (iv)\Leftrightarrow (v)$ is restatement
    of definition~(\ref{Adj:def:left:right:adjoint}).
    \newline\noindent
    $(i)\Leftrightarrow (vi)$: is a restatement of 
    definition~(\ref{Adj:def:adjunction}).
    \newline\noindent
    $(vi)\Rightarrow(vii)$: Suppose $(F,G)$ has a unit $\eta$. Using
    proposition~(\ref{Adj:prop:unit:has:related:counit}), $\eta$ has a unique
    related counit. In particular, $(F,G)$ has a counit.
    \newline\noindent
    $(vii)\Rightarrow(vi)$: Suppose $(F,G)$ has a counit $\epsilon$. Using
    proposition~(\ref{Adj:prop:counit:has:related:unit}), $\epsilon$ has a unique
    related unit. In particular, $(F,G)$ has a unit.
\end{proof}

The following propostion provides an easy criterium to determine whether
a pair of functors $(F,G)$ is an adjunction between two categories \Cat\ and
\Dat. This criterium only demands that two natural transformations
with the appropriate properties be defined. It does not require proving
any universal property.
\begin{prop}\label{Adj:prop:adjunction:criteria}
    Let $F:\Cat\to\Dat$ and $G:\Dat\to\Cat$ be functors where \Cat\ and \Dat\ 
    are categories. Let $\eta:I_{\cal C}\Rightarrow G\circ F$ and 
    $\epsilon:F\circ G\Rightarrow I_{\cal D}$ be natural transformations
    which satisfy both of the following equations:
        \begin{eqnarray*}
            (1)&\ &G\epsilon\circ\eta G=\iota_{G}\\
            (2)&\ &\epsilon F \circ F\eta=\iota_{F}
        \end{eqnarray*}
    Then $(F,G)$ is an adjunction between the categories \Cat\ and \Dat, 
    i.e. $F\dashv G$.
\end{prop}
\begin{proof}
    Suppose such natural transformations $\eta$ and $\epsilon$ do exist.
    Using proposition~(\ref{Adj:prop:related:just:natural}) it follows
    that $\eta$ and $\epsilon$ are related unit and counit of $(F,G)$
    respectively. In particular, $(F,G)$ has a unit and it is therefore
    an adjunction between \Cat\ and \Dat.
\end{proof}



