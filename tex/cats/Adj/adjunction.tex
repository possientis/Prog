\begin{defin}\label{Adj:def:adjunction}
    We call {\em adjunction} any tuple $(\Cat,\Dat,F,G,\eta)$ where:
        \begin{eqnarray*}
            (1)& &\mbox{\Cat, \Dat\ are categories}\\
            (2)& &\mbox{$F:\Cat\to\Dat$ is a functor}\\
            (3)& &\mbox{$G:\Dat\to\Cat$ is a functor}\\
            (4)& &\mbox{$\eta:I_{\cal C}\Rightarrow G\circ F$ is a natural
                transformation}
        \end{eqnarray*}
    such that for all objects $c\in\Cat$, $d\in\Dat$ and arrow $f:c\to G(d)$, 
    there exists a unique arrow $g:F(c) \to d$ satisfying the equality 
    $f = G(g) \circ \eta_{c}$.
\end{defin}

\noindent
{\bf Remark}: The equality $f = G(g) \circ \eta_{c}$ can be visualized as
the diagram:
    \[
        \begin{tikzcd}
            c \arrow[rd,swap, "f"]\arrow[r,"\eta_{c}"] 
            &G(\,F(c)\,)\arrow[d,"G(\,g\,)"]\\
            & G(\,d\,)
        \end{tikzcd}
    \]

\begin{defin}\label{Adj:def:unit}
    If $(\Cat,\Dat,F,G,\eta)$ is an adjunction, $\eta$ is called the {\em unit} of     
    the adjunction, while $F$ is called the {\em left-adjoint} and $G$
    is called the {\em right-adjoint}.
\end{defin}

%    We call {\em adjunction} an ordered pair $(F,G)$ where $F$ is a functor
%    $F:\Cat\to\Dat$ and $G$ is a functor $G:\Dat\to\Cat$ while \Cat\  and 
%    \Dat\ are two locally-small categories for which there exists a natural 
%    isomorphism: 
%        \[
%            \alpha\ :\ 
%            \Dat\circ(F\times I_{\cal D})
%            \ \Rightarrow\ 
%            \Cat\circ(I_{{\cal C}^{op}}\times G)
%        \]
%    in the functor category $[\,\Cop\times\Dat\, ,\,\Set\,]$,
%    where $F$ also denotes $F:\Cop\rightarrow\Dop$.





