\begin{defin}\label{Adj:def:related}
    Let $F:\Cat\to\Dat$ and $G:\Dat\to\Cat$ be functors where \Cat\ and \Dat\ 
    are categories. Let $\eta:I_{\cal C}\Rightarrow G\circ F$ and 
    $\epsilon:F\circ G\Rightarrow I_{\cal D}$ be a unit and counit of $(F,G)$ 
    respectively. We say that $\eta$ and $\epsilon$ are {\em related} \ifand\ 
    one of these holds:
        \begin{eqnarray*}
            (1)&\ &G\epsilon\circ\eta G=\iota_{G}\\
            (2)&\ &\epsilon F \circ F\eta=\iota_{F}
        \end{eqnarray*}
\end{defin}
{\bf Remark}: recall that $\iota_{F}$ and $\iota_{G}$ are identity natural
transformations on $F$ and $G$ respectively, as per 
definition~(\ref{Nat:def:identity}). Hence he have $\iota_{F}:F\Rightarrow F$
as well as $\iota_{G}:G\Rightarrow G$. Furthermore, since $\eta:I_{\cal C}
\Rightarrow G\circ F$ and $G:\Dat\to\Cat$, $\eta G$ is the natural transformation 
$\eta G:I_{\cal C}\circ G\Rightarrow G\circ F\circ G$ as per 
definition~(\ref{Nat:def:rightmul}), which is the same as $\eta G:G\Rightarrow
G\circ F\circ G$. Since $F:\Cat\to\Dat$, we also obtain $F\eta:F\Rightarrow 
F\circ G\circ F$ from definition~(\ref{Nat:def:leftmul}). Likewise, since 
$\epsilon:F\circ G\Rightarrow I_{\cal D}$, we have $\epsilon F:F\circ G\circ F
\Rightarrow F$ from definition~(\ref{Nat:def:rightmul}) and $G\epsilon:G\circ 
F\circ G\Rightarrow G$ from definition~(\ref{Nat:def:leftmul}). Hence we 
see that $\epsilon F\circ F\eta:F\Rightarrow F$ is a well-defined natural
transformation, as per definition~(\ref{Nat:def:composition}) and likewise
$G\epsilon\circ\eta G:G\Rightarrow G$ is well-defined. So both equations~$(1)$
and~$(2)$ of definition~(\ref{Adj:def:related}) make perfect sense.

\begin{prop}\label{Adj:prop:related:both}
    Let $F:\Cat\to\Dat$ and $G:\Dat\to\Cat$ be functors where \Cat\ and \Dat\ 
    are categories. Let $\eta$ and $\epsilon$ be a unit and counit of $(F,G)$ 
    respectively. Then if $\eta$ and $\epsilon$ are related, both equations~$(1)$
    and $(2)$ of definition~(\ref{Adj:def:related}) hold.
\end{prop}
\begin{proof}
    We assume that $\eta$ and $\epsilon$ are related unit and counit of $(F,G)$.
    Then one of equation $(1)$ and $(2)$ holds, and we need to show that both
    equations $(1)$ and $(2)$ hold. It is therefore sufficient to prove that
    given a unit $\eta$ and a counit $\epsilon$, we have $(1)\Rightarrow(2)$
    and $(2)\Rightarrow(1)$. First we show that $(1)\Rightarrow(2)$. So we
    assume that $(1)$ holds, and we need to show $(2)$, which is an equality
    between two natural transformations. Using proposition~(\ref{Nat:prop:equal}),
    if sufficient to show that for all $c\in\Cat$:
        \[
            (\epsilon F\circ F\eta)_{c} = (\iota_{F})_{c}
        \]
    However, from definition~(\ref{Nat:def:identity}) we have $(\iota_{F})_{c}=
    \id(\,F(c)\,)$ and furthermore:
        \begin{eqnarray*}(\epsilon F\circ F\eta)_{c}
            &=&(\epsilon F)_{c}\circ(F\eta)_{c}
            \ \leftarrow\ \mbox{def.~(\ref{Nat:def:composition})}\\
            \mbox{def.~(\ref{Nat:def:rightmul})}\ \to\ 
            &=&\epsilon_{F(c)}\circ(F\eta)_{c}\\
            \mbox{def.~(\ref{Nat:def:leftmul})}\ \to\ 
            &=&\epsilon_{F(c)}\circ F(\eta_{c})\\
        \end{eqnarray*}
    So we need to show that $\epsilon_{F(c)}\circ F(\eta_{c})=\id(\,F(c)\,)$. 
    Note that assumption $(1)$ TODO
\end{proof}

