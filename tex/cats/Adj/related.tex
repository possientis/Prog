\begin{defin}\label{Adj:def:related}
    Let $F:\Cat\to\Dat$ and $G:\Dat\to\Cat$ be functors where \Cat\ and \Dat\ 
    are categories. Let $\eta:I_{\cal C}\Rightarrow G\circ F$ and 
    $\epsilon:F\circ G\Rightarrow I_{\cal D}$ be a unit and counit of $(F,G)$ 
    respectively. We say that $\eta$ and $\epsilon$ are {\em related} \ifand\ 
    one of these holds:
        \begin{eqnarray*}
            (1)&\ &G\epsilon\circ\eta G=\iota_{G}\\
            (2)&\ &\epsilon F \circ F\eta=\iota_{F}
        \end{eqnarray*}
\end{defin}
{\bf Remark}: recall that $\iota_{F}$ and $\iota_{G}$ are identity natural
transformations on $F$ and $G$ respectively, as per 
definition~(\ref{Nat:def:identity}). Hence he have $\iota_{F}:F\Rightarrow F$
as well as $\iota_{G}:G\Rightarrow G$. Furthermore, since $\eta:I_{\cal C}
\Rightarrow G\circ F$ and $G:\Dat\to\Cat$, $\eta G$ is the natural transformation 
$\eta G:I_{\cal C}\circ G\Rightarrow G\circ F\circ G$ as per 
definition~(\ref{Nat:def:rightmul}), which is the same as $\eta G:G\Rightarrow
G\circ F\circ G$. Since $F:\Cat\to\Dat$, we also obtain $F\eta:F\Rightarrow 
F\circ G\circ F$ from definition~(\ref{Nat:def:leftmul}). Likewise, since 
$\epsilon:F\circ G\Rightarrow I_{\cal D}$, we have $\epsilon F:F\circ G\circ F
\Rightarrow F$ from definition~(\ref{Nat:def:rightmul}) and $G\epsilon:G\circ 
F\circ G\Rightarrow G$ from definition~(\ref{Nat:def:leftmul}). Hence we 
see that $\epsilon F\circ F\eta:F\Rightarrow F$ is a well-defined natural
transformation, as per definition~(\ref{Nat:def:composition}) and likewise
$G\epsilon\circ\eta G:G\Rightarrow G$ is well-defined. So both equations~$(1)$
and~$(2)$ of definition~(\ref{Adj:def:related}) make perfect sense.


