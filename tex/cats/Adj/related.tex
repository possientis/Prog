\begin{defin}\label{Adj:def:related}
    Let $F:\Cat\to\Dat$ and $G:\Dat\to\Cat$ be functors where \Cat\ and \Dat\ 
    are categories. Let $\eta:I_{\cal C}\Rightarrow G\circ F$ and 
    $\epsilon:F\circ G\Rightarrow I_{\cal D}$ be a unit and counit of $(F,G)$ 
    respectively. We say that $\eta$ and $\epsilon$ are {\em related} \ifand\ 
    one of these holds:
        \begin{eqnarray*}
            (1)&\ &G\epsilon\circ\eta G=\iota_{G}\\
            (2)&\ &\epsilon F \circ F\eta=\iota_{F}
        \end{eqnarray*}
\end{defin}
{\bf Remark}: recall that $\iota_{F}$ and $\iota_{G}$ are identity natural
transformations on $F$ and $G$ respectively, as per 
definition~(\ref{Nat:def:identity}). Hence he have $\iota_{F}:F\Rightarrow F$
as well as $\iota_{G}:G\Rightarrow G$. Furthermore, since $\eta:I_{\cal C}
\Rightarrow G\circ F$ and $G:\Dat\to\Cat$, $\eta G$ is the natural transformation 
$\eta G:I_{\cal C}\circ G\Rightarrow G\circ F\circ G$ as per 
definition~(\ref{Nat:def:rightmul}), which is the same as $\eta G:G\Rightarrow
G\circ F\circ G$. Since $F:\Cat\to\Dat$, we also obtain $F\eta:F\Rightarrow 
F\circ G\circ F$ from definition~(\ref{Nat:def:leftmul}). Likewise, since 
$\epsilon:F\circ G\Rightarrow I_{\cal D}$, we have $\epsilon F:F\circ G\circ F
\Rightarrow F$ from definition~(\ref{Nat:def:rightmul}) and $G\epsilon:G\circ 
F\circ G\Rightarrow G$ from definition~(\ref{Nat:def:leftmul}). Hence we 
see that $\epsilon F\circ F\eta:F\Rightarrow F$ is a well-defined natural
transformation, as per definition~(\ref{Nat:def:composition}) and likewise
$G\epsilon\circ\eta G:G\Rightarrow G$ is well-defined. So both equations~$(1)$
and~$(2)$ of definition~(\ref{Adj:def:related}) make perfect sense.

In order for a unit $\eta$ and counit $\epsilon$ to be related, according 
to definition~(\ref{Adj:def:related}) only one of equations $(1)$ and $(2)$
needs to be satisfied. However, as the following proposition shows, each of
these equalities imply the other, so that related units and counits will 
always in fact satisfy both equations.

\begin{prop}\label{Adj:prop:related:both}
    Let $F:\Cat\to\Dat$ and $G:\Dat\to\Cat$ be functors where \Cat\ and \Dat\ 
    are categories. Let $\eta$ and $\epsilon$ be a unit and counit of $(F,G)$ 
    respectively. Then if $\eta$ and $\epsilon$ are related, both equations~$(1)$
    and $(2)$ of definition~(\ref{Adj:def:related}) hold.
\end{prop}
\begin{proof}
    We assume that $\eta$ and $\epsilon$ are related unit and counit of $(F,G)$.
    Then one of equation $(1)$ and $(2)$ holds, and we need to show that both
    equations $(1)$ and $(2)$ hold. It is therefore sufficient to prove that
    given a unit $\eta$ and a counit $\epsilon$, we have $(1)\Rightarrow(2)$
    and $(2)\Rightarrow(1)$. First we show that $(1)\Rightarrow(2)$. So we
    assume that $(1)$ holds, and we need to show $(2)$, which is an equality
    between two natural transformations. Using proposition~(\ref{Nat:prop:equal}),
    if sufficient to show that for all $c\in\Cat$:
        \[
            (\epsilon F\circ F\eta)_{c} = (\iota_{F})_{c}
        \]
    However, from definition~(\ref{Nat:def:identity}) we have $(\iota_{F})_{c}=
    \id(\,F(c)\,)$ and furthermore:
        \begin{eqnarray*}(\epsilon F\circ F\eta)_{c}
            &=&(\epsilon F)_{c}\circ(F\eta)_{c}
            \ \leftarrow\ \mbox{def.~(\ref{Nat:def:composition})}\\
            \mbox{def.~(\ref{Nat:def:rightmul})}\ \to\ 
            &=&\epsilon_{F(c)}\circ(F\eta)_{c}\\
            \mbox{def.~(\ref{Nat:def:leftmul})}\ \to\ 
            &=&\epsilon_{F(c)}\circ F(\eta_{c})\\
        \end{eqnarray*}
    So we need to show that $\epsilon_{F(c)}\circ F(\eta_{c})=\id(\,F(c)\,)$
    for all $c\in\Cat$. Note that a similar derivation using assumption $(1)$ 
    allows us to obtain for all $d\in\Dat$\,:
        \begin{equation}\label{Adj:eqn:related:both:1}
            G(\epsilon_{d})\circ\eta_{G(d)}=\id(\,G(d)\,)
        \end{equation}
    Now given $c\in\Cat$, let us pick $d=F(c)$. Then both $\epsilon_{F(c)}\circ 
    F(\eta_{c})$ and $\id(\,F(c)\,)$ are arrows from $F(c)$ to $d$ in \Dat.
    In order to show that these two arrows coincide, having assumed $\eta$ is
    a unit of $(F,G)$, from proposition~(\ref{Adj:prop:unit:proving:equality})
    it is sufficient to prove that $G(\,\epsilon_{F(c)}\circ F(\eta_{c})\,)
    \circ\eta_{c}=G(\,\id(\,F(c)\,)\,)\circ\eta_{c}$ which goes as follows:
        \begin{eqnarray*}G(\,\epsilon_{F(c)}\circ F(\eta_{c})\,)\circ\eta_{c}
            &=&G(\,\epsilon_{d}\circ F(\eta_{c})\,)\circ\eta_{c}\\
            \mbox{$G$ functor}\ \to\ 
            &=&G(\epsilon_{d})\circ (G \circ F)(\eta_{c})\circ\eta_{c}\\
            \mbox{prop~(\ref{Adj:prop:unit:natural}), $\eta_{c}:c\to G(d)$}\ \to\ 
            &=&G(\epsilon_{d})\circ \eta_{G(d)}\circ\eta_{c}\\
            \mbox{eqn.~(\ref{Adj:eqn:related:both:1})}\ \to\ 
            &=&\id(\,G(d)\,)\circ\eta_{c}\\
            \mbox{$G$ functor}\ \to\ 
            &=&G(\,\id(d)\,)\circ\eta_{c}\\
            &=&G(\,\id(\,F(c)\,)\,)\circ\eta_{c}
        \end{eqnarray*}
    So we now show that $(2)\Rightarrow(1)$. In this case for all $c\in\Cat$
    we have:
        \begin{equation}\label{Adj:eqn:related:both:2}
            \epsilon_{F(c)}\circ F(\eta_{c})=\id(\,F(c)\,)
        \end{equation}
    and we need to prove that equation~(\ref{Adj:eqn:related:both:1}) holds
    for all $d\in\Dat$. However, given $d\in\Dat$ and setting $c=G(d)$, both
    arrows of equation~(\ref{Adj:eqn:related:both:1}) are arrows from $c$ to
    $G(d)$ in \Cat. In order to show that these two arrows coincide, having
    assumed $\epsilon$ is a counit of $(F,G)$, from
    proposition~(\ref{Adj:prop:counit:proving:equality}) it is sufficient to
    prove the equality $\epsilon_{d}\circ F(\,G(\epsilon_{d})\circ\eta_{G(d)}\,)=
    \epsilon_{d}\circ F(\,\id(\,G(d)\,)\,)$ which goes as follows:
        \begin{eqnarray*}\epsilon_{d}\circ F(\,G(\epsilon_{d})\circ\eta_{G(d)}\,)
            &=&\epsilon_{d}\circ F(\,G(\epsilon_{d})\circ\eta_{c}\,)\\
            \mbox{$F$ functor}\ \to\ 
            &=&\epsilon_{d}\circ (F\circ G)(\epsilon_{d})\circ F(\eta_{c})\\
            \mbox{prop~(\ref{Adj:prop:counit:natural}), $\epsilon_{d}:F(c)\to d$}
            \ \to\ 
            &=&\epsilon_{d}\circ\epsilon_{F(c)}\circ F(\eta_{c})\\
            \mbox{eqn.~(\ref{Adj:eqn:related:both:2})}\ \to\ 
            &=&\epsilon_{d}\circ\id(\,F(c)\,)\\
            \mbox{$F$ functor}\ \to\ 
            &=&\epsilon_{d}\circ F(\,\id(c)\,)\\
            &=&\epsilon_{d}\circ F(\,\id(\,G(d)\,)\,)\\
        \end{eqnarray*}
\end{proof}

\begin{prop}\label{Adj:prop:unit:has:related:counit}
    Let $F:\Cat\to\Dat$ and $G:\Dat\to\Cat$ be functors where \Cat\ and \Dat\ 
    are categories. Then every unit of $(F,G)$ has a related counit.
\end{prop}
\begin{proof}
    Let $\eta:I_{\cal C}\Rightarrow G\circ F$ be a unit of $(F,G)$. We need to 
    show the existence of a counit $\epsilon:F\circ G\Rightarrow I_{\cal D}$ 
    which is related to $\eta$ as per definition~(\ref{Adj:def:related}).
    First we shall define a natural transformation $\epsilon:F\circ G\Rightarrow
    I_{\cal D}$, then prove that it is a counit and finally that it is related
    to $\eta$. So let $d\in\Dat$. We need to define an arrow $\epsilon_{d}:
    (F\circ G)(d)\to d$ in the category \Dat. Define $c=G(d)$. Then we need
    to define an arrow $\epsilon_{d}:F(c)\to d$. However, $\eta$ is a unit
    of $(F,G)$ and $\id(c)$ is an arrow $\id(c):c\to G(d)$. Applying
    definition~(\ref{Adj:def:unit}) we can define $\epsilon_{d}$ to be
    the unique arrow $\epsilon_{d}:F(c)\to d$ such that $\id(c)=G(\epsilon_{d})
    \circ\eta_{c}$. Rewriting $c=G(d)$ we obtain:
        \begin{equation}\label{Adj:eqn:unit:has:related:counit:1}
            G(\epsilon_{d})\circ\eta_{G(d)}=\id(\,G(d)\,)
        \end{equation}
    Having defined $\epsilon_{d}:(F\circ G)(d)\to d$ for all $d\in\Dat$ we have 
    a map $\epsilon:\ob\ \Dat\to\arr\ \Dat$ which satisfies~$(1)$ of 
    definition~(\ref{Nat:def:natural}) in relation to the functors $F\circ G$
    and $I_{\cal D}$. In order to show that $\epsilon$ is a natural 
    transformation, it remains to prove property~$(2)$, namely that the 
    naturality square commutes. So let $a,b\in\Dat$ and $g:a\to b$:
    \[
        \begin{tikzcd}
            F(G(a))\arrow[r, "\epsilon_{a}"]\arrow[d, swap,"(F\circ G)(g)\ "]
            &a\arrow[d, "g"]
            \\
            F(G(b))\arrow[r, swap, "\epsilon_{b}"]
            &b
        \end{tikzcd}
    \]
    We need to show that $g\circ\epsilon_{a}=\epsilon_{b}\circ(F\circ G)(g)$.
    However, defining $c=G(a)$ and $d=b$, both arrows are from $F(c)$ to $d$.
    Having assumed that $\eta$ is a unit of $(F,G)$, in order to show that
    these arrows are equal, 
    from proposition~(\ref{Adj:prop:unit:proving:equality}) it is sufficient
    to prove that $G(\,g\circ\epsilon_{a}\,)\circ\eta_{c}=G(\,\epsilon_{b}
    \circ(F\circ G)(g)\,)\circ\eta_{c}$. Both of these arrows are equal to
    $G(g)$ as can be seen from:
        \begin{eqnarray*}G(\,g\circ\epsilon_{a}\,)\circ\eta_{c}
            &=&G(g)\circ G(\epsilon_{a})\circ\eta_{c}
            \ \leftarrow\ \mbox{$G$ functor}\\
            &=&G(g)\circ G(\epsilon_{a})\circ\eta_{G(a)}\\
            \mbox{eqn.~\ref{Adj:eqn:unit:has:related:counit:1}}\ \to\ 
            &=&G(g)\circ\id(\,G(a)\,)\\
            &=&G(g)
        \end{eqnarray*}
    and:
        \begin{eqnarray*}G(\,\epsilon_{b}\circ(F\circ G)(g)\,)\circ\eta_{c}
            &=&G(\epsilon_{b})\circ(G\circ F\circ G)(g)\circ\eta_{c}
            \ \leftarrow\ \mbox{$G$ functor}\\
            &=&G(\epsilon_{b})\circ(G\circ F)(G(g))\circ\eta_{G(a)}\\
            \mbox{prop.~(\ref{Adj:prop:unit:natural}), $G(g):G(a)\to G(b)$}\ \to\ 
            &=&G(\epsilon_{b})\circ\eta_{G(b)}\circ G(g)\\
            \mbox{eqn.~\ref{Adj:eqn:unit:has:related:counit:1}}\ \to\ 
            &=&\id(\,G(b)\,)\circ G(g)\\
            &=&G(g)
        \end{eqnarray*}
    So we have now proved that $\epsilon$ is a natural transformation
    $\epsilon:F\circ G\Rightarrow I_{\cal D}$, and we need to show that
    it is a counit of $(F,G)$ as per definition~(\ref{Adj:def:counit}).
    So let $c\in\Cat$, $d\in\Dat$ and $g:F(c)\to d$. We need to show
    the existence of a unique $f:c\to G(d)$ such that $g=\epsilon_{d}\circ F(f)$.
    We first show the existence: let $f=G(g)\circ\eta_{c}$. Since $\eta$
    is a natural transformation $\eta:I_{\cal C}\Rightarrow G\circ F$, in 
    particular we have $\eta_{c}:c\to (G\circ F)(c)$. Since $G$ is a functor
    and $g:F(c)\to d$, we have $G(g):(G\circ F)(c)\to G(d)$. It follows that
    $f$ is a well-defined arrow $f:c\to G(d)$. We need to show that
    $g=\epsilon_{d}\circ F(f)$. Having assumed that $\eta$ is a unit
    of $(F,G)$, using proposition~(\ref{Adj:prop:unit:proving:equality}) it
    is sufficient to prove that $G(g)\circ\eta_{c}=G(\,\epsilon_{d}\circ F(f)\,)
    \circ\eta_{c}$, which is $f=G(\,\epsilon_{d}\circ F(f)\,)\circ\eta_{c}$.
    The proof goes as follows:
        \begin{eqnarray*}G(\,\epsilon_{d}\circ F(f)\,)\circ\eta_{c}
            &=&G(\epsilon_{d})\circ(G\circ F)(f)\circ\eta_{c}
            \ \leftarrow\ \mbox{$G$ functor}\\
            \mbox{prop.~(\ref{Adj:prop:unit:natural}), $f:c\to G(d)$}\ \to\ 
            &=&G(\epsilon_{d})\circ\eta_{G(d)}\circ f\\
            \mbox{eqn.~\ref{Adj:eqn:unit:has:related:counit:1}}\ \to\ 
            &=&\id(\,G(d)\,)\circ f\\
            &=&f
        \end{eqnarray*}
    We now prove the uniqueness. So we assume that $f_{1},f_{2}:c\to G(d)$ are
    such that $g=\epsilon_{d}\circ F(f_{1})=\epsilon_{d}\circ F(f_{2})$ and
    we need to show that $f_{1}=f_{2}$:
        \begin{eqnarray*}f_{1}
            &=&\id(\,G(d)\,)\circ f_{1}\\
            \mbox{eqn.~\ref{Adj:eqn:unit:has:related:counit:1}}\ \to\ 
            &=&G(\epsilon_{d})\circ\eta_{G(d)}\circ f_{1}\\
            \mbox{prop.~(\ref{Adj:prop:unit:natural}), $f_{1}:c\to G(d)$}\ \to\ 
            &=&G(\epsilon_{d})\circ (G\circ F)(f_{1})\circ\eta_{c}\\
            \mbox{$G$ functor}\ \to\ 
            &=&G(\,\epsilon_{d}\circ F(f_{1})\,)\circ\eta_{c}\\
            \mbox{assumption}\ \to\ 
            &=&G(\,\epsilon_{d}\circ F(f_{2})\,)\circ\eta_{c}\\
            \mbox{same derivation}\ \to\ 
            &=&f_{2}
        \end{eqnarray*}
    So we have now proved that $\epsilon$ is a counit of $(F,G)$ as per
    definition~(\ref{Adj:def:counit}) and it remains to show that it is
    related to $\eta$. From definition~(\ref{Adj:def:related}), 
    it is sufficient to show that $G\epsilon\circ\eta G=\iota_{G}$ or
    equivalently that $G(\epsilon_{d})\circ\eta_{G(d)}=\id(\,G(d)\,)$ for
    all $d\in\Dat$. This follows directly from 
    equation~(\ref{Adj:eqn:unit:has:related:counit:1}).
\end{proof}

\begin{prop}\label{Adj:prop:counit:has:related:unit}
    Let $F:\Cat\to\Dat$ and $G:\Dat\to\Cat$ be functors where \Cat\ and \Dat\ 
    are categories. Then every counit of $(F,G)$ has a related unit.
\end{prop}
\begin{proof}
    Let $\epsilon:F\circ G\Rightarrow I_{\cal D}$ be a counit of $(F,G)$. We need 
    to show the existence of a unit $\eta:I_{\cal C}\Rightarrow G\circ F$ 
    which is related to $\epsilon$ as per definition~(\ref{Adj:def:related}).
    First we shall define a natural transformation $\eta:I_{\cal C}\Rightarrow
    G\circ F$, then prove that it is a unit and finally that it is related
    to $\epsilon$. So let $c\in\Cat$. We need to define an arrow $\eta_{c}:
    c\to(G\circ F)(c)$ in the category \Cat. Define $d=F(c)$. Then we need
    to define an arrow $\eta_{c}:c\to G(d)$. However, $\epsilon$ is a counit
    of $(F,G)$ and $\id(d)$ is an arrow $\id(d):F(c)\to d$. Applying
    definition~(\ref{Adj:def:counit}) we can define $\eta_{c}$ to be
    the unique arrow $\eta_{c}:c\to G(d)$ such that $\id(d)=\epsilon_{d}
    \circ F(\eta_{c})$. Rewriting $d=F(c)$ we obtain:
        \begin{equation}\label{Adj:eqn:counit:has:related:unit:1}
            \epsilon_{F(c)}\circ F(\eta_{c})=\id(\,F(c)\,)
        \end{equation}
    Having defined $\eta_{c}:c\to(G\circ F)(c)$ for all $c\in\Cat$ we have 
    $\eta:\ob\ \Cat\to\arr\ \Cat$ which satisfies~$(1)$ of 
    definition~(\ref{Nat:def:natural}) in relation to the functors $I_{\cal C}$
    and $G\circ F$. In order to show that $\eta$ is a natural 
    transformation, it remains to prove property~$(2)$, namely that the 
    naturality square commutes. So let $a,b\in\Cat$ and $f:a\to b$:
    \[
        \begin{tikzcd}
            a\arrow[r, "\eta_{a}"]\arrow[d, swap,"f\ "]
            &G(F(a))\arrow[d, "(G\circ F)(f)"]
            \\
            b\arrow[r, swap, "\eta_{b}"]
            &G(F(b))
        \end{tikzcd}
    \]
    We need to show that $(G\circ F)(f)\circ\eta_{a}=\eta_{b}\circ f$.
    However, defining $c=a$ and $d=F(b)$, both arrows are from $c$ to $G(d)$.
    Having assumed that $\epsilon$ is a counit of $(F,G)$, in order to show that
    these arrows are equal, 
    from proposition~(\ref{Adj:prop:counit:proving:equality}) it is sufficient
    to prove that $\epsilon_{d}\circ F(\,(G\circ F)(f)\circ\eta_{a}\,) =
    \epsilon_{d}\circ F(\,\eta_{b}\circ f)$. Both of these arrows are equal to
    $F(f)$ as can be seen from:
        \begin{eqnarray*}\epsilon_{d}\circ F(\,(G\circ F)(f)\circ\eta_{a}\,)
            &=&\epsilon_{d}\circ (F\circ G\circ F)(f)\circ F(\eta_{a})
            \ \leftarrow\ \mbox{$F$ functor}\\
            &=&\epsilon_{F(b)}\circ (F\circ G)(F(f))\circ F(\eta_{a})\\
            \mbox{prop.~(\ref{Adj:prop:counit:natural}), $F(f):F(a)\to F(b)$}
            \ \to\ 
            &=&F(f)\circ\epsilon_{F(a)}\circ F(\eta_{a})\\
            \mbox{eqn.~\ref{Adj:eqn:counit:has:related:unit:1}}\ \to\ 
            &=&F(f)\circ\id(\,F(a)\,)\\
            &=&F(f)
        \end{eqnarray*}
    and:
        \begin{eqnarray*}\epsilon_{d}\circ F(\,\eta_{b}\circ f)
            &=&\epsilon_{d}\circ F(\eta_{b})\circ F(f)
            \ \leftarrow\ \mbox{$F$ functor}\\
            &=&\epsilon_{F(b)}\circ F(\eta_{b})\circ F(f)\\
            \mbox{eqn.~\ref{Adj:eqn:counit:has:related:unit:1}}\ \to\ 
            &=&\id(\,F(b)\,)\circ F(f)\\
            &=&F(f)
        \end{eqnarray*}
    TODO
    So we have now proved that $\epsilon$ is a natural transformation
    $\epsilon:F\circ G\Rightarrow I_{\cal D}$, and we need to show that
    it is a counit of $(F,G)$ as per definition~(\ref{Adj:def:counit}).
    So let $c\in\Cat$, $d\in\Dat$ and $g:F(c)\to d$. We need to show
    the existence of a unique $f:c\to G(d)$ such that $g=\epsilon_{d}\circ F(f)$.
    We first show the existence: let $f=G(g)\circ\eta_{c}$. Since $\eta$
    is a natural transformation $\eta:I_{\cal C}\Rightarrow G\circ F$, in 
    particular we have $\eta_{c}:c\to (G\circ F)(c)$. Since $G$ is a functor
    and $g:F(c)\to d$, we have $G(g):(G\circ F)(c)\to G(d)$. It follows that
    $f$ is a well-defined arrow $f:c\to G(d)$. We need to show that
    $g=\epsilon_{d}\circ F(f)$. Having assumed that $\eta$ is a unit
    of $(F,G)$, using proposition~(\ref{Adj:prop:unit:proving:equality}) it
    is sufficient to prove that $G(g)\circ\eta_{c}=G(\,\epsilon_{d}\circ F(f)\,)
    \circ\eta_{c}$, which is $f=G(\,\epsilon_{d}\circ F(f)\,)\circ\eta_{c}$.
    The proof goes as follows:
        \begin{eqnarray*}G(\,\epsilon_{d}\circ F(f)\,)\circ\eta_{c}
            &=&G(\epsilon_{d})\circ(G\circ F)(f)\circ\eta_{c}
            \ \leftarrow\ \mbox{$G$ functor}\\
            \mbox{prop.~(\ref{Adj:prop:unit:natural}), $f:c\to G(d)$}\ \to\ 
            &=&G(\epsilon_{d})\circ\eta_{G(d)}\circ f\\
            \mbox{eqn.~\ref{Adj:eqn:counit:has:related:unit:1}}\ \to\ 
            &=&\id(\,G(d)\,)\circ f\\
            &=&f
        \end{eqnarray*}
    We now prove the uniqueness. So we assume that $f_{1},f_{2}:c\to G(d)$ are
    such that $g=\epsilon_{d}\circ F(f_{1})=\epsilon_{d}\circ F(f_{2})$ and
    we need to show that $f_{1}=f_{2}$:
        \begin{eqnarray*}f_{1}
            &=&\id(\,G(d)\,)\circ f_{1}\\
            \mbox{eqn.~\ref{Adj:eqn:counit:has:related:unit:1}}\ \to\ 
            &=&G(\epsilon_{d})\circ\eta_{G(d)}\circ f_{1}\\
            \mbox{prop.~(\ref{Adj:prop:unit:natural}), $f_{1}:c\to G(d)$}\ \to\ 
            &=&G(\epsilon_{d})\circ (G\circ F)(f_{1})\circ\eta_{c}\\
            \mbox{$G$ functor}\ \to\ 
            &=&G(\,\epsilon_{d}\circ F(f_{1})\,)\circ\eta_{c}\\
            \mbox{assumption}\ \to\ 
            &=&G(\,\epsilon_{d}\circ F(f_{2})\,)\circ\eta_{c}\\
            \mbox{same derivation}\ \to\ 
            &=&f_{2}
        \end{eqnarray*}
    So we have now proved that $\epsilon$ is a counit of $(F,G)$ as per
    definition~(\ref{Adj:def:counit}) and it remains to show that it is
    related to $\eta$. From definition~(\ref{Adj:def:related}), 
    it is sufficient to show that $G\epsilon\circ\eta G=\iota_{G}$ or
    equivalently that $G(\epsilon_{d})\circ\eta_{G(d)}=\id(\,G(d)\,)$ for
    all $d\in\Dat$. This follows directly from 
    equation~(\ref{Adj:eqn:unit:has:related:counit:1}).
\end{proof}


