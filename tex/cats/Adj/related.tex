\begin{defin}\label{Adj:def:related}
    Let $F:\Cat\to\Dat$ and $G:\Dat\to\Cat$ be functors where \Cat\ and \Dat\ 
    are categories. Let $\eta:I_{\cal C}\Rightarrow G\circ F$ and 
    $\epsilon:F\circ G\Rightarrow I_{\cal D}$ be a unit and counit of $(F,G)$ 
    respectively. We say that $\eta$ and $\epsilon$ are {\em related} \ifand\ 
    one of these holds:
        \begin{eqnarray*}
            (1)&\ &G\epsilon\circ\eta G=\iota_{G}\\
            (2)&\ &\epsilon F \circ F\eta=\iota_{F}
        \end{eqnarray*}
\end{defin}
{\bf Remark}: recall that $\iota_{F}$ and $\iota_{G}$ are identity natural
transformations on $F$ and $G$ respectively, as per 
definition~(\ref{Nat:def:identity}). Hence he have $\iota_{F}:F\Rightarrow F$
as well as $\iota_{G}:G\Rightarrow G$. Furthermore, since $\eta:I_{\cal C}
\Rightarrow G\circ F$ and $G:\Dat\to\Cat$, $\eta G$ is the natural transformation 
$\eta G:I_{\cal C}\circ G\Rightarrow G\circ F\circ G$ as per 
definition~(\ref{Nat:def:rightmul}), which is the same as $\eta G:G\Rightarrow
G\circ F\circ G$. Since $F:\Cat\to\Dat$, we also obtain $F\eta:F\Rightarrow 
F\circ G\circ F$ from definition~(\ref{Nat:def:leftmul}). Likewise, since 
$\epsilon:F\circ G\Rightarrow I_{\cal D}$, we have $\epsilon F:F\circ G\circ F
\Rightarrow F$ from definition~(\ref{Nat:def:rightmul}) and $G\epsilon:G\circ 
F\circ G\Rightarrow G$ from definition~(\ref{Nat:def:leftmul}). Hence we 
see that $\epsilon F\circ F\eta:F\Rightarrow F$ is a well-defined natural
transformation, as per definition~(\ref{Nat:def:composition}) and likewise
$G\epsilon\circ\eta G:G\Rightarrow G$ is well-defined. So both equations~$(1)$
and~$(2)$ of definition~(\ref{Adj:def:related}) make perfect sense.

In order for a unit $\eta$ and counit $\epsilon$ to be related, according 
to definition~(\ref{Adj:def:related}) only one of equations $(1)$ and $(2)$
needs to be satisfied. However, as the following proposition shows, each of
these equalities imply the other, so that related units and counits will 
always in fact satisfy both equations.

\begin{prop}\label{Adj:prop:related:both}
    Let $F:\Cat\to\Dat$ and $G:\Dat\to\Cat$ be functors where \Cat\ and \Dat\ 
    are categories. Let $\eta$ and $\epsilon$ be a unit and counit of $(F,G)$ 
    respectively. Then if $\eta$ and $\epsilon$ are related, both equations~$(1)$
    and $(2)$ of definition~(\ref{Adj:def:related}) hold.
\end{prop}
\begin{proof}
    We assume that $\eta$ and $\epsilon$ are related unit and counit of $(F,G)$.
    Then one of equation $(1)$ and $(2)$ holds, and we need to show that both
    equations $(1)$ and $(2)$ hold. It is therefore sufficient to prove that
    given a unit $\eta$ and a counit $\epsilon$, we have $(1)\Rightarrow(2)$
    and $(2)\Rightarrow(1)$. First we show that $(1)\Rightarrow(2)$. So we
    assume that $(1)$ holds, and we need to show $(2)$, which is an equality
    between two natural transformations. Using proposition~(\ref{Nat:prop:equal}),
    if sufficient to show that for all $c\in\Cat$:
        \[
            (\epsilon F\circ F\eta)_{c} = (\iota_{F})_{c}
        \]
    However, from definition~(\ref{Nat:def:identity}) we have $(\iota_{F})_{c}=
    \id(\,F(c)\,)$ and furthermore:
        \begin{eqnarray*}(\epsilon F\circ F\eta)_{c}
            &=&(\epsilon F)_{c}\circ(F\eta)_{c}
            \ \leftarrow\ \mbox{def.~(\ref{Nat:def:composition})}\\
            \mbox{def.~(\ref{Nat:def:rightmul})}\ \to\ 
            &=&\epsilon_{F(c)}\circ(F\eta)_{c}\\
            \mbox{def.~(\ref{Nat:def:leftmul})}\ \to\ 
            &=&\epsilon_{F(c)}\circ F(\eta_{c})\\
        \end{eqnarray*}
    So we need to show that $\epsilon_{F(c)}\circ F(\eta_{c})=\id(\,F(c)\,)$
    for all $c\in\Cat$. Note that a similar derivation using assumption $(1)$ 
    allows us to obtain for all $d\in\Dat$\,:
        \begin{equation}\label{Adj:eqn:related:both:1}
            G(\epsilon_{d})\circ\eta_{G(d)}=\id(\,G(d)\,)
        \end{equation}
    Now given $c\in\Cat$, let us pick $d=F(c)$. Then both $\epsilon_{F(c)}\circ 
    F(\eta_{c})$ and $\id(\,F(c)\,)$ are arrows from $F(c)$ to $d$ in \Dat.
    In order to show that these two arrows coincide, having assumed $\eta$ is
    a unit of $(F,G)$, from proposition~(\ref{Adj:prop:unit:proving:equality})
    it is sufficient to prove that $G(\,\epsilon_{F(c)}\circ F(\eta_{c})\,)
    \circ\eta_{c}=G(\,\id(\,F(c)\,)\,)\circ\eta_{c}$ which goes as follows:
        \begin{eqnarray*}G(\,\epsilon_{F(c)}\circ F(\eta_{c})\,)\circ\eta_{c}
            &=&G(\,\epsilon_{d}\circ F(\eta_{c})\,)\circ\eta_{c}\\
            \mbox{$G$ functor}\ \to\ 
            &=&G(\epsilon_{d})\circ (G \circ F)(\eta_{c})\circ\eta_{c}\\
            \mbox{prop~(\ref{Adj:prop:unit:natural}), $\eta_{c}:c\to G(d)$}\ \to\ 
            &=&G(\epsilon_{d})\circ \eta_{G(d)}\circ\eta_{c}\\
            \mbox{eqn.~(\ref{Adj:eqn:related:both:1})}\ \to\ 
            &=&\id(\,G(d)\,)\circ\eta_{c}\\
            \mbox{$G$ functor}\ \to\ 
            &=&G(\,\id(d)\,)\circ\eta_{c}\\
            &=&G(\,\id(\,F(c)\,)\,)\circ\eta_{c}
        \end{eqnarray*}
    So we now show that $(2)\Rightarrow(1)$. In this case for all $c\in\Cat$
    we have:
        \begin{equation}\label{Adj:eqn:related:both:2}
            \epsilon_{F(c)}\circ F(\eta_{c})=\id(\,F(c)\,)
        \end{equation}
    and we need to prove that equation~(\ref{Adj:eqn:related:both:1}) holds
    for all $d\in\Dat$. However, given $d\in\Dat$ and setting $c=G(d)$, both
    arrows of equation~(\ref{Adj:eqn:related:both:1}) are arrows from $c$ to
    $G(d)$ in \Cat. In order to show that these two arrows coincide, having
    assumed $\epsilon$ is a counit of $(F,G)$, from
    proposition~(\ref{Adj:prop:counit:proving:equality}) it is sufficient to
    prove the equality $\epsilon_{d}\circ F(\,G(\epsilon_{d})\circ\eta_{G(d)}\,)=
    \epsilon_{d}\circ F(\,\id(\,G(d)\,)\,)$ which goes as follows:
        \begin{eqnarray*}\epsilon_{d}\circ F(\,G(\epsilon_{d})\circ\eta_{G(d)}\,)
            &=&\epsilon_{d}\circ F(\,G(\epsilon_{d})\circ\eta_{c}\,)\\
            \mbox{$F$ functor}\ \to\ 
            &=&\epsilon_{d}\circ (F\circ G)(\epsilon_{d})\circ F(\eta_{c})\\
            \mbox{prop~(\ref{Adj:prop:counit:natural}), $\epsilon_{d}:F(c)\to d$}
            \ \to\ 
            &=&\epsilon_{d}\circ\epsilon_{F(c)}\circ F(\eta_{c})\\
            \mbox{eqn.~(\ref{Adj:eqn:related:both:2})}\ \to\ 
            &=&\epsilon_{d}\circ\id(\,F(c)\,)\\
            \mbox{$F$ functor}\ \to\ 
            &=&\epsilon_{d}\circ F(\,\id(c)\,)\\
            &=&\epsilon_{d}\circ F(\,\id(\,G(d)\,)\,)\\
        \end{eqnarray*}
\end{proof}

\begin{prop}\label{Adj:prop:unit:has:related:counit}
    Let $F:\Cat\to\Dat$ and $G:\Dat\to\Cat$ be functors where \Cat\ and \Dat\ 
    are categories. Then every unit of $(F,G)$ has a related counit.
\end{prop}
\begin{proof}
    Let $\eta:I_{\cal C}\Rightarrow G\circ F$ be a unit of $(F,G)$. We need to 
    show the existence of a counit $\epsilon:F\circ G\Rightarrow I_{\cal D}$ 
    which is related to $\eta$ as per definition~(\ref{Adj:def:related}).
\end{proof}
