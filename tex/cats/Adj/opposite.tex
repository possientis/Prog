Let $F:\Cat\to\Dat$ and $G:\Dat\to\Cat$ be functors where \Cat\ and \Dat\ 
are categories. Suppose $\eta$ is a unit of $(F,G)$. Then in particular,
$\eta$ is a natural transformation $\eta:I_{\cal C}\Rightarrow G\circ F$.
Now if we are precise and wish to spell out the relevant typed functors,
$\eta$ is a natural transformation between $(\Cat,\Cat,I_{\cal C})$ and
$(\Cat,\Cat,G\circ F)$. However, using proposition~(\ref{Nat:prop:opposite}),
it follows that $\eta$ is also a natural transformation between the typed
functors $(\Cop,\Cop,G\circ F)$ and $(\Cop,\Cop,I_{\cal C})$. Furthermore,
from proposition~(\ref{Fun:prop:identity:opposite}) we have $I_{\cal C}=
I_{{\cal C}^{op}}$ so $\eta$ is in fact a natural transformation between
the typed functors $(\Cop,\Cop,G\circ F)$ and $(\Cop,\Cop,I_{{\cal C}^{op}})$.
In short, we have $\eta:G\circ F\Rightarrow I_{{\cal C}^{op}}$ where 
$G:\Dop\to\Cop$ and $F:\Cop\to\Dop$. This is exactly the signature we would 
expect from a counit of $(G,F)$ in relation to the categories \Dop\ and \Cop. 
As the following proposition shows, whenever $\eta$ is a unit of $(F,G)$ in
relation to categories \Cat\ and \Dat, it is indeed also a counit of $(G,F)$ in 
relation to the categories \Dop\ and \Cop.
    \[
        \begin{tikzcd}
              \Cat \arrow[r, "F", bend left  = 50, ""{name=U, below}]
            & \Dat \arrow[l, "G", bend left = 50, ""{name=D, above}]
            & \Dop \arrow[r, "G", bend left  = 50, ""{name=U, below}]
            & \Cop \arrow[l, "F", bend left  = 50, ""{name=U, below}]
        \end{tikzcd}
    \]
\
