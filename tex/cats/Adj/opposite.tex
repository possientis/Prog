Let $F:\Cat\to\Dat$ and $G:\Dat\to\Cat$ be functors where \Cat\ and \Dat\ 
are categories. Suppose $\eta$ is a unit of $(F,G)$. Then in particular,
$\eta$ is a natural transformation $\eta:I_{\cal C}\Rightarrow G\circ F$.
Now if we are precise and wish to spell out the relevant typed functors,
$\eta$ is a natural transformation between $(\Cat,\Cat,I_{\cal C})$ and
$(\Cat,\Cat,G\circ F)$. However, using proposition~(\ref{Nat:prop:opposite}),
it follows that $\eta$ is also a natural transformation between the typed
functors $(\Cop,\Cop,G\circ F)$ and $(\Cop,\Cop,I_{\cal C})$. Furthermore,
from proposition~(\ref{Fun:prop:identity:opposite}) we have $I_{\cal C}=
I_{{\cal C}^{op}}$ so $\eta$ is in fact a natural transformation between
the typed functors $(\Cop,\Cop,G\circ F)$ and $(\Cop,\Cop,I_{{\cal C}^{op}})$.
In short, we have $\eta:G\circ F\Rightarrow I_{{\cal C}^{op}}$ where 
$G:\Dop\to\Cop$ and $F:\Cop\to\Dop$, as per proposition~(\ref{Fun:prop:opposite}).
This is exactly the signature we would expect from a counit of $(G,F)$ in 
relation to the categories \Dop\ and \Cop. As the following proposition shows, 
whenever $\eta$ is a unit of $(F,G)$ in relation to categories \Cat\ and \Dat, 
it is indeed also a counit of $(G,F)$ in relation to the categories 
\Dop\ and \Cop.
    \[
        \begin{tikzcd}
              \Cat \arrow[r, "F", bend left  = 50, ""{name=U, below}]
            & \Dat \arrow[l, "G", bend left = 50, ""{name=D, above}]
            & \Dop \arrow[r, "G", bend left  = 50, ""{name=U, below}]
            & \Cop \arrow[l, "F", bend left  = 50, ""{name=U, below}]
        \end{tikzcd}
    \]
\begin{prop}\label{Adj:prop:opposite}
    Let $F:\Cat\to\Dat$ and $G:\Dat\to\Cat$ be functors where \Cat\ and \Dat\ 
    are categories. Then every unit of $(F,G)$ is
    a counit of $(G,F)$ w.r. to \Dop\ and \Cop.
\end{prop}
\begin{proof}
    Let $\eta$ be a unit of $(F,G)$, i.e a unit between the typed functors 
    $(\Cat,\Dat,F)$ and $(\Dat,\Cat,G)$ as per definition~(\ref{Adj:def:unit}).
    We need to show that $\eta$ is a counit of $(G,F)$ in relation to the 
    categories \Dop\ and \Cop, i.e. a counit between the typed functors
    $(\Dop,\Cop,G)$ and $(\Cop,\Dop,F)$ as per definition~(\ref{Adj:def:counit}).
    First we need to show that $\eta$ is a natural transformation
    $\eta:G\circ F\Rightarrow I_{{\cal C}^{op}}$, or more precisely a natural
    transformation between the typed functors $(\Cop,\Cop,G\circ F)$ and
    $(\Cop,\Cop,I_{{\cal C}^{op}})$. From proposition~(\ref{Nat:prop:opposite}),
    this amounts to showing that $\eta$ is a natural transformation between
    $(\Cat,\Cat,I_{{\cal C}^{op}})$ and $(\Cat,\Cat,G\circ F)$. Since 
    $I_{{\cal C}^{op}}=I_{\cal C}$ as per 
    proposition~(\ref{Fun:prop:identity:opposite}), we need to show that
    $\eta$ is natural transformation between the typed functors 
    $(\Cat,\Cat,I_{\cal C})$ and $(\Cat,\Cat,G\circ F)$ which is
    $\eta:I_{\cal C}\Rightarrow G\circ F$. This follows immediately from 
    definition~(\ref{Adj:def:unit}) and our assumption that $\eta$ is 
    a unit of $(F,G)$. So it remains to show that $\eta$ satisfies the 
    universal property of definition~(\ref{Adj:def:counit}), where
    $F$ and $G$ have been swapped and \Cat,\Dat\ replaced by \Dop,\Cop\
    respectively. So let $d\in\Dop$, $c\in\Cop$ and $f:G(d)\to c\ @\ \Cop$.
    We need to show the existence of a unique $g:d\to F(c)\ @\ \Dop$ such
    $f=\eta_{c}\circ G(g)\ @\ \Cop$. In other words, given $d\in\Dat$, $c\in\Cat$
    and $f:c\to G(d)$, we need to show the existence of a unique
    $g:F(c)\to d$ such that $f=G(g)\circ\eta_{c}$. Once again, this 
    follows from definition~(\ref{Adj:def:unit}) and our assumption that 
    $\eta$ is a unit of $(F,G)$. 
\end{proof}
