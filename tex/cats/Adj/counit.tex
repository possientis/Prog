\begin{defin}\label{Adj:def:counit}
    Let $F:\Cat\to\Dat$ and $G:\Dat\to\Cat$ be functors where \Cat\ and \Dat\ 
    are categories. We call {\em counit} of the typed functors $(\Cat,\Dat,F)$
    and $(\Dat,\Cat,G)$, a natural transformation 
    $\epsilon:F\circ G\Rightarrow I_{\cal D}$ such that for all 
    $c\in\Cat$, $d\in\Dat$ and $g:F(c)\to d$, there exists a
    unique $f:c \to G(d)$ such that $g = \epsilon_{d}\circ F(f)$.
\end{defin}

\noindent
{\bf Remark}: The equality $g = \epsilon_{d}\circ F(f)$ can be visualized with
the diagram:
    \[
        \begin{tikzcd}
            F(\,G(d)\,)\arrow[r,"\epsilon_{d}"]
            &d\\
            F(\,c\,)\arrow[ru,swap,"g"]\arrow[u,"F(\,f\,)"]
        \end{tikzcd}
    \]

\noindent
{\bf Remark}: Just like for {\em units}, being a {\em counit} is a statement
relating typed functors (not just functors), and the order between $F$ and 
$G$ matters.

\begin{notation}\label{Adj:notation:counit}
    Whenever the categories \Cat\ and \Dat\ are clearly undertood from the 
    context, we shall simply say that $\epsilon$ {\em is a counit of} the
    ordered pair $(F,G)$.
\end{notation}

\begin{prop}\label{Adj:prop:counit:proving:equality}
    Let $F:\Cat\to\Dat$ and $G:\Dat\to\Cat$ be functors where \Cat\ and \Dat\ 
    are categories, and let $\epsilon:F\circ G\Rightarrow I_{\cal D}$ be a counit 
    of $(F,G)$. Then, for all $c\in\Cat$, $d\in\Dat$ and $f_{1},f_{2}:c\to G(d)$,
    we have the implication:
        \[
            \epsilon_{d}\circ F(f_{1}) = \epsilon_{d}\circ F(f_{2})\ 
            \Rightarrow\ 
            f_{1}=f_{2}
        \]
\end{prop}
\begin{proof}
    This is an immediate consequence of the uniqueness property of
    definition~(\ref{Adj:def:counit}). Consider the arrow 
    $g=\epsilon_{d}\circ F(f_{1})$. Since $f_{1}:c\to G(d)$ and $F$ is a 
    functor, we have 
    $F(f_{1}):F(c)\to F(G(d))$ and since $\epsilon_{d}:F(G(d))\to d$, 
    $g$ is a well-defined arrow $g:F(c)\to d$ in the category \Dat. Having
    assumed that $\epsilon$ is a counit of $(F,G)$, from 
    definition~(\ref{Adj:def:counit}) there exists a unique arrow 
    $f:c\to G(d)$ in the category \Cat\ such that $g=\epsilon_{d}\circ F(f)$.
    However the equality $g=\epsilon_{d}\circ F(f)$ is satified by the 
    arrow $f=f_{1}$ since we defined $g$ as $g=\epsilon_{d}\circ F(f_{1})$.
    Now if we assume that $\epsilon_{d}\circ F(f_{1})=\epsilon_{d}\circ F(f_{2})$,
    then the equality $g=\epsilon_{d}\circ F(f)$ is also satisfied by $f_{2}$.
    By uniquenes, it follows that $f_{1}=f_{2}$.
\end{proof}

\begin{prop}\label{Adj:prop:counit:natural}
    Let $F:\Cat\to\Dat$ and $G:\Dat\to\Cat$ be functors where \Cat\ and \Dat\ 
    are categories, and let $\epsilon:F\circ G\Rightarrow I_{\cal D}$ be a counit 
    of $(F,G)$. Then we have:
        \[
            \epsilon_{b}\circ(F\circ G)(g)=g\circ\epsilon_{a}
        \]
    for all objects $a,b\in\Cat$ and arrow $g:a\to b$.
\end{prop}
\begin{proof}
    This is an immediate consequence of definition~(\ref{Nat:def:natural}) 
    and the fact that $\epsilon$ is a natural transformation 
    $\epsilon:F\circ G\Rightarrow I_{\cal D}$, so the following square commutes:
    \[
        \begin{tikzcd}
            F(G(a))\arrow[r, "\epsilon_{a}"]\arrow[d, swap,"(F\circ G)(g)\ "]
            &a\arrow[d, "g"]
            \\
            F(G(b))\arrow[r, swap, "\epsilon_{b}"]
            &b
        \end{tikzcd}
    \]
\end{proof}

Given functors $F:\Cat\to\Dat$ and $G:\Dat\to\Cat$ where \Cat, \Dat\ are 
categories, if $\epsilon:F\circ G\Rightarrow I_{\cal D}$ is a counit of $(F,G)$,
then a question arises as to whether this counit is unique. As the following
proposition shows, the answer is 'no' in general as every natural 
isomorphism $\alpha : G\Rightarrow G$ potentially gives rise to a 
new counit.

\begin{prop}\label{Adj:prop:counit:not:unique}
    Let $F:\Cat\to\Dat$ and $G:\Dat\to\Cat$ be functors where \Cat\ and \Dat\ 
    are categories, and let $\epsilon:F\circ G\Rightarrow I_{\cal D}$ be a counit
    of $(F,G)$. Then, a natural transformation $\epsilon':F\circ G\Rightarrow 
    I_{\cal D}$ is a counit of $(F,G)$ \ifand\ there exists a natural isomorphism 
    $\alpha:G\Rightarrow G$ such that:
        \[
            \epsilon'=\epsilon\circ(F\alpha)
        \]
\end{prop}

\noindent
{\bf Remark}: If $\alpha: G\Rightarrow G$ and $F:\Cat\to\Dat$, then $F\alpha$
is the natural transformation $F\alpha:F\circ G\Rightarrow F\circ G$ as per
definition~(\ref{Nat:def:leftmul}), and if $\epsilon:F\circ G\Rightarrow 
I_{\cal D}$ then the composition $\epsilon\circ(F\alpha)$ is a well-defined 
natural transformation from the functor $F\circ G$ to the functor $I_{\cal D}$ 
as per definition~(\ref{Nat:def:composition}).

\noindent
\begin{proof}
    First we show the {\em if} part. So we assume that $\epsilon'=\epsilon
    \circ(F\alpha)$ for some natural isomorphism $\alpha:G\Rightarrow G$. We 
    need to show that $\epsilon'$ is a counit of $(F,G)$. As already seen
    $\epsilon'$ is a well-defined natural transformation $\epsilon':F\circ 
    G\Rightarrow I_{\cal D}$. Hence we simply need to show that it satisfies 
    the universal property of definition~(\ref{Adj:def:counit}). So let 
    $c\in\Cat$, $d\in\Dat$ and $g:F(c)\to d$. We need to show the existence 
    of a unique $f:c\to G(d)$ such that $g = \epsilon_{d}'\circ F(f)$. However, 
    by assumption we have:
        \begin{eqnarray*}\epsilon'_{d}
            &=&(\epsilon\circ(F\alpha))_{d}\\
            \mbox{$(1)$ of def.~(\ref{Nat:def:composition})}\ \to\ 
            &=&\epsilon_{d}\circ(F\alpha)_{d}\\
            \mbox{$(1)$ of def.~(\ref{Nat:def:leftmul})}\ \to\ 
            &=&\epsilon_{d}\circ F(\,\alpha_{d}\,)\\
        \end{eqnarray*}
    So we need to prove the existence of a unique arrow $f:c\to G(d)$
    such that $g = \epsilon_{d}\circ F(\alpha_{d})\circ F(f)$, which is
     $g = \epsilon_{d}\circ F(\alpha_{d}\circ f)$ since $F$ is a functor:

    Existence: having assumed $\epsilon$ is a counit of $(F,G)$, since 
    $g:F(c)\to d$ using definition~(\ref{Adj:def:counit}) there exists
    a (unique) $h:c\to G(d)$ with $g = \epsilon_{d}\circ F(h)$. However, 
    by assumption $\alpha:G\Rightarrow G$ is a natural isomorphism. 
    From proposition~(\ref{Nat:prop:isomorphism:component}),
    every component of $\alpha$ is an isomorphism and in particular
    $\alpha_{d}:G(d)\to G(d)$ has an inverse $\alpha_{d}^{-1}:G(d)\to G(d)$.
    Defining $f=\alpha_{d}^{-1}\circ h$ we obtain an arrow $f:c\to G(d)$ such
    that $g=\epsilon_{d}\circ F(\alpha_{d}\circ f)$ as requested.

    Uniqueness: Suppose $f_{1},f_{2}:c\to G(d)$ are two arrows in \Cat\ 
    with the equality $\epsilon_{d}\circ F(\alpha_{d}\circ f_{1})=g=
    \epsilon_{d}\circ F(\alpha_{d}\circ f_{2})$. Having assumed 
    $\epsilon$ is a counit of $(F,G)$ using
    proposition~(\ref{Adj:prop:counit:proving:equality}) we obtain 
    $\alpha_{d}\circ f_{1}=\alpha_{d}\circ f_{2}$ and composing both sides
    to the left by $\alpha_{d}^{-1}$ we conclude that $f_{1}=f_{2}$
    as requested.

    We now prove the {\em only if} part. So we assume that 
    $\epsilon':F\circ G\Rightarrow I_{\cal D}$ is a counit of $(F,G)$ and we need
    to show the existence of a natural isomorphism $\alpha:G\Rightarrow G$
    such that $\epsilon'=\epsilon\circ(F\alpha)$. So let $d\in\Dat$. First we need
    to define $\alpha_{d}:G(d)\to G(d)$. Define $c=G(d)\in\Cat$. Then we 
    have $\epsilon'_{d}:F(c)\to d$. Since $\epsilon$ is a counit of $(F,G)$ 
    there exists a unique arrow $f:c\to G(d)$ such that $\epsilon'_{d}
    =\epsilon_{d}\circ F(f)$.
    Define $\alpha_{d}$ to be precisely this unique arrow $f$. Then we 
    have $\alpha_{d}:G(d)\to G(d)$ and:
        \begin{equation}\label{Adj:eqn:counit:not:unique}
            \epsilon'_{d}=\epsilon_{d}\circ F(\alpha_{d})
        \end{equation}
    Collecting all these arrows $\alpha_{d}:G(d)\to G(d)$ for $d\in\Dat$,
    we obtain a map $\alpha:\ob\ \Dat\to\arr\ \Cat$. Next we need to show 
    that $\alpha$ is a natural transformation, so it remains to show that
    condition~$(2)$ of definition~(\ref{Nat:def:natural}) is satisfied. So
    let $a,b\in\Dat$ and $g:a\to b$. We need to show the equality
    $G(g)\circ\alpha_{a}=\alpha_{b}\circ G(g)$:
    \[
        \begin{tikzcd}
            a\arrow[d,swap, "g"]
            &G(a)\arrow[r, "\alpha_{a}"]\arrow[d, swap,"G(g)"]
            &G(a)\arrow[d, "G(g)"]
            \\
            b
            &G(b)\arrow[r, swap, "\alpha_{b}"]
            &G(b)
        \end{tikzcd}
    \]
    Define $c=G(a)\in\Cat$ and $d=b\in\Dat$. Then both arrows 
    $G(g)\circ\alpha_{a}$ and $\alpha_{b}\circ G(g)$ have domain $c$ and
    codomain $G(d)$ in \Cat. Appplying 
    proposition~(\ref{Adj:prop:counit:proving:equality}) 
    since $\epsilon$ is a counit of $(F,G)$, in order to show that these are equal
    it is sufficient to prove:
        \[
            \epsilon_{d}\circ F(\,G(g)\circ\alpha_{a}\,)=
            \epsilon_{d}\circ F(\,\alpha_{b}\circ G(g)\,)
        \]
    The proof goes as follows:
        \begin{eqnarray*}\epsilon_{d}\circ F(\,G(g)\circ\alpha_{a}\,)
            &=&\epsilon_{b}\circ F(\,G(g)\circ\alpha_{a}\,)\\
            \mbox{$F$ functor}\ \to\ 
            &=&\epsilon_{b}\circ F(G(g)) \circ F(\alpha_{a})\\
            \mbox{prop.~(\ref{Adj:prop:counit:natural}), $\epsilon$ counit}\ \to\ 
            &=&g\circ\epsilon_{a}\circ F(\alpha_{a})\\
            \mbox{eqn.~(\ref{Adj:eqn:counit:not:unique})}\ \to\ 
            &=&g\circ\epsilon_{a}'\\
            \mbox{prop.~(\ref{Adj:prop:counit:natural}), $\epsilon$' counit}\ \to\ 
            &=&\epsilon_{b}'\circ F(G(g))\\
            \mbox{eqn.~(\ref{Adj:eqn:counit:not:unique})}\ \to\ 
            &=&\epsilon_{b}\circ F(\alpha_{b})\circ F(G(g))\\
            \mbox{$F$ functor}\ \to\ 
            &=&\epsilon_{b}\circ F(\,\alpha_{b}\circ G(g)\,)\\
            &=&\epsilon_{d}\circ F(\,\alpha_{b}\circ G(g)\,)\\
        \end{eqnarray*}
    So we have now proved that $\alpha$ is a natural transformation
    $\alpha:G\Rightarrow G$. However, we aim to show that it is actually
    a natural isomorphism. 
    Using proposition~(\ref{Nat:prop:isomorphism:component}), it is sufficient
    to prove that each component $\alpha_{d}:G(d)\to G(d)$ for $d\in\Dat$ is 
    an isomorphisn in \Cat. So let $d\in\Dat$. Define $c=G(d)\in\Cat$. Then we 
    have $\epsilon_{d}:F(c)\to d$. Having assumed that $\epsilon'$ is a counit 
    of $(F,G)$ there exists a unique arrow $f:c\to G(d)$ such that 
    $\epsilon_{d}=\epsilon'_{d}\circ F(f)$. Define $\beta_{d}$ to be precisely this 
    unique arrow $f$. Then we have $\beta_{d}:G(d)\to G(d)$ and:
        \begin{equation}\label{Adj:eqn:counit:not:unique:2}
            \epsilon_{d}=\epsilon_{d}'\circ F(\beta_{d})
        \end{equation}
    We claim that $\beta_{d}$ is an inverse of $\alpha_{d}$. So we need to prove 
    that $\beta_{d}\circ\alpha_{d}=\id(\,G(d)\,)$ and 
    $\alpha_{d}\circ\beta_{d}=\id(\,G(d)\,)$.
    In order to show that $\beta_{d}\circ\alpha_{d}=\id(\,G(d)\,)$, since both 
    arrows are from $c=G(d)$ to $G(d)$ and $\epsilon'$ is assumed to be a counit 
    of $(F,G)$, applying proposition~(\ref{Adj:prop:counit:proving:equality}) it 
    is sufficient to prove that:
        \[
            \epsilon_{d}'\circ F(\,\beta_{d}\circ\alpha_{d}\,)
            = 
            \epsilon_{d}'\circ F(\,\id(G(d))\,)
        \]
    The proof goes as follows:
        \begin{eqnarray*}\epsilon_{d}'\circ F(\,\beta_{d}\circ\alpha_{d}\,)
            &=&\epsilon_{d}'\circ F(\beta_{d})\circ F(\alpha_{d})
            \ \leftarrow\ \mbox{$F$ functor}\\
            \mbox{eqn.~(\ref{Adj:eqn:counit:not:unique:2})}\ \to\ 
            &=&\epsilon_{d}\circ F(\alpha_{d})\\
            \mbox{eqn.~(\ref{Adj:eqn:counit:not:unique})}\ \to\ 
            &=&\epsilon_{d}'\\
            \mbox{identity}\ \to\ 
            &=&\epsilon_{d}'\circ \id(\,F(G(d))\,)\\
            \mbox{$F$ functor}\ \to\ 
            &=&\epsilon_{d}'\circ F(\,\id(G(d))\,)\\
        \end{eqnarray*}
    In order to show that $\alpha_{d}\circ\beta_{d}=\id(\,G(d)\,)$, since both 
    arrows are from $c=G(d)$ to $G(d)$ and $\epsilon$ is a counit of 
    $(F,G)$, applying proposition~(\ref{Adj:prop:counit:proving:equality}) it is 
    sufficient to prove that:
        \[
            \epsilon_{d}\circ F(\,\alpha_{d}\circ\beta_{d}\,)
            = 
            \epsilon_{d}\circ F(\,\id(G(d))\,)
        \]
    The proof goes as follows:
        \begin{eqnarray*}\epsilon_{d}\circ F(\,\alpha_{d}\circ\beta_{d}\,)
            &=&\epsilon_{d}\circ F(\alpha_{d})\circ F(\beta_{d})
            \ \leftarrow\ \mbox{$F$ functor}\\
            \mbox{eqn.~(\ref{Adj:eqn:counit:not:unique})}\ \to\ 
            &=&\epsilon_{d}'\circ F(\beta_{d})\\
            \mbox{eqn.~(\ref{Adj:eqn:counit:not:unique:2})}\ \to\ 
            &=&\epsilon_{d}\\
            \mbox{identity}\ \to\ 
            &=&\epsilon_{d}\circ \id(\,F(G(d))\,)\\
            \mbox{$F$ functor}\ \to\ 
            &=&\epsilon_{d}\circ F(\,\id(G(d))\,)\\
        \end{eqnarray*}
    So we now know that $\alpha:G\Rightarrow G$ is a natural isomorphism and it 
    remains to show that $\epsilon'=\epsilon\circ (F\alpha)$. Using 
    proposition~(\ref{Nat:prop:equal}), this equality between natural
    transformations is obtained if we can show that all components are equal:
        \begin{eqnarray*}\epsilon'_{d}
            &=&\epsilon_{d}\circ F(\alpha_{d})
            \ \leftarrow\ \mbox{eqn.~(\ref{Adj:eqn:counit:not:unique})}\\
            \mbox{$(1)$ of def.~(\ref{Nat:def:leftmul})}\ \to\ 
            &=&\epsilon_{d}\circ (F\alpha)_{d}\\
            \mbox{$(1)$ of def.~(\ref{Nat:def:composition})}\ \to\ 
            &=&(\,\epsilon\circ F\alpha\,)_{d}\\
        \end{eqnarray*}
\end{proof}
