In this section, we consider the case when both categories \Cat\ and \Dat\ are
locally small, as per definition~(\ref{Cat:def:locally:small}). The difference
this makes is the availability of the associated hom-functors
$\Cat:\Cop\times\Cat\to\Set$ and $\Dat:\Dop\times\Dat\to\Set$ as per
definition~(\ref{Fun:def:hom:functor}). So suppose
we have two functors $F:\Cat\to\Dat$ and $G:\Dat\to\Cat$. We also have
identity functors $I_{\cal C}:\Cat\to\Cat$ and $I_{\cal D}:\Dat\to\Dat$
as per definition~(\ref{Fun:def:identity}). Using 
proposition~(\ref{Fun:prop:opposite}), we know that $F$ is also a functor 
$F:\Cop\to\Dop$, while from proposition~(\ref{Fun:prop:identity:opposite})
we see that $I_{\cal C}$ and $I_{{\cal C}^{op}}$ are the same functors.
Using definition~(\ref{Fun:def:canonical:product}) we can define the 
product functors $F\times I_{\cal D}:\Cop\times\Dat\to\Dop\times\Dat$.
and $I_{\cal C}\times G:\Cop\times\Dat\to\Cop\times\Cat$. Composing 
these functors with the hom-functors \Dat\ and \Cat\ respectively as per
definition~(\ref{Fun:def:composition}), we obtain
two functors $\Dat\circ(F\times I_{\cal D}):\Cop\times\Dat\to\Set$ and
$\Cat\circ(I_{\cal C}\times G):\Cop\times\Dat\to\Set$. These functors 
are very important in what follows, so we shall give then a name:

\begin{defin}\label{Adj:def:lhs:functor}
    Let $F:\Cat\to\Dat$ and $G:\Dat\to\Cat$ be functors where \Cat\ and \Dat\ 
    are locally small categories. We call {\em left-hand-side functor} associated
    with the pair $(F,G)$ the functor $\Dat\circ(F\times I_{\cal D}):\Cop\times
    \Dat\to\Set$.
\end{defin}

\begin{defin}\label{Adj:def:rh:functor}
    Let $F:\Cat\to\Dat$ and $G:\Dat\to\Cat$ be functors where \Cat\ and \Dat\ 
    are locally small categories. We call {\em right-hand-side functor} associated
    with the pair $(F,G)$ the functor $\Cat\circ(I_{\cal C}\times G):\Cop\times
    \Dat\to\Set$.
\end{defin}

