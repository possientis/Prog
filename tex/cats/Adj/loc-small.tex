In this section, we consider the case when both categories \Cat\ and \Dat\ are
locally small, as per definition~(\ref{Cat:def:locally:small}). The difference
this makes is the availability of the associated hom-functors
$\Cat:\Cop\times\Cat\to\Set$ and $\Dat:\Dop\times\Dat\to\Set$ as per
definition~(\ref{Fun:def:hom:functor}). So suppose
we have two functors $F:\Cat\to\Dat$ and $G:\Dat\to\Cat$. We also have
identity functors $I_{\cal C}:\Cat\to\Cat$ and $I_{\cal D}:\Dat\to\Dat$
as per definition~(\ref{Fun:def:identity}). Using 
proposition~(\ref{Fun:prop:opposite}), we know that $F$ is also a functor 
$F:\Cop\to\Dop$, while from proposition~(\ref{Fun:prop:identity:opposite})
we see that $I_{\cal C}$ and $I_{{\cal C}^{op}}$ are the same functors.
Using definition~(\ref{Fun:def:canonical:product}) we can define the 
product functors $F\times I_{\cal D}:\Cop\times\Dat\to\Dop\times\Dat$.
and $I_{\cal C}\times G:\Cop\times\Dat\to\Cop\times\Cat$. Composing 
these functors with the hom-functors \Dat\ and \Cat\ respectively as per
definition~(\ref{Fun:def:composition}), we obtain
two functors $\Dat\circ(F\times I_{\cal D}):\Cop\times\Dat\to\Set$ and
$\Cat\circ(I_{\cal C}\times G):\Cop\times\Dat\to\Set$. These functors 
are very important in what follows, so we shall give then a name:

\begin{defin}\label{Adj:def:lhs:functor}
    Let $F:\Cat\to\Dat$ and $G:\Dat\to\Cat$ be functors where \Cat\ and \Dat\ 
    are locally small categories. We call {\em left-hand-side functor} associated
    with the pair $(F,G)$ the functor $\Dat\circ(F\times I_{\cal D}):\Cop\times
    \Dat\to\Set$.
\end{defin}

\begin{defin}\label{Adj:def:rhs:functor}
    Let $F:\Cat\to\Dat$ and $G:\Dat\to\Cat$ be functors where \Cat\ and \Dat\ 
    are locally small categories. We call {\em right-hand-side functor} associated
    with the pair $(F,G)$ the functor $\Cat\circ(I_{\cal C}\times G):\Cop\times
    \Dat\to\Set$.
\end{defin}

\begin{prop}\label{Adj:prop:lhs:object}
    Let $F:\Cat\to\Dat$ and $G:\Dat\to\Cat$ be functors where \Cat\ and \Dat\ 
    are locally small categories and let $F^{*}=\Dat\circ(F\times I_{\cal D})$
    be the left-hand-side functor associated with $(F,G)$. Then for all $c\in\Cat$ 
    and $d\in\Dat$, we have:
        \[
            F^{*}(c,d) = \Dat(F(c),d)
        \]
\end{prop}
\begin{proof}
    \begin{eqnarray*}F^{*}(c,d)
        &=&\Dat\circ(F\times I_{\cal D})(c,d)\\
        \mbox{$(1)$ of def.~(\ref{Fun:def:composition})}\ \to\ 
        &=&\Dat(\,(F\times I_{\cal D})(c,d)\,)\\
        \mbox{$(1)$ of def.~(\ref{Fun:def:canonical:product})}\ \to\ 
        &=&\Dat(F(c),I_{\cal D}(d))\\
        \mbox{$(1)$ of def.~(\ref{Fun:def:identity})}\ \to\ 
        &=&\Dat(F(c),d)
    \end{eqnarray*}
\end{proof}

\noindent
{\bf Remark}: In light of proposition~(\ref{Adj:prop:lhs:object}), it is
common to refer to the left-hand-side functor casually as '$\Dat(F(c),d)$'
where '$c$' and '$d$' are dummy variables. We should not forget however
that a functor is more than a mere transformation on objects, and that 
the left-hand-side functor is really $\Dat\circ(F\times I_{\cal D})$.

\begin{prop}\label{Adj:prop:rhs:object}
    Let $F:\Cat\to\Dat$ and $G:\Dat\to\Cat$ be functors where \Cat\ and \Dat\ 
    are locally small categories and let $G^{*}=\Cat\circ(I_{\cal C}\times G)$
    be the right-hand-side functor associated with $(F,G)$. Then for all 
    $c\in\Cat$ and $d\in\Dat$, we have:
        \[
            G^{*}(c,d) = \Cat(c,G(d))
        \]
\end{prop}
\begin{proof}
    \begin{eqnarray*}G^{*}(c,d)
        &=&\Cat\circ(I_{\cal C}\times G)(c,d)\\
        \mbox{$(1)$ of def.~(\ref{Fun:def:composition})}\ \to\ 
        &=&\Cat(\,(I_{\cal C}\times G)(c,d)\,)\\
        \mbox{$(1)$ of def.~(\ref{Fun:def:canonical:product})}\ \to\ 
        &=&\Cat(I_{\cal C}(c),G(d))\\
        \mbox{$(1)$ of def.~(\ref{Fun:def:identity})}\ \to\ 
        &=&\Cat(c,G(d))
    \end{eqnarray*}
\end{proof}

\noindent
{\bf Remark}: In light of proposition~(\ref{Adj:prop:rhs:object}), it is
common to refer to the right-hand-side functor casually as '$\Cat(c,G(d))$'
where '$c$' and '$d$' are dummy variables. Once again, a functor is more than 
a mere transformation on objects, and we should remember that the right-hand-side 
functor is really $\Cat\circ(I_{\cal C}\times G)$.

The left-hand side functor $F^{*}=\Dat\circ(F\times I_{\cal D})$ is a functor
$F^{*}:\Cop\times\Dat\to\Set$. If $c,c'\in\Cat$ and $d,d'\in\Dat$, if 
$f:c'\to c\ @\ \Cat$ and $g:d\to d'\ @\ \Dat$ then we have $(f,g):(c,d)\to
(c',d')\ @\ \Cop\times\Dat$ and $F^{*}(f,g):F^{*}(c,d)\to F^{*}(c',d')\ @\ \Set$, 
which is $F^{*}(f,g):\Dat(F(c),d)\to\Dat(F(c'),d')$. So if 
$g':F(c)\to d\ @\ \Dat$ then $F^{*}(f,g)(g')$ is well-defined in
$\Dat(F(c'),d')$, i.e. $F^{*}(f,g)(g'):F(c')\to d'\ @\ \Dat$.
\begin{prop}\label{Adj:prop:lhs:arrow}
    Let $F:\Cat\to\Dat$ and $G:\Dat\to\Cat$ be functors where \Cat\ and \Dat\ 
    are locally small categories and let $F^{*}=\Dat\circ(F\times I_{\cal D})$
    be the left-hand-side functor associated with $(F,G)$. Then for all 
    $c,c'\in\Cat$ and $d,d'\in\Dat$, for all $f:c'\to c$, $g:d\to d'$
    and $g':F(c)\to d$, we have:
        \[
            F^{*}(f,g)(g') = g \circ g'\circ F(f)
        \]
\end{prop}
\noindent
{\bf Remark}: Note that the expression $g\circ g'\circ F(f)$ is a well-defined
arrow in \Dat\ from $F(c')$ to $d'$, since $F(f):F(c')\to F(c)$, $g':F(c)\to d$
and $g:d\to d'$.

\noindent
\begin{proof}
    \begin{eqnarray*}F^{*}(f,g)(g')
        &=&\Dat\circ(F\times I_{\cal D})(f,g)(g')\\
        \mbox{$(2)$ of def.~(\ref{Fun:def:composition})}\ \to\ 
        &=&\Dat(\,(F\times I_{\cal D})(f,g)\,)(g')\\
        \mbox{$(2)$ of def.~(\ref{Fun:def:canonical:product})}\ \to\ 
        &=&\Dat(F(f),I_{\cal D}(g))(g')\\
        \mbox{$(2)$ of def.~(\ref{Fun:def:identity})}\ \to\ 
        &=&\Dat(F(f),g)(g')\\
        \mbox{$(2)$ of def.~(\ref{Fun:def:hom:functor})}\ \to\ 
        &=&g\circ g'\circ F(f)
    \end{eqnarray*}
\end{proof}

The right-hand side functor $G^{*}=\Cat\circ(I_{\cal C}\times G)$ is a functor
$G^{*}:\Cop\times\Dat\to\Set$. If $c,c'\in\Cat$ and $d,d'\in\Dat$, if 
$f:c'\to c\ @\ \Cat$ and $g:d\to d'\ @\ \Dat$ then we have $(f,g):(c,d)\to
(c',d')\ @\ \Cop\times\Dat$ and $G^{*}(f,g):G^{*}(c,d)\to G^{*}(c',d')\ @\ \Set$, 
which is $G^{*}(f,g):\Cat(c,G(d))\to\Cat(c',G(d'))$. So if 
$f':c\to G(d)\ @\ \Cat$ then $G^{*}(f,g)(f')$ is well-defined in
$\Cat(c',G(d'))$, i.e. $G^{*}(f,g)(f'):c'\to G(d')\ @\ \Cat$.
\begin{prop}\label{Adj:prop:rhs:arrow}
    Let $F:\Cat\to\Dat$ and $G:\Dat\to\Cat$ be functors where \Cat\ and \Dat\ 
    are locally small categories and let $G^{*}=\Cat\circ(I_{\cal C}\times G)$
    be the right-hand-side functor associated with $(F,G)$. Then for all 
    $c,c'\in\Cat$ and $d,d'\in\Dat$, for all $f:c'\to c$, $g:d\to d'$
    and $f':c\to G(d)$, we have:
        \[
            G^{*}(f,g)(f') = G(g) \circ f'\circ f
        \]
\end{prop}
\noindent
{\bf Remark}: Note that the expression $G(g)\circ f'\circ f$ is a well-defined
arrow in \Cat\ from $c'$ to $G(d')$, since $f:c'\to c$, $f':c\to G(d)$
and $G(g):G(d)\to G(d')$.

\noindent
\begin{proof}
    \begin{eqnarray*}G^{*}(f,g)(f')
        &=&\Cat\circ(I_{\cal C}\times G)(f,g)(f')\\
        \mbox{$(2)$ of def.~(\ref{Fun:def:composition})}\ \to\ 
        &=&\Cat(\,(I_{\cal C}\times G)(f,g)\,)(f')\\
        \mbox{$(2)$ of def.~(\ref{Fun:def:canonical:product})}\ \to\ 
        &=&\Cat(I_{\cal C}(f),G(g))(f')\\
        \mbox{$(2)$ of def.~(\ref{Fun:def:identity})}\ \to\ 
        &=&\Cat(f,G(g))(f')\\
        \mbox{$(2)$ of def.~(\ref{Fun:def:hom:functor})}\ \to\ 
        &=&G(g)\circ f'\circ f
    \end{eqnarray*}
\end{proof}

Suppose $F:\Cat\to\Dat$ and $G:\Dat\to\Cat$ are functors where \Cat\ and \Dat\ 
are locally small categories and let us assume that $(F,G)$ is an adjunction, 
or equivalently that we have a unit $\eta:I_{\cal C}\Rightarrow G\circ F$. 
Then given $c\in\Cat$ and $d\in\Dat$, we can consider the function
$\alpha(c,d):\Dat(F(c),d)\to\Cat(c,G(d))$ which maps any arrow $g:F(c)\to d\ @\ 
\Dat$ to the arrow $\alpha(c,d)(g):c\to G(d)\ @\ \Cat$ defined by the equation
$\alpha(c,d)(g)=G(g)\circ\eta_{c}$. This is a valid definition since we have
$\eta_{c}:c\to G(F(c))$ and $G(g):G(F(c))\to G(d)$ and $G(g)\circ\eta_{c}$
is a well-defined element of $\Cat(c,G(g))$. So $\alpha(c,d)$ is a well-defined
function from the hom-set $\Dat(F(c),d)$ to the hom-set $\Cat(c,G(d))$. In short,
we have $\alpha(c,d):F^{*}(c,d)\to G^{*}(c,d)$ where $F^{*}$ and $G^{*}$
are the left and right-hand side functors associated with $(F,G)$. 
    \[
        \begin{tikzcd}
            \Cop\times\Dat \arrow[r, "F^{*}", bend left  = 50, ""{name=U, below}]
                 \arrow[r, swap, "G^{*}", bend right = 50, ""{name=D, above}]
              & \Set
            \arrow[Rightarrow, "\,\alpha", from = U, to = D]
        \end{tikzcd}
    \]
The obvious question to ask is whether the map $\alpha:\ob\ (\Cop\times\Dat)
\to\arr\ \Set$ we have just defined is in fact a natural transformation.
The answer is 'yes':

\begin{prop}\label{Adj:prop:natural:left:right}
    Let $F:\Cat\to\Dat$ and $G:\Dat\to\Cat$ be functors where \Cat\ and \Dat\ 
    are locally small categories. Let $\eta$ be a unit of $(F,G)$. Consider the
    map $\alpha:\ob\ (\Cop\times\Dat)\to\arr\ \Set$ defined for all $c\in\Cat$ and
    $d\in\Dat$ by:
        \[
            \alpha(c,d)(g)=G(g)\circ\eta_{c}
        \]
    for all $g:F(c)\to d$. Then $\alpha$ is a natural transformation 
    $\alpha:F^{*}\Rightarrow G^{*}$, where $F^{*}$ and $G^{*}$ denote
    the left-hand-side and right-hand-side functors respectively.
\end{prop}
\begin{proof}
    The left-hand-side functor $F^{*}$ as defined in~(\ref{Adj:def:lhs:functor})
    and the right-hand-side functor $G^{*}$ as defined 
    in~(\ref{Adj:def:rhs:functor}) are both functors 
    $F^{*},G^{*}:\Cop\times\Dat\to\Set$. Using definition~(\ref{Nat:def:natural}),
    in order for $\alpha$ to qualify as a natural transformation 
    $\alpha:F^{*}\Rightarrow G^{*}$ it needs to be a map
    $\alpha:\ob\ (\Cop\times\Dat)\to\arr\ \Set$ (which is the case) satisfying
    $(1)$ and $(2)$ of definition~(\ref{Nat:def:natural}). We have
    seen that $\alpha(c,d):F^{*}(c,d)\to G^{*}(c,d)$ for all $c\in\Cat$
    and $d\in\Dat$, that is for all $(c,d)\in\Cop\times\Dat$. Hence $(1)$ of
    definition~(\ref{Nat:def:natural}) is satisfied, and it remains to show~$(2)$.
    So let $c,c'\in\Cat$, $d,d'\in\Dat$ and $k:(c,d)\to(c',d')\ @\ \Cop\times\Dat$.
    Then $k=(f,g)$ for some $f:c'\to c\ @\ \Cat$ and $g:d \to d'\ @\ \Dat$, and
    we need to show that the following square commutes:
    \[
        \begin{tikzcd}
            c
            &d\arrow[d,swap, "g"]
            &\Dat(F(c),d)\arrow[r, "\alpha\mbox{$(c,d)$}"]
             \arrow[d, swap,"F^{*}\mbox{$(f,g)$}"]
            &\Cat(c,G(d))
             \arrow[d, "G^{*}\mbox{$(f,g)$}"]
            \\
            c'\arrow[u, "f"]
            &d'
            &\Dat(F(c'),d')\arrow[r, swap, "\alpha\mbox{$(c',d')$}"]
            &\Cat(c',G(d'))
        \end{tikzcd}
    \]
    Hence we need to show that $G^{*}(f,g)\circ\alpha(c,d)=\alpha(c',d')\circ
    F^{*}(f,g)$. This is an equality between arrows in the category \Set. Using
    proposition~(\ref{Cat:prop:set:arrow:equal}), we can prove this equality
    simply by showing the two underlying untyped functions coincide on 
    every $g'\in\Dat(F(c),d)$. So let $g'\in\Dat(F(c),d)$. We need to show that
    $(G^{*}(f,g)\circ\alpha(c,d))(g')=(\alpha(c',d')\circ F^{*}(f,g))(g')$
    which goes as follows:
        \begin{eqnarray*}(G^{*}(f,g)\circ\alpha(c,d))(g')
            &=&G^{*}(f,g)(\,\alpha(c,d)(g')\,)\\
            &=&G^{*}(f,g)(\,G(g')\circ\eta_{c}\,)\\
            \mbox{prop.~(\ref{Adj:prop:rhs:arrow})}\ \to\ 
            &=&G(g)\circ G(g')\circ\eta_{c}\circ f\\
            \mbox{prop.~(\ref{Adj:prop:unit:natural})}\ \to\ 
            &=&G(g)\circ G(g')\circ (G\circ F)(f)\circ\eta_{c'}\\
            \mbox{$G$ functor}\ \to\ 
            &=&G(\,g\circ g'\circ F(f)\,)\circ\eta_{c'}\\
            \mbox{prop.~(\ref{Adj:prop:lhs:arrow})}\ \to\ 
            &=&G(\,F^{*}(f,g)(g')\,)\circ\eta_{c'}\\
            &=&\alpha(c',d')(\,F^{*}(f,g)(g'))\\
            &=&(\alpha(c',d')\circ F^{*}(f,g))(g')\\
        \end{eqnarray*}
\end{proof}

The natural transformation $\alpha:F^{*}\Rightarrow G^{*}$ above deserves to 
be named:
\begin{defin}\label{Adj:def:natural:associated:unit}
    Let $F:\Cat\to\Dat$ and $G:\Dat\to\Cat$ be functors where \Cat\ and \Dat\ 
    are locally small categories. Let $\eta$ be a unit of $(F,G)$. We call
    {\em natural transformation associated with} $\eta$ the transformation 
    $\alpha:F^{*}\Rightarrow G^{*}$, defined by:
        \[
            \alpha(c,d)(g)=G(g)\circ\eta_{c}
        \]
    for all $c\in\Cat$, $d\in\Dat$ and $g:F(c)\to d$, where $F^{*}$ and $G^{*}$ 
    denote the left-hand-side and right-hand-side functors respectively,
    as per definitions~(\ref{Adj:def:lhs:functor}) and~(\ref{Adj:def:rhs:functor}).
\end{defin}

Suppose $F:\Cat\to\Dat$ and $G:\Dat\to\Cat$ are functors where \Cat\ and \Dat\ 
are locally small categories and let us assume that $(F,G)$ is an adjunction, 
or equivalently that we have a counit $\epsilon:F\circ G\Rightarrow I_{\cal D}$. 
Then given $c\in\Cat$ and $d\in\Dat$, we can consider the function
$\beta(c,d):\Cat(c,G(d))\to\Dat(F(c),d)$ which maps any arrow $f:c\to G(d)\ @\ 
\Cat$ to the arrow $\beta(c,d)(f):F(c)\to d\ @\ \Dat$ defined by the equation
$\beta(c,d)(f)=\epsilon_{d}\circ F(f)$. This is a valid definition since we see 
that
$F(f):F(c)\to F(G(d))$ and $\epsilon_{d}:F(G(d))\to d$ and $\epsilon_{d}\circ 
F(f)$ is a well-defined element of $\Dat(F(c),d)$. So $\beta(c,d)$ is a 
well-defined function from the hom-set $\Cat(c,G(d))$ to the hom-set 
$\Dat(F(c),d)$. In short, $\beta(c,d):G^{*}(c,d)\to F^{*}(c,d)$ 
where $F^{*}$ and $G^{*}$
are the left and right-hand side functors of $(F,G)$. 
    \[
        \begin{tikzcd}
            \Cop\times\Dat \arrow[r, "G^{*}", bend left  = 50, ""{name=U, below}]
                 \arrow[r, swap, "F^{*}", bend right = 50, ""{name=D, above}]
              & \Set
            \arrow[Rightarrow, "\,\beta", from = U, to = D]
        \end{tikzcd}
    \]
The obvious question to ask is whether the map $\beta:\ob\ (\Cop\times\Dat)
\to\arr\ \Set$ we have just defined is in fact a natural transformation.
The answer is 'yes':

\begin{prop}\label{Adj:prop:natural:right:left}
    Let $F:\Cat\to\Dat$ and $G:\Dat\to\Cat$ be functors where \Cat\ and \Dat\ 
    are locally small categories. Let $\epsilon$ be a counit of $(F,G)$. 
    Consider the map $\beta:\ob\ (\Cop\times\Dat)\to\arr\ \Set$ defined 
    for all $c\in\Cat$ and $d\in\Dat$ by:
        \[
            \beta(c,d)(f)=\epsilon_{d}\circ F(f)
        \]
    for all $f:c\to G(d)$. Then $\beta$ is a natural transformation 
    $\beta:G^{*}\Rightarrow F^{*}$, where $F^{*}$ and $G^{*}$ denote
    the left-hand-side and right-hand-side functors respectively.
\end{prop}
\begin{proof}
    The left-hand-side functor $F^{*}$ as defined in~(\ref{Adj:def:lhs:functor})
    and the right-hand-side functor $G^{*}$ as defined 
    in~(\ref{Adj:def:rhs:functor}) are both functors 
    $F^{*},G^{*}:\Cop\times\Dat\to\Set$. Using definition~(\ref{Nat:def:natural}),
    in order for $\beta$ to qualify as a natural transformation 
    $\beta:G^{*}\Rightarrow F^{*}$ it needs to be a map
    $\beta:\ob\ (\Cop\times\Dat)\to\arr\ \Set$ (which is the case) satisfying
    $(1)$ and $(2)$ of definition~(\ref{Nat:def:natural}). We have
    seen that $\beta(c,d):G^{*}(c,d)\to F^{*}(c,d)$ for all $c\in\Cat$
    and $d\in\Dat$, that is for all $(c,d)\in\Cop\times\Dat$. Hence $(1)$ of
    definition~(\ref{Nat:def:natural}) is satisfied, and it remains to show~$(2)$.
    So let $c,c'\in\Cat$, $d,d'\in\Dat$ and $k:(c,d)\to(c',d')\ @\ \Cop\times\Dat$.
    Then $k=(f,g)$ for some $f:c'\to c\ @\ \Cat$ and $g:d \to d'\ @\ \Dat$, and
    we need to show that the following square commutes:
    \[
        \begin{tikzcd}
            c
            &d\arrow[d,swap, "g"]
            &\Cat(c,G(d))\arrow[r, "\beta\mbox{$(c,d)$}"]
             \arrow[d, swap,"G^{*}\mbox{$(f,g)$}"]
            &\Dat(F(c),d)
             \arrow[d, "F^{*}\mbox{$(f,g)$}"]
            \\
            c'\arrow[u, "f"]
            &d'
            &\Cat(c',G(d'))\arrow[r, swap, "\beta\mbox{$(c',d')$}"]
            &\Dat(F(c'),d')
        \end{tikzcd}
    \]
    Hence we need to show that $F^{*}(f,g)\circ\beta(c,d)=\beta(c',d')\circ
    G^{*}(f,g)$. This is an equality between arrows in the category \Set. Using
    proposition~(\ref{Cat:prop:set:arrow:equal}), we can prove this equality
    simply by showing the two underlying untyped functions coincide on 
    every $f'\in\Cat(c,G(d))$. So let $f'\in\Cat(c,G(d))$. We need to show that
    $(F^{*}(f,g)\circ\beta(c,d))(f')=(\beta(c',d')\circ G^{*}(f,g))(f')$
    which goes as follows:
        \begin{eqnarray*}(F^{*}(f,g)\circ\beta(c,d))(f')
            &=&F^{*}(f,g)(\,\beta(c,d)(f')\,)\\
            &=&F^{*}(f,g)(\,\epsilon_{d}\circ F(f')\,)\\
            \mbox{prop.~(\ref{Adj:prop:lhs:arrow})}\ \to\ 
            &=&g\circ \epsilon_{d}\circ F(f')\circ F(f)\\
            \mbox{prop.~(\ref{Adj:prop:counit:natural})}\ \to\ 
            &=&\epsilon_{d'}\circ (F\circ G)(g)\circ F(f')\circ F(f)\\
            \mbox{$F$ functor}\ \to\ 
            &=&\epsilon_{d'}\circ F(\,G(g)\circ f'\circ f\,)\\
            \mbox{prop.~(\ref{Adj:prop:rhs:arrow})}\ \to\ 
            &=&\epsilon_{d'}\circ F(\,G^{*}(f,g)(f')\,)\\
            &=&\beta(c',d')(\,G^{*}(f,g)(f')\,)\\
            &=&(\beta(c',d')\circ G^{*}(f,g))(f')\\
        \end{eqnarray*}
\end{proof}

The natural transformation $\beta:G^{*}\Rightarrow F^{*}$ above deserves to 
be named:
\begin{defin}\label{Adj:def:natural:associated:counit}
    Let $F:\Cat\to\Dat$ and $G:\Dat\to\Cat$ be functors where \Cat\ and \Dat\ 
    are locally small categories. Let $\epsilon$ be a counit of $(F,G)$. We call
    {\em natural transformation associated with} $\epsilon$ the transformation 
    $\beta:G^{*}\Rightarrow F^{*}$, defined by:
        \[
            \beta(c,d)(f)=\epsilon_{d}\circ F(f)
        \]
    for all $c\in\Cat$, $d\in\Dat$ and $f:c\to G(d)$, where $F^{*}$ and $G^{*}$ 
    denote the left-hand-side and right-hand-side functors respectively,
    as per definitions~(\ref{Adj:def:lhs:functor}) and~(\ref{Adj:def:rhs:functor}).
\end{defin}

\begin{prop}\label{Adj:prop:natural:associated:unit:isomorphism}
    Let $F:\Cat\to\Dat$ and $G:\Dat\to\Cat$ be functors where \Cat\ and \Dat\ 
    are locally small categories. Let $\eta$ be a unit of $(F,G)$ and 
    $\alpha:F^{*}\Rightarrow G^{*}$ be the natural transformation associated
    with $\eta$, as per~(\ref{Adj:def:natural:associated:unit}). Then $\alpha$ 
    is a natural isomorphism with inverse $\beta:G^{*}\Rightarrow F^{*}$ 
    associated with the related counit $\epsilon$.
\end{prop}
\noindent
{\bf Remark}: Recall from proposition~(\ref{Adj:prop:unit:has:related:counit})
that every unit $\eta$ of $(F,G)$ has a unique related counit $\epsilon$ and
$\beta:G^{*}\Rightarrow F^{*}$ is therefore well-defined by virtue 
of~(\ref{Adj:def:natural:associated:counit}).
\begin{proof}
    Using proposition~(\ref{Nat:prop:isomorphism}), we need to show that 
    $\beta\circ\alpha=\iota_{F^{*}}$ and $\alpha\circ\beta
    =\iota_{G^{*}}$, where $F^{*}$ and $G^{*}$ denote the left-hand-side 
    and right-hand-side functors respectively, as per 
    definitions~(\ref{Adj:def:lhs:functor}) and~(\ref{Adj:def:rhs:functor}).
    We shall start with $\beta\circ\alpha=\iota_{F^{*}}$. From
    proposition~(\ref{Nat:prop:equal}), in order to prove this equality it 
    is sufficient to show that all components are equal. So given
    $c\in\Cat$ and $d\in\Dat$, i.e. given an arbitrary $(c,d)\in\Cop\times\Dat$,
    we need to show that $(\beta\circ\alpha)(c,d)=\iota_{F^{*}}(c,d)$.
    This is an equality between two arrows of the category \Set\ with
    domain and codomain $F^{*}(c,d)=\Dat(F(c),d)$. Using 
    proposition~(\ref{Cat:prop:set:arrow:equal}), it is sufficient to show
    that for all $g:F(c)\to d\ @\ \Dat$ we have $(\beta\circ\alpha)(c,d)(g)=
    \iota_{F^{*}}(c,d)(g)$:
        \begin{eqnarray*}(\beta\circ\alpha)(c,d)(g)
            &=&(\,\beta(c,d)\circ\alpha(c,d)\,)(g)
            \ \leftarrow\ \mbox{def.~(\ref{Nat:def:composition})}\\
            \mbox{$\circ$ on \Set}\ \to\ 
            &=&\beta(c,d)(\,\alpha(c,d)(g)\,)\\
            \mbox{def.~(\ref{Adj:def:natural:associated:unit})}\ \to\ 
            &=&\beta(c,d)(\,G(g)\circ\eta_{c}\,)\\
            \mbox{def.~(\ref{Adj:def:natural:associated:counit})}\ \to\ 
            &=&\epsilon_{d}\circ F(\,G(g)\circ\eta_{c}\,)\\
            \mbox{$F$ functor}\ \to\ 
            &=&\epsilon_{d}\circ (F\circ G)(g)\circ F(\eta_{c})\\
            \mbox{prop.~(\ref{Adj:prop:counit:natural})}\ \to\ 
            &=&g\circ\epsilon_{F(c)}\circ F(\eta_{c})\\
            \mbox{def.~(\ref{Nat:def:leftmul}) and~(\ref{Nat:def:rightmul})}\ \to\ 
            &=&g\circ(\epsilon F\circ F\eta)(c)\\
            \mbox{def.~(\ref{Adj:prop:related:both}), $\eta$ and 
            $\epsilon$ related}\ \to\ 
            &=&g\circ\iota_{F}(c)\\
            \mbox{def.~(\ref{Nat:def:identity})}\ \to\ 
            &=&g\circ\id(\,F(c)\,)\\
            &=&g\\
            \mbox{$g:F(c)\to d\ @\ \Dat$}\ \to\ 
            &=&\id(\,\Dat(F(c),d)\,)(g)\\
            \mbox{prop.~(\ref{Adj:prop:lhs:object})}\ \to\ 
            &=&\id(\,F^{*}(c,d)\,)(g)\\
            \mbox{def.~(\ref{Nat:def:identity})}\ \to\ 
            &=&\iota_{F^{*}}(c,d)(g)\\
        \end{eqnarray*} 
    So it remains to show that $\alpha\circ\beta=\iota_{G^{*}}$. Similarly,
    given $c\in\Cat$, $d\in\Dat$ and $f:c\to G(d)\ @\ \Cat$, we need to 
    show that $(\alpha\circ\beta)(c,d)(f)=\iota_{G^{*}}(c,d)(f)$:
        \begin{eqnarray*}(\alpha\circ\beta)(c,d)(f)
            &=&(\,\alpha(c,d)\circ\beta(c,d)\,)(f)
            \ \leftarrow\ \mbox{def.~(\ref{Nat:def:composition})}\\
            \mbox{$\circ$ on \Set}\ \to\ 
            &=&\alpha(c,d)(\,\beta(c,d)(f)\,)\\
            \mbox{def.~(\ref{Adj:def:natural:associated:counit})}\ \to\ 
            &=&\alpha(c,d)(\,\epsilon_{d}\circ F(f)\,)\\
            \mbox{def.~(\ref{Adj:def:natural:associated:unit})}\ \to\ 
            &=&G(\,\epsilon_{d}\circ F(f)\,)\circ\eta_{c}\\
            \mbox{$G$ functor}\ \to\ 
            &=&G(\epsilon_{d})\circ(G\circ F)(f)\circ\eta_{c}\\
            \mbox{prop.~(\ref{Adj:prop:unit:natural})}\ \to\ 
            &=&G(\epsilon_{d})\circ\eta_{G(d)}\circ f\\
            \mbox{def.~(\ref{Nat:def:leftmul}) and~(\ref{Nat:def:rightmul})}\ \to\ 
            &=&(G\epsilon\circ\eta G)(d)\circ f\\
            \mbox{def.~(\ref{Adj:prop:related:both}), $\eta$ and 
            $\epsilon$ related}\ \to\ 
            &=&\iota_{G}(d)\circ f\\
            \mbox{def.~(\ref{Nat:def:identity})}\ \to\ 
            &=&\id(\,G(d)\,)\circ f\\
            &=&f\\
            \mbox{$f:c\to G(d)\ @\ \Cat$}\ \to\ 
            &=&\id(\,\Cat(c,G(d))\,)(f)\\
            \mbox{prop.~(\ref{Adj:prop:rhs:object})}\ \to\ 
            &=&\id(\,G^{*}(c,d)\,)(f)\\
            \mbox{def.~(\ref{Nat:def:identity})}\ \to\ 
            &=&\iota_{G^{*}}(c,d)(f)\\
        \end{eqnarray*} 
\end{proof}

\begin{prop}\label{Adj:prop:natural:associated:counit:isomorphism}
    Let $F:\Cat\to\Dat$ and $G:\Dat\to\Cat$ be functors where \Cat\ and \Dat\ 
    are locally small categories. Let $\epsilon$ be a counit of $(F,G)$ and 
    $\beta:G^{*}\Rightarrow F^{*}$ be the natural transformation associated
    with $\epsilon$, as per~(\ref{Adj:def:natural:associated:counit}). Then 
    $\beta$ is a natural isomorphism with inverse $\alpha:F^{*}\Rightarrow G^{*}$ 
    associated with the related unit $\eta$.
\end{prop}
\noindent
{\bf Remark}: Recall from proposition~(\ref{Adj:prop:counit:has:related:unit})
that every counit $\epsilon$ of $(F,G)$ has a unique related unit $\eta$ and
$\alpha:F^{*}\Rightarrow G^{*}$ is therefore well-defined by virtue 
of~(\ref{Adj:def:natural:associated:unit}).

\noindent
\begin{proof}
    Saying that $\beta$ is a natural isomorphism with inverse $\alpha$ is the
    same as saying that $\alpha$ is a natural isomorphism with inverse $\beta$.
    So the result follows immediately from 
    proposition~(\ref{Adj:prop:natural:associated:unit:isomorphism}),
    since $\eta$ and $\epsilon$ are related unit and counit of $(F,G)$ and
    $\alpha:F^{*}\Rightarrow G^{*}$ is associated with $\eta$ while
    $\beta:G^{*}\Rightarrow F^{*}$ is associated with $\epsilon$.
\end{proof}

\begin{prop}\label{Adj:prop:adjunction:F*:G*:iso}
    Let $F:\Cat\to\Dat$ and $G:\Dat\to\Cat$ be functors where \Cat\ and \Dat\ 
    are locally small categories. Then we have the implication:
        \[
            F\dashv G\ \Rightarrow\  F^{*}\simeq G^{*}
        \]
    where $F^{*}$ and $G^{*}$ denote the left-hand-side and right-hand-side 
    functors respectively associated with $(F,G)$, as per 
    definitions~(\ref{Adj:def:lhs:functor}) and~(\ref{Adj:def:rhs:functor}). 
    In other words, if $(F,G)$ is an adjunction then $F^{*}$ and $G^{*}$ are 
    naturally isomorphic.
\end{prop}
\begin{proof}
    We assume that $F\dashv G$, i.e. that $(F,G)$ is an adjunction. In other
    words we assume that $(F,G)$ has a unit $\eta:I_{\cal C}\Rightarrow G\circ F$
    as per definition~(\ref{Adj:def:unit}) and we need to show that 
    $F^{*}\simeq G^{*}$ i.e. that the left-hand-side functor $F^{*}$ is 
    isomorphic to the right-hand-side functor $G^{*}$ as per
    definition~(\ref{Cat:def:isomorphic}). Note that in order for the 
    statement $F^{*}\simeq G^{*}$ to make sense, we need $F^{*}$ and $G^{*}$
    to be objects of a common category. Since both $F^{*}$ and $G^{*}$ are
    functors between the categories $\Cop\times\Dat$ and \Set, it is 
    understood that the common category under consideration is the functor
    category $[\,\Cop\times\Dat\,,\,\Set\,]$ as per 
    definition~(\ref{Nat:def:functor:category}). From 
    definition~(\ref{Cat:def:isomorphic}), in order to prove that $F^{*}\simeq 
    G^{*}$ we need to show the existence of an isomophism $\alpha:F^{*}
    \Rightarrow G^{*}$. However, having assumed that $(F,G)$ has a unit $\eta$,
    we can consider the associated natural transformation $\alpha : F^{*}
    \Rightarrow G^{*}$ as per definition~(\ref{Adj:def:natural:associated:unit})
    which is a natural isomorphism as follows 
    from~(\ref{Adj:prop:natural:associated:unit:isomorphism}).
\end{proof}

Suppose $F:\Cat\to\Dat$ and $G:\Dat\to\Cat$ are functors where \Cat\ and \Dat\ 
are locally small categories, and let $\alpha:F^{*}\Rightarrow G^{*}$ be a 
natural transformation between the associated left-hand-side and right-hand-side
functors. Then for all $c\in\Cat$ and $d\in\Dat$, i.e. for all $(c,d)\in
\Cop\times\Dat$, the component $\alpha(c,d)$ is an arrow
$\alpha(c,d):\Dat(F(c),d)\to\Cat(c,G(d))\ @\ \Set$. Hence for all $c\in\Cat$,
taking $d=F(c)$, we have $\alpha(c,F(c)):\Dat(F(c),F(c))\to\Cat(c,G(F(c)))$.
Consequently, if we define $\eta_{c}=\alpha(c,F(c))(\,\id(\,F(c)\,)\,)$ then
$\eta_{c}$ is a well-defined element of $\Cat(c,G(F(c)))$, i.e.
$\eta_{c}$ is a well-defined arrow $\eta_{c}:c\to G(F(c))\ @\ \Cat$. The question
arises as to whether the map $\eta:\ob\ \Cat\to\arr\ \Cat$ we have just defined
is in fact a natural transformation $\eta:I_{\cal C}\Rightarrow G\circ F$.
This is in fact the case:

\begin{prop}\label{Adj:prop:alpha:F*:G*:eta}
    Let $F:\Cat\to\Dat$ and $G:\Dat\to\Cat$ be functors where \Cat\ and \Dat\ 
    are locally small categories, and let $F^{*}$ and $G^{*}$ denote the associated
    left-hand-side and right-hand-side functors respectively as per 
    definitions~(\ref{Adj:def:lhs:functor}) and~(\ref{Adj:def:rhs:functor}). Let
    $\alpha:F^{*}\Rightarrow G^{*}$ be a natural transformation and for all
    $c\in\Cat$ define:
        \begin{equation}\label{Adj:eqn:alpha:F*:G*:eta:1}
            \eta_{c}=\alpha(c,F(c))(\,\id(\,F(c)\,)\,)
        \end{equation}
    Then $\eta$ is a natural transformation $\eta:I_{\cal C}\Rightarrow G\circ F$
    and furthermore the equation:
        \begin{equation}\label{Adj:eqn:alpha:F*:G*:eta:2}
            \alpha(c,d)(g)=G(g)\circ\eta_{c}
        \end{equation}
    holds for all $c\in\Cat$, $d\in\Dat$ and $g:F(c)\to d$.
\end{prop}
\noindent
{\bf Remark}: Note that for all $c\in\Cat$, $d\in\Dat$ and $g:F(c)\to d$, since
we have $\alpha(c,d):\Dat(F(c),d)\to\Cat(c,G(d))$, the left-hand-side
$\alpha(c,d)(g)$ of equation~(\ref{Adj:eqn:alpha:F*:G*:eta:2}) is a well-defined
arrow $\alpha(c,d)(g):c\to G(d)\ @\ \Cat$. Also, since $\eta_{c}:c\to
G(F(c))$ and $G(g):G(F(c))\to G(d)$, the right-hand-side $G(g)\circ\eta_{c}$ of
equation~(\ref{Adj:eqn:alpha:F*:G*:eta:2}) is likewise a well-defined arrow
from $c$ to $G(d)$ in the category \Cat.

\noindent
\begin{proof}
    Given $c\in\Cat$, $d\in\Dat$ and $g:F(c)\to d$, we shall first establish 
    equation~(\ref{Adj:eqn:alpha:F*:G*:eta:2}). Since $\alpha$
    is a natural transformation, the following square commutes:
        \[
            \begin{tikzcd}
                c
                &F(c)\arrow[d,swap, "g"]
                &\Dat(F(c),F(c))\arrow[r, "\alpha\mbox{$(c,F(c))$}"]
                \arrow[d, swap,"F^{*}\mbox{$(\id(c),g)$}"]
                &\Cat(c,G(F(c)))
                \arrow[d, "G^{*}\mbox{$(\id(c),g)$}"]
                \\
                c\arrow[u, "\id(c)"]
                &d
                &\Dat(F(c),d)\arrow[r, swap, "\alpha\mbox{$(c,d)$}"]
                &\Cat(c,G(d))
            \end{tikzcd}
        \]
    In other words the following equation holds: 
        \begin{equation}\label{Adj:eqn:alpha:F*:G*:eta:3}
            \alpha(c,d)\circ F^{*}(\,\id(c)\,,\,g\,) = 
            G^{*}(\,\id(c)\,,\,g\,)\circ\alpha(c,F(c))
        \end{equation}
    Hence we have:
        \begin{eqnarray*}\alpha(c,d)(g)
            &=&\alpha(c,d)(\,g\circ\id(\,F(c)\,)\circ\id(\,F(c)\,)\,)\\
            \mbox{$F$ functor}\ \to\ 
            &=&\alpha(c,d)(\,g\circ\id(\,F(c)\,)\circ F(\,\id(c)\,)\,)\\
            \mbox{prop.~(\ref{Adj:prop:lhs:arrow})}\ \to\ 
            &=&\alpha(c,d)(\,F^{*}(\,\id(c)\,,\,g\,)(\,\id(\,F(c)\,)\,)\,)\\
            &=&(\,\alpha(c,d)\circ F^{*}(\,\id(c)\,,\,g\,)\,)(\,\id(\,F(c)\,)\,)\\
            \mbox{eqn.~(\ref{Adj:eqn:alpha:F*:G*:eta:3})}\ \to\ 
            &=&(\,G^{*}(\,\id(c)\,,\,g\,)\circ\alpha(c,F(c))\,)(\,\id(\,F(c)\,)\,)\\
            &=&G^{*}(\,\id(c)\,,\,g\,)(\,\alpha(c,F(c))(\,\id(\,F(c)\,)\,)\,)\\
            \mbox{eqn.~(\ref{Adj:eqn:alpha:F*:G*:eta:1})}\ \to\ 
            &=&G^{*}(\,\id(c)\,,\,g\,)(\eta_{c})\\
            \mbox{prop.~(\ref{Adj:prop:rhs:arrow})}\ \to\ 
            &=&G(g)\circ\eta_{c}\circ\id(c)\\
            &=&G(g)\circ\eta_{c}\\
        \end{eqnarray*}
    So it remains to show that $\eta$ is a natural transformation 
    $\eta:I_{\cal C}\Rightarrow G\circ F$. So we need to check that 
    $\eta$ is a map $\eta:\ob\ \Cat\to\arr\ \Cat$ (which it is) satisfying
    $(1)$ and $(2)$ of definition~(\ref{Nat:def:natural}). We have already
    seen that for all $c\in\Cat$, $\eta_{c}$ of
    equation~(\ref{Adj:eqn:alpha:F*:G*:eta:1}) is a well defined arrow
    $\eta_{c}:c\to G(F(c))$. So $(1)$ is satisfied and it remains to show $(2)$.
    So given $c,c'\in\Cat$ and $f:c\to c'$ we need to show that
    $G(F(f))\circ\eta_{c}=\eta_{c'}\circ f$, that is the following 
    square commutes:
        \[
            \begin{tikzcd}
                c\arrow[r, "\eta_{c}"]\arrow[d, swap,"f\ "]
                &G(F(c))\arrow[d, "(G\circ F)(f)"]
                \\
                c'\arrow[r, swap, "\eta_{c'}"]
                &G(F(c'))
            \end{tikzcd}
        \]
    However, since $\alpha$ is natural, setting $d=F(c')$, the following
    square commutes:
        \[
            \begin{tikzcd}
                c'
                &d\arrow[d,swap, "\id(d)"]
                &\Dat(F(c'),d)\arrow[r, "\alpha\mbox{$(c',d)$}"]
                \arrow[d, swap,"F^{*}\mbox{$(f,\id(d))$}"]
                &\Cat(c',G(d))
                \arrow[d, "G^{*}\mbox{$(f,\id(d))$}"]
                \\
                c\arrow[u, "f"]
                &d
                &\Dat(F(c),d)\arrow[r, swap, "\alpha\mbox{$(c,d)$}"]
                &\Cat(c,G(d))
            \end{tikzcd}
        \]
    In other words...TODO
    Hence we have: 
        \begin{eqnarray*}G(F(f))\circ\eta_{c}
            &=&TODO
        \end{eqnarray*}
\end{proof}
