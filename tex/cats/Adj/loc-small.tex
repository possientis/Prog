In this section, we consider the case when both categories \Cat\ and \Dat\ are
locally small, as per definition~(\ref{Cat:def:locally:small}). The difference
this makes is the availability of the associated hom-functors
$\Cat:\Cop\times\Cat\to\Set$ and $\Dat:\Dop\times\Dat\to\Set$ as per
definition~(\ref{Fun:def:hom:functor}). So suppose
we have two functors $F:\Cat\to\Dat$ and $G:\Dat\to\Cat$. We also have
identity functors $I_{\cal C}:\Cat\to\Cat$ and $I_{\cal D}:\Dat\to\Dat$
as per definition~(\ref{Fun:def:identity}). Using 
proposition~(\ref{Fun:prop:opposite}), we know that $F$ is also a functor 
$F:\Cop\to\Dop$, while from proposition~(\ref{Fun:prop:identity:opposite})
we see that $I_{\cal C}$ and $I_{{\cal C}^{op}}$ are the same functors.
Using definition~(\ref{Fun:def:canonical:product}) we can define the 
product functors $F\times I_{\cal D}:\Cop\times\Dat\to\Dop\times\Dat$.
and $I_{\cal C}\times G:\Cop\times\Dat\to\Cop\times\Cat$. Composing 
these functors with the hom-functors \Dat\ and \Cat\ respectively as per
definition~(\ref{Fun:def:composition}), we obtain
two functors $\Dat\circ(F\times I_{\cal D}):\Cop\times\Dat\to\Set$ and
$\Cat\circ(I_{\cal C}\times G):\Cop\times\Dat\to\Set$. These functors 
are very important in what follows, so we shall give then a name:

\begin{defin}\label{Adj:def:lhs:functor}
    Let $F:\Cat\to\Dat$ and $G:\Dat\to\Cat$ be functors where \Cat\ and \Dat\ 
    are locally small categories. We call {\em left-hand-side functor} associated
    with the pair $(F,G)$ the functor $\Dat\circ(F\times I_{\cal D}):\Cop\times
    \Dat\to\Set$.
\end{defin}

\begin{defin}\label{Adj:def:rhs:functor}
    Let $F:\Cat\to\Dat$ and $G:\Dat\to\Cat$ be functors where \Cat\ and \Dat\ 
    are locally small categories. We call {\em right-hand-side functor} associated
    with the pair $(F,G)$ the functor $\Cat\circ(I_{\cal C}\times G):\Cop\times
    \Dat\to\Set$.
\end{defin}

\begin{prop}\label{Adj:prop:lhs:object}
    Let $F:\Cat\to\Dat$ and $G:\Dat\to\Cat$ be functors where \Cat\ and \Dat\ 
    are locally small categories and let $F^{*}=\Dat\circ(F\times I_{\cal D})$
    be the left-hand-side functor associated with $(F,G)$. Then for all $c\in\Cat$ 
    and $d\in\Dat$, we have:
        \[
            F^{*}(c,d) = \Dat(F(c),d)
        \]
\end{prop}
\begin{proof}
    \begin{eqnarray*}F^{*}(c,d)
        &=&\Dat\circ(F\times I_{\cal D})(c,d)\\
        \mbox{$(1)$ of def.~(\ref{Fun:def:composition})}\ \to\ 
        &=&\Dat(\,(F\times I_{\cal D})(c,d)\,)\\
        \mbox{$(1)$ of def.~(\ref{Fun:def:canonical:product})}\ \to\ 
        &=&\Dat(F(c),I_{\cal D}(d))\\
        \mbox{$(1)$ of def.~(\ref{Fun:def:identity})}\ \to\ 
        &=&\Dat(F(c),d)
    \end{eqnarray*}
\end{proof}

\noindent
{\bf Remark}: In light of proposition~(\ref{Adj:prop:lhs:object}), it is
common to refer to the left-hand-side functor casually as '$\Dat(F(c),d)$'
where '$c$' and '$d$' are dummy variables. We should not forget however
that a functor is more than a mere transformation on objects, and that 
the left-hand-side functor is really $\Dat\circ(F\times I_{\cal D})$.

\begin{prop}\label{Adj:prop:rhs:object}
    Let $F:\Cat\to\Dat$ and $G:\Dat\to\Cat$ be functors where \Cat\ and \Dat\ 
    are locally small categories and let $G^{*}=\Cat\circ(I_{\cal C}\times G)$
    be the right-hand-side functor associated with $(F,G)$. Then for all 
    $c\in\Cat$ and $d\in\Dat$, we have:
        \[
            G^{*}(c,d) = \Cat(c,G(d))
        \]
\end{prop}
\begin{proof}
    \begin{eqnarray*}G^{*}(c,d)
        &=&\Cat\circ(I_{\cal C}\times G)(c,d)\\
        \mbox{$(1)$ of def.~(\ref{Fun:def:composition})}\ \to\ 
        &=&\Cat(\,(I_{\cal C}\times G)(c,d)\,)\\
        \mbox{$(1)$ of def.~(\ref{Fun:def:canonical:product})}\ \to\ 
        &=&\Cat(I_{\cal C}(c),G(d))\\
        \mbox{$(1)$ of def.~(\ref{Fun:def:identity})}\ \to\ 
        &=&\Cat(c,G(d))
    \end{eqnarray*}
\end{proof}

\noindent
{\bf Remark}: In light of proposition~(\ref{Adj:prop:rhs:object}), it is
common to refer to the right-hand-side functor casually as '$\Cat(c,G(d))$'
where '$c$' and '$d$' are dummy variables. Once again, a functor is more than 
a mere transformation on objects, and we should remember that the right-hand-side 
functor is really $\Cat\circ(I_{\cal C}\times G)$.

The left-hand side functor $F^{*}=\Dat\circ(F\times I_{\cal D})$ is a functor
$F^{*}:\Cop\times\Dat\to\Set$. If $c,c'\in\Cat$ and $d,d'\in\Dat$, if 
$f:c'\to c\ @\ \Cat$ and $g:d\to d'\ @\ \Dat$ then we have $(f,g):(c,d)\to
(c',d')\ @\ \Cop\times\Dat$ and $F^{*}(f,g):F^{*}(c,d)\to F^{*}(c',d')\ @\ \Set$, 
which is $F^{*}(f,g):\Dat(F(c),d)\to\Dat(F(c'),d')$. So if 
$g':F(c)\to d\ @\ \Dat$ then $F^{*}(f,g)(g')$ is well-defined in
$\Dat(F(c'),d')$, i.e. $F^{*}(f,g)(g'):F(c')\to d'\ @\ \Dat$.
\begin{prop}\label{Adj:prop:lhs:arrow}
    Let $F:\Cat\to\Dat$ and $G:\Dat\to\Cat$ be functors where \Cat\ and \Dat\ 
    are locally small categories and let $F^{*}=\Dat\circ(F\times I_{\cal D})$
    be the left-hand-side functor associated with $(F,G)$. Then for all 
    $c,c'\in\Cat$ and $d,d'\in\Dat$, for all $f:c'\to c$, $g:d\to d'$
    and $g':F(c)\to d$, we have:
        \[
            F^{*}(f,g)(g') = g \circ g'\circ F(f)
        \]
\end{prop}
\noindent
{\bf Remark}: Note that the expression $g\circ g'\circ F(f)$ is a well-defined
arrow in \Dat\ from $F(c')$ to $d'$, since $F(f):F(c')\to F(c)$, $g':F(c)\to d$
and $g:d\to d'$.

\noindent
\begin{proof}
    \begin{eqnarray*}F^{*}(f,g)(g')
        &=&\Dat\circ(F\times I_{\cal D})(f,g)(g')\\
        \mbox{$(2)$ of def.~(\ref{Fun:def:composition})}\ \to\ 
        &=&\Dat(\,(F\times I_{\cal D})(f,g)\,)(g')\\
        \mbox{$(2)$ of def.~(\ref{Fun:def:canonical:product})}\ \to\ 
        &=&\Dat(F(f),I_{\cal D}(g))(g')\\
        \mbox{$(2)$ of def.~(\ref{Fun:def:identity})}\ \to\ 
        &=&\Dat(F(f),g)(g')\\
        \mbox{$(2)$ of def.~(\ref{Fun:def:hom:functor})}\ \to\ 
        &=&g\circ g'\circ F(f)
    \end{eqnarray*}
\end{proof}

The right-hand side functor $G^{*}=\Cat\circ(I_{\cal C}\times G)$ is a functor
$G^{*}:\Cop\times\Dat\to\Set$. If $c,c'\in\Cat$ and $d,d'\in\Dat$, if 
$f:c'\to c\ @\ \Cat$ and $g:d\to d'\ @\ \Dat$ then we have $(f,g):(c,d)\to
(c',d')\ @\ \Cop\times\Dat$ and $G^{*}(f,g):G^{*}(c,d)\to G^{*}(c',d')\ @\ \Set$, 
which is $G^{*}(f,g):\Cat(c,G(d))\to\Cat(c',G(d'))$. So if 
$f':c\to G(d)\ @\ \Cat$ then $G^{*}(f,g)(f')$ is well-defined in
$\Cat(c',G(d'))$, i.e. $G^{*}(f,g)(f'):c'\to G(d')\ @\ \Cat$.
\begin{prop}\label{Adj:prop:rhs:arrow}
    Let $F:\Cat\to\Dat$ and $G:\Dat\to\Cat$ be functors where \Cat\ and \Dat\ 
    are locally small categories and let $G^{*}=\Cat\circ(I_{\cal C}\times G)$
    be the right-hand-side functor associated with $(F,G)$. Then for all 
    $c,c'\in\Cat$ and $d,d'\in\Dat$, for all $f:c'\to c$, $g:d\to d'$
    and $f':c\to G(d)$, we have:
        \[
            G^{*}(f,g)(f') = G(g) \circ f'\circ f
        \]
\end{prop}
\noindent
{\bf Remark}: Note that the expression $G(g)\circ f'\circ f$ is a well-defined
arrow in \Cat\ from $c'$ to $G(d')$, since $f:c'\to c$, $f':c\to G(d)$
and $G(g):G(d)\to G(d')$.

\noindent
\begin{proof}
    \begin{eqnarray*}G^{*}(f,g)(f')
        &=&\Cat\circ(I_{\cal C}\times G)(f,g)(f')\\
        \mbox{$(2)$ of def.~(\ref{Fun:def:composition})}\ \to\ 
        &=&\Cat(\,(I_{\cal C}\times G)(f,g)\,)(f')\\
        \mbox{$(2)$ of def.~(\ref{Fun:def:canonical:product})}\ \to\ 
        &=&\Cat(I_{\cal C}(f),G(g))(f')\\
        \mbox{$(2)$ of def.~(\ref{Fun:def:identity})}\ \to\ 
        &=&\Cat(f,G(g))(f')\\
        \mbox{$(2)$ of def.~(\ref{Fun:def:hom:functor})}\ \to\ 
        &=&G(g)\circ f'\circ f
    \end{eqnarray*}
\end{proof}

\begin{prop}\label{Adj:prop:natural:left:right}
    Let $F:\Cat\to\Dat$ and $G:\Dat\to\Cat$ be functors where \Cat\ and \Dat\ 
    are locally small categories. Let $\eta$ be a unit of $(F,G)$. Consider the
    map $\alpha:\ob\ (\Cop\times\Dat)\to\arr\ \Set$ defined for all $c\in\Cat$ and
    $d\in\Dat$ by:
        \[
            \alpha(c,d)(g)=G(g)\circ\eta_{c}
        \]
    for all $g:F(c)\to d$. Then $\alpha$ is a natural transformation 
    $\alpha:F^{*}\Rightarrow G^{*}$, where $F^{*}$ and $G^{*}$ denote
    the left-hand-side and right-hand-side functors respectively.
\end{prop}
