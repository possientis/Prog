In this section we introduce the notion of universal sub-algebras.
The main motivation for this will appear later when we investigate
equivalence relations and congruences on universal algebras. An
equivalence relation on a universal algebra $X$ is a subset of
$X\times X$ with certain closure properties, and can therefore be
regarded as a universal sub-algebra of $X\times X$, provided the
latter has been embedded with the appropriate structure of universal
algebra. Rather, than focussing on the specifics of $X\times X$, we
will study universal sub-algebras in a general setting.
\index{subalgebra@Sub-algebra of universal algebra}
\begin{defin}\label{logic:def:universal:sub:algebra}
Let $X$ be a universal algebra of type $\alpha$. We say that
$Y\subseteq X$ is a {\em Universal Sub-Algebra} of $X$ \ifand\ it is
closed under every operator i.e.
    \[
    \forall f\in\alpha\ ,\ \forall x\in X^{\alpha(f)}\ ,\ [\ x\in
    Y^{\alpha(f)}\ \Rightarrow\ f(x)\in Y\ ]
    \]
If $Y$ is a universal sub-algebra of $X$, we call {\em induced
structure on} $Y$ the map $T$ with domain $\alpha$ such that for all
$f\in\alpha$, the operator $T(f):Y^{\alpha(f)}\to Y$ is:
    \[
    \forall x\in Y^{\alpha(f)}\ ,\ T(f)(x) = f(x)\in Y
    \]
\end{defin}

Let $X$ be a universal algebra of type $\alpha$ and $Y$ be a
universal sub-algebra of $X$. Given $f\in\alpha$, when using loose
notations we have two operators $f:X^{\alpha(f)}\to X$ and
$f:Y^{\alpha(f)}\to Y$. However, the latter is simply the
restriction of the former to the smaller domain
$Y^{\alpha(f)}\subseteq X^{\alpha(f)}$, and the notation '$f(x)$' is
therefore unambiguous whenever $x$ is an element of $Y^{\alpha(f)}$.
The best example of universal sub-algebras are possibly homomorphic
images of universal algebras. We have:

\begin{prop}\label{logic:prop:UA:subalg:homomorphic:image}
Let $h:X\to Y$ be a homomorphism between two universal algebras $X$
and $Y$ of type $\alpha$. Then the image $h(X)$ is a sub-algebra of
$Y$.
\end{prop}
\begin{proof}
Let $f\in\alpha$ and $y\in h(X)^{\alpha(f)}$. We need to show that
$f(y)\in h(X)$. However, for all $i\in\alpha(f)$ we have $y(i)\in
h(X)$. Hence there exists $x_{i}\in X$ such that $y(i)=h(x_{i})$.
Let $x\in X^{\alpha(f)}$ be defined by $x(i)=x_{i}$ for all
$i\in\alpha(f)$. Then we have $y(i)=h(x_{i})=h(x(i))=h(x)(i)$ for
all $i\in\alpha(f)$ and consequently $y=h(x)$. Since $h:X\to Y$ is a
homomorphism, it follows that $f(y)=f\circ h(x)=h\circ
f(x)=h(x^{*})$ with $x^{*}=f(x)$. So we have found $x^{*}\in X$ such
that $f(y)=h(x^{*})$ and we conclude that $f(y)\in h(X)$.
\end{proof}

The restrictions of homomorphisms to universal sub-algebras are
themselves homomorphisms. We had to say this once to avoid future logical flaws
in forthcoming arguments. For example, if $h:X\to Y$ is a morphism
and $X_{0}$ is a sub-algebra of $X$, we would like to use
proposition~(\ref{logic:prop:UA:subalg:homomorphic:image}) to argue
that $h(X_{0})$ is a sub-algebra of $Y$. This is indeed the case,
but we need to argue that $h_{|X_{0}}:X_{0}\to Y$ is a morphism.

\begin{prop}\label{logic:prop:UA:subalg:restriction:morphism}
Let $h:X\to Y$ be a morphism  between two universal algebras of type
$\alpha$. Given a sub-algebra $X_{0}\subseteq X$, the restriction
$h_{|X_{0}}$ is a morphism.
\end{prop}
\begin{proof}
Given $f\in\alpha$ and $x\in X_{0}^{\alpha(f)}$ we need to check the
equality $h\circ f(x)=f\circ h(x)$. This follows from the fact that
every operator $f:X_{0}^{\alpha(f)}\to X_{0}$ is the restriction of
$f:X^{\alpha(f)}\to X$ and $h:X_{0}\to Y$ is the restriction of
$h:X\to Y$.
\end{proof}


Let ${\cal A}$ be a non-empty set. Recall that $\cap{\cal A}$ is the
set defined by:
    \[
    \cap{\cal A}=\{x:\forall Z\in{\cal A}\ ,\ x\in Z\}
    \]
When ${\cal A}=\emptyset$, the set $\cap{\cal A}$ is normally not
defined, as there is no such things as {\em the set of all sets}.
Suppose now that ${\cal A}$ is a non-empty set of subsets of a given
set $X$. Since ${\cal A}$ is non-empty, it has an element $Z\in{\cal
A}$ which by assumption is a subset of $X$. So every element $x\in
Z$ is also an element of $X$ and it follows that $\cap{\cal A}$
could equally have been defined as:
    \[
    \cap{\cal A}=\{x\in X: \forall Z\in{\cal A}\ ,\ x\in Z\}
    \]
This last expression is now meaningful when ${\cal A}=\emptyset$,
and yields $\cap\emptyset = X$. Hence we shall adopt the convention
that $\cap{\cal A}=X$ whenever ${\cal A}=\emptyset$ is viewed as a
set of subsets of $X$, and $X$ is clearly given from the context.
\begin{prop}\label{logic:prop:intersection}
Let $X$ be a universal algebra of type $\alpha$ and ${\cal A}$ be a
set of universal sub-algebras of $X$. Then $\cap{\cal A}$ is a
universal sub-algebra of $X$.
\end{prop}
\begin{proof}
Let $Y=\cap{\cal A}$. We need to show that $Y$ is a universal
sub-algebra of $X$. So let $f\in\alpha$ and $x\in Y^{\alpha(f)}$. We
need to show that $f(x)\in Y$. So let $Z\in{\cal A}$. We need to
show that $f(x)\in Z$. But $Y\subseteq Z$ and $x\in Y^{\alpha(f)}$.
So $x\in Z^{\alpha(f)}$. Since $Z$ is a universal sub-algebra of
$X$, we conclude that $f(x)\in Z$.
\end{proof}
\index{x@$\langle X_{0}\rangle\,$: sub-algebra generated by $X_{0}$}
\begin{defin}\label{logic:def:generated:sub:algebra}
Let $X$ be a universal algebra of type $\alpha$ and $X_{0}\subseteq
X$. Let ${\cal A}$ be the set of universal sub-algebras of $X$
defined by:
    \[
    {\cal A}=\{Y: X_{0}\subseteq Y\mbox{\ and\ } Y\mbox{\ universal sub-algebra of\ }X\}
    \]
We call {\em Universal Sub-Algebra of $X$ generated by $X_{0}$} the
universal sub-algebra of $X$ denoted $\langle X_{0}\rangle$ and
defined by $\langle X_{0}\rangle=\cap{\cal A}$.
\end{defin}
\begin{prop}\label{logic:prop:generated:smallest}
Let $X$ be a universal algebra of type $\alpha$ and $X_{0}\subseteq
X$. Then, the universal sub-algebra $\langle X_{0}\rangle$ generated
by $X_{0}$ is the smallest universal sub-algebra of $X$ {\em
containing} $X_{0}$, i.e. such that $X_{0}\subseteq \langle
X_{0}\rangle$.
\end{prop}
\begin{proof}
Define ${\cal A}=\{Y: X_{0}\subseteq Y\mbox{\ and\ } Y\mbox{\
universal sub-algebra of\ }X\}$. From
definition~(\ref{logic:def:generated:sub:algebra}), we have $\langle
X_{0}\rangle = \cap{\cal A}$. We need to show that $\langle
X_{0}\rangle$ is a universal sub-algebra of $X$ which contains
$X_{0}$, and that it is the smallest universal sub-algebra of $X$
with such property. Since every element of ${\cal A}$ is a universal
sub-algebra of $X$, from
proposition~(\ref{logic:prop:intersection}), $\langle X_{0}\rangle
=\cap{\cal A}$ is also a universal sub-algebra of $X$. To show that
$X_{0}\subseteq \langle X_{0}\rangle$ assume that $x\in X_{0}$. We
need to show that $x\in\cap{\cal A}$. So let $Y\in{\cal A}$. We need
to show that $x\in Y$. But this is clear since $X_{0}\subseteq Y$.
So we have proved that $\langle X_{0}\rangle$ is a universal
sub-algebra of $X$ containing $X_{0}$. Suppose that $Y$ is another
universal sub-algebra of $X$ such that $X_{0}\subseteq Y$. We need
to show that $\langle X_{0}\rangle$ is {\em smaller} than $Y$, that
is $\langle X_{0}\rangle\subseteq Y$. But $Y$ is clearly an element
of ${\cal A}$. Hence if $x\in\langle X_{0}\rangle =\cap{\cal A}$, it
follows immediately that $x\in Y$. This shows that $\langle
X_{0}\rangle \subseteq Y$.
\end{proof}

Suppose $X$ and $Y$ are universal algebras of type $\alpha$ and
$g:X\to Y$ is a morphism. Given $X_{0}\subseteq X$, the following
proposition asserts that the direct image $g(\langle X_{0}\rangle)$
of the universal sub-algebra of $X$ generated by $X_{0}$ is also the
universal sub-algebra $\langle g(X_{0})\rangle$ of $Y$ generated by
the direct image $g(X_{0})$.

\begin{prop}\label{logic:prop:UA:subalg:generated:direct:image}
Let $X,Y$ be universal algebras of type $\alpha$ and $g:X\to Y$ be a
morphism. For every subset $X_{0}\subseteq X$ we have the following
equality:
    \[
    g(\langle X_{0}\rangle)=\langle g(X_{0})\rangle
    \]
\end{prop}
\begin{proof}
First we show $\supseteq$\,: by virtue of
proposition~(\ref{logic:prop:generated:smallest}), $\langle
g(X_{0})\rangle$ is the smallest sub-algebra of $Y$ containing the
direct image $g(X_{0})$. In order to prove $\supseteq$\,, it is
therefore sufficient to show that $g(\langle X_{0}\rangle)$ is a
sub-algebra of $Y$ with $g(\langle X_{0}\rangle)\supseteq g(X_{0})$.
This last inclusion follows immediately from $\langle
X_{0}\rangle\supseteq X_{0}$. Being a homomorphic image of the
sub-algebra $\langle X_{0}\rangle$, the fact that $g(\langle
X_{0}\rangle)$ is a sub-algebra of $Y$ follows from
proposition~(\ref{logic:prop:UA:subalg:homomorphic:image}). So we
now prove $\subseteq$\,. Consider the subset $X_{1}$ defined by:
    \[
    X_{1}=\{\,x\in \langle X_{0}\rangle\ :\ g(x)\in\langle
    g(X_{0})\rangle\,\}
    \]
In order to prove $\subseteq$\,, it is sufficient to prove that
$\langle X_{0}\rangle\subseteq X_{1}$. Thus we need to show that
$X_{1}$ is a sub-algebra of $X$ with $X_{0}\subseteq X_{1}$. The
fact that $X_{0}\subseteq X_{1}$ follows from the inclusions
$X_{0}\subseteq\langle X_{0}\rangle$ and $g(X_{0})\subseteq\langle
g(X_{0})\rangle$. So it remains to show that $X_{1}$ is a
sub-algebra of $X$. Consider $f\in\alpha$ and $x\in
X_{1}^{\alpha(f)}$. We need to show that $f(x)\in X_{1}$. However,
from $X_{1}\subseteq \langle X_{0}\rangle$ we obtain $x\in\langle
X_{0}\rangle^{\alpha(f)}$ and it follows that $f(x)\in\langle
X_{0}\rangle$ since $\langle X_{0}\rangle$ is a sub-algebra of $X$.
In order to show that $f(x)\in X_{1}$, it remains to prove that
$g\circ f(x)\in\langle g(X_{0})\rangle$. Having assumed that $g:X\to
Y$ is a morphism, this amounts to showing that $f\circ
g(x)\in\langle g(X_{0})\rangle$. However, since $\langle
g(X_{0})\rangle$ is a sub-algebra of $Y$, in order to show that
$f\circ g(x)\in\langle g(X_{0})\rangle$ we simply need to prove that
$g(x)\in\langle g(X_{0})\rangle^{\alpha(f)}$. Thus given
$i\in\alpha(f)$, we need to show that $g(x)(i)=g(x(i))\in\langle
g(X_{0})\rangle$. It is therefore sufficient to prove that $x(i)\in
X_{1}$ which follows from $x\in X_{1}^{\alpha(f)}$.
\end{proof}

Let $X$ be a universal algebra of type $\alpha$ and $X_{0}\subseteq
X$. The universal sub-algebra $\langle X_{0}\rangle$ generated by
$X_{0}$ was defined in terms of an intersection $\cap{\cal A}$. This
may be viewed as a {\em definition from above}: we start from
universal sub-algebras which are larger than $\langle X_{0}\rangle$,
including $X$ itself, and we {\em reduce} these universal
sub-algebras by taking the intersection between them, until we
arrive at $\langle X_{0}\rangle$. The following proposition may be
viewed as a {\em definition from below}. We start from the smallest
possible set, which is the generator $X_{0}$, and we gradually {\em
add} all the elements which should belong to $\langle X_{0}\rangle$
in view of its closure properties. Note that the {\em definition
from below} makes use of a recursion principle, whereas the {\em
definition from above} does not.
\begin{prop}\label{logic:prop:characterisation:generated}
Let $X$ be a universal algebra of type $\alpha$ and $X_{0}\subseteq
X$. Define:
    \[
    X_{n+1}=X_{n}\cup\bar{X}_{n}\ ,\ n\in\N
    \]
with:
    \[
    \bar{X}_{n}=\left\{f(x):\ f\in\alpha\ ,\ x\in X_{n}^{\alpha(f)}\right\}
    \]
Then:
    \[
    \langle X_{0}\rangle=\bigcup_{n=0}^{+\infty}X_{n}
    \]
\end{prop}
\begin{proof}
Let $Y=\cup_{n\in\N}X_{n}$. We need to show that $Y=\langle
X_{0}\rangle$. First we show that $Y\subseteq \langle X_{0}\rangle$.
It is sufficient to prove by induction that $X_{n}\subseteq\langle
X_{0}\rangle$ for all $n\in\N$. Since $X_{0}\subseteq \langle
X_{0}\rangle$, this is obviously true for $n=0$. Suppose we have
$X_{n}\subseteq \langle X_{0}\rangle$. We claim that
$X_{n+1}\subseteq\langle X_{0}\rangle$. It is sufficient to prove
that $\bar{X}_{n}\subseteq\langle X_{0}\rangle$. So let $f\in\alpha$
and $x\in X_{n}^{\alpha(f)}$. We have to show that $f(x)\in\langle
X_{0}\rangle$. But since $X_{n}\subseteq\langle X_{0}\rangle$, we
have $X_{n}^{\alpha(f)}\subseteq\langle X_{0}\rangle^{\alpha(f)}$.
It follows that $x\in\langle X_{0}\rangle^{\alpha(f)}$. But $\langle
X_{0}\rangle$ is a universal sub-algebra of $X$, and we conclude
that $f(x)\in\langle X_{0}\rangle$. This completes our induction
argument, and we have proved that $Y\subseteq \langle X_{0}\rangle$.
We now show that $\langle X_{0}\rangle\subseteq Y$. Since
$X_{0}\subseteq Y$, from
proposition~(\ref{logic:prop:generated:smallest}), $\langle
X_{0}\rangle$ being the smallest universal sub-algebra containing
$X_{0}$, it is sufficient to show that $Y$ is also a universal
sub-algebra of $X$. So let $f\in\alpha$ and $x\in Y^{\alpha(f)}$. We
need to show that $f(x)\in Y$.  It is sufficient to show that $x\in
X_{n}^{\alpha(f)}$ for some $n\in\N$, as this will imply
$f(x)\in\bar{X}_{n}\subseteq X_{n+1}\subseteq Y$. If $\alpha(f)=0$
then $X_{n}^{\alpha(f)}=1=Y^{\alpha(f)}$ for all $n\in\N$ and $x\in
X_{n}^{\alpha(f)}$ for all $n\in\N$. Suppose $\alpha(f)\geq 1$.
Given $i\in\alpha(f)$, since $x\in Y^{\alpha(f)}$ we have $x(i)\in
Y$. So there exists $n_{i}\in\N$ such that $x(i)\in X_{n_{i}}$.
Define $n\in\N$ by:
     \[
     n=\max(n_{0},\ldots,n_{\alpha(f)-1})
     \]
Since $X_{n_{i}}\subseteq X_{n}$, we have $x(i)\in X_{n}$ for all $i\in\alpha(f)$.
So $x\in X_{n}^{\alpha(f)}$.
\end{proof}
