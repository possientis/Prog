The objective of this section is to show that $\preceq$ is a partial
order.

\begin{prop}\label{logic:prop:subformula:reflexive}
Let $X$ be a free universal algebra of type $\alpha$ with free
generator $X_{0}\subseteq X$. The relation $\preceq$ on $X$ is
reflexive.
\end{prop}
\begin{proof}
We need to show that $x\in\subf(x)$ for all $x\in X$. We shall prove
this result by structural induction using
theorem~(\ref{logic:the:proof:induction}) of
page~\pageref{logic:the:proof:induction}. If $x\in X_{0}$ then
$\subf(x)=\{x\}$ and the property $x\in\subf(x)$ is clear. Suppose
$f\in\alpha$ and $x\in X^{\alpha(f)}$ are such that
$x(i)\in\subf(x(i))$ for all $i\in\alpha(f)$. We need to show the
property is also true for~$f(x)$, i.e that $f(x)\in\subf(f(x))$.
This follows immediately from:
    \[
    \subf(f(x)) = \{f(x)\}\cup\bigcup_{i\in\alpha(f)}\subf(x(i))
    \]
\end{proof}
\begin{prop}\label{logic:prop:subformula:transitive}
Let $X$ be a free universal algebra of type $\alpha$ with free
generator $X_{0}\subseteq X$. For all $x,y\in X$ we have the
equivalence:
    \[
    x\preceq y\ \Leftrightarrow\ \subf(x)\subseteq\subf(y)
    \]
In particular, the relation $\preceq$ on $X$ is transitive.
\end{prop}
\begin{proof}
First we show the implication $\Leftarrow$. Let $x,y\in X$ such that
$\subf(x)\subseteq\subf(y)$. From
proposition~(\ref{logic:prop:subformula:reflexive}) we have
$x\in\subf(x)$ and consequently $x\in\subf(y)$ which shows that
$x\preceq y$. We now show the reverse implication $\Rightarrow$. We
shall do so by considering the property $P(y)$ defined for $y\in X$
as:
    \[
    \forall x\in X\ ,\ [\ x\in\subf(y)\ \Rightarrow\
    \subf(x)\subseteq\subf(y)\ ]
    \]
We shall prove $P(y)$ for all $y\in X$ by structural induction using
theorem~(\ref{logic:the:proof:induction}) of
page~\pageref{logic:the:proof:induction}. First we assume that $y\in
X_{0}$. Then $\subf(y) = \{y\}$ and the condition $x\in\subf(y)$
implies that $x=y$. In particular $x\in X_{0}$ and $\subf(x)=\{x\}$
which shows that the inclusion $\subf(x)\subseteq\subf(y)$ is indeed
true. So the property $P(y)$ holds for $y\in X_{0}$. Let
$f\in\alpha$ and $y\in X^{\alpha(f)}$ be such that $P(y(i))$ holds
for all $i\in\alpha(f)$. We need to show that the property $P(f(y))$
is also true. So let $x\in X$ be such that $x\in\subf(f(y))$. We
need to show that $\subf(x)\subseteq\subf(f(y))$. However, from
definition~(\ref{logic:def:subformula}) we have:
    \begin{equation}\label{logic:eq:subformula:1}
    \subf(f(y)) =
    \{f(y)\}\cup\bigcup_{i\in\alpha(f)}\subf(y(i))
    \end{equation}
Hence the condition $x\in\subf(f(y))$ implies that $x=f(y)$ or
$x\in\subf(y(i))$ for some $i\in\alpha(f)$. Suppose first that
$x=f(y)$. Then $\subf(x)\subseteq\subf(f(y))$ follows immediately.
Suppose now that $x\in\subf(y(i))$ for some $i\in\alpha(f)$. Having
assumed the property $P(y(i))$ is true, it follows that
$\subf(x)\subseteq\subf(y(i))$ and finally from
equation~(\ref{logic:eq:subformula:1}) we conclude that
$\subf(x)\subseteq\subf(f(y))$. This completes our induction
argument and the proof that $P(y)$ holds for all $y\in X$.
\end{proof}

Proving the relation $\preceq$ is reflexive and transitive worked
pretty well with structural induction arguments. The proof that
$\preceq$ is anti-symmetric will also rely on induction but will be
seen to be more difficult. Given $f\in\alpha$ and $x\in
X^{\alpha(f)}$ a key step in the argument will be to claim that
$f(x)$ cannot be a sub-formula of any $x(i)$ for all
$i\in\alpha(f)$. This seems pretty obvious but requires some care.
We shall achieve this by using the order mapping $\om:X\to\N$ as per
definition~(\ref{logic:def:order}) of
page~\pageref{logic:def:order}. Recall that given $x\in X$ the order
$\om(x)$ offers some measure of complexity for the formula $x$. The
next proposition shows that if $x$ is a sub-formula of $y$, then it
has no greater complexity than $y$. Since we already know from
proposition~(\ref{logic:prop:order}) that $f(x)$ has greater
complexity than every $x(i)$, we shall easily conclude that $f(x)$
cannot be a sub-formula of $x(i)$.

\begin{prop}\label{logic:prop:subformula:order}
Let $X$ be a universal algebra of type $\alpha$ with free generator
$X_{0}\subseteq X$. Then for all $x,y\in X$ we have:
    \[
    x\preceq y\ \Rightarrow\ \om(x)\leq\om(y)
    \]
where $\om:X\to\N$ is the order mapping of {\em
definition~(\ref{logic:def:order})}.
\end{prop}
\begin{proof}
Given $y\in X$ consider the property $P(y)$ defined by:
    \[
    \forall x\in X\ ,\ [\ x\in\subf(y)\ \Rightarrow\
    \om(x)\leq\om(y)\ ]
    \]
We shall complete the proof of this proposition by showing $P(y)$ is
true for all $y\in X$, which we shall do by a structural induction
argument using theorem~(\ref{logic:the:proof:induction}) of
page~\pageref{logic:the:proof:induction}. First we assume that $y\in
X_{0}$. Then $\subf(y)=\{y\}$ and the condition $x\in\subf(y)$
implies that $x=y$ and in particular $\om(x)\leq\om(y)$. So the
property $P(y)$ is true whenever $y\in X_{0}$. Next we assume that
$f\in\alpha$ and $y\in X^{\alpha(f)}$ are such that $P(y(i))$ is
true for all $i\in\alpha(f)$. We need to show the property $P(f(y))$
is also true. So let $x\in X$ be such that $x\in\subf(f(y))$. We
need to show that $\om(x)\leq\om(f(y))$. From
definition~(\ref{logic:def:subformula}) we have:
   \[
    \subf(f(y)) =
    \{f(y)\}\cup\bigcup_{i\in\alpha(f)}\subf(y(i))
    \]
Hence the condition $x\in\subf(f(y))$ implies that $x=f(y)$ or
$x\in\subf(y(i))$ for some $i\in\alpha(f)$. Suppose first that
$x=f(y)$. Then $\om(x)\leq\om(f(y))$ follows immediately. Suppose
now that $x\in\subf(y(i))$ for some $i\in\alpha(f)$. Having assumed
the property $P(y(i))$ is true, it follows that
$\om(x)\leq\om(y(i))$. Using proposition~(\ref{logic:def:order}) we
obtain:
    \[
    \om(x)\leq\om(y(i))<1 +
    \max\{\om(y(i)):i\in\alpha(f)\}=\om(f(y))
    \]
So we have proved that $\om(x)<\om(f(y))$ and in particular
$\om(x)\leq\om(f(y))$. This completes our induction argument and the
proof that $P(y)$ holds for all $y\in X$.
\end{proof}
\begin{prop}\label{logic:prop:subformula:antisymmetric}
Let $X$ be a free universal algebra of type $\alpha$ with free
generator $X_{0}\subseteq X$. The relation $\preceq$ on $X$ is
anti-symmetric.
\end{prop}
\begin{proof}
Given $x,y\in X$ we have to show that if $x\preceq y$ and $y\preceq
x$ then $x=y$. Given $y\in X$ consider the property $P(y)$ defined
by:
    \[
    \forall x\in X\ ,\ [ (x\preceq y)\land(y\preceq x)\ \Rightarrow\
    (x=y)\ ]
    \]
We shall complete the proof of this proposition by showing $P(y)$ is
true for all $y\in X$, which we shall do by a structural induction
argument using theorem~(\ref{logic:the:proof:induction}) of
page~\pageref{logic:the:proof:induction}. First we assume that $y\in
X_{0}$. We need to show that $P(y)$ is true. So let $x\in X$ be such
that $x\preceq y$ and $y\preceq x$. In particular we have $x\preceq
y$, i.e. $x\in\subf(y)$. From $y\in X_{0}$ we obtain
$\subf(y)=\{y\}$ and we conclude that $x=y$ as requested. So $P(y)$
is indeed true for all $y\in X_{0}$. Next we assume that
$f\in\alpha$ and $y\in X^{\alpha(f)}$ are such that $P(y(i))$ is
true for all $i\in\alpha(f)$. We need to show that $P(f(y))$ is also
true. So let $x\in X$ be such that $x\preceq f(y)$ and $f(y)\preceq
x$. We need to show that $x=f(y)$. Suppose to the contrary that
$x\neq f(y)$. We shall arrive at a contradiction. From $x\preceq
f(y)$ we obtain $x\in\subf(f(y))$ and from
definition~(\ref{logic:def:subformula}) we have:
    \begin{equation}\label{logic:eq:subformula:2}
    \subf(f(y)) =
    \{f(y)\}\cup\bigcup_{i\in\alpha(f)}\subf(y(i))
    \end{equation}
Having assumed that $x\neq f(y)$ it follows that $x\in\subf(y(i))$
for some $i\in\alpha(f)$. Hence we have proved that $x\preceq y(i)$
for some $i\in\alpha(f)$. However, it is clear from
equation~(\ref{logic:eq:subformula:2}) that
$\subf(y(i))\subseteq\subf(f(y))$ and it follows from
proposition~(\ref{logic:prop:subformula:transitive}) that
$y(i)\preceq f(y)$. From the assumption $f(y)\preceq x$, we obtain
$y(i)\preceq x$ by transitivity. Hence we see that $x\in X$ is an
element such that $x\preceq y(i)$ and $y(i)\preceq x$. From our
induction hypothesis, we know that $P(y(i))$ is true. From $x\preceq
y(i)$ and $y(i)\preceq x$ we therefore obtain $x=y(i)$. In
particular, using proposition~(\ref{logic:prop:order}) of
page~\pageref{logic:prop:order} we have:
    \[
    \om(x)=\om(y(i))<1 +
    \max\{\om(y(i)):i\in\alpha(f)\}=\om(f(y))
    \]
So we have proved that $\om(x)<\om(f(y))$. However, from
$f(y)\preceq x$ and proposition~(\ref{logic:prop:subformula:order})
we obtain $\om(f(y))\leq\om(x)$. This is our desired contradiction.
\end{proof}
\index{subformula@Sub-formula partial order}
\begin{prop}\label{logic:prop:sunformula:partial:order}
Let $X$ be a free universal algebra of type $\alpha$ with free
generator $X_{0}\subseteq X$. The relation $\preceq$ on $X$ is a
partial order.
\end{prop}
\begin{proof}
The relation $\preceq$ is reflexive, anti-symmetric and transitive
as can be seen from
propositions~(\ref{logic:prop:subformula:reflexive}),
(\ref{logic:prop:subformula:antisymmetric})
and~(\ref{logic:prop:subformula:transitive}) respectively. It is
therefore a partial order.
\end{proof}
