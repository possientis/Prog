Recall that a map $f$ is said to be {\em injective} or {\em
one-to-one} \ifand\ $x=x'$ whenever $(x,y)\in f$ and $(x',y)\in f$.
When this is the case, the set:
    \[
    f^{-1}=\{(y,x): (x,y)\in f\}
    \]
is also a map. If $f:A\to B$ then saying that $f$ is injective can be expressed as:
    \[
    \forall x,x'\in A\ ,\ f(x)=f(x')\ \Rightarrow\ x=x'
    \]
However the domain of $f^{-1}$ is the range of $f$ or $\rng(f)$,
which may not be the whole set $B$. We say that $f:A\to B$ is {\em
surjective} or {\em onto} \ifand\ $\rng(f)=B$, or equivalently:
    \[
    \forall y\in B\ ,\ \exists x\in A\ ,\ y=f(x)
    \]
Note that a map $f:A\to B$ is simply a set of ordered pairs with no
{\em knowledge} of the set $B$. In other words, being given the set
$f$ will not give you the set $B$, which can be any set with
$\rng(f)\subseteq B$. Saying that $f$ is surjective is therefore not
meaningful on its own, unless the set $B$ is clear from the context.

We say that $f:A\to B$ is {\em bijective}, or that it is a {\em
bijection} or {\em one-to-one correspondence} \ifand\ it is both
injective and surjective. When this is the case we have $f^{-1}:B\to
A$, and it is also a bijective map.
\index{morphism@Isomorpshism of universal algebra}
\begin{defin}\label{logic:def:isomorphism}
Let $X$ and $Y$ be universal algebras of type $\alpha$. We say that
a map $g:X\to Y$ is an {\em isomorphism} \ifand\ $g$ is a bijective
morphism. If there exists an isomorphism $g:X\to Y$, we say that $X$
and $Y$ are {\em isomorphic}.
\end{defin}
\begin{prop}
Let $X$ and $Y$ be universal algebras of type $\alpha$. If $g:X\to
Y$ is an isomorphism, then $g^{-1}:Y\to X$ is also an isomorphism.
\end{prop}
\begin{proof}
Suppose $g:X\to Y$ is a morphism which is bijective. Then
$g^{-1}:Y\to X$ is also bijective and we only need to check that it
is a morphism. Let $f\in\alpha$ and $y\in Y^{\alpha(f)}$. We need to
check that $g^{-1}\circ f(y) = f\circ g^{-1}(y)$, and since $g$ is
an injective map, this is equivalent to:
    \begin{equation}\label{logic:eqn:isomorphism}
    g\circ g^{-1}\circ f(y) =g\circ f\circ g^{-1}(y)
    \end{equation}
The l.h.s of~(\ref{logic:eqn:isomorphism}) is clearly $f(y)$ and
since $g:X\to Y$ is a morphism, the r.h.s is $f\circ g\circ
g^{-1}(y)$ . So it remains to check that $g\circ g^{-1}(y)=y$ for
all $y\in Y^{\alpha(f)}$. This follows immediately from:
    \[
    g\circ g^{-1}(y)(i) = g(g^{-1}(y)(i))=g(g^{-1}(y(i)))=y(i)
    \]
which is true for all $i\in\alpha(f)$.
\end{proof}
