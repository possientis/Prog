In this section we prove that any set $A$ has a disjoint copy of
itself, namely that there exists a set $B$ with $A\cap B=\emptyset$
and a bijection $j:A\to B$. We shall prove this result using
Cantor's Theorem (lemma~(\ref{logic:cantor:theorem})) as well as the
Axiom of Choice. There are probably many other ways to prove it,
some of which not involving the Axiom of Choice. It is in fact
unknown to me whether this can be proved within {\bf ZF} rather than
{\bf ZFC}.\footnote{For those interested, there is a thread on
\texttt{sci.math} entitled "[Set Theory] A set has a disjoint copy
of itself" (which may have the answer to the {\bf ZF} question). I
am grateful to those who contributed to this thread. The proof of
lemma~(\ref{logic:lemma:disjoint:copy}) is due to Robin Chapman.}


Given a set $A$, recall that the Axiom of Choice states the
existence of a {\em choice function} on $A$, i.e. the existence of a
map $k:{\cal P}(A)\setminus\{\emptyset\}\to A$ such that $k(x)\in x$
for all $x\in{\cal P}(A)\setminus\{\emptyset\}$. A choice function
on $A$ is simply a map which {\em chooses} an arbitrary element from
every non-empty subset of $A$.

\begin{lemma}\label{logic:lemma:disjoint:copy}
Let $A$ be an arbitrary set. There exists a set $B$ with $A\cap
B=\emptyset$ and a bijective map $j:A\to B$.
\end{lemma}
\begin{proof}
For all $x\in {\cal P}(A)$ define $A_{x}=\{(x,y): y\in A\}$. There
is a clear bijection between $A$ and $A_{x}$, namely $j_{x}:A\to
A_{x}$ defined by $j_{x}(y)=(x,y)$. We shall complete the proof by
taking $B=A_{x}$ for some $x\in{\cal P}(A)$ for which $A\cap
A_{x}=\emptyset$. Of course we need to check that it is always
possible to do so. Suppose to the contrary that $A\cap
A_{x}\not=\emptyset$ for all $x\in{\cal P}(A)$. Using the Axiom of
Choice, there exists a choice function $k:{\cal
P}(A)\setminus\{\emptyset\}\to A$. We define $j:{\cal P}(A)\to A$ by
setting $j(x)=k(A\cap A_{x})$. Note that $j(x)$ is well defined
since $A\cap A_{x}\neq\emptyset$. We obtain a map $j:{\cal P}(A)\to
A$ such that $j(x)\in A\cap A_{x}$ for all $x\in {\cal P}(A)$. It is
now sufficient to show that $j$ is injective, as this will
contradict Cantor's Theorem. So suppose $x,x'\in{\cal P}(A)$ are
such that $j(x)=j(x')$. Since $j(x)\in A_{x}$ there exists $y\in A$
such that $j(x)=(x,y)$. Similarly, there exists $y'\in A$ such that
$j(x')=(x',y')$. From $j(x)=j(x')$ we obtain $(x,y)=(x',y')$ and it
follows in particular that $x=x'$. So we have proved that $j:{\cal
P}(A)\to A$ is indeed an injective map.
\end{proof}


Given a set $X_{0}$, our goal is to obtain a free universal algebra
of type $\alpha$ with free generator the set $X_{0}$ itself.
Defining $Y_{0}=\{(0,x):x\in X_{0}\}$, we were able to construct a
free universal algebra $Y$ with free generator $Y_{0}$, as described
in proposition~(\ref{logic:prop:construction}). So we have an
injection $j:X_{0}\to Y$, where $Y$ is a free universal algebra of
type $\alpha$. In fact, this injection is a bijection between
$X_{0}$ and $Y_{0}$, the free generator of $Y$. In order for us to
construct a free universal algebra with free generator $X_{0}$, all
we need is a copy of $Y$ containing $X_{0}$. In other words we need
a set $X$ with $X_{0}\subseteq X$ and a bijection $g:X\to Y$. Once
we have this set $X$ and the bijection $g:X\to Y$, we can easily
{\em carry over} the structure of universal algebra from $Y$ onto
$X$, by choosing the structure on $X$ which turns the bijection
$g:X\to Y$ into an isomorphism. However, we want $X_{0}$ to be a
free generator of $X$. One way to ensure this is to request that the
bijection $g:X\to Y$ carries $X_{0}$ onto $Y_{0}$. In other words,
we want the restriction $g_{|X_{0}}$ to coincide with the bijection
$j:X_{0}\to Y_{0}$. The following lemma shows the existence of the
set $X$ with the bijection $g:X\to Y$ satisfying the required
property $g_{|X_{0}}=j$.


Note that the idea of starting from an injection $X_{0}\to Y$ and
obtaining a copy of $Y$ containing $X_{0}$ while {\em preserving}
this injection can be used in many other cases. Most of us are
familiar with the way the ring of integers $\Z$ is created as a
quotient set $(\N\times\N)/\sim$ with the appropriate equivalence
relation. We obtain an embedding $j_{1}:\N\to\Z$. We then construct
the field $\Q$ of rational numbers and we obtain another embedding
$j_{2}:\Z\to\Q$. We go on by constructing the field $\R$ of real
numbers and the field $\C$ of complex numbers with their associated
embedding $j_{3}:\Q\to\R$ and $j_{4}:\R\to\C$. Strictly speaking,
the set theoretic inclusions
$\N\subseteq\Z\subseteq\Q\subseteq\R\subseteq\C$ are false and most
living mathematicians do not care. The inclusion $\N\subseteq\Z$ has
to be true. If it is not, it needs to be re-interpreted in a way
which makes it true. In computing terms, mathematicians are
effectively {\em overloading} the operators $\in$ and $\subseteq$.
This is all very nice and fine in practice. But assuming we wanted
to {\em code} or {\em compile} a {\em high level} mathematical
statement into a {\em low level} expression of first order predicate
logic, it may be that defining $\Z$, $\Q$, $\R$ and $\C$ in a way
which avoids the overloading of primitive symbols, will make our
life a lot easier. We believe the aim of creating a formal high
level mathematical language whose statements could be compiled as
formulas of first order logic is a worthy objective. Such high level
language would most likely need to allow the overloading of symbols.
So it may be that whether $\N\subseteq\Z$ is literally true or not
will not matter. There is a fascinating website~\cite{Metamath} on
\texttt{http://www.metamath.org/} created by Norman Megill where
these questions are likely to have been answered. It is certainly
our intention to use this reference more and more as this document
progresses.

Recall that if $f:A\to B$ is a map and $A'\subseteq A$, then $f(A')$ is the set:
    \[
    f(A')=\{y:\exists x\ ,\ x\in A'\mbox{\ and\ }\ (x,y)\in f\}
    \]
which coincides with $\rng(f_{|A'})$. In particular, $f(A)=\rng(f)$. The notation $f(A')$ is a case of overloading. The meaning of $f(x)$ must be derived from the context, and depends on whether $x$ is viewed as an element of $A$ or a subset of $A$. It can easily be both, as $0$ is both an element and a subset of $1$. Some authors will use the notation $f[A']$ rather than $f(A')$.

\begin{lemma}\label{logic:lemma:pullback}
Let $X_{0}$ and $Y$ be sets and $j:X_{0}\to Y$ be an injective map. There exist a set $X$ and a bijection $g:X\to Y$ such that $X_{0}\subseteq X$ and $g_{|X_{0}}=j$.
\end{lemma}
\begin{proof}
Consider the set $A=X_{0}\cup Y$. Using lemma~(\ref{logic:lemma:disjoint:copy}) there exists a set $B$ with $A\cap B=\emptyset$ and a bijective map $i:A\to B$.
Define:
    \[
    X=X_{0}\cup i(Y\setminus j(X_{0}))
    \]
and $g:X\to Y$ by setting $g(x)=j(x)$ for all $x\in X_{0}$ and $g(x)=i^{-1}(x)$ for all $x\in i(Y\setminus j(X_{0}))$. Note first that $g(x)$ is well defined  since $x\in X_{0}$ and $x\in i(Y\setminus j(X_{0}))$ cannot occur at the same time, as $X_{0}\subseteq A$, $i(Y\setminus j(X_{0}))\subseteq B$ and $A\cap B=\emptyset$. Note also that $g(x)$ is defined for all $x\in X$. To show that we have $g:X\to Y$ we need to check that $g(x)\in Y$ for all $x\in X$. If $x\in X_{0}$ then this is clear since $j:X_{0}\to Y$ and $g(x)=j(x)$. If $x\in i(Y\setminus j(X_{0}))$, then $x$ can be written as $x=i(y)$ for some $y\in Y\setminus j(X_{0})$. It follows that $g(x)=i^{-1}(x)=y\in Y$. So we have proved that $g:X\to Y$. We obviously have $X_{0}\subseteq X$ and $g_{|X_{0}}=j$. So we shall complete the proof by showing that $g:X\to Y$ is a bijection. First we show that it is surjective. So let $y\in Y$. We need to show that $y=g(x)$ for some $x\in X$. If $y\in j(X_{0})$ then $y=j(x)$ for some $x\in X_{0}$ and in particular $y=g(x)$. Otherwise $y\in Y\setminus j(X_{0})$ and taking $x=i(y)\in i(Y\setminus j(X_{0}))$ we see that $y=i^{-1}(x)$ and in particular $y=g(x)$. So we have proved that $g$ is surjective. To show that it is injective, let $x,x'\in X$ and assume that $g(x)=g(x')$. We need to show that $x=x'$. Note first that we cannot have $x\in X_{0}$ while $x'\in i(Y\setminus j(X_{0}))$. If this was the case, we would have $g(x)=j(x)\in j(X_{0})$ while $g(x')=i^{-1}(x')\in Y\setminus j(X_{0})$ contradicting the fact that $g(x)=g(x')$ as one lies in $j(X_{0})$ while the other does not. Similarly, we cannot have $x\in i(Y\setminus j(X_{0}))$ while $x'\in X_{0}$. It follows that either both $x$ and $x'$ lie in $X_{0}$ or they both lie in $i(Y\setminus j(X_{0}))$. In the first case we obtain $j(x)=g(x)=g(x')=j(x')$ and since $j:X_{0}\to Y$ is injective we see that $x=x'$. In the second case, we obtain $i^{-1}(x)=g(x)=g(x')=i^{-1}(x')$ and since $i^{-1}:B\to A$ is injective we see that $x=x'$. So we have proved that $g$ is injective, and finally $g:X\to Y$ is a bijection.
\end{proof}
