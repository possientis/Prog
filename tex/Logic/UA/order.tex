In this section, we define the notion of {\em order} on a free
universal algebra. This notion will prove useful on several
occasions, and specifically when proving in
proposition~(\ref{logic:prop:free:generator:generator}) that a free
generator is also a {\em generator}. Loosely speaking, given a
universal algebra $X$ of type $\alpha$, the {\em order on $X$} is a
map $\om:X\to\N$ representing the degree of complexity of the
elements of $X$, viewed as expressions in a formal language. For
instance suppose $X$ is of type $\alpha=\{(0,2)\}$ with one single
binary operator $\oplus:X^{2}\to X$ and free generator
$X_{0}=\{0\}$. We have $\om(0)=0$ while $\om(0\oplus 0)=1$ and
$\om(0\oplus(0\oplus 0))=2$. The complexity of a formula is
sometimes a useful tool to carry out induction arguments over $\N$.
For those already familiar with the subject, we do not define the
order $\om:X\to\N$ by structural recursion, e.g. $\om(0)=0$ and
$\om(x\oplus y)=1+\max(\om(x),\om(y))$. Instead, we shall embed $\N$
with the appropriate structure of universal algebra of type
$\alpha$, and consider $\om:X\to\N$ as the unique morphism such that
$\om_{|X_{0}}=0$. In effect, we are limiting ourselves to
definitions by recursions over $\N$, which is a standard
mathematical argument already used in
proposition~(\ref{logic:prop:construction}). We shall not make use
of definitions by structural recursion on free universal algebras,
until the principle has been justified in a later section.
\begin{defin}\label{logic:def:default:structure}
Let $\alpha$ be a type of universal algebra. We call {\em default
structure of type $\alpha$} on $\N$ the map $T$ with domain $\alpha$
such that $T(f):\N^{\alpha(f)}\to\N$ is:
    \[
    \forall n\in\N^{\alpha(f)}\ ,\ T(f)(n)=1+\max\{n(i):i\in\alpha(f)\}
    \]
for all $f\in\alpha$, where it is understood that $\max(\emptyset)=0$.
\end{defin}
\index{order@Order in free universal
algebra}\index{omega@$\om(x)\,$: the order of $x$}
\begin{defin}\label{logic:def:order}
Let $X$ be a free universal algebra of type $\alpha$ with free
generator $X_{0}\subseteq X$. We call {\em order} on $X$ the unique
morphism $\om:X\to\N$ such that $\om_{|X_{0}}=0$, where $\N$ is
embedded with its default structure of type $\alpha$. Given $x\in
X$, we call {\em order of $x$} the integer $\om(x)\in\N$ where $\om$
is the order on $X$.
\end{defin}
\begin{prop}\label{logic:prop:order}
Let $X$ be a free universal algebra of type $\alpha$ with free
generator $X_{0}\subseteq X$ and $\om:X\to\N$ be the order on $X$.
Then for all $x\in X$ we have:
    \[
    \om(x)=0\ \Leftrightarrow\ x\in X_{0}
    \]
Furthermore, for all $f\in\alpha$ and $x\in X^{\alpha(f)}$ we have:
    \[
    \om(f(x)) = 1 + \max\{\om(x(i)):i\in\alpha(f)\}
    \]
\end{prop}
\begin{proof}
First we show the equality. Suppose $f\in\alpha$ and $x\in
X^{\alpha(f)}$. Let $T$ denote the default structure of type
$\alpha$ on $\N$ as per
definition~(\ref{logic:def:default:structure}). Since $\om:X\to\N$
is a morphism, we have:
    \begin{eqnarray*}
    \om(f(x))&=&T(f)(\om(x))\\
    &=&1+\max\{\om(x)(i):i\in\alpha(f)\}\\
    &=&1+\max\{\om(x(i)):i\in\alpha(f)\}
    \end{eqnarray*}
We now show the equivalence. Since $\om_{|X_{0}}=0$ from
definition~(\ref{logic:def:order}), it is clear that $y\in X_{0}$
implies $\om(y)=0$. Suppose conversely that $y\in X$ and $\om(y)=0$.
We need to show that $y\in X_{0}$. Suppose to the contrary that
$y\not\in X_{0}$. From
theorem~(\ref{logic:the:unique:representation}) of
page~\pageref{logic:the:unique:representation} there exist
$f\in\alpha$ and $x\in X^{\alpha(f)}$ such that $y=f(x)$. So:
    \[
    \om(y)=\om(f(x))=1+\max\{\om(x(i)):i\in\alpha(f)\}\geq 1
    \]
which contradicts the initial assumption of $\om(y)=0$.
\end{proof}

Suppose $X=\N$ is the free universal algebra of type
$\alpha=\{(0,1)\}$ with the single successor operator $s:X^{1}\to X$
and free generator $X_{0}=\{0\}$. Then $\om(0)=0$ while
$\om(1)=\om(s(0))=1+\om(0)=1$. If we now regard $X=\N$ as the free
universal algebra of type $\alpha=\{(0,1),(1,0)\}$ by adding the
constant $0$, then $X$ now has $X_{0}=\emptyset$ as free generator.
Furthermore:
    \[
    \om(0)=\om(T(1,0)(0))=1+max(\emptyset)=1
    \]
More generally, the order of constants in a free universal algebra
is $1$ and not $0$. Only the elements of the free generator have
order $0$.

We shall now complete this section with a small proposition on the
structure of a free universal algebra in relation to its order
mapping. Note that the definition of $X_{0}$ below is consistent
with the existing notation $X_{0}$ referring to the free generator
of $X$, since we have $\om(x)=0\ \Leftrightarrow\ x\in X_{0}$.

\begin{prop}\label{logic:prop:order:structure}
Let $X$ be a free universal algebra of type $\alpha$ with free
generator $X_{0}\subseteq X$. Let $\om:X\to\N$ be the order on $X$
and define:
    \[
    X_{n}=\{x\in X\ : \ \om(x)\leq n\}\ ,\ n\in\N
    \]
Then for all $n\in\N$ we have:
    \[
    X_{n+1}=X_{n}\cup\{f(x)\ :\ f\in\alpha\ ,\
    x\in(X_{n})^{\alpha(f)}\}
    \]
\end{prop}
\begin{proof}
First we show the inclusion $\supseteq$. It is clear that
$X_{n}\subseteq X_{n+1}$. So let $f\in\alpha$ and
$x\in(X_{n})^{\alpha(f)}$. We need to show that $f(x)\in X_{n+1}$
that is $\om(f(x))\leq n+1$ which follows from
proposition~(\ref{logic:prop:order}) and:
    \[
    \om(f(x))=1 + \max\{\om(x(i)):i\in\alpha(f)\}\leq 1 + n
    \]
where the inequality stems from $\om(x(i))\leq n$ for all
$i\in\alpha(f)$. We now prove the inclusion $\subseteq$. So suppose
$y\in X_{n+1}\setminus X_{n}$. Then we must have $\om(y)=n+1$ and in
particular $\om(y)\neq 0$. From proposition~(\ref{logic:prop:order})
it follows that $y\not\in X_{0}$ and consequently using
theorem~(\ref{logic:the:unique:representation}) of
page~\pageref{logic:the:unique:representation} we see that $y$ can
be uniquely represented as $y=f(x)$ for some $f\in\alpha$ and $x\in
X^{\alpha(f)}$. It remains to show that $x$ is actually an element
of $(X_{n})^{\alpha(f)}$, i.e. that $x(i)\in X_{n}$ for all
$i\in\alpha(f)$ which is vacuously true in the case when
$\alpha(f)=0$. So let $i\in\alpha(f)$. We have:
    \[
    1 +\om(x(i))\leq 1 +
    \max\{\om(x(i)):i\in\alpha(f)\}=\om(f(x))=\om(y)=n+1
    \]
from which we conclude that $\om(x(i))\leq n$ as requested.
\end{proof}
