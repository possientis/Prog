Let $\alpha=\{(0,1),(1,2)\}$ and $X_{0}$ be a set. We know from
theorem~(\ref{logic:the:main:existence}) of
page~\pageref{logic:the:main:existence} that there exists a free
universal algebra $X$ of type $\alpha$ whose free generator is
$X_{0}$. Let us denote $\lnot:X^{1}\to X$ and $\land:X^{2}\to X$ the
two operators on $X$. This particular choice of notations should
invite us to view $X$ as a free universal algebra of {\em
propositional logic}. We know from
theorem~(\ref{logic:the:unique:representation}) of
page~\pageref{logic:the:unique:representation} that any {\em
proposition} $\phi$ of $X$  is either an {\em atomic proposition}
$\phi=p$ for some $p\in X_{0}$, or the {\em negation} of a
proposition $\phi=\lnot \phi_{1}$ or the {\em conjunction} of two
propositions $\phi=\phi_{1}\land\phi_{2}$. These three cases are
exclusive, and these representations are unique. In effect, the free
universal algebra $X$ is partitioned into three parts $X=X_{0}\uplus
X_{\lnot}\uplus X_{\land}$. If $A$ is a set, then any map $g:X\to A$
could be defined by specifying the three restrictions $g_{|X_{0}}$,
$g_{|X_{\lnot}}$ and $g_{|X_{\land}}$ separately. We saw from
proposition~(\ref{logic:prop:order}) that the order $\om:X\to\N$ had
this sort of three-fold structure:
    \begin{equation}\label{logic:eqn:order:propositional}
    \forall\phi\in X\ ,\ \om(\phi)=\left\{
                    \begin{array}{lcl}
                    0&\mbox{\ if\ }&\phi=p\in X_{0}\\
                    1+\om(\phi_{1})&\mbox{\ if\ }&\phi=\lnot\phi_{1}\\
                    1+\max(\om(\phi_{1}),\om(\phi_{2}))&\mbox{\ if\ }&\phi=\phi_{1}\land\phi_{2}
                    \end{array}\right.
    \end{equation}
However, we did not invoke
equation~(\ref{logic:eqn:order:propositional}) as a definition of
the order $\om:X\to\N$ as this would not guarantee the existence of
$\om$. This is very similar to the simple recursion problem:
defining a map $g:\N\to\N$ by setting $g(0)=1$ and
$g(n+1)=(n+1).g(n)$ does not in itself guarantee the existence of
$g$. We now know that $g$ exists, but we needed to justify the
principle of {\em definition by recursion over $\N$} in the form of
lemma~(\ref{logic:lemma:recursion:over:N}) and
meta-theorem~(\ref{logic:meta:recursion:over:N}) of
page~\pageref{logic:meta:recursion:over:N}.
Equation~(\ref{logic:eqn:order:propositional}) is clearly recursive
as it links $\om(\phi)$ on the left-hand-side to other values
$\om(\phi_{1})$ and $\om(\phi_{2})$ assigned by the map \om\ to
other points. It is not obvious that $\om$ exists so we had to find
some other way to define it. And yet we may feel there is nothing
wrong with equation~(\ref{logic:eqn:order:propositional}). From a
computing point of view, it seems pretty obvious that any
computation of $\om(\phi)$ would terminate. For instance taking
$\phi=\lnot(p_{1}\land((\lnot p_{2})\land p_{3}))$,
equation~(\ref{logic:eqn:order:propositional}) would lead to:
    \begin{eqnarray*}
        \om(\phi)&=&\om(\lnot(p_{1}\land((\lnot p_{2})\land p_{3})))\\
        &=&1+\om(p_{1}\land((\lnot p_{2})\land p_{3}))\\
        &=&2+\max(\om(p_{1}),\om((\lnot p_{2})\land p_{3}))\\
        &=&2+\om((\lnot p_{2})\land p_{3})\\
        &=&3+\max(\om(\lnot p_{2}),\om(p_{3}))\\
        &=&3+\om(\lnot p_{2})\\
        &=&4+\om(p_{2})\\
        &=&4
    \end{eqnarray*}
Equation~(\ref{logic:eqn:order:propositional}) is a case of {\em
definition by structural recursion}. There are many natural mappings
on $X$ which could be defined in a similar way. For instance,
suppose we needed a map $g:X\to{\cal P}(X_{0})$ returning the set of
all atomic propositions involved in a formula $\phi\in X$. In the
case of $\phi=\lnot(p_{1}\land((\lnot p_{2})\land p_{3}))$, we would
have $g(\phi)=\{p_{1},p_{2},p_{3}\}$. A natural way to define the
map $g$ is:
    \begin{equation}\label{logic:eqn:set:propositional}
    \forall\phi\in X\ ,\ g(\phi)=\left\{
                    \begin{array}{lcl}
                    \{p\}&\mbox{\ if\ }&\phi=p\in X_{0}\\
                    g(\phi_{1})&\mbox{\ if\ }&\phi=\lnot\phi_{1}\\
                    g(\phi_{1})\cup g(\phi_{2})&\mbox{\ if\ }&\phi=\phi_{1}\land\phi_{2}
                    \end{array}\right.
    \end{equation}
As another example, suppose we had a truth table $v:X_{0}\to\{0,1\}$
indicating whether an atomic proposition $p$ is true ($v(p)=1$) or
false ($v(p)=0$) for all $p\in X_{0}$. We would like to define a map
$g:X\to\{0,1\}$ indicating whether a proposition $\phi$ is true
($g(\phi)=1$) or false ($g(\phi)=0$). Once again, it is natural to
define the map $g$ using structural recursion:
 \begin{equation}\label{logic:eqn:truth:propositional}
    \forall\phi\in X\ ,\ g(\phi)=\left\{
                    \begin{array}{lcl}
                    v(p)&\mbox{\ if\ }&\phi=p\in X_{0}\\
                    1-g(\phi_{1})&\mbox{\ if\ }&\phi=\lnot\phi_{1}\\
                    g(\phi_{1}).\,g(\phi_{2})&\mbox{\ if\ }&\phi=\phi_{1}\land\phi_{2}
                    \end{array}\right.
    \end{equation}
We know that equations~(\ref{logic:eqn:set:propositional})
and~(\ref{logic:eqn:truth:propositional}) are fine: we just need to
prove~it.

So let us go back to a more general setting and consider a free
universal algebra $X$ of type $\alpha$ with free generator
$X_{0}\subseteq X$. Let $A$ be a set. We would like to define a map
$g:X\to A$ by structural recursion. We know from
theorem~(\ref{logic:the:unique:representation}) of
page~\pageref{logic:the:unique:representation} that any element of
$X$ is either an element of $X_{0}$, or an element of the form
$f(x)$ for some unique $f\in\alpha$ and $x\in X^{\alpha(f)}$.
Defining $g:X\to A$ by structural recursion consists first in
specifying $g(x)$ for all $x\in X_{0}$. So we need a map
$g_{0}:X_{0}\to A$ with the understanding that $g(x)=g_{0}(x)$ for
all $x\in X_{0}$. For all $f\in\alpha$ and $x\in X^{\alpha(f)}$, we
then want to specify $g(f(x))$ as a function of all the individual
$g(x(i))$'s for all $i\in\alpha(f)$. So we need a map
$h(f):A^{\alpha(f)}\to A$ with the understanding that
$g(f(x))=h(f)(g(x))$. Note that if $\alpha(f)=0$, this amounts to
having a map $h(f):\{0\}\to A$ and requesting that
$g(f(0))=h(f)(0)$. In other words, it amounts to specifying the
value $h(f)(0)$ assigned by the map $g$ to the constant $f(0)$. We
are now in a position to ask the key question: given a map
$g_{0}:X_{0}\to A$ and given a map $h(f):A^{\alpha(f)}\to A$ for all
$f\in\alpha$, does there exist a map $g:X\to A$ such that
$g_{|X_{0}}=g_{0}$ and $g(f(x))=h(f)(g(x))$ for all $f\in\alpha$ and
$x\in X^{\alpha(f)}$? This question is dealt with in the following
theorem.

To prove theorem~(\ref{logic:the:structural:recursion}), we are
effectively resorting to the same trick as the one used to define
the order $\om:X\to\N$ in definition~(\ref{logic:def:order}): once
we give ourselves a map $h(f):A^{\alpha(f)}\to A$ for all
$f\in\alpha$, we are effectively specifying a structure of universal
algebra of type $\alpha$ on $A$, and requesting that
$g(f(x))=h(f)(g(x))$ is simply asking that $g:X\to A$ be a morphism.
The existence of $g$ such that $g_{|X_{0}}=g_{0}$ is guaranteed by
the fact that $X_{0}$ is a free generator of $X$.
\index{recursion@Structural recursion}
\begin{theorem}\label{logic:the:structural:recursion}
Let $X$ be a free universal algebra of type $\alpha$  with free
generator $X_{0}\subseteq X$. Let $A$ be a set and $g_{0}:X_{0}\to
A$ be a map. Let $h$ be a map with domain $\alpha$ such that $h(f)$
is a map $h(f):A^{\alpha(f)}\to A$ for all $f\in\alpha$. Then, there
exists a unique map $g:X\to A$ such that:
    \begin{eqnarray*}
    (i)&& x\in X_{0}\ \Rightarrow g(x)=g_{0}(x)\\
    (ii)&& x\in X^{\alpha(f)}\ \Rightarrow g(f(x)) = h(f)(g(x))
    \end{eqnarray*}
where $(ii)$ holds for all $f\in\alpha$.
\end{theorem}
\begin{proof}
Since $h$ is a map with domain $\alpha$ and we have
$h(f):A^{\alpha(f)}\to A$ for all $f\in\alpha$, the ordered pair
$(A,h)$ is in fact a universal algebra of type $\alpha$. Since
$g_{0}:X\to A$ is a map and $X$ is a free universal algebra of type
$\alpha$ with free generator $X_{0}$, there exists a unique morphism
$g:X\to A$ such that $g_{|X_{0}}=g_{0}$. From this last equality, we
see that $(i)$ is satisfied. Since $g$ is a morphism, for all
$f\in\alpha$ and $x\in X^{\alpha(f)}$ we have:
    \[
    g(f(x))=g\circ f(x)=h(f)\circ g(x)=h(f)(g(x))
    \]
where it is understood that '$g(x)$' on the right-hand-side of this
equation refers to the element of $A^{\alpha(f)}$ defined by
$g(x)(i)=g(x(i))$ for all $i\in\alpha(f)$. Hence we see that $(ii)$
is satisfied by $g:X\to A$ for all $f\in\alpha$. So we have proved
the existence of $g:X\to A$ such that $(i)$ and $(ii)$ holds for all
$f\in\alpha$. Suppose $g':X\to A$ is another map with such property.
Then $g'$ is clearly a morphism with $g'_{|X_{0}}=g_{0}$ and it
follows from the uniqueness of $g$ that $g'=g$.
\end{proof}

This was disappointingly simple. One would have expected the
principle of {\em definition by structural recursion} to be a lot
more complicated than that of recursion over $\N$. And it can be. We
effectively restricted ourselves to the simple case of $g:X\to A$
where $A$ is a set rather than a class, and $g(f(x))$ is simply a
function of the $g(x(i))$'s. This would be similar to considering
$g:\N\to A$ given a set $A$, and requesting that $g(n)=h\circ
g(n-1)$ rather than $g(n)=h(g_{|n})$. Anticipating on future events,
suppose $X$ is a free algebra of first order logic, generated by
atomic propositions $(x\in y)$ (with $x,y$ belonging to a set of
{\em variables} $V$),  with the constant $\bot$, the binary operator
$\to$ and a unary quantification operator $\forall x$ for every
variable $x\in V$. Given a set $M$ with a binary relation
$r\subseteq M\times M$ and a map $a:V\to M$ (a {\em variables
assignment}), a natural thing to ask is which of the formulas
$\phi\in X$ are {\em true} with respect to the model $(M,r)$ under
the assignment~$a:V\to M$. In other words, it is very tempting to
define a truth function $g_{a}:X\to 2=\{0,1\}$ with the formula:
    \[
                    g_{a}(\phi)=\left\{
                    \begin{array}{lcl}
                    1_{r}(\,a(x),a(y)\,)&\mbox{\ if\ }&\phi=(x\in y)\\
                    0&\mbox{\ if\ }&\phi=\bot\\
                    g_{a}(\phi_{1})\to g_{a}(\phi_{2})&\mbox{\ if\ }&
                    \phi=\phi_{1}\to\phi_{2}\\
                    \min\left\{\,g_{b}(\phi_{1})\ :\  b=a\mbox{\ on\
                    }V\setminus\{x\}\,\right\}
                    &\mbox{\ if\ }&\phi=\forall x\phi_{1}\\
                    \end{array}\right.
    \]
being understood that $g_{a}(\phi_{1})\to g_{a}(\phi_{2})$ is the
usual $\lnot g_{a}(\phi_{1})\lor g_{a}(\phi_{2})$ in $\{0,1\}$. This
definition of the concept of {\em truth} will be seen to be flawed:
it is clear enough at this stage that
theorem~(\ref{logic:the:structural:recursion}) will not be directly
applicable to this case, since we are attempting to define
$g_{a}:X\to 2$ in terms of other functions $g_{b}:X\to 2$. We will
need to find some other way to prove the existence of $g_{a}$. Going
back to propositional logic, another possible example is the
following:
\begin{equation}\label{logic:eqn:class:propositional}
    \forall\phi\in X\ ,\ g(\phi)=\left\{
                    \begin{array}{lcl}
                    \{0\}&\mbox{\ if\ }&\phi=p\in X_{0}\\
                    {\cal P}(g(\phi_{1}))&\mbox{\ if\ }&\phi=\lnot\phi_{1}\\
                    g(\phi_{1})\times g(\phi_{2})&\mbox{\ if\ }&\phi=\phi_{1}\land\phi_{2}
                    \end{array}\right.
    \end{equation}
This would constitute a case when
theorem~(\ref{logic:the:structural:recursion}) fails to be
applicable, as there is no obvious set $A$ for $g:X\to A$. We shall
not provide a meta-theorem for this.

We shall however now provide a stronger version of
theorem~(\ref{logic:the:structural:recursion}) which will be
required at a later stage of this document. For example, when
studying the Hilbert deductive system on the free algebra of first
order predicate logic \pv, we shall define the notion of {\em
proofs} as elements of another free algebra \pvs, which in turn will
lead to the following recursion:
\[
    \forall\pi\in\pvs\ ,\ \val(\pi)=\left\{
                    \begin{array}{lcl}
                    \phi&\mbox{\ if\ }&\pi=\phi\in\pv\\
                    \phi&\mbox{\ if\ }&\pi=\axi\phi,\ \phi\in\av\\
                    \bot\to\bot&\mbox{\ if\ }&\pi=\axi\phi,\ \phi\not\in\av\\
                    M(\val(\pi_{1}), \val(\pi_{2})) &\mbox{\ if\ }&\pi=\pi_{1}\pon\pi_{2}\\
                    \forall x\val(\pi_{1})&\mbox{\ if\ }&\pi=\gen
                    x\pi_{1},\  x\not\in\spec(\pi_{1})\\
                    \bot\to\bot&\mbox{\ if\ }&\pi=\gen
                    x\pi_{1},\  x\in\spec(\pi_{1})\\
                    \end{array}\right.
\]
when attempting to define $\val(\pi)$ as representing the conclusion
being proved by the proof $\pi$. The details of this definition are
not important at this stage, but we should simply notice that given
a variable $x$ and the associated unary operator $\gen
x:\pvs\to\pvs$, we were unable to define $\val(\gen x\pi_{1})$
simply in terms of $\val(\pi_{1})$ as an application of
theorem~(\ref{logic:the:structural:recursion}) would require. The
definition of $\val(\gen x\pi_{1})$ is also based on whether
$x\in\spec(\pi_{1})$, i.e. on whether $x$ is a free variable of the
premises involved in the proof $\pi_{1}$. Going back to the general
setting of an arbitrary free algebra $X$ of type $\alpha$, this is
therefore a case when given $f\in\alpha$ and $x\in X^{\alpha(f)}$,
$g(f(x))$ is not simply defined in terms of $g(x)$, but is also
based on $x$ itself. So we would like to set $g(f(x))=h(f)(g(x),x)$
rather than the more restrictive $g(f(x))=h(f)(g(x))$ of
theorem~(\ref{logic:the:structural:recursion}).

The proof of the following theorem is identical in spirit to that of
Lemma~(\ref{logic:lemma:recursion:over:N}). The only new idea is the
introduction of the subsets $X_{n}\subseteq X$ of
proposition~(\ref{logic:prop:order:structure}):
    \[
    X_{n}=\{x\in X\ :\ \om(x)\leq n\}\ ,\ n\in\N
    \]
where $\om:X\to\N$ is the order mapping of
definition~(\ref{logic:def:order}) which allows to reduce our
problem to that of induction over \N. Once we have shown the
existence of maps $g_{n}:X_{n}\to A$ with the right property, and
which are extensions of one another, we prove the existence of the
larger map $g:X\to A$ simply by collecting the appropriate ordered
pairs in one single set $g=\cup_{n\in\N}g_{n}$. Note that it is
probably possible to prove
theorem~(\ref{logic:the:structural:recursion:2}) using
lemma~(\ref{logic:lemma:recursion:over:N}) rather than duplicating
what is essentially the same argument. \index{recursion@Structural
recursion}
\begin{theorem}\label{logic:the:structural:recursion:2}
Let $X$ be a free universal algebra of type $\alpha$  with free
generator $X_{0}\subseteq X$. Let $A$ be a set and $g_{0}:X_{0}\to
A$ be a map. Let $h$ be a map with domain $\alpha$ such that $h(f)$
is a map $h(f):A^{\alpha(f)}\times X^{\alpha(f)}\to A$ for all
$f\in\alpha$. Then, there exists a unique map $g:X\to A$ such that:
    \begin{eqnarray*}
    (i)&& x\in X_{0}\ \Rightarrow g(x)=g_{0}(x)\\
    (ii)&& x\in X^{\alpha(f)}\ \Rightarrow g(f(x)) = h(f)(g(x),x)
    \end{eqnarray*}
where $(ii)$ holds for all $f\in\alpha$.
\end{theorem}
\begin{proof}
We shall first prove the uniqueness property. So suppose $g,g':X\to
A$ are two maps satisfying $(i)$ and $(ii)$ above. We need to show
that $g=g'$ or equivalently that $g(x)=g'(x)$ for all $x\in X$. We
shall do so using a structural induction argument based on
theorem~(\ref{logic:the:proof:induction}) of
page~\pageref{logic:the:proof:induction}. Since $X_{0}$ is a
generator of $X$, we shall first check this is the case when $x\in
X_{0}$. This follows immediately from~$(i)$ and
$g(x)=g_{0}(x)=g'(x)$. We proceed with our induction argument by
considering an arbitrary $f\in\alpha$ and assuming $x\in
X^{\alpha(f)}$ is such that $g(x(i))=g'(x(i))$ for all
$i\in\alpha(f)$. We need to show that $g(f(x))=g'(f(x))$. Recall
that the notation $g(x)$ in $(ii)$ above refers to the element of
$A^{\alpha(f)}$ defined by $g(x)(i)=g(x(i))$ for all
$i\in\alpha(f)$. In the case when $\alpha(f)=0$ we have $g(x)=0$. So
with this in mind and a similar notational convention for $g'(x)$,
our hypothesis leads to $g(x)=g'(x)$, and using $(ii)$ above we
obtain:
    \[
    g(f(x))=h(f)(g(x),x)=h(f)(g'(x),x)=g'(f(x))
    \]
This completes our induction argument and the proof of uniqueness.
We shall now prove the existence of the map $g:X\to A$. We define:
    \[
    X_{n}=\{x\in X\ :\ \om(x)\leq n\}\ ,\ n\in\N
    \]
where $\om:X\to\N$ is the order on $X$ as per
definition~(\ref{logic:def:order}). Note that from
proposition~(\ref{logic:prop:order}) we have $x\in X_{0}\
\Leftrightarrow\ \om(x)=0$ so this new definition does not conflict
with the notation $X_{0}$ used for the free generator on $X$. We now
consider the subset $Y\subseteq\N$ of all $n\in\N$ which satisfy the
property that there exists a unique map $g_{n}:X_{n}\to A$ such
that:
    \begin{eqnarray*}
    (iii)&& x\in X_{0}\ \Rightarrow g_{n}(x)=g_{0}(x)\\
    (iv)&& f(x)\in X_{n}\ \Rightarrow g_{n}(f(x)) = h(f)(g_{n}(x),x)
    \end{eqnarray*}
where $(iv)$ holds for all $f\in\alpha$ and $x\in X^{\alpha(f)}$. In
order to prove the existence of the map $g:X\to A$, it is sufficient
to prove that $Y=N$. We shall first prove this is the case. So
suppose $Y=\N$. We need to show the existence of the map $g:X\to A$.
Define $g=\cup_{n\in N}g_{n}$. Each $g_{n}$ being a map is a set of
ordered pairs. So $g$ is also a set of ordered pairs. In fact, it is
a functional set of ordered pairs:
    \[
    \forall x,y,y'\ ,\ ((x,y)\in g)\land((x,y')\in g)\ \Rightarrow\
    (y=y')
    \]
Suppose this is true for the time being. Then $g$ is a map.
Furthermore, since $g_{n}:X_{n}\to A$ and $X_{n}\subseteq X$ for all
$n\in\N$ we have:
    \[
    (x,y)\in g\ \Rightarrow\ \exists n\in\N [(x,y)\in g_{n}]\ \Rightarrow\
    (x\in X)\land(y\in A)
    \]
So it is clear that $\dom(g)\subseteq X$ and $\rng(g)\subseteq A$.
In fact, since every $x\in X$ is an element of $X_{n}$ for any
$n\geq\om(x)$, we have $\dom(g)=X$ and we conlude that $g:X\to A$.
In order to prove the existence of $g:X\to A$ it remains to show
that $(i)$ and $(ii)$ above are satisfied. So let $x\in X_{0}$. From
$(x,g_{0}(x))\in g_{0}\subseteq g$ it follows immediately that
$g(x)=g_{0}(x)$ which shows that $(i)$ is indeed true. Let
$f\in\alpha$ and $x\in X^{\alpha(f)}$. Picking $n\in\N$ large enough
so that $f(x)\in X_{n}$ we obtain $(f(x),g_{n}(f(x)))\in
g_{n}\subseteq g$ from which we conclude that $g(f(x))=g_{n}(f(x))$.
Furthermore, for all $i\in\alpha(f)$ from
proposition~(\ref{logic:prop:order}) we have
$\om(f(x))\geq\om(x(i))$ which shows that $x(i)\in X_{n}$ and
consequently $(x_{i},g_{n}(x(i)))\in g_{n}\subseteq g$ i.e.
$g(x(i))=g_{n}(x(i))$. This being true for all $i\in\alpha(f)$, we
obtain $g(x)=g_{n}(x)$, an equality which is still true when
$\alpha(f)=0$. Hence, using $(iv)$ above we have:
    \[
    g(f(x))=g_{n}(f(x))=h(f)(g_{n}(x),x)=h(f)(g(x),x)
    \]
which shows that $(ii)$ is indeed true. This completes the proof of
the existence of $g:X\to A$ having admitted that $g$ is a functional
relation. We shall now go back to this particular point and show
that $g$ is indeed functional. So let $x,y,y'$ be sets and assume
that $(x,y)\in g$ and $(x,y')\in g$. We need to show that $y=y'$.
Let $n\in\N$ be the smallest integer such that $(x,y)\in g_{n}$ and
likewise let $n'\in\N$ be the smallest integer such that $(x,y')\in
g_{n'}$. Without loss of generality, we may assume that $n\leq n'$
and consequently $X_{n}\subseteq X_{n'}$. Consider the restriction
$(g_{n'})_{|X_{n}}:X_{n}\to A$. Since $g_{n'}:X_{n'}\to A$ satisfies
$(iii)$ and $(iv)$ above for $n'$, for all $z\in X_{0}$ we have
$(g_{n'})_{|X_{n}}(z) = g_{n'}(z)=g_{0}(z)$ and furthermore given
$f\in\alpha$ and $z\in X^{\alpha(f)}$ such that $f(z)\in X_{n}$ we
have:
    \[
    (g_{n'})_{|X_{n}}(f(z)) = g_{n'}(f(z)) =
    h(f)(g_{n'}(z),z)=h(f)((g_{n'})_{|X_{n}}(z),z)
    \]
where the last equality follows from the fact that $z(i)\in X_{n}$
for all $i\in\alpha(f)$. Hence we see that the map
$(g_{n'})_{|X_{n}}:X_{n}\to A$ also satisfies $(iii)$ and $(iv)$
above and from the uniqueness property of $g_{n}$ we conclude that
$(g_{n'})_{|X_{n}}=g_{n}$. We are now in a position to prove that
$y=y'$. From $(x,y)\in g_{n}$ we obtain $x\in X_{n}$ and
$y=g_{n}(x)$ while from $(x,y')\in g_{n'}$ we obtain $y'=g_{n'}(x) =
(g_{n'})_{|X_{n}}(x)$. From $(g_{n'})_{|X_{n}}=g_{n}$ we therefore
conclude that $y=y'$. This completes our proof of the existence of
the map $g:X\to A$ having assumed the equality $Y=\N$. We shall
complete the proof of this theorem by showing the equality $Y=\N$ is
indeed true. So given $n\in\N$ we need to show the existence and
uniqueness of a map $g_{n}:X_{n}\to A$ which satisfies $(iii)$ and
$(iv)$ above. First we prove the uniqueness. So we assume that
$g,g':X_{n}\to A$ are two maps satisfying $(iii)$ and $(iv)$ above
and we need to show that $g(x) = g'(x)$ for all $x\in X_{n}$. We
shall do so by proving the implication $x\in X_{n}\ \Rightarrow\
g(x)=g'(x)$ by structural induction on $X$, using
theorem~(\ref{logic:the:proof:induction}) of
page~\pageref{logic:the:proof:induction}. First we check that this
is true for $x\in X_{0}$. In this case from $(iii)$ above we obtain
immediately $g(x)=g_{0}(x)=g'(x)$. Next we consider $f\in\alpha$ and
$x\in X^{\alpha(f)}$ such that the implication $x(i)\in X_{n}\
\Rightarrow\ g(x(i))=g'(x(i))$ is true for all $i\in\alpha(f)$, i.e.
$g(x)=g'(x)$, an equality which is still true in the case when
$\alpha(f)=0$. We need to check the implication $f(x)\in X_{n}\
\Rightarrow\ g(f(x))=g'(f(x))$,  which in fact follows immediately
from $(iv)$ and:
    \[
    g(f(x))=h(f)(g(x),x)=h(f)(g'(x),x)=g'(f(x))
    \]
This completes our structural induction argument and $g_{n}:X_{n}\to
A$ is indeed unique. We shall now prove the existence
$g_{n}:X_{n}\to A$ using an induction argument on $\N$. For $n=0$,
we already have a map $g_{0}:X_{0}\to A$ which clearly satisfies
$(iii)$ above. It also vacuously satifies $(iv)$ as
theorem~(\ref{logic:the:unique:representation}) of
page~\pageref{logic:the:unique:representation} shows no element of
$X_{0}$ can be of the form $f(x)$ for $f\in\alpha$ and $x\in
X^{\alpha(f)}$. So we assume that $g_{n}:X_{n}\to A$ exists for a
given $n\in\N$ and we need to show the existence of
$g_{n+1}:X_{n+1}\to A$. Recall the equality from
proposition~(\ref{logic:prop:order:structure}):
    \[
    X_{n+1}=X_{n}\cup\{f(x)\ :\ f\in\alpha\ ,\
    x\in(X_{n})^{\alpha(f)}\}
    \]
So we shall define $g_{n+1}:X_{n+1}\to A$ by setting
$(g_{n+1})_{|X_{n}}=g_{n}$ and:
     \[
     g_{n+1}(y) = h(f)(g_{n}(x),x)
     \]
for all $y\in X_{n+1}\setminus X_{n}$, where $y = f(x)$ is the
unique representation of $y$ with $f\in\alpha$ and
$x\in(X_{n})^{\alpha(f)}$. Such representation is indeed unique, as
follows from theorem~(\ref{logic:the:unique:representation}) of
page~\pageref{logic:the:unique:representation}. We have thus defined
a map $g_{n+1}:X_{n+1}\to A$ and it remains to check that $g_{n+1}$
satisfies $(iii)$ and $(iv)$ above. So let $x\in X_{0}$. Since
$g_{n}$ satisfies $(iii)$ we have
$g_{n+1}(x)=(g_{n+1})_{|X_{n}}(x)=g_{n}(x)=g_{0}(x)$ and $g_{n+1}$
also satisfies $(iii)$. So let $f\in\alpha$ and $x\in X^{\alpha(f)}$
be such that $y=f(x)\in X_{n+1}$. We need to show that
$g_{n+1}(f(x)) = h(f)(g_{n+1}(x),x)$. We shall distinguish two
cases: first we assume that $f(x)\in X_{n}$. Then, since $(iv)$ is
true for $g_{n}$:
    \begin{eqnarray*}
    g_{n+1}(f(x)) &=& (g_{n+1})_{|X_{n}}(f(x))\\
        &=& g_{n}(f(x))\\
        (iv)\mbox{\ true\ }g_{n}\ \rightarrow\ &=&h(f)(g_{n}(x),x)\\
        &=&h(f)((g_{n+1})_{|X_{n}}(x),x)\\
        &=&h(f)(g_{n+1}(x),x)
    \end{eqnarray*}
We now assume that $f(x)\in X_{n+1}\setminus X_{n}$. From
$x\in(X_{n})^{\alpha(f)}$ we obtain:
    \begin{eqnarray*}
    g_{n+1}(f(x))&=&h(f)(g_{n}(x),x)\\
        &=& h(f)((g_{n+1})_{|X_{n}}(x),x)\\
        &=& h(f)(g_{n+1}(x),x)
    \end{eqnarray*}
So we have proved that $g_{n+1}$ satisfies $(iv)$ above as
requested.
\end{proof}
