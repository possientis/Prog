We all know that $\forall x(x\in z)$ and $\forall y(y\in z)$ should
be identical mathematical statements when $z\not\in\{x,y\}$. In this
section, we shall attempt to formally define a congruence $\sim$ on
\pv\ such that $\forall x(x\in z)\sim\forall y(y\in z)$ is satisfied
for all $z\not\in\{x,y\}$. More generally, given a set $V$ and a
formula $\phi_{1}\in\pv$, we would like our congruence $\sim$ to be
such that $\forall x\phi_{1}\sim\forall y\phi_{1}[y/x]$ for all
$x,y\in V$. In other words, if we were to replace the variable $x$
by the variable $y$ in the formula $\phi=\forall x\phi_{1}$, the
formula $\psi=\forall y\phi_{1}[y/x]$ resulting from the
substitution should have the same {\em meaning} as that of $\phi$.
One of the difficulties in pinning down a formal definition for the
congruence $\sim$ is to know exactly what we want. If
$\phi_{1}=(x\in y)$ with $x\neq y$, then $\phi=\forall x(x\in y)$
while $\psi=\forall y(y\in y)$ and it is clear that $\phi$ and
$\psi$ do not represent the same mathematical statement. It would
therefore be inappropriate to require that $\forall
x\phi_{1}\sim\forall y\phi_{1}[y/x]$ for all $x,y\in V$, without
some form of restriction on the variables $x$ and $y$. Looking back
at the example of $\phi_{1}=(x\in y)$, the problem arises from the
fact that the substitution $[y/x]$ is not a valid substitution for
$\forall x\phi_{1}$. It is therefore tempting to require that
$\forall x\phi_{1}\sim\forall y\phi_{1}[y/x]$ for all $x,y\in V$
such that $[y/x]$ is valid for $\forall x\phi_{1}$. From
proposition~(\ref{logic:prop:FOPL:validsub:singlevar}), we can
easily check that $[y/x]$ is automatically valid for $\forall
x\phi_{1}$ whenever the condition $y\not\in\var(\phi_{1})$ is
satisfied. So we could also require that $\forall
x\phi_{1}\sim\forall y\phi_{1}[y/x]$ solely when the condition
$y\not\in\var(\phi_{1})$ is met. This is obviously simpler as it
does not require the concept of valid substitution, and it is also
in line with the existing literature (e.g. see Donald W. Barnes,
John M. Mack~\cite{AlgLog} page~28). As it turns out, whether we
decide to go for the condition $y\not\in\var(\phi_{1})$ or {\em
$[y/x]$ is valid for $\forall x\phi_{1}$} will lead to the same
congruence, as can be seen from
proposition~(\ref{logic:prop:admissible:strong}). So the question is
settled: we shall require that $\forall x\phi_{1}\sim\forall
y\phi_{1}[y/x]$ whenever $y\not\in\var(\phi_{1})$. Since a
congruence is always reflexive, there is no harm in imposing that
$x\neq y$. We shall define our congruence in terms of a generator,
in the sense of definition~(\ref{logic:def:generated:congruence}).
Such a congruence exists and is unique by virtue of
theorem~(\ref{logic:the:generated:congruence}) of
page~\pageref{logic:the:generated:congruence}. 


However despite our
best efforts, it will appear later that  the congruence $\sim$ is
not the right one. It will in fact be too strong when the set $V$ is
a finite set. For this reason, the congruence presented in this
section will be called {\em strong $\alpha$-equivalence}. In
later parts of this document, we shall alter our definition slightly
so as to reach what we believe is a definitive notion of {\em
$\alpha$-equivalence} working equally well for $V$ finite and $V$
infinite. Despite its minor flaws, the {\em strong $\alpha$-equivalence} 
deserves to be studied in its own right, as it seems to
be the one studied in every known textbook (which assumes an
infinite set $V$).

\index{congruence@Strong substitution congruence}
\begin{defin}\label{logic:def:strong:sub:congruence}
Let $V$ be a set. We call {\em strong $\alpha$-equivalence on
\pv\ }the congruence on \pv\ generated by the following set
$R_{0}\subseteq \pv\times\pv$:
    \[
    R_{0}=\left\{\,(\,\forall x\phi_{1}\,,\,\forall
    y\,\phi_{1}[y/x]\,):\phi_{1}\in\pv\ ,\ x,y\in V\ ,\ x\neq y\ ,\
    y\not\in\var(\phi_{1})\,\right\}
    \]
where $[y/x]$ denotes the substitution of $y$ in place of $x$ as per
{\em definition~(\ref{logic:def:single:var:substitution})}.
\end{defin}
