Prior to defining the strong $\alpha$-equivalence on
page~\pageref{logic:def:strong:sub:congruence}, we discussed what we
believed were the requirements an {\em $\alpha$-equivalence} ought
to meet. Unfortunately, despite our best efforts, we failed to
define the right notion. So we are back to our starting point,
enquiring about the appropriate definition of an 
{\em $\alpha$-equivalence}. In 
proposition~(\ref{logic:prop:substitution:invariant}) we showed that
given a formula $\phi$, we have the equivalence $\phi[y/x]\sim\phi$
provided $y\not\in\var(\phi)$ and $x\not\in\free(\phi)$. The key
idea underlying this proposition is that some substitution should
not affect the {\em meaning} of a formula. This may give us a new
angle of attack. In this section, we define those substitutions
$\sigma$ which we believe should have this invariance property of
not altering the $\alpha$-equivalence class of a given formula $\phi$, i.e.
such that $\sigma(\phi)\sim\phi$. We shall then define a new 
$\alpha$-equivalence.

In proposition~(\ref{logic:prop:substitution:invariant}) we required
that $x\not\in\free(\phi)$. This condition ensures the substitution
$[y/x]$ does not affect free variables of the formula~$\phi$. There
is clearly a good reason for this. An $\alpha$-equivalence is all
about potentially differing quantification variables. We cannot
expect to have $\sigma(\phi)\sim\phi$ unless the free variables of
$\phi$ are invariant under the substitution $\sigma$. Furthermore,
proposition~(\ref{logic:prop:substitution:invariant}) also required
$y\not\in\var(\phi)$. This condition guarantees the substitution
$[y/x]$ is valid for $\phi$. This motivates the following
definition:

\index{admissible@Admissible substitution for formula}
\begin{defin}\label{logic:def:admissible:substitution}
Let $V$ be a set and $\sigma:V\to V$ be a map. Let $\phi\in\pv$. We
say that $\sigma$ is an {\em admissible substitution for $\phi$}
\ifand\ it satisfies:
    \begin{eqnarray*}
    (i)&&\mbox{$\sigma$ valid for $\phi$}\\
    (ii)&&\forall u\in\free(\phi)\ ,\ \sigma(u)=u
    \end{eqnarray*}
\end{defin}

Having defined admissible substitutions with respect to a formula
$\phi$, our belief is that an appropriate {\em $\alpha$-equivalence} 
$\sim$ should be such that $\sigma(\phi)\sim\phi$
whenever $\sigma$ is admissible for $\phi$. If we consider
$\phi=\forall x\forall y(x \in y)$ and the permutation
$\sigma=[y\!:\!x]$ for $x\neq y$, then $\sigma$ is clearly
admissible for $\phi$. Yet we know from
proposition~(\ref{logic:prop:counter:strong:1}) that the equivalence
$\sigma(\phi)\sim\phi$ fails when $V=\{x,y\}$ and $\sim$ is the
strong $\alpha$-equivalence on \pv. To remedy this failure, we
may conjecture that an appropriate definition of {\em $\alpha$-equivalence} 
should be made in reference to ordered pairs
$(\phi,\sigma(\phi))$ where $\sigma$ is admissible for $\phi$. In
other words, if we define:
    \[
    R_{1}=\left\{\,(\,\phi\,,\,\sigma(\phi)\,):\phi\in\pv\ ,\
    \mbox{$\sigma:V\to V$ admissible for $\phi$} \right\}
    \]
the {\em right} $\alpha$-equivalence should simply be defined as
the congruence on \pv\ generated by $R_{1}$. As it turns out, we
shall adopt a different but equivalent definition of 
$\alpha$-equivalence, in terms of a generator $R_{0}$ which is a
lot smaller than $R_{1}$. In effect, rather than considering all
possible ordered pair $(\phi,\sigma(\phi))$ where $\sigma$ is
admissible for $\phi$, we restrict our attention to the case when
$\phi=\forall x\phi_{1}$ and $\sigma=[y\!:\!x]$ for $x\neq y$ and
$y\not\in\free(\phi_{1})$, thereby obtaining a definition of 
$\alpha$-equivalence which is formally very similar to
definition~(\ref{logic:def:strong:sub:congruence}). 

\index{congruence@Substitution
congruence}\index{alpha@$\alpha$-equivalence}
\begin{defin}\label{logic:def:sub:congruence}
Let $V$ be a set. We call {\em $\alpha$-equivalence on \pv\ }the
congruence on \pv\ generated by the following set $R_{0}\subseteq
\pv\times\pv$:
    \[
    R_{0}=\left\{\,(\,\forall x\phi_{1}\,,\,\forall y\,\phi_{1}[y\!:\!x]\,):
    \phi_{1}\in\pv\ ,\ x,y\in V\ ,\ x\neq y\ ,\ y\not\in\free(\phi_{1})\,\right\}
    \]
where $[y\!:\!x]$ denotes the permutation of $x$ and $y$ as per {\em
definition~(\ref{logic:def:single:var:permutation})}.
\end{defin}

We have now defined what we hope will be a definitive notion of 
{\em $\alpha$-equivalence}. However, the
definition~(\ref{logic:def:sub:congruence}) we chose does not make
reference to ordered pairs $(\phi,\sigma(\phi))$ where $\sigma$ is
an admissible substitution for $\phi$. One of our first task is to
make sure definition~(\ref{logic:def:sub:congruence}) is equivalent
to having defined $\alpha$-equivalence as generated by these
ordered pairs. We start by showing that $\phi\sim\sigma(\phi)$
whenever $\sigma$ is admissible for $\phi$.

\begin{prop}\label{logic:prop:admissible:sub:congruence}
Let $\sim$ be the $\alpha$-equivalence on \pv\ where $V$ is a
set. Let $\phi\in\pv$ and $\sigma:V\to V$ be an admissible
substitution for $\phi$. Then:
    \[
    \phi\sim\sigma(\phi)
    \]
\end{prop}

\noindent
\begin{proof}
We need to show the property $\forall\sigma[\,(\sigma\mbox{\
admissible for\ }\phi)\ \Rightarrow\ \phi\sim\sigma(\phi)\,]$ for
all $\phi\in\pv$. We shall do so by structural induction, using
theorem~(\ref{logic:the:proof:induction}) of
page~\pageref{logic:the:proof:induction}. Since \pvo\ is a generator
of \pv, we shall first show that the property is true on \pvo. So
let $\phi=(x\in y)\in\pvo$ where $x,y\in V$. We assume that
$\sigma:V\to V$ is an admissible substitution for $\phi$. We need to
show that $\phi\sim\sigma(\phi)$. However since
$\free(\phi)=\{x,y\}$ and $\sigma$ is an admissible substitution for
$\phi$, we have $\sigma(x)=x$ and $\sigma(y)=y$ and consequently:
    \[
    \sigma(\phi)=\sigma(x\in y)=(\,\sigma(x)\in\sigma(y)\,)=(x\in y)=\phi
    \]
and in particular we see that $\phi\sim\sigma(\phi)$. We now check
that the property is true for $\phi=\bot$. Note that any map
$\sigma:V\to V$ is an admissible substitution for $\bot$. Since we
always have $\bot=\sigma(\bot)$, it follows that
$\bot\sim\sigma(\bot)$. We now check that the property is true for
$\phi=\phi_{1}\to\phi_{2}$ if it is true for
$\phi_{1},\phi_{2}\in\pv$. So we assume that $\sigma:V\to V$ is an
admissible substitution for $\phi$. We need to show that
$\phi\sim\sigma(\phi)$. Since $\phi=\phi_{1}\to\phi_{2}$ and
$\sigma(\phi)=\sigma(\phi_{1})\to\sigma(\phi_{2})$, the 
$\alpha$-equivalence being a congruent relation on \pv, it is sufficient to
show that $\phi_{1}\sim\sigma(\phi_{1})$ and
$\phi_{2}\sim\sigma(\phi_{2})$. First we show that
$\phi_{1}\sim\sigma(\phi_{1})$. Having assumed the property is true
for $\phi_{1}$, it is sufficient to show that $\sigma$ is an
admissible substitution for $\phi_{1}$. Since $\sigma$ admissible
for $\phi$, in particular it is valid for $\phi$ and it follows from
proposition~(\ref{logic:prop:FOPL:valid:recursion:imp}) that it is
also valid for $\phi_{1}$. So it remains to show that $\sigma(u)=u$
for all $u\in\free(\phi_{1})$ which follows immediately from
$\free(\phi)=\free(\phi_{1})\cup\free(\phi_{2})$ and the fact that
$\sigma(u)=u$ for all $u\in\free(\phi)$. So we have proved that
$\phi_{1}\sim\sigma(\phi_{1})$ and we show similarly that
$\phi_{2}\sim\sigma(\phi_{2})$. We now need to check that the
property is true for $\phi=\forall x\phi_{1}$ if it is true for
$\phi_{1}\in\pv$. So we assume that $\sigma:V\to V$ is an admissible
substitution for $\phi$. We need to show that
$\phi\sim\sigma(\phi)$. We shall distinguish two cases: first we
assume that $\sigma(x)=x$. Then $\sigma(\phi)=\forall
x\,\sigma(\phi_{1})$ and in order to show $\phi\sim\sigma(\phi)$,
the $\alpha$-equivalence being a congruent relation on \pv, it is
sufficient to show that $\phi_{1}\sim\sigma(\phi_{1})$. Having
assumed the property is true for $\phi_{1}$, it is therefore
sufficient to prove that $\sigma$ is an admissible substitution for
$\phi_{1}$. Since $\sigma$ admissible for $\phi$, in particular it
is valid for $\phi$ and it follows from
proposition~(\ref{logic:prop:FOPL:valid:recursion:quant}) that it is
also valid for $\phi_{1}$. So it remains to show that $\sigma(u)=u$
for all $u\in\free(\phi_{1})$. We shall distinguish two further
cases: first we assume that $u=x$. Then $\sigma(u)=u$ is true from
our assumption $\sigma(x)=x$. So we assume that $u\neq x$. It
follows that $u\in\free(\phi_{1})\setminus\{x\}=\free(\phi)$, and
since $\sigma$ is admissible for $\phi$, we conclude that
$\sigma(u)=u$. This completes our proof of $\phi\sim\sigma(\phi)$ in
the case when $\sigma(x)=x$. We now assume that $\sigma(x)\neq x$.
Let $y=\sigma(x)$. Then $\sigma(\phi)=\forall y\,\sigma(\phi_{1})$
and we need to show that $\forall x\phi_{1}\sim\forall
y\,\sigma(\phi_{1})$. However, since $[y\!:\!x]\circ[y\!:\!x]$ is
the identity mapping we have $\sigma=[y\!:\!x]\circ\sigma^{*}$ where
the map $\sigma^{*}:V\to V$ is defined as
$\sigma^{*}=[y\!:\!x]\circ\sigma$. It follows that
$\sigma(\phi_{1})=\sigma^{*}(\phi_{1})[y\!:\!x]$ and we need to show
that $\forall x\phi_{1}\sim\forall y\sigma^{*}(\phi_{1})[y\!:\!x]$.
Let us accept for now that $y\not\in\free(\sigma^{*}(\phi_{1}))$.
Then from definition~(\ref{logic:def:sub:congruence}) we obtain
$\forall x\sigma^{*}(\phi_{1})\sim\forall
y\sigma^{*}(\phi_{1})[y\!:\!x]$, and it is therefore sufficient to
prove that $\forall x\phi_{1}\sim\forall x\sigma^{*}(\phi_{1})$. So
we see that it is sufficient to prove
$\phi_{1}\sim\sigma^{*}(\phi_{1})$ provided we can justify the fact
that $y\not\in\free(\sigma^{*}(\phi_{1}))$. First we show that
$\phi_{1}\sim\sigma^{*}(\phi_{1})$. Having assumed our property is
true for $\phi_{1}$ it is sufficient to prove that $\sigma^{*}$ is
admissible for $\phi_{1}$. However, we have already seen that
$\sigma$ is valid for $\phi_{1}$. Furthermore, since $[y\!:\!x]$ is
an injective map, from
proposition~(\ref{logic:prop:FOPL:valid:injective}) it is a valid
substitution for $\sigma(\phi_{1})$. It follows from
proposition~(\ref{logic:prop:FOPL:valid:composition}) that
$\sigma^{*}=[y\!:\!x]\circ\sigma$ is valid for $\phi_{1}$. So in
order to prove that $\sigma^{*}$ is admissible for $\phi_{1}$, it
remains to show that $\sigma^{*}(u)=u$ for all
$u\in\free(\phi_{1})$. So let $u\in\free(\phi_{1})$. We shall
distinguish two cases: first we assume that $u=x$. Then
$\sigma^{*}(u)=[y\!:\!x](\sigma(x))=[y\!:\!x](y)=x=u$. Next we
assume that $u\neq x$. Then $u$ is in fact an element of
$\free(\phi)$. Having assumed $\sigma$ is admissible for $\phi$ we
obtain $\sigma(u)=u$. We also obtain the fact that $\sigma$ is valid
for $\phi=\forall x\phi_{1}$ and consequently
$\sigma(u)\neq\sigma(x)$, i.e. $u\neq y$. Thus
$\sigma^{*}(u)=[y\!:\!x](\sigma(u))=[y\!:\!x](u)=u$. This completes
our proof that $\sigma^{*}$ is admissible for $\phi_{1}$ and
$\phi_{1}\sim\sigma^{*}(\phi_{1})$. It remains to show that
$y\not\in\free(\sigma^{*}(\phi_{1}))$. So suppose to the contrary
that $y\in\free(\sigma^{*}(\phi_{1}))$. We shall obtain a
contradiction. Using
proposition~(\ref{logic:prop:freevar:of:substitution:inclusion})
there exists $u\in\free(\phi_{1})$ such that $y=\sigma^{*}(u)$.
Having proven that $\sigma^{*}$ is admissible for $\phi_{1}$ we have
$\sigma^{*}(u)=u$ and consequently $y=u\in\free(\phi_{1})$. From the
assumption $y=\sigma(x)\neq x$ we in fact have $y\in\free(\phi)$. So
from the admissibility of $\sigma$ for $\phi$ we obtain
$\sigma(y)=y$ and furthermore from the validity of $\sigma$ for
$\phi=\forall x\phi_{1}$ we obtain $\sigma(y)\neq\sigma(x)$. So we
conclude that $y\neq\sigma(x)$ which contradicts our very definition
of $y$.
\end{proof}

We are now in a position to check that the $\alpha$-equivalence  
is also generated by the set of ordered pairs $(\phi,\sigma(\phi))$
where $\sigma$ is admissible for $\phi$.

\begin{prop}\label{logic:prop:sub:congruence:from:admissible}
Let $V$ be a set. Then the $\alpha$-equivalence on \pv\ is also
generated by the following set $R_{1}\subseteq \pv\times\pv$:
    \[
    R_{1}=\left\{\,(\,\phi\,,\,\sigma(\phi)\,):\phi\in\pv\ ,\
    \mbox{$\sigma:V\to V$ admissible for $\phi$} \right\}
    \]
\end{prop}

\noindent
\begin{proof}
Let $\sim$ denote the $\alpha$-equivalence on \pv\ and $\equiv$
be the congruence on \pv\ generated by $R_{1}$. We need to show that
$\sim\,=\,\equiv\,$. First we show $\sim\,\subseteq\,\equiv\,$.
Since $\sim$ is the smallest congruence on \pv\ which contains the
set $R_{0}$ of definition~(\ref{logic:def:sub:congruence}), in order
to prove $\sim\,\subseteq\,\equiv$ it is sufficient to prove that
$R_{0}\subseteq\equiv\,$. So let $\phi_{1}\in\pv$ and $x,y\in V$ be
such that $x\neq y$ and $y\not\in\free(\phi_{1})$. Define
$\phi=\forall x\phi_{1}$ and $\psi=\forall y\,\phi_{1}[y\!:\!x]$. We
need to show that $\phi\equiv\psi$. The congruence $\equiv$ being
generated by $R_{1}$ it is sufficient to prove that $(\phi,\psi)\in
R_{1}$. However, if we define $\sigma:V\to V$ by setting
$\sigma=[y\!:\!x]$ we have:
    \[
    \psi=\forall
    y\,\phi_{1}[y\!:\!x]=\forall\sigma(x)\,\sigma(\phi_{1})=\sigma(\forall
    x\phi_{1})=\sigma(\phi)
    \]
Hence, in order to show $(\phi,\psi)\in R_{1}$, it is sufficient to
prove that $\sigma$ is an admissible substitution for $\phi$. Being
injective, it is clear from
proposition~(\ref{logic:prop:FOPL:valid:injective}) that $\sigma$ is
valid for $\phi$. So we need to prove that $\sigma(u)=u$ for all
$u\in\free(\phi)$. So let $u\in\free(\phi)$. It is sufficient to
prove that $u\not\in\{x,y\}$. The fact that $u\neq x$ is clear from
$u\in\free(\phi)=\free(\phi_{1})\setminus\{x\}$. The fact that
$u\neq y$ follows from $u\in\free(\phi)\subseteq\free(\phi_{1})$ and
the assumption $y\not\in\free(\phi_{1})$. We now show that
$\equiv\,\subseteq\,\sim\,$. Since $\equiv$ is the smallest
congruence on \pv\ which contains the set $R_{1}$, it is sufficient
to show that $R_{1}\subseteq\sim$. So let $\phi\in\pv$ and
$\sigma:V\to V$ be an admissible substitution for $\phi$. We need to
show that $\phi\sim\sigma(\phi)$. But this follows immediately from
proposition~(\ref{logic:prop:admissible:sub:congruence}).
\end{proof}
