Just like the strong $\alpha$-equivalence, the $\alpha$-equivalence 
was defined in terms of a generator. This makes it
convenient to prove an equivalence $\phi\sim\psi$, but inconvenient
when it comes to proving this equivalence does not hold. We shall
therefore follow a similar strategy to that leading up to
theorem~(\ref{logic:the:strong:sub:congruence:charac}) of
page~\pageref{logic:the:strong:sub:congruence:charac}. We refer the
reader to the discussion preceding
definition~(\ref{logic:def:almost:strong:equivalent}). We shall
define a new relation $\simeq$ on \pv\ and prove that $\phi\sim\psi\
\Leftrightarrow\ \phi\simeq\psi$. The point of the relation $\simeq$
is to give us more insight about the formulas $\phi$ and $\psi$
whenever we have $\phi\sim\psi$. Most of the hard work consists in
proving that $\simeq$ is a transitive relation. However, this will
be a lot simpler than in the case of the strong $\alpha$-equivalence. 
In fact comparing
definition~(\ref{logic:def:almost:equivalent}) below to its
counterpart definition~(\ref{logic:def:almost:strong:equivalent}),
we see that these definitions are formally identical with a
difference occurring solely in $(v)$. 
For the strong $\alpha$-equivalence :
    \begin{eqnarray*}
    (v)&&\phi=\forall x\phi_{1}\ ,\ \psi=\forall y\psi_{1}\ ,\ x\neq y\ ,\
    \phi_{1}\sim \theta\ ,\ \psi_{1}\sim \theta[y/x]\ ,\ y\not\in\var(\theta)
    \end{eqnarray*}
while for the $\alpha$-equivalence we have:
    \begin{eqnarray*}
    (v)&&\phi=\forall x\phi_{1}\ ,\ \psi=\forall y\psi_{1}\ ,\ x\neq y\ ,\
    \psi_{1}\sim \phi_{1}[y\!:\!x]\ ,\ y\not\in\free(\phi_{1})
    \end{eqnarray*}
This is the beauty of the $\alpha$-equivalence. Not only does it
work equally well for $V$ finite and $V$ infinite or so we believe,
but it is also a lot simpler in many respects. It involves the
injective permutation $[y\!:\!x]$ rather than the non-injective
substitution $[y/x]$. It makes no reference to the existence of a
third formula $\theta$ in the case of $(v)$ but directly gives us
$\psi_{1}\sim \phi_{1}[y\!:\!x]$ instead. The proofs in the case of
the $\alpha$-equivalence are simpler than their counterparts of
the strong $\alpha$-equivalence.

\begin{defin}\label{logic:def:almost:equivalent}
Let $\sim$ be the $\alpha$-equivalence on \pv\ where $V$ is a
set. Let $\phi,\psi\in\pv$. We say that $\phi$ is {\em almost
equivalent to $\psi$} and we write $\phi\simeq\psi$, \ifand\ one of
the following is the case:
    \begin{eqnarray*}
    (i)&&\phi\in\pvo\ ,\ \psi\in\pvo\ ,\ \mbox{and}\ \phi=\psi\\
    (ii)&&\phi=\bot\ \mbox{and}\ \psi=\bot\\
    (iii)&&\phi=\phi_{1}\to\phi_{2}\ ,\ \psi=\psi_{1}\to\psi_{2}\ ,\
    \phi_{1}\sim\psi_{1}\ \mbox{and}\ \phi_{2}\sim\psi_{2}\\
    (iv)&&\phi=\forall x\phi_{1}\ ,\ \psi=\forall x\psi_{1}\ \mbox{and}\ \phi_{1}\sim\psi_{1}\\
    (v)&&\phi=\forall x\phi_{1}\ ,\ \psi=\forall y\psi_{1}\ ,\ x\neq y\ ,\
    \psi_{1}\sim \phi_{1}[y\!:\!x]\ ,\ y\not\in\free(\phi_{1})
    \end{eqnarray*}
\end{defin}


\begin{prop}
$(i),(ii),(iii),(iv), (v)$ of {\em
def.~(\ref{logic:def:almost:equivalent})} are mutually exclusive.
\end{prop}

\noindent
\begin{proof}
This is an immediate consequence of
theorem~(\ref{logic:the:unique:representation}) of
page~\pageref{logic:the:unique:representation} applied to the free
universal algebra \pv\ with free generator \pvo, where a formula
$\phi\in\pv$ is either an element of \pvo, or the contradiction
constant $\phi=\bot$, or an implication $\phi=\phi_{1}\to\phi_{2}$,
or a quantification $\phi=\forall x\phi_{1}$, but cannot be equal to
any two of those things simultaneously. Since $(v)$ can only occur
with $x\neq y$, it also follows from
theorem~(\ref{logic:the:unique:representation}) that $(v)$ cannot
occur at the same time as $(iv)$.
\end{proof}

Having defined the relation $\simeq$, we now proceed to prove the
equivalence $\phi\sim\psi\ \Leftrightarrow\ \phi\simeq\psi$. The
strategy is the same as the one adopted for the strong 
$\alpha$-equivalence. The difficult part is to show the inclusion
$\sim\,\subseteq\,\simeq$, which we achieve by showing
$R_{0}\subseteq\,\simeq$ together with the fact that $\simeq$ is a
congruence.

\begin{prop}\label{logic:prop:almost:contains:r0}
Let $\simeq$ be the almost equivalence relation on \pv\ where $V$ is
a set. Then $\simeq$ contains the generator $R_{0}$ of {\em
definition~(\ref{logic:def:sub:congruence})}.
\end{prop}

\noindent
\begin{proof}
Suppose $x,y\in V$ and $\phi_{1}\in\pv$ are such that $x\neq y$ and
$y\not\in\free(\phi_{1})$. Note that this cannot happen unless $V$
has at least two elements. We define $\phi=\forall x\phi_{1}$ and
$\psi=\forall y\,\phi_{1}[y\!:\!x]$. We need to show that
$\phi\simeq\psi$. We shall do so by proving that $(v)$ of
definition~(\ref{logic:def:almost:equivalent}) is the case. Define
$\psi_{1}=\phi_{1}[y\!:\!x]$. Then we have $\phi=\forall x\phi_{1}$,
$\psi=\forall y\psi_{1}$, $x\neq y$ and $y\not\in\free(\phi_{1})$.
So it remains to show that $\psi_{1}\sim\phi_{1}[y\!:\!x]$ which is
immediate from the reflexivity of $\sim$\,.
\end{proof}

\begin{prop}\label{logic:prop:almost:reflexive}
Let $\simeq$ be the almost equivalence relation on \pv\ where $V$ is
a set. Then $\simeq$ is a reflexive relation on \pv.
\end{prop}

\noindent
\begin{proof}
Let $\phi\in\pv$. We need to show that $\phi\simeq\phi$. From
theorem~(\ref{logic:the:unique:representation}) of
page~\pageref{logic:the:unique:representation} we know that $\phi$
is either an element of \pvo, or $\phi=\bot$ or
$\phi=\phi_{1}\to\phi_{2}$ or $\phi=\forall x\phi_{1}$ for some
$\phi_{1},\phi_{2}\in\pv$ and $x\in V$. We shall consider these four
mutually exclusive cases separately. Suppose first that
$\phi\in\pvo$: then from $\phi=\phi$ we obtain $\phi\simeq\phi$.
Suppose next that $\phi=\bot$: then it is clear that
$\phi\simeq\phi$. Suppose now that $\phi=\phi_{1}\to\phi_{2}$. Since
the $\alpha$-equivalence $\sim$ is reflexive, we have
$\phi_{1}\sim\phi_{1}$ and $\phi_{2}\sim\phi_{2}$. It follows from
$(iii)$ of definition~(\ref{logic:def:almost:equivalent}) that
$\phi\simeq\phi$. Suppose finally that $\phi=\forall x\phi_{1}$.
From $\phi_{1}\sim\phi_{1}$ and $(iv)$ of
definition~(\ref{logic:def:almost:equivalent}) we conclude that
$\phi\simeq\phi$. In all cases, we have proved that
$\phi\simeq\phi$.
\end{proof}


\begin{prop}\label{logic:prop:almost:symmetric}
Let $\simeq$ be the almost equivalence relation on \pv\ where $V$ is
a set. Then $\simeq$ is a symmetric relation on \pv.
\end{prop}

\noindent
\begin{proof}
Let $\phi,\psi\in\pv$ be such that $\phi\simeq\psi$. We need to show
that $\psi\simeq\phi$. We shall consider the five possible cases of
definition~(\ref{logic:def:almost:equivalent}): suppose first that
$\phi\in\pvo$, $\psi\in\pvo$ and $\phi=\psi$. Then it is clear that
$\psi\simeq\phi$. Suppose next that $\phi=\bot$ and $\psi=\bot$.
Then we also have $\psi\simeq\phi$. We now assume that
$\phi=\phi_{1}\to\phi_{2}$ and $\psi=\psi_{1}\to\psi_{2}$ with
$\phi_{1}\sim\psi_{1}$ and $\phi_{2}\sim\psi_{2}$. Since the 
$\alpha$-equivalence on \pv\ is symmetric, we have
$\psi_{1}\sim\phi_{1}$ and $\psi_{2}\sim\phi_{2}$. Hence we have
$\psi\simeq\phi$. We now assume that $\phi=\forall x\phi_{1}$ and
$\psi=\forall x\psi_{1}$ with $\phi_{1}\sim\psi_{1}$. Then once
again by symmetry of the $\alpha$-equivalence we have
$\psi_{1}\sim\phi_{1}$ and consequently $\psi\simeq\phi$. We finally
consider the last possible case of $\phi=\forall x\phi_{1}$,
$\psi=\forall y\psi_{1}$ with $x\neq y$,
$\psi_{1}\sim\phi_{1}[y\!:\!x]$ and $y\not\in\free(\phi_{1})$. We
need to show that $\phi_{1}\sim\psi_{1}[x\!:\!y]$ and
$x\not\in\free(\psi_{1})$. First we show that
$\phi_{1}\sim\psi_{1}[x\!:\!y]$. Note that $[x\!:\!y]$ and
$[y\!:\!x]$ are in fact the same substitutions. So we need to show
that $\phi_{1}\sim\psi_{1}[y\!:\!x]$. Since
$\psi_{1}\sim\phi_{1}[y\!:\!x]$ and $[y\!:\!x]:V\to V$ is injective,
from
proposition~(\ref{logic:prop:sub:congruence:injective:substitution})
we obtain:
    \[
    \psi_{1}[y\!:\!x]\sim\phi_{1}[y\!:\!x][y\!:\!x]
    \]
It is therefore sufficient to show that
$\phi_{1}\sim\phi_{1}[y\!:\!x][y\!:\!x]$ which follows from
proposition~(\ref{logic:prop:substitution:identity}) and the fact
that $[y\!:\!x]\circ[y\!:\!x]$ is the identity mapping. We now show
that $x\not\in\free(\psi_{1})$. From $\psi_{1}\sim\phi_{1}[y\!:\!x]$
and proposition~(\ref{logic:prop:sub:congruence:freevar}) we obtain
$\free(\psi_{1})=\free(\phi_{1}[y\!:\!x])$. So we need to show that
$x\not\in\free(\phi_{1}[y\!:\!x])$. So suppose to the contrary that
$x\in\free(\phi_{1}[y\!:\!x])$. We shall derive a contradiction.
Since $[y\!:\!x]$ is injective, from
proposition~(\ref{logic:prop:freevar:of:substitution}) we have
$\free(\phi_{1}[y\!:\!x])=[y\!:\!x](\free(\phi_{1}))$ and
consequently there exists $u\in\free(\phi_{1})$ such that
$x=[y\!:\!x](u)$. It follows that $u=y$ which contradicts the
assumption $y\not\in\free(\phi_{1})$.
\end{proof}

Showing that $\simeq$ is a transitive relation is the difficult part
of this section. The following proposition is the counterpart of
proposition~(\ref{logic:prop:almost:strong:transitive}) of the
strong $\alpha$-equivalence . Its proof is however a lot simpler.

\begin{prop}\label{logic:prop:almost:transitive}
Let $\simeq$ be the almost equivalence relation on \pv\ where $V$ is
a set. Then $\simeq$ is a transitive relation on \pv.
\end{prop}

\noindent
\begin{proof}
Let $\phi,\psi$ and $\chi\in\pv$ be such that $\phi\simeq\psi$ and
$\psi\simeq\chi$. We need to show that $\phi\simeq\chi$. We shall
consider the five possible cases of
definition~(\ref{logic:def:almost:equivalent}) in relation to
$\phi\simeq\psi$. Suppose first that $\phi,\psi\in\pvo$ and
$\phi=\psi$. Then from $\psi\simeq\chi$ we obtain $\psi,\chi\in\pvo$
and $\psi=\chi$. It follows that $\phi,\chi\in\pvo$ and $\phi=\chi$.
Hence we see that $\phi\simeq\chi$. We now assume that
$\phi=\psi=\bot$. Then from $\psi\simeq\chi$ we obtain
$\psi=\chi=\bot$. It follows that $\phi=\chi=\bot$ and consequently
$\phi\simeq\chi$. We now assume that $\phi=\phi_{1}\to\phi_{2}$ and
$\psi=\psi_{1}\to\psi_{2}$ with $\phi_{1}\sim\psi_{1}$ and
$\phi_{2}\sim\psi_{2}$. From $\psi\simeq\chi$ we obtain
$\chi=\chi_{1}\to\chi_{2}$ with $\psi_{1}\sim\chi_{1}$ and
$\psi_{2}\sim\chi_{2}$. The $\alpha$-equivalence being
transitive, it follows that $\phi_{1}\sim\chi_{1}$ and
$\phi_{2}\sim\chi_{2}$. Hence we see that $\phi\simeq\chi$. We now
assume that $\phi=\forall x\phi_{1}$ and $\psi=\forall x\psi_{1}$
with $\phi_{1}\sim\psi_{1}$, for some $x\in V$. From
$\psi\simeq\chi$ only the cases $(iv)$ and $(v)$ of
definition~(\ref{logic:def:almost:equivalent}) are possible. First
we assume that $(iv)$ is the case. Then $\chi=\forall x\chi_{1}$
with $\psi_{1}\sim\chi_{1}$. The $\alpha$-equivalence being
transitive, we obtain $\phi_{1}\sim\chi_{1}$ and consequently
$\phi\simeq\chi$. We now assume that $(v)$ is the case. Then
$\chi=\forall y\chi_{1}$ for some $y\in V$ with $x\neq y$,
$\chi_{1}\sim\psi_{1}[y\!:\!x]$ and $y\not\in\free(\psi_{1})$. In
order to prove $\phi\simeq\chi$ it is sufficient to show that
$\chi_{1}\sim\phi_{1}[y\!:\!x]$ and $y\not\in\free(\phi_{1})$. First
we show that $\chi_{1}\sim\phi_{1}[y\!:\!x]$. It is sufficient to
prove that $\phi_{1}[y\!:\!x]\sim\psi_{1}[y\!:\!x]$ which follows
from $\phi_{1}\sim\psi_{1}$, using
proposition~(\ref{logic:prop:sub:congruence:injective:substitution})
and the fact that $[y\!:\!x]:V\to V$ is injective. We now show that
$y\not\in\free(\phi_{1})$. From $\phi_{1}\sim\psi_{1}$ and
proposition~(\ref{logic:prop:sub:congruence:freevar}) we obtain
$\free(\phi_{1})=\free(\psi_{1})$. Hence it is sufficient to show
that $y\not\in\free(\psi_{1})$ which is true by assumption. It
remains to consider the last possible case of
definition~(\ref{logic:def:almost:equivalent}). So we assume that
$\phi=\forall x\phi_{1}$ and $\psi=\forall y\psi_{1}$ with $x\neq
y$, $\psi_{1}\sim\phi_{1}[y\!:\!x]$ and $y\not\in\free(\phi_{1})$.
From $\psi\simeq\chi$ only the cases $(iv)$ and $(v)$ of
definition~(\ref{logic:def:almost:equivalent}) are possible. First
we assume that $(iv)$ is the case. Then $\chi=\forall y\chi_{1}$
with $\psi_{1}\sim\chi_{1}$. The $\alpha$-equivalence being
transitive, we obtain $\chi_{1}\sim\phi_{1}[y\!:\!x]$ and
consequently $\phi\simeq\chi$. We now assume that $(v)$ is the case.
Then $\chi=\forall z\chi_{1}$ for some $z\in V$ with $y\neq z$,
$\chi_{1}\sim\psi_{1}[z\!:\!y]$ and $z\not\in\free(\psi_{1})$. We
shall now distinguish two cases. First we assume that $x=z$. Then
$\phi=\forall x\phi_{1}$ and $\chi=\forall x\chi_{1}$ and in order
to show that $\phi\simeq\chi$ it is sufficient to prove that
$\phi_{1}\sim\chi_{1}$. From $\chi_{1}\sim\psi_{1}[z\!:\!y]$ and
$z=x$ we obtain $\chi_{1}\sim\psi_{1}[y\!:\!x]$ and it is therefore
sufficient to prove that $\phi_{1}\sim\psi_{1}[y\!:\!x]$. However,
we know that $\psi_{1}\sim\phi_{1}[y\!:\!x]$ and since
$[y\!:\!x]:V\to V$ is injective, from
proposition~(\ref{logic:prop:sub:congruence:injective:substitution})
we obtain $\psi_{1}[y\!:\!x]\sim\phi_{1}[y\!:\!x][y\!:\!x]$. It is
therefore sufficient to show that
$\phi_{1}\sim\phi_{1}[y\!:\!x][y\!:\!x]$ which follows from
proposition~(\ref{logic:prop:substitution:identity}) and the fact
that $[y\!:\!x]\circ[y\!:\!x]$ is the identity mapping. This
completes our proof in the case when $x=z$. We now assume that
$x\neq z$. So we have $x\neq y$, $y\neq z$ and $x\neq z$, with
$\phi=\forall x\phi_{1}$, $\psi=\forall y\psi_{1}$ and $\chi=\forall
z\chi_{1}$. Furthermore, $\psi_{1}\sim\phi_{1}[y\!:\!x]$ and
$\chi_{1}\sim\psi_{1}[z\!:\!y]$ while we have
$y\not\in\free(\phi_{1})$ and $z\not\in\free(\psi_{1})$, and we need
to show that $\phi\simeq\chi$. So we need to prove that
$\chi_{1}\sim\phi_{1}[z\!:\!x]$ and $z\not\in\free(\phi_{1})$. First
we show that $z\not\in\free(\phi_{1})$. So suppose to the contrary
that $z\in\free(\phi_{1})$. We shall derive a contradiction. Since
$[y\!:\!x]$ is injective, from
proposition~(\ref{logic:prop:freevar:of:substitution}) we have
$\free(\phi_{1}[y\!:\!x]) = [y\!:\!x](\free(\phi_{1}))$. It follows
that $z=[y\!:\!x](z)$ is also an element of
$\free(\phi_{1}[y\!:\!x])$. However we have
$\psi_{1}\sim\phi_{1}[y\!:\!x]$ and consequently from
proposition~(\ref{logic:prop:sub:congruence:freevar}) we obtain
$\free(\psi_{1})=\free(\phi_{1}[y\!:\!x])$. Hence we see that
$z\in\free(\psi_{1})$ which is our desired contradiction. We shall
now prove that $\chi_{1}\sim\phi_{1}[z\!:\!x]$. Since
$\chi_{1}\sim\psi_{1}[z\!:\!y]$, it is sufficient to show that
$\psi_{1}[z\!:\!y]\sim\phi_{1}[z\!:\!x]$. However we know that
$\psi_{1}\sim\phi_{1}[y\!:\!x]$ and since $[z\!:\!y]:V\to V$ is
injective, from
proposition~(\ref{logic:prop:sub:congruence:injective:substitution})
we obtain $\psi_{1}[z\!:\!y]\sim\phi_{1}[y\!:\!x][z\!:\!y]$. It is
therefore sufficient to show that
$\phi_{1}[y\!:\!x][z\!:\!y]\sim\phi_{1}[z\!:\!x]$. Let us accept for
now:
    \begin{equation}\label{logic:eqn:quant:transitive:z:x}
    [z\!:\!x]=[y\!:\!x]\circ[z\!:\!y]\circ[y\!:\!x]
    \end{equation}
Then we simply need to show that
$\phi_{1}[y\!:\!x][z\!:\!y]\sim\phi_{1}[y\!:\!x][z\!:\!y][y\!:\!x]$.
Using proposition~(\ref{logic:prop:admissible:sub:congruence}), it
is therefore sufficient to prove that $[y\!:\!x]$ is an admissible
substitution for $\phi_{1}[y\!:\!x][z\!:\!y]$. Since $[y\!:\!x]:V\to
V$ is injective, from
proposition~(\ref{logic:prop:FOPL:valid:injective}) it is valid for
$\phi_{1}[y\!:\!x][z\!:\!y]$. So it remains to show that
$[y\!:\!x](u)=u$ for all $u\in\free(\phi_{1}[y\!:\!x][z\!:\!y])$. It
is therefore sufficient to prove that neither $x$ nor $y$ are
elements of $\free(\phi_{1}[y\!:\!x][z\!:\!y])$. However, since
$[z\!:\!y]\circ[y\!:\!x]:V\to V$ is injective, from
proposition~(\ref{logic:prop:freevar:of:substitution}) we have
$\free(\phi_{1}[y\!:\!x][z\!:\!y])=[z\!:\!y]\circ[y\!:\!x](\free(\phi_{1}))$.
So we need to show that $x$ and $y$ do not belong to
$[z\!:\!y]\circ[y\!:\!x](\free(\phi_{1}))$. First we do this for
$x$. Suppose $x=[z\!:\!y]\circ[y\!:\!x](u)$ for some
$u\in\free(\phi_{1})$. By injectivity we must have $u=y$,
contradicting the assumption $y\not\in\free(\phi_{1})$. We now deal
with $y$. So suppose $y=[z\!:\!y]\circ[y\!:\!x](u)$ for some
$u\in\free(\phi_{1})$. By injectivity we must have $u=z$,
contradicting the fact that $z\not\in\free(\phi_{1})$ which we have
already proven. It remains to prove that
equation~(\ref{logic:eqn:quant:transitive:z:x}) holds. So let $u\in
V$. We need:
    \[
    [z\!:\!x](u)=[y\!:\!x]\circ[z\!:\!y]\circ[y\!:\!x](u)
    \]
This is clearly the case when $u\not\in\{x,y,z\}$. The cases $u=x$,
$u=y$ and $u=z$ are easily checked.
\end{proof}

The implication $\phi\simeq\psi\ \Rightarrow\ \phi\sim\psi$ is the
simple one. We need to prove it now in order to show that $\simeq$
is a congruent relation.

\begin{prop}\label{logic:prop:almost:implies:sub:congruence}
Let $\simeq$ be the almost equivalence and $\sim$ be the 
$\alpha$-equivalence on \pv, where $V$ is a set. For all
$\phi,\psi\in\pv$:
    \[
    \phi\simeq\psi\ \Rightarrow\ \phi\sim\psi
    \]
\end{prop}

\noindent
\begin{proof}
Let $\phi,\psi\in\pv$ such that $\phi\simeq\psi$. We need to show
that $\phi\sim\psi$. We shall consider the five possible cases of
definition~(\ref{logic:def:almost:equivalent}) in relation to
$\phi\simeq\psi$. Suppose first that $\phi=\psi\in\pvo$. From the
reflexivity of the $\alpha$-equivalence, it is clear that
$\phi\sim\psi$. Suppose next that $\phi=\psi=\bot$. Then we also
have $\phi\sim\psi$. We now assume that $\phi=\phi_{1}\to\phi_{2}$
and $\psi=\psi_{1}\to\psi_{2}$ where $\phi_{1}\sim\psi_{1}$ and
$\phi_{2}\sim\psi_{2}$. The $\alpha$-equivalence being a
congruent relation on \pv, we obtain $\phi\sim\psi$. Next we assume
that $\phi=\forall x\phi_{1}$ and $\psi=\forall x\psi_{1}$ where
$\phi_{1}\sim\psi_{1}$ and $x\in\ V$. Again, the $\alpha$-equivalence 
being a congruent relation we obtain $\phi\sim\psi$.
Finally we assume that $\phi=\forall x\phi_{1}$ and $\psi=\forall
y\psi_{1}$ where $x\neq y$, $\psi_{1}\sim\phi_{1}[y\!:\!x]$ and
$y\not\in\free(\phi_{1})$. For the last time, the $\alpha$-equivalence 
being a congruent relation we obtain $\psi\sim \forall
y\, \phi_{1}[y\!:\!x]$. Hence in order to show $\phi\sim\psi$ it is
sufficient to show that $\forall x\phi_{1}\sim\forall
y\,\phi_{1}[y\!:\!x]$ which follows immediately from $x\neq y$,
$y\not\in\free(\phi_{1})$ and
definition~(\ref{logic:def:sub:congruence}).
\end{proof}

\begin{prop}\label{logic:prop:almost:congruent}
Let $\simeq$ be the almost equivalence relation on \pv\ where $V$ is
a set. Then $\simeq$ is a congruent relation on \pv.
\end{prop}

\noindent
\begin{proof}
From proposition~(\ref{logic:prop:almost:reflexive}), the almost
equivalence $\simeq$ is reflexive and so $\bot\simeq\bot$. We now
assume that $\phi=\phi_{1}\to\phi_{2}$ and
$\psi=\psi_{1}\to\psi_{2}$ where $\phi_{1}\simeq\psi_{1}$ and
$\phi_{2}\simeq\psi_{2}$. We need to show that $\phi\simeq\psi$.
However from
proposition~(\ref{logic:prop:almost:implies:sub:congruence}) we have
$\phi_{1}\sim\psi_{1}$ and $\phi_{2}\sim\psi_{2}$ and it follows
from definition~(\ref{logic:def:almost:equivalent}) that
$\phi\simeq\psi$. We now assume that $\phi=\forall x\phi_{1}$ and
$\psi=\forall x\psi_{1}$ where $\phi_{1}\simeq\psi_{1}$ and $x\in
V$. We need to show that $\phi\simeq\psi$. Once again from
proposition~(\ref{logic:prop:almost:implies:sub:congruence}) we have
$\phi_{1}\sim\psi_{1}$ and consequently from
definition~(\ref{logic:def:almost:equivalent}) we obtain
$\phi\simeq\psi$.
\end{proof}

\begin{prop}\label{logic:prop:almost:congruence}
Let $\simeq$ be the almost equivalence relation on \pv\ where $V$ is
a set. Then $\simeq$ is a congruence on \pv.
\end{prop}


\noindent
\begin{proof}
We need to show that $\simeq$ is reflexive, symmetric, transitive
and that it is a congruent relation on \pv. From
proposition~(\ref{logic:prop:almost:reflexive}), the
relation~$\simeq$ is reflexive. From
proposition~(\ref{logic:prop:almost:symmetric}) it is symmetric
while from proposition~(\ref{logic:prop:almost:transitive}) it is
transitive. Finally from
proposition~(\ref{logic:prop:almost:congruent}) the
relation~$\simeq$ is a congruent relation.
\end{proof}

So we have shown that $\simeq$ is a congruence on \pv\ which
contains the generator $R_{0}$ of the $\alpha$-equivalence. The
equality $\simeq\,=\,\sim$ follows immediately.

\begin{prop}\label{logic:prop:almost:is:sub:congruence}
Let $\simeq$ be the almost equivalence and $\sim$ be the 
$\alpha$-equivalence on \pv, where $V$ is a set. For all
$\phi,\psi\in\pv$:
    \[
    \phi\simeq\psi\ \Leftrightarrow\ \phi\sim\psi
    \]
\end{prop}

\noindent
\begin{proof}
From proposition~(\ref{logic:prop:almost:implies:sub:congruence}) it
is sufficient to show the implication $\Leftarrow$ or equivalently
the inclusion $\sim\,\subseteq\,\simeq\,$. Since $\sim$ is the 
$\alpha$-equivalence on \pv, it is the smallest congruence on
\pv\ which contains the set $R_{0}$ of
definition~(\ref{logic:def:sub:congruence}). In order to show the
inclusion $\sim\,\subseteq\,\simeq$ it is therefore sufficient to
show that $\simeq$ is a congruence on \pv\ such that
$R_{0}\subseteq\,\simeq$. The fact that it is a congruence stems
from proposition~(\ref{logic:prop:almost:congruence}). The fact that
$R_{0}\subseteq\,\simeq$ follows from
proposition~(\ref{logic:prop:almost:contains:r0}).
\end{proof}

We have no need to remember the relation $\simeq$. The equality
$\simeq\,=\,\sim$ is wrapped up in the following theorem for future
reference. This is the counterpart of
theorem~(\ref{logic:the:strong:sub:congruence:charac}) of
page~\pageref{logic:the:strong:sub:congruence:charac} of the strong $\alpha$-equivalence.

\index{congruence@Charact. of sub. congruence}
\begin{theorem}\label{logic:the:sub:congruence:charac}
Let $\sim$ be the $\alpha$-equivalence on \pv\ where $V$ is a
set. For all $\phi,\psi\in\pv$, $\phi\sim\psi$ \ifand\ one of the
following is the case:
    \begin{eqnarray*}
    (i)&&\phi\in\pvo\ ,\ \psi\in\pvo\ ,\ \mbox{and}\ \phi=\psi\\
    (ii)&&\phi=\bot\ \mbox{and}\ \psi=\bot\\
    (iii)&&\phi=\phi_{1}\to\phi_{2}\ ,\ \psi=\psi_{1}\to\psi_{2}\ ,\
    \phi_{1}\sim\psi_{1}\ \mbox{and}\ \phi_{2}\sim\psi_{2}\\
    (iv)&&\phi=\forall x\phi_{1}\ ,\ \psi=\forall x\psi_{1}\ \mbox{and}\ \phi_{1}\sim\psi_{1}\\
    (v)&&\phi=\forall x\phi_{1}\ ,\ \psi=\forall y\psi_{1}\ ,\ x\neq y\ ,\
    \psi_{1}\sim \phi_{1}[y\!:\!x]\ ,\ y\not\in\free(\phi_{1})
    \end{eqnarray*}
\end{theorem}

\noindent
\begin{proof}
Immediately follows from
proposition~(\ref{logic:prop:almost:is:sub:congruence}) and
definition~(\ref{logic:def:almost:equivalent}).
\end{proof}
