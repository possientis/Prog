In definition~(\ref{logic:def:strong:sub:congruence}) we initially
defined the strong $\alpha$-equivalence thinking it was the
appropriate notion to study. We then realized in
proposition~(\ref{logic:prop:counter:strong:1}) and
proposition~(\ref{logic:prop:counter:strong:2}) that the strong 
$\alpha$-equivalence is in fact unsatisfactory in the case when
the set of variables $V$ has two or three elements. For example,
when $V=\{x,y\}$ with $x\neq y$, we saw that the formulas
$\phi=\forall x\forall y(x\in y)$ and $\psi=\forall y\forall x(y\in
x)$ failed to be strongly $\alpha$-equivalent. It is easy to believe 
that similar failures can be uncovered whenever $V$ is a finite set. So we
decided to search for a new notion of {\em $\alpha$-equivalence}
which led to definition~(\ref{logic:def:sub:congruence}). In this
section, we shall attempt to show that this new $\alpha$-equivalence 
has all the advantages of the strong $\alpha$-equivalence, 
but without its~flaws.

First we look at the paradox of
proposition~(\ref{logic:prop:counter:strong:1}). So let $V=\{x,y\}$
with $x\neq y$ and $\phi=\forall x\forall y(x\in y)$ with
$\psi=\forall y\forall x(y\in x)$. Then setting $\phi_{1}=\forall
y(x\in y)$ we have $\phi=\forall x\phi_{1}$ and $\psi=\forall
y\,\phi_{1}[y\!:\!x]$. Since $x\neq y$ and $y\not\in\free(\phi_{1})$
we conclude from definition~(\ref{logic:def:sub:congruence}) that
$\phi\sim\psi$ where $\sim$ denotes the $\alpha$-equivalence. So
the paradox of proposition~(\ref{logic:prop:counter:strong:1}) is
lifted. Suppose now that $V=\{x,y,z\}$ where $x$, $y$ and $z$ are
distinct, and define $\phi= \forall x\forall y\forall z\,[(x\in
y)\to(y\in z)]$ and $\psi=\forall y\forall z\forall x\,[(y\in
z)\to(z\in x)]$. We claim that $\phi\sim\psi$. If this is the case,
then the paradox of proposition~(\ref{logic:prop:counter:strong:2})
is also lifted. Let $\chi$ be the formula defined by $\chi=\forall
y\forall x\forall z\,[(y\in x)\to(x\in z)]$. Then $\chi$ is simply
the formula obtained from $\phi$ by permuting the variables $x$ and
$y$. In order to show $\phi\sim\psi$ it is sufficient to prove that
$\phi\sim\chi$ and $\chi\sim\psi$. First we show that
$\phi\sim\chi$. Defining $\phi_{1}=\forall y\forall z\,[(x\in
y)\to(y\in z)]$ we have $\phi=\forall x\phi_{1}$ and $\chi=\forall
y\phi_{1}[y\!:\!x]$. Since $x\neq y$ and $y\not\in\free(\phi_{1})$
we conclude from definition~(\ref{logic:def:sub:congruence}) that
$\phi\sim\chi$. We now show that $\chi\sim\psi$. Setting
$\theta=\forall x\forall z\,[(y\in x)\to(x\in z)]$ together with
$\theta^{*}=\forall z\forall x\,[(y\in z)\to(z\in x)]$ we have
$\chi=\forall y\theta$ and $\psi=\forall y\theta^{*}$. It is
therefore sufficient to show that $\theta\sim\theta^{*}$. Defining
$\theta_{1}=\forall z\,[(y\in x)\to(x\in z)]$ we obtain
$\theta=\forall x\theta_{1}$ and $\theta^{*}=\forall
z\theta_{1}[z\!:\!x]$. Since $x\neq z$ and
$z\not\in\free(\theta_{1})$ we conclude from
definition~(\ref{logic:def:sub:congruence}) that
$\theta\sim\theta^{*}$. So we have proved that the $\alpha$-equivalence 
is not subject to the paradoxes of
proposition~(\ref{logic:prop:counter:strong:1}) and
proposition~(\ref{logic:prop:counter:strong:2}).


Having reviewed the known shortcomings of the strong 
$\alpha$-equivalence, we shall now establish that the new $\alpha$-equivalence 
is in fact pretty much the same notion as the old.
Specifically, we shall see that both congruences coincide when $V$
is an infinite set. Furthermore, even when $V$ is a finite set we
shall see that two formulas $\phi$ and $\psi$ which are just
$\alpha$-equivalent, are in fact strongly $\alpha$-equivalent 
provided $\phi$ and $\psi$ do not use up all the variables in $V$, i.e. 
$\var(\phi)\neq V$ or $\var(\psi)\neq V$.
First we check that strong $\alpha$-equivalence is indeed a
relation which is {\em stronger} than $\alpha$-equivalence:

\begin{prop}\label{logic:prop:strong:implies:quant}
Let $\sim$ be the $\alpha$-equivalence and $\simeq$ be the strong
$\alpha$-equivalence on \pv\ where $V$ is a set. Then for all
$\phi,\psi\in\pv$:
    \[
    \phi\simeq\psi\ \Rightarrow\ \phi\sim\psi
    \]
\end{prop}

\noindent
\begin{proof}
We need to show the inclusion $\simeq\,\subseteq\,\sim$\,. However,
using definition~(\ref{logic:def:strong:sub:congruence}) the strong 
$\alpha$-equivalence $\simeq$ is generated by the following set:
    \[
    R_{0}=\left\{\,(\,\forall x\phi_{1}\,,\,\forall
    y\,\phi_{1}[y/x]\,):\phi_{1}\in\pv\ ,\ x,y\in V\ ,\ x\neq y\ ,\
    y\not\in\var(\phi_{1})\,\right\}
    \]
while from definition~(\ref{logic:def:sub:congruence}) the
congruence $\sim$ is generated by the set:
     \[
    R_{1}=\left\{\,(\,\forall x\phi_{1}\,,\,\forall y\,\phi_{1}[y\!:\!x]\,):
    \phi_{1}\in\pv\ ,\ x,y\in V\ ,\ x\neq y\ ,\ y\not\in\free(\phi_{1})\,\right\}
    \]
So it is sufficient to show that $R_{0}\subseteq R_{1}$ which
follows from the fact that $\phi_{1}[y/x]=\phi_{1}[y\!:\!x]$
whenever $y\not\in\var(\phi_{1})$ as can be seen from
proposition~(\ref{logic:prop:permutation:is:substitution}).
\end{proof}


\begin{prop}\label{logic:prop:link:strong:sub:congruence}
Let $\simeq$ be the strong $\alpha$-equivalence on \pv\ where $V$
is a set. Let $\phi\in\pv$ such that $y\not\in\free(\phi)$ and
$\var(\phi)\neq V$. Then we have:
    \[
    \forall x\,\phi\simeq\forall y\,\phi[y\!:\!x]
    \]
\end{prop}

\noindent
\begin{proof}
Suppose for now that $y\not\in\var(\phi)$. From
proposition~(\ref{logic:prop:permutation:is:substitution}) we have
the equality $\phi[y\!:\!x]=\phi[y/x]$ and the strong equivalence
$\forall x\,\phi\simeq\forall y\,\phi[y\!:\!x]$ is therefore a of
consequence $y\not\in\var(\phi)$ and
definition~(\ref{logic:def:strong:sub:congruence}). So the
conclusion of the proposition follows easily with the simple
assumption $y\not\in\var(\phi)$. Thus we may assume that
$y\in\var(\phi)$ from now on. Suppose now that $\psi\in\pv$ is such
that $y\not\in\var(\psi)$ and $\phi\simeq\psi$. Then for the reasons
just indicated we obtain $\forall x\,\psi\simeq\forall
y\,\psi[y\!:\!x]$. Furthermore, the strong$\alpha$-equivalence  
being a congruent relation on \pv, from $\phi\simeq\psi$ we have
$\forall x\,\phi\simeq\forall x\,\psi$ and $\forall
y\,\phi[y\!:\!x]\simeq\forall y\,\psi[y\!:\!x]$, where we have used
the fact that $\phi[y\!:\!x]\simeq\psi[y\!:\!x]$ which itself
follows from the injectivity of $[y\!:\!x]:V\to V$ and
proposition~(\ref{logic:prop:strong:injective:substitution}). By
transitivity it follows that $\forall x\,\phi\simeq\forall
y\,\phi[y\!:\!x]$ and the proposition is proved. Thus, it is
sufficient to show the existence of $\psi\in\pv$ with
$y\not\in\var(\psi)$ and $\phi\simeq\psi$. Having assumed the
existence of $z\in V$ such that $z\not\in\var(\phi)$, let
$\psi\in\pv$ be defined as $\psi=\phi[z/y]$. Note in particular that
$z\neq y$ since we have assumed $y\in\var(\phi)$. The fact that
$y\not\in\var(\psi)$ follows immediately from $z\neq y$ and
proposition~(\ref{logic:prop:inplaceof:notvar}). So it remains to
show that $\phi\simeq\psi$ or equivalently that
$\phi\simeq\phi[z/y]$ which is a consequence of
proposition~(\ref{logic:prop:substitution:invariant}) and the facts
that $z\not\in\var(\phi)$ and $y\not\in\free(\phi)$.
\end{proof}

\begin{prop}\label{logic:prop:FOPL:subcongvstrong:reverse}
Let $\sim$ be the $\alpha$-equivalence and $\simeq$ be the strong 
$\alpha$-equivalence on \pv\ where $V$ is a set. Let
$\phi,\psi\in\pv$ be such that $\var(\phi)\neq V$. Then we have the
equivalence:
    \[
    \phi\simeq\psi\ \Leftrightarrow\
    \phi\sim\psi
    \]
\end{prop}

\noindent
\begin{proof}
The implication $\Rightarrow$ follows from
proposition~(\ref{logic:prop:strong:implies:quant}). So we need to
prove $\Leftarrow$\,. Specifically, given $\phi\in\pv$, we need to
show that $\phi$ satisfies the property:
    \[
    \var(\phi)\neq V\ \Rightarrow\ \forall\psi[\,\phi\sim\psi\
    \Rightarrow\ \phi\simeq\psi\,]
    \]
We shall do so by a structural induction argument, using
theorem~(\ref{logic:the:proof:induction}) of
page~\pageref{logic:the:proof:induction}. First we assume that
$\phi=(x\in y)$ for some $x,y\in V$. As we shall see the condition
$\var(\phi)\neq V$ will not be used here. So let $\psi\in\pv$ such
that $\phi\sim\psi$. It is sufficient to prove that
$\phi\simeq\psi$. In fact it is sufficient to show that $\phi=\psi$
which follows immediately from $\phi=(x\in y)$ and $\phi\sim\psi$,
using theorem~(\ref{logic:the:sub:congruence:charac}) of
page~\pageref{logic:the:sub:congruence:charac}. So we now assume
that $\phi=\bot$. Then again from
theorem~(\ref{logic:the:sub:congruence:charac}) the condition
$\phi\sim\psi$ implies that $\phi=\psi$ and we are done. So we now
assume that $\phi=\phi_{1}\to\phi_{2}$ where
$\phi_{1},\phi_{2}\in\pv$ satisfy our property. We need to show the
same is true of $\phi$. So we assume that $\var(\phi)\neq V$ and we
consider $\psi\in\pv$ such that $\phi\sim\psi$. We need to show that
$\phi\simeq\psi$. However, using
theorem~(\ref{logic:the:sub:congruence:charac}) once more the
condition $\phi\sim\psi$ implies that $\psi$ must be of the form
$\psi=\psi_{1}\to\psi_{2}$ where $\phi_{1}\sim\psi_{1}$ and
$\phi_{2}\sim\psi_{2}$. Hence the strong $\alpha$-equivalence $\simeq$ being
a congruent relation, in order to show $\phi\simeq\psi$ it is
sufficient to prove that $\phi_{1}\simeq\psi_{1}$ and
$\phi_{2}\simeq\psi_{2}$. Having assumed our induction property is
true for $\phi_{1}$, it follows from
$\var(\phi_{1})\subseteq\var(\phi)$ that $\var(\phi_{1})\neq V$ and
$\phi_{1}\simeq\psi_{1}$ is therefore an immediate consequence of
$\phi_{1}\sim\psi_{1}$. We prove similarly that
$\phi_{2}\simeq\psi_{2}$ which completes the case when
$\phi=\phi_{1}\to\phi_{2}$. We now assume that $\phi=\forall
x\phi_{1}$ where $x\in V$ and $\phi_{1}\in\pv$ satisfies our
property. We need to show the same is true for $\phi$. So we assume
that $\var(\phi)\neq V$ and consider $\psi\in\pv$ such that
$\phi\sim\psi$. We need to show that $\phi\simeq\psi$. From
theorem~(\ref{logic:the:sub:congruence:charac}), the condition
$\phi\sim\psi$ leads to two possible cases: $\psi$ must be of the
form $\psi=\forall x\psi_{1}$ with $\phi_{1}\sim\psi_{1}$, or it
must be of the form $\psi=\forall y\psi_{1}$ with $x\neq y$,
$\psi_{1}\sim\phi_{1}[y\!:\!x]$ and $y\not\in\free(\phi_{1})$. First
we assume that $\psi=\forall x\psi_{1}$ which
$\phi_{1}\sim\psi_{1}$. Then in order to show that $\phi\simeq\psi$
it is sufficient to prove that $\phi_{1}\simeq\psi_{1}$. Having
assumed our induction property is true for $\phi_{1}$, it follows
from $\var(\phi_{1})\subseteq\var(\phi)$ that $\var(\phi_{1})\neq V$
and $\phi_{1}\simeq\psi_{1}$ is therefore an immediate consequence
of $\phi_{1}\sim\psi_{1}$. So we now consider the second case when
$\psi=\forall y\psi_{1}$ with $x\neq y$,
$\psi_{1}\sim\phi_{1}[y\!:\!x]$ and $y\not\in\free(\phi_{1})$. We
need to show that $\phi\simeq\psi$. However, using
proposition~(\ref{logic:prop:sub:congruence:injective:substitution})
and the fact that $[y\!:\!x]:V\to V$ is an injective map such that
$[y\!:\!x]\circ[y\!:\!x]$ is the identity mapping, we obtain
immediately $\psi_{1}[y\!:\!x]\sim\phi_{1}$. Having assumed our
property is true for $\phi_{1}$, from $\var(\phi_{1})\neq V$ we
obtain $\psi_{1}[y\!:\!x]\simeq\phi_{1}$. Composing once again on
both side by $[y\!:\!x]$ we now argue from
proposition~(\ref{logic:prop:strong:injective:substitution}) that
$\psi_{1}\simeq\phi_{1}[y\!:\!x]$. Thus in order to show that
$\phi\simeq\psi$ it is sufficient to prove that $\forall
x\phi_{1}\simeq\forall y\phi_{1}[y\!:\!x]$ which follows immediately
from proposition~(\ref{logic:prop:link:strong:sub:congruence}),
$y\not\in\free(\phi_{1})$ and $\var(\phi_{1})\neq V$.
\end{proof}

So it is now clear that the $\alpha$-equivalence and 
strong $\alpha$-equivalence are pretty much the same thing. We just need
to bear in mind that an $\alpha$-equivalence $\phi\sim\psi$ may fail to imply
a strong equivalence $\phi\simeq\psi$ in the case when all variables
of $V$ have been used up by both $\phi$ and $\psi$. We shall
conclude this section by providing counterparts to
proposition~(\ref{logic:prop:admissible:sub:congruence}) and
proposition~(\ref{logic:prop:sub:congruence:from:admissible}) for
the strong $\alpha$-equivalence.

\begin{prop}\label{logic:prop:admissible:strong}
Let $\simeq$ be the strong $\alpha$-equivalence on \pv\ where $V$
is a set. Let $\phi\in\pv$ and $\sigma:V\to V$ be an admissible
substitution for $\phi$ such that $\var(\sigma(\phi))\neq V$. Then,
we have:
    \[
    \phi\simeq\sigma(\phi)
    \]
\end{prop}

\noindent
\begin{proof}
Let $\sim$ be the $\alpha$-equivalence on \pv. Having assumed
$\sigma$ is admissible for $\phi$, using
proposition~(\ref{logic:prop:admissible:sub:congruence}) we obtain
$\phi\sim\sigma(\phi)$. Since $\var(\sigma(\phi))\neq V$ we conclude
from proposition~(\ref{logic:prop:FOPL:subcongvstrong:reverse}) that
$\phi\simeq\sigma(\phi)$.
\end{proof}

Proposition~(\ref{logic:prop:admissible:strong}) allows us to state
another characterization of the strong $\alpha$-equivalence as
the following proposition shows. In
definition~(\ref{logic:def:strong:sub:congruence}), the strong 
$\alpha$-equivalence was defined in terms of a generator:
 \[
    R_{0}=\{(\phi,\sigma(\phi))\ :\ \phi=\forall x\phi_{1}\ ,\
    \sigma=[y/x]\ ,x\neq y\ , \ y\not\in\var(\phi_{1})\}
 \]
As we shall soon discover, this generator is in fact a subset of the
set $R_{1}$ of ordered pairs $(\phi,\sigma(\phi))$ where $\sigma$ is
admissible for $\phi$ and such that $\var(\sigma(\phi))\neq V$.
Since we now know from
proposition~(\ref{logic:prop:admissible:strong}) that $R_{1}$ is
itself a subset of the strong $\alpha$-equivalence, it follows
that $R_{1}$ is also a generator of the strong $\alpha$-equivalence, 
which we shall now prove formally:

\begin{prop}\label{logic:prop:strong:quant:congruence:from:admissible}
Let $V$ be a set. Then the strong $\alpha$-equivalence on \pv\ is
also generated by the following set $R_{1}\subseteq \pv\times\pv$:
    \[
    R_{1}=\left\{\,(\,\phi\,,\,\sigma(\phi)\,):\phi\in\pv\ ,\
    \mbox{$\sigma:V\to V$ admissible for $\phi$}\ ,\
    \var(\sigma(\phi))\neq V\ \right\}
    \]
\end{prop}

\noindent
\begin{proof}
Let $\simeq$ denote the strong $\alpha$-equivalence on \pv\ and
$\equiv$ be the congruence on \pv\ generated by $R_{1}$. We need to
show that $\simeq\,=\,\equiv\,$. First we show that
$\simeq\,\subseteq\,\equiv\,$. Since $\simeq$ is the smallest
congruence on \pv\ which contains the set $R_{0}$ of
definition~(\ref{logic:def:strong:sub:congruence}), in order to
prove $\simeq\,\subseteq\,\equiv$ it is sufficient to prove that
$R_{0}\subseteq\,\equiv\,$. So let $\phi_{1}\in\pv$ and $x,y\in V$
be such that $x\neq y$ and $y\not\in\var(\phi_{1})$. Define
$\phi=\forall x\phi_{1}$ and $\psi=\forall y\,\phi_{1}[y/x]$. We
need to show that $\phi\equiv\psi$. The congruence $\equiv$ being
generated by $R_{1}$ it is sufficient to prove that $(\phi,\psi)\in
R_{1}$. However, if we define $\sigma:V\to V$ by setting
$\sigma=[y/x]$ we have:
    \[
    \psi=\forall
    y\,\phi_{1}[y/x]=\forall\sigma(x)\,\sigma(\phi_{1})=\sigma(\forall
    x\phi_{1})=\sigma(\phi)
    \]
Hence, in order to show $(\phi,\psi)\in R_{1}$, it is sufficient to
prove that $\sigma$ is an admissible substitution for $\phi$ and
furthermore that $\var(\sigma(\phi))\neq V$. The fact that
$\var(\sigma(\phi))\neq V$ follows immediately from
$x\not\in\var(\phi[y/x])$, which is itself a consequence of $x\neq
y$ and proposition~(\ref{logic:prop:inplaceof:notvar}). So we need
to show that $\sigma$ is an admissible substitution for $\phi$.
First we show that $\sigma$ is valid for $\phi=[y/x]$. This follows
immediately from
proposition~(\ref{logic:prop:FOPL:validsub:singlevar}) and the fact
that $y$ is not an element of $\var(\phi)=\var(\phi_{1})\cup\{x\}$.
Next we show that $\sigma(u)=u$ for all $u\in\free(\phi)$. So let
$u\in\free(\phi)=\free(\phi_{1})\setminus\{x\}$. Then in particular
$u\neq x$ and consequently $\sigma(u)=[y/x](u)=u$. This completes
our proof of $\simeq\,\subseteq\,\equiv\,$. We now show that
$\equiv\,\subseteq\,\simeq\,$. Since $\equiv$ is the smallest
congruence on \pv\ which contains the set $R_{1}$, it is sufficient
to show that $R_{1}\,\subseteq\,\simeq\,$. So let $\phi\in\pv$ and
$\sigma:V\to V$ be an admissible substitution for $\phi$ such that
$\var(\sigma(\phi))\neq V$. We need to show that
$\phi\simeq\sigma(\phi)$. But this follows immediately from
proposition~(\ref{logic:prop:admissible:strong}).
\end{proof}

From
proposition~(\ref{logic:prop:strong:quant:congruence:from:admissible})
the strong $\alpha$-equivalence is generated by the set of
ordered pairs $(\phi,\sigma(\phi))$ where $\sigma$ is admissible for
$\phi$ and such that $\var(\sigma(\phi))\neq V$. This is also a flaw
of the strong $\alpha$-equivalence as we feel the condition
$\var(\sigma(\phi))\neq V$ should not be there. In contrast, as can
be seen from
proposition~(\ref{logic:prop:sub:congruence:from:admissible}), the 
$\alpha$-equivalence does not suffer from this condition, as it
is simply generated by the set of ordered pairs
$(\phi,\sigma(\phi))$ where $\sigma$ is admissible for $\phi$.
Whichever way we look at it, it is clear that the $\alpha$-equivalence 
is a better notion to look at than the strong $\alpha$-equivalence. 
In many respects, it is also a lot simpler. Before we close this section, 
we may go back to a question we raised before defining the 
strong $\alpha$-equivalence on
page~\pageref{logic:def:strong:sub:congruence}. We decided to define
the strong $\alpha$-equivalence terms of ordered pairs $(\forall
x\phi_{1},\forall y\phi_{1}[y/x])$ where $x\neq y$ and
$y\not\in\var(\phi_{1})$. What if we had opted for the condition
'{\em $[y/x]$ valid for $\forall x\phi_{1}$}' rather than
'$y\not\in\var(\phi_{1})$'? It is now clear from
proposition~(\ref{logic:prop:admissible:strong}) that this would
have yielded a congruence identical to the strong $\alpha$-equivalence. 
So we did well to keep it simple.
