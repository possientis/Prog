Strong $\alpha$-equivalence is designed in such a way that if
you take a formula $\phi$ and replace its bound variables by other
variables, you obtain a formula $\psi$ which is equivalent to
$\phi$, provided the variable substitution is valid. Strong
$\alpha$-equivalence is also designed to be the smallest
congruence on \pv\ with such property. So one would expect two
equivalent formulas to be identical in all respects, except possibly
in relation to their bound variables. In particular, if a formula
$\phi$ is equivalent to a formula $\psi$, we should expect both
formulas to have the same free variables. This is indeed the case,
as the following proposition shows. Since we already know from
proposition~(\ref{logic:prop:congruence:freevar}) that
$\free(\phi)=\free(\psi)$ defines a congruence on \pv, the proof
reduces to checking every ordered pair $(\phi,\psi)$
belonging to the generator $R_{0}$ of the strong $\alpha$-equivalence, 
is such that $\free(\phi)=\free(\psi)$.
\begin{prop}\label{logic:prop:strong:freevar}
Let $\sim$ be the strong $\alpha$-equivalence on \pv\ where $V$
is a set. Then for all $\phi,\psi\in\pv$ we have the implication:
    \[
    \phi\sim\psi\ \Rightarrow\ \free(\phi)=\free(\psi)
    \]
\end{prop}
\begin{proof}
Let $\equiv$ be the relation on \pv\ defined by $\phi\equiv\psi\
\Leftrightarrow\ \free(\phi)=\free(\psi)$. We need to show that
$\phi\sim\psi\ \Rightarrow\ \phi\equiv\psi$ or equivalently that the
inclusion $\sim\,\subseteq\,\equiv$ holds. Since $\sim$ is the
strong $\alpha$-equivalence on \pv, it is the smallest congruence
on \pv\ which contains the set $R_{0}$ of
definition~(\ref{logic:def:strong:sub:congruence}). In order to show
the inclusion $\sim\,\subseteq\,\equiv$ it is therefore sufficient
to show that $\equiv$ is a congruence on \pv\ such that
$R_{0}\subseteq\,\equiv$. However, we already know from
proposition~(\ref{logic:prop:congruence:freevar}) that $\equiv$ is a
congruence on \pv. So it remains to show that
$R_{0}\subseteq\,\equiv$. So let $\phi_{1}\in\pv$ and $x,y\in V$ be
such that $x\neq y$ and $y\not\in\var(\phi_{1})$. Define the two
formulas $\phi=\forall x\phi_{1}$ and $\psi=\forall
y\,\phi_{1}[y/x]$. We need to show that $\phi\equiv\psi$ or
equivalently that $\free(\phi)=\free(\psi)$. We shall distinguish
two cases. First we assume that $x\not\in\free(\phi_{1})$. From the
assumption $y\not\in\var(\phi_{1})$ and
proposition~(\ref{logic:prop:freevar:single:subst}) we obtain
$\free(\phi_{1}[y/x])=\free(\phi_{1})$. Furthermore, since
$\free(\phi_{1})\subseteq\var(\phi_{1})$, we also have
$y\not\in\free(\phi_{1})$. It follows that:
    \begin{eqnarray*}
    \free(\psi)&=&\free(\forall y\,\phi_{1}[y/x])\\
    &=&\free(\phi_{1}[y/x])\setminus\{y\}\\
    &=&\free(\phi_{1})\setminus\{y\}\\
    &=&\free(\phi_{1})\\
    &=&\free(\phi_{1})\setminus\{x\}\\
    &=&\free(\forall x\phi_{1})\\
    &=&\free(\phi)
    \end{eqnarray*}
We now consider the case when $x\in\free(\phi_{1})$. From the
assumption $y\not\in\var(\phi_{1})$ and
proposition~(\ref{logic:prop:freevar:single:subst}) we obtain
$\free(\phi_{1}[y/x])=\free(\phi_{1})\setminus\{x\}\cup\{y\}$, and
so:
\begin{eqnarray*}
    \free(\psi)&=&\free(\forall y\,\phi_{1}[y/x])\\
    &=&\free(\phi_{1}[y/x])\setminus\{y\}\\
    &=&(\,\free(\phi_{1})\setminus\{x\}\cup\{y\}\,)\setminus\{y\}\\
    &=&(\,\free(\phi_{1})\setminus\{x\}\,)\setminus\{y\}\\
    &=&\free(\phi_{1})\setminus\{x\}\\
    &=&\free(\forall x\phi_{1})\\
    &=&\free(\phi)
    \end{eqnarray*}
\end{proof}
