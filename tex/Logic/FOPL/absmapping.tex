In definition~(\ref{logic:def:FOPL:mintransform:transform}) we
introduced the minimal transform ${\cal M}:\pv\to\pvb$ which allowed
us to characterize an $\alpha$-equivalence $\phi\sim\psi$ with a
simple equality ${\cal M}(\phi)={\cal M}(\psi)$, as follows from
theorem~(\ref{logic:the:FOPL:mintransfsubcong:kernel}) of
page~\pageref{logic:the:FOPL:mintransfsubcong:kernel}. We would like
to achieve a similar result for the absorption congruence. As we
shall see, things are a lot simpler in this case. We shall define a
mapping ${\cal A}:\pv\to\pv$ which effectively removes {\em
pointless quantifications} from a formula. We shall then prove that
the absorption equivalence $\phi\sim\psi$ can be reduced to the
equality ${\cal A}(\phi)={\cal A}(\psi)$. This will be done in
proposition~(\ref{logic:prop:FOPL:absmapping:kernel}) below. In
general given a congruence $\sim$ on \pv, we believe there is
enormous benefit in reducing the equivalence $\phi\sim\psi$ to an
equality $f(\phi)=f(\psi)$ for some map $f:\pv\to A$ where $A$ is an
arbitrary set. This can always be done of course, as $\phi\sim\psi$
is equivalent to an equality $[\phi]=[\psi]$ between equivalence
classes. However, if we can find a map $f$ which is {\em computable}
in some sense with a codomain $A$ in which the equality is {\em
decidable}, this will ensure that the congruence $\sim$ is itself
{\em decidable}. Granted we have no idea what {\em computable} and
{\em decidable} mean at this stage, but at some point we shall want
to know. This is for later.

\index{absorption@Absorption mapping}\index{a@${\cal A}$ :
absorption mapping}
\begin{defin}\label{logic:def:FOPL:absmapping:mapping}
Let $V$ be a set. We call {\em absorption mapping on \pv\ }the map
${\cal A}:\pv\to\pv$ defined by the following recursive equation:
    \[
    \forall\phi\in\pv\ ,\ {\cal A}(\phi)=\left\{
                    \begin{array}{lcl}
                    (x\in y)&\mbox{\ if\ }&\phi=(x\in y)\\
                    \bot&\mbox{\ if\ }&\phi=\bot\\
                    {\cal A}(\phi_{1})\to{\cal A}(\phi_{2})&\mbox{\ if\ }&
                    \phi=\phi_{1}\to\phi_{2}\\
                    \forall x{\cal A}(\phi_{1})&\mbox{\ if\ }&\phi=\forall x\phi_{1},\ x\in\free(\phi_{1})\\
                    {\cal A}(\phi_{1})&\mbox{\ if\ }&\phi=\forall x\phi_{1},\ x\not\in\free(\phi_{1})\\
                    \end{array}\right.
    \]
\end{defin}
\begin{prop}\label{logic:prop:FOPL:absmapping:recursion}
The structural recursion of {\em
definition~(\ref{logic:def:FOPL:absmapping:mapping})} is legitimate.
\end{prop}
\begin{proof}
We have to be slightly careful in this case. As usual, we have a
choice between theorem~(\ref{logic:the:structural:recursion}) of
page~\pageref{logic:the:structural:recursion} and
theorem~(\ref{logic:the:structural:recursion:2}) of
page~\pageref{logic:the:structural:recursion:2}. Given $\phi=\forall
x\phi_{1}$, looking at
definition~(\ref{logic:def:FOPL:absmapping:mapping}) it seems that
${\cal A}(\phi)$ is not simply a function of ${\cal A}(\phi_{1})$
but also involves a dependence in $\phi_{1}$. So it would seem that
theorem~(\ref{logic:the:structural:recursion:2}) has to be used in
this case. Of course, it is not difficult to see that an equivalent
definition could be obtained by replacing $\free(\phi_{1})$ by
$\free({\cal A}(\phi_{1}))$, allowing us to use
theorem~(\ref{logic:the:structural:recursion}) instead. However, we
have no need to worry about that. So let us use
theorem~(\ref{logic:the:structural:recursion:2}) with $X=\pv$,
$X_{0}=\pvo$ and $A=\pv$. We define $g_{0}:X_{0}\to A$ by setting
$g_{0}(x\in y)=(x\in y)$ which ensures the first condition of
definition~(\ref{logic:def:FOPL:absmapping:mapping}) is met. Next,
we define $h(\bot):A^{0}\times X^{0}\to A$ by setting
$h(\bot)(0,0)=\bot$ which gives us the second condition. Next, we
define $h(\to):A^{2}\times X^{2}\to A$ be setting $h(a,\phi)=a(0)\to
a(1)$. Finally, we define $h(\forall x):A^{1}\times X^{1}\to A$ be
setting $h(\forall x)(a,\phi)=\forall x a(0)$ if
$x\in\free(\phi(0))$ and $h(\forall x)(a,\phi)=a(0)$ otherwise. So
let us check that conditions $3$, $4$ and $5$ are met. If
$\phi=\phi_{1}\to\phi_{2}$ we have the following equalities:
    \begin{eqnarray*}
    {\cal A}(\phi)&=&{\cal A}(\phi_{1}\to\phi_{2})\\
    f=\to,\ \phi=(\phi_{1},\phi_{2})\ \rightarrow
    &=&{\cal A}(f(\phi))\\
    {\cal A}:X^{2}\to A^{2}\ \rightarrow
    &=&h(f)({\cal A}(\phi),\phi)\\
    &=&h(\to)({\cal A}(\phi),\phi)\\
    &=&{\cal A}(\phi)(0)\to{\cal A}(\phi)(1)\\
    {\cal A}:X\to A\ \rightarrow
    &=&{\cal A}(\phi(0))\to{\cal A}(\phi(1))\\
    &=&{\cal A}(\phi_{1})\to{\cal A}(\phi_{2})\\
    \end{eqnarray*}
If $\phi=\forall x\phi_{1}$ with $x\in\free(\phi_{1})$, then we have
the following equalities:
    \begin{eqnarray*}
    {\cal A}(\phi)&=&{\cal A}(\forall x\phi_{1})\\
    f=\forall x,\ \phi(0)=\phi_{1}\ \rightarrow
    &=&{\cal A}(f(\phi))\\
    {\cal A}:X^{1}\to A^{1}\ \rightarrow
    &=&h(f)({\cal A}(\phi),\phi)\\
    &=&h(\forall x)({\cal A}(\phi),\phi)\\
    x\in\free(\phi(0))\ \rightarrow
    &=&\forall x{\cal A}(\phi)(0)\\
    {\cal A}:X\to A\ \rightarrow
    &=&\forall x{\cal A}(\phi(0))\\
    &=&\forall x{\cal A}(\phi_{1})\\
    \end{eqnarray*}
The case $x\not\in\free(\phi_{1})$ is dealt with in a similar way.
\end{proof}

In proposition~(\ref{logic:prop:FOPL:mintransform:eqivalence}) we
showed that ${\cal M}(\phi)\sim i(\phi)$ where $\sim$ is the
substitution congruence and $i:V\to\bar{V}$ is the inclusion map. In
effect, this shows that the minimal transform preserves the
equivalence class of $\phi$. Of course, this is not quite true as
the minimal transform takes values in \pvb\ and not \pv. So the
equivalence class of $\phi$ is preserved only after we embed $\phi$
in \pvb\ as $i(\phi)$. In the case of the absorption congruence
things are simpler: since the absorption mapping takes values in
\pv\ itself, we are able to write equivalence ${\cal
A}(\phi)\sim\phi$.

\begin{prop}\label{logic:prop:FOPL:absmapping:equivalence:A:phi}
Let $V$ be a set and ${\cal A}:\pv\to\pv$ be the absorption mapping.
Let $\sim$ be the absorption congruence on \pv. Then given
$\phi\in\pv$:
    \[
    {\cal A}(\phi)\sim\phi
    \]
\end{prop}
\begin{proof}
We shall prove this equivalence with a structural induction, using
theorem~(\ref{logic:the:proof:induction}) of
page~\pageref{logic:the:proof:induction}. The equivalence is clear
in the case when $\phi=(x\in y)$ and $\phi=\bot$. So we assume that
$\phi=\phi_{1}\to\phi_{2}$ where $\phi_{1},\phi_{2}\in\pv$ satisfy
our equivalence:
    \begin{eqnarray*}
    {\cal A}(\phi)&=&{\cal A}(\phi_{1}\to\phi_{2})\\
    &=&{\cal A}(\phi_{1})\to{\cal A}(\phi_{2})\\
    &\sim&\phi_{1}\to\phi_{2}\\
    &=&\phi
    \end{eqnarray*}
Next we assume that $\phi=\forall x\phi_{1}$ for some $x\in V$ and
$\phi_{1}\in\pv$ satisfying our equivalence. We shall distinguish
two cases. First we assume that $x\in\free(\phi_{1})$:
    \begin{eqnarray*}
    {\cal A}(\phi)&=&{\cal A}(\forall x\phi_{1})\\
    x\in\free(\phi_{1})\ \rightarrow
    &=&\forall x{\cal A}(\phi_{1})\\
    &\sim&\forall x\phi_{1}\\
    &=&\phi
    \end{eqnarray*}
Next we assume that $x\not\in\free(\phi_{1})$. Then we have:
\begin{eqnarray*}
    {\cal A}(\phi)&=&{\cal A}(\forall x\phi_{1})\\
    x\not\in\free(\phi_{1})\ \rightarrow
    &=&{\cal A}(\phi_{1})\\
    &\sim&\phi_{1}\\
    x\not\in\free(\phi_{1})\ \rightarrow
    &\sim&\forall x\phi_{1}\\
    &=&\phi
    \end{eqnarray*}
\end{proof}


Before we prove that the equivalence $\phi\sim\psi$ can be reduced
to the quality ${\cal A}(\phi)={\cal A}(\psi)$ we shall need to show
that this equality defines a congruence:


\begin{prop}\label{logic:prop:FOPL:absmapping:congruence}
Let $V$ be a set and $\equiv$ be the relation on \pv\ defined by:
    \[
    \phi\equiv\psi\ \Leftrightarrow\ {\cal A}(\phi)={\cal A}(\psi)
    \]
where ${\cal A}:\pv\to\pv$ is the absorption mapping. Then $\equiv$
is a congruence.
\end{prop}
\begin{proof}
The relation $\equiv$ is clearly an equivalence relation on \pv. So
it remains to show that it is also a congruent relation. By
reflexivity, we already know that $\bot\equiv\bot$. So let
$\phi=\phi_{1}\to\phi_{2}$ and $\psi=\psi_{1}\to\psi_{2}$ be such
that $\phi_{1}\equiv\psi_{1}$ and $\phi_{2}\equiv\psi_{2}$. We need
to show that $\phi\equiv\psi$ or equivalently ${\cal A}(\phi)={\cal
A}(\psi)$ which goes as follows:
    \begin{eqnarray*}
    {\cal A}(\phi)&=&{\cal A}(\phi_{1}\to\phi_{2})\\
    &=&{\cal A}(\phi_{1})\to{\cal A}(\phi_{2})\\
    \phi_{i}\equiv\psi_{i}\ \rightarrow
    &=&{\cal A}(\psi_{1})\to{\cal A}(\psi_{2})\\
    &=&{\cal A}(\psi_{1}\to\psi_{2})\\
    &=&{\cal A}(\psi)
    \end{eqnarray*}
We now assume that $\phi=\forall x\phi_{1}$ and $\psi=\forall
x\psi_{1}$ where $\phi_{1}\equiv\psi_{1}$. We need to show that
$\phi\equiv\psi$ or equivalently ${\cal A}(\phi)={\cal A}(\psi)$. We
shall distinguish two cases: first we assume that
$x\in\free(\phi_{1})$. Having assumed that $\phi_{1}\equiv\psi_{1}$,
we obtain ${\cal A}(\phi_{1})={\cal A}(\psi_{1})$ and consequently
from
proposition~(\ref{logic:prop:FOPL:absmapping:equivalence:A:phi}) we
have $\phi_{1}\sim\psi_{1}$, where $\sim$ is the absorption
congruence on \pv. It follows from
proposition~(\ref{logic:prop:FOPL:abscong:freevar}) that
$\free(\phi_{1})=\free(\psi_{1})$ and thus $x\in\free(\psi_{1})$.
Hence we have:
    \begin{eqnarray*}
    {\cal A}(\phi)&=&{\cal A}(\forall x\phi_{1})\\
    x\in\free(\phi_{1})\ \rightarrow
    &=&\forall x{\cal A}(\phi_{1})\\
    \phi_{1}\equiv\psi_{1}\ \rightarrow
    &=&\forall x{\cal A}(\psi_{1})\\
    x\in\free(\psi_{1})\ \rightarrow
    &=&{\cal A}(\forall x\psi_{1})\\
    &=&{\cal A}(\psi)
    \end{eqnarray*}
We now assume that $x\not\in\free(\phi_{1})$. Then we also have
$x\not\in\free(\psi_{1})$ and hence:
    \begin{eqnarray*}
    {\cal A}(\phi)&=&{\cal A}(\forall x\phi_{1})\\
    x\not\in\free(\phi_{1})\ \rightarrow
    &=&{\cal A}(\phi_{1})\\
    \phi_{1}\equiv\psi_{1}\ \rightarrow
    &=&{\cal A}(\psi_{1})\\
    x\not\in\free(\psi_{1})\ \rightarrow
    &=&{\cal A}(\forall x\psi_{1})\\
    &=&{\cal A}(\psi)
    \end{eqnarray*}
\end{proof}

\begin{prop}\label{logic:prop:FOPL:absmapping:kernel}
Let $\sim$ be the absorption congruence on \pv\ where $V$ is a set.
Then for all $\phi,\psi\in\pv$ we have the following equivalence:
    \[
    \phi\sim\psi\ \Leftrightarrow\ {\cal A}(\phi)={\cal A}(\psi)
    \]
where ${\cal A}:\pv\to\pv$ is the absorption mapping as per {\em
definition~(\ref{logic:def:FOPL:absmapping:mapping})}.
\end{prop}
\begin{proof}
First we show $\Rightarrow$\,: Let $\equiv$ be the relation on \pv\
defined by $\phi\equiv\psi$ \ifand\ ${\cal A}(\phi)={\cal A}(\psi)$.
We need to show the inclusion $\sim\,\subseteq\,\equiv\,$. However,
since $\sim$ is the smallest congruence on \pv\ which contains the
set $R_{0}$ of definition~(\ref{logic:def:FOPL:abscong:absorption}),
it is sufficient to show that $\equiv$ is a congruence on \pv\ which
contains $R_{0}$. We already know from
proposition~(\ref{logic:prop:FOPL:absmapping:congruence}) that
$\equiv$ is a congruence on \pv. So it remains to show that
$R_{0}\subseteq\equiv\,$. So let $\phi_{1}\in\pv$ and $x\in V$ such
that $x\not\in\free(\phi_{1})$. We need to show that ${\cal
A}(\phi_{1})={\cal A}(\forall x\phi_{1})$ which follows immediately
from definition~(\ref{logic:def:FOPL:absmapping:mapping}). We now
prove $\Leftarrow$\,: so we assume that ${\cal A}(\phi)={\cal
A}(\psi)$. We need to show that $\phi\sim\psi$, which follows
immediately from
proposition~(\ref{logic:prop:FOPL:absmapping:equivalence:A:phi}).
\end{proof}
