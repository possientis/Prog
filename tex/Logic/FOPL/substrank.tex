When $V$ and $W$ are sets and $\sigma:V\to W$ is a map, given
$\phi\in\pv$ our motivating force in the last few sections has been
to define a map $\sigma(\phi)$ even in the case when $\sigma$ is not
a valid substitution for $\phi$. Our strategy so far has been to
{\em step outside} of ${\bf P}(W)$ into the bigger space ${\bf
P}(\bar{W})$, and define $\sigma(\phi)$ as $\bar{\sigma}\circ{\cal
M}(\phi)$ where ${\cal M}(\phi)$ is the minimal transform of $\phi$
and $\bar{\sigma}:\bar{V}\to\bar{W}$ is the minimal extension of
$\sigma$, as per
definitions~(\ref{logic:def:FOPL:mintransform:transform})
and~(\ref{logic:def:FOPL:commute:minextensioon:map}) respectively.
In this section and the next, we wish to go further than this and
define $\sigma(\phi)$ as an actual element of ${\bf P}(W)$. This
shouldn't be too hard. Indeed, when computing the minimal transform
${\cal M}(\phi)$, all the bound variables have been ordered and
rationalized in a given way. The effect of $\bar{\sigma}$ on ${\cal
M}(\phi)$ is simply to move the free variables from $V$ to $W$ which
should have no impact of the configuration of the bound variables.
It is therefore not unreasonable to think that the formula
$\bar{\sigma}\circ{\cal M}(\phi)$ is still properly configured so to
speak, in other words that $\bar{\sigma}\circ{\cal M}(\phi)={\cal
M}(\psi)$ for some $\psi\in{\bf P}(W)$. If this is the case, we can
simply define $\sigma(\phi)=\psi$. This definition would not
uniquely characterize the formula $\sigma(\phi)$, but it would
certainly define a unique class modulo the substitution congruence,
by virtue of theorem~(\ref{logic:the:FOPL:mintransfsubcong:kernel})
of page~\pageref{logic:the:FOPL:mintransfsubcong:kernel}. So let's
have a look at an example with $\phi=\forall y(x\in y)$ and $x\neq
y$. Assuming $\sigma(x)=u$ we obtain $\bar{\sigma}\circ{\cal
M}(\phi)=\forall\,0(u\in 0)$, and sure enough this is indeed the
minimal transform of $\forall v(u\in v)$ where $v\in W$ and $u\neq
v$. So our strategy is clear: all we need to do is prove that
$\bar{\sigma}\circ{\cal M}(\phi)$ is indeed equal to the minimal
transform ${\cal M}(\psi)$ for some $\psi\in{\bf P}(W)$. There is
however a glitch: the set $W$ may be too small. Our example was
successful because we assumed the existence of $v\in W$ such that
$u\neq v$. This need not be the case. We cannot have $W=\emptyset$
as otherwise there would be no map $\sigma:V\to W$ when
$\{x,y\}\subseteq V\neq\emptyset$, but it is possible that $W$ be
limited to the singleton $W=\{u\}$. When this is the case, there
exists no $\psi\in{\bf P}(W)$ such that ${\cal
M}(\psi)=\forall\,0(u\in 0)$. So it is clear we cannot hope to
achieve $\bar{\sigma}\circ{\cal M}(\phi)={\cal M}(\psi)$ in general
unless $W$ has sufficiently many variables to {\em accommodate} the
formula $\bar{\sigma}\circ{\cal M}(\phi)$. It appears the formula
$\forall\,0(u\in 0)$ requires at least two variables. This is also
the case for $\forall\,1\forall\,0(0\in 1)$ or $(u\in v)$ with
$u\neq v$, while $\forall\,0(0\in 0)$ requires one variable only. As
for the formula $\bot\to(\bot\to\bot)$, it requires no variable at
all. So we need to make this idea precise: given the formula
$\bar{\sigma}\circ{\cal M}(\phi)$ we have to define the minimum
required number of variables in $W$ so the formula can be {\em
squeezed} into ${\bf P}(W)$. This motivates the following
definition:
\index{rank@Substitution rank of
formula}\index{r@$\rnk(\phi)$ : substitution rank of $\phi$}
\begin{defin}\label{logic:def:FOPL:substrank:substrank}
Let $V$ be a set and $\sim$ be the substitution congruence on $\pv$.
For all $\phi\in\pv$ we call {\em substitution rank} of $\phi$ the
integer $\rnk(\phi)$ defined by:
    \[
    \rnk(\phi)=\min\{\,|\var(\psi)|\ :\
    \psi\in\pv\ ,\ \phi\sim\psi\,\}
    \]
where $|\var(\psi)|$ denotes the cardinal of the set $\var(\psi)$,
for all $\psi\in\pv$.
\end{defin}
Given a set $A$, it is customary to use the notation $|A|$ to refer
to the {\em cardinal number} of~$A$. We have not formally defined
the notion of ordinal and cardinal numbers at this stage, but for
all $\psi\in\pv$ the set $\var(\psi)$ is finite and $|\var(\psi)|$
is simply the number of elements it has. So this should be enough
for now. Given a formula $\phi\in\pv$, we defined $\rnk(\phi)$ as
the smallest number of variables used by a formula $\psi$, which is
equivalent to $\phi$ modulo the substitution congruence. Our hope is
that provided $\rnk(\,\bar{\sigma}\circ{\cal M}(\phi)\,)\leq|W|$,
there should be enough variables in $W$ for us to find $\psi\in{\bf
P}(W)$ such that $\bar{\sigma}\circ{\cal M}(\phi)={\cal M}(\psi)$.
So let us see if this works: in the case $\bar{\sigma}\circ{\cal
M}(\phi)=\bot$, the condition $\rnk(\,\bar{\sigma}\circ{\cal
M}(\phi)\,)\leq|W|$ becomes $0\leq|W|$ and it is clear we do not
require $W$ to have any element. In the case when
$\bar{\sigma}\circ{\cal M}(\phi)=\forall\,0(0\in 0)$ the condition
becomes $1\leq |W|$ and it is clear $W$ should have at least one
variable. If $\bar{\sigma}\circ{\cal M}(\phi)=\forall\,0(u\in 0)$
then the condition becomes $2\leq |W|$ which is another success. A
more interesting example is $\bar{\sigma}\circ{\cal
M}(\phi)=\forall\,1\forall\,0(1\in 0)\to(u\in v)$ with $u\neq v$.
This formula is equivalent to $\psi=\forall u\forall v(u\in
v)\to(u\in v)$ and consequently we have
$\rnk(\,\bar{\sigma}\circ{\cal M}(\phi)\,)=2$. So the condition
becomes $2\leq|W|$ which is indeed the correct condition to impose
on $W$ since we do not need more than two variables, as
$\bar{\sigma}\circ{\cal M}(\phi)={\cal M}(\psi)$ with $\psi=\forall
u\forall v(u\in v)\to(u\in v)$. Hence we believe
definition~(\ref{logic:def:FOPL:substrank:substrank}) refers to the
correct notion and we shall endeavor to prove the existence of
$\psi\in{\bf P}(W)$ such that $\bar{\sigma}\circ{\cal M}(\phi)={\cal
M}(\psi)$ whenever the condition $\rnk(\,\bar{\sigma}\circ{\cal
M}(\phi)\,)\leq|W|$ is satisfied. For now, we shall establish some
of the rank's properties:
\begin{prop}\label{logic:prop:FOPL:substrank:basic:ineq}
Let $V$ be a set and $\phi\in\pv$. Then, we have:
    \[
    |\free(\phi)|\leq\rnk(\phi)\leq|\var(\phi)|
    \]
\end{prop}
\begin{proof}
The inequality $\rnk(\phi)\leq|\var(\phi)|$ follows immediately from
definition~(\ref{logic:def:FOPL:substrank:substrank}). So it remains
to show that $|\free(\phi)|\leq\rnk(\phi)$. So let $\psi\in\pv$ such
that $\phi\sim\psi$ where $\sim$ is the substitution congruence. We
need to show $|\free(\phi)|\leq|\var(\psi)|$. From
proposition~(\ref{logic:prop:sub:congruence:freevar}) we have
$\free(\phi)=\free(\psi)$. Hence we need to show that
$|\free(\psi)|\leq|\var(\psi)|$ which follows from
$\free(\psi)\subseteq\var(\psi)$.
\end{proof}

Given $\phi\in\pv$, we know from
definition~(\ref{logic:def:FOPL:substrank:substrank}) that
$\rnk(\phi)=|\var(\psi)|$ for some $\psi\sim\phi$. Given a suitably
large subset $V_{0}\subseteq V$, the following proposition allows us
to choose such a $\psi$ with the additional property
$\var(\psi)\subseteq V_{0}$ provided $\free(\phi)\subseteq V_{0}$.
Obviously we cannot hope to have $\var(\psi)\subseteq V_{0}$ unless
$V_{0}$ is large enough and $\free(\phi)\subseteq V_{0}$ since two
substitution equivalent formulas have the same free variables. Being
able to find a $\psi$ with $\var(\psi)\subseteq V_{0}$ is crucially
important for us. If we go back to the case when
$\bar{\sigma}\circ{\cal M}(\phi)=\forall\,0(u\in 0)$, our first step
in proving the existence of $\psi$ such that $\bar{\sigma}\circ{\cal
M}(\phi)={\cal M}(\psi)$ will be to show the existence of
$\psi_{1}=\forall v(u\in v)$, which is a $\psi_{1}$ with
$\var(\psi_{1})\subseteq V_{0}$ where $V_{0}=W\subseteq\bar{W}$.
When $V_{0}$ is not large enough, we cannot hope to find  a $\psi$
with $\var(\psi)\subseteq V_{0}$. However, we can ensure that
$V_{0}\subseteq\var(\psi)$ which will also be a useful property when
calculating the substitution rank of $\phi_{1}\to\phi_{2}$.


\begin{prop}\label{logic:prop:FOPL:substrank:changeofvar}
Let $V$ be a set and $\phi\in\pv$ such that $\free(\phi)\subseteq
V_{0}$ for some $V_{0}\subseteq V$.  Let $\sim$ denote the
substitution congruence on \pv. Then there exists $\psi\in\pv$ such
that $\phi\sim\psi$ and $|\var(\psi)|=\rnk(\phi)$ such that:
    \begin{eqnarray*}
    (i)&&\rnk(\phi)\leq|V_{0}|\ \Rightarrow\ \var(\psi)\subseteq
    V_{0}\\
    (ii)&&|V_{0}|\leq\rnk(\phi)\ \Rightarrow\
    V_{0}\subseteq\var(\psi)
    \end{eqnarray*}
\end{prop}
\begin{proof}
Without loss of generality, we may assume that
$|\var(\phi)|=\rnk(\phi)$. Indeed, suppose the proposition has been
established with the additional assumption
$|\var(\phi)|=\rnk(\phi)$. We need to show it is then true in the
general case. So given $V_{0}\subseteq V$, consider $\phi\in\pv$
such that $\free(\phi)\subseteq V_{0}$. From
definition~(\ref{logic:def:FOPL:substrank:substrank}) there exists
$\phi_{1}\in\pv$ such that $\phi\sim\phi_{1}$ with the equality
$|\var(\phi_{1})|=\rnk(\phi)$. From
proposition~(\ref{logic:prop:sub:congruence:freevar}) we obtain
$\free(\phi)=\free(\phi_{1})$ and so $\free(\phi_{1})\subseteq
V_{0}$. Hence we see that $\phi_{1}$ satisfies the assumption of the
proposition with the additional property
$|\var(\phi_{1})|=\rnk(\phi_{1})$. Having assumed the proposition is
true in this case, we obtain the existence of $\psi\in\pv$ such that
$\phi_{1}\sim\psi$, $|\var(\psi)|=\rnk(\phi_{1})$ and which
satisfies $(i)$ and $(ii)$ where $\phi$ is replaced by $\phi_{1}$.
However $\rnk(\phi_{1})=\rnk(\phi)$ and replacing $\phi$ by
$\phi_{1}$ in $(i)$ and $(ii)$ has no impact. So $\psi$ satisfies
$(i)$ and $(ii)$. Hence we have $\phi\sim\psi$ and
$|\var(\psi)|=\rnk(\phi)$ together with $(i)$ and $(ii)$ which
establishes the proposition in the general case. So we now assume
without loss of generality that $|\var(\phi)|=\rnk(\phi)$ and
$\free(\phi)\subseteq V_{0}$. We need to show the existence of
$\psi\sim\phi$ such that $|\var(\psi)|=|\var(\phi)|$ and which
satisfies $(i)$ and $(ii)$. Note that if $V=\emptyset$  then
$V_{0}=\emptyset$ and we can take $\psi=\phi$. So we assume
$V\neq\emptyset$. We shall first consider the case when
$\rnk(\phi)\leq |V_{0}|$ and show the existence of $\psi$ such that
$\var(\psi)\subseteq V_{0}$. Since $\free(\phi)\subseteq V_{0}$ the
set $V_{0}$ is the disjoint union of $\free(\phi)$ and
$V_{0}\setminus\free(\phi)$, giving us the equality $|V_{0}|
=|\free(\phi)|+|V_{0}\setminus\free(\phi)|$. Since we also have
$|\var(\phi)|=|\free(\phi)|+|\var(\phi)\setminus\free(\phi)|$, we
obtain from $|\var(\phi)|\leq|V_{0}|$:
    \[
    |\free(\phi)|+|\var(\phi)\setminus\free(\phi)|\leq|\free(\phi)|
    +|V_{0}\setminus\free(\phi)|
    \]
Since $|\free(\phi)|$ is a finite cardinal it follows that
$|\var(\phi)\setminus\free(\phi)|\leq|V_{0}\setminus\free(\phi)|$.
Hence, there is an injection mapping
$i:\var(\phi)\setminus\free(\phi)\to V_{0}\setminus\free(\phi)$.
Having assumed $V\neq\emptyset$ consider $x^{*}\in V$ and define the
map $\sigma:V\to V$ as follows:
    \[
    \forall u\in V\ ,\ \sigma(u)=\left\{
        \begin{array}{lcl}
        u&\mbox{\ if\ }&u\in\free(\phi)\\
        i(u)&\mbox{\ if\ }&u\in\var(\phi)\setminus\free(\phi)\\
        x^{*}&\mbox{\ if\ }&u\not\in\var(\phi)
        \end{array}
    \right.
    \]
Define $\psi=\sigma(\phi)$. It remains to show that $\psi$ has the
desired properties, namely that $\psi\sim\phi$,
$|\var(\psi)|=|\var(\phi)|$ and $\var(\psi)\subseteq V_{0}$. First
we show $\psi\sim\phi$. Using
proposition~(\ref{logic:prop:admissible:sub:congruence}) it is
sufficient to prove that $\sigma$ is an admissible substitution for
$\phi$. It is clear that $\sigma(u)=u$ for all $u\in\free(\phi)$. So
it remains to show that $\sigma$ is valid for $\phi$. From
proposition~(\ref{logic:prop:FOPL:valid:injective}) it is sufficient
to prove that $\sigma_{|\var(\phi)}$ is an injective map. So let
$u,v\in\var(\phi)$ such that $\sigma(u)=\sigma(v)$. We need to show
that $u=v$. We shall distinguish four cases: first we assume that
$u\in\free(\phi)$ and $v\in\free(\phi)$. Then the equality
$\sigma(u)=\sigma(v)$ leads to $u=v$. Next we assume that
$u\not\in\free(\phi)$ and $v\not\in\free(\phi)$. Then we obtain
$i(u)=i(v)$ which also leads to $u=v$ since
$i:\var(\phi)\setminus\free(\phi)\to V_{0}\setminus\free(\phi)$ is
an injective map. So we now assume that $u\in\free(\phi)$ and
$v\not\in\free(\phi)$. Then from $\sigma(u)=\sigma(v)$ we obtain
$u=i(v)$ which is in fact impossible since $u\in\free(\phi)$ and
$i(v)\in V_{0}\setminus\free(\phi)$. The last case
$u\not\in\free(\phi)$ and $v\in\free(\phi)$ is equally impossible
which completes our proof of $\psi\sim\phi$. So we now prove that
$|\var(\psi)|=|\var(\phi)|$. From
proposition~(\ref{logic:prop:var:of:substitution}) we have
$\var(\psi)=\var(\sigma(\phi))=\sigma(\var(\phi))$. So we need
$|\sigma(\var(\phi))|=|\var(\phi)|$ which is clear since
$\sigma_{|\var(\phi)}:\var(\phi)\to\sigma(\var(\phi))$ is a
bijection. So it remains to show that $\var(\psi)\subseteq V_{0}$,
or equivalently that $\sigma(\var(\phi))\subseteq V_{0}$. So let
$u\in\var(\phi)$. We need to show that $\sigma(u)\in V_{0}$. We
shall distinguish two cases: first we assume that $u\in\free(\phi)$.
Then $\sigma(u)=u$ and the property $\sigma(u)\in V_{0}$ follows
from the inclusion $\free(\phi)\subseteq V_{0}$. Next we assume that
$u\not\in\free(\phi)$. Then we have $\sigma(u)=i(u)\in
V_{0}\setminus\free(\phi)\subseteq V_{0}$. So in the case when
$\rnk(\phi)\leq|V_{0}|$ we have been able to prove the existence of
$\psi$ satisfying $(i)$. In fact we claim that $\psi$ also satisfies
$(ii)$. So let us assume that $|V_{0}|\leq\rnk(\phi)$. Then we must
have $\rnk(\phi)=|V_{0}|$ and we need to show that
$V_{0}\subseteq\var(\psi)$. However, we have
$|\var(\psi)|=\rnk(\phi)$ and consequently $|\var(\psi)|=|V_{0}|$
together with $\var(\psi)\subseteq V_{0}$. Two finite subsets
ordered by inclusion and with the same cardinal must be equal. So
$\var(\psi)=V_{0}$. We now consider the case when
$|V_{0}|<\rnk(\phi)$. We need to show the existence of
$\psi\sim\phi$ such that $|\var(\psi)|=|\var(\phi)|$ satisfying
$(i)$ and $(ii)$. In the case when $|V_{0}|<\rnk(\phi)$, $(i)$ is
vacuously true, so we simply need to ensure that
$V_{0}\subseteq\var(\psi)$. Since $|V_{0}|<|\var(\phi)|$ we obtain:
    \begin{eqnarray*}
    |V_{0}\setminus\var(\phi)|&=&|V_{0}|-|V_{0}\cap\var(\phi)|\\
    &<&|\var(\phi)|-|V_{0}\cap\var(\phi)|\\
    &=&|\var(\phi)\setminus V_{0}|
    \end{eqnarray*}
So there is an injective map
$i:V_{0}\setminus\var(\phi)\to\var(\phi)\setminus V_{0}$. Given
$x^{*}\in V$, define:
    \[
    \forall u\in V\ ,\ \sigma(u)=\left\{
        \begin{array}{lcl}
        u&\mbox{\ if\ }&u\in\var(\phi)\setminus i(\,V_{0}\setminus\var(\phi)\,)\\
        i^{-1}(u)&\mbox{\ if\ }&u\in i(\,V_{0}\setminus\var(\phi)\,)\\
        x^{*}&\mbox{\ if\ }&u\not\in\var(\phi)
        \end{array}
    \right.
    \]
Let $\psi=\sigma(\phi)$. It remains to show that $\psi\sim\phi$,
$|\var(\psi)|=|\var(\phi)|$ and $V_{0}\subseteq\var(\psi)$. First we
show that $\psi\sim\phi$. Using
proposition~(\ref{logic:prop:admissible:sub:congruence}) it is
sufficient to prove that $\sigma$ is an admissible substitution for
$\phi$. So let $u\in\free(\phi)$. We need to show that
$\sigma(u)=u$. So it is sufficient to prove that $u\not\in
i(\,V_{0}\setminus\var(\phi)\,)$ which follows from the fact that
$u\in V_{0}$, itself a consequence of $\free(\phi)\subseteq V_{0}$.
In order to show that $\sigma$ is also valid for $\phi$, from
proposition~(\ref{logic:prop:FOPL:valid:injective}) it is sufficient
to prove that $\sigma$ is injective on $\var(\phi)$. So let
$u,v\in\var(\phi)$ such that $\sigma(u)=\sigma(v)$. We need to prove
that $u=v$. The only case when this may not be clear is when
$\sigma(u)=u$ and $\sigma(v)=i^{-1}(v)$ or vice versa. So we assume
that $u\in\var(\phi)\setminus i(\,V_{0}\setminus\var(\phi)\,)$ and $
v\in i(\,V_{0}\setminus\var(\phi)\,)$. Then we see that
$\sigma(u)=u\in\var(\phi)$ while $\sigma(v)=i^{-1}(v)\in
V_{0}\setminus\var(\phi)$. So the equality $\sigma(u)=\sigma(v)$ is
in fact impossible, which completes our proof of $\psi\sim\phi$. As
before, the fact that $|\var(\psi)|=|\var(\phi)|$ follows from the
injectivity of $\sigma_{|\var(\phi)}$ and it remains to prove that
$V_{0}\subseteq\var(\psi)$. So let $u\in V_{0}$ we need to show that
$u\in\var(\psi)=\sigma(\var(\phi))$ and we shall distinguish two
cases: first we assume that $u\in\var(\phi)$. Since $u\in V_{0}$, it
cannot be an element of $i(\,V_{0}\setminus\var(\phi)\,)$. It
follows that $u\in\var(\phi)\setminus
i(\,V_{0}\setminus\var(\phi)\,)$ and consequently
$u=\sigma(u)\in\sigma(\var(\phi))=\var(\psi)$. Next we assume that
$u\in V_{0}\setminus\var(\phi)$. Then $i(u)$ is an element of
$i(\,V_{0}\setminus\var(\phi)\,)$ and therefore
$\sigma(i(u))=i^{-1}(i(u))=u$. Since $i(u)$ is an element of
$\var(\phi)\setminus V_{0}$ we conclude that
$u=\sigma(i(u))\in\sigma(\var(\phi))=\var(\psi)$, which completes
our proof.
\end{proof}

The substitution rank of a formula is essentially the minimum number
of variables needed to describe a representative of its class modulo
substitution. We should expect injective variable substitutions to
have no effect on the rank.

\begin{prop}\label{logic:prop:FOPL:substrank:injective}
Let $V,W$ be sets and $\sigma:V\to W$ be a map. Then for all
$\phi\in\pv$, if $\sigma_{|\var(\phi)}$ is an injective map, we have
the equality:
    \[
    \rnk(\sigma(\phi))=\rnk(\phi)
    \]
where $\sigma:\pv\to{\bf P}(W)$ denotes the associated substitution
mapping.
\end{prop}
\begin{proof}
Let $\sigma:V\to W$ and $\phi\in\pv$ such that
$\sigma_{|\var(\phi)}$ is an injective map. We need to show that
$\rnk(\sigma(\phi))=\rnk(\phi)$. First we shall show
$\rnk(\sigma(\phi))\leq\rnk(\phi)$. Using
proposition~(\ref{logic:prop:FOPL:substrank:changeofvar}) with
$V_{0}=\var(\phi)$, since we have $\free(\phi)\subseteq\var(\phi)$
and $\rnk(\phi)\leq|\var(\phi)|$, there exists $\psi\in\pv$ such
that $\phi\sim\psi$, $|\var(\psi)|=\rnk(\phi)$ and
$\var(\psi)\subseteq\var(\phi)$. Having assumed $\sigma$ is
injective on $\var(\phi)$, it is therefore injective on both
$\var(\phi)$ and $\var(\psi)$. From
proposition~(\ref{logic:prop:FOPL:valid:injective}) it follows that
$\sigma$ is a valid substitution for both $\phi$ and $\psi$. Hence
from theorem~(\ref{logic:the:FOPL:mintransfsubcong:valid}) of
page~\pageref{logic:the:FOPL:mintransfsubcong:valid} we obtain
$\sigma(\phi)\sim\sigma(\psi)$ and consequently using
proposition~(\ref{logic:prop:var:of:substitution})\,:
    \[
    \rnk(\sigma(\phi))\leq|\var(\sigma(\psi))|=|\sigma(\var(\psi))|
    =|\var(\psi)|=\rnk(\phi)
    \]
So it remains to show that $\rnk(\phi)\leq\rnk(\sigma(\phi))$. If
$V=\emptyset$ then $\rnk(\phi)=0$ and we are done. So we may assume
that $V\neq\emptyset$. So let $x^{*}\in V$ and define:
    \[
    \forall u\in W\ ,\ \tau(u)=\left\{
        \begin{array}{lcl}
        \sigma^{-1}(u)&\mbox{\ if\ }&u\in\sigma(\var(\phi))\\
        x^{*}&\mbox{\ if\ }&u\not\in\sigma(\var(\phi))
        \end{array}
    \right.
    \]
Then $\tau:W\to V$ is injective on
$\var(\sigma(\phi))=\sigma(\var(\phi))$ and hence:
    \[
    \rnk(\,\tau(\sigma(\phi))\,)\leq\rnk(\sigma(\phi))
    \]
So it suffices for us to show that  $\tau\circ\sigma(\phi)=\phi$.
From proposition~(\ref{logic:prop:substitution:support}) it is thus
sufficient to show that $\tau\circ\sigma(x)=x$ for all
$x\in\var(\phi)$, which is clear.
\end{proof}

If $\phi\in\pv$ we know from
proposition~(\ref{logic:prop:FOPL:mintransform:eqivalence}) that
${\cal M}(\phi)$ is substitution equivalent to $i(\phi)$ where
$i:V\to\bar{V}$ is the inclusion map. Hence we have:

\begin{prop}\label{logic:prop:FOPL:substrank:minrank}
Let $V$ be a set and $\phi\in\pv$. Then the formula $\phi$ and its
minimal transform have equal substitution rank, i.e.:
    \[
    \rnk({\cal M}(\phi))=\rnk(\phi)
    \]
\end{prop}
\begin{proof}
Let $\sim$ denote the substitution congruence on \pvb\ and
$i:V\to\bar{V}$ be the inclusion map. From
proposition~(\ref{logic:prop:FOPL:mintransform:eqivalence}) we have
${\cal M}(\phi)\sim\, i(\phi)$ and consequently $\rnk({\cal
M}(\phi))=\rnk(i(\phi))$. So we need to show that
$\rnk(i(\phi))=\rnk(\phi)$ which follows from
proposition~(\ref{logic:prop:FOPL:substrank:injective}) and the fact
that $i:V\to\bar{V}$ is injective.
\end{proof}

We have already established that an injective variable substitution
does not change the rank of a formula. If $\phi\in\pv$ and
$\sigma:V\to V$ is an admissible substitution for $\phi$, we also
know from proposition~(\ref{logic:prop:admissible:sub:congruence})
that $\phi\sim\sigma(\phi)$. Hence if $\sigma:V\to V$ is valid for
$\phi$ while leaving the free variables unchanged, it will not
change the rank of the formula $\phi$ either. It is not unreasonable
to think that this property will remain true if we only impose that
$\sigma$ be injective on $\free(\phi)$. In fact, this should also
apply to any $\sigma:V\to W$, without assuming $W=V$.

\begin{prop}\label{logic:prop:FOPL:substrank:invariant:rank}
Let $V,W$ be sets and $\sigma:V\to W$ be a map. Let $\phi\in\pv$. We
assume that $\sigma$ is valid for $\phi$ and $\sigma_{|\free(\phi)}$
is an injective map. Then:
    \[
    \rnk(\sigma(\phi))=\rnk(\phi)
    \]
where $\sigma:\pv\to{\bf P}(W)$ denotes the associated substitution
mapping.
\end{prop}
\begin{proof}
We shall proceed in two steps: first we shall prove the proposition
is true in the case when $W=V$. We shall then extend the result to
arbitrary $W$. So let us assume $W=V$. Let $\sigma:V\to V$ valid for
$\phi\in\pv$ such that $\sigma_{|\free(\phi)}$ is an injective map.
We need to show that $\rnk(\sigma(\phi))=\rnk(\phi)$. If
$V=\emptyset$ then $\sigma:V\to V$ is the map with empty domain,
namely the empty set which is injective on $\var(\phi)=\emptyset$
and $\rnk(\sigma(\phi))=\rnk(\phi)$ follows from
proposition~(\ref{logic:prop:FOPL:substrank:injective}). So we
assume that $V\neq\emptyset$. The idea of the proof is to write
$\sigma(\phi)=\tau_{1}\circ\tau_{0}(\phi)$ where each substitution
$\tau_{0}, \tau_{1}$ is rank preserving. Having assumed $\sigma$
injective on $\free(\phi)$, we have the equality
$|\sigma(\free(\phi)|=|\free(\phi)|$ and consequently:
    \begin{eqnarray*}
    |\var(\phi)\setminus\free(\phi)|&=&|\var(\phi)|-|\free(\phi)|\\
    &\leq&|V|-|\free(\phi)|\\
    &=&|V|-|\,\sigma(\free(\phi))\,|\\
    &=&|\,V\setminus\sigma(\free(\phi))\,|
    \end{eqnarray*}
So let $i:\var(\phi)\setminus\free(\phi)\to
V\setminus\sigma(\free(\phi))$ be an injective map. Let $x^{*}\in V$
and define the substitution $\tau_{0}:V\to V$ as follows:
    \[
    \forall x\in V\ ,\ \tau_{0}(x)=\left\{
        \begin{array}{lcl}
        \sigma(x)&\mbox{\ if\ }&x\in\free(\phi)\\
        i(x)&\mbox{\ if\ }&x\in\var(\phi)\setminus\free(\phi)\\
        x^{*}&\mbox{\ if\ }&x\not\in\var(\phi)
        \end{array}
    \right.
    \]
Let us accept for now that $\tau_{0}$ is injective on $\var(\phi)$.
Then using proposition~(\ref{logic:prop:FOPL:substrank:injective})
we obtain $\rnk(\tau_{0}(\phi))=\rnk(\phi)$. So consider
$\tau_{1}:V\to V$\,:
    \[
    \forall u\in V\ ,\ \tau_{1}(u)=\left\{
        \begin{array}{lcl}
        u&\mbox{\ if\ }&u\in\sigma(\free(\phi))\\
        \sigma\circ i^{-1}(u)&\mbox{\ if\ }&u\in i(\,\var(\phi)\setminus\free(\phi)\,)\\
        x^{*}&\mbox{\ \ \ \ }&\mbox{otherwise}
        \end{array}
    \right.
    \]
Note that $\tau_{1}$ is well defined since $\sigma(\free(\phi))\cap
i(\,\var(\phi)\setminus\free(\phi)\,)=\emptyset$. So let us accept
for now that $\tau_{1}$ is admissible for $\tau_{0}(\phi)$. Then
using proposition~(\ref{logic:prop:admissible:sub:congruence}) we
obtain $\tau_{1}\circ\tau_{0}(\phi)\sim\tau_{0}(\phi)$ where $\sim$
is the substitution congruence on \pv. Hence in particular we have
$\rnk(\tau_{1}\circ\tau_{0}(\phi))=\rnk(\tau_{0}(\phi))=\rnk(\phi)$.
So in order to show that the proposition is true in the case when
$W=V$, it remains to prove that $\tau_{0}$ is injective on
$\var(\phi)$, $\tau_{1}$ is admissible for $\tau_{0}(\phi)$ and
furthermore that $\tau_{1}\circ\tau_{0}(\phi)=\sigma(\phi)$. First
we show that $\tau_{0}$ is injective on $\var(\phi)$. So let
$x,y\in\var(\phi)$ such that $\tau_{0}(x)=\tau_{0}(y)$. We need to
show that $x=y$. We shall distinguish four cases: first we assume
that $x\in\free(\phi)$ and $y\in\free(\phi)$. Then the equality
$\tau_{0}(x)=\tau_{0}(y)$ leads to $\sigma(x)=\sigma(y)$. Having
assumed $\sigma_{|\free(\phi)}$ is an injective map, we obtain
$x=y$. Next we assume that $x\in\var(\phi)\setminus\free(\phi)$ and
$y\in\var(\phi)\setminus\free(\phi)$. Then the equality
$\tau_{0}(x)=\tau_{0}(y)$ leads to $i(x)=i(y)$ and consequently
$x=y$. So we now assume that $x\in\free(\phi)$ and
$y\in\var(\phi)\setminus\free(\phi)$. Then
$\tau_{0}(x)=\sigma(x)\in\sigma(\free(\phi))$ and
$\tau_{0}(y)=i(y)\in V\setminus\sigma(\free(\phi))$. So the equality
$\tau_{0}(x)=\tau_{0}(y)$ is in fact impossible. We show similarly
that the final case $x\in\var(\phi)\setminus\free(\phi)$ and
$y\in\free(\phi)$ is also impossible which completes the proof that
$\tau_{0}$ is injective on $\var(\phi)$. We shall now show that
$\tau_{1}\circ\tau_{0}(\phi)=\sigma(\phi)$. From
proposition~(\ref{logic:prop:substitution:support}) it is sufficient
to prove that $\tau_{1}\circ\tau_{0}(x)=\sigma(x)$ for all
$x\in\var(\phi)$. We shall distinguish two cases: first we assume
that $x\in\free(\phi)$. Then
$\tau_{0}(x)=\sigma(x)\in\sigma(\free(\phi))$ and consequently
$\tau_{1}\circ\tau_{0}(x)=\sigma(x)$ as requested. Next we assume
that $x\in\var(\phi)\setminus\free(\phi)$. Then $\tau_{0}(x)=i(x)\in
i(\,\var(\phi)\setminus\free(\phi)\,)$ and consequently
$\tau_{1}\circ\tau_{0}(x)=\sigma\circ i^{-1}(i(x))=\sigma(x)$. So it
remains to show that $\tau_{1}$ is admissible for $\tau_{0}(\phi)$,
i.e. that it is valid for $\tau_{0}(\phi)$ and $\tau_{1}(u)=u$ for
all $u\in\free(\tau_{0}(\phi))$. So let $u\in\free(\tau_{0}(\phi))$.
We need to show that $\tau_{1}(u)=u$. So it is sufficient to prove
that $u\in\sigma(\free(\phi))$. However from
proposition~(\ref{logic:prop:freevar:of:substitution:inclusion}) we
have $\free(\tau_{0}(\phi))\subseteq\tau_{0}(\free(\phi))$ and
consequently there exists $x\in\free(\phi)$ such that
$u=\tau_{0}(x)=\sigma(x)$. It follows that $u\in\sigma(\free(\phi))$
as requested and it remains to show that $\tau_{1}$ is valid for
$\tau_{0}(\phi)$. From
proposition~(\ref{logic:prop:FOPL:valid:composition}) it is
sufficient to show that $\tau_{1}\circ\tau_{0}$ is valid for $\phi$.
However, having proved that
$\tau_{1}\circ\tau_{0}(\phi)=\sigma(\phi)$ from
proposition~(\ref{logic:prop:FOPL:validsub:image}) it is sufficient
to prove that $\sigma$ is valid for $\phi$ which is in fact true by
assumption. This completes our proof of the proposition in the case
when $W=V$. We shall now prove the proposition in the general case.
So we assume that $\sigma:V\to W$ is a map and $\phi\in\pv$ is such
that $\sigma$ is valid for $\phi$ and $\sigma_{|\free(\phi)}$ is an
injective map. We need to prove that
$\rnk(\sigma(\phi))=\rnk(\phi)$. Let $U$ be the disjoint union of
the sets $V$ and $W$, specifically:
    \[
    U=\{0\}\times V\uplus\{1\}\times W
    \]
Let $i:V\to U$ and $j:W\to U$ be the corresponding inclusion maps.
Consider the formula $\phi^{*}=i(\phi)\in{\bf P}(U)$ and let
$\sigma^{*}:U\to U$ be defined as:
    \[
    \forall u\in U\ ,\ \sigma^{*}(u)=\left\{
        \begin{array}{lcl}
        j\circ\sigma(u)&\mbox{\ if\ }&u\in V\\
        u&\mbox{\ if\ }&u\in W\\
        \end{array}
    \right.
    \]
Let us accept for now that $\sigma^{*}$ is valid for $\phi^{*}$ and
that $\sigma^{*}_{|\free(\phi^{*})}$ is an injective map. Having
proved the proposition in the case when $W=V$, it can be applied to
$\sigma^{*}:U\to U$ and $\phi^{*}\in{\bf P}(U)$. Hence, since $i$
and $j$ are injective maps, using
proposition~(\ref{logic:prop:FOPL:substrank:injective}) we obtain
the following equalities:
    \begin{eqnarray*}
    \rnk(\sigma(\phi))&=&\rnk(j\circ\sigma(\phi))\\
    \mbox{A: to be proved}\ \rightarrow&=&\rnk(\sigma^{*}(\phi^{*}))\\
    \mbox{case $W=V$}\ \rightarrow&=&\rnk(\phi^{*})\\
    &=&\rnk(i(\phi))\\
    \mbox{prop.~(\ref{logic:prop:FOPL:substrank:injective})}\ \rightarrow
    &=&\rnk(\phi)
    \end{eqnarray*}
So it remains to show that $j\circ\sigma(\phi)=\sigma^{*}(\phi^{*})$
and furthermore that $\sigma^{*}$ is valid for $\phi^{*}$ while
$\sigma^{*}_{|\free(\phi^{*})}$ is an injective map. First we show
that $j\circ\sigma(\phi)=\sigma^{*}(\phi^{*})$. Since
$\phi^{*}=i(\phi)$ from
proposition~(\ref{logic:prop:substitution:support}) it is sufficient
to prove that $j\circ\sigma(u)=\sigma^{*}\circ i(u)$ for all
$u\in\var(\phi)$. So let $u\in\var(\phi)$. In particular $u\in V$
and consequently $i(u)\in i(V)\subseteq U$. From the above
definition of $\sigma^{*}$ we obtain immediately
$\sigma^{*}(i(u))=j\circ\sigma(u)$ as requested. So we now prove
that $\sigma^{*}$ is valid for $\phi^{*}=i(\phi)$. Using
proposition~(\ref{logic:prop:FOPL:valid:composition}) it is
sufficient to prove that $\sigma^{*}\circ i$ is valid for $\phi$.
However, having proved that $\sigma^{*}\circ
i(\phi)=j\circ\sigma(\phi)$, from
proposition~(\ref{logic:prop:FOPL:validsub:image}) it is sufficient
to prove that $j\circ\sigma$ is valid for $\phi$. Having assumed
that $\sigma$ is valid for $\phi$, using
proposition~(\ref{logic:prop:FOPL:valid:composition}) once more it
remains to show that $j$ is valid for $\sigma(\phi)$ which follows
from the injectivity of $j$ and
proposition~(\ref{logic:prop:FOPL:valid:injective}). So it remains
to prove that $\sigma^{*}_{|\free(\phi^{*})}$ is an injective map.
So let $u,v\in \free(\phi^{*})$ such that
$\sigma^{*}(u)=\sigma^{*}(v)$. We need to show that $u=v$. However
 since $\phi^{*}=i(\phi)$, from
proposition~(\ref{logic:prop:freevar:of:substitution:inclusion}) we
have $\free(\phi^{*})\subseteq i(\free(\phi))$. Hence, there exists
$x,y\in\free(\phi)$ such that $u=i(x)$ and $v=i(y)$. Having proved
that $j\circ\sigma =\sigma^{*}\circ i$ on $\var(\phi)$, from the
equality $\sigma^{*}(u)=\sigma^{*}(v)$ we obtain
$j\circ\sigma(x)=j\circ\sigma(y)$. It follows from the injectivity
of $j$ that $\sigma(x)=\sigma(y)$. Having assumed that $\sigma$ is
injective on $\free(\phi)$ we conclude that $x=y$ and finally that
$u=v$ as requested.
\end{proof}

As with anything involving the algebra $\pv$, it is difficult to
establish deeper properties without resorting to some form of
structural induction argument. Hence if we want to say anything more
substantial about the substitution rank of a formula $\phi\in\pv$ we
need to relate the rank of $\phi_{1}\to\phi_{2}$ and $\forall
x\phi_{1}$ to the ranks of $\phi_{1}$ and $\phi_{2}$. These
relationships are not as simple as one would wish.

\begin{prop}\label{logic:prop:FOPL:substrank:impl}
Let $V$ be a set and $\phi\in\pv$ of the form
$\phi=\phi_{1}\to\phi_{2}$ where $\phi_{1},\phi_{2}\in\pv$. Then the
substitution ranks of $\phi$, $\phi_{1}$ and $\phi_{2}$ satisfy the
equality:
    \[
    \rnk(\phi)=\max(\,|\free(\phi)|\,,\,\rnk(\phi_{1})\,,\,\rnk(\phi_{2})\,)
    \]
\end{prop}
\begin{proof}
First we show that
$\max(\,|\free(\phi)|\,,\,\rnk(\phi_{1})\,,\,\rnk(\phi_{2})\,)\leq\rnk(\phi)$.
From proposition~(\ref{logic:prop:FOPL:substrank:basic:ineq}) we
already know that $|\free(\phi)|\leq\rnk(\phi)$. So it remains to
show that $\rnk(\phi_{1})\leq\rnk(\phi)$ and
$\rnk(\phi_{2})\leq\rnk(\phi)$. So let $\sim$ be the substitution
congruence on \pv\ and $\psi\sim\phi$. We need to show that
$\rnk(\phi_{1})\leq|\var(\psi)|$ and
$\rnk(\phi_{2})\leq|\var(\psi)|$. However, from
$\psi\sim\phi=\phi_{1}\to\phi_{2}$ and
theorem~(\ref{logic:the:sub:congruence:charac}) of
page~\pageref{logic:the:sub:congruence:charac} we see that $\psi$
must be of the form $\psi=\psi_{1}\to\psi_{2}$ where
$\psi_{1}\sim\phi_{1}$ and $\psi_{2}\sim\phi_{2}$. Hence we have
$\rnk(\phi_{1})\leq|\var(\psi_{1})|\leq|\var(\psi)|$ and similarly
$\rnk(\phi_{2})\leq|\var(\psi_{2})|\leq|\var(\psi)|$. So it remains
to show the inequality
$\rnk(\phi)\leq\max(\,|\free(\phi)|\,,\,\rnk(\phi_{1})\,,\,\rnk(\phi_{2})\,)$.
We shall distinguish two cases: first we assume that
$\max(\rnk(\phi_{1}),\rnk(\phi_{2}))\leq|\free(\phi)|$. Since
$\free(\phi_{1})\subseteq\free(\phi)$ and
$\free(\phi_{2})\subseteq\free(\phi)$ using
proposition~(\ref{logic:prop:FOPL:substrank:changeofvar}) we obtain
the existence of $\psi_{1}\sim\phi_{1}$ and $\psi_{2}\sim\phi_{2}$
such that $|\var(\psi_{1})|=\rnk(\phi_{1})$ and
$|\var(\psi_{2})|=\rnk(\phi_{2})$ with the inclusions
$\var(\psi_{1})\subseteq\free(\phi)$ and
$\var(\psi_{2})\subseteq\free(\phi)$. Since
$\phi\sim\psi_{1}\to\psi_{2}$\,:
    \begin{eqnarray*}
    \rnk(\phi)&\leq&|\var(\psi_{1}\to\psi_{2})|\\
    &=&|\var(\psi_{1})\cup\var(\psi_{2})|\\
    \var(\psi_{i})\subseteq\free(\phi)\ \rightarrow&\leq&|\free(\phi)|\\
    &=&\max(\,|\free(\phi)|\,,\,\rnk(\phi_{1})\,,\,\rnk(\phi_{2})\,)
    \end{eqnarray*}
Next we assume that
$|\free(\phi)|\leq\max(\rnk(\phi_{1}),\rnk(\phi_{2}))$. We shall
distinguish two further cases: first we assume that
$\rnk(\phi_{1})\leq\rnk(\phi_{2})$. Since we have both
$\free(\phi_{2})\subseteq\free(\phi)$ and
$|\free(\phi)|\leq\rnk(\phi_{2})$, from
proposition~(\ref{logic:prop:FOPL:substrank:changeofvar}) we can
find $\psi_{2}\sim\phi_{2}$ such that
$|\var(\psi_{2})|=\rnk(\phi_{2})$ and
$\free(\phi)\subseteq\var(\psi_{2})$. In particular we obtain the
inclusion $\free(\phi_{1})\subseteq\var(\psi_{2})$ and applying
proposition~(\ref{logic:prop:FOPL:substrank:changeofvar}) once more,
from  $\rnk(\phi_{1})\leq\rnk(\phi_{2})=|\var(\psi_{2})|$ we obtain
the existence of $\psi_{1}\sim\phi_{1}$ such that
$|\var(\psi_{1})|=\rnk(\phi_{1})$ and
$\var(\psi_{1})\subseteq\var(\psi_{2})$. It follows that:
    \begin{eqnarray*}
    \rnk(\phi)&\leq&|\var(\psi_{1}\to\psi_{2})|\\
    &=&|\var(\psi_{1})\cup\var(\psi_{2})|\\
    \var(\psi_{1})\subseteq\var(\psi_{2})\ \rightarrow&=&|\var(\psi_{2})|\\
    &=&\rnk(\phi_{2})\\
    &=&\max(\,|\free(\phi)|\,,\,\rnk(\phi_{1})\,,\,\rnk(\phi_{2})\,)
    \end{eqnarray*}
The case $\rnk(\phi_{2})\leq\rnk(\phi_{1})$ is dealt with similarly.
\end{proof}


\begin{prop}\label{logic:prop:FOPL:substrank:quant}
Let $V$ be a set and $\phi\in\pv$ of the form $\phi=\forall
x\phi_{1}$ where $\phi_{1}\in\pv$ and $x\in V$. Then the
substitution ranks of $\phi$ and $\phi_{1}$ satisfy:
    \[
    \rnk(\phi)=\rnk(\phi_{1})+\epsilon
    \]
where $\epsilon\in 2=\{0,1\}$ is given by the equivalence
$\epsilon=1$ \ifand:
    \[
    (\,x\not\in\free(\phi_{1})\,)\land(\,|\free(\phi_{1})|
    =\rnk(\phi_{1})\,)
    \]
\end{prop}
\begin{proof}
Let $\phi=\forall x\phi_{1}$ where $\phi_{1}\in\pv$ and $x\in V$.
Define $\epsilon = \rnk(\phi)-\rnk(\phi_{1})$. Then $\epsilon$ is an
integer, possibly negative. In order to prove that $\epsilon\in 2$
it is therefore sufficient to prove that the following inequalities
hold:
    \begin{equation}\label{logic:eqn:FOPL:substrank:quant:eq1}
    \rnk(\phi_{1})\leq\rnk(\phi)\leq\rnk(\phi_{1})+1
    \end{equation}
This will be the first part of our proof. Next we shall show the
equivalence:
    \begin{equation}\label{logic:eqn:FOPL:substrank:quant:eq2}
    (\epsilon =1)\ \Leftrightarrow\
    (\,x\not\in\free(\phi_{1})\,)\land(\,|\free(\phi_{1})|
    =\rnk(\phi_{1})\,)
    \end{equation}
So first we show that $\rnk(\phi_{1})\leq\rnk(\phi)$. Let
$\psi\sim\phi$ where $\sim$ denotes the substitution congruence on
\pv. We need to show that $\rnk(\phi_{1})\leq|\var(\psi)|$. Using
theorem~(\ref{logic:the:sub:congruence:charac}) of
page~\pageref{logic:the:sub:congruence:charac}, from the equivalence
$\psi\sim\phi$ we see that $\psi$ is either of the form
$\psi=\forall x\psi_{1}$ where $\psi_{1}\sim\phi_{1}$, or $\psi$ is
of the form $\psi=\forall y\psi_{1}$ where
$\psi_{1}\sim\phi_{1}[y\!:\!x]$, $x\neq y$ and
$y\not\in\free(\phi_{1})$. First we assume that $\psi=\forall
x\psi_{1}$ where $\psi_{1}\sim\phi_{1}$. Then we have the following
inequalities:
    \begin{eqnarray*}
    \rnk(\phi_{1})&\leq&|\var(\psi_{1})|\\
    &\leq&|\{x\}\cup\var(\psi_{1})|\\
    &=&|\var(\forall x\psi_{1})|\\
    &=&|\var(\psi)|
    \end{eqnarray*}
Next we assume that $\psi=\forall y\psi_{1}$ where
$\psi_{1}\sim\phi_{1}[y\!:\!x]$, $x\neq y$ and
$y\not\in\free(\phi_{1})$. The permutation $[y\!:\!x]$ being
injective, from
proposition~(\ref{logic:prop:sub:congruence:injective:substitution})
we obtain $\psi_{1}^{*}\sim\phi_{1}$ where
$\psi_{1}^{*}=\psi_{1}[y\!:\!x]$. Furthermore, defining
$\psi^{*}=\forall x\psi_{1}^{*}$ we have $\psi=\psi^{*}[y\!:\!x]$.
Using the injectivity of $[y\!:\!x]$ once more and
proposition~(\ref{logic:prop:var:of:substitution}) we obtain:
    \[
    |\var(\psi)|=|\var(\psi^{*}[y\!:\!x])|=
    |\,[y\!:\!x](\var(\psi^{*}))\,|=|\var(\psi^{*})|
    \]
So we need to prove that $\rnk(\phi_{1})\leq|\var(\psi^{*})|$ where
$\psi^{*}=\forall x\psi_{1}^{*}$ and $\psi_{1}^{*}\sim\phi_{1}$.
Hence we are back to our initial case and we have proved that
$\rnk(\phi_{1})\leq\rnk(\phi)$. Next we show that
$\rnk(\phi)\leq\rnk(\phi_{1})+1$. So let $\psi_{1}\sim\phi_{1}$. We
need to show that $\rnk(\phi)-1\leq|\var(\psi_{1})|$ or equivalently
that $\rnk(\phi)\leq|\var(\psi_{1})|+1$:
    \begin{eqnarray*}
    \rnk(\phi)&=&\rnk(\forall x\phi_{1})\\
    \forall x\phi_{1}\sim\forall x\psi_{1}\ \rightarrow
    &\leq&|\var(\forall x\psi_{1})|\\
    &=&|\{x\}\cup\var(\psi_{1})|\\
    &\leq&|\var(\psi_{1})|+1
    \end{eqnarray*}
So we are done proving the
inequalities~(\ref{logic:eqn:FOPL:substrank:quant:eq1}). We shall
complete the proof of this proposition by showing the
equivalence~(\ref{logic:eqn:FOPL:substrank:quant:eq2}). First we
show $\Rightarrow$\,: so we assume that $\epsilon=1$. We need to
show that $x\not\in\free(\phi_{1})$ and furthermore that
$|\free(\phi_{1})|=\rnk(\phi_{1})$. First we show that
$x\not\in\free(\phi_{1})$. So suppose to the contrary that
$x\in\free(\phi_{1})$. We shall obtain a contradiction by showing
$\epsilon=0$, that is $\rnk(\phi_{1})=\rnk(\phi)$. We already know
that $\rnk(\phi_{1})\leq\rnk(\phi)$. So we need to show that
$\rnk(\phi)\leq\rnk(\phi_{1})$. So let $\psi_{1}\sim\phi_{1}$. We
need to show that $\rnk(\phi)\leq|\var(\psi_{1})|$. However, from
$\psi_{1}\sim\phi_{1}$ and
proposition~(\ref{logic:prop:sub:congruence:freevar}) we obtain
$\free(\psi_{1})=\free(\phi_{1})$ and in particular
$x\in\free(\psi_{1})$. It follows that:
    \begin{eqnarray*}
    \rnk(\phi)&=&\rnk(\forall x\phi_{1})\\
    \forall x\phi_{1}\sim\forall x\psi_{1}\ \rightarrow
    &\leq&|\var(\forall x\psi_{1})|\\
    &=&|\{x\}\cup\var(\psi_{1})|\\
    x\in\free(\psi_{1})\subseteq\var(\psi_{1})\ \rightarrow
    &=&|\var(\psi_{1})|\\
    \end{eqnarray*}
This is our desired contradiction and we conclude that
$x\not\in\free(\phi_{1})$. It remains to show that
$|\free(\phi_{1})|=\rnk(\phi_{1})$. So suppose this equality does
not hold. We shall obtain a contradiction by showing $\epsilon=0$,
that is $\rnk(\phi)\leq\rnk(\phi_{1})$. So let
$\psi_{1}\sim\phi_{1}$. We need to show once again that
$\rnk(\phi)\leq|\var(\psi_{1})|$. However, having assumed the
equality $|\free(\phi_{1})|=\rnk(\phi_{1})$ does not hold, from
proposition~(\ref{logic:prop:FOPL:substrank:basic:ineq}) we obtain
$|\free(\phi_{1})|<\rnk(\phi_{1})$ and consequently from
$\psi_{1}\sim\phi_{1}$ we have:
    \[
    |\free(\psi_{1})|=|\free(\phi_{1})|
    <\rnk(\phi_{1})
    \leq|\var(\psi_{1})|
    \]
It follows that the set $\var(\psi_{1})\setminus\free(\psi_{1})$
cannot be empty, and there exists
$y\in\var(\psi_{1})\setminus\free(\psi_{1})$. From
$x\not\in\free(\phi_{1})$ and
$y\not\in\free(\psi_{1})=\free(\phi_{1})$ using
proposition~(\ref{logic:prop:FOPL:freesubcong:xy:not:free}) we
obtain the equivalence $\forall x\phi_{1}\sim\forall y\phi_{1}$.
Hence we also have the equivalence $\forall x\phi_{1}\sim\forall
y\psi_{1}$ and consequently:
    \begin{eqnarray*}
    \rnk(\phi)&=&\rnk(\forall x\phi_{1})\\
        \forall x\phi_{1}\sim\forall y\psi_{1}\ \rightarrow
        &\leq&|\var(\forall y\psi_{1})|\\
        &=&|\{y\}\cup\var(\psi_{1})|\\
        y\in\var(\psi_{1})\ \rightarrow
        &=&|\var(\psi_{1})|
    \end{eqnarray*}
which is our desired contradiction and we conclude that
$|\free(\phi_{1})|=\rnk(\phi_{1})$. This completes our proof of
$\Rightarrow$ in the
equivalence~(\ref{logic:eqn:FOPL:substrank:quant:eq2}). We now prove
$\Leftarrow$\,: so we assume that $x\not\in\free(\phi_{1})$ and
$|\free(\phi_{1})|=\rnk(\phi_{1})$. We need to show that
$\epsilon=1$, that is $\rnk(\phi)=\rnk(\phi_{1})+1$. We already have
the inequality $\rnk(\phi)\leq\rnk(\phi_{1})+1$. So it remains to
show that $\rnk(\phi_{1})+1\leq\rnk(\phi)$ or equivalently
$\rnk(\phi_{1})<\rnk(\phi)$. So let $\psi\sim\phi$. We need to show
that $\rnk(\phi_{1})<|\var(\psi)|$. Once again, using
theorem~(\ref{logic:the:sub:congruence:charac}) of
page~\pageref{logic:the:sub:congruence:charac}, from the equivalence
$\psi\sim\phi$ we see that $\psi$ is either of the form
$\psi=\forall x\psi_{1}$ where $\psi_{1}\sim\phi_{1}$, or $\psi$ is
of the form $\psi=\forall y\psi_{1}$ where
$\psi_{1}\sim\phi_{1}[y\!:\!x]$, $x\neq y$ and
$y\not\in\free(\phi_{1})$. First we assume that $\psi=\forall
x\psi_{1}$ where $\psi_{1}\sim\phi_{1}$. Hence
$\rnk(\phi_{1})\leq|\var(\psi_{1})|$ and we shall distinguish two
further cases: first we assume that
$\rnk(\phi_{1})=|\var(\psi_{1})|$. Having assumed that
$|\free(\phi_{1})|=\rnk(\phi_{1})$ we obtain:
    \[
    |\free(\psi_{1})|=|\free(\phi_{1})|=\rnk(\phi_{1})=|\var(\psi_{1})|
    \]
from which we see that $\var(\psi_{1})=\free(\psi_{1})$ and
consequently it follows that
$x\not\in\free(\phi_{1})=\free(\psi_{1})=\var(\psi_{1})$. Hence we
see that:
    \begin{eqnarray*}
    \rnk(\phi_{1})&\leq&|\var(\psi_{1})|\\
    x\not\in\var(\psi_{1})\ \rightarrow
    &<&|\{x\}\cup\var(\psi_{1})|\\
    &=&|\var(\forall x\psi_{1})|\\
    &=&|\var(\psi)|
    \end{eqnarray*}
So we now assume that $\rnk(\phi_{1})<|\var(\psi_{1})|$, in which
case we obtain:
    \begin{eqnarray*}
    \rnk(\phi_{1})&<&|\var(\psi_{1})|\\
    &\leq&|\{x\}\cup\var(\psi_{1})|\\
    &=&|\var(\forall x\psi_{1})|\\
    &=&|\var(\psi)|
    \end{eqnarray*}
We now consider the case when $\psi=\forall y\psi_{1}$ where
$\psi_{1}\sim\phi_{1}[y\!:\!x]$, $x\neq y$ and
$y\not\in\free(\phi_{1})$. Once again, defining
$\psi_{1}^{*}=\psi_{1}[y\!:\!x]$ and $\psi^{*}=\forall
x\psi_{1}^{*}$ we obtain the equivalence $\psi_{1}^{*}\sim\phi_{1}$
and $|\var(\psi^{*})|=|\var(\psi)|$. So we need to prove that
$\rnk(\phi_{1})<|\var(\psi^{*})|$ knowing that
$\psi_{1}^{*}\sim\phi_{1}$ which follows from our initial case.
\end{proof}

As already discussed, our objective is to show the existence of a
$\psi\in{\bf P}(W)$ such that $\bar{\sigma}\circ{\cal M}(\phi)={\cal
M}(\psi)$ provided we have $\rnk(\bar{\sigma}\circ{\cal
M}(\phi))\leq|W|$. Assuming we prove this result, it will be
important for us to have a way of telling whether the condition
$\rnk(\bar{\sigma}\circ{\cal M}(\phi))\leq|W|$ is satisfied. The
following proposition allows us to do this. Given any map
$\sigma:V\to W$, the associated variable substitution
$\sigma:\pv\to{\bf P}(W)$ cannot increase the substitution rank of a
formula. Hence:
    \[
    \rnk(\bar{\sigma}\circ{\cal M}(\phi))\leq \rnk({\cal
    M}(\phi))=\rnk(\phi)\leq|\var(\phi)|\leq|V|
    \]
So we see for example that the condition $|V|\leq|W|$ is a
sufficient condition. Note the following proposition makes no
assumption on the validity of $\sigma$ for~$\phi$.
\begin{prop}\label{logic:prop:FOPL:subst:rank:substitution}
Let $V,W$ be sets and $\sigma:V\to W$ be a map. Let $\phi\in\pv$:
    \[
    \rnk(\sigma(\phi))\leq\rnk(\phi)
    \]
where $\sigma:\pv\to{\bf P}(W)$ denotes the associated substitution
mapping.
\end{prop}
\begin{proof}
We shall prove $\rnk(\sigma(\phi))\leq\rnk(\phi)$ by structural
induction using theorem~(\ref{logic:the:proof:induction}) of
page~\pageref{logic:the:proof:induction}. First we assume that
$\phi=(x\in y)$ for some $x,y\in V$. Then we have:
    \[
    \rnk(\sigma(\phi))=|\,\{\sigma(x),\sigma(y)\}\,|\leq|\,\{x,y\}\,|=\rnk(\phi)
    \]
Next we assume that $\phi=\bot$. Then
$\rnk(\sigma(\phi))=\rnk(\bot)=0=\rnk(\phi)$. So we now assume that
$\phi=\phi_{1}\to\phi_{2}$ where $\phi_{1},\phi_{2}\in\pv$ satisfy
our property:
    \begin{eqnarray*}
    \rnk(\sigma(\phi))&=&\rnk(\sigma(\phi_{1}\to\phi_{2}))\\
    &=&\rnk(\sigma(\phi_{1})\to\sigma(\phi_{2}))\\
    \mbox{prop.~(\ref{logic:prop:FOPL:substrank:impl})}\ \rightarrow
    &=&\max(\,|\,\free(\sigma(\phi))\,|\,,\,
    \rnk(\sigma(\phi_{1}))\,,\,\rnk(\sigma(\phi_{2}))\,)\\
    &\leq&\max(\,|\,\free(\sigma(\phi))\,|\,,\,
    \rnk(\phi_{1})\,,\,\rnk(\phi_{2})\,)\\
    \mbox{prop.~(\ref{logic:prop:freevar:of:substitution:inclusion})}\ \rightarrow
    &\leq&\max(\,|\,\sigma(\free(\phi))\,|\,,\,
    \rnk(\phi_{1})\,,\,\rnk(\phi_{2})\,)\\
    &\leq&\max(\,|\free(\phi)|\,,\,
    \rnk(\phi_{1})\,,\,\rnk(\phi_{2})\,)\\
    \mbox{prop.~(\ref{logic:prop:FOPL:substrank:impl})}\ \rightarrow
    &=&\rnk(\phi)
    \end{eqnarray*}
Finally we assume that $\phi=\forall x\phi_{1}$ where $x\in V$ and
$\phi_{1}\in\pv$ satisfy our induction property. Then
$\sigma(\phi)=\forall\sigma(x)\sigma(\phi_{1})$. Let $\epsilon
=\rnk(\phi)-\rnk(\phi_{1})$ and let
$\eta=\rnk(\sigma(\phi))-\rnk(\sigma(\phi_{1}))$. From
proposition~(\ref{logic:prop:FOPL:substrank:quant}) we know that
$\epsilon,\eta\in 2$. We shall distinguish two cases: first we
assume that $\eta=0$. Then we have:
    \[
    \rnk(\sigma(\phi))=\rnk(\sigma(\phi_{1}))\leq\rnk(\phi_{1})\leq\rnk(\phi)
    \]
Next we assume that $\eta=1$. We shall distinguish two further
cases: if $\epsilon=1$\,:
    \[
    \rnk(\sigma(\phi))=1+\rnk(\sigma(\phi_{1}))\leq1+\rnk(\phi_{1})=\rnk(\phi)
    \]
So it remains to deal with the last possibility when $\eta=1$ and
$\epsilon=0$. Using
proposition~(\ref{logic:prop:FOPL:substrank:quant}), from
$\epsilon=0$ we see that $x\in\free(\phi_{1})$ or
$|\free(\phi_{1})|<\rnk(\phi_{1})$. So we shall distinguish two
further cases. These cases may not be exclusive of one another but
we don't need them to be. So first we assume $x\in\free(\phi_{1})$.
Using proposition~(\ref{logic:prop:FOPL:substrank:quant}) again,
from $\eta=1$ we obtain
$|\free(\sigma(\phi_{1}))|=\rnk(\sigma(\phi_{1}))$ and furthermore
$\sigma(x)\not\in\free(\sigma(\phi_{1}))$. Having assumed that
$x\in\free(\phi_{1})$ it follows that $\sigma(x)$ is an element of
$\sigma(\free(\phi_{1}))$ but not an element of
$\free(\sigma(\phi_{1}))$. Hence, we see that the inclusion
$\free(\sigma(\phi_{1}))\subseteq\sigma(\free(\phi_{1}))$ which we
know is true from
proposition~(\ref{logic:prop:freevar:of:substitution:inclusion}) is
in fact a strict inclusion. So we have a strict inequality between
the finite cardinals
$|\free(\sigma(\phi_{1}))|<|\sigma(\free(\phi_{1}))|$. Thus:
    \begin{eqnarray*}
    \rnk(\sigma(\phi))&=&1+\rnk(\sigma(\phi_{1}))\\
    |\free(\sigma(\phi_{1}))|=\rnk(\sigma(\phi_{1}))\ \rightarrow
    &=&1+|\free(\sigma(\phi_{1}))|\\
    |\free(\sigma(\phi_{1}))|<|\sigma(\free(\phi_{1}))|\ \rightarrow
    &\leq&|\sigma(\free(\phi_{1}))|\\
    &\leq&|\free(\phi_{1})|\\
    \mbox{prop.~(\ref{logic:prop:FOPL:substrank:basic:ineq})}\ \rightarrow
    &\leq&\rnk(\phi_{1})\\
    \epsilon=0\ \rightarrow&=&\rnk(\phi)
    \end{eqnarray*}
So we now assume that $|\free(\phi_{1})|<\rnk(\phi_{1})$. In this
case we have:
    \begin{eqnarray*}
    \rnk(\sigma(\phi))&=&1+\rnk(\sigma(\phi_{1}))\\
    |\free(\sigma(\phi_{1}))|=\rnk(\sigma(\phi_{1}))\ \rightarrow
    &=&1+|\free(\sigma(\phi_{1}))|\\
    \mbox{prop.~(\ref{logic:prop:freevar:of:substitution:inclusion})}\ \rightarrow
    &\leq&1+|\sigma(\free(\phi_{1}))|\\
    &\leq&1+|\free(\phi_{1})|\\
    |\free(\phi_{1})|<\rnk(\phi_{1})\ \rightarrow
    &\leq&\rnk(\phi_{1})\\
    \epsilon=0\ \rightarrow&=&\rnk(\phi)
    \end{eqnarray*}
\end{proof}
\begin{prop}\label{logic:prop:FOPL:substrank:closed:fullrank}
Let $V$ be a set and $n\in\N$ such that $n\leq|V|$. Then there
exists a formula $\phi\in\pv$ such that $\free(\phi)=\emptyset$ and
$\rnk(\phi)=n$.
\end{prop}
\begin{proof}
From the inequality $n\leq |V|$ there exists an injective map
$u:n\to V$. Define $\psi_{0}=\bot$ and $\psi_{k+1}=(u(k)\in
u(k))\to\psi_{k}$ for $k\in n$. Let $\psi=\psi_{n}$. For example, in
the case when $n=2$ the formula $\psi$ is given by:
    \[
    \psi=(\,u(1)\in u(1)\,)\to[\,(\,u(0)\in u(0)\,)\to\bot\,]
    \]
Define $\phi_{0}=\psi$ and $\phi_{k+1}= \forall u(k)\, \phi_{k}$ for
$k\in n$. Let $\phi=\phi_{n}$. We shall complete the proof of this
proposition by showing that $\free(\phi)=\emptyset$ and
$\rnk(\phi)=n$. First we prove that $\free(\phi)=\emptyset$. From
the recursion $\psi_{k+1}=(u(k)\in u(k))\to\psi_{k}$\,:
    \[
    \free(\psi_{k+1})=\{u(k)\}\cup\free(\psi_{k})\ ,\ k\in n
    \]
Since $\free(\psi_{0})=\free(\bot)=\emptyset$ an easy induction
argument shows that we have $\free(\psi_{k})= u[k]$ for all $k\in
n+1$. Note that $u[k]$ denotes the image of the set $k$ by $u$, i.e.
the range of the restricted map $u_{|k}$. We are using the square
brackets~$[\,.\,]$ to avoid any confusion between $u[k]$ and $u(k)$.
In particular for $k=n$ we see that
$\free(\psi)=\free(\psi_{n})=u[n]=\rng(u)$. Furthermore, from the
recursion formula $\phi_{k+1}= \forall u(k)\, \phi_{k}$ we obtain
the equality:
    \[
    \free(\phi_{k+1})=\free(\phi_{k})\setminus\{u(k)\}\ ,\ k\in n
    \]
Since $\free(\phi_{0})=\free(\psi)=\rng(u)$, an easy induction
argument shows that $\free(\phi_{k})=\rng(u)\setminus u[k]$ for all
$k\in n+1$. In particular for $k=n$ we have the equality
$\free(\phi)=\free(\phi_{n})=\rng(u)\setminus u[n]=\emptyset$ as
requested. So it remains to show that $rnk(\phi)=n$. Using
proposition~(\ref{logic:prop:FOPL:substrank:quant}), from the
recursion formula $\phi_{k+1}= \forall u(k)\, \phi_{k}$ and the fact
that $u(k)\in \rng(u)\setminus u[k]=\free(\phi_{k})$ we obtain
$\rnk(\phi_{k})=\rnk(\phi_{k+1})$. Note that $u(k)\in
\rng(u)\setminus u[k]$ is true since $u$ is injective. Hence we see
that $\rnk(\psi)=\rnk(\phi_{0})=\rnk(\phi)$ and it is sufficient to
prove that $\rnk(\psi)=n$. However having proved $\free(\psi_{k})=
u[k]$, it follows from the injectivity of $u$ that
$|\free(\psi_{k})|=k$ for all $k\in n+1$. From the recursion formula
$\psi_{k+1}=(u(k)\in u(k))\to\psi_{k}$ using
proposition~(\ref{logic:prop:FOPL:substrank:impl}) we obtain:
    \[
    \rnk(\psi_{k+1})=\max(\,|\free(\psi_{k+1})|\,,\,\rnk(u(k)\in
    u(k))\,,\,\rnk(\psi_{k})\,)
    \]
that is $\rnk(\psi_{k+1})=\max(k+1,1,\rnk(\psi_{k}))$ for all $k\in
n$. Since $\rnk(\psi_{0})=0$, a simple induction argument shows that
$\rnk(\psi_{k})=k$ for all $k\in n+1$. In particular for $k=n$ we
obtain $\rnk(\psi)=\rnk(\psi_{n})=n$ as requested.
\end{proof}
