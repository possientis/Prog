We now show that equivalence between formulas is invariant under
injective substitution. The following proposition is the counterpart
of proposition~(\ref{logic:prop:strong:injective:substitution}) of
the strong substitution congruence.

\begin{prop}\label{logic:prop:sub:congruence:injective:substitution}
Let $V$ and $W$ be sets and $\sigma:V\to W$ be an injective map.
Let~$\sim$ be the substitution congruence both on \pv\ and ${\bf
P}(W)$. Then:
    \[
    \phi\sim\psi\ \Rightarrow\ \sigma(\phi)\sim\sigma(\psi)
    \]
for all $\phi,\psi\in\pv$, where $\sigma:\pv\to{\bf P}(W)$ is also
the substitution mapping.
\end{prop}
\begin{proof}
Let $\equiv$ be the relation on \pv\ defined by $\phi\equiv\psi\
\Leftrightarrow\ \sigma(\phi)\sim\sigma(\psi)$. We need to show that
$\phi\sim\psi\ \Rightarrow\ \phi\equiv\psi$ or equivalently that the
inclusion $\sim\,\subseteq\,\equiv$ holds. Since $\sim$ is the
substitution congruence on \pv, it is the smallest congruence on
\pv\ which contains the set $R_{0}$ of
definition~(\ref{logic:def:sub:congruence}). In order to show the
inclusion $\sim\,\subseteq\,\equiv$ it is therefore sufficient to
show that $\equiv$ is a congruence on \pv\ such that
$R_{0}\subseteq\,\equiv$. However, we already know from
proposition~(\ref{logic:prop:substitution:congruence}) that $\equiv$
is a congruence on \pv. So it remains to show that
$R_{0}\subseteq\,\equiv$. So let $\phi_{1}\in\pv$ and $x,y\in V$ be
such that $x\neq y$ and $y\not\in\free(\phi_{1})$. Define
$\phi=\forall x\phi_{1}$ and $\psi=\forall y\,\phi_{1}[y\!:\!x]$. We
need to show that $\phi\equiv\psi$ or equivalently that
$\sigma(\phi)\sim\sigma(\psi)$. In order to do so, it is sufficient
to show that the ordered pair $(\sigma(\phi),\sigma(\psi))$ belongs
to the generator $R_{0}'$ of the substitution congruence on ${\bf
P}(W)$ as per definition~(\ref{logic:def:sub:congruence}). In other
words, it is sufficient to show the existence of $\phi_{1}'\in{\bf
P}(W)$ and $x',y'\in W$ with $x'\neq y'$ and
$y'\not\in\free(\phi_{1}')$, such that $\sigma(\phi)=\forall
x'\phi_{1}'$ and $\sigma(\psi)=\forall y'\,\phi_{1}'[y'\!:\!x']$.
Take $\phi_{1}'=\sigma(\phi_{1})\in{\bf P}(W)$ together with
$x'=\sigma(x)\in W$ and $y'=\sigma(y)\in W$. Then:
    \[
    \sigma(\phi)=\sigma(\forall x\phi_{1})\\
        =\forall\sigma(x)\,\sigma(\phi_{1})\\
        =\forall x'\phi_{1}'
    \]
Furthermore, from
proposition~(\ref{logic:prop:FUAP:singlevar:composition:injective})
we have $\sigma\circ[y\!:\!x]=[\sigma(y)\!:\!\sigma(x)]\circ\sigma$
and so:
    \begin{eqnarray*}
    \sigma(\psi)&=&\sigma(\forall y\,\phi_{1}[y\!:\!x])\\
    &=&\forall\sigma(y)\,\sigma(\phi_{1}[y\!:\!x])\\
    &=&\forall y'\,\sigma([y\!:\!x](\phi_{1}))\\
    &=&\forall y'\,\sigma\circ [y\!:\!x]\,(\phi_{1})\\
    &=&\forall y'\, [\sigma(y)\!:\!\sigma(x)]\circ\sigma\,(\phi_{1})\\
    &=&\forall y'\,[y'\!:\!x'](\phi_{1}')\\
    &=&\forall y'\,\phi_{1}'[y'\!:\!x']
    \end{eqnarray*}
So it remains to show that $x'\neq y'$ and
$y'\not\in\free(\phi_{1}')$. Since $\sigma:V\to W$ is an injective
map, $x'\neq y'$ follows immediately from $x\neq y$. We now show
that $y'\not\in\free(\phi_{1}')$. So suppose to the contrary that
$y'\in\free(\phi_{1}')$. We shall arrive at a contradiction. Since
$\phi_{1}'=\sigma(\phi_{1})$, from
proposition~(\ref{logic:prop:freevar:of:substitution}) we have:
 \[ \free(\phi_{1}')=\{\sigma(u):u\in\free(\phi_{1})\}
 \]
It follows that there exists $u\in\free(\phi_{1})$ such that
$y'=\sigma(u)$. However, $y'=\sigma(y)$ and $\sigma:V\to W$ is an
injective map. Hence we see that $u=y$ and consequently
$y\in\free(\phi_{1})$ which contradicts our initial assumption of
$y\not\in\free(\phi_{1})$.
\end{proof}
