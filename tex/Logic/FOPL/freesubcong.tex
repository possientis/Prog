A lot of the work which was done for the strong $\alpha$-equivalence
needs to be done all over again for the $\alpha$-equivalence. 
We start by showing that equivalent formulas have the
same free variables. The following proposition is the counterpart of
proposition~(\ref{logic:prop:strong:freevar}) of the strong 
$\alpha$-equivalence.

\begin{prop}\label{logic:prop:sub:congruence:freevar}
Let $\sim$ denote the $\alpha$-equivalence on \pv\ where $V$ is a
set. Then for all $\phi,\psi\in\pv$ we have the implication:
    \[
    \phi\sim\psi\ \Rightarrow\ \free(\phi)=\free(\psi)
    \]
\end{prop}

\noindent
\begin{proof}
Let $\equiv$ be the relation on \pv\ defined by $\phi\equiv\psi\
\Leftrightarrow\ \free(\phi)=\free(\psi)$. We need to show that
$\phi\sim\psi\ \Rightarrow\ \phi\equiv\psi$ or equivalently that the
inclusion $\sim\,\subseteq\,\equiv$ holds. Since $\sim$ is the
$\alpha$-equivalence on \pv, it is the smallest congruence on
\pv\ which contains the set $R_{1}$ of
proposition~(\ref{logic:prop:sub:congruence:from:admissible}). In
order to show the inclusion $\sim\,\subseteq\,\equiv$ it is
therefore sufficient to show that $\equiv$ is a congruence on \pv\
such that $R_{1}\subseteq\,\equiv$. However, we already know from
proposition~(\ref{logic:prop:congruence:freevar}) that $\equiv$ is a
congruence on \pv. So it remains to show that
$R_{1}\subseteq\,\equiv$. So let $\phi\in\pv$ and $\sigma:V\to V$ be
an admissible substitution for $\phi$. We need to show that
$\phi\equiv\sigma(\phi)$ or equivalently that
$\free(\phi)=\free(\sigma(\phi))$. However, since $\sigma$ is valid
for $\phi$, from
proposition~(\ref{logic:prop:FOPL:valid:free:commute}) we have
$\free(\sigma(\phi))=\sigma(\free(\phi))$ so we need to show that
$\free(\phi)=\sigma(\free(\phi))$ which follows from the fact that
$\sigma(u)=u$ for all $u\in\free(\phi)$.
\end{proof}

\begin{prop}\label{logic:prop:FOPL:freesubcong:xy:not:free}
Let $\sim$ denote the $\alpha$-equivalence on \pv\ where $V$ is a
set. Then for all $\phi_{1}\in\pv$ and $x,y\in V$ such that
$x,y\not\in\free(\phi_{1})$ we have:
    \[
    \forall x\phi_{1}\sim\forall y\phi_{1}
    \]
\end{prop}

\noindent
\begin{proof}
From $y\not\in\free(\phi_{1})$ and
definition~(\ref{logic:def:sub:congruence}) we see that $\forall
x\phi_{1}\sim\forall y\phi_{1}[y\!:\!x]$. So we need to show that
$\forall y\phi_{1}[y\!:\!x]\sim\forall y\phi_{1}$. Hence it is
sufficient to prove that $\phi_{1}[y\!:\!x]\sim\phi_{1}$. Using
proposition~(\ref{logic:prop:admissible:sub:congruence}), we simply
need to argue that $[y\!:\!x]$ is an admissible substitution for
$\phi_{1}$. Being injective, from
proposition~(\ref{logic:prop:FOPL:valid:injective}) it is a valid
substitution for $\phi_{1}$. So it remains to show that
$[y\!:\!x](u)=u$ for all $u\in\free(\phi_{1})$ which follows
immediately from $x,y\not\in\free(\phi_{1})$.
\end{proof}
