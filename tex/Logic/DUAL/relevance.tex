Looking back at
definition~(\ref{logic:def:FOPL:model:valuation:function}), if $M$
is a model of \pv\ with model valuation function $\beta:\pv\to{\cal
P}(M^{V})$, given a variables assignment $a:V\to M$ we have:
    \[
    \beta(\forall x\phi_{1})(a)=\min\left\{\,\beta(\phi_{1})(b)\ :\  b=a\mbox{\ on\
                    }V_{x}\,\right\}
    \]
So the truth value of $\phi=\forall x\phi_{1}$ under the assignment
$a:V\to M$ is determined by the set of all assignments $b:V\to M$
which coincide with $a$ on the set $V\setminus\{x\}$. In particular,
the value $a(x)$ i.e. the interpretation of the variable $x$ under
the assignment $a$ is not relevant. More generally, it is easy to
believe that unless $x\in V$ is a free variable of the formula
$\phi\in\pv$, the truth value $\beta(\phi)(a)$ is not impacted by
the value $a(x)$. In other words, we expect $\beta(\phi)(a)$ to
solely depend on the restriction $a_{|\free(\phi)}$. The following
proposition establishes that fact. We chose to entitle this section
{\em The Relevance Lemma} following lecture~$26$ of the online
course of Prof. Arindama Singh~\cite{Singh} which offers a proof of
the result on the YouTube link
\texttt{http://nptel.iitm.ac.in/courses/111106052/26}.

\index{relevance@Relevance lemma}
\begin{prop}\label{logic:prop:FOPL:model:assignment:support}
Let $V$ be a set and $M$ be a model of\, \pv\ with model valuation
function $\beta:\pv\to{\cal P}(M^{V})$. Then for all $\phi\in\pv$
and $a,b:V\to M$:
    \begin{equation}\label{logic:eqn:FOPL:model:assignment:support:1}
    a_{|\free(\phi)}=b_{|\free(\phi)}\ \Rightarrow\
    \beta(\phi)(a)=\beta(\phi)(b)
    \end{equation}
\end{prop}
\begin{proof}
For all $\phi\in\pv$ we need to show that for all assignments
$a,b:V\to M$, the
implication~(\ref{logic:eqn:FOPL:model:assignment:support:1}) is
true. We shall do so by structural induction, using
theorem~(\ref{logic:the:proof:induction}) of
page~\pageref{logic:the:proof:induction}. First we assume that
$\phi=(x\in y)$ where $x,y\in V$. Let $a,b:V\to M$ be two
assignments such that $a=b$ on $\free(\phi)$. Then in particular we
have $a(x)=b(x)$ and $a(y)=b(y)$. We need to show that
$\beta(\phi)(a)=\beta(\phi)(b)$, which goes as follows, denoting
$r\subseteq M\times M$ the relation on $M$:
    \begin{eqnarray*}
    \beta(\phi)(a)&=&\beta(x\in y)(a)\\
    &=&1_{r}(a(x),a(y))\\
    &=&1_{r}(b(x),b(y))\\
    &=&\beta(x\in y)(b)\\
    &=&\beta(\phi)(b)\\
    \end{eqnarray*}
Next we assume that $\phi=\bot$. Let $a,b:V\to M$ be two
assignments. Then $a$ and $b$ vacuously coincide on $\free(\phi)$.
In any case we have $\beta(\bot)(a)=0=\beta(\bot)(b)$. So we now
assume that $\phi=\phi_{1}\to\phi_{2}$ where
$\phi_{1},\phi_{2}\in\pv$ are such that the
implication~(\ref{logic:eqn:FOPL:model:assignment:support:1}) is
true for all $a,b:V\to M$. We need to show the same is true of
$\phi$. So let $a,b:V\to M$ be two assignments such that $a=b$ on
$\free(\phi)=\free(\phi_{1})\cup\free(\phi_{2})$. We need to show
that $\beta(\phi)(a)=\beta(\phi)(b)$:
    \begin{eqnarray*}
    \beta(\phi)(a)&=&\beta(\phi_{1}\to\phi_{2})(a)\\
    &=&\beta(\phi_{1})(a)\to\beta(\phi_{2})(a)\\
    a=b\mbox{\ on\ }\free(\phi_{1})\ \rightarrow
    &=&\beta(\phi_{1})(b)\to\beta(\phi_{2})(a)\\
    a=b\mbox{\ on\ }\free(\phi_{2})\ \rightarrow
    &=&\beta(\phi_{1})(b)\to\beta(\phi_{2})(b)\\
    &=&\beta(\phi_{1}\to\phi_{2})(b)\\
    &=&\beta(\phi)(b)\\
    \end{eqnarray*}
Finally we assume that $\phi=\forall x\phi_{1}$ where $x\in V$ and
$\phi_{1}\in\pv$ is such that the
implication~(\ref{logic:eqn:FOPL:model:assignment:support:1}) is
true for all $a,b:V\to M$. We need to show the same is true of
$\phi$. So let $a,b:V\to M$ be two assignments such that $a=b$ on
$\free(\phi)$. We need to show that $\beta(\phi)(a)=\beta(\phi)(b)$
which goes as follows:
    \begin{eqnarray*}
    \beta(\phi)(a)&=&\beta(\forall x\phi_{1})(a)\\
    &=&\min\left\{\,\beta(\phi_{1})(c)\ :\ c=a\mbox{\ on\
    }V_{x}\,\right\}\\
    \mbox{A: to be proved}\ \rightarrow
    &=&\min\left\{\,\beta(\phi_{1})(d)\ :\ d=b\mbox{\ on\
    }V_{x}\,\right\}\\
    &=&\beta(\forall x\phi_{1})(b)\\
    &=&\beta(\phi)(b)
    \end{eqnarray*}
So it remains to prove point A, for which it is sufficient to show
the set equality:
    \[
    X=\left\{\,\beta(\phi_{1})(c)\ :\ c=a\mbox{\ on\
    }V_{x}\,\right\}=\left\{\,\beta(\phi_{1})(d)\ :\ d=b\mbox{\ on\
    }V_{x}\,\right\}=Y
    \]
First we show that $X\subseteq Y$: so let $\epsilon\in X$. There
exists an assignment $c:V\to M$ such that $c=a$ on $V\setminus\{x\}$
and $\epsilon=\beta(\phi_{1})(c)$. Define the assignment $d:V\to M$
by setting $d=b$ on $V\setminus\{x\}$ and $d(x)=c(x)$. In order to
show that $\epsilon\in Y$, it is sufficient to prove that
$\epsilon=\beta(\phi_{1})(d)$. So we need to prove that
$\beta(\phi_{1})(c)=\beta(\phi_{1})(d)$. Having assumed the
implication~(\ref{logic:eqn:FOPL:model:assignment:support:1}) is
true for $\phi_{1}$ and all assignments $a,b$, it is sufficient to
show that $c$ and $d$ coincide on $\free(\phi_{1})$. So let
$u\in\free(\phi_{1})$. We need to show that $c(u)=d(u)$. We shall
distinguish two cases: first we assume that $u=x$. Then $c(x)=d(x)$
is true by definition of $d$. Next we assume that $u\neq x$. Then
$u\in\free(\phi_{1})\setminus\{x\}=\free(\phi)$. Having assumed $a$
and $b$ coincide on $\free(\phi)$, we obtain $a(u)=b(u)$.
Furthermore, since $u\neq x$ and $c=a$ and $V\setminus\{x\}$ we have
$c(u)=a(u)$. By definition, $d=b$ on $V\setminus\{x\}$ and
consequently $d(u)=b(u)$. We conclude that $c(u)=d(u)$ as requested,
and we have proved that $X\subseteq Y$. By symmetry, we show
similarly that $Y\subseteq X$.
\end{proof}
