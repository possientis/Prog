The equivalence between syntactic and semantic entailments is a
consequence of
theorem~(\ref{logic:the:FOPL:semantics:satis:equiv:consistent}) and
therefore a consequence of Lindenbaum's
lemma~(\ref{logic:lemma:FOPL:semantics:lindenbaum}). The implication
$\Rightarrow$ of
theorem~(\ref{logic:the:FOPL:semantics:syn:equiv:sem}) below was
stated as proposition~(\ref{logic:prop:FOPL:semantics:syn:imp:sem}).
The other implication $\Leftarrow$ will be proved now and is often
referred to as the {\em strong completeness property}. The
implication $\vdash\phi\ \Leftarrow\ \vDash\phi$ is normally known
as G\"odel's completeness theorem and referred to as the {\em weak
completeness property}. However, it would be wrong for us to use the
word {\em completeness} when referring to
theorem~(\ref{logic:the:FOPL:semantics:syn:equiv:sem}). As already
explained in the discussion preceding
definition~(\ref{logic:def:FOPL:semantics:valuation}), our notion of
semantic entailment is defined solely in terms of the dual space
\pvd\ without reference to any model theory. We have many good
reasons to do this and one of these reasons is precisely to show
that results such as
theorem~(\ref{logic:the:FOPL:semantics:syn:equiv:sem}) below can be
obtained independently of any model theory, opening the way for
possible generalization to axiomatic systems which may not be
complete. The notion of completeness expresses the idea that a proof
system is rich enough to prove any formula which is true in every
model. If we remove axioms or replace them by weaker axioms, there
are fewer formulas which can be proven from our deductive system,
which has become weaker and may no longer be complete. In terms of
the deductive congruence which stems from the preorder
$\vdash(\phi\to\psi)$, a weaker deductive system leads to fewer
pairs $(\phi,\psi)$ of {\em logically equivalent} formulas, so the
congruence is stronger, assuming it is still a congruence, depending
on the exact nature of the deductive system. From the point of view
of the dual space \pvd\ which is defined by the soundness property
$\vdash\phi\ \Rightarrow\ v(\phi)=1$, a weaker deductive system
leads to a larger dual space and therefore fewer {\em valid}
formulas, in the sense of
definition~(\ref{logic:def:FOPL:semantics:entailments}). So although
we may have a weaker deductive system which is no longer complete,
since there are fewer provable formulas {\em as well as} fewer valid
formulas, it is conceivable that the equivalence between syntactic
and semantic entailment still holds. This is worth investigating and
we may do so at a later stage. This is one of the key reason for us
to decouple semantics from model theory: we wanted to present an
argument leading to
theorem~(\ref{logic:the:FOPL:semantics:syn:equiv:sem}) below, which
is specific to the Hilbert deductive congruence, and yet can
probably be made to work for other cases of congruence. For us, the
notion of completeness expresses the idea that the Hilbert deductive
proof system is rich enough, so its associated dual space \pvd\ is
small enough so that {\em every valuation has a model}. Note that
theorem~(\ref{logic:the:FOPL:semantics:syn:equiv:sem}) below is
established prior to proving $\pvd\neq\emptyset$.

\index{entailment@Entailment : Syntactic v
semantic}\index{semantic@Semantics equivalent to syntax}
\index{syntax@Syntax equivalent to semantics}
\begin{theorem}\label{logic:the:FOPL:semantics:syn:equiv:sem}
Let $V$ be a set and $\Gamma\subseteq\pv$. Then for all
$\phi\in\pv$\,:
    \begin{equation}\label{logic:eqn:FOPL:semantics:syn:equiv:sem:1}
    \Gamma\vdash\phi\ \Leftrightarrow\ \Gamma\vDash\phi
    \end{equation}
\end{theorem}
\begin{proof}
The implication $\Rightarrow$ follows from
proposition~(\ref{logic:prop:FOPL:semantics:syn:imp:sem}). So we now
prove $\Leftarrow$\,: we assume that $\Gamma\subseteq\pv$ and
$\phi\in\pv$ are such that $\Gamma\vDash\phi$. We need to show that
$\Gamma\vdash\phi$. Suppose this is not the case. Using
proposition~(\ref{logic:prop:FOPL:semantics:consistent:sequent}) we
see that $\Gamma\cup\{\phi\to\bot\}$ is consistent. Using
theorem~(\ref{logic:the:FOPL:semantics:satis:equiv:consistent}), it
is therefore satisfiable. So there exists $v\in\pvd$ which satisfies
$\Gamma\cup\{\phi\to\bot\}$. It follows that $v\vDash\Gamma$ and
$v\vDash(\phi\to\bot)$. However, from $v\vDash\Gamma$ and
$\Gamma\vDash\phi$ we obtain $v\vDash\phi$, i.e. $v(\phi)=1$. From
$v\vDash(\phi\to\bot)$ we obtain $1=v(\phi\to\bot)=v(\phi)\to
v(\bot)=v(\phi)\to 0$ and consequently $v(\phi)=0$ which is our
desired contradiction.
\end{proof}
