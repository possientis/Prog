We want to prove G\"odel's completeness theorem. Our strategy to do
so will be to show that {\em every valuation has a model}, namely
that for all $v:\pv\to 2$ element of the dual space \pvd\ of
definition~(\ref{logic:def:FOPL:semantics:valuation}), there exists
a model $(M,r)$ together with an assignment $a:V\to M$ such that
$v=\beta(\,\cdot\,)(a)$, where $\beta$ denotes the associated model
valuation function of
definition~(\ref{logic:def:FOPL:model:valuation:function}). This
strategy is clearly sufficient: G\"odel's completeness theorem is
usually known as the statement that a valid formula must be
provable. However, the word {\em valid} is usually understood as
{\em being true under every model and assignment} which does not
correspond to
definition~(\ref{logic:def:FOPL:semantics:entailments}) where we
defined a {\em valid} formula in terms of valuations and not models.
So for us G\"odel's completeness theorem is the statement that if a
formula is true under any model and assignment, then it must be
provable. Now suppose we have proved that every valuation has a
model. Then if $\phi\in\pv$ is a formula which is true under any
model and assignment, then it is also true under any valuation.
Indeed, let $v\in\pvd$. Then $v$ has a model, i.e.
$v=\beta(\,\cdot\,)(a)$ for some model $(M,r)$ and assignment
$a:V\to M$. By assumption we have $M\vDash\phi[a]$ which is
$\beta(\phi)(a)=1$ and consequently $v(\phi)=1$, i.e. $v\vDash\phi$.
Having proved that $\phi$ is true under any valuation, we conclude
from theorem~(\ref{logic:the:FOPL:semantics:syn:equiv:sem}) of
page~\pageref{logic:the:FOPL:semantics:syn:equiv:sem} that $\phi$ is
provable, and G\"odel's completeness theorem is true.

In this section, we shall study the notion of {\em regular
valuation} which will allow us to follow the spirit of Leon Henkin's
proof of G\"odel's completeness theorem, by considering maximal
consistent sets in which every existential statement has a {\em
Henkin witness}. A regular valuation $v:\pv\to 2$ is a valuation for
which:
    \begin{equation}\label{logic:eqn:FOPL:regval:regular:1}
    v(\forall x\phi_{1})=\min\left\{v(\phi_{1}[y/x])\ :\ y\in V\right\}
    \end{equation}
Since $\phi=\forall x\phi_{1}\to\phi_{1}[y/x]$ is an axiom of first
order logic, it is a provable formula and consequently $v(\phi)=1$
which is $v(\forall x\phi_{1})\leq v(\phi_{1}[y/x])$ for all
$v\in\pvd$. So the interpretation of
equation~(\ref{logic:eqn:FOPL:regval:regular:1}) can be thought of
as follows: whenever a valuation $v$ is regular, if a statement
$\forall x\phi_{1}$ is false then there must exist some $y\in V$ for
which the statement $\phi_{1}[y/x]$ is also false. We already know
from
proposition~(\ref{logic:prop:FOPL:semantics:bijection:max:cons:val})
that a valuation can be identified with a maximal consistent set. We
are also familiar with the duality $\forall
x\leftrightarrow\lnot\exists x\lnot$. Hence another way to think of
equation~(\ref{logic:eqn:FOPL:regval:regular:1}) is as follows:
whenever a maximal consistent set is regular, if it contains a
statement $\exists x\phi_{1}$ then it must also contain
$\phi_{1}[y/x]$ for some $y\in V$. In other words, every true
statement $\exists x\phi_{1}$ has a {\em Henkin witness}.

One interesting point to note is that
equation~(\ref{logic:eqn:FOPL:regval:regular:1}) is in general very
hard to write with full mathematical rigor, since the substitution
$[y/x]$ is usually not valid for $\phi_{1}$. Fortunately for us, the
notation $[y/x]:\pv\to\pv$ refers to an essential substitution of
$y$ in place of $x$, which eliminates the issue altogether.

\index{regular@Regular valuation}\index{valuation@Regular valuation}
\begin{defin}\label{logic:def:FOPL:regval:regular}
Let $V$ be a set and $v:\pv\to 2$ be a valuation on \pv. we say that
$v$ is {\em regular}, \ifand\ for all $\phi_{1}\in\pv$ and $x\in V$
we have:
    \[
    v(\forall x\phi_{1})=\min\left\{v(\phi_{1}[y/x])\ :\ y\in V\right\}
    \]
where $[y/x]:\pv\to\pv$ denotes an essential substitution of $y$ in
place of $x$.
\end{defin}

Regular valuations have the very appealing property of being induced
by their {\em naive model}. In other words, if $v:\pv\to 2$ is a
regular valuation then $v=\beta(\,\cdot\,)(i)$ where $\beta$ is the
model valuation function associated with the naive model of $v$ and
$i:V\to V$ is the identity assignment. The {\em naive model} of $v$
is simply the model $(M,r)$ where $M=V$ and $r=\{(x,y)\in V\times V\
:\ v(x\in y)=1\}$.

\index{model@Naive model of valuation} \index{naive@Naive model of
valuation}
\begin{defin}\label{logic:def:FOPL:naive:model}
Let $V$ be a set and $v:\pv\to 2$ be a valuation on \pv. We call
{\em naive model} of\, $v$ the ordered pair $(M,r)$ defined by $M=V$
and:
    \[
    r=\{(x,y)\in V\times V\ :\ v(x\in y)=1\}
    \]
\end{defin}

In fact the converse is also true: if a valuation is induced by its
naive model then it is regular, as we shall now see from the
following theorem:

\index{regular@Regular valuation and naive model}\index{naive@Naive
model and regular valuation}
\begin{theorem}\label{logic:the:FOPL:regval:regular:naive}
Let $V$ be a set and $v\in\pvd$. Then $v$ is regular, \ifand\ it is
induced by its naive model and the identity assignment $i:V\to V$,
i.e.\,:
    \[
    \forall \phi\in\pv\ ,\ v(\phi)=\beta(\phi)(i)
    \]
where $\beta$ is the model valuation function associated with the
naive model of $v$.
\end{theorem}
\begin{proof}
First we show the 'if' part: so we assume that $v:\pv\to 2$ is a
valuation such that $v(\phi)=\beta(\phi)(i)$ for all $\phi\in\pv$,
where $\beta$ is the model valuation function associated with the
naive model of $v$, and $i:V\to V$ is the identity. We need to show
that $v$ is regular. So let $\phi_{1}\in\pv$ and $x\in V$. We have:
    \begin{eqnarray*}
    v(\forall x\phi_{1})&=&\beta(\forall x\phi_{1})(i)\\
    &=&\min\left\{\beta(\phi_{1})(b)\ :\ b=i\mbox{\ on\
    }V_{x}\right\}\\
    \mbox{A: to be proved}\ \rightarrow
    &=&\min\left\{\beta(\phi_{1})([y/x])\ :\ y\in V\right\}\\
    &=&\min\left\{\beta(\phi_{1})(\,i\circ[y/x]\,)\ :\ y\in V\right\}\\
    \mbox{B: to be proved}\ \rightarrow
    &=&\min\left\{\beta(\,[y/x](\phi_{1})\,)(i)\ :\ y\in V\right\}\\
    &=&\min\left\{\beta(\phi_{1}[y/x])(i)\ :\ y\in V\right\}\\
    &=&\min\left\{v(\phi_{1}[y/x])\ :\ y\in V\right\}\\
    \end{eqnarray*}
So it remains to justify points A and B. For point A, we need to
show:
    \[
    X=\{b:V\to V\ :\ b=i\mbox{\ on\ }V_{x}\}=\{[y/x]:V\to V\ ,\ y\in
    V\}=Y
    \]
First we show that $X\subseteq Y$: so suppose $b=i$ on
$V\setminus\{x\}$. Define $y=b(x)$. In order to show that $b\in Y$
it is sufficient to prove that $b=[y/x]$. So suppose $u\in V$. If
$u=x$, then $b(x)=[y/x](x)$ follows from our definition of $y$. If
$u\neq x$ then we have $b(u)=i(u)=u=[y/x](u)$. So we now prove that
$Y\subseteq X$: so let $b=[y/x]$ for some $y\in V$. Then it is clear
that $b=i$ on $V\setminus\{x\}$. It remains to justify point B,
which is an immediate consequence of
theorem~(\ref{logic:the:FOPL:model:essential:substitution}) of
page~\pageref{logic:the:FOPL:model:essential:substitution} provided
the map $[y/x]:\pv\to\pv$ is understood to be an arbitrary essential
substitution associated with the map $[y/x]:V\to V$. Note that such
an essential substitution always exists by virtue of
theorem~(\ref{logic:the:FOPL:esssubst:existence}) of
page~\pageref{logic:the:FOPL:esssubst:existence}. We now show the
'only if' part: so we assume that $v:\pv\to 2$ is a valuation which
is regular. We need to show that $\beta(\phi)(i)=v(\phi)$ for all
$\phi\in\pv$, where $\beta$ is the model valuation function of the
naive model of $v$ and $i:V\to V$ is the identity. We shall prove
the seemingly stronger result that
$\beta(\phi)(\sigma)=v(\sigma(\phi))$ for all essential
substitutions $\sigma:\pv\to\pv$. We shall do so by structural
induction, using theorem~(\ref{logic:the:proof:induction}) of
page~\pageref{logic:the:proof:induction}. Note that from
proposition~(\ref{logic:prop:FOPL:esssubstprop:injective}) the
identity $i:\pv\to\pv$ being injective, is an essential substitution
associated with the identity $i:V\to V$. So what we set out to prove
is indeed sufficient. First we assume that $\phi=(x\in y)$ for some
$x,y\in V$. Let $\sigma:\pv\to\pv$ be an arbitrary essential
substitution:
    \begin{eqnarray*}
    \beta(\phi)(\sigma)&=&\beta(x\in y)(\sigma)\\
        &=&1_{r}(\sigma(x),\sigma(y))\\
        \mbox{def.~(\ref{logic:def:FOPL:naive:model})}\ \rightarrow
        &=&v(\sigma(x)\in \sigma(y))\\
        \mbox{prop.~(\ref{logic:prop:FOPL:esssubstprop:charac})}\ \rightarrow
        &=&v(\sigma(\phi))\\
    \end{eqnarray*}
Next we assume that $\phi=\bot$. We have
$\beta(\bot)(\sigma)=0=v(\bot)=v(\sigma(\bot))$, where
$\bot=\sigma(\bot)$ follows from
proposition~(\ref{logic:prop:FOPL:esssubstprop:charac}). So we now
assume that $\phi=\phi_{1}\to\phi_{2}$ where
$\phi_{1},\phi_{2}\in\pv$ satisfy our equality for any
$\sigma:\pv\to\pv$ essential:
    \begin{eqnarray*}
    \beta(\phi)(\sigma)&=&\beta(\phi_{1}\to\phi_{2})(\sigma)\\
    &=&\beta(\phi_{1})(\sigma)\to\beta(\phi_{2})(\sigma)\\
    &=&v(\sigma(\phi_{1}))\to v(\sigma(\phi_{2}))\\
    &=&v(\,\sigma(\phi_{1})\to\sigma(\phi_{2})\,)\\
    \mbox{A: to be proved}\ \rightarrow&=&v(\sigma(\phi))\\
    \end{eqnarray*}
It remains to show that
$v(\,\sigma(\phi_{1})\to\sigma(\phi_{2})\,)=v(\sigma(\phi))$. From
proposition~(\ref{logic:prop:FOPL:esssubstprop:charac}) we have
$\sigma(\phi)\sim\sigma(\phi_{1})\to\sigma(\phi_{2})$ where $\sim$
is the substitution congruence. So our desired equality follows from
proposition~(\ref{logic:prop:FOPL:semantics:stronger:congruence}).
We now assume that $\phi=\forall x\phi_{1}$ where $\phi_{1}\in\pv$
satisfies our equality for all $\sigma:\pv\to\pv$ essential. We need
to show the same is true of $\phi$. So let $\sigma:\pv\to\pv$ be an
essential substitution. Using
proposition~(\ref{logic:prop:FOPL:esssubstprop:tau}), let
$\tau:\pv\to\pv$ be an essential substitution such that
$\tau=\sigma$ on $V\setminus\{x\}$, and
$\tau(x)\not\in\free(\sigma(\phi))$. Note that we have the
substitution equivalence
$\sigma(\phi)\sim\forall\tau(x)\tau(\phi_{1})$ and consequently:
    \begin{eqnarray*}
    \beta(\phi)(\sigma)&=&\beta(\forall x\phi_{1})(\sigma)\\
    &=&\min\left\{\beta(\phi_{1})(b)\ :\ b=\sigma\mbox{\ on\
    }V_{x}\right\}\\
    \tau=\sigma\mbox{\ on\ }V_{x}\ \rightarrow
    &=&\min\left\{\beta(\phi_{1})(b)\ :\ b=\tau\mbox{\ on\
    }V_{x}\right\}\\
    \mbox{A: to be proved}\ \rightarrow
    &=&\min\left\{\beta(\phi_{1})(\,[y/\tau(x)]\circ\tau\,)\ :\ y\in V\right\}\\
    \mbox{B: to be proved}\ \rightarrow
    &=&\min\left\{v(\,\,[y/\tau(x)]\circ\tau(\phi_{1})\,\,)\ :\ y\in V\right\}\\
    &=&\min\left\{v(\,\tau(\phi_{1})[y/\tau(x)]\,)\ :\ y\in V\right\}\\
    \mbox{$v$ is regular}\ \rightarrow&=&v(\,\forall\tau(x)\tau(\phi_{1})\,)\\
    \mbox{prop.~(\ref{logic:prop:FOPL:semantics:stronger:congruence})}\ \rightarrow
    &=&v(\sigma(\phi))
    \end{eqnarray*}
So it remains to prove points A and B. First we start with point
$B$: we need to show the equality
$\beta(\phi_{1})(\,[y/\tau(x)]\circ\tau\,)=v(\,\,[y/\tau(x)]\circ\tau(\phi_{1})\,\,)$
which is in fact simply our induction hypothesis, provided
$[y/\tau(x)]\circ\tau:\pv\to\pv$ is an essential substitution.
However, we already know that $\tau:\pv\to\pv$ is essential, and
from theorem~(\ref{logic:the:FOPL:esssubst:existence}) of
page~\pageref{logic:the:FOPL:esssubst:existence} we can choose
$[y/\tau(x)]:\pv\to\pv$ to be an essential map associated with
$[y/\tau(x)]:V\to V$. From
proposition~(\ref{logic:prop:FOPL:esssubstprop:composition}), the
composition $[y/\tau(x)]\circ\tau:\pv\to\pv$ is indeed an essential
substitution, associated with $[y/\tau(x)]\circ\tau: V\to V$. So it
remains to prove A, which is:
    \[
    X=\left\{\beta(\phi_{1})(b)\ :\ b=\tau\mbox{\ on\
    }V_{x}\right\}=\left\{\beta(\phi_{1})(\,[y/\tau(x)]\circ\tau\,)\ :\ y\in
    V\right\}=Y
    \]
First we show that $X\subseteq Y$: so let $\epsilon\in X$. There
exists an assignment $b:V\to V$ such that $b=\tau$ on
$V\setminus\{x\}$ and $\epsilon=\beta(\phi_{1})(b)$. We need to show
that $\epsilon\in Y$. Let $y=b(x)$. In order to show that
$\epsilon\in Y$, it is sufficient to show that $\epsilon
=\beta(\phi_{1})(\,[y/\tau(x)]\circ\tau\,)$. So we need to show
$\beta(\phi_{1})(b)=\beta(\phi_{1})(\,[y/\tau(x)]\circ\tau\,)$.
Using proposition~(\ref{logic:prop:FOPL:model:assignment:support}),
it is sufficient to prove that $b$ and $[y/\tau(x)]\circ\tau$
coincide on $\free(\phi_{1})$. So let $u\in\free(\phi_{1})$. We need
to show that $b(u)=[y/\tau(x)]\circ\tau(u)$. We shall distinguish
two cases: first we assume that $u=x$. Then we need to show that
$b(x)=y$ which is true by definition of $y$. Next we assume that
$u\neq x$. Then $u\in\free(\phi_{1})\setminus\{x\}=\free(\phi)$.
From proposition~(\ref{logic:prop:FOPL:esssubstprop:free:commute})
we have $\free(\sigma(\phi))=\sigma(\free(\phi))$. It follows that
$\sigma(u)\in\free(\sigma(\phi))$. Furthermore, since $\tau=\sigma$
on $V\setminus\{x\}$ we have $\tau(u)=\sigma(u)$. So we see that
$\tau(u)\in\free(\sigma(\phi))$. Having assumed that
$\tau(x)\not\in\free(\sigma(\phi))$ it follows that
$\tau(u)\neq\tau(x)$. Hence, we see that
$[y/\tau(x)]\circ\tau(u)=\tau(u)$, and the equality
$b(u)=[y/\tau(x)]\circ\tau(u)$ follows from the fact that $b=\tau$
on $V\setminus\{x\}$. So we now prove that $Y\subseteq X$: let
$\epsilon\in Y$. There exists $y\in V$ such that
$\epsilon=\beta(\phi_{1})(\,[y/\tau(x)]\circ\tau\,)$. We need to
show that $\epsilon\in X$. Define the assignment $b:V\to V$ by
setting $b=\tau$ on $V\setminus\{x\}$ and $b(x)=y$. In order to show
that $\epsilon\in X$, it is sufficient to prove that $\epsilon
=\beta(\phi_{1})(b)$. So we need to show that
$\beta(\phi_{1})(b)=\beta(\phi_{1})(\,[y/\tau(x)]\circ\tau\,)$.
Using proposition~(\ref{logic:prop:FOPL:model:assignment:support}),
it is sufficient to prove that $b$ and $[y/\tau(x)]\circ\tau$
coincide on $\free(\phi_{1})$. So let $u\in\free(\phi_{1})$. We need
to show that $b(u)=[y/\tau(x)]\circ\tau(u)$. We shall distinguish
two cases: first we assume that $u=x$. Then we need to show that
$b(x)=y$ which is true by definition of $b$. Next we assume that
$u\neq x$. Then using an identical argument as previously we obtain
$\tau(u)\neq\tau(x)$ and we therefore need to show that
$b(u)=\tau(u)$ which is true by definition of $b$.
\end{proof}
