In this section we prove Lindenbaum's lemma from which follow many
interesting and deeper results such as
theorem~(\ref{logic:the:FOPL:semantics:syn:equiv:sem}) of
page~\pageref{logic:the:FOPL:semantics:syn:equiv:sem} establishing
the equivalence between syntactic and semantic entailments. Our
proof relies on Zorn's lemma and therefore on the axiom of choice.
It is usually possible to provide a proof of Lindenbaum's lemma
which does not rely on the axiom of choice. However our set $V$
being arbitrary, we do not have this option when dealing with the
general case. We shall provide a statement of Zorn's lemma below,
but are not able to spell out a proof, as this would take us too
long. Zorn's lemma is a monster of usefulness and we certainly hope
to design a proof of it, once we are able to deal with proper
axiomatic set theory. Recall that a {\em partially ordered set} is
an ordered pair $(X\,\leq)$ where $\leq$ is a {\em partial order} on
$X$, namely a relation on $X$ which is reflexive, anti-symmetric and
transitive on $X$. We have already studied a case of partial order
when defining the {\em sub-formula relation} $\preceq$ of
definition~(\ref{logic:def:subformula}), where $X$ is a free
universal algebra. If $(X,\leq)$ is a partially ordered set and
$Y\subseteq X$, there is an obvious relation on $Y$ namely:
    \[
    R_{Y}=\{(x,y)\in Y\times Y\ :\ x\leq y\}
    \]
It is easy to check that $R_{Y}$ is a partial order on $Y$ which we
also denote $\leq$. So $(Y,\leq)$ becomes a partially ordered set.
We say that $Y$ is a {\em totally ordered subset} of $X$ \ifand\ the
partially ordered set $(Y,\leq)$ is {\em totally ordered}. A
partially ordered set $(X,\leq)$ is said to be {\em totally
ordered}, \ifand\ for all $x,y\in X$ we have $x\leq y$ or $y\leq x$.
We say that $Y\subseteq X$ {\em has an upper-bound in } $X$, \ifand\
there exists $a\in X$ such that $y\leq a$ for all $y\in Y$. Zorn's
lemma says that if $(X,\leq)$ is a partially ordered set for which
every totally ordered subset $Y\subseteq X$ has an upper-bound in
$X$, then $X$ has a maximal element. Note that it is not necessary
to assume $X\neq\emptyset$ in the statement of Zorn's lemma. Indeed
$Y=\emptyset$ is a subset of $X$ which is vacuously totally ordered.
Any element $a\in X$ is vacuously an upper-bound of $Y=\emptyset$.
So if we assume that every totally ordered subset of $X$ has an
upper-bound in $X$, then in particular $Y=\emptyset$ has an
upper-bound in $X$ and consequently $X$ cannot be empty. We are now
ready to state:

\index{zorn@Zorn's lemma}
\begin{lemma}[Zorn's Lemma]\label{logic:lemma:FOPL:semantics:zorn}
Let $(X,\leq)$ be a partially ordered set for which every totally
ordered subset has an upper-bound in $X$. Then $X$ has a maximal
element, i.e. there exists $x_{0}\in X$ satisfying the property:
    \[
    \forall x\in X\ ,\ (x_{0}\leq x\ \Rightarrow\ x_{0}=x)
    \]
\end{lemma}
\begin{proof}
See for example Walter Rudin~\cite{Rudin}.
\end{proof}

Lindenbaum's lemma states that every consistent subset of \pv\ can
be extended to a maximal consistent subset. Its proof relies on
Zorn's lemma by considering the set $X$ of all consistent extensions
$\Gamma$ of a given consistent set $\Gamma_{0}$, partially ordered
by inclusion. In order to successfully apply Zorn's lemma, we need
to show that every totally ordered subset $Y$ of $X$ has an
upper-bound in $X$. An upper-bound candidate will naturally be the
union $\Gamma=\cup Y$, which we shall need to check is indeed a
consistent extension of $\Gamma_{0}$. The following lemma will allow
us to do that. Note that we are assuming $Y$ to be non-empty in the
following statement. If $Y=\emptyset$, then $\Gamma=\cup Y$ is also
the empty set which is consistent but we cannot yet prove that. So
removing the assumption $Y\neq\emptyset$ in the following lemma
would yield a true conclusion, but with the wrong proof.

\begin{lemma}\label{logic:lemma:FOPL:semantics:chain:consistent}
Let $V$ be a set and $Y$ be a set of consistent subsets of\, \pv\
which is non-empty and totally ordered by inclusion. Then
$\Gamma=\cup Y$ is consistent.
\end{lemma}
\begin{proof}
We have $\Gamma=\cup Y=\{\,\phi\in\pv\ :\ \exists\Delta\in Y\ ,\
\phi\in\Delta\,\}$. Suppose $\Gamma$ is not consistent. Then we have
$\Gamma\vdash\bot$. So there exists $\Gamma_{0}$ finite such that
$\Gamma_{0}\subseteq\Gamma$ and $\Gamma_{0}\vdash\bot$. Note that
$\Gamma_{0}$ cannot be the empty set, as otherwise from $\vdash\bot$
we would have $\Delta\vdash\bot$ for every $\Delta\subseteq\pv$, and
no subset of \pv\ would be consistent, contradicting the fact that
$Y$ is non-empty. Let $n=|\Gamma_{0}|$ be the cardinal of
$\Gamma_{0}$. Then there exists a bijection $\psi:n\to\Gamma_{0}$.
For all $k\in n$ we have $\psi(k)\in\Gamma_{0}$ and in particular
$\psi(k)\in\Gamma$. Hence, there exists $\Delta(k)\in Y$ such that
$\psi(k)\in\Delta(k)$. Since $\Gamma_{0}\neq\emptyset$ we have
$n\geq 1$ and the set $\{\Delta(k)\ : k\in n\}$ is therefore a
non-empty subset of $Y$. Having assumed $Y$ is totally ordered by
inclusion, this set has a maximum. So there exists $k^{*}\in n$ such
that $\Delta(k)\subseteq\Delta(k^{*})$ for all $k\in n$. It follows
that $\psi(k)\in\Delta(k^{*})$ for all $k\in n$. The map
$\psi:n\to\Gamma_{0}$ being a surjection, we see that
$\Gamma_{0}\subseteq\Delta(k^{*})$. So from $\Gamma_{0}\vdash\bot$
we have $\Delta(k^{*})\vdash\bot$. It follows that $\Delta(k^{*})$
is not consistent, contradicting the fact that $\Delta(k^{*})\in Y$.
\end{proof}

As already mentioned on prior occasions, we still do not know for a
fact that consistent subsets do exist, as we haven't proved the
sequent $\vdash\bot$ is false. If there was no consistent set at
all, the statement of Lindenbaum's lemma which follows would be
vacuously true. Luckily, consistent subsets will be seen to exist.

\index{lindenbaum@Lindenbaum's lemma}
\begin{lemma}[Lindenbaum]\label{logic:lemma:FOPL:semantics:lindenbaum}
Let $V$ be a set and $\Gamma_{0}\subseteq\pv$ be a consistent
subset. Then there exists $\Gamma\subseteq\pv$ maximal consistent
subset such that $\Gamma_{0}\subseteq\Gamma$.
\end{lemma}
\begin{proof}
Consider the partially ordered set $(X,\subseteq)$ defined by the
following:
    \[
    X=\{\,\Gamma\subseteq\pv\ :\ \Gamma_{0}\subseteq\Gamma\ ,\
    \mbox{$\Gamma$ consistent}\,\}
    \]
where $\subseteq$ denotes the standard inclusion on ${\cal P}(\pv)$.
We wish to apply Zorn's lemma to $(X,\subseteq)$. Let us accept the
conclusion of Zorn's lemma for now, i.e. that $X$ has a maximal
element $\Gamma$. Then in particular we have $\Gamma\in X$. So
$\Gamma$ is consistent and $\Gamma_{0}\subseteq\Gamma$. It remains
to prove that $\Gamma$ is in fact maximal consistent. So suppose
$\Delta\subseteq\pv$ is consistent with $\Gamma\subseteq\Delta$. We
need to show that $\Gamma=\Delta$. So it is sufficient to show that
$\Delta\in X$ as the equality $\Gamma=\Delta$ will then follow from
the maximality of $\Gamma$ in $X$ and $\Gamma\subseteq\Delta$.
However, we have $\Gamma_{0}\subseteq\Gamma$ and
$\Gamma\subseteq\Delta$. It follows that
$\Gamma_{0}\subseteq\Delta$. Having assumed $\Delta$ is consistent,
we see that $\Delta$ is indeed an element of $X$, and the
proposition is proved. So it remains to show that $X$ has a maximal
element. Using Zorn's lemma~(\ref{logic:lemma:FOPL:semantics:zorn}),
we need to show that if $Y\subseteq X$ is a totally ordered subset
of $X$, then $Y$ has an upper-bound in $X$. So let $Y\subseteq X$ be
totally ordered. We shall distinguish two cases: first we assume
that $Y=\emptyset$. Then any element of $X$ is vacuously an
upper-bound of $Y$. So it is sufficient to show that
$X\neq\emptyset$ which is clearly the case since $\Gamma_{0}$ is
consistent and consequently $\Gamma_{0}\in X$. So we assume
$Y\neq\emptyset$. Let $\Gamma=\cup Y$, i.e.\,:
    \[
    \Gamma=\{\,\phi\in\pv\ :\ \exists\Delta\in Y\ ,\ \phi\in\Delta\,\}
    \]
It is sufficient to show that $\Gamma$ is an upper-bound of $Y$ in
$X$, namely that $\Gamma\in X$ and $\Delta\subseteq\Gamma$ for all
$\Delta\in Y$. First we show that $\Delta\subseteq\Gamma$ for all
$\Delta\in Y$. So let $\Delta\in Y$ and consider $\phi\in\Delta$.
Then it is clear that $\phi\in\Gamma$ and the inclusion
$\Delta\subseteq\Gamma$ follows. So we now prove that $\Gamma\in X$.
We need to show that $\Gamma_{0}\subseteq\Gamma$ and $\Gamma$ is
consistent. First we show that $\Gamma_{0}\subseteq\Gamma$. So let
$\phi\in\Gamma_{0}$. We need to show that $\phi\in\Gamma$. However,
since $Y\neq\emptyset$, there exists $\Delta\in Y\subseteq X$. So in
particular, we have $\Delta\in X$ and consequently
$\Gamma_{0}\subseteq\Delta$. It follows that $\phi\in\Delta$ and
from $\Delta\in Y$ we see that $\phi\in\Gamma$ as requested. It
remains to show that $\Gamma$ is consistent. However, we have
$\Gamma=\cup Y$ where $Y$ is a set of consistent subsets of \pv\
which is non-empty and totally ordered by inclusion. The fact that
$\Gamma$ is consistent follows immediately from
lemma~(\ref{logic:lemma:FOPL:semantics:chain:consistent}).
\end{proof}
