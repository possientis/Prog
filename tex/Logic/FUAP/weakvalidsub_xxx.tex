In definition~(\ref{logic:def:FOPL:valid:substitution}) we defined
what it meant for a substitution $\sigma:V\to W$ to be valid for a
formula $\phi\in\pv$. We needed to introduce the notion that
$\sigma$ would not randomly distort the {\em logical structure} of
$\phi$, a property which is commonly known as {\em avoiding
capture}. The purpose was obvious: we wanted to {\em carry over}
properties of $\phi\in\pv$ into corresponding properties of
$\sigma(\phi)\in{\bf P}(W)$. For example,
theorem~(\ref{logic:the:FOPL:mintransfsubcong:valid}) of
page~\pageref{logic:the:FOPL:mintransfsubcong:valid} tells us that a
substitution equivalence $\phi\sim\psi$ is preserved by $\sigma$
provided it is valid for both $\phi$ and $\psi$. Another example
will be seen as the {\em substitution lemma} in the context of model
theory, where
proposition~(\ref{logic:prop:FOPL:model:valid:substitution}) will
show that if $\sigma$ is valid for $\phi$, then the truth of
$\sigma(\phi)$ in a model $M$ under an assignment $a:W\to M$ is
equivalent to the truth of $\phi$ under the assignment
$a\circ\sigma$. The notion of $\sigma$ being valid for $\phi$ is
crucially important.

One of the most interesting properties of formulas which we may wish
to {\em carry over} is that of provability. If a sequent
$\Gamma\vdash\phi$ is true, we all want to know under what
conditions the corresponding sequent
$\sigma(\Gamma)\vdash\sigma(\phi)$ is also true. It is very tempting
to conjecture that provided $\sigma$ is valid for $\phi$ and is also
valid for every hypothesis $\psi\in\Gamma$, then the result will
hold. Of course, an obvious issue is the fact that a sequent
$\Gamma\vdash\phi$ does not tell us anything about the axioms being
used in a proof underlying the sequent. So we cannot hope to simply
{\em carry over} every step of the proof with $\sigma$, as our
assumption does not guarantee $\sigma$ is valid for every axiom
involved in the proof. It only guarantees $\sigma$ is valid for the
conclusion, and for every hypothesis. In fairness, this issue only
arises because we stubbornly insist on working with actual maps
$\sigma:V\to W$, rather than essential substitutions
$\sigma:\pv\to{\bf P}(W)$ of
definition~(\ref{logic:def:FOPL:esssubstprop:essential}). We know
essential substitutions are the right concept. We somehow suspect
our scheme to {\em carry over} proofs with the substitution
$\sigma:\pvs\to{\bf\Pi}(W)$ of
definition~(\ref{logic:def:FUAP:substitution:substitution}) is most
likely a waste of time. We should {\em carry over} proofs using
essential substitutions only. There is however one thing to
remember: from theorem~(\ref{logic:the:FOPL:esssubst:existence}) of
page~\pageref{logic:the:FOPL:esssubst:existence} an essential
substitution $\sigma:\pv\to{\bf P}(W)$ can only exist if $W$ is an
infinite set, or $V$ is a smaller set than $W$. So if we restrict
our analysis to essential substitutions, we cannot hope to {\em
carry over} the sequent $\Gamma\vdash\phi$ into
$\sigma(\Gamma)\vdash\sigma(\phi)$ in the case when $W$ is a finite
set of smaller cardinality than $V$.

We are certainly interested in saying something about
$\sigma(\Gamma)\vdash\sigma(\phi)$ in the case when $W$ is a smaller
finite set. For example, suppose we have an embedding $j:W\to V$
between two finite sets, and $\Delta$ is a consistent subset of
${\bf P}(W)$. We certainly hope that $\Gamma=j(\Delta)$ is a
consistent subset of \pv. If $\Gamma$ is not consistent, as we shall
see from
definition~(\ref{logic:def:FOPL:semantics:consistent:subset}) the
sequent $\Gamma\vdash\bot$ is true. Assuming $W\neq\emptyset$ and
$j$ has a left inverse $\sigma:V\to W$, here is a case when we want
to {\em carry over} the sequent $\Gamma\vdash\bot$ into a smaller
finite set $W$: we want to argue that $\sigma(\Gamma)\vdash\bot$ is
true so as to obtain $\Delta\vdash\bot$, contradicting the
consistency of $\Delta$. Most proofs of G\"odel completeness theorem
involve successive embeddings where constants are continually added
to the language, and it is usually taken for granted that
consistency is preserved as the language is extended. This is not
completely obvious for us. We do not have constants, but only
variables. As the set of variables is increased, the range of
possible proofs becomes larger and it is conceivable that a
contradiction may appear. In general, it is not clear that a sequent
$j(\Delta)\vdash j(\phi)$ can be transported back into the smaller
finite set, as the variables involved in the axioms may have no
obvious counterpart in $W$.

In any case, although we strongly suspect essential substitutions
will play an important role in allowing us to {\em carry over}
sequents, we should not attempt to run before we can walk: when
attempting to establish the sequent
$\sigma(\Gamma)\vdash\sigma(\phi)$ we should start with the obvious,
and the obvious consists in following every step of the proof
underlying $\Gamma\vdash\phi$, and replacing all variables in line
with the substitution $\sigma$. This is the purpose of
definition~(\ref{logic:def:FUAP:substitution:substitution}),
creating $\sigma(\pi)\in{\bf\Pi}(W)$ from $\pi\in\pvs$, in the hope
that $\sigma(\pi)$ will become a proof of $\sigma(\phi)$, from the
set of hypothesis $\sigma(\Gamma)$. Of course, this is not going to
work in all cases. The proof $\sigma(\pi)$ is pretty worthless,
unless we control its valuation $\val(\sigma(\pi))$, that is:
    \begin{equation}\label{logic:eqn:FUAP:validsub:intro:1}
    \val\circ\sigma(\pi)=\sigma\circ\val(\pi)
    \end{equation}
It is clear equation~(\ref{logic:eqn:FUAP:validsub:intro:1}) will
fail unless $\sigma:V\to W$ has the right properties in relation to
$\pi$. For example, if $\pi=\axi\phi$ where $\phi\in\av$ is an axiom
of first order logic, then $\sigma(\pi)=\axi\sigma(\phi)$ and if
$\sigma(\phi)$ fails to be an axiom, then from
definition~(\ref{logic:def:FOPL:proof:valuation}) we obtain
$\val\circ\sigma(\pi)=\bot\to\bot$. So
equation~(\ref{logic:eqn:FUAP:validsub:intro:1}) will fail unless
$\sigma(\phi)$ is itself an axiom. We shall see later in
lemma~(\ref{logic:lemma:FUAP:substitution:axiom}) that
$\sigma(\phi)$ is indeed an axiom, provided $\sigma$ is valid for
$\phi$. It is not difficult to design a counterexample otherwise. So
here is a case when the condition {\em $\sigma$ is valid for $\phi$}
is a key condition to ensure that
equation~(\ref{logic:eqn:FUAP:validsub:intro:1}) is met by
$\pi=\axi\phi$, at least in the case when $\pi$ is totaly clean,
that is when $\phi$ is an axiom. In our discussion preceding
definition~(\ref{logic:def:FUAP:clean:clean:proof}) we explained the
importance of $\pi$ being totally clean.

In this section, we want to define what it is for $\sigma$ to be
{\em valid for $\pi$}. Our purpose to the design the right condition
so as to ensure equation~(\ref{logic:eqn:FUAP:validsub:intro:1})
will hold, whenever $\pi$ is totally clean and $\sigma$ is {\em
valid for $\pi$}. The result will be proved in
proposition~(\ref{logic:prop:FUAP:validsub:valuation:commute}).



\begin{defin}\label{logic:def:FUAP:validsub:validsub}
Let $V$ and $W$ be sets and $\sigma:V\to W$ be a map. Then $\sigma$
is {\em weakly valid for} $\pi\in\pvs$, \ifand\ for all
$\pi_{1}\in\pvs$, $\phi\in\pv$, $x\in V$:
    \begin{eqnarray*}
    (i)&&\phi\in\hyp(\pi)\ \Rightarrow\ \mbox{$\sigma$ valid for $\phi$}\\
    (ii)&&\phi\in\ax(\pi)\ \Rightarrow\ \mbox{$\sigma$ valid for $\phi$}\\
    (iii)&&\gen x\pi_{1}\preceq\pi\ \Rightarrow\ [\
    u\in\spec(\pi_{1})\setminus\{x\}\ \Rightarrow\
    \sigma(u)\neq\sigma(x)\ ]
    \end{eqnarray*}
\end{defin}

\begin{prop}\label{logic:prop:FUAP:validsub:subformula}
Let $V, W$ be sets and $\sigma:V\to W$ be a map. Then $\sigma$ is
weakly valid for $\pi\in\pvs$ \ifand\ it is weakly valid for any
sub-proof $\rho\preceq\pi$.
\end{prop}
\begin{proof}
If $\sigma$ is weakly valid for any sub-proof of $\pi$, then in
particular it is weakly valid for $\pi$. So we assume that $\sigma$
is weakly valid for $\pi$ and consider a sub-proof $\rho\preceq\pi$.
We need to show that $\sigma$ is also weakly valid for $\rho$. So
suppose $\phi\in\hyp(\rho)\cup\ax(\rho)$. We need to show that
$\sigma$ is valid for $\phi$. However, from
proposition~(\ref{logic:prop:FUAP:hypothesis:subformula})
and~(\ref{logic:prop:FUAP:axiomset:subformula}) we have the
inclusions $\hyp(\rho)\subseteq\hyp(\pi)$  and
$\ax(\rho)\subseteq\ax(\pi)$. It follows that
$\phi\in\hyp(\pi)\cup\ax(\pi)$. Since $\sigma$ is weakly valid for
$\pi$ we see that $\sigma$ is valid for $\phi$. We now assume that
$\gen x\pi_{1}\preceq\rho$. We need to show that the implication
$u\in\spec(\pi_{1})\setminus\{x\}\ \Rightarrow\
    \sigma(u)\neq\sigma(x)$ is true. However, since $\rho\preceq\pi$ by transitivity we obtain
$\gen x\pi_{1}\preceq\pi$ and since $\sigma$ is weakly valid for
$\pi$, the implication is true.
\end{proof}

\begin{prop}\label{logic:prop:FUAP:validsub:recursion:formula}
Let $V, W$ be sets and $\sigma:V\to W$ be a map. Let $\pi$ of the
form $\pi=\phi\in\pv$. Then $\sigma$ is weakly valid for $\pi$
\ifand\ it is valid for~$\phi$.
\end{prop}
\begin{proof}
Suppose $\sigma$ is weakly valid for $\pi=\phi$. Since
$\phi\in\hyp(\pi)$ we see that $\sigma$ is valid for~$\phi$.
Conversely, suppose $\sigma$ is valid for $\phi$. We need to show
that $\sigma$ is weakly valid for $\pi$. So suppose
$\psi\in\hyp(\pi)$. We need to show that $\sigma$ is valid for
$\psi$. However, $\hyp(\pi)=\{\phi\}$ and consequently $\psi=\phi$.
So $\sigma$ is indeed valid for $\psi$. Furthermore since
$\ax(\pi)=\emptyset$, property $(ii)$ of
definition~(\ref{logic:def:FUAP:validsub:validsub}) is vacuously
true. So we now check property $(iii)$: note that since
$\pi=\phi\in\pv$, from definition~(\ref{logic:def:subformula}) we
have $\subf(\pi)=\{\pi\}$. In other words, the only sub-proof of
$\pi$ is $\pi$ itself. Using
theorem~(\ref{logic:the:unique:representation}) of
page~\pageref{logic:the:unique:representation} it follows that no
sub-proof of $\pi$ can be of the form $\gen x\pi_{1}$. So $(iii)$ of
definition~(\ref{logic:def:FUAP:validsub:validsub}) is also
vacuously satisfied.
\end{proof}

\begin{prop}\label{logic:prop:FUAP:validsub:recursion:axiom}
Let $V, W$ be sets and $\sigma:V\to W$ be a map. Let $\pi=\axi\phi$,
where $\phi\in\pv$. Then $\sigma$ is weakly valid for $\pi$ \ifand\
it is valid for $\phi$.
\end{prop}
\begin{proof}
Suppose $\sigma$ is weakly valid for $\pi=\axi\phi$. Since
$\phi\in\ax(\pi)$ we see that $\sigma$ is valid for~$\phi$.
Conversely, suppose $\sigma$ is valid for $\phi$. We need to show
that $\sigma$ is weakly valid for~$\pi$. since
$\hyp(\pi)=\emptyset$, property $(i)$ of
definition~(\ref{logic:def:FUAP:validsub:validsub}) is vacuously
true. So we now check property $(ii)$: suppose $\psi\in\ax(\pi)$. We
need to show that $\sigma$ is valid for $\psi$. However,
$\ax(\pi)=\{\phi\}$ and consequently $\psi=\phi$. So $\sigma$ is
indeed valid for $\psi$. We now check property $(iii)$: from
definition~(\ref{logic:def:subformula}) we have
$\subf(\pi)=\subf(\axi\phi(0))=\{\axi\phi(0)\}=\{\pi\}$. In other
words, the only sub-proof of $\pi$ is $\pi$ itself. Using
theorem~(\ref{logic:the:unique:representation}) of
page~\pageref{logic:the:unique:representation} it follows that no
sub-proof of $\pi$ can be of the form $\gen x\pi_{1}$. So $(iii)$ of
definition~(\ref{logic:def:FUAP:validsub:validsub}) is vacuously
satisfied.
\end{proof}

\begin{prop}\label{logic:prop:FUAP:validsub:recursion:pon}
Let $V, W$ be sets and $\sigma:V\to W$ be a map. Let
$\pi=\pi_{1}\pon\pi_{2}$. Then $\sigma$ is weakly valid for $\pi$
\ifand\ it is weakly valid for $\pi_{1}$ and $\pi_{2}$.
\end{prop}
\begin{proof}
First we show the 'only if' part: so we assume that $\sigma$ is
weakly valid for $\pi=\pi_{1}\pon\pi_{2}$. From
proposition~(\ref{logic:prop:FUAP:validsub:subformula}) it is
therefore weakly valid for every sub-proof of $\pi$. Since
$\pi_{1}\preceq\pi$ and $\pi_{2}\preceq\pi$ we conclude that
$\sigma$ is weakly valid for both $\pi_{1}$ and $\pi_{2}$. We now
show the 'if' part: so we assume that $\sigma$ is weakly valid for
$\pi_{1}$ and $\pi_{2}$. We need to show that $\sigma$ is weakly
valid for $\pi=\pi_{1}\pon\pi_{2}$. So let $\phi\in\hyp(\pi)$. We
need to show that $\sigma$ is valid for $\phi$. However
$\hyp(\pi)=\hyp(\pi_{1})\cup\hyp(\pi_{2})$. So $\phi$ is an element
of $\hyp(\pi_{1})$ or $\hyp(\pi_{2})$ and in both cases we see that
$\sigma$ is valid for $\phi$. So we now assume that
$\phi\in\ax(\pi)$. We need to show that $\sigma$ is valid for
$\phi$. Once again, since $\ax(\pi)=\ax(\pi_{1})\cup\ax(\pi_{2})$
the formula $\phi$ must be an element of $\ax(\pi_{1})$ or
$\ax(\pi_{2})$. Either way, we see that $\sigma$ is valid for
$\phi$. In order to show that $\sigma$ is weakly valid for $\pi$, it
remains to show that $(iii)$ of
definition~(\ref{logic:def:FUAP:validsub:validsub}) is satisfied. So
suppose $\gen x\rho\preceq\pi$. We need to show the implication
$u\in\spec(\rho)\setminus\{x\}\ \Rightarrow\
\sigma(u)\neq\sigma(x)$. However, since $\pi=\pi_{1}\pon\pi_{2}$,
from theorem~(\ref{logic:the:unique:representation}) of
page~\pageref{logic:the:unique:representation} we have $\gen
x\rho\neq\pi$. It follows that $\gen x\rho\in\subf(\pi_{1})$ or
$\gen x\rho\in\subf(\pi_{2})$. In the first case, $\gen
x\rho\preceq\pi_{1}$ and our desired implication follows from the
weak validity of $\sigma$ for $\pi_{1}$. In the second case, $\gen
x\rho\preceq\pi_{2}$  and we conclude from the weak validity of
$\sigma$ for $\pi_{2}$.
\end{proof}

\begin{prop}\label{logic:prop:FUAP:validsub:recursion:gen}
Let $V, W$ be sets and $\sigma:V\to W$ be a map. Let $\pi=\gen
x\pi_{1}$. Then $\sigma$ is weakly valid for $\pi$ \ifand\ it is
weakly valid for $\pi_{1}$ and:
    \begin{equation}\label{logic:eqn:FUAP:validsub:recursion:gen:1}
    u\in\spec(\pi_{1})\setminus\{x\}\ \Rightarrow\
    \sigma(u)\neq\sigma(x)
    \end{equation}
\end{prop}
\begin{proof}
First we show the 'only if' part: so we assume that $\sigma$ is
weakly valid for the proof $\pi=\gen x\pi_{1}$ where $x\in V$ and
$\pi_{1}\in\pvs$. From
proposition~(\ref{logic:prop:FUAP:validsub:subformula}), $\sigma$ is
weakly valid for $\pi_{1}\preceq\pi$. So it remains to show that the
implication~(\ref{logic:eqn:FUAP:validsub:recursion:gen:1}) holds.
So suppose $u\in\spec(\pi_{1})\setminus\{x\}$. We need to show that
$\sigma(u)\neq\sigma(x)$. However, this follows immediately from
$(iii)$ of definition~(\ref{logic:def:FUAP:validsub:validsub}), the
weak validity of $\sigma$ for $\pi$ and the fact that $\pi$ is a
sub-proof of itself, i.e. $\gen x\pi_{1}\preceq\pi$. We now show the
'if' part: so we assume that $\sigma$ is weakly valid for $\pi_{1}$
and furthermore that~(\ref{logic:eqn:FUAP:validsub:recursion:gen:1})
holds. We need to show that $\sigma$ is weakly valid for~$\pi$. So
let $\phi\in\hyp(\pi)\cup\ax(\pi)$. We need to show that $\sigma$ is
valid for~$\phi$. However, we have $\hyp(\pi)=\hyp(\pi_{1})$ and
$\ax(\pi)=\ax(\pi_{1})$ and consequently
$\phi\in\hyp(\pi_{1})\cup\ax(\pi_{1})$. Having assumed that $\sigma$
is weakly valid for $\pi_{1}$, it follows that $\sigma$ is valid for
$\phi$ as requested. In order to show that $\sigma$ is weakly valid
for $\pi$, it remains to prove that $(iii)$ of
definition~(\ref{logic:def:FUAP:validsub:validsub}) holds. So
suppose $\gen y\rho\preceq\pi=\gen x\pi_{1}$. We need to show:
    \[
    u\in\spec(\rho)\setminus\{y\}\ \Rightarrow\ \sigma(u)\neq\sigma(y)
    \]
We shall distinguish two cases: first we assume that $\gen
y\rho\preceq\pi_{1}$. Then our desired implication follows from the
weak validity of $\sigma$ for $\pi_{1}$. Next we assume that $\gen
y\rho=\pi=\gen x\pi_{1}$. From
theorem~(\ref{logic:the:unique:representation}) of
page~\pageref{logic:the:unique:representation} we obtain $y=x$ and
$\rho=\pi_{1}$ and our desired implication is exactly our
assumption~(\ref{logic:eqn:FUAP:validsub:recursion:gen:1}).
\end{proof}


\begin{prop}\label{logic:prop:FUAP:validsub:freevar}
Let $V, W$ be sets and $\sigma:V\to W$ be a map. Then for every
proof $\pi\in\pvs$, if  the substitution $\sigma$ is weakly valid
for $\pi$ we have:
    \[
    \spec(\sigma(\pi))=\sigma(\spec(\pi))
    \]
where $\sigma:\pvs\to{\bf\Pi}(W)$ also denotes the proof
substitution mapping.
\end{prop}
\begin{proof}
We assume that $\sigma$ is weakly valid for $\pi$. Then $\sigma$ is
valid for every $\phi\in\hyp(\pi)$. Hence from
proposition~(\ref{logic:prop:FOPL:valid:free:commute}),
$\free(\sigma(\phi))=\sigma(\free(\phi))$ for all $\phi\in\hyp(\pi)$
and:
    \begin{eqnarray*}
    \spec(\sigma(\pi))&=&\spec(\,\hyp(\sigma(\pi))\,)\\
    \mbox{prop.~(\ref{logic:prop:FUAP:substitution:hypothesis})}\ \rightarrow
    &=&\spec(\,\sigma(\hyp(\pi))\,)\\
    &=&\cup\{\,\free(\psi)\ :\ \psi\in\sigma(\hyp(\pi))\,\}\\
    &=&\cup\{\,\free(\sigma(\phi))\ :\ \phi\in\hyp(\pi)\,\}\\
    \mbox{$\sigma$ valid for $\phi\in\hyp(\pi)$}\ \rightarrow
    &=&\cup\{\,\sigma(\free(\phi))\ :\ \phi\in\hyp(\pi)\,\}\\
    &=&\sigma(\,\cup\{\,\free(\phi)\ :\ \phi\in\hyp(\pi)\,\}\,)\\
    &=&\sigma(\,\spec(\hyp(\pi))\,)\\
    &=&\sigma(\spec(\pi))
    \end{eqnarray*}
\end{proof}

\begin{prop}\label{logic:prop:FUAP:validsub:injective}
Let $V, W$ be sets and $\sigma:V\to W$ be a map. Let $\pi\in\pvs$.
We assume that $\sigma_{|\var(\pi)}$ is an injective map. Then
$\sigma$ is weakly valid for $\pi$.
\end{prop}
\begin{proof}
We assume that $\sigma:V\to W$ is injective on $\var(\pi)$. We need
to show that $\sigma$ is weakly valid for $\pi$. So let
$\phi\in\hyp(\pi)$. We need to show that $\sigma$ is valid for
$\phi$. Using proposition~(\ref{logic:prop:FOPL:valid:injective}) it
is sufficient to prove that $\sigma$ is injective on $\var(\phi)$.
It is therefore sufficient to show that
$\var(\phi)\subseteq\var(\pi)$, which follows from
proposition~(\ref{logic:prop:FUAP:variable:subformula}) and
$\phi\preceq\pi$, this last inequality being itself a consequence of
proposition~(\ref{logic:prop:FUAP:hypothesis:charac}) and
$\phi\in\hyp(\pi)$. So we now assume that $\phi\in\ax(\pi)$. We need
to show that $\sigma$ is valid for $\phi$. Using
proposition~(\ref{logic:prop:FOPL:valid:injective}) it is sufficient
to prove that $\sigma$ is injective on $\var(\phi)$. It is therefore
sufficient to show that $\var(\phi)\subseteq\var(\pi)$ or
equivalently $\var(\axi\phi)\subseteq\var(\pi)$. This follows from
proposition~(\ref{logic:prop:FUAP:variable:subformula}) and
$\axi\phi\preceq\pi$, this last inequality being itself a
consequence of proposition~(\ref{logic:prop:FUAP:axiomset:charac})
and $\phi\in\ax(\pi)$. In order to show that $\sigma$ is weakly
valid for $\pi$, we finally consider $\gen x\pi_{1}\preceq\pi$. It
remains to show that the following implication holds:
    \[
    u\in\spec(\pi_{1})\setminus\{x\}\ \Rightarrow\
    \sigma(u)\neq\sigma(x)
    \]
So suppose that $u\in\spec(\pi_{1})\setminus\{x\}$. In particular we
have $u\neq x$ and we need to show that $\sigma(u)\neq\sigma(x)$.
Having assumed $\sigma$ is injective on $\var(\pi)$, it is
sufficient to prove that $\{u,x\}\subseteq\var(\pi)$. From $\gen
x\pi_{1}\preceq\pi$ and
proposition~(\ref{logic:prop:FUAP:variable:subformula}) we obtain:
    \[
    x\in\var(\gen x\pi_{1})\subseteq\var(\pi)
    \]
So it remains to show that $u\in\var(\pi)$. Using
propositions~(\ref{logic:prop:FUAP:freevar:subset:variable})
and~(\ref{logic:prop:FUAP:variable:subformula})\,:
    \[
    u\in\spec(\pi_{1})\subseteq\var(\pi_{1})\subseteq \var(\gen x\pi_{1})\subseteq\var(\pi)
    \]
\end{proof}

\begin{prop}\label{logic:prop:FUAP:validsub:singlevar}
Let $V$ be a set and $x,y\in V$. Let $\pi\in\pvs$. Then we have:
    \[
    y\not\in\var(\pi)\ \Rightarrow\ (\mbox{$[y/x]$ weakly valid for
    $\pi$})
    \]
\end{prop}
\begin{proof}
We assume that $y\not\in\var(\pi)$. We need to show that $[y/x]$ is
weakly valid for $\pi$. Using
proposition~(\ref{logic:prop:FUAP:validsub:injective}), it is
sufficient to prove that $[y/x]_{|\var(\pi)}$ is an injective map.
This follows immediately from $y\not\in\var(\pi)$ and
proposition~(\ref{logic:prop:FOPL:singlevar:support}).
\end{proof}

\begin{prop}\label{logic:prop:FUAP:validsub:composition}
Let $U$, $V$, $W$ be sets and $\tau:U\to V$ and $\sigma:V\to W$ be
maps. Then for all $\pi\in{\bf\Pi}(U)$ we have the equivalence:
\[
    (\mbox{$\tau$ w-valid for $\pi$})\land(\mbox{$\sigma$ w-valid for
    $\tau(\pi)$})\ \Leftrightarrow\ (\mbox{$\sigma\circ\tau$ w-valid for
    $\pi$})
\]
where $\tau:{\bf\Pi}(U)\to{\bf\Pi}(V)$ also denotes the associated
proof substitution mapping.
\end{prop}
\begin{proof}
First we show $\Rightarrow$\,: so we assume that $\tau$ is weakly
valid for $\pi$ and $\sigma$ is weakly valid for $\tau(\pi)$. We
need to show that $\sigma\circ\tau$ is weakly valid for $\pi$. So we
consider a formula $\phi\in\hyp(\pi)\cup\ax(\pi)$. We need to show
that $\sigma\circ\tau$ is valid for $\phi$. Using
proposition~(\ref{logic:prop:FOPL:valid:composition}), it is
sufficient to prove that $\tau$ is valid for $\phi$ and that
$\sigma$ is valid for $\tau(\phi)$. The fact that $\tau$ is valid
for $\phi$ follows immediately from the weak validity of $\tau$ for
$\pi$. So it remains to show that $\sigma$ is valid for
$\tau(\phi)$. However, from
proposition~(\ref{logic:prop:FUAP:substitution:hypothesis}) we have
$\tau(\hyp(\pi))=\hyp(\tau(\pi))$ and furthermore it follows from
proposition~(\ref{logic:prop:FUAP:axiomset:substitution}) that
$\tau(\ax(\pi))=\ax(\tau(\pi))$. Hence we see that
$\tau(\phi)\in\hyp(\tau(\pi))\cup\ax(\tau(\pi))$, and having assumed
that $\sigma$ is weakly valid for $\tau(\pi)$ we conclude that
$\sigma$ is valid for $\tau(\phi)$ as requested. We now consider
$\gen x\pi_{1}\preceq\pi$. In order to show that $\sigma\circ\tau$
is weakly valid for $\pi$, given $u\in U$ we need:
    \begin{equation}\label{logic:eqn:FUAP:validsub:composition:1}
    u\in\spec(\pi_{1})\setminus\{x\}\ \Rightarrow\
    \sigma\circ\tau(u)\neq\sigma\circ\tau(x)
    \end{equation}
However, since $\tau$ is weakly valid for $\pi$, we know the
following implication is true:
    \begin{equation}\label{logic:eqn:FUAP:validsub:composition:2}
    u\in\spec(\pi_{1})\setminus\{x\}\ \Rightarrow\
    \tau(u)\neq\tau(x)
    \end{equation}
So we assume that $u\in\spec(\pi_{1})\setminus\{x\}$. We need to
show that $\sigma\circ\tau(u)\neq\sigma\circ\tau(x)$. Defining
$v=\tau(u)\in V$ and $y=\tau(x)\in V$, we need to show that
$\sigma(v)\neq\sigma(y)$. However, from $\gen x\pi_{1}\preceq\pi$
and proposition~(\ref{logic:prop:FUAP:substitution:subformula}) we
obtain $\gen y\tau(\pi_{1})\preceq\tau(\pi)$. Having assumed that
$\sigma$ is weakly valid for $\tau(\pi)$, the following implication
holds:
    \begin{equation}\label{logic:eqn:FUAP:validsub:composition:3}
    v\in\spec(\tau(\pi_{1}))\setminus\{y\}\ \Rightarrow\
    \sigma(v)\neq\sigma(y)
    \end{equation}
Thus, in order to show $\sigma(v)\neq\sigma(y)$ it is sufficient to
prove $v\in\spec(\tau(\pi_{1}))\setminus\{y\}$. We already know from
the implication~(\ref{logic:eqn:FUAP:validsub:composition:2}) that
$v\neq y$. So it remains to show that $v\in\spec(\tau(\pi_{1}))$.
However we have $v=\tau(u)\in\tau(\spec(\pi_{1}))$. It is therefore
sufficient to prove that
$\tau(\spec(\pi_{1}))=\spec(\tau(\pi_{1}))$. Using
proposition~(\ref{logic:prop:FUAP:validsub:freevar}) we simply need
to show that $\tau$ is weakly valid for $\pi_{1}$ which follows from
proposition~(\ref{logic:prop:FUAP:validsub:subformula}) and the fact
that $\pi_{1}\preceq\gen x\pi_{1}\preceq\pi$, i.e. that $\pi_{1}$ is
a sub-proof of $\pi$. We now show $\Leftarrow$\,: so we assume that
$\sigma\circ\tau$ is weakly valid for $\pi$. We need to show that
$\tau$ is weakly valid for $\pi$ and $\sigma$ is weakly valid for
$\tau(\pi)$. First we show that $\tau$ is weakly valid for $\pi$: so
let $\phi\in\hyp(\pi)\cup\ax(\pi)$. We need to show that $\tau$ is
valid for $\phi$. However we know by assumption that
$\sigma\circ\tau$ is valid for $\phi$. Using
proposition~(\ref{logic:prop:FOPL:valid:composition}) it follows
that $\tau$ is valid for $\phi$ as requested. We now consider $\gen
x\pi_{1}\preceq\pi$. In order to show that $\tau$ is weakly valid
for $\pi$ we need to show the implication
$u\in\spec(\pi_{1})\setminus\{x\}\ \Rightarrow\ \tau(u)\neq\tau(x)$
holds. So let $u\in\spec(\pi_{1})\setminus\{x\}$. We need to show
that $\tau(u)\neq\tau(x)$. However, we know by assumption that
$\sigma\circ\tau(u)\neq\sigma\circ\tau(x)$. So $\tau(u)\neq\tau(x)$
must follow. We now show that $\sigma$ is weakly valid for
$\tau(\pi)$. So let $\psi\in\hyp(\tau(\pi))\cup\ax(\tau(\pi))$. We
need to show that $\sigma$ is valid for $\psi$. From
proposition~(\ref{logic:prop:FUAP:substitution:hypothesis}),
$\hyp(\tau(\pi))=\tau(\hyp(\pi))$ and from
proposition~(\ref{logic:prop:FUAP:axiomset:substitution})
$\ax(\tau(\pi))=\tau(\ax(\pi))$. It follows that $\psi=\tau(\phi)$
for some $\phi\in\hyp(\pi)\cup\ax(\pi)$. Having assumed that
$\sigma\circ\tau$ is weakly valid for $\pi$, it follows that
$\sigma\circ\tau$ is valid for $\phi$. Using
proposition~(\ref{logic:prop:FOPL:valid:composition}) we see that
$\sigma$ is valid for $\tau(\phi)$. In other words, $\sigma$ is
valid for $\psi$ as requested. We now consider $\gen
y\rho_{1}\preceq\tau(\pi)$. In order to show that $\sigma$ is weakly
valid for $\tau(\pi)$ we need to prove:
    \begin{equation}\label{logic:eqn:FUAP:validsub:composition:4}
    v\in\spec(\rho_{1})\setminus\{y\}\ \Rightarrow\
    \sigma(v)\neq\sigma(y)
    \end{equation}
However since $\gen y\rho_{1}$ is a sub-proof of $\tau(\pi)$, from
proposition~(\ref{logic:prop:FUAP:substitution:subformula}) we have
$\gen y\rho_{1}=\tau(\rho)$ for some $\rho\preceq\pi$. Furthermore,
from theorem~(\ref{logic:the:unique:representation}) of
page~\pageref{logic:the:unique:representation} the proof
$\rho\in{\bf\Pi}(U)$ can only be of four types, namely $\rho=\phi$
for some $\phi\in{\bf P}(U)$ or $\rho=\axi\phi$ for some
$\phi\in{\bf P}(U)$ or $\rho=\pi_{1}\pon\pi_{2}$ or $\rho=\gen
x\pi_{1}$. From the equation $\gen y\rho_{1}=\tau(\rho)$ and the
uniqueness of representation stated in
theorem~(\ref{logic:the:unique:representation}) it is clear the only
possibility is $\rho=\gen x\pi_{1}$ for some $x\in U$ and
$\pi_{1}\in{\bf\Pi}(U)$. Hence, we have found $x$ and $\pi_{1}$ such
that $\gen x\pi_{1}\preceq\pi$ and $\gen y\rho_{1}=\tau(\gen
x\pi_{1})=\gen\tau(x)\tau(\pi_{1})$, i.e. $y=\tau(x)$ and
$\rho_{1}=\tau(\pi_{1})$. So the
implication~(\ref{logic:eqn:FUAP:validsub:composition:4}) can be
stated as:
    \begin{equation}\label{logic:eqn:FUAP:validsub:composition:5}
    v\in\spec(\tau(\pi_{1}))\setminus\{\tau(x)\}\ \Rightarrow\
    \sigma(v)\neq\sigma\circ\tau(x)
    \end{equation}
So let $v\in\spec(\tau(\pi_{1}))\setminus\{\tau(x)\}$. We need to
show that $\sigma(v)\neq\sigma\circ\tau(x)$. However, from
proposition~(\ref{logic:prop:FUAP:freevar:substitution}) we have
$\spec(\tau(\pi_{1}))\subseteq\tau(\spec(\pi_{1}))$. It follows that
we have $v=\tau(u)$ for some $u\in\spec(\pi_{1})$. Furthermore,
since $v\neq\tau(x)$ we have $u\neq x$ and consequently
$u\in\spec(\pi_{1})\setminus\{x\}$. Furthermore, recall that $\gen
x\pi_{1}\preceq\pi$. Having assumed that $\sigma\circ\tau$ is weakly
valid for $\pi$ the following implication holds:
    \begin{equation}\label{logic:eqn:FUAP:validsub:composition:6}
    u\in\spec(\pi_{1})\setminus\{x\}\ \Rightarrow\
    \sigma\circ\tau(u)\neq\sigma\circ\tau(x)
    \end{equation}
Hence, we conclude that $\sigma\circ\tau(u)\neq\sigma\circ\tau(x)$
which is $\sigma(v)\neq\sigma\circ\tau(x)$ as requested.
\end{proof}

\begin{prop}\label{logic:prop:FUAP:validsub:equal:image}
Let $V,W$ be sets and $\sigma,\tau:V\to W$ be maps. Let $\pi\in\pvs$
such that the equality $\sigma(\pi)=\tau(\pi)$ holds. Then we have
the equivalence:
    \[
    (\mbox{$\sigma$ weakly valid for $\pi$})\ \Leftrightarrow\
    (\mbox{$\tau$ weakly valid for $\pi$})
    \]
\end{prop}
\begin{proof}
It is sufficient to prove $\Rightarrow$\,: so we assume that
$\sigma(\pi)=\tau(\pi)$ and that $\sigma$ is weakly valid for $\pi$.
We need to show that $\tau$ is weakly valid for $\pi$. Using
proposition~(\ref{logic:prop:FUAP:variable:support}) we see that
$\sigma$ and $\tau$ coincide on $\var(\pi)$. So let
$\phi\in\hyp(\pi)\cup\ax(\pi)$. We need to show that $\tau$ is valid
for $\phi$. However since $\sigma$ is weakly valid for $\pi$, we
know that $\sigma$ is itself valid for $\phi$. Using
proposition~(\ref{logic:prop:FOPL:validsub:image}), in order to show
that $\tau$ is also valid for $\phi$ it is sufficient to prove that
$\sigma(\phi)=\tau(\phi)$. From
proposition~(\ref{logic:prop:substitution:support}) it is therefore
sufficient to show that $\sigma$ and $\tau$ coincide on
$\var(\phi)$. Having established that $\sigma$ and $\tau$ coincide
on $\var(\pi)$, we simply need to prove that
$\var(\phi)\subseteq\var(\pi)$. We shall distinguish two cases:
first we assume that $\phi\in\hyp(\pi)$. Then from
proposition~(\ref{logic:prop:FUAP:hypothesis:charac}) we have
$\phi\preceq\pi$ and using
proposition~(\ref{logic:prop:FUAP:variable:subformula}) it follows
that $\var(\phi)\subseteq\var(\pi)$ as requested. Next we assume
that $\phi\in\ax(\pi)$. Then from
proposition~(\ref{logic:prop:FUAP:axiomset:charac}) we see that
$\axi\phi\preceq\pi$ and again from
proposition~(\ref{logic:prop:FUAP:variable:subformula}) we have
$\var(\phi)=\var(\axi\phi)\subseteq\var(\pi)$. So we now consider
$\gen x\pi_{1}\preceq\pi$. In order to show that $\tau$ is weakly
valid for $\pi$, for all $u\in V$ we need to show:
    \begin{equation}\label{logic:eqn:FUAP:validsub:equal:image:1}
    u\in\spec(\pi_{1})\setminus\{x\}\ \Rightarrow\
    \tau(u)\neq\tau(x)
    \end{equation}
However, from the weak validity of $\sigma$ for $\pi$ we know the
following is true:
    \begin{equation}\label{logic:eqn:FUAP:validsub:equal:image:2}
    u\in\spec(\pi_{1})\setminus\{x\}\ \Rightarrow\
    \sigma(u)\neq\sigma(x)
    \end{equation}
In order to establish~(\ref{logic:eqn:FUAP:validsub:equal:image:1}),
since $\sigma$ and $\tau$ coincide on $\var(\pi)$ it is therefore
sufficient to prove that $\{u,x\}\subseteq\var(\pi)$. From $\gen
x\pi_{1}\preceq\pi$ and
proposition~(\ref{logic:prop:FUAP:variable:subformula})\,:
    \[
    x\in\var(\gen x\pi_{1})\subseteq\var(\pi)
    \]
So it remains to show that $u\in\var(\pi)$. Using
propositions~(\ref{logic:prop:FUAP:freevar:subset:variable})
and~(\ref{logic:prop:FUAP:variable:subformula})\,:
    \[
    u\in\spec(\pi_{1})\subseteq\var(\pi_{1})\subseteq \var(\gen x\pi_{1})\subseteq\var(\pi)
    \]
\end{proof}
