In definition~(\ref{logic:def:substitution}) we introduced the
substitution $\sigma:\pv\to{\bf P}(W)$ associated with a map
$\sigma:V\to W$. This substitution is crude in the sense that it is
generally not capture-avoiding. In other words, given a formula
$\phi\in\pv$, the substitution $\sigma$ is generally not valid for
$\phi$, as per definition~(\ref{logic:def:FOPL:valid:substitution}).
However, the introduction of the minimal transform of
definition~(\ref{logic:def:FOPL:mintransform:transform}) allowed us
to build on this elementary substitution and derive a new type of
variable substitution mapping $\sigma:\pv\to{\bf P}(W)$, the
so-called essential substitutions of
definition~(\ref{logic:def:FOPL:esssubstprop:essential}). Essential
substitutions are capture-avoiding and their existence is guaranteed
by theorem~(\ref{logic:the:FOPL:esssubst:existence}) with a mere
cardinality assumption on $V$ and $W$. A notable application of
essential substitutions is our ability to put forward an
axiomatization of first order logic where the specialization axioms
of definition~(\ref{logic:def:FOPL:specialization:axiom}), i.e.
$\forall x\phi_{1}\to\phi_{1}[y/x]$ are stated without the usual
caveat of {\em $y$ being free for $x$ in $\phi_{1}$}.

What was done for formulas can be done for proofs. Given a map
$\sigma:V\to W$ and a proof $\pi\in\pvs$, it is meaningful to ask
which proof $\sigma(\pi)\in{\bf\Pi}(W)$ can be derived by
systematically substituting variables in $\pi$ according to the
map~$\sigma$. As in the case of formulas, we cannot expect miracles
from this as the transformation will most likely be too crude in
general. However, this may be a good starting point which is
certainly worth investigating. There are many reasons for us to
attempt building substitutions on proofs which are associated in
some way with a map $\sigma:V\to W$. Any $\sigma:\pvs\to{\bf\Pi}(W)$
can be viewed as a tool to create new proofs. This tool may prove
invaluable when attempting to show that a {\em proof exists}.
G\"odel's completeness theorem is a case when the existence of a
proof has to be established. Many common arguments in mathematical
logic involve {\em language extensions} which can be viewed as
embeddings. It is usually implicit in the argument that what is
provable (or inconsistent) in one space, remains provable in the
other. As we do not wish to take anything for granted, this is one
reason for us to look into substitution of variables in proofs.
Another reason is to establish some form of {\em structurality
condition} for our consequence relation~$\vdash$. This condition is
stated in~\cite{BlokPigozzi} and~\cite{Rivieccio} as
$\Gamma\vdash\phi\ \Rightarrow\ \sigma(\Gamma)\vdash\sigma(\phi)$.
However, at this point of the document it is not clear to us what
type of substitution $\sigma$ would allow the structurality
condition to hold in the context of first order logic with terms.
Our hope is that essential substitutions will work, and indeed they
will as theorem~(\ref{logic:the:FUAP:substitutiontheorem:main}) of
page~\pageref{logic:the:FUAP:substitutiontheorem:main} will show.
The work leading up to this important theorem requires a new tool,
namely the ability to consider variable substitutions in proofs.
With this in mind, we define: \index{proof@Proof
substitution}\index{substitution@Proof
substitution}\index{substitution@$\pi[y/x]$ : image of $\pi$ by
$[y/x]$}\index{substitution@$\sigma(\pi)$ : image
$\sigma:\pvs\to{\bf\Pi}(W)$}\index{substitution@$\sigma$ :
associated
$\sigma:\pvs\to{\bf\Pi}(W)$}\index{substitution@Substitution of
variables}
\begin{defin}\label{logic:def:FUAP:substitution:substitution}
Let $V$, $W$ be sets and $\sigma:V\to W$ be a map. We call {\em
proof substitution associated with $\sigma$}, the map
$\sigma^{*}:\pvs\to{\bf \Pi}(W)$ defined by:
    \begin{equation}\label{logic:eqn:FUAP:substitution:substitution:1}
    \forall\pi\in\pvs\ ,\ \sigma^{*}(\pi)=\left\{
                    \begin{array}{lcl}
                    \sigma(\phi)&\mbox{\ if\ }&\pi=\phi\in\pv\\
                    \axi\sigma(\phi)&\mbox{\ if\ }&\pi=\axi\phi\\
                    \sigma^{*}(\pi_{1})\pon\,\sigma^{*}(\pi_{2})
                    &\mbox{\ if\ }&\pi=\pi_{1}\pon\pi_{2}\\
                    \gen \sigma(x)\sigma^{*}(\pi_{1})&\mbox{\ if\ }&\pi=\gen
                    x\pi_{1}\\
                    \end{array}\right.
    \end{equation}
where $\sigma:\pv\to{\bf P}(W)$ also denotes the associated
substitution mapping.
\end{defin}
Given a map $\sigma:V\to W$ and $\phi\in\pv$ the notation
$\sigma(\phi)$ is potentially ambiguous. Since $\pv\subseteq\pvs$,
it may refer to the proof of
definition~(\ref{logic:def:FUAP:substitution:substitution}) or to
the formula of definition~(\ref{logic:def:substitution}). Luckily,
the two notions coincide.
\begin{prop}
The structural recursion of {\em
definition~(\ref{logic:def:FUAP:substitution:substitution})} is
legitimate.
\end{prop}
\begin{proof}
We need to prove the existence and uniqueness of the map
$\sigma^{*}:\pvs\to{\bf \Pi}(W)$ satisfying the four conditions of
equation~(\ref{logic:eqn:FUAP:substitution:substitution:1}). We
shall do so using theorem~(\ref{logic:the:structural:recursion}) of
page~\pageref{logic:the:structural:recursion} applied to the free
universal algebra $X=\pvs$ with $X_{0}=\pv$ and $A={\bf\Pi}(W)$.
First we define $g_{0}:X_{0}\to A$ by $g_{0}(\phi)=\sigma(\phi)$
which ensures the first condition is met. Next, given a formula
$\phi\in\pv$ we define $h(\axi\phi):A^{0}\to A$ by setting
$h(\axi\phi)(0)=\axi\sigma(\phi)$ which ensures the second condition
is met. Next we define $h(\pon):A^{2}\to A$ by setting
$h(\pon)(\pi_{1},\pi_{2})=\pi_{1}\pon\pi_{2}$ which ensures the
third condition is met. Finally, given $x\in V$ we define $h(\gen
x):A^{1}\to A$ by setting $h(\gen x)(\pi_{1})=\gen\sigma(x)\pi_{1}$.
This guarantees our fourth condition is met.
\end{proof}

Given $\sigma:V\to W$, the associated proof substitution
$\sigma^{*}:\pvs\to{\bf\Pi}(W)$ is denoted $\sigma^{*}$ and not
$\sigma$. The following proposition will allow us to change that and
revert to the simple notation $\sigma:\pvs\to{\bf\Pi}(W)$ just as we
have done with $\sigma:\pv\to{\bf P}(W)$. This proposition should be
compared with
proposition~(\ref{logic:prop:substitution:composition}). Once we
know that $(\sigma\circ\tau)^{*}=\sigma^{*}\circ\tau^{*}$ we can
safely get rid of the star '$*$'.

\begin{prop}\label{logic:prop:FUAP:substitution:composition}
Let $U$, $V$, $W$ be sets. Let $\tau:U\to V$ and $\sigma:V\to W$ be
maps. Let $\tau^{*}:{\bf\Pi}(U)\to\pvs$ and
$\sigma^{*}:\pvs\to{\bf\Pi}(W)$ be the proof substitutions
associated with $\tau$ and $\sigma$ respectively. Then we have the
equality:
    \[
    (\sigma\circ\tau)^{*}=\sigma^{*}\circ\tau^{*}
    \]
where $(\sigma\circ\tau)^{*}:{\bf\Pi}(U)\to {\bf\Pi}(W)$ is the
proof substitution associated with $\sigma\circ\tau$.
\end{prop}
\begin{proof}
We need to show that
$(\sigma\circ\tau)^{*}(\pi)=\sigma^{*}\circ\tau^{*}(\pi)$ for all
$\pi\in{\bf\Pi}(U)$. We shall do so with a structural induction
argument, using theorem~(\ref{logic:the:proof:induction}) of
page~\pageref{logic:the:proof:induction}. First we assume that
$\pi=\phi$ for some $\phi\in{\bf P}(U)$. Then we have the
equalities:
    \begin{eqnarray*}
    (\sigma\circ\tau)^{*}(\pi)&=&(\sigma\circ\tau)^{*}(\phi)\\
    &=&\sigma\circ\tau(\phi)\\
    &=&\sigma^{*}(\tau(\phi))\\
    &=&\sigma^{*}\circ\tau^{*}(\phi)\\
    \end{eqnarray*}
Next we assume that $\pi=\axi\phi$ for some $\phi\in{\bf P}(U)$.
Then we have:
    \begin{eqnarray*}
    (\sigma\circ\tau)^{*}(\pi)&=&(\sigma\circ\tau)^{*}(\axi\phi)\\
    &=&\axi(\sigma\circ\tau(\phi))\\
    &=&\sigma^{*}(\axi\tau(\phi))\\
    &=&\sigma^{*}\circ\tau^{*}(\axi\phi)\\
    &=&\sigma^{*}\circ\tau^{*}(\pi)\\
    \end{eqnarray*}
Next we assume that $\pi=\pi_{1}\pon\pi_{2}$ where
$\pi_{1},\pi_{2}\in{\bf\Pi}(U)$ satisfy our property:
    \begin{eqnarray*}
    (\sigma\circ\tau)^{*}(\pi)&=&(\sigma\circ\tau)^{*}(\pi_{1}\pon\pi_{2})\\
    &=&(\sigma\circ\tau)^{*}(\pi_{1})\pon\,(\sigma\circ\tau)^{*}(\pi_{2})\\
    &=&\sigma^{*}\circ\tau^{*}(\pi_{1})\pon\,\sigma^{*}\circ\tau^{*}(\pi_{2})\\
    &=&\sigma^{*}(\,\tau^{*}(\pi_{1})\pon\,\tau^{*}(\pi_{2})\,)\\
    &=&\sigma^{*}\circ\tau^{*}(\pi_{1}\pon\pi_{2})\\
    &=&\sigma^{*}\circ\tau^{*}(\pi)\\
    \end{eqnarray*}
Finally, we assume that $\pi=\gen x\pi_{1}$ where $x\in U$ and
$\pi_{1}$ satisfies our property:
    \begin{eqnarray*}
    (\sigma\circ\tau)^{*}(\pi)&=&(\sigma\circ\tau)^{*}(\gen x\pi_{1})\\
    &=&\gen \sigma\circ\tau(x)\,(\sigma\circ\tau)^{*}(\pi_{1})\\
    &=&\gen \sigma\circ\tau(x)\,\sigma^{*}\circ\tau^{*}(\pi_{1})\\
    &=&\sigma^{*}(\,\gen\tau(x)\,\tau^{*}(\pi_{1})\,)\\
    &=&\sigma^{*}\circ\tau^{*}(\gen x\pi_{1})\\
    &=&\sigma^{*}\circ\tau^{*}(\pi)\\
    \end{eqnarray*}
\end{proof}

As in the case of
proposition~(\ref{logic:prop:substitution:identity}), the following
spells out the obvious:

\begin{prop}\label{logic:prop:FUAP:substitution:identity}
Let $V$ be a set and $i:V\to V$ be the identity mapping. Then, the
associated proof substitution mapping $i:\pvs\to\pvs$ is also the
identity.
\end{prop}
\begin{proof}
We need to show that $i(\pi)=\pi$ for all $\pi\in\pvs$. We shall do
so with a structural induction argument using
theorem~(\ref{logic:the:proof:induction}) of
page~\pageref{logic:the:proof:induction}. First we assume that
$\pi=\phi$ for some $\phi\in\pv$. Then $i(\pi)=i(\phi)$ and using
proposition~(\ref{logic:prop:substitution:identity}) it follows that
$i(\pi)=\phi=\pi$. Next we assume that $\pi=\axi\phi$ for some
$\phi\in\pv$. Then we have $i(\pi)=\axi i(\phi)=\axi\phi=\pi$. Next
we assume that $\pi=\pi_{1}\pon\pi_{2}$ in which case we obtain
$i(\pi)=i(\pi_{1})\pon\, i(\pi_{2})=\pi_{1}\pon\pi_{2}=\pi$. Finally
we assume that $\pi=\gen x\pi_{1}$ which yields $i(\pi)=\gen
i(x)i(\pi_{1})=\gen x\pi_{1}=\pi$.
\end{proof}

The map $\sigma:\pvs\to{\bf\Pi}(W)$ is a structural substitution as
per
definition~(\ref{logic:def:UA:structuralsub:structural:substitution}).
This allows us to claim that
$\subf(\sigma(\pi))=\sigma(\subf(\pi))$. Hence:
    \[
    \rho\preceq\pi\ \Rightarrow\ \sigma(\rho)\preceq\sigma(\pi)
    \]
In other words, if $\rho$ is a sub-proof of $\pi$ then
$\sigma(\rho)$ is a sub-proof of $\sigma(\pi)$. We also have the
converse: any sub-proof of $\sigma(\pi)$ is of the form
$\sigma(\rho)$ where $\rho$ is a sub-proof of $\pi$. The following
proposition is the analogue of
proposition~(\ref{logic:prop:FOPL:substitution:subformula}).


\begin{prop}\label{logic:prop:FUAP:substitution:subformula}
Let $V,W$ be sets and $\sigma:V\to W$ be a map. The associated proof
substitution $\sigma:\pvs\to{\bf\Pi}(W)$ is structural and for all
$\pi\in\pvs$\,:
    \[
    \subf(\sigma(\pi))=\sigma(\subf(\pi))
    \]
\end{prop}
\begin{proof}
By virtue of
proposition~(\ref{logic:prop:UA:structuralsub:subformula}) it is
sufficient to prove that $\sigma:\pvs\to{\bf\Pi}(W)$ is a structural
substitution as per
definition~(\ref{logic:def:UA:structuralsub:structural:substitution}).
Let $\alpha(V)$ and $\alpha(W)$ denote the Hilbert deductive proof
types associated with $V$ and $W$ respectively as per
definition~(\ref{logic:def:FOPL:proof:type}). Let
$q:\alpha(V)\to\alpha(W)$ be the map defined by:
    \[
    \forall f\in\alpha(V)\ ,\ q(f)=\left\{
        \begin{array}{lcl}
        \axi\sigma(\phi)&\mbox{\ if\ }&f=\axi\phi\\
        \pon&\mbox{\ if\ }&f=\pon\\
        \gen\sigma(x)&\mbox{\ if\ }&f=\gen x
        \end{array}
    \right.
    \]
Then $q$ is clearly arity preserving. In order to show that $\sigma$
is a structural substitution, we simply need to check that
properties $(i)$ and $(ii)$ of
definition~(\ref{logic:def:UA:structuralsub:structural:substitution})
are met. First we start with property $(i)$: so let $\pi=\phi$ for
some $\phi\in\pv$. We need to show that $\sigma(\pi)\in{\bf P}(W)$
which follows immediately from $\sigma(\pi)=\sigma(\phi)$. So we now
show property $(ii)$: given $f\in\alpha(V)$, given
$\pi\in\pvs^{\alpha(f)}$ we need to show that
$\sigma(f(\pi))=q(f)(\sigma(\pi))$. First we assume that
$f=\axi\phi$ for some $\phi\in\pv$. Then $\alpha(f)=0$, $\pi=0$ and
consequently we have:
    \begin{eqnarray*}
    \sigma(f(\pi))&=&\sigma(\axi\phi(0))\\
    \mbox{$\axi\phi(0)$ denoted '$\axi\phi$'}\ \rightarrow&=&\sigma(\axi\phi)\\
    \mbox{def.~(\ref{logic:def:FUAP:substitution:substitution})}\ \rightarrow
    &=&\axi\sigma(\phi)\\
    \mbox{$\axi\sigma(\phi)(0)$ denoted '$\axi\sigma(\phi)$'}\ \rightarrow
    &=&\axi\sigma(\phi)(0)\\
    \sigma:\{0\}\to\{0\}\ \rightarrow&=&q(\axi\phi)(\sigma(0))\\
    &=&q(f)(\sigma(\pi))
    \end{eqnarray*}
Next we assume that $f=\pon$. Then $\alpha(f)=2$ and given
$\pi=(\pi_{0},\pi_{1})$\,:
    \begin{eqnarray*}
    \sigma(f(\pi))&=&\sigma(\pi_{0}\pon\pi_{1})\\
    \mbox{def.~(\ref{logic:def:FUAP:substitution:substitution})}\ \rightarrow
    &=&\sigma(\pi_{0})\pon\,\sigma(\pi_{1})\\
    \sigma:\pvs^{2}\to{\bf\Pi}(W)^{2}\ \rightarrow&=&q(\pon)(\sigma(\pi_{0},\pi_{1}))\\
    &=&q(f)(\sigma(\pi))
    \end{eqnarray*}
Finally we assume that $f=\gen x$, $x\in V$. Then $\alpha(f)=1$ and
given $\pi=(\pi_{0})$\,:
    \begin{eqnarray*}
    \sigma(f(\pi))&=&\sigma(\gen x\pi_{0})\\
    \mbox{def.~(\ref{logic:def:FUAP:substitution:substitution})}\ \rightarrow
    &=&\gen\sigma(x)\sigma(\pi_{0})\\
    \sigma:\pvs^{1}\to{\bf\Pi}(W)^{1}\ \rightarrow
    &=&q(\gen x)(\sigma(\pi_{0}))\\
    &=&q(f)(\sigma(\pi))
    \end{eqnarray*}
\end{proof}

When dealing with congruences on \pvs\ we shall need the following:
\begin{prop}\label{logic:prop:FUAP:substitution:congruence}
Let $V$ and $W$ be sets and $\sigma:V\to W$ be a map. Let $\simeq$
be an arbitrary congruence on ${\bf\Pi}(W)$ and let $\equiv$ be the
relation on \pvs\ defined by:
    \[
    \pi\equiv\rho\ \Leftrightarrow\ \sigma(\pi)\simeq\sigma(\rho)
    \]
for all $\pi,\rho\in\pvs$. Then $\equiv$ is a congruence on \pvs.
\end{prop}
\begin{proof}
Since the congruence $\simeq$ on ${\bf\Pi}(W)$ is an equivalence
relation, $\equiv$ is clearly reflexive, symmetric and transitive on
\pvs. So we simply need to show that $\equiv$ is a congruent
relation on \pvs. Since $\simeq$ is reflexive, we have
$\sigma(\axi\phi)\simeq\sigma(\axi\phi)$ for all $\phi\in\pv$,  and
so $\axi\phi\equiv\axi\phi$. Suppose $\pi_{1},\pi_{2},\rho_{1}$ and
$\rho_{2}\in\pvs$ are such that $\pi_{1}\equiv\rho_{1}$ and
$\pi_{2}\equiv\rho_{2}$. Define $\pi=\pi_{1}\pon\pi_{2}$ and
$\rho=\rho_{1}\pon\rho_{2}$. We need to show that $\pi\equiv\rho$,
or equivalently that $\sigma(\pi)\simeq\sigma(\rho)$. This follows
from the fact that $\sigma(\pi_{1})\simeq\sigma(\rho_{1})$,
$\sigma(\pi_{2})\simeq\sigma(\rho_{2})$ and furthermore:
    \begin{eqnarray*}
    \sigma(\pi)&=&\sigma(\pi_{1}\pon\pi_{2})\\
    &=&\sigma(\pi_{1})\pon\,\sigma(\pi_{2})\\
    &\simeq&\sigma(\rho_{1})\pon\,\sigma(\rho_{2})\\
    &=&\sigma(\rho_{1}\pon\rho_{2})\\
    &=&\sigma(\rho)
    \end{eqnarray*}
where the intermediate $\simeq$ crucially depends on $\simeq$ being
a congruent relation on ${\bf\Pi}(W)$. We now suppose that
$\pi_{1},\rho_{1}\in\pvs$ are such that $\pi_{1}\equiv\rho_{1}$. Let
$x\in V$ and define $\pi=\gen x\pi_{1}$ and $\rho=\gen x\rho_{1}$.
We need to show that $\pi\equiv\rho$, or equivalently that
$\sigma(\pi)\simeq\sigma(\rho)$. This follows from
$\sigma(\pi_{1})\simeq\sigma(\rho_{1})$ and:
    \begin{eqnarray*}
    \sigma(\pi)&=&\sigma(\gen x\pi_{1})\\
    &=&\gen\sigma(x)\,\sigma(\pi_{1})\\
    &\simeq&\gen\sigma(x)\sigma(\rho_{1})\\
    &=&\sigma(\gen x\rho_{1})\\
    &=&\sigma(\rho)
    \end{eqnarray*}
where the intermediate $\simeq$ crucially depends on $\simeq$ being
a congruent relation.
\end{proof}
