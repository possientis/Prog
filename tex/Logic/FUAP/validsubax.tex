Given a map $\sigma:V\to W$ our plan is to carry over the sequent
$\Gamma\vdash\phi$ into another sequent
$\sigma(\Gamma)\vdash\sigma(\phi)$. Our strategy is based around the
study of variable substitutions in proofs. So from $\sigma:V\to W$
we define $\sigma:\pvs\to{\bf\Pi}(W)$ as per
definition~(\ref{logic:def:FUAP:substitution:substitution}). We know
that some of these substitutions will not be capture-avoiding, and
we extended the notion of {\em valid substitution} for proofs in
definition~(\ref{logic:def:FUAP:validsubproof:validsub}) to
formalize the idea of capture-avoidance. The hope is that a valid
substitution will satisfy the equality
$\val\circ\sigma(\pi)=\sigma\circ\val(\pi)$ provided the proof $\pi$
is totally clean, as per
definition~(\ref{logic:def:FUAP:clean:clean:proof}). From
proposition~(\ref{logic:prop:FUAP:clean:counterpart}), we know that
a sequent $\Gamma\vdash\phi$ always has an underlying proof $\pi$
which is totally clean. So if $\sigma$ happens to be valid for $\pi$
and satisfies our equality, then we see that $\sigma(\pi)$ is a
proof of $\sigma(\phi)$. We also know from
proposition~(\ref{logic:prop:FUAP:substitution:hypothesis}) that
$\hyp(\sigma(\pi))=\sigma(\hyp(\pi))\subseteq\sigma(\Gamma)$ and we
conclude that $\sigma(\pi)$ is a proof underlying the sequent
$\sigma(\Gamma)\vdash\sigma(\phi)$. There is of course a big
outstanding problem: in general, we have no way to tell whether a
substitution $\sigma$ is valid for the proof underlying the sequent
$\Gamma\vdash\phi$. The data which defines the sequent namely the
set $\Gamma\subseteq\pv$ and the formula $\phi\in\pv$ do not tell us
anything about the axioms being used in the proof. So we shall need
to have another idea. In the meantime, if we assume that $\sigma$ is
valid for $\pi$ and $\pi$ is totally clean, we want the conclusion
of $\sigma(\pi)$ to be the image by $\sigma$ of the conclusion of
$\pi$. Fundamentally, this cannot happen unless the axioms of $\pi$
remain axioms of first order logic after substitution by $\sigma$.
So let us look as the simple example $\pi=\axi\phi$ for some
$\phi\in\pv$. Saying that $\pi$ is totally clean amounts to saying
that the axiom invocation $\axi\phi$ is legitimate. In other words,
the formula $\phi$ is indeed an axiom of first order logic, i.e. we
have $\phi\in\av$ and consequently $\val(\pi)=\phi$. In order to
have $\val(\sigma(\pi))=\sigma(\phi)$, since
$\sigma(\pi)=\axi\sigma(\phi)$ we need $\sigma(\phi)$ to be itself
an axiom of first order logic. This will not be the case in general,
but having assumed that $\sigma$ is valid for $\pi$, it is valid for
the formula $\phi$. If the substitution $\sigma$ is capture-avoiding
and $\phi$ is an axiom of first order logic, we should hope that
$\sigma(\phi)$ is itself an axiom. Proving this fact is the purpose
of lemma~(\ref{logic:lemma:FUAP:substitution:axiom}) below. Note
that the validity of $\sigma$ for $\phi$ is crucially important when
dealing with quantification axioms and specialization axioms, as per
definition~(\ref{logic:def:FOPL:quantification:axiom})
and~(\ref{logic:def:FOPL:specialization:axiom}). It is not difficult
to see that $\sigma(\phi)$ is always an axiom without the validity
assumption, whenever $\phi$ is a simplification, Frege or
transposition axiom. But consider the case when $\phi=\forall
x[\,(y\in y)\to(x\in x)\,]\to[\,(y\in y)\to\forall x(x\in x)\,]$
with $x\neq y$ and $\sigma=[y/x]$. Then $\phi$ is a quantification
axiom while $\sigma$ is not valid for $\phi$. It is clear that
$\sigma(\phi)=\forall y[\,(y\in y)\to(y\in y)\,]\to[\,(y\in
y)\to\forall y(y\in y)\,]$ is not an axiom of first order logic. In
fact, for those already familiar with model theory and the soundness
theorem~(\ref{logic:the:FOPL:soundness:soundness}) of
page~\pageref{logic:the:FOPL:soundness:soundness}, it is clear that
$\sigma(\phi)$ is not provable, i.e. the sequent
$\vdash\sigma(\phi)$ is false. Now consider $\phi=\forall y(x\in
y)\to(x\in z)$ where $x,y,z$ are distinct. Then $\phi$ is a
specialization axiom while $\sigma=[y/x]$ is not valid for $\phi$.
We have $\sigma(\phi)=\forall y(y\in y)\to(y\in z)$ which is not an
axiom of first order logic and is easily seen to be false for some
model and assignment.

\index{axiom@Image of axiom by valid sub}\index{valid@Image of axiom
by valid sub}
\begin{lemma}\label{logic:lemma:FUAP:substitution:axiom}
Let $V$, $W$ be sets and $\sigma:V\to W$ be a map. Given
$\phi\in\av$\,:
    \[
    (\mbox{$\sigma$ valid for $\phi$})\ \Rightarrow\
    \sigma(\phi)\in{\bf A}(W)
    \]
In other words, the image of an axiom by a valid substitution is an
axiom.
\end{lemma}
\begin{proof}
Let $\phi\in\av$ and $\sigma:V\to W$ be valid for $\phi$. We need to
show that $\sigma(\phi)$ is an axiom of first order logic. Following
definition~(\ref{logic:def:FOPL:first:order:axiom}) we shall
consider the five possible types of axioms separately: first we
assume that $\phi$ is a simplification axiom. Using
definition~(\ref{logic:def:FOPL:simplification:axiom}) we have $\phi
= \phi_{1}\to(\phi_{2}\to\phi_{1})$ for $\phi_{1},\phi_{2}\in\pv$.
It follows that $\sigma(\phi)=\psi_{1}\to(\psi_{2}\to\psi_{1})$
where $\psi_{1}=\sigma(\phi_{1})$ and $\psi_{2}=\sigma(\phi_{2})$.
So $\sigma(\phi)$ is also a simplification axiom. Next we assume
that $\phi$ is a Frege axiom. Using
definition~(\ref{logic:def:FOPL:frege:axiom}), $ \phi =
    [\phi_{1}\to(\phi_{2}\to\phi_{3})]\to[(\phi_{1}\to\phi_{2})\to(\phi_{1}\to\phi_{3})]$
for $\phi_{1},\phi_{2},\phi_{3}\in\pv$. Setting
$\psi_{1}=\sigma(\phi_{1})$, $\psi_{2}=\sigma(\phi_{2})$ and
$\psi_{3}=\sigma(\phi_{3})$ we have:
    \[
    \sigma(\phi) =
    [\psi_{1}\to(\psi_{2}\to\psi_{3})]\to[(\psi_{1}\to\psi_{2})\to(\psi_{1}\to\psi_{3})]
    \]
So $\sigma(\phi)$ is also a Frege axiom. Next we assume that $\phi$
is a transposition axiom. Using
definition~(\ref{logic:def:FOPL:transposition:axiom}) we have $\phi
=[(\phi_{1}\to\bot)\to\bot]\to\phi_{1}$ for some $\phi_{1}\in\pv$.
It follows that $\sigma(\phi)
=[(\psi_{1}\to\bot)\to\bot]\to\psi_{1}$ where
$\psi_{1}=\sigma(\phi_{1})$ and $\sigma(\phi)$ is also a
transposition axiom. Next we assume that $\phi$ is a quantification
axiom. Using definition~(\ref{logic:def:FOPL:quantification:axiom})
we have $\phi =\forall x(\phi_{1}\to\phi_{2})\to(\phi_{1} \to\forall
x \phi_{2})$ where $\phi_{1},\phi_{2}\in\pv$ and
$x\not\in\free(\phi_{1})$. Setting $\psi_{1}=\sigma(\phi_{1})$ and
$\psi_{2}=\sigma(\phi_{2})$ we obtain:
    \[
    \sigma(\phi) =\forall y(\psi_{1}\to\psi_{2})\to(\psi_{1} \to\forall
    y \psi_{2})
    \]
where $y=\sigma(x)$. However, we cannot argue that $\sigma(\phi)$ is
a quantification axiom until we prove that
$y\not\in\free(\psi_{1})$. So suppose to the contrary that
$y\in\free(\psi_{1})$. From
proposition~(\ref{logic:prop:freevar:of:substitution:inclusion}) we
have $\free(\sigma(\phi_{1}))\subseteq\sigma(\free(\phi_{1}))$ and
it follows that $y\in\sigma(\free(\phi_{1}))$. Hence, there exists
$u\in\free(\phi_{1})$ such that $\sigma(x)=y=\sigma(u)$. However, we
know that $x\not\in\free(\phi_{1})$ and consequently $u\neq x$.
Thus, we see that $u\in\free(\phi_{1})\setminus\{x\}=\free(\forall
x\phi_{1})$ and furthermore $\sigma(x)=\sigma(u)$. By virtue of
proposition~(\ref{logic:prop:FOPL:valid:recursion:quant}), this
means $\sigma$ is not a valid substitution for $\forall x\phi_{1}$.
We shall obtain our desired contradiction by showing that $\sigma$
is in fact valid for $\forall x\phi_{1}$. Indeed, by assumption we
know that $\sigma$ is valid for $\phi$. Using
proposition~(\ref{logic:prop:FOPL:valid:recursion:imp}) it follows
that $\sigma$ is valid for $\forall x(\phi_{1}\to\phi_{2})$. Hence,
from proposition~(\ref{logic:prop:FOPL:valid:recursion:quant}) we
see that $\sigma$ is valid for $\phi_{1}\to\phi_{2}$ and
furthermore, given $v\in V$ we have the implication:
    \[
    v\in\free(\forall x(\phi_{1}\to\phi_{2}))\ \Rightarrow\
    \sigma(v)\neq\sigma(x)
    \]
Using proposition~(\ref{logic:prop:FOPL:valid:recursion:imp}) once
more, it follows that $\sigma$ is valid for $\phi_{1}$ and:
    \[
    v\in\free(\forall x\phi_{1})\ \Rightarrow\
    \sigma(v)\neq\sigma(x)
    \]
this last implication being a consequence of $\free(\forall
x\phi_{1})\subseteq\free(\forall x(\phi_{1}\to\phi_{2}))$. So we
conclude from
proposition~(\ref{logic:prop:FOPL:valid:recursion:quant}) that
$\sigma$ is valid for $\forall x\phi_{1}$, which completes our proof
that $\sigma(\phi)$ is indeed a quantification axiom. Note that the
quantification axiom is the first case where we needed to use the
assumption of validity of $\sigma$ for $\phi$. Finally, we assume
that $\phi$ is a specialization axiom. Using
definition~(\ref{logic:def:FOPL:specialization:axiom}) we have $\phi
=\forall x\phi_{1}\to\phi_{1}[y/x]$ where $\phi_{1}\in V$, $x,y\in
V$ and $[y/x]:\pv\to\pv$ refers to an essential substitution of $y$
in place of $x$, i.e. an essential substitution associated with
$[y/x]:V\to V$. Defining $\psi_{1}=\sigma(\phi_{1})$  and
$\psi_{1}^{*}=\sigma\circ[y/x](\phi_{1})$\,:
    \[
    \sigma(\phi)=\forall z\psi_{1}\to\psi_{1}^{*}
    \]
where $z=\sigma(x)$. In order to show that $\sigma(\phi)$ is itself
a specialization axiom, from
proposition~(\ref{logic:prop:FOPL:specialization:axiom:2}) it is
sufficient to show that $\psi_{1}^{*}\sim\psi_{1}[\sigma(y)/z]$
where the map $[\sigma(y)/z]:{\bf P}(W)\to{\bf P}(W)$ refers to an
essential substitution of $\sigma(y)$ in place of $z=\sigma(x)$, and
$\sim$ is the substitution congruence on ${\bf P}(W)$. In order to
prove such equivalence, using
theorem~(\ref{logic:the:FOPL:mintransfsubcong:kernel}) of
page~\pageref{logic:the:FOPL:mintransfsubcong:kernel} it is
sufficient to prove the equality between the corresponding minimal
transforms, which goes as follows:
    \begin{eqnarray*}
    {\cal M}(\psi_{1}^{*})&=&{\cal
    M}\circ\sigma\circ[y/x](\phi_{1})\\
    \mbox{A: to be proved}\ \rightarrow
    &=&\bar{\sigma}\circ{\cal M}\circ[y/x](\phi_{1})\\
    \mbox{$[y/x]$ essential}\ \rightarrow
    &=&\bar{\sigma}\circ[y/x]\circ{\cal M}(\phi_{1})\\
    \mbox{B: to be proved}\ \rightarrow
    &=&[\sigma(y)/\sigma(x)]\circ\bar{\sigma}\circ{\cal M}(\phi_{1})\\
    \mbox{C: to be proved}\ \rightarrow
    &=&[\sigma(y)/\sigma(x)]\circ{\cal M}\circ\sigma(\phi_{1})\\
    &=&[\sigma(y)/z]\circ{\cal M}\circ\sigma(\phi_{1})\\
    \mbox{$[\sigma(y)/z]$ essential}\ \rightarrow
    &=&{\cal M}\circ[\sigma(y)/z]\circ\sigma(\phi_{1})\\
    &=&{\cal M}(\,\psi_{1}[\sigma(y)/z]\,)\\
    \end{eqnarray*}
So it remains to prove points A, B and C. First we start with point
A: using
theorem~(\ref{logic:the:FOPL:commute:mintransform:validsub}) of
page~\pageref{logic:the:FOPL:commute:mintransform:validsub} we
simply need to show that $\sigma$ is valid for $[y/x](\phi_{1})$,
which follows from our assumption that $\sigma$ is valid for
$\phi=\forall x\phi_{1}\to[y/x](\phi_{1})$ and
proposition~(\ref{logic:prop:FOPL:valid:recursion:imp}). So we now
prove point B: let $u\in\var({\cal M}(\phi_{1}))$. Using
proposition~(\ref{logic:prop:substitution:support}) we need to show
that
$\bar{\sigma}\circ[y/x](u)=[\sigma(y)/\sigma(x)]\circ\bar{\sigma}(u)$.
Since $\bar{V}$ is the disjoint union of $V$ and \N, we shall
distinguish two cases: first we assume that $u\in\N$. Then $u\neq x$
and $\bar{\sigma}(u)=u$. It follows that
$\bar{\sigma}(u)\neq\sigma(x)$ and the equality is true. Next we
assume that $u\in V$. Using
proposition~(\ref{logic:prop:FOPL:mintransform:variables}) it
follows that $u\in\var({\cal M}(\phi_{1}))\cap V=\free(\phi_{1})$.
We shall distinguish two further cases: first we assume that $u=x$.
Then $\bar{\sigma}(u)=\sigma(x)$ and the equality is true. Finally
we assume that $u\neq x$. Then
$u\in\free(\phi_{1})\setminus\{x\}=\free(\forall x\phi_{1})$. Having
assumed that $\sigma$ is valid for $\phi=\forall
x\phi_{1}\to[y/x](\phi_{1})$, in particular from
proposition~(\ref{logic:prop:FOPL:valid:recursion:imp}) it is valid
for $\forall x\phi_{1}$. Hence from $u\in\free(\forall x\phi_{1})$
and proposition~(\ref{logic:prop:FOPL:valid:recursion:quant}) we
obtain $\bar{\sigma}(u)=\sigma(u)\neq\sigma(x)$. So we conclude the
equality is once again true, which completes our proof of point B.
So we now prove point C: using
theorem~(\ref{logic:the:FOPL:commute:mintransform:validsub}) of
page~\pageref{logic:the:FOPL:commute:mintransform:validsub} once
more, we need to show that $\sigma$ is valid for $\phi_{1}$.
However, we already know that $\sigma$ is valid for $\forall
x\phi_{1}$, and the validity of $\sigma$ for $\phi_{1}$ follows from
proposition~(\ref{logic:prop:FOPL:valid:recursion:quant}). This
completes our proof of point C.
\end{proof}
