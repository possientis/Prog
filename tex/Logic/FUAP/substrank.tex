We are now getting closer to our objective of defining essential
proof substitutions $\sigma:\pvs\to{\bf\Pi}(W)$ associated with a
map $\sigma:V\to W$. Our strategy so far has been to follow the
trail of what was done to construct essential substitutions
$\sigma:\pv\to{\bf P}(W)$ for formulas. There is nothing glamorous
in doing so, but we do enjoy a reasonable chance of success by
exploiting safe routes, with minimal time and effort. Looking back
at the work done for formulas, one natural question comes to mind:
why do we need to bother about {\em substitution rank}\,? Our final
result in theorem~(\ref{logic:the:FOPL:esssubst:existence}) of
page~\pageref{logic:the:FOPL:esssubst:existence} was to prove the
existence of essential substitutions $\sigma:\pv\to{\bf P}(W)$
associated with $\sigma:V\to W$, provided $|W|$ is an infinite
cardinal or $|V|\leq|W|$. This result does not involve {\em
substitution rank} and it is very likely that it could be proved
without any reference to it. So why do we care so much about {\em
substitution rank}\,? The reason is this: as a general comment, it
is no good to know that something is true. What is important is to
understand why this is the case. We feel so much better about it.
When it comes to the existence of essential substitutions
$\sigma:\pv\to{\bf P}(W)$, the notion of {\em substitution rank}
brings us some valuable light. It allows us to understand why the
restrictions on the cardinals exist. So let us see why this is:
given a map $\sigma:V\to W$ and a formula $\phi\in\pv$, the formula
$\sigma(\phi)$ is essentially $\bar{\sigma}\circ{\cal M}(\phi)$. The
problem is $\bar{\sigma}\circ{\cal M}(\phi)$ is an element of ${\bf
P}(\bar{W})$. So it needs to be interpreted as an element of ${\bf
P}(W)$. This cannot be done unless the set ${\bf P}(W)$ is {\em
large enough} for the formula $\bar{\sigma}\circ{\cal M}(\phi)$ to
be {\em squeezed} into it. In other words, the set $W$ needs to have
sufficiently many variables. This naturally leads to the condition
$\rnk(\bar{\sigma}\circ{\cal M}(\phi))\leq|W|$ which is where the
{\em substitution rank} comes in. If we want to define an essential
substitution as a total map $\sigma:\pv\to{\bf P}(W)$, we need this
condition to hold everywhere, which is clearly the case when $|W|$
is an infinite cardinal. Now we have the following inequalities:
    \[
    \rnk(\bar{\sigma}\circ{\cal M}(\phi))\leq\rnk({\cal
    M}(\phi))=\rnk(\phi)\leq|\var(\phi)|\leq|V|
    \]
It follows that a sufficient condition is $|V|\leq|W|$. This
condition is in fact necessary when $W$ is a finite set as it is not
difficult to design a formula $\phi$ for which
$\rnk(\bar{\sigma}\circ{\cal M}(\phi))=n$ for any $n\leq |V|$. So
here we are: the notion of {\em substitution rank} gives us the
insight we need to understand why the condition $(|W|\mbox{\
infinite})\lor(|V|\leq|W|)$ is equivalent to the existence of
essential substitutions $\sigma:\pv\to{\bf P}(W)$ associated to
$\sigma:V\to W$. For those who are surprised by the fact that this
condition makes no allowance for the specifics of the map
$\sigma:V\to W$, one should remember that an {\em essential
substitution of variables} has no impact on closed formulas, modulo
$\alpha$-equivalence. So if a closed formula needs many variables to
be expressed in \pv, it needs an equal number of variables to exist
in ${\bf P}(W)$. We are now ready to move on and quote:

\index{rank@Substitution rank of proof}\index{r@$\rnk(\pi)$ :
substitution rank of $\pi$}
\begin{defin}\label{logic:def:FUAP:substrank:main}
Let $V$ be a set and $\sim$ be the alpha equivalence on
$\pvs$. For all $\pi\in\pvs$ we call {\em substitution rank} of
$\pi$ the integer $\rnk(\pi)$ defined by:
    \[
    \rnk(\pi)=\min\{\,|\var(\rho)|\ :\
    \rho\in\pvs\ ,\ \pi\sim\rho\,\}
    \]
where $|\var(\rho)|$ denotes the cardinal of the set $\var(\rho)$,
for all $\rho\in\pvs$.
\end{defin}

Given a formula $\phi\in\pv$ the notation $\rnk(\phi)$ is
potentially ambiguous. Since $\pv\subseteq\pvs$, it may refer to the
usual $\rnk(\phi)$ of
definition~(\ref{logic:def:FOPL:substrank:substrank}), or to the
newly defined integer $\rnk(\pi)$ where $\pi=\phi$ of
definition~(\ref{logic:def:FUAP:substrank:main}). Luckily, the two
notions coincide as the following proposition shows:


\begin{prop}\label{logic:prop:FUAP:substrank:recursion:formula}
Let $V$ be a set and $\pi=\phi$ for some $\phi\in\pv$. Then:
    \[
    \rnk(\pi)=\rnk(\phi)
    \]
where $\rnk(\phi)$ is the substitution rank of $\phi$ as per {\em
definition~(\ref{logic:def:FOPL:substrank:substrank})}.
\end{prop}
\begin{proof}
First we show $\rnk(\pi)\leq\rnk(\phi)$\,: so let $\psi\in\pv$ such
that $\phi\sim\psi$, where $\sim$ is also the alpha equivalence on \pv. 
We need to show $\rnk(\pi)\leq|\var(\psi)|$.
However take $\rho=\psi$. Then $\rho$ is a proof such that
$\pi\sim\rho$. It follows from
definition~(\ref{logic:def:FUAP:substrank:main}) that
$\rnk(\pi)\leq|\var(\rho)|$. So $\rnk(\pi)\leq|\var(\psi)|$ as
requested. We now show $\rnk(\phi)\leq\rnk(\pi)$\,: so let
$\rho\in\pvs$ such that $\pi\sim\rho$. We need to show that
$\rnk(\phi)\leq|\var(\rho)|$. However from $\pi\sim\rho$ and
theorem~(\ref{logic:the:FUAP:charsubcong:charac}) of
page~\pageref{logic:the:FUAP:charsubcong:charac}, we see that $\rho$
must be of the form $\rho=\psi$ for some $\psi\in\pv$ with
$\phi\sim\psi$. From
definition~(\ref{logic:def:FOPL:substrank:substrank}) we obtain
$\rnk(\phi)\leq|\var(\psi)|$. It follows that
$\rnk(\phi)\leq|\var(\rho)|$.
\end{proof}


\begin{prop}\label{logic:prop:FUAP:substrank:recursion:axiom}
Let $V$ be a set and $\pi=\axi\phi$ for some $\phi\in\pv$. Then:
    \[
    \rnk(\pi)=\rnk(\phi)
    \]
where $\rnk(\phi)$ is the substitution rank of $\phi$ as per {\em
definition~(\ref{logic:def:FOPL:substrank:substrank})}.
\end{prop}
\begin{proof}
First we show $\rnk(\pi)\leq\rnk(\phi)$\,: so let $\psi\in\pv$ such
that $\phi\sim\psi$, where $\sim$ is also the alpha equivalence on \pv. 
We need to show $\rnk(\pi)\leq|\var(\psi)|$.
However take $\rho=\axi\psi$. Then $\rho$ is a proof such that
$\pi\sim\rho$. It follows from
definition~(\ref{logic:def:FUAP:substrank:main}) that
$\rnk(\pi)\leq|\var(\rho)|$. So $\rnk(\pi)\leq|\var(\psi)|$ as
requested. We now show $\rnk(\phi)\leq\rnk(\pi)$\,: so let
$\rho\in\pvs$ such that $\pi\sim\rho$. We need to show that
$\rnk(\phi)\leq|\var(\rho)|$. However from $\pi\sim\rho$ and
theorem~(\ref{logic:the:FUAP:charsubcong:charac}) of
page~\pageref{logic:the:FUAP:charsubcong:charac}, we see that $\rho$
must be of the form $\rho=\axi\psi$ for some $\psi\in\pv$ with
$\phi\sim\psi$. From
definition~(\ref{logic:def:FOPL:substrank:substrank}) we obtain
$\rnk(\phi)\leq|\var(\psi)|$. It follows that
$\rnk(\phi)\leq|\var(\rho)|$.
\end{proof}

The following proposition is the counterpart of
proposition~(\ref{logic:prop:FOPL:substrank:basic:ineq})\,:

\begin{prop}\label{logic:prop:FUAP:substrank:basic:ineq}
Let $V$ be a set and $\pi\in\pvs$. Then, we have:
    \[
    |\free(\pi)|\leq\rnk(\pi)\leq|\var(\pi)|
    \]
\end{prop}
\begin{proof}
The inequality $\rnk(\pi)\leq|\var(\pi)|$ follows immediately from
definition~(\ref{logic:def:FUAP:substrank:main}). So it remains to
show that $|\free(\pi)|\leq\rnk(\pi)$. So let $\rho\in\pvs$ such
that $\pi\sim\rho$ where $\sim$ is the alpha equivalence. We
need to show $|\free(\pi)|\leq|\var(\rho)|$. From
proposition~(\ref{logic:prop:FUAP:charsubcong:freevar}) we have
$\free(\pi)=\free(\rho)$. Hence we need to show that
$|\free(\rho)|\leq|\var(\rho)|$ which follows from
$\free(\rho)\subseteq\var(\rho)$.
\end{proof}

The following proposition is the counterpart of
proposition~(\ref{logic:prop:FOPL:substrank:changeofvar}). It is the
most important result of this section, as everything else depends on
it. When attempting to prove anything about $\rnk(\pi)$, it is very
important to find some representative $\rho\sim\pi$ such that
$\rnk(\pi)=|\var(\rho)|$. We know that such representative exists
from definition~(\ref{logic:def:FUAP:substrank:main}). However, we
usually want to find $\rho$ while controlling the set $\var(\rho)$.
Since $\rho\sim\pi$ we have $\free(\rho)=\free(\pi)$ so we cannot
control $\free(\rho)$. However, if $\free(\pi)\subseteq V_{0}$ for
some $V_{0}\subseteq V$, we can always find $\rho$ such that
$\var(\rho)\subseteq V_{0}$ provided $V_{0}$ is a large enough set.
If $V_{0}$ is too small to accommodate every variable of $\rho$,
then we can find $\rho$ such that $V_{0}\subseteq\var(\rho)$.

\begin{prop}\label{logic:prop:FUAP:substrank:changeofvar}
Let $V$ be a set and $\pi\in\pvs$ such that $\free(\pi)\subseteq
V_{0}$ for some $V_{0}\subseteq V$.  Let $\sim$ denote the
alpha equivalence on \pvs. Then there exists $\rho\in\pvs$
such that $\pi\sim\rho$ and $|\var(\rho)|=\rnk(\pi)$ such that:
    \begin{eqnarray*}
    (i)&&\rnk(\pi)\leq|V_{0}|\ \Rightarrow\ \var(\rho)\subseteq
    V_{0}\\
    (ii)&&|V_{0}|\leq\rnk(\pi)\ \Rightarrow\
    V_{0}\subseteq\var(\rho)
    \end{eqnarray*}
\end{prop}
\begin{proof}
Without loss of generality, we may assume that
$|\var(\pi)|=\rnk(\pi)$. Indeed, suppose the proposition has been
established with the additional assumption $|\var(\pi)|=\rnk(\pi)$.
We need to show it is then true in the general case. So given
$V_{0}\subseteq V$, consider $\pi\in\pvs$ such that
$\free(\pi)\subseteq V_{0}$. From
definition~(\ref{logic:def:FUAP:substrank:main}) there exists
$\pi_{1}\in\pvs$ such that $\pi\sim\pi_{1}$ with the equality
$|\var(\pi_{1})|=\rnk(\pi)$. From
proposition~(\ref{logic:prop:FUAP:charsubcong:freevar}) we obtain
$\free(\pi)=\free(\pi_{1})$ and so $\free(\pi_{1})\subseteq V_{0}$.
Hence we see that $\pi_{1}$ satisfies the assumption of the
proposition with the additional property
$|\var(\pi_{1})|=\rnk(\pi_{1})$. Having assumed the proposition is
true in this case, we obtain the existence of $\rho\in\pvs$ such
that $\pi_{1}\sim\rho$, $|\var(\rho)|=\rnk(\pi_{1})$ and which
satisfies $(i)$ and $(ii)$ where $\pi$ is replaced by $\pi_{1}$.
However $\rnk(\pi_{1})=\rnk(\pi)$ and replacing $\pi$ by $\pi_{1}$
in $(i)$ and $(ii)$ has no impact. So $\rho$ satisfies $(i)$ and
$(ii)$. Hence we have $\pi\sim\rho$ and $|\var(\rho)|=\rnk(\pi)$
together with $(i)$ and $(ii)$ which establishes the proposition in
the general case. So we now assume without loss of generality that
$|\var(\pi)|=\rnk(\pi)$ and $\free(\pi)\subseteq V_{0}$. We need to
show the existence of $\rho\sim\pi$ such that
$|\var(\rho)|=|\var(\pi)|$ and which satisfies $(i)$ and $(ii)$.
Note that if $V=\emptyset$  then $V_{0}=\emptyset$ and we can take
$\rho=\pi$. So we assume $V\neq\emptyset$. We shall first consider
the case when $\rnk(\pi)\leq |V_{0}|$ and show the existence of
$\rho$ such that $\var(\rho)\subseteq V_{0}$. Since
$\free(\pi)\subseteq V_{0}$ the set $V_{0}$ is the disjoint union of
$\free(\pi)$ and $V_{0}\setminus\free(\pi)$, giving us the equality
$|V_{0}| =|\free(\pi)|+|V_{0}\setminus\free(\pi)|$. Since we also
have $|\var(\pi)|=|\free(\pi)|+|\var(\pi)\setminus\free(\pi)|$, we
obtain from $|\var(\pi)|\leq|V_{0}|$:
    \[
    |\free(\pi)|+|\var(\pi)\setminus\free(\pi)|\leq|\free(\pi)|
    +|V_{0}\setminus\free(\pi)|
    \]
Since $|\free(\pi)|$ is a finite cardinal it follows that
$|\var(\pi)\setminus\free(\pi)|\leq|V_{0}\setminus\free(\pi)|$.
Hence, there is an injection mapping
$i:\var(\pi)\setminus\free(\pi)\to V_{0}\setminus\free(\pi)$. Having
assumed $V\neq\emptyset$ consider $x^{*}\in V$ and define the map
$\sigma:V\to V$ as follows:
    \[
    \forall u\in V\ ,\ \sigma(u)=\left\{
        \begin{array}{lcl}
        u&\mbox{\ if\ }&u\in\free(\pi)\\
        i(u)&\mbox{\ if\ }&u\in\var(\pi)\setminus\free(\pi)\\
        x^{*}&\mbox{\ if\ }&u\not\in\var(\pi)
        \end{array}
    \right.
    \]
Define $\rho=\sigma(\pi)$. It remains to show that $\rho$ has the
desired properties, namely that $\rho\sim\pi$,
$|\var(\rho)|=|\var(\pi)|$ and $\var(\rho)\subseteq V_{0}$. First we
show $\rho\sim\pi$. Using
proposition~(\ref{logic:prop:FUAP:subcong:admissible:subcong}) it is
sufficient to prove that $\sigma$ is an admissible substitution for
$\pi$. It is clear that $\sigma(u)=u$ for all $u\in\free(\pi)$. So
it remains to show that $\sigma$ is valid for $\pi$. From
proposition~(\ref{logic:prop:FUAP:validsubproof:injective}) it is
sufficient to prove that $\sigma_{|\var(\pi)}$ is an injective map.
So let $u,v\in\var(\pi)$ such that $\sigma(u)=\sigma(v)$. We need to
show that $u=v$. We shall distinguish four cases: first we assume
that $u\in\free(\pi)$ and $v\in\free(\pi)$. Then the equality
$\sigma(u)=\sigma(v)$ leads to $u=v$. Next we assume that
$u\not\in\free(\pi)$ and $v\not\in\free(\pi)$. Then we obtain
$i(u)=i(v)$ which also leads to $u=v$ since
$i:\var(\pi)\setminus\free(\pi)\to V_{0}\setminus\free(\pi)$ is an
injective map. So we now assume that $u\in\free(\pi)$ and
$v\not\in\free(\pi)$. Then from $\sigma(u)=\sigma(v)$ we obtain
$u=i(v)$ which is in fact impossible since $u\in\free(\pi)$ and
$i(v)\in V_{0}\setminus\free(\pi)$. The last case
$u\not\in\free(\pi)$ and $v\in\free(\pi)$ is equally impossible
which completes our proof of $\rho\sim\pi$. So we now prove that
$|\var(\rho)|=|\var(\pi)|$. From
proposition~(\ref{logic:prop:FUAP:variable:substitution}) we have
$\var(\rho)=\var(\sigma(\pi))=\sigma(\var(\pi))$. So we need
$|\sigma(\var(\pi))|=|\var(\pi)|$ which is clear since
$\sigma_{|\var(\pi)}:\var(\pi)\to\sigma(\var(\pi))$ is a bijection.
So it remains to show that $\var(\rho)\subseteq V_{0}$, or
equivalently that $\sigma(\var(\pi))\subseteq V_{0}$. So let
$u\in\var(\pi)$. We need to show that $\sigma(u)\in V_{0}$. We shall
distinguish two cases: first we assume that $u\in\free(\pi)$. Then
$\sigma(u)=u$ and the property $\sigma(u)\in V_{0}$ follows from the
inclusion $\free(\pi)\subseteq V_{0}$. Next we assume that
$u\not\in\free(\pi)$. Then we have $\sigma(u)=i(u)\in
V_{0}\setminus\free(\pi)\subseteq V_{0}$. So in the case when
$\rnk(\pi)\leq|V_{0}|$ we have been able to prove the existence of
$\rho$ satisfying $(i)$. In fact we claim that $\rho$ also satisfies
$(ii)$. So let us assume that $|V_{0}|\leq\rnk(\pi)$. Then we must
have $\rnk(\pi)=|V_{0}|$ and we need to show that
$V_{0}\subseteq\var(\rho)$. However, we have
$|\var(\rho)|=\rnk(\pi)$ and consequently $|\var(\rho)|=|V_{0}|$
together with $\var(\rho)\subseteq V_{0}$. Two finite subsets
ordered by inclusion and with the same cardinal must be equal. So
$\var(\rho)=V_{0}$. We now consider the case when
$|V_{0}|<\rnk(\pi)$. We need to show the existence of $\rho\sim\pi$
such that $|\var(\rho)|=|\var(\pi)|$ satisfying $(i)$ and $(ii)$. In
the case when $|V_{0}|<\rnk(\pi)$, $(i)$ is vacuously true, so we
simply need to ensure that $V_{0}\subseteq\var(\rho)$. Since
$|V_{0}|<|\var(\pi)|$ we obtain:
    \begin{eqnarray*}
    |V_{0}\setminus\var(\pi)|&=&|V_{0}|-|V_{0}\cap\var(\pi)|\\
    &<&|\var(\pi)|-|V_{0}\cap\var(\pi)|\\
    &=&|\var(\pi)\setminus V_{0}|
    \end{eqnarray*}
So there is an injective map
$i:V_{0}\setminus\var(\pi)\to\var(\pi)\setminus V_{0}$. Given
$x^{*}\in V$, define:
    \[
    \forall u\in V\ ,\ \sigma(u)=\left\{
        \begin{array}{lcl}
        u&\mbox{\ if\ }&u\in\var(\pi)\setminus i(\,V_{0}\setminus\var(\pi)\,)\\
        i^{-1}(u)&\mbox{\ if\ }&u\in i(\,V_{0}\setminus\var(\pi)\,)\\
        x^{*}&\mbox{\ if\ }&u\not\in\var(\pi)
        \end{array}
    \right.
    \]
Let $\rho=\sigma(\pi)$. It remains to show that $\rho\sim\pi$,
$|\var(\rho)|=|\var(\pi)|$ and furthermore
$V_{0}\subseteq\var(\rho)$. First we show that $\rho\sim\pi$. Using
proposition~(\ref{logic:prop:FUAP:subcong:admissible:subcong}) it is
sufficient to prove that $\sigma$ is an admissible substitution for
$\pi$. So let $u\in\free(\pi)$. We need to show that $\sigma(u)=u$.
So it is sufficient to prove that $u\not\in
i(\,V_{0}\setminus\var(\pi)\,)$ which follows from the fact that
$u\in V_{0}$, itself a consequence of $\free(\pi)\subseteq V_{0}$.
In order to show that $\sigma$ is also valid for $\pi$, from
proposition~(\ref{logic:prop:FUAP:validsubproof:injective}) it is
sufficient to prove that $\sigma$ is injective on $\var(\pi)$. So
let $u,v\in\var(\pi)$ such that $\sigma(u)=\sigma(v)$. We need to
prove that $u=v$. The only case when this may not be clear is when
$\sigma(u)=u$ and $\sigma(v)=i^{-1}(v)$ or vice versa. So we assume
that $u\in\var(\pi)\setminus i(\,V_{0}\setminus\var(\pi)\,)$ and $
v\in i(\,V_{0}\setminus\var(\pi)\,)$. Then we see that
$\sigma(u)=u\in\var(\pi)$ while $\sigma(v)=i^{-1}(v)\in
V_{0}\setminus\var(\pi)$. So the equality $\sigma(u)=\sigma(v)$ is
in fact impossible, which completes our proof of $\rho\sim\pi$. As
before, the fact that $|\var(\rho)|=|\var(\pi)|$ follows from the
injectivity of $\sigma_{|\var(\pi)}$ and it remains to prove that
$V_{0}\subseteq\var(\rho)$. So let $u\in V_{0}$ we need to show that
$u\in\var(\rho)=\sigma(\var(\pi))$ and we shall distinguish two
cases: first we assume that $u\in\var(\pi)$. Since $u\in V_{0}$, it
cannot be an element of $i(\,V_{0}\setminus\var(\pi)\,)$. It follows
that $u\in\var(\pi)\setminus i(\,V_{0}\setminus\var(\pi)\,)$ and
thus $u=\sigma(u)\in\sigma(\var(\pi))=\var(\rho)$. Next we assume
that $u\in V_{0}\setminus\var(\pi)$. Then $i(u)$ is an element of
$i(\,V_{0}\setminus\var(\pi)\,)$ and therefore
$\sigma(i(u))=i^{-1}(i(u))=u$. Since $i(u)$ is an element of
$\var(\pi)\setminus V_{0}$ we conclude that
$u=\sigma(i(u))\in\sigma(\var(\pi))=\var(\rho)$, which completes our
proof.
\end{proof}

Just like it is for formulas, the substitution rank is invariant by
injective substitution. The following proposition is the counterpart
of proposition~(\ref{logic:prop:FOPL:substrank:injective})\,:

\begin{prop}\label{logic:prop:FUAP:substrank:injective}
Let $V,W$ be sets and $\sigma:V\to W$ be a map. Then for all
$\pi\in\pvs$, if $\sigma_{|\var(\pi)}$ is an injective map, we have
the equality:
    \[
    \rnk(\sigma(\pi))=\rnk(\pi)
    \]
where $\sigma:\pvs\to{\bf\Pi}(W)$ denotes the associated
substitution mapping.
\end{prop}
\begin{proof}
Let $\sigma:V\to W$ and $\pi\in\pvs$ such that $\sigma_{|\var(\pi)}$
is an injective map. We need to show that
$\rnk(\sigma(\pi))=\rnk(\pi)$. First we shall show
$\rnk(\sigma(\pi))\leq\rnk(\pi)$. Using
proposition~(\ref{logic:prop:FUAP:substrank:changeofvar}) with
$V_{0}=\var(\pi)$, since we have $\free(\pi)\subseteq\var(\pi)$ and
$\rnk(\pi)\leq|\var(\pi)|$, there exists $\rho\in\pvs$ such that
$\pi\sim\rho$, $|\var(\rho)|=\rnk(\pi)$ and
$\var(\rho)\subseteq\var(\pi)$. Having assumed $\sigma$ is injective
on $\var(\pi)$, it is therefore injective on both $\var(\pi)$ and
$\var(\rho)$. From
proposition~(\ref{logic:prop:FUAP:validsubproof:injective}) it
follows that $\sigma$ is a valid substitution for both $\pi$ and
$\rho$. Hence from
theorem~(\ref{logic:the:FUAP:mintransfsubcong:valid}) of
page~\pageref{logic:the:FUAP:mintransfsubcong:valid} we obtain
$\sigma(\pi)\sim\sigma(\rho)$ and consequently using
proposition~(\ref{logic:prop:FUAP:variable:substitution})\,:
    \[
    \rnk(\sigma(\pi))\leq|\var(\sigma(\rho))|=|\sigma(\var(\rho))|
    =|\var(\rho)|=\rnk(\pi)
    \]
So it remains to show that $\rnk(\pi)\leq\rnk(\sigma(\pi))$. If
$V=\emptyset$ then $\rnk(\pi)=0$ and we are done. So we may assume
that $V\neq\emptyset$. So let $x^{*}\in V$ and define:
    \[
    \forall u\in W\ ,\ \tau(u)=\left\{
        \begin{array}{lcl}
        \sigma^{-1}(u)&\mbox{\ if\ }&u\in\sigma(\var(\pi))\\
        x^{*}&\mbox{\ if\ }&u\not\in\sigma(\var(\pi))
        \end{array}
    \right.
    \]
Then $\tau:W\to V$ is injective on
$\var(\sigma(\pi))=\sigma(\var(\pi))$ and hence:
    \[
    \rnk(\,\tau(\sigma(\pi))\,)\leq\rnk(\sigma(\pi))
    \]
So it suffices for us to show that  $\tau\circ\sigma(\pi)=\pi$. From
proposition~(\ref{logic:prop:FUAP:variable:support}) it is thus
sufficient to show that $\tau\circ\sigma(x)=x$ for all
$x\in\var(\pi)$, which is clear.
\end{proof}

The following proposition is the counterpart of
proposition~(\ref{logic:prop:FOPL:substrank:minrank})\,:

\begin{prop}\label{logic:prop:FUAP:substrank:minrank}
Let $V$ be a set and $\pi\in\pvs$. Then the proof $\pi$ and its
minimal transform have equal substitution rank, i.e.:
    \[
    \rnk({\cal M}(\pi))=\rnk(\pi)
    \]
\end{prop}
\begin{proof}
Let $\sim$ denote the alpha equivalence on ${\bf\Pi}(V)$ and
$i:V\to\bar{V}$ be the inclusion map. From
proposition~(\ref{logic:prop:FUAP:mintransfsubcong:equivalence}) we
have ${\cal M}(\pi)\sim\, i(\pi)$ and consequently $\rnk({\cal
M}(\pi))=\rnk(i(\pi))$. So we need to show that
$\rnk(i(\pi))=\rnk(\pi)$ which follows from
proposition~(\ref{logic:prop:FUAP:substrank:injective}) and the fact
that $i:V\to\bar{V}$ is injective.
\end{proof}

The following proposition is the counterpart of
proposition~(\ref{logic:prop:FOPL:substrank:invariant:rank})\,:

\begin{prop}\label{logic:prop:FUAP:substrank:invariant:rank}
Let $V,W$ be sets and $\sigma:V\to W$ be a map. Let $\pi\in\pvs$. We
assume that $\sigma$ is valid for $\pi$ and $\sigma_{|\free(\pi)}$
is an injective map. Then:
    \[
    \rnk(\sigma(\pi))=\rnk(\pi)
    \]
where $\sigma:\pvs\to{\bf\Pi}(W)$ denotes the associated
substitution mapping.
\end{prop}
\begin{proof}
We shall proceed in two steps: first we shall prove the proposition
is true in the case when $W=V$. We shall then extend the result to
arbitrary $W$. So let us assume $W=V$. Let $\sigma:V\to V$ valid for
$\pi\in\pvs$ such that $\sigma_{|\free(\pi)}$ is an injective map.
We need to show that $\rnk(\sigma(\pi))=\rnk(\pi)$. If $V=\emptyset$
then $\sigma:V\to V$ is the map with empty domain, namely the empty
set which is injective on $\var(\pi)=\emptyset$ and
$\rnk(\sigma(\pi))=\rnk(\pi)$ follows from
proposition~(\ref{logic:prop:FUAP:substrank:injective}). So we
assume that $V\neq\emptyset$. The idea of the proof is to write
$\sigma(\pi)=\tau_{1}\circ\tau_{0}(\pi)$ where each substitution
$\tau_{0}, \tau_{1}$ is rank preserving. Having assumed $\sigma$
injective on $\free(\pi)$, we have the equality
$|\sigma(\free(\pi)|=|\free(\pi)|$ and consequently:
    \begin{eqnarray*}
    |\var(\pi)\setminus\free(\pi)|&=&|\var(\pi)|-|\free(\pi)|\\
    &\leq&|V|-|\free(\pi)|\\
    &=&|V|-|\,\sigma(\free(\pi))\,|\\
    &=&|\,V\setminus\sigma(\free(\pi))\,|
    \end{eqnarray*}
So let $i:\var(\pi)\setminus\free(\pi)\to
V\setminus\sigma(\free(\pi))$ be an injective map. Let $x^{*}\in V$
and define the substitution $\tau_{0}:V\to V$ as follows:
    \[
    \forall x\in V\ ,\ \tau_{0}(x)=\left\{
        \begin{array}{lcl}
        \sigma(x)&\mbox{\ if\ }&x\in\free(\pi)\\
        i(x)&\mbox{\ if\ }&x\in\var(\pi)\setminus\free(\pi)\\
        x^{*}&\mbox{\ if\ }&x\not\in\var(\pi)
        \end{array}
    \right.
    \]
Let us accept for now that $\tau_{0}$ is injective on $\var(\pi)$.
Then using proposition~(\ref{logic:prop:FUAP:substrank:injective})
we obtain $\rnk(\tau_{0}(\pi))=\rnk(\pi)$. So consider
$\tau_{1}:V\to V$\,:
    \[
    \forall u\in V\ ,\ \tau_{1}(u)=\left\{
        \begin{array}{lcl}
        u&\mbox{\ if\ }&u\in\sigma(\free(\pi))\\
        \sigma\circ i^{-1}(u)&\mbox{\ if\ }&u\in i(\,\var(\pi)\setminus\free(\pi)\,)\\
        x^{*}&\mbox{\ \ \ \ }&\mbox{otherwise}
        \end{array}
    \right.
    \]
Note that $\tau_{1}$ is well defined since $\sigma(\free(\pi))\cap
i(\,\var(\pi)\setminus\free(\pi)\,)=\emptyset$. So let us accept for
now that $\tau_{1}$ is admissible for $\tau_{0}(\pi)$. Then using
proposition~(\ref{logic:prop:FUAP:subcong:admissible:subcong}) we
obtain $\tau_{1}\circ\tau_{0}(\pi)\sim\tau_{0}(\pi)$ where $\sim$ is
the alpha equivalence on \pvs. Hence in particular we have
$\rnk(\tau_{1}\circ\tau_{0}(\pi))=\rnk(\tau_{0}(\pi))=\rnk(\pi)$. So
in order to show that the proposition is true in the case when
$W=V$, it remains to prove that $\tau_{0}$ is injective on
$\var(\pi)$, $\tau_{1}$ is admissible for $\tau_{0}(\pi)$ and
furthermore that $\tau_{1}\circ\tau_{0}(\pi)=\sigma(\pi)$. First we
show that $\tau_{0}$ is injective on $\var(\pi)$. So let
$x,y\in\var(\pi)$ such that $\tau_{0}(x)=\tau_{0}(y)$. We need to
show that $x=y$. We shall distinguish four cases: first we assume
that $x\in\free(\pi)$ and $y\in\free(\pi)$. Then the equality
$\tau_{0}(x)=\tau_{0}(y)$ leads to $\sigma(x)=\sigma(y)$. Having
assumed $\sigma_{|\free(\pi)}$ is an injective map, we obtain $x=y$.
Next we assume that $x\in\var(\pi)\setminus\free(\pi)$ and
$y\in\var(\pi)\setminus\free(\pi)$. Then the equality
$\tau_{0}(x)=\tau_{0}(y)$ leads to $i(x)=i(y)$ and consequently
$x=y$. So we now assume that $x\in\free(\pi)$ and
$y\in\var(\pi)\setminus\free(\pi)$. Then
$\tau_{0}(x)=\sigma(x)\in\sigma(\free(\pi))$ and
$\tau_{0}(y)=i(y)\in V\setminus\sigma(\free(\pi))$. So the equality
$\tau_{0}(x)=\tau_{0}(y)$ is in fact impossible. We show similarly
that the final case $x\in\var(\pi)\setminus\free(\pi)$ and
$y\in\free(\pi)$ is also impossible which completes the proof that
$\tau_{0}$ is injective on $\var(\pi)$. We shall now show that
$\tau_{1}\circ\tau_{0}(\pi)=\sigma(\pi)$. From
proposition~(\ref{logic:prop:FUAP:variable:support}) it is
sufficient to prove that $\tau_{1}\circ\tau_{0}(x)=\sigma(x)$ for
all $x\in\var(\pi)$. We shall distinguish two cases: first we assume
that $x\in\free(\pi)$. Then
$\tau_{0}(x)=\sigma(x)\in\sigma(\free(\pi))$ and consequently
$\tau_{1}\circ\tau_{0}(x)=\sigma(x)$ as requested. Next we assume
that $x\in\var(\pi)\setminus\free(\pi)$. Then $\tau_{0}(x)=i(x)\in
i(\,\var(\pi)\setminus\free(\pi)\,)$ and consequently
$\tau_{1}\circ\tau_{0}(x)=\sigma\circ i^{-1}(i(x))=\sigma(x)$. So it
remains to show that $\tau_{1}$ is admissible for $\tau_{0}(\pi)$,
i.e. that it is valid for $\tau_{0}(\pi)$ and $\tau_{1}(u)=u$ for
all $u\in\free(\tau_{0}(\pi))$. So let $u\in\free(\tau_{0}(\pi))$.
We need to show that $\tau_{1}(u)=u$. So it is sufficient to prove
that $u\in\sigma(\free(\pi))$. However from
proposition~(\ref{logic:prop:FUAP:freevarproof:substitution:inclusion})
we have $\free(\tau_{0}(\pi))\subseteq\tau_{0}(\free(\pi))$ and
consequently there exists $x\in\free(\pi)$ such that
$u=\tau_{0}(x)=\sigma(x)$. It follows that $u\in\sigma(\free(\pi))$
as requested and it remains to show that $\tau_{1}$ is valid for
$\tau_{0}(\pi)$. From
proposition~(\ref{logic:prop:FUAP:validsubproof:composition}) it is
sufficient to show that $\tau_{1}\circ\tau_{0}$ is valid for $\pi$.
However, having proved that $\tau_{1}\circ\tau_{0}(\pi)=\sigma(\pi)$
from proposition~(\ref{logic:prop:FUAP:validsubproof:equal:image})
it is sufficient to prove that $\sigma$ is valid for $\pi$ which is
in fact true by assumption. This completes our proof of the
proposition in the case when $W=V$. We shall now prove the
proposition in the general case. So we assume that $\sigma:V\to W$
is a map and $\pi\in\pvs$ is such that $\sigma$ is valid for $\pi$
and $\sigma_{|\free(\pi)}$ is an injective map. We need to prove
that $\rnk(\sigma(\pi))=\rnk(\pi)$. Let $U$ be the disjoint union of
the sets $V$ and $W$, specifically:
    \[
    U=\{0\}\times V\uplus\{1\}\times W
    \]
Let $i:V\to U$ and $j:W\to U$ be the corresponding inclusion maps.
Consider the proof $\pi^{*}=i(\pi)\in{\bf\Pi}(U)$ and let
$\sigma^{*}:U\to U$ be defined as:
    \[
    \forall u\in U\ ,\ \sigma^{*}(u)=\left\{
        \begin{array}{lcl}
        j\circ\sigma(u)&\mbox{\ if\ }&u\in V\\
        u&\mbox{\ if\ }&u\in W\\
        \end{array}
    \right.
    \]
Let us accept for now that $\sigma^{*}$ is valid for $\pi^{*}$ and
that $\sigma^{*}_{|\free(\pi^{*})}$ is an injective map. Having
proved the proposition in the case when $W=V$, it can be applied to
$\sigma^{*}:U\to U$ and $\pi^{*}\in{\bf\Pi}(U)$. Hence, since $i$
and $j$ are injective maps, using
proposition~(\ref{logic:prop:FUAP:substrank:injective}) we obtain
the following equalities:
    \begin{eqnarray*}
    \rnk(\sigma(\pi))&=&\rnk(j\circ\sigma(\pi))\\
    \mbox{A: to be proved}\ \rightarrow&=&\rnk(\sigma^{*}(\pi^{*}))\\
    \mbox{case $W=V$}\ \rightarrow&=&\rnk(\pi^{*})\\
    &=&\rnk(i(\pi))\\
    \mbox{prop.~(\ref{logic:prop:FUAP:substrank:injective})}\ \rightarrow
    &=&\rnk(\pi)
    \end{eqnarray*}
So it remains to show that $j\circ\sigma(\pi)=\sigma^{*}(\pi^{*})$
and furthermore that $\sigma^{*}$ is valid for $\pi^{*}$ while
$\sigma^{*}_{|\free(\pi^{*})}$ is an injective map. First we show
that $j\circ\sigma(\pi)=\sigma^{*}(\pi^{*})$. Since $\pi^{*}=i(\pi)$
from proposition~(\ref{logic:prop:FUAP:variable:support}) it is
sufficient to prove the equality $j\circ\sigma(u)=\sigma^{*}\circ
i(u)$ for all $u\in\var(\pi)$. So let $u\in\var(\pi)$. In particular
$u\in V$ and consequently $i(u)\in i(V)\subseteq U$. From the above
definition of $\sigma^{*}$ we obtain immediately
$\sigma^{*}(i(u))=j\circ\sigma(u)$ as requested. So we now prove
that $\sigma^{*}$ is valid for $\pi^{*}=i(\pi)$. Using
proposition~(\ref{logic:prop:FUAP:validsubproof:composition}) it is
sufficient to prove that $\sigma^{*}\circ i$ is valid for $\pi$.
However, having proved that $\sigma^{*}\circ
i(\pi)=j\circ\sigma(\pi)$, from
proposition~(\ref{logic:prop:FUAP:validsubproof:equal:image}) it is
sufficient to prove that $j\circ\sigma$ is valid for $\pi$. Having
assumed that $\sigma$ is valid for $\pi$, using
proposition~(\ref{logic:prop:FUAP:validsubproof:composition}) once
more it remains to show that $j$ is valid for $\sigma(\pi)$ which
follows from the injectivity of $j$ and
proposition~(\ref{logic:prop:FUAP:validsubproof:injective}). So it
remains to prove that $\sigma^{*}_{|\free(\pi^{*})}$ is an injective
map. So let $u,v\in \free(\pi^{*})$ such that
$\sigma^{*}(u)=\sigma^{*}(v)$. We need to show that $u=v$. However
since $\pi^{*}=i(\pi)$, from
proposition~(\ref{logic:prop:FUAP:freevarproof:substitution:inclusion})
we have $\free(\pi^{*})\subseteq i(\free(\pi))$. Hence, there exists
$x,y\in\free(\pi)$ such that $u=i(x)$ and $v=i(y)$. Having proved
that $j\circ\sigma =\sigma^{*}\circ i$ on $\var(\pi)$, from the
equality $\sigma^{*}(u)=\sigma^{*}(v)$ we obtain
$j\circ\sigma(x)=j\circ\sigma(y)$. It follows from the injectivity
of $j$ that $\sigma(x)=\sigma(y)$. Having assumed that $\sigma$ is
injective on $\free(\pi)$ we conclude that $x=y$ and finally that
$u=v$ as requested.
\end{proof}

The following proposition is the counterpart of
proposition~(\ref{logic:prop:FOPL:substrank:impl})\,:

\begin{prop}\label{logic:prop:FUAP:substrank:recursion:pon}
Let $V$ be a set and $\pi\in\pvs$ of the form
$\pi=\pi_{1}\pon\pi_{2}$ where $\pi_{1},\pi_{2}\in\pvs$. Then the
substitution ranks of $\pi$, $\pi_{1}$ and $\pi_{2}$ satisfy the
equality:
    \[
    \rnk(\pi)=\max(\,|\free(\pi)|\,,\,\rnk(\pi_{1})\,,\,\rnk(\pi_{2})\,)
    \]
\end{prop}
\begin{proof}
First we show that
$\max(\,|\free(\pi)|\,,\,\rnk(\pi_{1})\,,\,\rnk(\pi_{2})\,)\leq\rnk(\pi)$.
From proposition~(\ref{logic:prop:FUAP:substrank:basic:ineq}) we
already know that $|\free(\pi)|\leq\rnk(\pi)$. So it remains to show
that $\rnk(\pi_{1})\leq\rnk(\pi)$ and $\rnk(\pi_{2})\leq\rnk(\pi)$.
So let $\sim$ be the alpha equivalence on \pvs\ and
$\rho\sim\pi$. We need to show that $\rnk(\pi_{1})\leq|\var(\rho)|$
and $\rnk(\pi_{2})\leq|\var(\rho)|$. However, from
$\rho\sim\pi=\pi_{1}\pon\pi_{2}$ and
theorem~(\ref{logic:the:FUAP:charsubcong:charac}) of
page~\pageref{logic:the:FUAP:charsubcong:charac} we see that $\rho$
must be of the form $\rho=\rho_{1}\pon\rho_{2}$ where
$\rho_{1}\sim\pi_{1}$ and $\rho_{2}\sim\pi_{2}$. Hence we have
$\rnk(\pi_{1})\leq|\var(\rho_{1})|\leq|\var(\rho)|$ and similarly
$\rnk(\pi_{2})\leq|\var(\rho_{2})|\leq|\var(\rho)|$. So it remains
to show the inequality
$\rnk(\pi)\leq\max(\,|\free(\pi)|\,,\,\rnk(\pi_{1})\,,\,\rnk(\pi_{2})\,)$.
We shall distinguish two cases: first we assume that
$\max(\rnk(\pi_{1}),\rnk(\pi_{2}))\leq|\free(\pi)|$. Since
$\free(\pi_{1})\subseteq\free(\pi)$ and
$\free(\pi_{2})\subseteq\free(\pi)$ using
proposition~(\ref{logic:prop:FUAP:substrank:changeofvar}) we obtain
the existence of $\rho_{1}\sim\pi_{1}$ and $\rho_{2}\sim\pi_{2}$
such that $|\var(\rho_{1})|=\rnk(\pi_{1})$ and
$|\var(\rho_{2})|=\rnk(\pi_{2})$ with the inclusions
$\var(\rho_{1})\subseteq\free(\pi)$ and
$\var(\rho_{2})\subseteq\free(\pi)$. Since
$\pi\sim\rho_{1}\pon\rho_{2}$\,:
    \begin{eqnarray*}
    \rnk(\pi)&\leq&|\var(\rho_{1}\pon\rho_{2})|\\
    &=&|\var(\rho_{1})\cup\var(\rho_{2})|\\
    \var(\rho_{i})\subseteq\free(\pi)\ \rightarrow&\leq&|\free(\pi)|\\
    &=&\max(\,|\free(\pi)|\,,\,\rnk(\pi_{1})\,,\,\rnk(\pi_{2})\,)
    \end{eqnarray*}
Next we assume that
$|\free(\pi)|\leq\max(\rnk(\pi_{1}),\rnk(\pi_{2}))$. We shall
distinguish two further cases: first we assume that
$\rnk(\pi_{1})\leq\rnk(\pi_{2})$. Since we have both
$\free(\pi_{2})\subseteq\free(\pi)$ and
$|\free(\pi)|\leq\rnk(\pi_{2})$, from
proposition~(\ref{logic:prop:FUAP:substrank:changeofvar}) we can
find $\rho_{2}\sim\pi_{2}$ such that
$|\var(\rho_{2})|=\rnk(\pi_{2})$ and
$\free(\pi)\subseteq\var(\rho_{2})$. In particular we obtain the
inclusion $\free(\pi_{1})\subseteq\var(\rho_{2})$ and applying
proposition~(\ref{logic:prop:FUAP:substrank:changeofvar}) once more,
from  $\rnk(\pi_{1})\leq\rnk(\pi_{2})=|\var(\rho_{2})|$ we obtain
the existence of $\rho_{1}\sim\pi_{1}$ such that
$|\var(\rho_{1})|=\rnk(\pi_{1})$ and
$\var(\rho_{1})\subseteq\var(\rho_{2})$. It follows that:
    \begin{eqnarray*}
    \rnk(\pi)&\leq&|\var(\rho_{1}\pon\rho_{2})|\\
    &=&|\var(\rho_{1})\cup\var(\rho_{2})|\\
    \var(\rho_{1})\subseteq\var(\rho_{2})\ \rightarrow&=&|\var(\rho_{2})|\\
    &=&\rnk(\pi_{2})\\
    &=&\max(\,|\free(\pi)|\,,\,\rnk(\pi_{1})\,,\,\rnk(\pi_{2})\,)
    \end{eqnarray*}
The case $\rnk(\pi_{2})\leq\rnk(\pi_{1})$ is dealt with similarly.
\end{proof}

The following proposition is the counterpart of
proposition~(\ref{logic:prop:FOPL:substrank:quant})\,:

\begin{prop}\label{logic:prop:FUAP:substrank:recursion:gen}
Let $V$ be a set and $\pi\in\pvs$ of the form $\pi=\gen x\pi_{1}$
where $\pi_{1}\in\pvs$ and $x\in V$. Then the substitution ranks of
$\pi$ and $\pi_{1}$ satisfy:
    \[
    \rnk(\pi)=\rnk(\pi_{1})+\epsilon
    \]
where $\epsilon\in 2=\{0,1\}$ is given by the equivalence
$\epsilon=1$ \ifand:
    \[
    (\,x\not\in\free(\pi_{1})\,)\land(\,|\free(\pi_{1})|
    =\rnk(\pi_{1})\,)
    \]
\end{prop}
\begin{proof}
Let $\pi=\gen x\pi_{1}$ where $\pi_{1}\in\pvs$ and $x\in V$. Define
$\epsilon = \rnk(\pi)-\rnk(\pi_{1})$. Then $\epsilon$ is an integer,
possibly negative. In order to prove that $\epsilon\in 2$ it is
therefore sufficient to prove that the following inequalities hold:
    \begin{equation}\label{logic:eqn:FUAP:substrank:recursion:gen:1}
    \rnk(\pi_{1})\leq\rnk(\pi)\leq\rnk(\pi_{1})+1
    \end{equation}
This will be the first part of our proof. Next we shall show the
equivalence:
    \begin{equation}\label{logic:eqn:FUAP:substrank:recursion:gen:2}
    (\epsilon =1)\ \Leftrightarrow\
    (\,x\not\in\free(\pi_{1})\,)\land(\,|\free(\pi_{1})|
    =\rnk(\pi_{1})\,)
    \end{equation}
So first we show that $\rnk(\pi_{1})\leq\rnk(\pi)$. Let
$\rho\sim\pi$ where $\sim$ denotes the alpha equivalence on
\pvs. We need to show that $\rnk(\pi_{1})\leq|\var(\rho)|$. Using
theorem~(\ref{logic:the:FUAP:charsubcong:charac}) of
page~\pageref{logic:the:FUAP:charsubcong:charac}, from the
equivalence $\rho\sim\pi$ we see that $\rho$ is either of the form
$\rho=\gen x\rho_{1}$ where $\rho_{1}\sim\pi_{1}$, or $\rho$ is of
the form $\rho=\gen y\rho_{1}$ where $\rho_{1}\sim\pi_{1}[y\!:\!x]$,
$x\neq y$ and $y\not\in\free(\pi_{1})$. First we assume that
$\rho=\gen x\rho_{1}$ where $\rho_{1}\sim\pi_{1}$. Then we have the
following inequalities:
    \begin{eqnarray*}
    \rnk(\pi_{1})&\leq&|\var(\rho_{1})|\\
    &\leq&|\{x\}\cup\var(\rho_{1})|\\
    &=&|\var(\gen x\rho_{1})|\\
    &=&|\var(\rho)|
    \end{eqnarray*}
Next we assume that $\rho=\gen y\rho_{1}$ where
$\rho_{1}\sim\pi_{1}[y\!:\!x]$, $x\neq y$ and
$y\not\in\free(\pi_{1})$. The permutation $[y\!:\!x]$ being
injective, from
proposition~(\ref{logic:prop:FUAP:charsubcong:injective:substitution})
we obtain $\rho_{1}^{*}\sim\pi_{1}$ where
$\rho_{1}^{*}=\rho_{1}[y\!:\!x]$. Furthermore, defining
$\rho^{*}=\gen x\rho_{1}^{*}$ we have $\rho=\rho^{*}[y\!:\!x]$.
Using the injectivity of $[y\!:\!x]$ once more and
proposition~(\ref{logic:prop:FUAP:variable:substitution}) we obtain:
    \[
    |\var(\rho)|=|\var(\rho^{*}[y\!:\!x])|=
    |\,[y\!:\!x](\var(\rho^{*}))\,|=|\var(\rho^{*})|
    \]
So we need to prove that $\rnk(\pi_{1})\leq|\var(\rho^{*})|$ where
$\rho^{*}=\gen x\rho_{1}^{*}$ and $\rho_{1}^{*}\sim\pi_{1}$. Hence
we are back to our initial case and we have proved that
$\rnk(\pi_{1})\leq\rnk(\pi)$. Next we show that
$\rnk(\pi)\leq\rnk(\pi_{1})+1$. So let $\rho_{1}\sim\pi_{1}$. We
need to show that $\rnk(\pi)-1\leq|\var(\rho_{1})|$ or equivalently
that $\rnk(\pi)\leq|\var(\rho_{1})|+1$:
    \begin{eqnarray*}
    \rnk(\pi)&=&\rnk(\gen x\pi_{1})\\
    \gen x\pi_{1}\sim\gen x\rho_{1}\ \rightarrow
    &\leq&|\var(\gen x\rho_{1})|\\
    &=&|\{x\}\cup\var(\rho_{1})|\\
    &\leq&|\var(\rho_{1})|+1
    \end{eqnarray*}
So we are done proving the
inequalities~(\ref{logic:eqn:FUAP:substrank:recursion:gen:1}). We
shall complete the proof of this proposition by showing the
equivalence~(\ref{logic:eqn:FUAP:substrank:recursion:gen:2}). First
we show $\Rightarrow$\,: so we assume that $\epsilon=1$. We need to
show that $x\not\in\free(\pi_{1})$ and furthermore that
$|\free(\pi_{1})|=\rnk(\pi_{1})$. First we show that
$x\not\in\free(\pi_{1})$. So suppose to the contrary that
$x\in\free(\pi_{1})$. We shall obtain a contradiction by showing
$\epsilon=0$, that is $\rnk(\pi_{1})=\rnk(\pi)$. We already know
that $\rnk(\pi_{1})\leq\rnk(\pi)$. So we need to show that
$\rnk(\pi)\leq\rnk(\pi_{1})$. So let $\rho_{1}\sim\pi_{1}$. We need
to show that $\rnk(\pi)\leq|\var(\rho_{1})|$. However, from
$\rho_{1}\sim\pi_{1}$ and
proposition~(\ref{logic:prop:FUAP:charsubcong:freevar}) we obtain
$\free(\rho_{1})=\free(\pi_{1})$ and in particular
$x\in\free(\rho_{1})$. It follows that:
    \begin{eqnarray*}
    \rnk(\pi)&=&\rnk(\gen x\pi_{1})\\
    \gen x\pi_{1}\sim\gen x\rho_{1}\ \rightarrow
    &\leq&|\var(\gen x\rho_{1})|\\
    &=&|\{x\}\cup\var(\rho_{1})|\\
    x\in\free(\rho_{1})\subseteq\var(\rho_{1})\ \rightarrow
    &=&|\var(\rho_{1})|\\
    \end{eqnarray*}
This is our desired contradiction and we conclude that
$x\not\in\free(\pi_{1})$. It remains to show that
$|\free(\pi_{1})|=\rnk(\pi_{1})$. So suppose this equality does not
hold. We shall obtain a contradiction by showing $\epsilon=0$, that
is $\rnk(\pi)\leq\rnk(\pi_{1})$. So let $\rho_{1}\sim\pi_{1}$. We
need to show once again that $\rnk(\pi)\leq|\var(\rho_{1})|$.
However, having assumed the equality
$|\free(\pi_{1})|=\rnk(\pi_{1})$ does not hold, from
proposition~(\ref{logic:prop:FUAP:substrank:basic:ineq}) we obtain
$|\free(\pi_{1})|<\rnk(\pi_{1})$ and consequently from
$\rho_{1}\sim\pi_{1}$ we have:
    \[
    |\free(\rho_{1})|=|\free(\pi_{1})|
    <\rnk(\pi_{1})
    \leq|\var(\rho_{1})|
    \]
It follows that the set $\var(\rho_{1})\setminus\free(\rho_{1})$
cannot be empty, and there exists
$y\in\var(\rho_{1})\setminus\free(\rho_{1})$. From
$x\not\in\free(\pi_{1})$ and
$y\not\in\free(\rho_{1})=\free(\pi_{1})$ using
proposition~(\ref{logic:prop:FUAP:charsubcong:xy:not:free}) we
obtain the equivalence $\gen x\pi_{1}\sim\gen y\pi_{1}$. Hence we
also have the equivalence $\gen x\pi_{1}\sim\gen y\rho_{1}$ and
consequently:
    \begin{eqnarray*}
    \rnk(\pi)&=&\rnk(\gen x\pi_{1})\\
        \gen x\pi_{1}\sim\gen y\rho_{1}\ \rightarrow
        &\leq&|\var(\gen y\rho_{1})|\\
        &=&|\{y\}\cup\var(\rho_{1})|\\
        y\in\var(\rho_{1})\ \rightarrow
        &=&|\var(\rho_{1})|
    \end{eqnarray*}
which is our desired contradiction and we conclude that
$|\free(\pi_{1})|=\rnk(\pi_{1})$. This completes our proof of
$\Rightarrow$ in the
equivalence~(\ref{logic:eqn:FUAP:substrank:recursion:gen:2}). We now
prove $\Leftarrow$\,: so we assume that $x\not\in\free(\pi_{1})$ and
$|\free(\pi_{1})|=\rnk(\pi_{1})$. We need to show that $\epsilon=1$,
that is $\rnk(\pi)=\rnk(\pi_{1})+1$. We already have
$\rnk(\pi)\leq\rnk(\pi_{1})+1$. So it remains to show that
$\rnk(\pi_{1})+1\leq\rnk(\pi)$ or equivalently
$\rnk(\pi_{1})<\rnk(\pi)$. So let $\rho\sim\pi$. We need to show
that $\rnk(\pi_{1})<|\var(\rho)|$. Once again, using
theorem~(\ref{logic:the:FUAP:charsubcong:charac}) of
page~\pageref{logic:the:FUAP:charsubcong:charac}, from the
equivalence $\rho\sim\pi$ we see that $\rho$ is either of the form
$\rho=\gen x\rho_{1}$ where $\rho_{1}\sim\pi_{1}$, or $\rho$ is of
the form $\rho=\gen y\rho_{1}$ where $\rho_{1}\sim\pi_{1}[y\!:\!x]$,
$x\neq y$ and $y\not\in\free(\pi_{1})$. First we assume that
$\rho=\gen x\rho_{1}$ where $\rho_{1}\sim\pi_{1}$. Hence
$\rnk(\pi_{1})\leq|\var(\rho_{1})|$ and we shall distinguish two
further cases: first we assume that
$\rnk(\pi_{1})=|\var(\rho_{1})|$. Having assumed that
$|\free(\pi_{1})|=\rnk(\pi_{1})$ we obtain:
    \[
    |\free(\rho_{1})|=|\free(\pi_{1})|=\rnk(\pi_{1})=|\var(\rho_{1})|
    \]
from which we see that $\var(\rho_{1})=\free(\rho_{1})$ and
consequently it follows that
$x\not\in\free(\pi_{1})=\free(\rho_{1})=\var(\rho_{1})$. Hence we
see that:
    \begin{eqnarray*}
    \rnk(\pi_{1})&\leq&|\var(\rho_{1})|\\
    x\not\in\var(\rho_{1})\ \rightarrow
    &<&|\{x\}\cup\var(\rho_{1})|\\
    &=&|\var(\gen x\rho_{1})|\\
    &=&|\var(\rho)|
    \end{eqnarray*}
So we now assume that $\rnk(\pi_{1})<|\var(\rho_{1})|$, in which
case we obtain:
    \begin{eqnarray*}
    \rnk(\pi_{1})&<&|\var(\rho_{1})|\\
    &\leq&|\{x\}\cup\var(\rho_{1})|\\
    &=&|\var(\gen x\rho_{1})|\\
    &=&|\var(\rho)|
    \end{eqnarray*}
We now consider the case when $\rho=\gen y\rho_{1}$ where
$\rho_{1}\sim\pi_{1}[y\!:\!x]$, $x\neq y$ and
$y\not\in\free(\pi_{1})$. Once again, defining
$\rho_{1}^{*}=\rho_{1}[y\!:\!x]$ and $\rho^{*}=\gen x\rho_{1}^{*}$
we obtain the equivalence $\rho_{1}^{*}\sim\pi_{1}$ and
$|\var(\rho^{*})|=|\var(\rho)|$. So we need to prove that
$\rnk(\pi_{1})<|\var(\rho^{*})|$ knowing that
$\rho_{1}^{*}\sim\pi_{1}$ which follows from our initial case.
\end{proof}

As already discussed at the start of this section, the existence of
essential substitutions crucially relies on the inequality
$\rnk(\bar{\sigma}\circ{\cal M}(\pi))\leq |W|$. The following
proposition allows us to write $\rnk(\bar{\sigma}\circ{\cal
M}(\pi))\leq\rnk({\cal M}(\pi))$. Since we know from
proposition~(\ref{logic:prop:FUAP:substrank:minrank}) that
$\rnk({\cal M}(\pi))=\rnk(\pi)$ we finally obtain
$\rnk(\bar{\sigma}\circ{\cal M}(\pi))\leq\rnk(\pi)$, which gives us
the sufficient condition $\rnk(\pi)\leq |W|$. Note that the result
makes no use of the fact that $\bar{\sigma}$ is valid for ${\cal
M}(\pi)$, and the inequality $\rnk(\sigma(\pi))\leq\rnk(\pi)$ is
always true, regardless of whether $\sigma$ is valid for $\pi$. The
following proposition is the counterpart of
proposition~(\ref{logic:prop:FOPL:subst:rank:substitution})\,:

\begin{prop}\label{logic:prop:FUAP:substrank:substitution}
Let $V,W$ be sets and $\sigma:V\to W$ be a map. Let $\pi\in\pvs$:
    \[
    \rnk(\sigma(\pi))\leq\rnk(\pi)
    \]
where $\sigma:\pvs\to{\bf\Pi}(W)$ is the associated proof
substitution mapping.
\end{prop}
\begin{proof}
We shall prove $\rnk(\sigma(\pi))\leq\rnk(\pi)$ by structural
induction using theorem~(\ref{logic:the:proof:induction}) of
page~\pageref{logic:the:proof:induction}. First we assume that
$\pi=\phi$ for some $\phi\in\pv$. Then the inequality follows from
proposition~(\ref{logic:prop:FOPL:subst:rank:substitution}). We now
assume that $\pi=\axi\phi$ for some $\phi\in\pv$. Then using
proposition~(\ref{logic:prop:FOPL:subst:rank:substitution}) once
more together with
proposition~(\ref{logic:prop:FUAP:substrank:recursion:axiom})\,:
    \begin{eqnarray*}
    \rnk(\sigma(\pi))&=&\rnk(\sigma(\axi\phi))\\
    &=&\rnk(\axi\sigma(\phi))\\
    \mbox{prop.~(\ref{logic:prop:FUAP:substrank:recursion:axiom})}\ \rightarrow
    &=&\rnk(\sigma(\phi))\\
    \mbox{prop.~(\ref{logic:prop:FOPL:subst:rank:substitution})}\ \rightarrow
    &\leq&\rnk(\phi)\\
    \mbox{prop.~(\ref{logic:prop:FUAP:substrank:recursion:axiom})}\ \rightarrow
    &=&\rnk(\axi\phi)\\
    &=&\rnk(\pi)\\
    \end{eqnarray*}
So we now assume that $\pi=\pi_{1}\pon\pi_{2}$ where
$\pi_{1},\pi_{2}\in\pvs$ satisfy our property:
    \begin{eqnarray*}
    \rnk(\sigma(\pi))&=&\rnk(\sigma(\pi_{1}\pon\pi_{2}))\\
    &=&\rnk(\sigma(\pi_{1})\pon\,\sigma(\pi_{2}))\\
    \mbox{prop.~(\ref{logic:prop:FUAP:substrank:recursion:pon})}\ \rightarrow
    &=&\max(\,|\,\free(\sigma(\pi))\,|\,,\,
    \rnk(\sigma(\pi_{1}))\,,\,\rnk(\sigma(\pi_{2}))\,)\\
    &\leq&\max(\,|\,\free(\sigma(\pi))\,|\,,\,
    \rnk(\pi_{1})\,,\,\rnk(\pi_{2})\,)\\
    \mbox{prop.~(\ref{logic:prop:FUAP:freevarproof:substitution:inclusion})}\ \rightarrow
    &\leq&\max(\,|\,\sigma(\free(\pi))\,|\,,\,
    \rnk(\pi_{1})\,,\,\rnk(\pi_{2})\,)\\
    &\leq&\max(\,|\free(\pi)|\,,\,
    \rnk(\pi_{1})\,,\,\rnk(\pi_{2})\,)\\
    \mbox{prop.~(\ref{logic:prop:FUAP:substrank:recursion:pon})}\ \rightarrow
    &=&\rnk(\pi)
    \end{eqnarray*}
Finally we assume that $\pi=\gen x\pi_{1}$ where $x\in V$ and
$\pi_{1}\in\pvs$ satisfies our induction property. Then
$\sigma(\pi)=\gen\sigma(x)\sigma(\pi_{1})$. Let $\epsilon
=\rnk(\pi)-\rnk(\pi_{1})$ and let
$\eta=\rnk(\sigma(\pi))-\rnk(\sigma(\pi_{1}))$. From
proposition~(\ref{logic:prop:FUAP:substrank:recursion:gen}) we know
that $\epsilon,\eta\in 2$. We shall distinguish two cases: first we
assume that $\eta=0$. Then we have:
    \[
    \rnk(\sigma(\pi))=\rnk(\sigma(\pi_{1}))\leq\rnk(\pi_{1})\leq\rnk(\pi)
    \]
Next we assume that $\eta=1$. We shall distinguish two further
cases: if $\epsilon=1$\,:
    \[
    \rnk(\sigma(\pi))=1+\rnk(\sigma(\pi_{1}))\leq1+\rnk(\pi_{1})=\rnk(\pi)
    \]
So it remains to deal with the last possibility when $\eta=1$ and
$\epsilon=0$. Using
proposition~(\ref{logic:prop:FUAP:substrank:recursion:gen}), from
$\epsilon=0$ we see that $x\in\free(\pi_{1})$ or
$|\free(\pi_{1})|<\rnk(\pi_{1})$. So we shall distinguish two
further cases. These cases may not be exclusive of one another but
we don't need them to be. So first we assume $x\in\free(\pi_{1})$.
Using proposition~(\ref{logic:prop:FUAP:substrank:recursion:gen})
again, from $\eta=1$ we obtain
$|\free(\sigma(\pi_{1}))|=\rnk(\sigma(\pi_{1}))$ and furthermore
$\sigma(x)\not\in\free(\sigma(\pi_{1}))$. Having assumed that
$x\in\free(\pi_{1})$ it follows that $\sigma(x)$ is an element of
$\sigma(\free(\pi_{1}))$ but not an element of
$\free(\sigma(\pi_{1}))$. Hence, we see that the inclusion
$\free(\sigma(\pi_{1}))\subseteq\sigma(\free(\pi_{1}))$ which we
know is true from
proposition~(\ref{logic:prop:FUAP:freevarproof:substitution:inclusion})
is in fact a strict inclusion. So we have a strict inequality
between the finite cardinals
$|\free(\sigma(\pi_{1}))|<|\sigma(\free(\pi_{1}))|$. Thus:
    \begin{eqnarray*}
    \rnk(\sigma(\pi))&=&1+\rnk(\sigma(\pi_{1}))\\
    |\free(\sigma(\pi_{1}))|=\rnk(\sigma(\pi_{1}))\ \rightarrow
    &=&1+|\free(\sigma(\pi_{1}))|\\
    |\free(\sigma(\pi_{1}))|<|\sigma(\free(\pi_{1}))|\ \rightarrow
    &\leq&|\sigma(\free(\pi_{1}))|\\
    &\leq&|\free(\pi_{1})|\\
    \mbox{prop.~(\ref{logic:prop:FUAP:substrank:basic:ineq})}\ \rightarrow
    &\leq&\rnk(\pi_{1})\\
    \epsilon=0\ \rightarrow&=&\rnk(\pi)
    \end{eqnarray*}
So we now assume that $|\free(\pi_{1})|<\rnk(\pi_{1})$. In this case
we have:
    \begin{eqnarray*}
    \rnk(\sigma(\pi))&=&1+\rnk(\sigma(\pi_{1}))\\
    |\free(\sigma(\pi_{1}))|=\rnk(\sigma(\pi_{1}))\ \rightarrow
    &=&1+|\free(\sigma(\pi_{1}))|\\
    \mbox{prop.~(\ref{logic:prop:FUAP:freevarproof:substitution:inclusion})}\ \rightarrow
    &\leq&1+|\sigma(\free(\pi_{1}))|\\
    &\leq&1+|\free(\pi_{1})|\\
    |\free(\pi_{1})|<\rnk(\pi_{1})\ \rightarrow
    &\leq&\rnk(\pi_{1})\\
    \epsilon=0\ \rightarrow&=&\rnk(\pi)
    \end{eqnarray*}
\end{proof}

This completes our study of substitution ranks for proofs. We are
now ready to prove the existence of essential substitutions
$\sigma:\pvs\to{\bf \Pi}(W)$ associated with $\sigma:V\to W$. This
will be done in the following section.
