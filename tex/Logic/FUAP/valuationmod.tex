In the previous section, we saw that a proof of the form
$\pi=\pi_{1}\pon\pi_{2} $ where $\val(\pi_{1})=\forall\,1\phi(1,1)$
and $\val(\pi_{2})=\forall\,0\phi(0,0)\to\phi(x,x)$ fails to be a
legitimate use of the modus ponens rule of inference. We are not
discovering anything, as we knew this all along. We are deliberately
being dumb, refusing to understand the proof $\pi$ while any
ordinary mathematician would most likely accept it as a legitimate
proof of $\phi(x,x)$, provided both $\pi_{1}$ and $\pi_{2}$ are
themselves acceptable. This is how we defined our key semantics on
$\pvs$. From definition~(\ref{logic:def:FOPL:proof:valuation}), the
valuation mapping $\val:\pvs\to\pv$ will not return anything
sensible unless its argument is a totally clean proof, as per
definition~(\ref{logic:def:FUAP:clean:clean:proof}). The benefit of
this approach is that we were able to define a notion of provability
and sequent $\Gamma\vdash\phi$  in
definition~(\ref{logic:def:FOPL:proof:of:formula}) while keeping to
a minimum the complexity associated with alpha equivalence. 
As you may recall, we were not able to completely strip
out references to this congruence, as our specialization axioms
namely $\forall x\phi_{1}\to\phi_{1}[y/x]$ are defined in terms of
essential substitutions of $y$ in place of $x$, as per
definition~(\ref{logic:def:FOPL:specialization:axiom}). However, we
avoided the use of quotients and equivalence classes, while staying
well clear of any informal treatment where the reader is casually
invited to assume renaming of variables is continually taking place
to avoid variable capture. In short, we kept it simple but without
waving our hands on variable binding, capture and
alpha equivalence. It is all very nice, but we have now
discovered a major flaw in this approach: as explained in the
previous section, our very natural definition of minimal transform
for proofs is failing, seemingly because $\pi=\pi_{1}\pon\pi_{2}$ is
not regarded as a legitimate use of the modus ponens rule of
inference, despite the equivalence
$\forall\,1\phi(1,1)\sim\forall\,0\phi(0,0)$. So we are back to the
drawing board. We are committed to defining a minimal transform
mapping on \pvs\ and our tentative definition is so natural that we
do not believe it should be changed: what is wrong is our deductive
system.

Luckily, we can address this issue with minimal work. We defined our
set of proofs as a set of abstract formal expressions. \pvs\ is a
free universal algebra. This is as general as it gets. We allowed a
proof $\pi_{1}\pon\pi_{2}$ to be meaningful at all times. A
generalization $\gen x\pi_{1}$ always exists. Even an axiom
invocation $\axi\phi$ can be done without $\phi$ being an axiom of
first order logic. A proof is just a skeleton. It doesn't say much
about provability which is defined solely in terms of semantics.
What matters is not so much the proof, but our interpretation of it.
If we need to adjust our deductive system, we simply need to change
the semantics. Right now, we have a valuation $\val:\pvs\to\pv$
which maps every proof $\pi\in\pvs$ to its conclusion
$\val(\pi)\in\pv$. Although this map is a total function, it is only
truly meaningful when $\pi$ is totally clean, as per
definition~(\ref{logic:def:FUAP:clean:clean:proof}). In effect, the
domain of the valuation $\val$ is the set of totally clean proofs.
This is our kernel for provability. We now wish to extend this
kernel by defining a new valuation $\vals:\pvs\to\pv$ which
coincides with $\val$ on totally clean proofs, but has a wider
domain of acceptable proofs. We want this new valuation to be
clever: it should know about $\alpha$-equivalence and be flexible
about it. It should not require that $\phi$ be an axiom of first
order logic to accept $\axi\phi$ as a legitimate axiom invocation,
but should return $\vals(\axi\phi)=\phi$ as long as $\phi$ is
equivalent to an axiom. With this in mind, we define:

\index{axiom@Axiom modulo}
\begin{defin}\label{logic:def:FUAP:valuationmod:axiom:modulo}
Let $V$ be a set and $\sim$ be the alpha equivalence. We say
that $\phi\in\pv$ is an {\em axiom modulo} \ifand\ there exists an
axiom $\psi\in\av$ such that $\phi\sim\psi$. The set of axioms
modulo on \pv\ is denoted \avs.
\end{defin}

So we should have $\vals(\axi\phi)=\phi$ whenever $\phi$ is an axiom
modulo, which improves on $\val(\axi\phi)=\bot\to\bot$ which may
occur if $\phi$ is not strictly an axiom of first order logic. Thus,
the valuation $\vals$ will have more flexibility than $\val$ with
regards to axioms. We also want more flexibility with regards to
modus ponens. At this point, we have
$\val(\pi_{1}\pon\pi_{2})=M(\val(\pi_{1}),\val(\pi_{2}))$ where
$M:\pv^{2}\to\pv$ is the modus ponens mapping of
definition~(\ref{logic:def:FOPL:modus:ponens}). This mapping does
not allow any equality modulo and insists on the strict equality
$\val(\pi_{2})=\val(\pi_{1})\to\phi$ for some $\phi\in\pv$. We shall
introduce more flexibility in the valuation $\vals$ by defining
$\vals(\pi_{1}\pon\pi_{2})=M^{+}(\vals(\pi_{1}),\vals(\pi_{2}))$
where $M^{+}:\pv^{2}\to\pv$ understands $\alpha$-equivalence. Note
that the following definition uses a notational shortcut for
readability: the mathematical statement 'if
$\phi_{2}=\psi_{1}\to\psi_{2}$ and $\psi_{1}\sim\phi_{1}$' should be
rightly understood to mean 'if there exist $\psi_{1},\psi_{2}\in\pv$
such that $\phi_{2}=\psi_{1}\to\psi_{2}$ and
$\psi_{1}\sim\phi_{1}$'\,:

\index{modus@Modus ponens
mapping}\index{m@$M^{+}(\phi_{1},\phi_{2})$ : mp map modulo}
\begin{defin}\label{logic:def:FUAP:valuationmod:mp:modulo}
Let $V$ be a set and $\sim$ be the alpha equivalence. We call
{\em modus ponens mapping modulo} the map $M^{+}:\pv^{2}\rightarrow
\pv$ defined by:
    \[
    \forall \phi_{1},\phi_{2}\in\pv\ ,\
    M^{+}(\phi_{1},\phi_{2})=\left\{
        \begin{array}{lcl}
        \psi_{2}&\mbox{\ if\ }&\phi_{2}=\psi_{1}\to\psi_{2}
        \mbox{\ and\ } \psi_{1}\sim\phi_{1}\\
        \bot\to\bot&\mbox{\ \ }&\mbox{otherwise}\\
        \end{array}
    \right.
    \]
\end{defin}

It remains to deal with generalization. At this point, we have the
equality $\val(\gen x\pi_{1})=\bot\to\bot$ whenever $x$ is not an
arbitrary variable, i.e. $x\in\spec(\pi_{1})$. Such a variable is
called a {\em specific} variable of the proof $\pi_{1}$ as per
definition~(\ref{logic:def:FOPL:proof:free:variable}). The specific
variables of $\pi_{1}$ are the free variables of the hypothesis of
$\pi_{1}$. The free variables of any axiom involved in $\pi_{1}$ are
considered {\em arbitrary}, i.e. not specific. If $x$ is not a
specific variable of $\pi_{1}$, then we have $\val(\gen
x\pi_{1})=\forall x\val(\pi_{1})$. When it comes to defining our
extended valuation $\vals:\pvs\to\pv$, there is not much flexibility
which can be added with regards to generalization. We want $\vals$
to be clever. We do not want it to accept proofs which are
irretrievably flawed. We cannot allow a mathematician to generalize
with respect to a variable which is not arbitrary. So we shall keep
$\vals(\gen x\pi_{1})=\forall x\vals(\pi_{1})$ whenever
$x\not\in\spec(\pi_{1})$ and set $\vals(\gen x\pi_{1})=\bot\to\bot$
otherwise, just like we did for $\val$\,:

\index{valuation@Valuation modulo on \pvs}\index{v@$\vals$ :
$\vals:\pvs\to\pv$}\index{v@$\vals(\pi)$ : conclusion modulo of
$\pi$}\index{semantic@Semantics on \pvs}
\begin{defin}\label{logic:def:FUAP:valuationmod:valuation:modulo}
Let $V$ be a set. We call {\em valuation mapping modulo} on \pvs\
the map $\vals:\pvs\to\pv$ defined by the recursion, given
$\pi\in\pvs$:
\[
                    \vals(\pi)=\left\{
                    \begin{array}{lcl}
                    \phi&\mbox{\ if\ }&\pi=\phi\in\pv\\
                    \phi&\mbox{\ if\ }&\pi=\axi\phi,\ \phi\in\avs\\
                    \bot\to\bot&\mbox{\ if\ }&\pi=\axi\phi,\ \phi\not\in\avs\\
                    M^{+}\,(\vals(\pi_{1}), \vals(\pi_{2})) &
                    \mbox{\ if\ }&\pi=\pi_{1}\pon\pi_{2}\\
                    \forall x\vals(\pi_{1})&\mbox{\ if\ }&\pi=\gen
                    x\pi_{1},\  x\not\in\spec(\pi_{1})\\
                    \bot\to\bot&\mbox{\ if\ }&\pi=\gen
                    x\pi_{1},\  x\in\spec(\pi_{1})\\
                    \end{array}\right.
\]
where $M^{+}:\pv^{2}\to\pv$ refers to the modus ponens mapping
modulo.
\end{defin}
\begin{prop}
The structural recursion of {\em
definition~(\ref{logic:def:FUAP:valuationmod:valuation:modulo})} is
legitimate.
\end{prop}
\begin{proof}
We need to prove the existence and uniqueness of $\vals:\pvs\to\pv$,
which satisfies the six conditions of
definition~(\ref{logic:def:FUAP:valuationmod:valuation:modulo}). We
cannot apply theorem~(\ref{logic:the:structural:recursion}) of
page~\pageref{logic:the:structural:recursion} in this case. The
reason for this is that we do not wish $\vals(\gen x\pi_{1})$ to be
simply a function of $\vals(\pi_{1})$. Indeed, the conclusion modulo
of the proof $\gen x\pi_{1}$ depends on whether $x\in\spec(\pi_{1})$
or not. So we want $\vals(\gen x\pi_{1})$ to be a function of both
$\vals(\pi_{1})$ and $\pi_{1}$. So the main point is to define the
mapping $h(\gen x):\pv\times\pvs\to\pv$ by $h(\gen
x)(\phi_{1},\pi_{1}) =\forall x\phi_{1}$ if $x\not\in\spec(\pi_{1})$
and $h(\gen x)(\phi_{1},\pi_{1})=\bot\to\bot$ otherwise. We can then
apply theorem~(\ref{logic:the:structural:recursion:2}) of
page~\pageref{logic:the:structural:recursion:2}. Note that the
mapping $h(\axi\phi):\pv^{0}\times\pvs^{0}\to\pv$ should be defined
differently, depending on whether $\phi$ is an axiom modulo or not.
If $\phi\in\avs$ we set $h(\axi\phi)(0,0)=\phi$ and otherwise
$h(\axi\phi)(0,0)=\bot\to\bot$.
\end{proof}

From the beginning, our intention was to define an extension of the
valuation $\val$ from the domain of totally clean proofs to a wider
domain. Having defined $\vals:\pvs\to\pv$, our first step is to
confirm $\vals$ is indeed an extension:
\begin{prop}\label{logic:prop:FUAP:valuationmod:clean:proof}
Let $V$ be a set and $\pi\in\pvs$ be a totally clean proof. Then:
    \[
    \vals(\pi)=\val(\pi)
    \]
\end{prop}
\begin{proof}
We need to show the implication $(\,\mbox{$\pi$ totally clean}\,)\
\Rightarrow\ \vals(\pi)=\val(\pi)$. We shall do so with a structural
induction argument, using theorem~(\ref{logic:the:proof:induction})
of page~\pageref{logic:the:proof:induction}. First we assume that
$\pi=\phi$ for some $\phi\in\pv$. Then $\pi$ is always totally clean
and we have $\vals(\pi)=\phi=\val(\pi)$. Next we assume that
$\pi=\axi\phi$ for some $\phi\in\pv$. We need to show the
implication is true for $\pi$. So we assume that $\pi$ is totally
clean. From definition~(\ref{logic:def:FUAP:clean:clean:proof}) we
obtain $\phi\in\av$. So $\phi\in\avs$ and it follows that
$\vals(\pi)=\phi=\val(\pi)$. Next we assume that
$\pi=\pi_{1}\pon\pi_{2}$ where $\pi_{1},\pi_{2}\in\pvs$ are proofs
satisfying the implication. We need to show the same is true
of~$\pi$. So we assume that $\pi$ is totally clean. From
proposition~(\ref{logic:prop:FUAP:clean:modus:ponens}) it follows
that both $\pi_{1}$ and $\pi_{2}$ are totally clean and furthermore
$\val(\pi_{2})=\val(\pi_{1})\to\val(\pi)$. Having assumed the
implication is true for $\pi_{1},\pi_{2}$, it follows that
$\vals(\pi_{1})=\val(\pi_{1})$ and $\vals(\pi_{2})=\val(\pi_{2})$
and consequently $\vals(\pi_{2})=\vals(\pi_{1})\to\val(\pi)$. In
particular, we see that $\vals(\pi_{2})=\psi_{1}\to\psi_{2}$ with
$\psi_{1}\sim\vals(\pi_{1})$ for some $\psi_{1},\psi_{2}\in\pv$,
where $\sim$ is the alpha equivalence on \pv. Using
definition~(\ref{logic:def:FUAP:valuationmod:valuation:modulo}) it
follows that $\vals(\pi)=\psi_{2}=\val(\pi)$. So we now assume that
$\pi=\gen x\pi_{1}$ where $x\in V$ and $\pi_{1}\in\pvs$ is a proof
satisfying the implication. We need to show the same is true of
$\pi$. So we assume that $\pi$ is totally clean. From
proposition~(\ref{logic:prop:FUAP:clean:generalization}) it follows
that $\pi_{1}$ is itself totally clean and furthermore
$x\not\in\spec(\pi_{1})$. Having assumed the implication is true for
$\pi_{1}$, we see that $\vals(\pi_{1})=\val(\pi_{1})$ and
consequently:
    \begin{eqnarray*}
    \vals(\pi)&=&\vals(\gen x\pi_{1})\\
    x\not\in\spec(\pi_{1})\ \rightarrow
    &=&\forall x\vals(\pi_{1})\\
    &=&\forall x\val(\pi_{1})\\
    x\not\in\spec(\pi_{1})\ \rightarrow
    &=&\val(\gen x\pi_{1})\\
    &=&\val(\pi)
    \end{eqnarray*}
\end{proof}

From the valuation modulo $\vals:\pvs\to\pv$ corresponds a notion of
provability modulo and sequent modulo. In order for $\pi\in\pvs$ to
be a proof modulo of $\phi$ from $\Gamma$, we are asking for the
strict equality $\vals(\pi)=\phi$ and not simply for the
substitution equivalence $\vals(\pi)\sim\phi$. We also require the
inclusion $\hyp(\pi)\subseteq\Gamma$ rather than a weaker condition
involving $\alpha$-equivalence. As we shall see in
theorem~(\ref{logic:the:FUAP:valuationmod:provability}) below, we
could equally provide a definition of provability modulo with weaker
conditions. However, we wanted the following definition to mirror
exactly definition~(\ref{logic:def:FOPL:proof:of:formula}) and
emphasize a key fact: all we are doing is change the semantics on
\pvs\ from $\val$ to $\vals$\,:

\index{proof@Proof modulo of a
formula}\index{provability@Provability modulo}
\begin{defin}\label{logic:def:FUAP:valuationmod:proof:modulo}
Let $V$ be a set. Let $\Gamma\subseteq\pv$ and $\phi\in\pv$. We say
that $\pi\in\pvs$ is a {\em proof modulo of the formula $\phi$ from
$\Gamma$} \ifand\ we have:
    \[
    \vals(\pi)=\phi\mbox{\ and\ }\hyp(\pi)\subseteq\Gamma
    \]
We say that $\phi$ is {\em provable modulo from $\Gamma$} or that
$\Gamma$ {\em entails} $\phi$ {\em modulo} denoted:
    \[
    \Gamma\vdash_{+}\phi
    \]
\ifand\ there exists a proof modulo $\pi\in\pvs$ of the formula
$\phi$ from $\Gamma$.
\end{defin}

As the following proposition shows, the notation
$\Gamma\vdash_{+}\phi$ is unlikely to be used again, as it is simply
equivalent to $\Gamma\vdash\phi$. Thus, our new notion of {\em
provability modulo} is in fact equivalent to our initial
definition~(\ref{logic:def:FOPL:proof:of:formula}). This is of
course extremely reassuring. We want our new valuation
$\vals:\pvs\to\pv$ to be clever enough to cope with the alpha
equivalence, in order to make sense of minimal transforms on \pvs. We
do not want $\vals$ to change the set of provable formulas on \pv.
Somehow we believe '$\vdash\phi$' is an {\em absolute} on \pv.

\index{provability@Equivalence of proof modulo}
\begin{prop}\label{logic:prop:FUAP:valuationmod:equivalence}
Let $V$ be a set. Let $\Gamma\subseteq\pv$ and $\phi\in\pv$. Then we
have:
    \[
    \Gamma\vdash\phi\ \Leftrightarrow\ \Gamma\vdash_{+}\phi
    \]
i.e. $\phi$ is provable from $\Gamma$ \ifand\ $\phi$ is provable
modulo from $\Gamma$.
\end{prop}
\begin{proof}
First we show $\Rightarrow$\,: so we assume that $\Gamma\vdash\phi$,
i.e. there exists $\pi\in\pvs$ which is a proof of $\phi$ from
$\Gamma$. Using
proposition~(\ref{logic:prop:FUAP:clean:counterpart}), without loss
of generality we may assume that $\pi$ is totally clean. So we have
found a totally clean proof $\pi\in\pvs$ such that
$\hyp(\pi)\subseteq\Gamma$ and $\val(\pi)=\phi$. Since $\pi$ is
totally clean, from
proposition~(\ref{logic:prop:FUAP:valuationmod:clean:proof}) we have
$\vals(\pi)=\val(\pi)$. It follows that $\hyp(\pi)\subseteq\Gamma$
and $\vals(\pi)=\phi$. This shows that $\pi$ is a proof modulo of
$\phi$ from $\Gamma$, and so $\Gamma\vdash_{+}\phi$. We now prove
$\Leftarrow$\,: in order to prove this implication, it is sufficient
to show:
    \begin{equation}\label{logic:eqn:FUAP:valuationmod:equivalence:1}
    \hyp(\pi)\vdash\vals(\pi)
    \end{equation}
for all $\pi\in\pvs$. In other words, the valuation modulo of a
proof is provable from its hypothesis. Suppose this property has
been proved. We shall show that the implication $\Leftarrow$ is
true. So we assume that $\Gamma\vdash_{+}\phi$. We need to show that
$\Gamma\vdash\phi$. By assumption, there exists a proof $\pi\in\pvs$
such that $\hyp(\pi)\subseteq\Gamma$ and $\vals(\pi)=\phi$. Having
assumed that~(\ref{logic:eqn:FUAP:valuationmod:equivalence:1}) is
true we obtain $\hyp(\pi)\vdash\phi$. From
$\hyp(\pi)\subseteq\Gamma$ we conclude that $\Gamma\vdash\phi$ as
requested. So it remains to show
that~(\ref{logic:eqn:FUAP:valuationmod:equivalence:1}) is true. We
shall do so with a structural induction argument, using
theorem~(\ref{logic:the:proof:induction}) of
page~\pageref{logic:the:proof:induction}. First we assume that
$\pi=\phi$ for some $\phi\in\pv$. From
definition~(\ref{logic:def:FUAP:valuationmod:valuation:modulo}) we
have $\vals(\pi)=\phi$. Hence we need to show that
$\{\phi\}\vdash\phi$ which is clear. Next we assume that
$\pi=\axi\phi$ for some $\phi\in\pv$. We shall distinguish two
cases: first we assume that $\phi$ is not an axiom modulo, i.e.
$\phi\not\in\avs$. From
definition~(\ref{logic:def:FUAP:valuationmod:valuation:modulo}) we
have $\vals(\pi)=\bot\to\bot$. Hence we need to show that
$\vdash(\bot\to\bot)$ which follows from
lemma~(\ref{logic:lemma:FUAP:clean:bot:bot}). Next we assume that
$\phi$ is an axiom modulo, i.e. $\phi\in\avs$. From
definition~(\ref{logic:def:FUAP:valuationmod:valuation:modulo}) we
have $\vals(\pi)=\phi$. So we need to show that $\vdash\phi$.
However, from $\phi\in\avs$ and
definition~(\ref{logic:def:FUAP:valuationmod:axiom:modulo}), there
exists an axiom $\psi\in\av$ such that $\phi\sim\psi$, where $\sim$
is the alpha equivalence on \pv. Since $\psi$ is an axiom,
from proposition~(\ref{logic:prop:FOPL:axiom}) it is provable i.e.
we have $\vdash\psi$. It follows that $\vdash\psi$ and
$\psi\sim\phi$, and from
proposition~(\ref{logic:prop:FOPL:HDC:sequent:stronger:congruence})
we conclude that $\vdash\phi$ as requested. So we now assume that
$\pi=\pi_{1}\pon\pi_{2}$ where $\pi_{1},\pi_{2}\in\pvs$ are proofs
which satisfy the
entailment~(\ref{logic:eqn:FUAP:valuationmod:equivalence:1}). We
need to show the same is true of $\pi$. We shall distinguish two
cases: First we assume that $\vals(\pi_{2})$ cannot be written as
$\vals(\pi_{2})=\psi_{1}\to\psi_{2}$ where
$\psi_{1}\sim\vals(\pi_{1})$. From
definition~(\ref{logic:def:FUAP:valuationmod:valuation:modulo}) we
have $\vals(\pi)=\bot\to\bot$ and we need to show that
$\hyp(\pi)\vdash(\bot\to\bot)$ which is clear. Next we assume that
$\vals(\pi_{2})=\psi_{1}\to\psi_{2}$ where $\psi_{1},\psi_{2}\in\pv$
and $\psi_{1}\sim\vals(\pi_{1})$. From
definition~(\ref{logic:def:FUAP:valuationmod:valuation:modulo}) we
have $\vals(\pi)=\psi_{2}$ and we need to show that
$\hyp(\pi)\vdash\psi_{2}$. Using the modus ponens property of
proposition~(\ref{logic:prop:FOPL:modus:ponens}) it is sufficient to
prove that $\hyp(\pi)\vdash\psi_{1}$ and
$\hyp(\pi)\vdash\psi_{1}\to\psi_{2}$. First we show that
$\hyp(\pi)\vdash\psi_{1}\to\psi_{2}$\,: We need to show that
$\hyp(\pi)\vdash\vals(\pi_{2})$ which follows immediately from
$\hyp(\pi)\supseteq\hyp(\pi_{2})$ and the
entailment~(\ref{logic:eqn:FUAP:valuationmod:equivalence:1}) applied
to $\pi_{2}$. So we now show that $\hyp(\pi)\vdash\psi_{1}$. Since
$\psi_{1}\sim\vals(\pi_{1})$, from
proposition~(\ref{logic:prop:FOPL:HDC:sequent:stronger:congruence})
it is sufficient to show that $\hyp(\pi)\vdash\vals(\pi_{1})$ which
follows immediately from $\hyp(\pi)\supseteq\hyp(\pi_{1})$ and the
entailment~(\ref{logic:eqn:FUAP:valuationmod:equivalence:1}) applied
to $\pi_{1}$. This completes our induction argument in the case when
$\pi=\pi_{1}\pon\pi_{2}$. So we now assume that $\pi=\gen x\pi_{1}$
where $x\in V$ and $\pi_{1}\in\pvs$ is a proof which satisfies the
entailment~(\ref{logic:eqn:FUAP:valuationmod:equivalence:1}). We
need to show the same is true of $\pi$. We shall distinguish two
cases: first we assume that $x\in\spec(\pi_{1})$. From
definition~(\ref{logic:def:FUAP:valuationmod:valuation:modulo}) we
have $\vals(\pi)=\bot\to\bot$ and we need to show that
$\hyp(\pi)\vdash(\bot\to\bot)$ which is clear. Next we assume that
$x\not\in\spec(\pi_{1})$. From
definition~(\ref{logic:def:FUAP:valuationmod:valuation:modulo}) we
have $\vals(\pi)=\forall x\vals(\pi_{1})$ and we need to show that
$\hyp(\pi)\vdash\forall x\vals(\pi_{1})$. Since
$\hyp(\pi)=\hyp(\pi_{1})$ we have to show that
$\hyp(\pi_{1})\vdash\forall x\vals(\pi_{1})$. Since
$x\not\in\spec(\pi_{1})=\free(\hyp(\pi_{1}))$, from the
generalization property of
proposition~(\ref{logic:prop:FOPL:generalization}) it is sufficient
to prove that $\hyp(\pi_{1})\vdash\vals(\pi_{1})$ which follows from
the entailment~(\ref{logic:eqn:FUAP:valuationmod:equivalence:1})
applied to $\pi_{1}$.
\end{proof}

Using proposition~(\ref{logic:prop:FUAP:valuationmod:equivalence}),
we can now prove a sequent $\Gamma\vdash\phi$ with less work than
before. Until now, we needed to find a proof $\pi\in\pvs$ with
$\val(\pi)=\phi$ and $\hyp(\pi)\subseteq\Gamma$. We no longer need
to have the equality $\val(\pi)=\phi$ anymore. It is sufficient to
have $\vals(\pi)=\phi$. So let us go back to our initial example of
$\pi=\pi_{1}\pon\pi_{2}$ where $\val(\pi_{1})=\forall\,1\phi(1,1)$
and $\val(\pi_{2})=\forall\,0\phi(0,0)\to\phi(x,x)$. From the
equality $\vals(\pi)=\phi(x,x)$ we can now claim that $\phi(x,x)$ is
provable from $\Gamma=\hyp(\pi)$ without any further justification.
Without
proposition~(\ref{logic:prop:FUAP:valuationmod:equivalence}), we
would need to specialize the proof $\pi_{1}$ using the axiom of
first order logic $\forall\,1\phi(1,1)\to\phi(0,0)$ and generalize
with respect to $0$ so as to obtain:
\[
\Tree[.$\pon$ [.$\gen 0$ [.$\pon$ $\pi_{1}$
$\axi(\forall\,1\phi(1,1)\to\phi(0,0))$ ] ] $\pi_{2}$ ]
\]
which is a proof $\pi^{*}$ such that $\val(\pi^{*})=\phi(x,x)$
(assuming generalization in $0$ is legitimate). So
proposition~(\ref{logic:prop:FUAP:valuationmod:equivalence}) is an
improvement allowing us to establish provability with the leaner
proof $\pi_{1}\pon\pi_{2}$ instead of $\pi^{*}$. However, we can
still do better than this: if $\sim$ is the alpha equivalence
on \pv, we know from
proposition~(\ref{logic:prop:FOPL:HDC:sequent:stronger:congruence})
that the sequent $\Gamma\vdash\phi$ is true whenever
$\Gamma\vdash\psi$ and $\psi\sim\phi$. Hence the equality
$\vals(\pi)=\phi$ should not be required. It should be sufficient to
have $\vals(\pi)\sim\phi$. Similarly by virtue of the deduction
theorem~(\ref{logic:the:FOPL:deduction}) of
page~\pageref{logic:the:FOPL:deduction}, the inclusion
$\hyp(\pi)\subseteq\Gamma$ should not be required either. It should
be sufficient that every element of $\hyp(\pi)$ be equivalent to
some element of $\Gamma$. Thus, in order to establish the sequent
$\Gamma\vdash\phi$ it should be sufficient to find a proof
$\pi\in\pvs$ such that $\vals(\pi)\sim\phi$ and
$\hyp(\pi)\precsim\Gamma$, where $\precsim$ is an inclusion modulo.
We shall prove this fact in
theorem~(\ref{logic:the:FUAP:valuationmod:provability}) below.
Before we can do so, we shall make precise the notion of {\em
inclusion modulo}\,:

\index{modulo@Inclusion modulo subsets}
\begin{defin}\label{logic:def:FUAP:valuationmod:inclusion:modulo}
Let $V$ be a set and $\simeq$ be a congruence on \pv. We call {\em
inclusion modulo} associated with $\simeq$ the relation $\precsim$
on ${\cal P}(\pv)$ defined by the equivalence $\Gamma\precsim\Delta$
\ifand\ for all $\phi\in\Gamma$ there exists $\psi\in\Delta$ with
$\phi\simeq\psi$.
\end{defin}

The notion of inclusion modulo leads naturally to that of {\em
equality modulo} for subsets, which we shall use again in
proposition~(\ref{logic:prop:substitutiontheorem:hypothesis}). For
reference:

\index{modulo@Equality modulo subsets}
\begin{defin}\label{logic:def:FUAP:valuationmod:equality:modulo}
Let $V$ be a set and $\simeq$ be a congruence on \pv. We call {\em
equality modulo} associated with $\simeq$ the relation on ${\cal
P}(\pv)$ defined by:
    \[
    \Gamma\simeq\Delta\ \Leftrightarrow\
    (\Gamma\precsim\Delta)\land(\Delta\precsim\Gamma)
    \]
where $\precsim$ is the inclusion modulo associated with the
congruence $\simeq$.
\end{defin}

Before we can prove
theorem~(\ref{logic:the:FUAP:valuationmod:provability}), we shall
establish an easy consequence of
proposition~(\ref{logic:prop:FOPL:HDC:sequent:stronger:congruence})
and the transitivity of the consequence relation $\vdash$\,:

\begin{prop}\label{logic:prop:FUAP:valuationmod:hypothesis:modulo}
Let $V$ be a set. Let $\Gamma, \Delta\subseteq\pv$ and $\phi\in\pv$.
Then:
    \[
    (\Gamma\succsim\Delta)\land(\Delta\vdash\phi)\ \Rightarrow\
    \Gamma\vdash\phi
    \]
where $\precsim$ denotes the inclusion modulo the alpha equivalence on \pv.
\end{prop}
\begin{proof}
Using the transitivity of the consequence relation $\vdash$ in the
form of proposition~(\ref{logic:prop:FOPL:deduction:transitivity})
it is sufficient to prove that $\Gamma\vdash\psi$ for all
$\psi\in\Delta$. So let $\psi\in\Delta$. We need to show that
$\Gamma\vdash\psi$. However, from the assumption
$\Gamma\succsim\Delta$, there exists $\phi\in\Gamma$ such that
$\phi\sim\psi$, where $\sim$ is the alpha equivalence on \pv.
It is clear that $\Gamma\vdash\phi$. Hence from
proposition~(\ref{logic:prop:FOPL:HDC:sequent:stronger:congruence})
we obtain $\Gamma\vdash\psi$ as requested.
\end{proof}

We can now prove
theorem~(\ref{logic:the:FUAP:valuationmod:provability}) which is the
main result of this section, and provide the weakest conditions
allowing us to establish provability so far:

\index{provability@Provability criterion modulo}
\begin{theorem}\label{logic:the:FUAP:valuationmod:provability}
Let $V$ be a set. Let $\Gamma\subseteq\pv$ and $\phi\in\pv$. Then
the sequent $\Gamma\vdash\phi$ is true \ifand\ there exists a proof
$\pi\in\pvs$ such that:
    \[
    \vals(\pi)\sim\phi\mbox{\ and\ }\hyp(\pi)\precsim\Gamma
    \]
where $\precsim$ denotes the inclusion modulo alpha equivalence $\sim$ on \pv.
\end{theorem}
\begin{proof}
First we show the 'only if' part: so we assume that
$\Gamma\vdash\phi$. Using
proposition~(\ref{logic:prop:FUAP:valuationmod:equivalence}) it
follows that $\phi$ is provable modulo from $\Gamma$. Thus, there
exists a proof $\pi\in\pvs$ such that $\vals(\pi)=\phi$ and
$\hyp(\pi)\subseteq\Gamma$. In particular we have
$\vals(\pi)\sim\phi$ and $\hyp(\pi)\precsim\Gamma$ as requested. We
now prove the 'if' part: so we assume that $\pi\in\pvs$ is a proof
such that $\vals(\pi)\sim\phi$ and $\hyp(\pi)\precsim\Gamma$. We
need to show that $\Gamma\vdash\phi$. From
proposition~(\ref{logic:prop:FOPL:HDC:sequent:stronger:congruence})
it is sufficient to prove $\Gamma\vdash\vals(\pi)$. However, since
$\hyp(\pi)\precsim\Gamma$, using
proposition~(\ref{logic:prop:FUAP:valuationmod:hypothesis:modulo})
it is sufficient to show that $\hyp(\pi)\vdash\vals(\pi)$. Finally,
using proposition~(\ref{logic:prop:FUAP:valuationmod:equivalence})
we simply need to show that $\vals(\pi)$ is provable modulo from
$\hyp(\pi)$. This is clear from
definition~(\ref{logic:def:FUAP:valuationmod:proof:modulo}) since
$\pi$ is a proof modulo of $\vals(\pi)$ from $\hyp(\pi)$.
\end{proof}

We have now achieved the  aim motivating this section. We wanted to
change our deductive system by creating a valuation
$\vals:\pvs\to\pv$ which extends $\val:\pvs\to\pv$ from totally
clean proofs to a wider domain, thereby introducing flexibility and
awareness of $\alpha$-equivalence. This allowed us to prove
theorem~(\ref{logic:the:FUAP:valuationmod:provability}) providing a
convenient way to establish the truth of any sequent
$\Gamma\vdash\phi$. It will also allow us to define the minimal
transform ${\cal M}(\pi)$ of any proof $\pi\in\pvs$. Before we
complete this section, we shall establish a few elementary results
regarding the valuation $\vals$. We start with the fact that
variables of conclusions of sub-proofs are variables of the proof:

\begin{prop}\label{logic:prop:FUAP:varvalmod:conclusions:sub:proof}
Let $V$ be a set and $\pi\in\pvs$. Then we have:
    \begin{equation}\label{logic:eqn:FUAP:varvalmod:conclusions:1}
    \cup\{\,\var(\vals(\rho))\ :\
    \rho\preceq\pi\,\}\subseteq\var(\pi)
    \end{equation}
i.e. the variables of conclusions modulo of sub-proofs of $\pi$ are
variables of $\pi$.
\end{prop}
\begin{proof}
We shall prove
inclusion~(\ref{logic:eqn:FUAP:varvalmod:conclusions:1}) with a
structural induction argument, using
theorem~(\ref{logic:the:proof:induction}) of
page~\pageref{logic:the:proof:induction}. First we assume that
$\pi=\phi$ for some $\phi\in\pv$. From
definition~(\ref{logic:def:subformula}), the only sub-proof of $\pi$
is $\pi$ itself. Hence we need to show the inclusion
$\var(\vals(\pi))\subseteq\var(\pi)$ which follows from
$\var(\pi)=\var(\phi)$ and $\vals(\pi)=\phi$. Next we assume that
$\pi=\axi\phi$ for some $\phi\in\pv$. Then again the only sub-proof
of $\pi$ is $\pi$ itself and we need to show that
$\var(\vals(\pi))\subseteq\var(\pi)$. We shall distinguish two
cases: first we assume that $\phi\in\avs$. Then we have
$\vals(\pi)=\phi$ while $\var(\pi)=\var(\phi)$ so the inclusion is
clear. Next we assume that $\phi\not\in\avs$. Then
$\vals(\pi)=\bot\to\bot$ so $\var(\vals(\pi))=\emptyset$ and the
inclusion is also true. So we now assume that
$\pi=\pi_{1}\pon\pi_{2}$ where $\pi_{1},\pi_{2}\in\pvs$ are proofs
for which the
inclusion~(\ref{logic:eqn:FUAP:varvalmod:conclusions:1}) is true.
Then we have:
    \begin{eqnarray*}
    \var(\pi)&=&\var(\pi_{1}\pon\pi_{2})\\
    &=&\var(\pi_{1})\cup\var(\pi_{2})\\
    &\supseteq&(\cup\{\var(\vals(\rho)):\rho\preceq\pi_{1}\})
    \cup(\cup\{\var(\vals(\rho)):\rho\preceq\pi_{2}\})\\
    &=&\cup\{\var(\vals(\rho)):\rho\in\subf(\pi_{1})\cup\subf(\pi_{2})\}\\
    \mbox{A: below}\rightarrow\!\!\!\!
    &=&\cup\{\var(\vals(\rho)):\rho\in\subf(\pi_{1})\cup\subf(\pi_{2})
    \cup\{\pi_{1}\pon\pi_{2}\}\}\\
    \mbox{def.~(\ref{logic:def:subformula})}\ \rightarrow
    &=&\cup\{\var(\vals(\rho)):\rho\in\subf(\pi_{1}\pon\pi_{2})\}\\
    &=&\cup\{\var(\vals(\rho)):\rho\preceq\pi\}\\
    \end{eqnarray*}
So it remains to show point A: The inclusion $\subseteq$ is clear.
In order to show $\supseteq$ it is sufficient to show that
$\rho=\pi_{1}\pon\pi_{2}$ does not enlarge the set, that is:
    \[
    \var(\vals(\pi_{1}\pon\pi_{2}))\subseteq
    \cup\{\var(\vals(\rho)):\rho\in\subf(\pi_{1})\cup\subf(\pi_{2})\}
    \]
We shall distinguish two cases: first we assume that
$\vals(\pi_{1}\pon\pi_{2})=\bot\to\bot$. Then the inclusion is
clear. Next we assume that
$\vals(\pi_{1}\pon\pi_{2})\neq\bot\to\bot$. Then we must have
$\vals(\pi_{2})=\psi_{1}\to\psi_{2}$ for some $\psi_{1},\psi_{2}$
such that $\psi_{1}\sim\vals(\pi_{1})$ and
$\psi_{2}=\vals(\pi_{1}\pon\pi_{2})$, where $\sim$ is the
alpha equivalence. Consequently:
    \[
    \var(\vals(\pi_{1}\pon\pi_{2}))=\var(\psi_{2})\subseteq\var(\vals(\pi_{2}))
    \]
and the inclusion is also clear. So we now assume that $\pi=\gen
x\pi_{1}$ where $x\in V$ and $\pi_{1}\in\pvs$ is a proof for which
inclusion~(\ref{logic:eqn:FUAP:varvalmod:conclusions:1}) is true:
    \begin{eqnarray*}
    \var(\pi)&=&\var(\gen x\pi_{1})\\
    &=&\{x\}\cup\var(\pi_{1})\\
    &\supseteq&\{x\}\cup(\,\cup\{\var(\vals(\rho)):\rho\in\subf(\pi_{1})\}\,)\\
    \mbox{A: to be proved}\ \rightarrow&=&
    \cup\{\var(\vals(\rho)):\rho\in\subf(\pi_{1})\cup\{\gen x\pi_{1}\}\}\\
    \mbox{def.~(\ref{logic:def:subformula})}\ \rightarrow
    &=&\cup\{\var(\vals(\rho)):\rho\in\subf(\gen x\pi_{1})\}\\
    &=&\cup\{\var(\vals(\rho)):\rho\preceq\pi\}\\
    \end{eqnarray*}
So it remains to show point A. We shall distinguish two cases: first
we assume that $x\in\spec(\pi_{1})$. Then $\vals(\gen
x\pi_{1})=\bot\to\bot$ and $\var(\vals(\rho))=\emptyset$ in the case
when $\rho=\gen x\pi_{1}$. So the inclusion $\supseteq$ is clear. In
order to show $\subseteq$, it is sufficient to prove that
$x\in\var(\vals(\rho))$ for some $\rho\in\subf(\pi_{1})$. However,
since $x\in\spec(\pi_{1})$, from
definition~(\ref{logic:def:FOPL:proof:free:variable}) we have
$x\in\free(\phi)$ for some $\phi\in\hyp(\pi_{1})$. Defining
$\rho=\phi$ from
proposition~(\ref{logic:prop:FUAP:hypothesis:charac}) we have
$\rho\in\subf(\pi_{1})$ and furthermore since $\vals(\rho)=\phi$ and
$\free(\phi)\subseteq\var(\phi)$ we conclude that
$x\in\var(\vals(\rho))$. So we now assume that
$x\not\in\spec(\pi_{1})$. Then $\vals(\gen x\pi_{1})=\forall
x\vals(\pi_{1})$ and:
    \begin{equation}\label{logic:eqn:FUAP:varvalmod:conclusions:2}
    \var(\vals(\gen x\pi_{1}))=\{x\}\cup\var(\vals(\pi_{1}))
    \end{equation}
In particular, we see that $x\in\var(\vals(\rho))$ in the case when
$\rho=\gen x\pi_{1}$ and the inclusion $\subseteq$ is clear. In
order to show $\supseteq$ we simply need to prove:
    \[
    \var(\vals(\gen x\pi_{1}))\subseteq
    \{x\}\cup(\,\cup\{\var(\vals(\rho)):\rho\in\subf(\pi_{1})\}\,)
    \]
which follows immediately from
equation~(\ref{logic:eqn:FUAP:varvalmod:conclusions:2}) and the fact
that $\pi_{1}\in\subf(\pi_{1})$.
\end{proof}

The free variables of the conclusion modulo of a proof are free
variables of the proof itself. Remember that free variables of a
sub-proof may not be free variables of the proof. So although we can
claim that $\free(\vals(\rho))\subseteq\free(\rho)$ for every
sub-proof $\rho\preceq\pi$, the inclusion
$\free(\vals(\rho))\subseteq\free(\pi)$ is false in general.


\begin{prop}\label{logic:prop:FUAP:valuationmod:freevar:conclusion}
Let $V$ be a set and $\pi\in\pvs$. Then we have the inclusion:
    \[
    \free(\vals(\pi))\subseteq\free(\pi)
    \]
i.e. the free variables of the conclusion modulo of $\pi$ are free
variables of $\pi$.
\end{prop}
\begin{proof}
We shall prove this inclusion with a structural induction, using
theorem~(\ref{logic:the:proof:induction}) of
page~\pageref{logic:the:proof:induction}. First we assume that
$\pi=\phi$ for some $\phi\in\pv$. Then the inclusion follows
immediately from $\vals(\phi)=\phi$. Next we assume that
$\pi=\axi\phi$ for some $\phi\in\pv$. We need to show the inclusion
holds for $\pi$. This is clearly the case if
$\vals(\pi)=\bot\to\bot$. So we may assume that
$\vals(\pi)\neq\bot\to\bot$ in which case $\phi$ is an axiom modulo,
i.e. $\phi\in\avs$ and consequently $\vals(\pi)=\phi$. It follows
that $\free(\vals(\pi))=\free(\phi)=\free(\axi\phi)=\free(\pi)$ and
in particular the inclusion holds. So we now assume that
$\pi=\pi_{1}\pon\pi_{2}$ where $\pi_{1},\pi_{2}\in\pvs$ are proofs
satisfying the inclusion. We need to show the same is true of $\pi$.
This is clearly the case if $\vals(\pi)=\bot\to\bot$. So we may
assume that $\vals(\pi)\neq\bot\to\bot$ in which case
$\vals(\pi_{2})=\psi_{1}\to\psi_{2}$ for some
$\psi_{1},\psi_{2}\in\pv$ with $\psi_{1}\sim\vals(\pi_{1})$ and
$\psi_{2}=\vals(\pi)$, where $\sim$ denotes the alpha equivalence
on \pv\,:
    \begin{eqnarray*}
    \free(\vals(\pi))&=&\free(\psi_{2})\\
    &\subseteq&\free(\psi_{1})\cup\free(\psi_{2})\\
    &=&\free(\psi_{1}\to\psi_{2})\\
    &=&\free(\vals(\pi_{2}))\\
    &\subseteq&\free(\pi_{2})\\
    &\subseteq&\free(\pi_{1})\cup\free(\pi_{2})\\
    &=&\free(\pi_{1}\pon\pi_{2})\\
    &=&\free(\pi)
    \end{eqnarray*}
So we now assume that $\pi=\gen x\pi_{1}$ where $x\in V$ and
$\pi_{1}\in\pvs$ is a proof satisfying our inclusion. We need to
show the same is true of $\pi$. This is clearly the case if
$\vals(\pi)=\bot\to\bot$. So we may assume that
$\vals(\pi)\neq\bot\to\bot$ in which case we have
$x\not\in\spec(\pi_{1})$ and $\vals(\pi)=\forall x\vals(\pi_{1})$.
Hence:
    \begin{eqnarray*}
    \free(\vals(\pi))&=&\free(\forall x\vals(\pi_{1}))\\
    &=&\free(\vals(\pi_{1}))\setminus\{x\}\\
    &\subseteq&\free(\pi_{1})\setminus\{x\}\\
    &=&\free(\gen x\pi_{1})\\
    &=&\free(\pi)\\
    \end{eqnarray*}
\end{proof}

In definition~(\ref{logic:def:FUAP:validsubproof:validsub}) we
extended the notion of {\em valid substitution} from formulas to
proofs. We made sure every valid substitution would be valid for
every hypothesis and every axiom of a given proof. We also made sure
the substitution would avoid variable capture which may arise from
generalization. The following proposition may be seen as a
vindication of the valuation modulo $\vals:\pvs\to\pv$ as well as of
definition~(\ref{logic:def:FUAP:validsubproof:validsub}): every
valid substitution is in fact also valid for all intermediary
conclusions arising from the proof:


\begin{prop}\label{logic:prop:FUAP:valuationmod:valid:vals}
Let $V$, $W$ be sets and $\sigma:V\to W$ be a map. Let $\pi\in\pvs$.
Then if $\sigma$ is valid for $\pi$, for all $\rho\in\pvs$ we have
the implication:
    \[
    \rho\preceq\pi\ \Rightarrow\ \mbox{$\sigma$ valid for $\vals(\rho)$}
    \]
\end{prop}
\begin{proof}
Given $\pi\in\pvs$, it is sufficient to show the following
implication:
    \[
    (\mbox{$\sigma$ valid for $\pi$})\ \Rightarrow\ \mbox{$\sigma$ valid for
    $\vals(\pi)$}
    \]
Indeed, suppose we have done so. Then if $\sigma$ is valid for $\pi$
and $\rho\preceq\pi$, from
proposition~(\ref{logic:prop:FUAP:validsubproof:subformula})
$\sigma$ is also valid for $\rho$ and consequently from the
implication it is valid for $\vals(\rho)$. So the proposition is
proved. We shall prove the implication with a structural induction
argument, using theorem~(\ref{logic:the:proof:induction}) of
page~\pageref{logic:the:proof:induction}. First we assume that
$\pi=\phi$ for some $\phi\in\pv$. Then the implication follows
immediately from $\vals(\phi)=\phi$. Next we assume that
$\pi=\axi\phi$ for $\phi\in\pv$. We need to show the implication is
true for $\pi$. So we assume that $\sigma$ is valid for $\pi$. We
need to show it is valid for $\vals(\pi)$. This is obviously the
case when $\vals(\pi)=\bot\to\bot$. So we may assume that
$\vals(\pi)\neq\bot\to\bot$ in which case $\phi$ is an axiom modulo,
i.e. $\phi\in\avs$. It follows that $\vals(\pi)=\phi$ and we need to
show that $\sigma$ is valid for $\phi$. This follows from
proposition~(\ref{logic:prop:FUAP:validsubproof:recursion:axiom})
and the fact that $\sigma$ is valid for $\pi=\axi\phi$. So we now
assume that $\pi=\pi_{1}\pon\pi_{2}$ where $\pi_{1},\pi_{2}\in\pvs$
are proofs satisfying our implication. We need to show the same is
true of $\pi$. So we assume that $\sigma$ is valid for $\pi$. We
need to show that $\sigma$ is valid for $\vals(\pi)$. This is
obviously the case if $\vals(\pi)=\bot\to\bot$. So we may assume
that $\vals(\pi)\neq\bot\to\bot$ in which case we have
$\vals(\pi_{2})=\psi_{1}\to\psi_{2}$ where $\psi_{1},\psi_{2}\in\pv$
are such that $\psi_{1}\sim\vals(\pi_{1})$ and
$\psi_{2}=\vals(\pi)$, where $\sim$ denotes alpha equivalence
on \pv. So we need to show that $\sigma$ is valid for $\psi_{2}$.
Using proposition~(\ref{logic:prop:FOPL:valid:recursion:imp}), it is
sufficient to prove that $\sigma$ is valid for $\vals(\pi_{2})$.
Having assumed the implication is true for $\pi_{2}$, we simply need
to show that $\sigma$ is valid for $\pi_{2}$ which follows from
proposition~(\ref{logic:prop:FUAP:validsubproof:recursion:pon}) and
the validity of $\sigma$ for $\pi$. So we now assume that $\pi=\gen
x\pi_{1}$ where $x\in V$ and $\pi_{1}\in\pvs$ is a proof satisfying
our implication. We need to show the same is true of $\pi$. So we
assume that $\sigma$ is valid for $\pi$. We need to show it is valid
for $\vals(\pi)$. This is obviously the case if
$\vals(\pi)=\bot\to\bot$. So we may assume that
$\vals(\pi)\neq\bot\to\bot$ in which case $x\not\in\spec(\pi_{1})$
and $\vals(\pi)=\forall x\vals(\pi_{1})$. So we need to show that
$\sigma$ is valid for $\forall x\vals(\pi_{1})$. Using
proposition~(\ref{logic:prop:FOPL:valid:recursion:quant}), it is
sufficient to show that $\sigma$ is valid for $\vals(\pi_{1})$ and
furthermore given $u\in\free(\forall x\vals(\pi_{1}))$, that
$\sigma(u)\neq\sigma(x)$. However, using
proposition~(\ref{logic:prop:FUAP:valuationmod:freevar:conclusion})
we have $\free(\vals(\pi_{1}))\subseteq\free(\pi_{1})$ and so:
    \[
    \free(\forall
    x\vals(\pi_{1}))\subseteq\free(\pi_{1})\setminus\{x\}=\free(\gen
    x\pi_{1})
    \]
So it is sufficient to show the implication $u\in\free(\gen
x\pi_{1})\ \Rightarrow\ \sigma(u)\neq\sigma(x)$ which follows from
proposition~(\ref{logic:prop:FUAP:validsubproof:recursion:gen}) and
the validity of $\sigma$ for $\pi=\gen x\pi_{1}$. So it remains to
show that $\sigma$ is valid for $\vals(\pi_{1})$. Having assumed our
implication is true for $\pi_{1}$, we simply need to prove that
$\sigma$ is valid for $\pi_{1}$ which follows once again from
proposition~(\ref{logic:prop:FUAP:validsubproof:recursion:gen}) and
the validity of $\sigma$ for $\pi=\gen x\pi_{1}$.
\end{proof}
