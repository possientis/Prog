Using
proposition~(\ref{logic:prop:FUAP:validsubtotclean:valuation:commute}),
we obtained
proposition~(\ref{logic:prop:FUAP:validsubtotclean:sequent})
allowing us to carry over sequents from $\Gamma\vdash\phi$ to
$\sigma(\Gamma)\vdash\sigma(\phi)$ in the case when $\sigma:V\to W$
is an injective map. There is only so much we can do with
proposition~(\ref{logic:prop:FUAP:validsubtotclean:valuation:commute}).
The equality $\val\circ\sigma(\pi)=\sigma\circ\val(\pi)$ cannot be
inferred from it, unless $\sigma$ is valid for $\pi$. In general, we
have no way to tell whether a substitution $\sigma$ is valid for a
proof $\pi$ underlying the sequent $\Gamma\vdash\phi$. We do not
know which axioms are being used in the proof, simply by looking at
the sequent. We need a whole new strategy.

We have dealt with this problem before. Given a map $\sigma:V\to W$
and a formula $\phi\in\pv$, we wanted to define a formula
$\sigma(\phi)$ which is meaningful even in the case when $\sigma$ is
not valid for $\phi$. Our solution was to introduce the minimal
transform ${\cal M}(\phi)$ of
definition~(\ref{logic:def:FOPL:mintransform:transform}) and
consider the formula $\bar{\sigma}\circ{\cal M}(\phi)$ where all the
free variables of $\phi$ are carried over by $\sigma$ while avoiding
capture. Likewise, given a map $\sigma:V\to W$ which may not be
valid for the proof $\pi\in\pvs$, we want to define a proof
$\sigma(\pi)$ which is meaningful even when $\sigma$ is not valid
for $\pi$. By {\em meaningful}, it is understood that the key
equality $\val\circ\sigma(\pi)=\sigma\circ\val(\pi)$ should be
satisfied. Of course, we also need to say something about the set of
hypothesis $\hyp(\sigma(\pi))$ but this is usually a lot easier to
achieve. So here is the plan: we should define the minimal transform
${\cal M}(\pi)$ which replaces all the bound variables of $\pi$ by
elements of a copy of \N\ which is disjoint from $V$, and consider
the proof $\bar{\sigma}\circ{\cal M}(\pi)$ in the hope that such a
proof will have the right property, as the variable substitution
will avoid capture. Recall that $\bar{\sigma}:\bar{V}\to\bar{W}$ is
the minimal extension of $\sigma$ as per
definition~(\ref{logic:def:FOPL:commute:minextensioon:map}). When
attempting to define the minimal transform ${\cal M}(\pi)$, the
obvious choice appears to set ${\cal M}(\pi)={\cal M}(\phi)$
whenever $\pi=\phi$, ${\cal M}(\axi\phi)=\axi{\cal M}(\phi)$, ${\cal
M}(\pi_{1}\pon\pi_{2})={\cal M}(\pi_{1})\pon\,{\cal M}(\pi_{2})$
and:
    \[
    {\cal M}(\gen x\pi_{1})=\gen n{\cal M}(\pi_{1})[n/x]
    \]
where $n=\min\{k\in\N:\mbox{$[k/x]$ valid for ${\cal
M}(\pi_{1})$}\}$. So let us look at an example:
    \[
    \Tree[.$\gen y$ [.$\pon$ [.$\gen x$ $\axi\phi_{1}$ ]
    $\axi(\forall x\phi_{1}\to\phi_{1}[y/x])$ ] ]
    \]
where $\phi_{1}\in\av$ is any axiom of first order logic and
$[y/x]:\pv\to\pv$ is an essential substitution of $y$ in place of
$x$. To make our discussion simpler, let us define the map
$\phi:V\times V\to\av$ by setting $\phi(x,y)=\bot\to(\,(x\in
y)\to\bot\,)$ and pick $\phi_{1}=\phi(x,y)$ for some $x\neq y$. Then
$\phi_{1}[y/x]=\phi(y,y)$. Our proof is:
    \[
    \Tree[.$\gen y$ [.$\pon$ [.$\gen x$ $\axi\phi(x,y)$ ]
    $\axi(\forall x\phi(x,y)\to\phi(y,y))$ ] ]
    \]
which can be expressed as $\pi=\gen y[\,\gen
x\axi\phi(x,y)\pon\,\axi(\,\forall x\phi(x,y)\to\phi(y,y)\, )\,]$.
This is arguably less readable than the tree representation. So
${\cal M}(\pi)$ becomes:
    \[
    \Tree[.$\gen\,1$ [.$\pon$ [.$\gen\,0$ $\axi\phi(0,1)$ ]
    $\axi(\forall\,0\phi(0,1)\to\phi(1,1))$ ] ]
    \]
The key observation in this case is that $[0/y]$ is not valid for
the proof:
    \[
    \Tree[.$\pon$ [.$\gen\,0$ $\axi\phi(0,y)$ ]
    $\axi(\forall\,0\phi(0,y)\to\phi(y,y))$ ]
    \]
and consequently the variable $y$ had to be replaced by $1$ instead
of $0$. Now it appears that the conclusion of ${\cal M}(\pi)$ is the
formula $\val\circ{\cal M}(\pi)=\forall\,1\phi(1,1)$, while the
conclusion of $\pi$ is the formula $\val(\pi)=\forall y\phi(y,y)$.
It follows that ${\cal M}\circ\val(\pi)=\forall\,0\phi(0,0)$ and we
are met with the fact that:
    \[
    \val\circ{\cal M}(\pi)\neq{\cal M}\circ\val(\pi)
    \]
This is hugely disappointing. The conclusion of ${\cal M}(\pi)$ is
not what we want it to be. Our plan is going to fail. However, it
should be noted that $\forall\,0\phi(0,0)$ and $\forall\,1\phi(1,1)$
are equivalent modulo the substitution congruence $\sim$ and thus:
    \[
    \val\circ{\cal M}(\pi)\sim{\cal M}\circ\val(\pi)
    \]
It may be that this equivalence is true at all times. It may also be
sufficient for our purpose to carry over sequents
$\Gamma\vdash\phi$. Our tentative definition of minimal transform
for proofs may be appropriate after all. Unfortunately, this is
seemingly not the case: consider the proof
$\rho=\pi\pon\,\axi(\forall y\phi(y,y)\to\phi(x,x))$. From
$\val(\pi)=\forall y\phi(y,y)$ we obtain $\val(\rho)=\phi(x,x)$.
Hence ${\cal M}\circ\val(\rho)=\phi(x,x)$. In contrast we have
${\cal M}(\rho)={\cal
M}(\pi)\pon\,\axi(\forall\,0\phi(0,0)\to\phi(x,x))$. From the
equality $\val\circ{\cal M}(\pi)=\forall\,1\phi(1,1)$ we see that
${\cal M}(\rho)$ is not a legitimate application of the modus ponens
rule of inference and consequently $\val\circ{\cal
M}(\rho)=\bot\to\bot$. Thus, the equivalence $\val\circ{\cal
M}(\rho)\sim{\cal M}\circ\val(\rho)$ is not even satisfied.
