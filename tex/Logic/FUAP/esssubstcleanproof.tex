It is all very nice to have constructed essential substitutions
$\sigma:\pvs\to{\bf\Pi}(W)$ associated with $\sigma:V\to W$. So we
have a tool substituting variables in proofs while avoiding capture.
However, given $\pi\in\pvs$ we need to check that the conclusion and
hypothesis of $\sigma(\pi)$ are what we expect. This is the purpose
of the present section. With regards to the conclusion, we cannot
hope to say much about the valuation $\val\circ\sigma(\pi)$, but we
certainly should be able to prove that
$\vals\circ\sigma(\pi)\sim\sigma\circ\vals(\pi)$ where $\sim$ is the
alpha equivalence on ${\bf P}(W)$. Note that the '$\sigma$'
appearing in $\sigma\circ\vals(\pi)$ refers to the restriction of
the essential proof substitution $\sigma:\pvs\to{\bf\Pi}(W)$  to
$\pv$, and not to the brute force substitution $\sigma:\pv\to{\bf
P}(W)$ of definition~(\ref{logic:def:substitution}). It would not
make sense otherwise: we have no assumption about the validity of
$\sigma$ for $\pi$ or the validity of $\sigma$ for $\vals(\pi)$. The
context gives us an essential substitution
$\sigma:\pvs\to{\bf\Pi}(W)$ which can equally act on formulas since
$\pv\subseteq\pvs$. So it is clear from the context that the only
sensible interpretation of $\sigma\circ\vals(\pi)$ is to regard
$\sigma$ as essential. This is in contrast with similar results we
previously obtained in these notes: for $\pi$ totally clean and
$\sigma$ valid for $\pi$, we obtained
$\val\circ\sigma(\pi)=\sigma\circ\val(\pi)$ in
proposition~(\ref{logic:prop:FUAP:validsubtotclean:valuation:commute}).
For $\pi$ clean and $\sigma$ valid for $\pi$ we obtained
$\vals\circ\sigma(\pi)=\sigma\circ\vals(\pi)$ in
proposition~(\ref{logic:prop:FUAP:strongvalsubalmostclean:valuation:commute}).
In both cases, the right-hand-side '$\sigma$' was the naive
substitution of definition~(\ref{logic:def:substitution}). One
question we may ask is whether we should hope to have
$\vals\circ\sigma(\pi)=\sigma\circ\vals(\pi)$ rather than a mere
$\alpha$-equivalence. The answer is no. By virtue of
proposition~(\ref{logic:prop:FUAP:esssubstprop:redefine}), an
essential proof substitution $\sigma:\pvs\to{\bf \Pi}(W)$ can be
redefined modulo alpha equivalence, and still qualify as
an essential substitution associated to the same $\sigma:V\to W$. So
if $\vals\circ\sigma(\pi)=\sigma\circ\vals(\pi)$ happened to be
true, we could redefine $\sigma$ on \pv\ alone, without changing
$\sigma(\pi)$.


\index{essential@Essential image of clean
proof}\index{clean@Essential image of clean proof}
\begin{prop}\label{logic:prop:FUAP:esssubstcleanproof:main}
Let $V$, $W$ be sets and $\sigma:\pvs\to{\bf\Pi}(W)$ be an essential
proof substitution. Let $\pi\in\pvs$ be a clean proof. Then
$\sigma(\pi)$ is clean and:
    \begin{equation}\label{logic:eqn:FUAP:esssubstcleanproof:main:1}
    \vals\circ\sigma(\pi)\sim\sigma\circ\vals(\pi)
    \end{equation}
where $\sigma:\pv\to{\bf P}(W)$ is the restriction and $\sim$ is the
alpha equivalence.
\end{prop}
\begin{proof}
Before we start, it should be noted that the map $\sigma:\pv\to{\bf
P}(W)$ on the right-hand-side
of~(\ref{logic:eqn:FUAP:esssubstcleanproof:main:1}) is simply the
restriction of $\sigma:\pvs\to{\bf\Pi}(W)$ and not the substitution
associated with $\sigma:V\to W$ as per
definition~(\ref{logic:def:substitution}). We know from
proposition~(\ref{logic:prop:FUAP:esssubst:proof:to:formula}) that
$\sigma:\pv\to{\bf P}(W)$ is an essential substitution associated
with the same $\sigma:V\to W$ as is $\sigma:\pvs\to{\bf\Pi}(W)$. We
now assume that $\pi$ is a clean proof. First we show that
$\sigma(\pi)$ is clean. From
proposition~(\ref{logic:prop:FUAP:proofwithcleanMT:mintrans:clean:equivalence})
it is sufficient to show that ${\cal M}\circ\sigma(\pi)$ is clean.
Having assumed $\sigma:\pvs\to{\bf\Pi}(W)$ is an essential proof
substitution, we have ${\cal
M}\circ\sigma(\pi)=\bar{\sigma}\circ{\cal M}(\pi)$. So we need to
show that $\bar{\sigma}\circ{\cal M}(\pi)$ is clean. However, from
proposition~(\ref{logic:prop:FUAP:proofwithcleanMT:mintrans:clean:equivalence})
we know that ${\cal M}(\pi)$ is clean, while from
proposition~(\ref{logic:prop:FUAP:mintransformproof:minextension:valid})
the minimal extension $\bar{\sigma}:\bar{V}\to\bar{W}$ is valid for
${\cal M}(\pi)$. It follows from
proposition~(\ref{logic:prop:FUAP:strongvalsubalmostclean:valuation:commute})
that $\bar{\sigma}\circ{\cal M}(\pi)$ is clean as requested. It
remains to prove the equivalence
$\vals\circ\sigma(\pi)\sim\sigma\circ\vals(\pi)$. Using
theorem~(\ref{logic:the:FOPL:mintransfsubcong:kernel}) of
page~\pageref{logic:the:FOPL:mintransfsubcong:kernel} it is
sufficient to prove the equality between minimal transforms ${\cal
M}\circ\vals\circ\sigma(\pi)={\cal M}\circ\sigma\circ\vals(\pi)$. In
fact, from
proposition~(\ref{logic:prop:FOPL:esssubst:mintransform:equiv:imp:equal})
we simply need to show the equivalence ${\cal
M}\circ\vals\circ\sigma(\pi)\sim{\cal M}\circ\sigma\circ\vals(\pi)$
where $\sim$ denotes the alpha equivalence on ${\bf
P}(\bar{W})$. Having established that $\sigma(\pi)$ is clean, using
proposition~(\ref{logic:prop:FUAP:mintransproof:valuation:commute})
we obtain the following:
    \begin{eqnarray*}
    {\cal M}\circ\vals\circ\sigma(\pi)&\sim&\vals\circ{\cal
    M}\circ\sigma(\pi)\\
    \mbox{$\sigma$ essential}\ \rightarrow
    &=&\vals\circ\bar{\sigma}\circ{\cal M}(\pi)\\
    \mbox{prop.~(\ref{logic:prop:FUAP:strongvalsubalmostclean:valuation:commute})}\ \rightarrow
    &=&\bar{\sigma}\circ\vals\circ{\cal M}(\pi)\\
    \mbox{A: to be proved}\ \rightarrow
    &\sim&\bar{\sigma}\circ{\cal M}\circ\vals(\pi)\\
    \mbox{$\sigma$ essential}\ \rightarrow
    &=&{\cal M}\circ\sigma\circ\vals(\pi)\\
    \end{eqnarray*}
So it remains to prove the equivalence
$\bar{\sigma}\circ\vals\circ{\cal M}(\pi)\sim\bar{\sigma}\circ{\cal
M}\circ\vals(\pi)$. Using
theorem~(\ref{logic:the:FOPL:mintransfsubcong:valid}) of
page~\pageref{logic:the:FOPL:mintransfsubcong:valid}, it is
sufficient to show $\vals\circ{\cal M}(\pi)\sim{\cal
M}\circ\vals(\pi)$ and furthermore that $\bar{\sigma}$ is valid for
both formulas. The equivalence follows from $\pi$ being clean and
proposition~(\ref{logic:prop:FUAP:mintransproof:valuation:commute}).
The fact that $\bar{\sigma}$ is valid for ${\cal M}\circ\vals(\pi)$
follows from
proposition~(\ref{logic:def:FOPL:commute:minextension:valid}). The
fact that $\bar{\sigma}$ is valid for $\vals\circ{\cal M}(\pi)$
follows from
proposition~(\ref{logic:prop:FUAP:valuationmod:valid:vals}) and the
validity of $\bar{\sigma}$ for ${\cal M}(\pi)$, which is itself a
consequence of
proposition~(\ref{logic:prop:FUAP:mintransformproof:minextension:valid})
and which completes our proof.
\end{proof}

We shall now check that $\hyp(\sigma(\pi))$ is also what we expect,
namely the image
$\sigma(\hyp(\pi))=\{\sigma(\phi):\phi\in\hyp(\pi)\}$. Once again,
the '$\sigma$' involved in this last expression can only be the
essential $\sigma$, not the substitution $\sigma:\pv\to{\bf P}(W)$
of definition~(\ref{logic:def:substitution}). Since $\sigma$ can be
arbitrarily redefined modulo alpha equivalence, we cannot
hope to have an exact equality
$\hyp(\sigma(\pi))=\sigma(\hyp(\pi))$. The best we can hope for is
that every element of $\hyp(\sigma(\pi))$ be $\alpha$-equivalent to
an element of $\sigma(\hyp(\pi))$ and conversely. We should also
remember that no sensible result can be obtained unless $\pi$ is a
clean proof. For example, consider $V=\{x,y\}$ with $x\neq y$,
$\pi=\gen x(x\in x)$ and $\sigma:\pvs\to\pvs$ an an essential
substitution associated with the identity mapping $i:V\to V$. Since
$\pi\sim\rho$ where $\rho=\gen y(y\in y)$, we can redefine $\sigma$
so as to have $\sigma(\pi)=\rho$. It is clear the equality modulo
$\hyp(\sigma(\pi))\sim\sigma(\hyp(\pi))$ is false in this case.

\index{hypothesis@Hypothesis of essential
image}\index{essential@Hypothesis of essential image}
\begin{prop}\label{logic:prop:substitutiontheorem:hypothesis}
Let $V$, $W$ be sets and $\sigma:\pvs\to{\bf\Pi}(W)$ be an essential
proof substitution. Let $\pi\in\pvs$ be a clean proof. Then we have:
    \begin{equation}\label{logic:eqn:substitutiontheorem:hypothesis:1}
    \hyp(\sigma(\pi))\sim\sigma(\hyp(\pi))
    \end{equation}
where $\sim$ is the equality modulo alpha equivalence as
per {\em
definition~(\ref{logic:def:FUAP:valuationmod:equality:modulo})}.
\end{prop}
\begin{proof}
Before we start, it should be noted that the map $\sigma:\pv\to{\bf
P}(W)$ on the right-hand-side
of~(\ref{logic:eqn:substitutiontheorem:hypothesis:1}) is simply the
restriction of $\sigma:\pvs\to{\bf\Pi}(W)$ and not the substitution
associated with $\sigma:V\to W$ as per
definition~(\ref{logic:def:substitution}). We know from
proposition~(\ref{logic:prop:FUAP:esssubst:proof:to:formula}) that
the map $\sigma:\pv\to{\bf P}(W)$ is an essential substitution
associated with the same $\sigma:V\to W$ as is
$\sigma:\pvs\to{\bf\Pi}(W)$. We now assume that $\pi$ is a clean
proof. We need to show that any formula of $\hyp(\sigma(\pi))$ is
substitution equivalent to a formula of $\sigma(\hyp(\pi))$ and
conversely that any formula of $\sigma(\hyp(\pi))$ is substitution
equivalent to a formula of $\hyp(\sigma(\pi))$. Using
theorem~(\ref{logic:the:FOPL:mintransfsubcong:kernel}) of
page~\pageref{logic:the:FOPL:mintransfsubcong:kernel} it is
therefore sufficient to prove ${\cal M}(\hyp(\sigma(\pi)))={\cal
M}(\sigma(\hyp(\pi)))$. Using
proposition~(\ref{logic:prop:FUAP:esssubstcleanproof:main}), the
proof $\sigma(\pi)$ is clean and consequently from
proposition~(\ref{logic:prop:FUAP:mintransproof:hypothesis}) we
obtain:
    \begin{eqnarray*}
    {\cal M}(\hyp(\sigma(\pi)))&=&\hyp({\cal M}\circ\sigma(\pi))\\
    \mbox{$\sigma$ essential}\ \rightarrow
    &=&\hyp(\bar{\sigma}\circ{\cal M}(\pi))\\
    \mbox{prop.~(\ref{logic:prop:FUAP:substitution:hypothesis})}\ \rightarrow
    &=&\bar{\sigma}(\,\hyp({\cal M}(\pi))\,)\\
    \mbox{prop.~(\ref{logic:prop:FUAP:mintransproof:hypothesis}), $\pi$ clean}\ \rightarrow
    &=&\bar{\sigma}(\,{\cal M}(\hyp(\pi))\,)\\
    &=&\bar{\sigma}\circ{\cal M}(\hyp(\pi))\\
    \mbox{$\sigma$ essential}\ \rightarrow
    &=&{\cal M}\circ\sigma(\hyp(\pi))\\
    &=&{\cal M}(\sigma(\hyp(\pi)))\\
    \end{eqnarray*}
\end{proof}
