Having established
proposition~(\ref{logic:prop:FUAP:proofwithcleanMT:mintrans:clean:equivalence})
in the previous section, we are now in a position to check that
$\alpha$-equivalence between proofs is a sensible notion. This means
that equivalent proofs prove the same thing from the same
hypothesis, modulo the substitution congruence, at least when one of
them is clean. When this is the case, then both proofs are in fact
clean. Note that $\pi\sim\rho$ may not imply
$\vals(\pi)\sim\vals(\rho)$ if $\pi$ and $\rho$ are not clean
proofs. As already pointed out on several occasions, dealing with
proofs which are not clean is typically a waste of time as nothing
sensible can be said about them. For example, suppose $x\neq y$ and
consider $\pi_{1}=\gen x(x\in x)$ and $\rho_{1}=\gen y(y\in y)$.
These proofs are not clean and we have
$\vals(\pi_{1})=\bot\to\bot=\vals(\rho_{1})$. However, from the
equivalence $\pi_{1}\sim\rho_{1}$ we see that $\pi\sim\rho$ where
$\pi=\gen x\pi_{1}$ and $\rho=\gen x\rho_{1}$. Furthermore, since
$x\in\spec(\pi_{1})=\{x\}$ we have $\vals(\pi)=\bot\to\bot$, while
from $x\not\in\spec(\rho_{1})=\{y\}$ we obtain $\vals(\rho)=\forall
x\vals(\rho_{1})=\forall x(\bot\to\bot)$. From
theorem~(\ref{logic:the:sub:congruence:charac}) of
page~\pageref{logic:the:sub:congruence:charac}, the formulas
$\forall x(\bot\to\bot)$ and $\bot\to\bot$ are not
$\alpha$-equivalent. Thus we see that it is possible to have the
equivalence $\pi\sim\rho$ and $\vals(\pi)\not\sim\vals(\rho)$. As
the following proposition shows, this possibility is removed by
assuming one of the proofs $\pi$ or $\rho$ is a clean proof.

\index{valuation@Valuation of equivalent proofs}
\begin{prop}\label{logic:prop:FUAP:valmodsubcong:main}
Let $V$ be a set and $\pi\in\pvs$ be a clean proof. Then for all
$\rho\in\pvs$, if we have the equivalence $\pi\sim\rho$ then $\rho$
is itself clean and:
    \[
    \vals(\pi)\sim\vals(\rho)
    \]
where $\sim$ denotes the substitution congruence both on \pv\ and
\pvs.
\end{prop}
\begin{proof}
We assume that $\pi$ is clean and $\pi\sim\rho$. We need to show
that $\rho$ is clean and $\vals(\pi)\sim\vals(\rho)$. First we show
that $\rho$ is clean: from
theorem~(\ref{logic:the:FUAP:mintransfsubcong:kernel}) of
page~\pageref{logic:the:FUAP:mintransfsubcong:kernel} we obtain
${\cal M}(\pi)={\cal M}(\rho)$. Furthermore since $\pi$ is clean,
from
proposition~(\ref{logic:prop:FUAP:proofwithcleanMT:mintrans:clean:equivalence})
it follows that ${\cal M}(\pi)$ is clean. Hence we see that ${\cal
M}(\rho)$ is clean and using
proposition~(\ref{logic:prop:FUAP:proofwithcleanMT:mintrans:clean:equivalence})
once more we conclude that $\rho$ is a clean proof. In order to show
the equivalence $\vals(\pi)\sim\vals(\rho)$, it is sufficient to
show the equality ${\cal M}\circ\vals(\pi)={\cal M}\circ\vals(\rho)$
by virtue of theorem~(\ref{logic:the:FOPL:mintransfsubcong:kernel})
of page~\pageref{logic:the:FOPL:mintransfsubcong:kernel}. In fact,
using
proposition~(\ref{logic:prop:FOPL:esssubst:mintransform:equiv:imp:equal}),
it is sufficient to show the equivalence ${\cal
M}\circ\vals(\pi)\sim{\cal M}\circ\vals(\rho)$ where $\sim$ now
denotes the substitution congruence on \pvb. Since $\pi$ and $\rho$
are clean proofs, using
proposition~(\ref{logic:prop:FUAP:mintransproof:valuation:commute})
we obtain:
    \begin{eqnarray*}
    {\cal M}\circ\vals(\pi)&\sim&\vals\circ{\cal M}(\pi)\\
    &=&\vals\circ{\cal M}(\rho)\\
    \mbox{prop.~(\ref{logic:prop:FUAP:mintransproof:valuation:commute})}\ \rightarrow
    &\sim&{\cal M}\circ\vals(\rho)\\
    \end{eqnarray*}
\end{proof}

We complete this section by showing the hypothesis of equivalent
proofs are the same, modulo the substitution congruence, provided
these proofs are clean. For a counter-example involving equivalent
proofs which are not clean, simply consider $\pi=\gen x(x\in x)$ and
$\rho=\gen y(y\in y)$ where $x\neq y$.

\begin{prop}\label{logic:prop:FUAP:valmodsubcong:hypothesis}
Let $V$ be a set and $\pi\in\pvs$ be a clean proof. Then for all
$\rho\in\pvs$, if we have the substitution equivalence $\pi\sim\rho$
then we also have:
    \[
    \hyp(\pi)\sim\hyp(\rho)
    \]
where $\sim$ is the equality modulo the substitution congruence as
per {\em
definition~(\ref{logic:def:FUAP:valuationmod:equality:modulo})}.
\end{prop}
\begin{proof}
If $\pi$ is clean and $\pi\sim\rho$ we know that $\rho$ is clean
from proposition~(\ref{logic:prop:FUAP:valmodsubcong:main}). We have
to show that the equality $\hyp(\pi)\sim\hyp(\rho)$ modulo the
substitution congruence is true. From
definition~(\ref{logic:def:FUAP:valuationmod:equality:modulo}), we
need to prove the inclusions modulo $\hyp(\pi)\precsim\hyp(\rho)$
and $\hyp(\rho)\precsim\hyp(\pi)$, and it is clearly sufficient to
focus on one of these inclusions. So let $\phi\in\hyp(\pi)$. From
definition~(\ref{logic:def:FUAP:valuationmod:inclusion:modulo}), we
need to show the existence of $\psi\in\hyp(\rho)$ such that
$\phi\sim\psi$. However, since $\pi$ is a clean proof, from
proposition~(\ref{logic:prop:FUAP:mintransproof:hypothesis}) we have
$\hyp({\cal M}(\pi))={\cal M}(\hyp(\pi))$. It follows that ${\cal
M}(\phi)\in\hyp({\cal M}(\pi))$. Having assumed that $\pi\sim\rho$,
from theorem~(\ref{logic:the:FUAP:mintransfsubcong:kernel}) of
page~\pageref{logic:the:FUAP:mintransfsubcong:kernel} we obtain
${\cal M}(\pi)={\cal M}(\rho)$ and consequently ${\cal
M}(\phi)\in\hyp({\cal M}(\rho))$. However, $\rho$ is also a clean
proof and we have $\hyp({\cal M}(\rho))={\cal M}(\hyp(\rho))$. It
follows that ${\cal M}(\phi)\in{\cal M}(\hyp(\rho))$ and we see that
there exists $\psi\in\hyp(\rho)$ such that ${\cal M}(\phi)={\cal
M}(\psi)$ which is $\phi\sim\psi$ by virtue of
theorem~(\ref{logic:the:FOPL:mintransfsubcong:kernel}) of
page~\pageref{logic:the:FOPL:mintransfsubcong:kernel}.
\end{proof}
