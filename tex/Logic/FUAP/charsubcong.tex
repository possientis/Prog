The purpose of this section is to provide a characterization of 
alpha equivalence on \pvs\ similar to
theorem~(\ref{logic:the:sub:congruence:charac}) of
page~\pageref{logic:the:sub:congruence:charac} for \pv. The result
will be proved in theorem~(\ref{logic:the:FUAP:charsubcong:charac})
below. This type of theorem is very useful as it confirms the
natural belief that alpha equivalence can only occur between
proofs which are {\em structurally identical}. So if $\equiv$ and
$\sim$ denote the alpha equivalence on \pvs\ and \pv\
respectively, then the equivalence $\pi\equiv\rho$ can only arise
when $\pi$ and $\rho$ are both elements of \pv\ with $\pi\sim\rho$,
or both 'axioms' with $\pi=\axi\phi$, $\rho=\axi\psi$ and
$\phi\sim\psi$, or both of the form $\pi=\pi_{1}\pon\pi_{2}$,
$\rho=\rho_{1}\pon\rho_{2}$ and $\pi_{1}\equiv\rho_{1}$,
$\pi_{2}\equiv\rho_{2}$ etc. We start by showing that equivalent
proofs have the same free variables, which is the counterpart of
proposition~(\ref{logic:prop:sub:congruence:freevar}).

\begin{prop}\label{logic:prop:FUAP:charsubcong:freevar}
Let $\equiv$ denote the alpha equivalence on \pvs\ where $V$
is a set. Then for all $\pi,\rho\in\pvs$ we have the implication:
    \[
    \pi\equiv\rho\ \Rightarrow\ \free(\pi)=\free(\rho)
    \]
\end{prop}
\begin{proof}
Let $\simeq$ be the relation on \pvs\ defined by $\pi\simeq\rho\
\Leftrightarrow\ \free(\pi)=\free(\rho)$. We need to show that
$\pi\equiv\rho\ \Rightarrow\ \pi\simeq\rho$ or equivalently that the
inclusion $\equiv\,\subseteq\,\simeq$ holds. Since $\equiv$ is the
alpha equivalence on \pvs, it is the smallest congruence on
\pvs\ which contains the sets $R_{0}$, $R_{1}$ and $R_{2}$ of
definition~(\ref{logic:def:FUAP:subcong:substitution:congruence}).
In order to show the inclusion $\equiv\,\subseteq\,\simeq$ it is
therefore sufficient to show that $\simeq$ is a congruence on \pvs\
such that $R_{i}\subseteq\,\simeq$ for $i\in 3$. However, we already
know from
proposition~(\ref{logic:prop:FUAP:freevarproof:congruence}) that
$\simeq$ is a congruence on \pvs. So it remains to show that
$R_{i}\subseteq\,\simeq$ for $i\in 3$. First we show that
$R_{0}\subseteq\,\simeq$\,. So let $\phi,\psi\in\pv$ such that
$\phi\sim\psi$ where $\sim$ denotes the alpha equivalence on
\pv. We need to show that $\phi\simeq\psi$ i.e. that
$\free(\phi)=\free(\psi)$ which follows from
proposition~(\ref{logic:prop:sub:congruence:freevar}). We now show
that $R_{1}\subseteq\,\simeq$\,. So let $\phi,\psi\in\pv$ such that
$\phi\sim\psi$. We need to show that $\axi\phi\simeq\axi\psi$ which
is $\free(\axi\phi)=\free(\axi\psi)$ or equivalently
$\free(\phi)=\free(\psi)$. Once again, this follows from
proposition~(\ref{logic:prop:sub:congruence:freevar}). So we now
show that $R_{2}\subseteq\,\simeq$\,. Let $\pi_{1}\in\pvs$ and
$x,y\in V$ such that $x\neq y$ and $y\not\in\free(\pi_{1})$. Define
$\pi=\gen x\pi_{1}$ and $\rho=\gen y\pi_{1}[y\!:\!x]$. We need to
show that $\pi\simeq\rho$, i.e. that $\free(\pi)=\free(\rho)$.
However, from
proposition~(\ref{logic:prop:FUAP:validsubproof:injective}), the map
$[y\!:\!x]$ is valid for $\pi_{1}$. Hence:
    \begin{eqnarray*}
    \free(\rho)&=&\free(\,\gen y\pi_{1}[y\!:\!x]\,)\\
    &=&\free(\pi_{1}[y\!:\!x])\setminus\{y\}\\
    \mbox{prop.~(\ref{logic:prop:FUAP:validsubproof:freevar})}\ \rightarrow
    &=&[y\!:\!x](\free(\pi_{1}))\setminus\{y\}\\
    y\not\in\free(\pi_{1})\ \rightarrow
    &=&[y\!:\!x](\,\free(\pi_{1})\setminus\{y\}\,)\setminus\{y\}\\
    \mbox{$[y\!:\!x](x)=y$}\ \rightarrow
    &=&[y\!:\!x](\,\free(\pi_{1})\setminus\{x,y\}\,)\setminus\{y\}\\
    \mbox{$u\not\in\{x,y\}\Rightarrow[y\!:\!x](u)=u$}\ \rightarrow
    &=&(\free(\pi_{1})\setminus\{x,y\})\setminus\{y\}\\
    &=&\free(\pi_{1})\setminus\{x,y\}\\
    y\not\in\free(\pi_{1})\ \rightarrow
    &=&\free(\pi_{1})\setminus\{x\}\\
    &=&\free(\gen x\pi_{1})\\
    &=&\free(\pi)\\
    \end{eqnarray*}
\end{proof}

The following proposition is the counterpart of
proposition~(\ref{logic:prop:FOPL:freesubcong:xy:not:free})\,:
\begin{prop}\label{logic:prop:FUAP:charsubcong:xy:not:free}
Let $\equiv$ denote the alpha equivalence on \pvs\ where $V$
is a set. Then for all $\pi_{1}\in\pvs$ and $x,y\in V$ such that
$x,y\not\in\free(\pi_{1})$ we have:
    \[
    \gen x\pi_{1}\equiv\gen y\pi_{1}
    \]
\end{prop}
\begin{proof}
From $y\not\in\free(\pi_{1})$ and
definition~(\ref{logic:def:FUAP:subcong:substitution:congruence}) we
see that $\gen x\pi_{1}\equiv\gen y\pi_{1}[y\!:\!x]$. So we need to
show that $\gen y\pi_{1}[y\!:\!x]\equiv\forall y\pi_{1}$. Hence it
is sufficient to prove that $\pi_{1}[y\!:\!x]\equiv\pi_{1}$. Using
proposition~(\ref{logic:prop:FUAP:subcong:admissible:subcong}), we
simply need to argue that $[y\!:\!x]$ is an admissible substitution
for $\pi_{1}$. Being injective, from
proposition~(\ref{logic:prop:FUAP:validsubproof:injective}) it is a
valid substitution for $\pi_{1}$. So it remains to show that
$[y\!:\!x](u)=u$ for all $u\in\free(\pi_{1})$ which follows
immediately from $x,y\not\in\free(\pi_{1})$.
\end{proof}


An injective substitution preserves $\alpha$-equivalence. This is of
course a temporary results which will be extended from injective to
valid substitutions in
theorem~(\ref{logic:the:FUAP:mintransfsubcong:valid}) of
page~\pageref{logic:the:FUAP:mintransfsubcong:valid}. The following
is the counterpart of
proposition~(\ref{logic:prop:sub:congruence:injective:substitution})\,:
\begin{prop}\label{logic:prop:FUAP:charsubcong:injective:substitution}
Let $V$ and $W$ be sets and $\sigma:V\to W$ be an injective map.
Let~$\equiv$ be the alpha equivalence both on \pvs\ and
${\bf\Pi}(W)$. Then:
    \[
    \pi\equiv\rho\ \Rightarrow\ \sigma(\pi)\equiv\sigma(\rho)
    \]
for all $\pi,\rho\in\pvs$, where $\sigma:\pvs\to{\bf\Pi}(W)$ is also
the substitution mapping.
\end{prop}
\begin{proof}
Let $\simeq$ be the relation on \pvs\ defined by $\pi\simeq\rho\
\Leftrightarrow\ \sigma(\pi)\equiv\sigma(\rho)$. We need to show
that $\pi\equiv\rho\ \Rightarrow\ \pi\simeq\rho$ or equivalently
that the inclusion $\equiv\,\subseteq\,\simeq$ holds. Since $\equiv$
is the alpha equivalence on \pvs, it is the smallest
congruence on \pvs\ which contains the sets $R_{0}$, $R_{1}$ and
$R_{2}$ of
definition~(\ref{logic:def:FUAP:subcong:substitution:congruence}).
In order to show the inclusion $\equiv\,\subseteq\,\simeq$ it is
therefore sufficient to show that $\simeq$ is a congruence on \pvs\
such that $R_{i}\subseteq\,\simeq$ for $i\in 3$. However, we already
know from
proposition~(\ref{logic:prop:FUAP:substitution:congruence}) that
$\simeq$ is a congruence on \pvs. So it remains to show that
$R_{i}\subseteq\,\simeq$ for $i\in 3$. First we show that
$R_{0}\subseteq\,\simeq$\,. So let $\phi,\psi\in\pv$ such that
$\phi\sim\psi$, where $\sim$ denotes the alpha equivalence on
\pv. We need to show that $\phi\simeq\psi$ or equivalently that
$\sigma(\phi)\equiv\sigma(\psi)$. Looking at
definition~(\ref{logic:def:FUAP:subcong:substitution:congruence}) it
is sufficient to show that $\sigma(\phi)\sim\sigma(\psi)$ where
$\sim$ now denotes the alpha equivalence on ${\bf P}(W)$.
However, the equivalence $\sigma(\phi)\sim\sigma(\psi)$ follows
immediately from the injectivity of $\sigma$ and
proposition~(\ref{logic:prop:sub:congruence:injective:substitution}).
We now show that $R_{1}\subseteq\,\simeq$\,. So let
$\phi,\psi\in\pv$ such that $\phi\sim\psi$. We need to show that
$\axi\phi\simeq\axi\psi$ or equivalently
$\sigma(\axi\phi)\equiv\sigma(\axi\psi)$ which is
$\axi\sigma(\phi)\equiv\axi\sigma(\psi)$. Once again from
definition~(\ref{logic:def:FUAP:subcong:substitution:congruence}) it
is sufficient to show that $\sigma(\phi)\sim\sigma(\psi)$ which
follows from the injectivity of $\sigma$. We now show that
$R_{2}\subseteq\,\simeq$\,. So let $\pi_{1}\in\pvs$ and $x,y\in V$
be such that $x\neq y$ and $y\not\in\free(\pi_{1})$. Define
$\pi=\gen x\pi_{1}$ and $\rho=\gen y\,\pi_{1}[y\!:\!x]$. We need to
show that $\pi\simeq\rho$ or equivalently that
$\sigma(\pi)\equiv\sigma(\rho)$. In order to do so, it is sufficient
to show that the ordered pair $(\sigma(\pi),\sigma(\rho))$ belongs
to the set $R_{2}'$ of the alpha equivalence on ${\bf\Pi}(W)$
as per
definition~(\ref{logic:def:FUAP:subcong:substitution:congruence}).
In other words, it is sufficient to show the existence of
$\pi_{1}'\in{\bf\Pi}(W)$ and $x',y'\in W$ with $x'\neq y'$ and
$y'\not\in\free(\pi_{1}')$, such that $\sigma(\pi)=\gen x'\pi_{1}'$
and $\sigma(\rho)=\gen y'\,\pi_{1}'[y'\!:\!x']$. Take
$\pi_{1}'=\sigma(\pi_{1})\in{\bf\Pi}(W)$ together with
$x'=\sigma(x)\in W$ and $y'=\sigma(y)\in W$. Then:
    \[
    \sigma(\pi)=\sigma(\gen x\pi_{1})\\
        =\gen\sigma(x)\,\sigma(\pi_{1})\\
        =\gen x'\pi_{1}'
    \]
Furthermore, from
proposition~(\ref{logic:prop:FUAP:singlevar:composition:injective})
we have $\sigma\circ[y\!:\!x]=[\sigma(y)\!:\!\sigma(x)]\circ\sigma$
and so:
    \begin{eqnarray*}
    \sigma(\rho)&=&\sigma(\gen y\,\pi_{1}[y\!:\!x])\\
    &=&\gen\sigma(y)\,\sigma(\pi_{1}[y\!:\!x])\\
    &=&\gen y'\,\sigma\circ [y\!:\!x]\,(\pi_{1})\\
    &=&\gen y'\, [\sigma(y)\!:\!\sigma(x)]\circ\sigma\,(\pi_{1})\\
    &=&\gen y'\,[y'\!:\!x'](\pi_{1}')\\
    &=&\gen y'\,\pi_{1}'[y'\!:\!x']
    \end{eqnarray*}
So it remains to show that $x'\neq y'$ and
$y'\not\in\free(\pi_{1}')$. Since $\sigma:V\to W$ is an injective
map, $x'\neq y'$ follows immediately from $x\neq y$. We now show
that $y'\not\in\free(\pi_{1}')$. So suppose to the contrary that
$y'\in\free(\pi_{1}')$. We shall arrive at a contradiction. Since
$\pi_{1}'=\sigma(\pi_{1})$, from
proposition~(\ref{logic:prop:FUAP:freevarproof:substitution:inclusion})
there exists $u\in\free(\pi_{1})$ such that $y'=\sigma(u)$. However,
$y'=\sigma(y)$ and $\sigma:V\to W$ is an injective map. Hence we see
that $u=y$ and consequently $y\in\free(\pi_{1})$ which contradicts
our initial assumption of $y\not\in\free(\pi_{1})$ and completes our
proof.
\end{proof}


We shall now follow a familiar strategy leading to
theorem~(\ref{logic:the:FUAP:charsubcong:charac}) of
page~\pageref{logic:the:FUAP:charsubcong:charac}. We shall first
define a new relation~$\simeq$ on \pvs\ reflecting our best estimate
of what the alpha equivalence~$\equiv$ should be like. We
shall then show that $\simeq$ is in fact a congruence which is
indeed equal to~$\equiv$ as expected. Note that some of the
mathematical statements of the following definition have been
shortened in the interest of readability. So in the case of $(iii)$,
what is meant is 'there exist $\pi_{1}$, $\pi_{2}$ and $\rho_{1}$,
$\rho_{2}$ such that $\pi=\pi_{1}\pon\pi_{2}$,
$\rho=\rho_{1}\pon\rho_{2}$, $\pi_{1}\equiv\rho_{1}$ and
$\pi_{2}\equiv\rho_{2}$' etc.


\begin{defin}\label{logic:def:FUAP:charsubcong:almost:equivalent}
Let $\equiv$ and $\sim$ be the alpha equivalence on \pvs\ and
\pv\ respectively where $V$ is a set. Let $\pi,\rho\in\pvs$. We say
that $\pi$ is {\em almost equivalent to $\rho$} denoted
$\pi\simeq\rho$, \ifand\ one of the following is the case:
    \begin{eqnarray*}
    (i)&&\pi=\phi\ ,\ \rho=\psi\ ,\ \mbox{for some}\ \phi,\psi\in\pv\ \mbox{and}\ \phi\sim\psi\\
    (ii)&&\pi=\axi\phi\ ,\ \rho=\axi\psi\ ,\ \mbox{for some}\ \phi,\psi\in\pv\ \mbox{and}\ \phi\sim\psi\\
    (iii)&&\pi=\pi_{1}\pon\pi_{2}\ ,\ \rho=\rho_{1}\pon\rho_{2}\ ,\
    \pi_{1}\equiv\rho_{1}\ \mbox{and}\ \pi_{2}\equiv\rho_{2}\\
    (iv)&&\pi=\gen x\pi_{1}\ ,\ \rho=\gen x\rho_{1}\ \mbox{and}\ \pi_{1}\equiv\rho_{1}\\
    (v)&&\pi=\gen x\pi_{1}\ ,\ \rho=\gen y\rho_{1}\ ,\ x\neq y\ ,\
    \rho_{1}\equiv \pi_{1}[y\!:\!x]\ ,\ y\not\in\free(\pi_{1})
    \end{eqnarray*}
\end{defin}

\begin{prop}
$(i),(ii),(iii),(iv), (v)$ of {\em
def.~(\ref{logic:def:FUAP:charsubcong:almost:equivalent})} are
mutually exclusive.
\end{prop}
\begin{proof}
This is an immediate consequence of
theorem~(\ref{logic:the:unique:representation}) of
page~\pageref{logic:the:unique:representation} applied to the free
universal algebra \pvs\ with free generator \pv, where a proof
$\pi\in\pvs$ is either an element of \pv, or an element of the form
$\pi=\axi\phi$ for some $\phi\in\pv$, or the result of a modus
ponens application $\pi=\pi_{1}\pon\pi_{2}$, or a generalization
$\pi=\gen x\pi_{1}$, but cannot be equal to any two of those things
simultaneously. Since $(v)$ can only occur with $x\neq y$, it also
follows from theorem~(\ref{logic:the:unique:representation}) that
$(v)$ cannot occur at the same time as $(iv)$ which completes our
proof.
\end{proof}

\begin{prop}\label{logic:prop:FUAP:charsubcong:almost:contains:r0:r1:r2}
Let $\simeq$ be the almost equivalence relation on \pvs\ where $V$
is a set. Then $\simeq$ contains the sets $R_{0}$, $R_{1}$ and
$R_{2}$ of {\em
definition~(\ref{logic:def:FUAP:subcong:substitution:congruence})}.
\end{prop}
\begin{proof}
First we show that $R_{0}\subseteq\,\simeq$\,. So let
$\phi,\psi\in\pv$ such that $\phi\sim\psi$. We need to show that
$\phi\simeq\psi$ which is clear from $(i)$ of
definition~(\ref{logic:def:FUAP:charsubcong:almost:equivalent}).
Next we show that $R_{1}\subseteq\,\simeq$\,. So let
$\phi,\psi\in\pv$ such that $\phi\sim\psi$. We need to show that
$\axi\phi\simeq\axi\psi$ which is clear from $(ii)$ of
definition~(\ref{logic:def:FUAP:charsubcong:almost:equivalent}). We
finally show $R_{3}\subseteq\,\simeq$\,. So we assume $x,y\in V$ and
$\pi_{1}\in\pvs$ are such that $x\neq y$ and
$y\not\in\free(\pi_{1})$. Note that this cannot happen unless $V$
has at least two elements. We define $\pi=\gen x\pi_{1}$ and
$\rho=\gen y\,\pi_{1}[y\!:\!x]$. We need to show that
$\pi\simeq\rho$. We shall do so by proving that $(v)$ of
definition~(\ref{logic:def:FUAP:charsubcong:almost:equivalent}) is
the case. Define $\rho_{1}=\pi_{1}[y\!:\!x]$. Then we have $\pi=\gen
x\pi_{1}$, $\rho=\gen y\rho_{1}$, $x\neq y$ and
$y\not\in\free(\pi_{1})$. So it remains to show that
$\rho_{1}\equiv\pi_{1}[y\!:\!x]$ which is immediate from the
reflexivity of $\equiv$\,.
\end{proof}

\begin{prop}\label{logic:prop:FUAP:subcong:almost:reflexive}
The almost equivalence relation on \pvs\ is reflexive.
\end{prop}
\begin{proof}
Let $\pi\in\pvs$. We need to show that $\pi\simeq\pi$. From
theorem~(\ref{logic:the:unique:representation}) of
page~\pageref{logic:the:unique:representation} we know that $\pi$ is
either an element of \pv, or $\pi=\axi\phi$ for some $\phi\in\pv$ or
$\pi=\pi_{1}\pon\pi_{2}$ or $\pi=\gen x\pi_{1}$ for some
$\pi_{1},\pi_{2}\in\pv$ and $x\in V$. In every one of these cases,
it is clear that $\pi\simeq\pi$ from the reflexivity of $\equiv$ and
$\sim$\,.
\end{proof}

\begin{prop}\label{logic:prop:FUAP:subcong:almost:symmetric}
The almost equivalence relation on \pvs\ is symmetric.
\end{prop}
\begin{proof}
Let $\pi,\rho\in\pvs$ be such that $\pi\simeq\rho$. We need to show
that $\rho\simeq\pi$. We shall consider the five possible cases of
definition~(\ref{logic:def:FUAP:charsubcong:almost:equivalent}): it
is clear that $\rho\simeq\pi$ is true in cases $(i)-(iv)$ from the
symmetry of $\equiv$ and $\sim$\,. So we consider the last possible
case of $\pi=\gen x\pi_{1}$, $\rho=\gen y\rho_{1}$ with $x\neq y$,
$\rho_{1}\equiv\pi_{1}[y\!:\!x]$ and $y\not\in\free(\pi_{1})$. We
need to show that $\pi_{1}\equiv\rho_{1}[x\!:\!y]$ and
$x\not\in\free(\rho_{1})$. First we show that
$\pi_{1}\equiv\rho_{1}[x\!:\!y]$. Note that $[x\!:\!y]$ and
$[y\!:\!x]$ are in fact the same substitutions. So we need to show
that $\pi_{1}\equiv\rho_{1}[y\!:\!x]$. Since
$\rho_{1}\equiv\pi_{1}[y\!:\!x]$ and $[y\!:\!x]:V\to V$ is
injective, from
proposition~(\ref{logic:prop:FUAP:charsubcong:injective:substitution})
we obtain $\rho_{1}[y\!:\!x]\equiv\pi_{1}[y\!:\!x][y\!:\!x]$. It is
therefore sufficient to show that
$\pi_{1}=\pi_{1}[y\!:\!x][y\!:\!x]$ which follows from
proposition~(\ref{logic:prop:FUAP:substitution:identity}) and the
fact that $[y\!:\!x]\circ[y\!:\!x]$ is the identity mapping. We now
show that $x\not\in\free(\rho_{1})$. From
$\rho_{1}\equiv\pi_{1}[y\!:\!x]$ and
proposition~(\ref{logic:prop:FUAP:charsubcong:freevar}) we obtain
$\free(\rho_{1})=\free(\pi_{1}[y\!:\!x])$. So we need to show that
$x\not\in\free(\pi_{1}[y\!:\!x])$. So suppose to the contrary that
$x\in\free(\pi_{1}[y\!:\!x])$. From
proposition~(\ref{logic:prop:FUAP:freevarproof:substitution:inclusion})
we have $\free(\pi_{1}[y\!:\!x])\subseteq[y\!:\!x](\free(\pi_{1}))$
and consequently there exists $u\in\free(\pi_{1})$ such that
$x=[y\!:\!x](u)$. By injectivity, It follows that $u=y$ which
contradicts the assumption $y\not\in\free(\pi_{1})$.
\end{proof}

As is now usual, establishing transitivity is the hardest part:

\begin{prop}\label{logic:prop:FUAP:subcong:almost:transitive}
The almost equivalence relation on \pvs\ is transitive.
\end{prop}
\begin{proof}
Let $\pi,\rho$ and $\kappa\in\pvs$ be such that $\pi\simeq\rho$ and
$\rho\simeq\kappa$. We need to show that $\pi\simeq\kappa$. We shall
consider the five possible cases of
definition~(\ref{logic:def:FUAP:charsubcong:almost:equivalent}) in
relation to $\pi\simeq\rho$. Suppose first that $\pi=\phi$ and
$\rho=\psi$ for some $\phi,\psi\in\pv$ with $\phi\sim\psi$. Then
from $\rho\simeq\kappa$ we obtain that $\kappa=\chi$ for some
$\chi\in\pv$ and $\psi\sim\chi$. It follows that $\pi=\phi$,
$\kappa=\chi$ with $\phi\sim\chi$. Hence we see that
$\pi\simeq\kappa$ as requested. We now assume that $\pi=\axi\phi$
and $\rho=\axi\psi$ for some $\phi,\psi\in\pv$ with $\phi\sim\psi$.
Then from $\rho\simeq\kappa$ we obtain that $\kappa=\axi\chi$ for
some $\chi\in\pv$ and $\psi\sim\chi$. It follows that
$\pi=\axi\phi$, $\kappa=\axi\chi$ with $\phi\sim\chi$. Once again we
see that $\pi\simeq\kappa$ as requested. We now assume that
$\pi=\pi_{1}\pon\pi_{2}$ and $\rho=\rho_{1}\pon\rho_{2}$ with
$\pi_{1}\equiv\rho_{1}$ and $\pi_{2}\equiv\rho_{2}$. From
$\rho\simeq\kappa$ we obtain $\kappa=\kappa_{1}\pon\kappa_{2}$ with
$\rho_{1}\equiv\kappa_{1}$ and $\rho_{2}\equiv\kappa_{2}$. It
follows that $\pi_{1}\equiv\kappa_{1}$ and
$\pi_{2}\equiv\kappa_{2}$. Hence we see that $\pi\simeq\kappa$ as
requested. We now assume that $\pi=\gen x\pi_{1}$ and $\rho=\gen
x\rho_{1}$ with $\pi_{1}\equiv\rho_{1}$, for some $x\in V$. From
$\rho\simeq\kappa$ only the cases $(iv)$ and $(v)$ of
definition~(\ref{logic:def:FUAP:charsubcong:almost:equivalent}) are
possible. First we assume that $(iv)$ is the case. Then $\kappa=\gen
x\kappa_{1}$ with $\rho_{1}\equiv\kappa_{1}$. Hence we obtain
$\pi_{1}\equiv\kappa_{1}$ and consequently $\pi\simeq\kappa$ as
requested. We now assume that $(v)$ is the case. Then $\kappa=\gen
y\kappa_{1}$ for some $y\in V$ with $x\neq y$,
$\kappa_{1}\equiv\rho_{1}[y\!:\!x]$ and $y\not\in\free(\rho_{1})$.
In order to prove $\pi\simeq\kappa$ it is sufficient to show that
$\kappa_{1}\equiv\pi_{1}[y\!:\!x]$ and $y\not\in\free(\pi_{1})$.
First we show that $\kappa_{1}\equiv\pi_{1}[y\!:\!x]$. It is
sufficient to prove that $\pi_{1}[y\!:\!x]\equiv\rho_{1}[y\!:\!x]$
which follows from $\pi_{1}\equiv\rho_{1}$, using
proposition~(\ref{logic:prop:FUAP:charsubcong:injective:substitution})
and the fact that $[y\!:\!x]:V\to V$ is injective. We now show that
$y\not\in\free(\pi_{1})$. From $\pi_{1}\equiv\rho_{1}$ and
proposition~(\ref{logic:prop:FUAP:charsubcong:freevar}) we obtain
$\free(\pi_{1})=\free(\rho_{1})$. Hence it is sufficient to show
that $y\not\in\free(\rho_{1})$ which is true by assumption. It
remains to consider the last possible case of
definition~(\ref{logic:def:FUAP:charsubcong:almost:equivalent}). So
we assume that $\pi=\gen x\pi_{1}$ and $\rho=\gen y\rho_{1}$ with
$x\neq y$, $\rho_{1}\equiv\pi_{1}[y\!:\!x]$ and
$y\not\in\free(\pi_{1})$. From $\rho\simeq\kappa$ only the cases
$(iv)$ and $(v)$ of
definition~(\ref{logic:def:FUAP:charsubcong:almost:equivalent}) are
possible. First we assume that $(iv)$ is the case. Then $\kappa=\gen
y\kappa_{1}$ with $\rho_{1}\equiv\kappa_{1}$. Then, we obtain
$\kappa_{1}\equiv\pi_{1}[y\!:\!x]$ and consequently
$\pi\simeq\kappa$. We now assume that $(v)$ is the case. Then
$\kappa=\gen z\kappa_{1}$ for some $z\in V$ with $y\neq z$,
$\kappa_{1}\equiv\rho_{1}[z\!:\!y]$ and $z\not\in\free(\rho_{1})$.
We shall now distinguish two cases. First we assume that $x=z$. Then
$\pi=\gen x\pi_{1}$ and $\kappa=\gen x\kappa_{1}$ and in order to
show that $\pi\simeq\kappa$ it is sufficient to prove that
$\pi_{1}\equiv\kappa_{1}$. From $\kappa_{1}\equiv\rho_{1}[z\!:\!y]$
and $z=x$ we obtain $\kappa_{1}\equiv\rho_{1}[y\!:\!x]$ and it is
therefore sufficient to prove that $\pi_{1}\equiv\rho_{1}[y\!:\!x]$.
However, we know that $\rho_{1}\equiv\pi_{1}[y\!:\!x]$ and since
$[y\!:\!x]:V\to V$ is injective, from
proposition~(\ref{logic:prop:FUAP:charsubcong:injective:substitution})
we obtain
$\rho_{1}[y\!:\!x]\equiv\pi_{1}[y\!:\!x][y\!:\!x]=\pi_{1}$. This
completes our proof in the case when $x=z$. We now assume that
$x\neq z$. So we have $x\neq y$, $y\neq z$ and $x\neq z$, with
$\pi=\gen x\pi_{1}$, $\rho=\gen y\rho_{1}$ and $\kappa=\gen
z\kappa_{1}$. Furthermore, $\rho_{1}\equiv\pi_{1}[y\!:\!x]$ and
$\kappa_{1}\equiv\rho_{1}[z\!:\!y]$ while we have
$y\not\in\free(\pi_{1})$ and $z\not\in\free(\rho_{1})$, and we need
to show that $\pi\simeq\kappa$. So we need to prove that
$\kappa_{1}\equiv\pi_{1}[z\!:\!x]$ and $z\not\in\free(\pi_{1})$.
First we show that $z\not\in\free(\pi_{1})$. So suppose to the
contrary that $z\in\free(\pi_{1})$. We shall derive a contradiction.
Since $[y\!:\!x]$ is injective it is valid for $\pi_{1}$ and from
proposition~(\ref{logic:prop:FUAP:validsubproof:freevar}) we have
$\free(\pi_{1}[y\!:\!x]) = [y\!:\!x](\free(\pi_{1}))$. It follows
that $z=[y\!:\!x](z)$ is also an element of
$\free(\pi_{1}[y\!:\!x])$. However we have
$\rho_{1}\equiv\pi_{1}[y\!:\!x]$ and consequently from
proposition~(\ref{logic:prop:FUAP:charsubcong:freevar}) we obtain
$\free(\rho_{1})=\free(\pi_{1}[y\!:\!x])$. Hence we see that
$z\in\free(\rho_{1})$ which is our desired contradiction. We shall
now prove that $\kappa_{1}\equiv\pi_{1}[z\!:\!x]$. Since
$\kappa_{1}\equiv\rho_{1}[z\!:\!y]$, it is sufficient to show that
$\rho_{1}[z\!:\!y]\equiv\pi_{1}[z\!:\!x]$. However we know that
$\rho_{1}\equiv\pi_{1}[y\!:\!x]$ and since $[z\!:\!y]:V\to V$ is
injective, from
proposition~(\ref{logic:prop:FUAP:charsubcong:injective:substitution})
we obtain $\rho_{1}[z\!:\!y]\equiv\pi_{1}[y\!:\!x][z\!:\!y]$. It is
therefore sufficient to show that
$\pi_{1}[y\!:\!x][z\!:\!y]\equiv\pi_{1}[z\!:\!x]$. Let us accept for
now:
    \begin{equation}\label{logic:eqn:FUAP:subcong:gen:transitive:z:x}
    [z\!:\!x]=[y\!:\!x]\circ[z\!:\!y]\circ[y\!:\!x]
    \end{equation}
Then we simply need to show that
$\pi_{1}[y\!:\!x][z\!:\!y]\equiv\pi_{1}[y\!:\!x][z\!:\!y][y\!:\!x]$.
Using
proposition~(\ref{logic:prop:FUAP:subcong:admissible:subcong}), it
is therefore sufficient to prove that $[y\!:\!x]$ is an admissible
substitution for $\pi_{1}[y\!:\!x][z\!:\!y]$. Since $[y\!:\!x]:V\to
V$ is injective, from
proposition~(\ref{logic:prop:FUAP:validsubproof:injective}) it is
valid for $\pi_{1}[y\!:\!x][z\!:\!y]$. So it remains to show that
$[y\!:\!x](u)=u$ for all $u\in\free(\pi_{1}[y\!:\!x][z\!:\!y])$. It
is therefore sufficient to prove that neither $x$ nor $y$ are
elements of $\free(\pi_{1}[y\!:\!x][z\!:\!y])$. However, from
proposition~(\ref{logic:prop:FUAP:freevarproof:substitution:inclusion})
we have
$\free(\pi_{1}[y\!:\!x][z\!:\!y])\subseteq[z\!:\!y]\circ[y\!:\!x](\free(\pi_{1}))$.
So it is sufficient to show that $x$ and $y$ do not belong to
$[z\!:\!y]\circ[y\!:\!x](\free(\pi_{1}))$. First we do this for $x$.
Suppose $x=[z\!:\!y]\circ[y\!:\!x](u)$ for some
$u\in\free(\pi_{1})$. By injectivity we must have $u=y$,
contradicting the assumption $y\not\in\free(\pi_{1})$. We now deal
with $y$. So suppose $y=[z\!:\!y]\circ[y\!:\!x](u)$ for some
$u\in\free(\pi_{1})$. By injectivity we must have $u=z$,
contradicting the fact that $z\not\in\free(\pi_{1})$ which we have
already proven. It remains to prove that
equation~(\ref{logic:eqn:FUAP:subcong:gen:transitive:z:x}) holds. So
let $u\in V$. We need:
    \[
    [z\!:\!x](u)=[y\!:\!x]\circ[z\!:\!y]\circ[y\!:\!x](u)
    \]
This is the case when $u\not\in\{x,y,z\}$. The cases $u=x$, $u=y$
and $u=z$ are easy.
\end{proof}

The almost equivalence is stronger than the alpha equivalence:

\begin{prop}\label{logic:prop:FUAP:subcong:almost:implies:sub:congruence}
Let $\simeq$ be the almost equivalence and $\equiv$ be the
alpha equivalence on \pvs, where $V$ is a set. Then for all
$\pi,\rho\in\pvs$ we have:
    \[
    \pi\simeq\rho\ \Rightarrow\ \pi\equiv\rho
    \]
\end{prop}
\begin{proof}
Let $\pi,\rho\in\pvs$ such that $\pi\simeq\rho$. We need to show
that $\pi\equiv\rho$. We shall consider the five possible cases of
definition~(\ref{logic:def:FUAP:charsubcong:almost:equivalent}) in
relation to $\pi\simeq\rho$. Suppose first that $\pi=\phi$ and
$\rho=\psi$ for some $\phi,\psi\in\pv$ with $\phi\sim\psi$. Then the
ordered pair $(\pi,\rho)$ belongs to the set $R_{0}$ of
definition~(\ref{logic:def:FUAP:subcong:substitution:congruence})
and in particular $\pi\equiv\rho$. Suppose next that $\pi=\axi\phi$
and $\rho=\axi\psi$ for some $\phi,\psi\in\pv$ with $\phi\sim\psi$.
Then the ordered pair $(\pi,\rho)$ belongs to the set $R_{1}$ of
definition~(\ref{logic:def:FUAP:subcong:substitution:congruence})
and in particular $\pi\equiv\rho$. We now assume that
$\pi=\pi_{1}\pon\pi_{2}$ and $\rho=\rho_{1}\pon\rho_{2}$ where
$\pi_{1}\equiv\rho_{1}$ and $\pi_{2}\equiv\rho_{2}$. The
alpha equivalence being a congruent relation on \pvs, we
obtain $\pi\equiv\rho$. Next we assume that $\pi=\gen x\pi_{1}$ and
$\rho=\gen x\rho_{1}$ where $\pi_{1}\equiv\rho_{1}$ and $x\in\ V$.
Again, the alpha equivalence being a congruent relation we
obtain $\pi\equiv\rho$. Finally we assume that $\pi=\gen x\pi_{1}$
and $\rho=\gen y\rho_{1}$ where $x\neq y$,
$\rho_{1}\equiv\pi_{1}[y\!:\!x]$ and $y\not\in\free(\pi_{1})$. For
the last time, the alpha equivalence being a congruent
relation we obtain $\rho\equiv\gen y\, \pi_{1}[y\!:\!x]$. Hence in
order to show $\pi\equiv\rho$ it is sufficient to show that $\gen
x\pi_{1}\equiv\gen y\,\pi_{1}[y\!:\!x]$ which follows immediately
from $x\neq y$, $y\not\in\free(\pi_{1})$ and the definition of
$R_{2}$ of
definition~(\ref{logic:def:FUAP:subcong:substitution:congruence}).
\end{proof}

\begin{prop}\label{logic:prop:FUAP:subcong:almost:congruent}
The almost equivalence relation on \pvs\ is congruent.
\end{prop}
\begin{proof}
From proposition~(\ref{logic:prop:FUAP:subcong:almost:reflexive}),
the almost equivalence $\simeq$ is reflexive and so
$\axi\phi\simeq\axi\phi$ for all $\phi\in\pv$. We now assume that
$\pi=\pi_{1}\pon\pi_{2}$ and $\rho=\rho_{1}\pon\rho_{2}$ where
$\pi_{1}\simeq\rho_{1}$ and $\pi_{2}\simeq\rho_{2}$. We need to show
that $\pi\simeq\rho$. However from
proposition~(\ref{logic:prop:FUAP:subcong:almost:implies:sub:congruence})
we obtain $\pi_{1}\equiv\rho_{1}$ and $\pi_{2}\equiv\rho_{2}$ and it
follows from
definition~(\ref{logic:def:FUAP:charsubcong:almost:equivalent}) that
$\pi\simeq\rho$. We now assume that $\pi=\gen x\pi_{1}$ and
$\rho=\gen x\rho_{1}$ where $\pi_{1}\simeq\rho_{1}$ and $x\in V$. We
need to show that $\pi\simeq\rho$. Once again from
proposition~(\ref{logic:prop:FUAP:subcong:almost:implies:sub:congruence})
we have $\pi_{1}\equiv\rho_{1}$ and consequently from
definition~(\ref{logic:def:FUAP:charsubcong:almost:equivalent}) we
obtain $\pi\simeq\rho$ as requested.
\end{proof}

\begin{prop}\label{logic:prop:FUAP:subcong:almost:congruence}
The almost equivalence relation on \pvs\ is a congruence.
\end{prop}
\begin{proof}
We need to show that $\simeq$ is reflexive, symmetric, transitive
and that it is a congruent relation on \pvs. From
proposition~(\ref{logic:prop:FUAP:subcong:almost:reflexive}), the
relation~$\simeq$ is reflexive. From
proposition~(\ref{logic:prop:FUAP:subcong:almost:symmetric}) it is
symmetric while from
proposition~(\ref{logic:prop:FUAP:subcong:almost:transitive}) it is
transitive. Finally from
proposition~(\ref{logic:prop:FUAP:subcong:almost:congruent}) the
relation~$\simeq$ is a congruent relation.
\end{proof}


\begin{prop}\label{logic:prop:FUAP:subcong:almost:is:sub:congruence}
Let $\simeq$ be the almost equivalence and $\equiv$ be the
alpha equivalence on \pvs, where $V$ is a set. For all
$\pi,\rho\in\pvs$ we have:
    \[
    \pi\simeq\rho\ \Leftrightarrow\ \pi\equiv\rho
    \]
\end{prop}
\begin{proof}
From
proposition~(\ref{logic:prop:FUAP:subcong:almost:implies:sub:congruence})
it is sufficient to show the implication $\Leftarrow$ or
equivalently the inclusion $\equiv\,\subseteq\,\simeq\,$. Since
$\equiv$ is the alpha equivalence on \pvs, it is the smallest
congruence on \pvs\ which contains the sets $R_{0}$, $R_{1}$ and
$R_{2}$ of
definition~(\ref{logic:def:FUAP:subcong:substitution:congruence}).
In order to show the inclusion $\equiv\,\subseteq\,\simeq$ it is
therefore sufficient to show that $\simeq$ is a congruence on \pvs\
such that $R_{i}\subseteq\,\simeq$ for $i\in 3$. The fact that it is
a congruence stems from
proposition~(\ref{logic:prop:FUAP:subcong:almost:congruence}). The
fact that $R_{i}\subseteq\,\simeq$ for $i\in 3$  follows from
proposition~(\ref{logic:prop:FUAP:charsubcong:almost:contains:r0:r1:r2}),
which completes our proof.
\end{proof}

The almost equivalence~$\simeq$ can now be forgotten. We obtain:

\index{congruence@Charact. of sub. congruence}
\begin{theorem}\label{logic:the:FUAP:charsubcong:charac}
Let $\equiv$ be the alpha equivalence on \pvs\ where $V$ is a
set. For all $\pi,\rho\in\pvs$, $\pi\equiv\rho$ \ifand\ one of the
following is the case:
    \begin{eqnarray*}
    (i)&&\pi=\phi\ ,\ \rho=\psi\ ,\ \mbox{for some}\ \phi,\psi\in\pv\ \mbox{and}\ \phi\sim\psi\\
    (ii)&&\pi=\axi\phi\ ,\ \rho=\axi\psi\ ,\ \mbox{for some}\ \phi,\psi\in\pv\ \mbox{and}\ \phi\sim\psi\\
    (iii)&&\pi=\pi_{1}\pon\pi_{2}\ ,\ \rho=\rho_{1}\pon\rho_{2}\ ,\
    \pi_{1}\equiv\rho_{1}\ \mbox{and}\ \pi_{2}\equiv\rho_{2}\\
    (iv)&&\pi=\gen x\pi_{1}\ ,\ \rho=\gen x\rho_{1}\ \mbox{and}\ \pi_{1}\equiv\rho_{1}\\
    (v)&&\pi=\gen x\pi_{1}\ ,\ \rho=\gen y\rho_{1}\ ,\ x\neq y\ ,\
    \rho_{1}\equiv \pi_{1}[y\!:\!x]\ ,\ y\not\in\free(\pi_{1})
    \end{eqnarray*}
where $\sim$ denotes the alpha equivalence on \pv.
\end{theorem}
\begin{proof}
Immediately follows from
proposition~(\ref{logic:prop:FUAP:subcong:almost:is:sub:congruence})
and definition~(\ref{logic:def:FUAP:charsubcong:almost:equivalent}).
\end{proof}

We can now prove what we claimed at the beginning of this section:
if $\sim$ and $\equiv$ denote the alpha equivalence on \pv\
and \pvs\ respectively, then $\phi\sim\psi$ is equivalent to
$\phi\equiv\psi$. It will no longer be necessary for us to use two
different notations for the alpha equivalence, and we shall
stick to the familiar $\sim$ going forward. When confronted with the
equivalence $\phi\sim\psi$, we do not need to worry as to whether
$\phi$, $\psi$ are regarded as formulas or proofs.

\begin{prop}\label{logic:prop:FUAP:charsubsong:equivalence:formula}
Let $V$ be a set and $\pi=\phi$, $\rho=\psi$ for some
$\phi,\psi\in\pv$:
    \[
    \pi\equiv\rho\ \Leftrightarrow\ \phi\sim\psi
    \]
where $\equiv$ and $\sim$ are the alpha equivalence on \pvs\
and \pv\ respectively.
\end{prop}
\begin{proof}
First we show $\Leftarrow$\,: So we assume that $\phi\sim\psi$. Then
the ordered pair $(\pi,\rho)$ is an element of the set $R_{0}$ of
definition~(\ref{logic:def:FUAP:subcong:substitution:congruence}).
In particular we have $\pi\equiv\rho$. We now prove $\Rightarrow$\,:
so we assume that $\pi\equiv\rho$. Using
theorem~(\ref{logic:the:FUAP:charsubcong:charac}), the only
possibility with $\pi,\rho\in\pv$ is $(i)$, in which case
$\phi\sim\psi$ as requested.
\end{proof}

Given $\phi,\psi\in\pv$, we already knew from
definition~(\ref{logic:def:FUAP:subcong:substitution:congruence})
that $\axi\phi\equiv\axi\psi$ whenever $\phi\sim\psi$. However, the
converse was not obvious until now:

\begin{prop}\label{logic:prop:FUAP:charsubsong:equivalence:axiom}
Let $V$ be a set and $\pi=\axi\phi$, $\rho=\axi\psi$ for some
$\phi,\psi\in\pv$:
    \[
    \pi\equiv\rho\ \Leftrightarrow\ \phi\sim\psi
    \]
where $\equiv$ and $\sim$ are the alpha equivalence on \pvs\
and \pv\ respectively.
\end{prop}
\begin{proof}
First we show $\Leftarrow$\,: So we assume that $\phi\sim\psi$. Then
the ordered pair $(\pi,\rho)$ is an element of the set $R_{1}$ of
definition~(\ref{logic:def:FUAP:subcong:substitution:congruence}).
In particular we have $\pi\equiv\rho$. We now prove $\Rightarrow$\,:
so we assume that $\pi\equiv\rho$. Using
theorem~(\ref{logic:the:FUAP:charsubcong:charac}), the only
possibility with $\pi=\axi\phi$ and $\rho=\axi\psi$ is $(ii)$, in
which case $\phi\sim\psi$ as requested.
\end{proof}
