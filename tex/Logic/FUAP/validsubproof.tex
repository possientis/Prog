In definition~(\ref{logic:def:FOPL:valid:substitution}) we defined
what it meant for a substitution $\sigma:V\to W$ to be valid for a
formula $\phi\in\pv$. We needed to introduce the notion that
$\sigma$ would not randomly distort the {\em logical structure} of
$\phi$, a property which is commonly known as {\em avoiding
capture}. The purpose was obvious: we wanted to {\em carry over}
properties of $\phi\in\pv$ into corresponding properties of
$\sigma(\phi)\in{\bf P}(W)$. For example,
theorem~(\ref{logic:the:FOPL:mintransfsubcong:valid}) of
page~\pageref{logic:the:FOPL:mintransfsubcong:valid} tells us that a
substitution equivalence $\phi\sim\psi$ is preserved by $\sigma$
provided it is valid for both $\phi$ and $\psi$. Another example
will be seen as the {\em substitution lemma} in the context of model
theory, where
proposition~(\ref{logic:prop:FOPL:model:valid:substitution}) will
show that if $\sigma$ is valid for $\phi$, then the truth of
$\sigma(\phi)$ in a model $M$ under an assignment $a:W\to M$ is
equivalent to the truth of $\phi$ under the assignment
$a\circ\sigma$. The notion of $\sigma$ being valid for $\phi$ is
crucially important.

One of the most interesting properties of formulas which we may wish
to {\em carry over} is that of provability. If a sequent
$\Gamma\vdash\phi$ is true, we all want to know under what
conditions the corresponding sequent
$\sigma(\Gamma)\vdash\sigma(\phi)$ is also true. It is very tempting
to conjecture that provided $\sigma$ is valid for $\phi$ and is also
valid for every hypothesis $\psi\in\Gamma$, then the result will
hold. Of course, an obvious issue is the fact that a sequent
$\Gamma\vdash\phi$ does not tell us anything about the axioms being
used in a proof underlying the sequent. So we cannot hope to simply
{\em carry over} every step of the proof with $\sigma$, as our
assumption does not guarantee $\sigma$ is valid for every axiom
involved in the proof. It only guarantees $\sigma$ is valid for the
conclusion, and for every hypothesis. In fairness, this issue only
arises because we stubbornly insist on working with actual maps
$\sigma:V\to W$, rather than essential substitutions
$\sigma:\pv\to{\bf P}(W)$ of
definition~(\ref{logic:def:FOPL:esssubstprop:essential}). We know
essential substitutions are the right concept. We somehow suspect
our scheme to {\em carry over} proofs with the substitution
$\sigma:\pvs\to{\bf\Pi}(W)$ of
definition~(\ref{logic:def:FUAP:substitution:substitution}) is most
likely a waste of time. We should {\em carry over} proofs using
essential substitutions only. There is however one thing to
remember: from theorem~(\ref{logic:the:FOPL:esssubst:existence}) of
page~\pageref{logic:the:FOPL:esssubst:existence} an essential
substitution $\sigma:\pv\to{\bf P}(W)$ can only exist if $W$ is an
infinite set, or $V$ is a smaller set than $W$. So if we restrict
our analysis to essential substitutions, we cannot hope to {\em
carry over} the sequent $\Gamma\vdash\phi$ into
$\sigma(\Gamma)\vdash\sigma(\phi)$ in the case when $W$ is a finite
set of smaller cardinality than $V$.

We are certainly interested in saying something about
$\sigma(\Gamma)\vdash\sigma(\phi)$ in the case when $W$ is a smaller
finite set. For example, suppose we have an embedding $j:W\to V$
between two finite sets, and $\Delta$ is a consistent subset of
${\bf P}(W)$. We certainly hope that $\Gamma=j(\Delta)$ is a
consistent subset of \pv. If $\Gamma$ is not consistent, as we shall
see from
definition~(\ref{logic:def:FOPL:semantics:consistent:subset}) the
sequent $\Gamma\vdash\bot$ is true. Assuming $W\neq\emptyset$ and
$j$ has a left inverse $\sigma:V\to W$, here is a case when we want
to {\em carry over} the sequent $\Gamma\vdash\bot$ into a smaller
finite set $W$: we want to argue that $\sigma(\Gamma)\vdash\bot$ is
true so as to obtain $\Delta\vdash\bot$, contradicting the
consistency of $\Delta$. Most proofs of G\"odel completeness theorem
involve successive embeddings where constants are continually added
to the language, and it is usually taken for granted that
consistency is preserved as the language is extended. This is not
completely obvious for us. We do not have constants, but only
variables. As the set of variables is increased, the range of
possible proofs becomes larger and it is conceivable that a
contradiction may appear. In general, it is not clear that a sequent
$j(\Delta)\vdash j(\phi)$ can be transported back into the smaller
finite set, as the variables involved in the axioms may have no
obvious counterpart in $W$.

In any case, although we strongly suspect essential substitutions
will play an important role in allowing us to {\em carry over}
sequents, we should not attempt to run before we can walk: when
attempting to establish the sequent
$\sigma(\Gamma)\vdash\sigma(\phi)$ we should start with the obvious,
and the obvious consists in following every step of the proof
underlying $\Gamma\vdash\phi$, and replacing all variables in line
with the substitution $\sigma$. This is the purpose of
definition~(\ref{logic:def:FUAP:substitution:substitution}),
creating $\sigma(\pi)\in{\bf\Pi}(W)$ from $\pi\in\pvs$, in the hope
that $\sigma(\pi)$ will become a proof of $\sigma(\phi)$, from the
set of hypothesis $\sigma(\Gamma)$. Of course, this is not going to
work in all cases. The proof $\sigma(\pi)$ is pretty worthless,
unless we control its valuation $\val(\sigma(\pi))$, that is:
    \begin{equation}\label{logic:eqn:FUAP:validsub:intro:1}
    \val\circ\sigma(\pi)=\sigma\circ\val(\pi)
    \end{equation}
It is clear equation~(\ref{logic:eqn:FUAP:validsub:intro:1}) will
fail unless $\sigma:V\to W$ has the right properties in relation to
$\pi$. For example, if $\pi=\axi\phi$ where $\phi\in\av$ is an axiom
of first order logic, then $\sigma(\pi)=\axi\sigma(\phi)$ and if
$\sigma(\phi)$ fails to be an axiom, then from
definition~(\ref{logic:def:FOPL:proof:valuation}) we obtain
$\val\circ\sigma(\pi)=\bot\to\bot$. So
equation~(\ref{logic:eqn:FUAP:validsub:intro:1}) will fail unless
$\sigma(\phi)$ is itself an axiom. We shall see later in
lemma~(\ref{logic:lemma:FUAP:substitution:axiom}) that
$\sigma(\phi)$ is indeed an axiom, provided $\sigma$ is valid for
$\phi$. It is not difficult to design a counterexample otherwise. So
here is a case when the condition {\em $\sigma$ is valid for $\phi$}
is a key condition to ensure that
equation~(\ref{logic:eqn:FUAP:validsub:intro:1}) is met by
$\pi=\axi\phi$, at least in the case when $\pi$ is totaly clean,
that is when $\phi$ is an axiom. In our discussion preceding
definition~(\ref{logic:def:FUAP:clean:clean:proof}) we explained the
importance of $\pi$ being totally clean.

In this section, we want to define what it is for $\sigma$ to be
{\em valid for $\pi$}. Our purpose to the design the right
conditions so as to ensure
equation~(\ref{logic:eqn:FUAP:validsub:intro:1}) will hold, whenever
$\pi$ is totally clean and $\sigma$ is {\em valid for $\pi$}. The
result will be proved in
proposition~(\ref{logic:prop:FUAP:validsubtotclean:valuation:commute}).
Choosing the right conditions which should be met by a substitution
$\sigma$ which is valid for a proof $\pi$ is not completely obvious.
We have seen that equation~(\ref{logic:eqn:FUAP:validsub:intro:1})
plays a key role and we want $\sigma$ to be valid for $\phi$
whenever $\phi$ is an axiom of $\pi$. This will be imposed by
condition~$(ii)$ of
definition~(\ref{logic:def:FUAP:validsubproof:validsub}) below. Now
given $\phi\in\hyp(\pi)$, should we impose that $\sigma$ be valid
for $\phi$? Yes, we should! Although it is possible to obtain a
proof $\sigma(\pi)$ which satisfies
equation~(\ref{logic:eqn:FUAP:validsub:intro:1}) without requesting
that $\sigma$ be valid for every hypothesis of $\pi$, another
consideration comes to mind: since $\pv\subseteq\pvs$ the
mathematical statement '$\sigma$ is valid for $\phi$' will become
ambiguous after we define the notion of validity for proofs. What we
certainly want to avoid, is designing a concept of {\em validity}
for proof whereby '$\sigma$ is valid for $\phi$' has a different
meaning depending on whether $\phi$ is viewed as a formula or as a
proof. So $\sigma$ has to be valid for every $\phi\in\hyp(\pi)$.
This will be imposed by condition~$(i)$ of
definition~(\ref{logic:def:FUAP:validsubproof:validsub}) below. Now
the real essence of {\em validity} is the idea that a substitution
should be {\em capture-avoiding}. This idea is somehow very general
and relevant to every formal language with variable binding. We have
formalized this idea for first order logic within the restricted
language \pv, using
definition~(\ref{logic:def:FOPL:valid:substitution}). We now want to
express the same idea in the formal language of Hilbert style proofs
\pvs. We could equally do so for untyped $\lambda$-calculus and many
other cases. Somehow, we know the analysis should be carried out in
a general setting and with the right degree of abstraction. We have
failed to provide this so far, leading to all sorts of tedious
repetitions whereby simple results established for formulas are
extended to proofs. For our defense, the situation is not as clear
cut as it may appear: although there is an obvious syntactic
parallel between \pv\ and \pvs\ with the constant operator $\bot$
versus $\axi\phi$, the binary operator $\to$ versus $\pon$ and unary
quantifications $\forall x$ versus $\gen x$, the generator of the
free universal algebra $\pvs$ is the algebra \pv\ itself, on which a
lot of structure has already been defined. So when it comes to
defining the notion of {\em validity} for proofs, we cannot simply
re-use the idea of
definition~(\ref{logic:def:FOPL:valid:substitution}). We need to
impose further conditions which are $(i)$ and $(ii)$ of
definition~(\ref{logic:def:FUAP:validsubproof:validsub}) below.
However, the idea of
definition~(\ref{logic:def:FOPL:valid:substitution}) is certainly
fundamental and should be retained for the purpose of
definition~(\ref{logic:def:FUAP:validsubproof:validsub}): if a
substitution $\sigma:V\to W$ is {\em capture-avoiding} with respect
to a proof $\pi\in\pvs$, any free variable of any sub-proof should
remain free after substitution. This will be imposed by condition
$(iii)$ below:

\index{valid@Valid substitution for
proof}\index{capture@Capture-avoiding substitution}
\begin{defin}\label{logic:def:FUAP:validsubproof:validsub}
Let $V$ and $W$ be sets and $\sigma:V\to W$ be a map. Then $\sigma$
is said to be {\em valid for} $\pi\in\pvs$, \ifand\ for all
$\phi\in\pv$ and $\rho\in\pvs$\,:
    \begin{eqnarray*}
    (i)&&\phi\in\hyp(\pi)\ \Rightarrow\ \mbox{$\sigma$ valid for $\phi$}\\
    (ii)&&\phi\in\ax(\pi)\ \Rightarrow\ \mbox{$\sigma$ valid for $\phi$}\\
    (iii)&&\rho\preceq\pi\ \Rightarrow\ \forall u\ [\ u\in\free(\rho)\
    \Rightarrow\ \sigma(u)\in\free(\sigma(\rho))\ ]
    \end{eqnarray*}
\end{defin}

Let $\sigma:V\to W$ be a map and $\phi\in\pv$. Since
$\pv\subseteq\pvs$, the mathematical statement '{\em $\sigma$ is
valid for $\phi$}' has become ambiguous, as it may refer to this new
definition~(\ref{logic:def:FUAP:validsubproof:validsub}) or to the
usual definition~(\ref{logic:def:FOPL:valid:substitution}). Luckily,
as the next proposition shows, the two definitions lead to
equivalent statements. The next proposition states that $\sigma$ is
valid for $\pi$ \ifand\ it is valid for $\phi$, whenever $\pi=\phi$.
This seems rather tautological but of course it is not, as the two
sides of the equivalence refer to
definition~(\ref{logic:def:FUAP:validsubproof:validsub}) and
definition~(\ref{logic:def:FOPL:valid:substitution})\,:
\begin{prop}\label{logic:prop:FUAP:validsubproof:recursion:formula}
Let $V, W$ be sets and $\sigma:V\to W$ be a map. Let $\pi$ be of the
form $\pi=\phi\in\pv$. Then $\sigma$ is valid for $\pi$ \ifand\ it
is valid for~$\phi$.
\end{prop}
\begin{proof}
Suppose $\sigma$ is valid for $\pi=\phi$. Since $\phi\in\hyp(\pi)$
we see that $\sigma$ is valid for~$\phi$. Conversely, suppose
$\sigma$ is valid for $\phi$. We need to show that $\sigma$ is valid
for $\pi$. So suppose $\psi\in\hyp(\pi)$. We need to show that
$\sigma$ is valid for $\psi$. However, $\hyp(\pi)=\{\phi\}$ and
consequently $\psi=\phi$. So $\sigma$ is indeed valid for $\psi$.
Furthermore since $\ax(\pi)=\emptyset$, property $(ii)$ of
definition~(\ref{logic:def:FUAP:validsubproof:validsub}) is
vacuously true. So we now check property $(iii)$: so let
$\rho\preceq\pi$ and $u\in\free(\rho)$. We need to show that
$\sigma(u)\in\free(\sigma(\rho))$. However since $\pi=\phi\in\pv$,
from definition~(\ref{logic:def:subformula}) we have
$\subf(\pi)=\{\pi\}$. In other words, the only sub-proof of $\pi$ is
$\pi$ itself. It follows that $\rho=\phi$ and we need to check that
$\sigma(u)\in\free(\sigma(\phi))$ knowing that $u\in\free(\phi)$.
This follows from
definition~(\ref{logic:def:FOPL:valid:substitution}) and the
assumption that $\sigma$ is valid for $\phi$.
\end{proof}


A map $\sigma:V\to W$ is valid for $\pi$ \ifand\ it is valid for
every sub-proof of $\pi$. This property is similar to that
encountered with formulas, as described in
proposition~(\ref{logic:prop:FOPL:valid:subformula}). However as
already pointed out, sub-formulas and sub-proofs are two different
things. If $\phi\in\pv$, then it may have many sub-formulas but only
one sub-proof, namely itself. The notion of sub-formula or sub-proof
in a free universal algebra is formally introduced in
definition~(\ref{logic:def:subformula}) of
page~\pageref{logic:def:subformula}.

\begin{prop}\label{logic:prop:FUAP:validsubproof:subformula}
Let $V, W$ be sets and $\sigma:V\to W$ be a map. Then $\sigma$ is
valid for $\pi\in\pvs$ \ifand\ it is valid for any sub-proof
$\rho\preceq\pi$ of the proof $\pi$.
\end{prop}
\begin{proof}
If $\sigma$ is valid for any sub-proof of $\pi$, then in particular
it is valid for $\pi$. So we assume that $\sigma$ is valid for $\pi$
and consider a sub-proof $\rho\preceq\pi$. We need to show that
$\sigma$ is also valid for $\rho$. So suppose
$\phi\in\hyp(\rho)\cup\ax(\rho)$. We need to show that $\sigma$ is
valid for $\phi$. However, from
proposition~(\ref{logic:prop:FUAP:hypothesis:subformula})
and~(\ref{logic:prop:FUAP:axiomset:subformula}) we have the
inclusions $\hyp(\rho)\subseteq\hyp(\pi)$  and
$\ax(\rho)\subseteq\ax(\pi)$. It follows that
$\phi\in\hyp(\pi)\cup\ax(\pi)$. Since $\sigma$ is valid for $\pi$ we
see that $\sigma$ is valid for $\phi$. We now assume that
$\kappa\preceq\rho$. Given $u\in\free(\kappa)$, we need to show that
$\sigma(u)\in\free(\sigma(\kappa))$. However, this follows
immediately from the validity of $\sigma$ for $\pi$ and the fact, by
transitivity, that $\kappa\preceq\pi$.
\end{proof}

Whenever $\pi=\axi\phi$ the validity of $\sigma$ for $\pi$ is
equivalent to that of $\sigma$ for $\phi$. This property is of
course natural but also crucial to guarantee the equality
$\val\circ\sigma(\pi)=\sigma\circ\val(\pi)$ whenever $\phi$ is an
axiom of first order logic and $\sigma$ is valid for $\pi=\axi\phi$.
Fundamentally, the image $\sigma(\phi)$ of an axiom by a valid
substitution remains an axiom of first order logic, as will be seen
from lemma~(\ref{logic:lemma:FUAP:substitution:axiom}).
\begin{prop}\label{logic:prop:FUAP:validsubproof:recursion:axiom}
Let $V, W$ be sets and $\sigma:V\to W$ be a map. Let $\pi=\axi\phi$,
where $\phi\in\pv$. Then $\sigma$ is valid for $\pi$ \ifand\ it is
valid for $\phi$.
\end{prop}
\begin{proof}
Suppose $\sigma$ is valid for $\pi=\axi\phi$. Since
$\phi\in\ax(\pi)$ we see that $\sigma$ is valid for~$\phi$.
Conversely, suppose $\sigma$ is valid for $\phi$. We need to show
that $\sigma$ is valid for~$\pi$. Since $\hyp(\pi)=\emptyset$,
property $(i)$ of
definition~(\ref{logic:def:FUAP:validsubproof:validsub}) is
vacuously true. So we now check property $(ii)$: suppose
$\psi\in\ax(\pi)$. We need to show that $\sigma$ is valid for
$\psi$. However, $\ax(\pi)=\{\phi\}$ and consequently $\psi=\phi$.
So $\sigma$ is indeed valid for $\psi$. We now check property
$(iii)$: Let $\rho\preceq\pi$ and $u\in\free(\rho)$. We need to show
that $\sigma(u)\in\free(\sigma(\rho))$. However, from
definition~(\ref{logic:def:subformula}) we have
$\subf(\pi)=\subf(\axi\phi(0))=\{\axi\phi(0)\}=\{\pi\}$. In other
words, the only sub-proof of $\pi$ is $\pi$ itself. It follows that
$\rho=\pi=\axi\phi$. So we need to show that
$\sigma(u)\in\free(\sigma(\phi))$ knowing that $u\in\free(\phi)$.
This follows from the validity of $\sigma$ for $\phi$.
\end{proof}

Just as in the case of formulas with
proposition~(\ref{logic:prop:FOPL:valid:recursion:imp}) and
proposition~(\ref{logic:prop:FOPL:valid:recursion:quant}) which
established a link between the validity of $\sigma$ for
$\phi=\phi_{1}\to\phi_{2}$ or $\phi=\forall x\phi_{1}$ with the
validity of $\sigma$ for $\phi_{1}$ and $\phi_{2}$, we need to have
a similar link between the validity of $\sigma$ for
$\pi=\pi_{1}\pon\pi_{2}$ or $\pi=\gen x\pi_{1}$, with the validity
of $\sigma$ for $\pi_{1}$ and $\pi_{2}$. These results are very
useful to carry out structural induction arguments.

\begin{prop}\label{logic:prop:FUAP:validsubproof:recursion:pon}
Let $V, W$ be sets and $\sigma:V\to W$ be a map. Let
$\pi=\pi_{1}\pon\pi_{2}$ where $\pi_{1},\pi_{2}\in\pvs$. Then
$\sigma$ valid for $\pi$ \ifand\ it is valid for $\pi_{1}$ and
$\pi_{2}$.
\end{prop}
\begin{proof}
First we show the 'only if' part: so we assume that $\sigma$ is
valid for $\pi=\pi_{1}\pon\pi_{2}$. From
proposition~(\ref{logic:prop:FUAP:validsubproof:subformula}) it is
therefore valid for every sub-proof of $\pi$. Since
$\pi_{1}\preceq\pi$ and $\pi_{2}\preceq\pi$ we conclude that
$\sigma$ is valid for both $\pi_{1}$ and $\pi_{2}$. We now show the
'if' part: so we assume that $\sigma$ is valid for $\pi_{1}$ and
$\pi_{2}$. We need to show that $\sigma$ is valid for
$\pi=\pi_{1}\pon\pi_{2}$. So let $\phi\in\hyp(\pi)$. We need to show
that $\sigma$ is valid for $\phi$. However
$\hyp(\pi)=\hyp(\pi_{1})\cup\hyp(\pi_{2})$. So $\phi$ is an element
of $\hyp(\pi_{1})$ or $\hyp(\pi_{2})$ and in both cases we see that
$\sigma$ is valid for $\phi$. So we now assume that
$\phi\in\ax(\pi)$. We need to show that $\sigma$ is valid for
$\phi$. Once again, since $\ax(\pi)=\ax(\pi_{1})\cup\ax(\pi_{2})$
the formula $\phi$ must be an element of $\ax(\pi_{1})$ or
$\ax(\pi_{2})$. Either way, we see that $\sigma$ is valid for
$\phi$. In order to show that $\sigma$ is valid for $\pi$, it
remains to show that $(iii)$ of
definition~(\ref{logic:def:FUAP:validsubproof:validsub}) is
satisfied. So let $\rho\preceq\pi$ and $u\in\free(\rho)$. We need to
show that $\sigma(u)\in\free(\sigma(\rho))$. If $\rho\preceq\pi_{1}$
or $\rho\preceq\pi_{2}$, then this follows immediately from the
validity of $\sigma$ for $\pi_{1}$ or $\pi_{2}$. So we assume that
$\rho=\pi_{1}\pon\pi_{2}$. Then
$\free(\rho)=\free(\pi_{1})\cup\free(\pi_{2})$ and we shall
distinguish two cases: first we assume that $u\in\free(\pi_{1})$.
Then from the validity of $\sigma$ for $\pi_{1}$\,:
    \begin{eqnarray*}
    \sigma(u)&\in&\free(\sigma(\pi_{1}))\\
    &\subseteq&\free(\sigma(\pi_{1}))\cup\free(\sigma(\pi_{2}))\\
    &=&\free(\,\sigma(\pi_{1})\pon\,\sigma(\pi_{2})\,)\\
    &=&\free(\sigma(\pi_{1}\pon\pi_{2}))\\
    &=&\free(\sigma(\rho))
    \end{eqnarray*}
Next we assume $u\in\free(\pi_{2})$ and in identical fashion we
obtain $\sigma(u)\in\free(\sigma(\rho))$.
\end{proof}

\begin{prop}\label{logic:prop:FUAP:validsubproof:recursion:gen}
Let $V, W$ be sets and $\sigma:V\to W$ be a map. Let $\pi=\gen
x\pi_{1}$. Then $\sigma$ is valid for $\pi$ \ifand\ it is valid for
$\pi_{1}$ and for all $u\in V$:
    \begin{equation}\label{logic:eqn:FUAP:validsubproof:recursion:gen:1}
    u\in\free(\gen x\pi_{1})\ \Rightarrow\
    \sigma(u)\neq\sigma(x)
    \end{equation}
\end{prop}
\begin{proof}
First we show the 'only if' part: so we assume that $\sigma$ is
valid for the proof $\pi=\gen x\pi_{1}$ where $x\in V$ and
$\pi_{1}\in\pvs$. From
proposition~(\ref{logic:prop:FUAP:validsubproof:subformula}),
$\sigma$ is valid for $\pi_{1}\preceq\pi$. So it remains to show
that the
implication~(\ref{logic:eqn:FUAP:validsubproof:recursion:gen:1})
holds. So suppose $u\in\free(\gen x\pi_{1})$. We need to show that
$\sigma(u)\neq\sigma(x)$. However since $\pi$ is a sub-proof of
itself, having assumed that $\sigma$ is valid for $\pi$ we obtain:
    \begin{eqnarray*}
    \sigma(u)&\in&\free(\sigma(\pi))\\
    &=&\free(\,\sigma(\gen x\pi_{1})\,)\\
    &=&\free(\,\gen\sigma(x)\sigma(\pi_{1})\,)\\
    &=&\free(\sigma(\pi_{1}))\setminus\{\sigma(x)\}\\
    \end{eqnarray*}
In particular we have $\sigma(u)\neq\sigma(x)$. We now show the 'if'
part: so we assume that $\sigma$ is valid for $\pi_{1}$ and
furthermore
that~(\ref{logic:eqn:FUAP:validsubproof:recursion:gen:1}) holds. We
need to show that $\sigma$ is valid for~$\pi$. So let
$\phi\in\hyp(\pi)\cup\ax(\pi)$. We need to show that $\sigma$ is
valid for~$\phi$. However, we have $\hyp(\pi)=\hyp(\pi_{1})$ and
$\ax(\pi)=\ax(\pi_{1})$ and consequently
$\phi\in\hyp(\pi_{1})\cup\ax(\pi_{1})$. Having assumed that $\sigma$
is valid for $\pi_{1}$, it follows that $\sigma$ is valid for $\phi$
as requested. In order to show that $\sigma$ is valid for $\pi$, it
remains to prove that $(iii)$ of
definition~(\ref{logic:def:FUAP:validsubproof:validsub}) holds. So
let $\rho\preceq\pi$ and $u\in\free(\rho)$. We need to show that
$\sigma(u)\in\free(\sigma(\rho))$. If $\rho\preceq\pi_{1}$ then this
follows immediately from the validity of $\sigma$ for $\pi_{1}$. So
we assume that $\rho=\gen x\pi_{1}$ in which case $u\in\free(\gen
x\pi_{1})$. We need to show that $\sigma(u)\in \free(\sigma(\gen
x\pi_{1}))=\free(\sigma(\pi_{1}))\setminus\{\sigma(x)\}$. Having
assumed that~(\ref{logic:eqn:FUAP:validsubproof:recursion:gen:1})
holds, we already know that $\sigma(u)\neq\sigma(x)$. So it remains
to show that $\sigma(u)\in\free(\sigma(\pi_{1}))$ which follows from
the validity of $\sigma$ for $\pi_{1}$.
\end{proof}

The following proposition is the counterpart of
proposition~(\ref{logic:prop:FOPL:validsub:criterion}) which was
established for formulas. It offers a simpler criterion to establish
validity. When attempting to prove $(iii)$ of
definition~(\ref{logic:def:FUAP:validsubproof:validsub}), it is no
longer necessary to consider every sub-proof $\rho\preceq\pi$ and we
can instead restrict our attention to sub-proofs of the form
$\rho=\gen x\pi_{1}$. Furthermore, rather than prove
$\sigma(u)\in\free(\sigma(\rho))$ whenever $u\in\free(\rho)$, it is
sufficient to prove the simpler property $\sigma(u)\neq\sigma(x)$.

\begin{prop}\label{logic:prop:FUAP:validsubproof:criterion}
Let $V$ and $W$ be sets and $\sigma:V\to W$ be a map. Then $\sigma$
is {\em valid for} $\pi\in\pvs$ \ifand\ for all $\phi\in\pv$,
$\pi_{1}\in\pvs$ and $x\in V$\,:
    \begin{eqnarray*}
    (i)&&\phi\in\hyp(\pi)\ \Rightarrow\ \mbox{$\sigma$ valid for $\phi$}\\
    (ii)&&\phi\in\ax(\pi)\ \Rightarrow\ \mbox{$\sigma$ valid for $\phi$}\\
    (iii)&&\gen x\pi_{1}\preceq\pi\ \Rightarrow\ \forall u\ [\ u\in\free(\gen x\pi_{1})\
    \Rightarrow\ \sigma(u)\neq\sigma(x)\ ]
    \end{eqnarray*}
\end{prop}
\begin{proof}
First we show the 'only if' part. So we assume that $\sigma$ is
valid for $\pi$. Then $(i)$ and $(ii)$ are satisfied by definition
and we simply need to prove $(iii)$. So we assume that $\gen
x\pi_{1}\preceq\pi$ and $u\in\free(\gen x\pi_{1})$. We need to show
$\sigma(u)\neq\sigma(x)$. However, from
proposition~(\ref{logic:prop:FUAP:validsubproof:subformula}) we see
that $\sigma$ is valid for $\gen x\pi_{1}$. Having assumed that
$u\in\free(\gen x\pi_{1})$ we must have
$\sigma(u)\in\free(\sigma(\gen
x\pi_{1}))=\free(\sigma(\pi_{1}))\setminus\{\sigma(x)\}$. It follows
in particular that $\sigma(u)\neq\sigma(x)$ as requested. We now
show the 'if' part: for every proof $\pi\in\pvs$ we need to show the
following implication:
     \[
     (i)+(ii)+(iii)\ \Rightarrow\ \mbox{$\sigma$ valid for $\pi$}
     \]
We shall do so with a structural induction, using
theorem~(\ref{logic:the:proof:induction}) of
page~\pageref{logic:the:proof:induction}. First we assume that
$\pi=\phi$ for some $\phi\in\pv$. We need to show our implication is
true for $\pi$. So we assume that $(i)$, $(ii)$ and $(iii)$ hold. We
need to show that $\sigma$ is valid for $\phi$. This follows
immediately from $(i)$ and the fact that
$\phi\in\hyp(\pi)=\{\phi\}$. So we now assume that $\pi=\axi\phi$
for some $\phi\in\pv$. We need to show our implication is true for
$\pi$. So we assume that $(i)$, $(ii)$ and $(iii)$ hold. We need to
show that $\sigma$ is valid for $\pi$. From
proposition~(\ref{logic:prop:FUAP:validsubproof:recursion:axiom}),
we simply need to show that $\sigma$ is valid for $\phi$, which
follows from $(ii)$ and the fact that $\phi\in\ax(\pi)=\{\phi\}$. We
now assume that $\pi=\pi_{1}\pon\pi_{2}$ where
$\pi_{1},\pi_{2}\in\pvs$ are proofs satisfying our implication. We
need to show the same is true of $\pi$.  So we assume that $(i)$,
$(ii)$ and $(iii)$ hold for $\pi$. We need to show that $\sigma$ is
valid for $\pi$. Using
proposition~(\ref{logic:prop:FUAP:validsubproof:recursion:pon}) it
is sufficient to show that $\sigma$ is valid for $\pi_{1}$ and
$\pi_{2}$. Having assumed our implication is true for $\pi_{1}$ and
$\pi_{2}$, we simply need to prove that $(i)$, $(ii)$ and $(iii)$
hold for $\pi_{1}$ and $\pi_{2}$. First we deal with $\pi_{1}$\,:
the fact that $(i)$ holds for $\pi_{1}$ follows from
$\hyp(\pi_{1})\subseteq\hyp(\pi)$ and the fact that $(i)$ is true
for $\pi$. Note that $\hyp(\pi_{1})\subseteq\hyp(\pi)$ is a
consequence of $\pi_{1}\preceq\pi$ and
proposition~(\ref{logic:prop:FUAP:hypothesis:subformula}). The fact
that $(ii)$ holds for $\pi_{1}$ follows from
$\ax(\pi_{1})\subseteq\ax(\pi)$ and the fact that $(ii)$ is true for
$\pi$. Note that $\ax(\pi_{1})\subseteq\ax(\pi)$ is a consequence of
$\pi_{1}\preceq\pi$ and
proposition~(\ref{logic:prop:FUAP:axiomset:subformula}). The fact
that $(iii)$ holds for $\pi_{1}$ follows from $\pi_{1}\preceq\pi$
and the transitivity of the sub-proof partial order $\preceq$ on
\pvs, as well as the fact that $(iii)$ is true for $\pi$. The case
of $\pi_{2}$ is dealt with similarly. This completes our induction
argument in the case when $\pi=\pi_{1}\pon\pi_{2}$. So we now assume
that $\pi=\gen x\pi_{1}$ where $x\in V$ and $\pi_{1}\in\pvs$ is a
proof satisfying our implication. We need to show the same is true
for $\pi$. So we assume that $(i)$, $(ii)$ and $(iii)$ hold for
$\pi$. We need to show that $\sigma$ is valid for $\pi$. Using
proposition~(\ref{logic:prop:FUAP:validsubproof:recursion:gen}) it
is sufficient to show that $\sigma$ is valid for $\pi_{1}$ and
furthermore, given $u\in V$:
    \[
    u\in\free(\gen x\pi_{1})\ \Rightarrow\
    \sigma(u)\neq\sigma(x)
    \]
This last implication follows immediately from $(iii)$. So it
remains to show that $\sigma$ is valid for $\pi_{1}$. Having assumed
our implication is true for $\pi_{1}$, we simply need to prove that
$(i)$, $(ii)$ and $(iii)$ hold for $\pi_{1}$. Again, this follows
from $\pi_{1}\preceq\pi$.
\end{proof}

In proposition~(\ref{logic:prop:FUAP:freevar:substitution}) we
established the inclusion
$\spec(\sigma(\pi))\subseteq\sigma(\spec(\pi))$ which holds for
every map $\sigma:V\to W$ and $\pi\in\pvs$. This inclusion states
that the specific variables of the proof $\sigma(\pi)$ must be
images by $\sigma$ of specific variables of the proof $\pi$. The
following proposition offers a stronger result with the equality
$\spec(\sigma(\pi))=\sigma(\spec(\pi))$, provided $\sigma$ is valid
for $\pi$. If $\sigma$ is not valid for $\pi$ then the equality may
fail, as the counterexample $\pi=\forall x(x\in y)$ with $x\neq y$
and $\sigma=[y/x]$ shows. In this case we obtain
$\spec(\sigma(\pi))=\emptyset$ and $\sigma(\spec(\pi))=\{y\}$.
\begin{prop}\label{logic:prop:FUAP:validsubproof:specvar}
Let $V, W$ be sets and $\sigma:V\to W$ be a map. Then for every
proof $\pi\in\pvs$, if  the substitution $\sigma$ is valid for $\pi$
we have:
    \[
    \spec(\sigma(\pi))=\sigma(\spec(\pi))
    \]
where $\sigma:\pvs\to{\bf\Pi}(W)$ also denotes the proof
substitution mapping.
\end{prop}
\begin{proof}
We assume that $\sigma$ is valid for $\pi$. Then $\sigma$ is valid
for every $\phi\in\hyp(\pi)$. Hence from
proposition~(\ref{logic:prop:FOPL:valid:free:commute}),
$\free(\sigma(\phi))=\sigma(\free(\phi))$ for all $\phi\in\hyp(\pi)$
and:
    \begin{eqnarray*}
    \spec(\sigma(\pi))&=&\free(\,\hyp(\sigma(\pi))\,)\\
    \mbox{prop.~(\ref{logic:prop:FUAP:substitution:hypothesis})}\ \rightarrow
    &=&\free(\,\sigma(\hyp(\pi))\,)\\
    &=&\cup\{\,\free(\psi)\ :\ \psi\in\sigma(\hyp(\pi))\,\}\\
    &=&\cup\{\,\free(\sigma(\phi))\ :\ \phi\in\hyp(\pi)\,\}\\
    \mbox{$\sigma$ valid for $\phi\in\hyp(\pi)$}\ \rightarrow
    &=&\cup\{\,\sigma(\free(\phi))\ :\ \phi\in\hyp(\pi)\,\}\\
    &=&\sigma(\,\cup\{\,\free(\phi)\ :\ \phi\in\hyp(\pi)\,\}\,)\\
    &=&\sigma(\,\free(\hyp(\pi))\,)\\
    &=&\sigma(\spec(\pi))
    \end{eqnarray*}
\end{proof}

The following proposition is the counterpart of
proposition~(\ref{logic:prop:FOPL:valid:free:commute}) which was
established for formulas. The inclusion
$\free(\sigma(\pi))\subseteq\sigma(\free(\pi))$ is already known in
general from
proposition~(\ref{logic:prop:FUAP:freevarproof:substitution:inclusion}).
The free variables of the proof $\sigma(\pi)$ must be images by
$\sigma$ of free variables of the proof $\pi$. We are now offering a
stronger conclusion with the equality
$\free(\sigma(\pi))=\sigma(\free(\pi))$ provided $\sigma$ is valid
for $\pi$. Note that if $\sigma$ is valid for $\pi$ then it is valid
for every sub-proof $\rho\preceq\pi$ and consequently the equality
$\free(\sigma(\rho))=\sigma(\free(\rho))$ is also true for every
$\rho\preceq\pi$.
\begin{prop}\label{logic:prop:FUAP:validsubproof:freevar}
Let $V, W$ be sets and $\sigma:V\to W$ be a map. Then for every
proof $\pi\in\pvs$, if  the substitution $\sigma$ is valid for $\pi$
we have:
    \[
    \free(\sigma(\pi))=\sigma(\free(\pi))
    \]
where $\sigma:\pvs\to{\bf\Pi}(W)$ also denotes the proof
substitution mapping.
\end{prop}
\begin{proof}
The inclusion $\subseteq$ follows from
proposition~(\ref{logic:prop:FUAP:freevarproof:substitution:inclusion}).
So it remains to show $\supseteq$. So let $u\in\free(\pi)$. We need
to show that $\sigma(u)\in\free(\sigma(\pi))$. However, this follows
immediately from the validity of $\sigma$ for $\pi$ and the fact
that $\pi\preceq\pi$.
\end{proof}


The most obvious example of valid substitutions are those which are
injective. In fact given $\sigma:V\to W$ and $\pi\in\pvs$, we only
only need $\sigma$ to be injective on $\var(\pi)$. This simple
proposition is one of the good reasons for us to define the set
$\var(\pi)$ in the first place. It is the counterpart of
proposition~(\ref{logic:prop:FOPL:valid:injective}).
\begin{prop}\label{logic:prop:FUAP:validsubproof:injective}
Let $V, W$ be sets and $\sigma:V\to W$ be a map. Let $\pi\in\pvs$.
We assume that $\sigma_{|\var(\pi)}$ is an injective map. Then
$\sigma$ is valid for $\pi$.
\end{prop}
\begin{proof}
We assume that $\sigma:V\to W$ is injective on $\var(\pi)$. We need
to show that $\sigma$ is valid for $\pi$. So let $\phi\in\hyp(\pi)$.
We need to show that $\sigma$ is valid for $\phi$. Using
proposition~(\ref{logic:prop:FOPL:valid:injective}) it is sufficient
to prove that $\sigma$ is injective on $\var(\phi)$. It is therefore
sufficient to show that $\var(\phi)\subseteq\var(\pi)$, which
follows from proposition~(\ref{logic:prop:FUAP:variable:subformula})
and $\phi\preceq\pi$, this last inequality being itself a
consequence of proposition~(\ref{logic:prop:FUAP:hypothesis:charac})
and $\phi\in\hyp(\pi)$. So we now assume that $\phi\in\ax(\pi)$. We
need to show that $\sigma$ is valid for $\phi$. Using
proposition~(\ref{logic:prop:FOPL:valid:injective}) it is sufficient
to prove that $\sigma$ is injective on $\var(\phi)$. It is therefore
sufficient to show that $\var(\phi)\subseteq\var(\pi)$ or
equivalently $\var(\axi\phi)\subseteq\var(\pi)$. This follows from
proposition~(\ref{logic:prop:FUAP:variable:subformula}) and
$\axi\phi\preceq\pi$, this last inequality being itself a
consequence of proposition~(\ref{logic:prop:FUAP:axiomset:charac})
and $\phi\in\ax(\pi)$. In order to show that $\sigma$ is valid for
$\pi$, we finally consider $\gen x\pi_{1}\preceq\pi$ and
$u\in\free(\gen x\pi_{1})$. Using
proposition~(\ref{logic:prop:FUAP:validsubproof:criterion}) it is
sufficient to prove that $\sigma(u)\neq\sigma(x)$. From
$u\in\free(\gen x\pi_{1})$ in particular $u\neq x$ and having
assumed $\sigma$ is injective on $\var(\pi)$, it is sufficient to
prove that $\{u,x\}\subseteq\var(\pi)$. From $\gen
x\pi_{1}\preceq\pi$ and
proposition~(\ref{logic:prop:FUAP:variable:subformula}) we obtain
$x\in\var(\gen x\pi_{1})\subseteq\var(\pi)$. So it remains to show
that $u\in\var(\pi)$. Using
propositions~(\ref{logic:prop:FUAP:boundvarproof:var:free:bound})
and~(\ref{logic:prop:FUAP:variable:subformula})\,:
    \[
    u\in\free(\gen x\pi_{1})\subseteq \var(\gen x\pi_{1})\subseteq\var(\pi)
    \]
\end{proof}

When replacing a variable $x$ by a variable $y$ in a formula $\phi$,
a standard textbook in mathematical logic will often assume that
$y\not\in\var(\phi)$ to avoid capture. For us, {\em avoiding
capture} means that a substitution of variable is valid for a given
formula. So $[y/x]$ is valid for $\phi$ whenever
$y\not\in\var(\phi)$, as can be seen from
proposition~(\ref{logic:prop:FOPL:validsub:singlevar}). A similar
property can now be established for proofs:
\begin{prop}\label{logic:prop:FUAP:validsubproof:singlevar}
Let $V$ be a set and $x,y\in V$. Let $\pi\in\pvs$. Then we have:
    \[
    y\not\in\var(\pi)\ \Rightarrow\ (\mbox{$[y/x]$ valid for
    $\pi$})
    \]
\end{prop}
\begin{proof}
We assume that $y\not\in\var(\pi)$. We need to show that $[y/x]$ is
valid for $\pi$. Using
proposition~(\ref{logic:prop:FUAP:validsubproof:injective}), it is
sufficient to prove that $[y/x]_{|\var(\pi)}$ is an injective map.
This follows immediately from $y\not\in\var(\pi)$ and
proposition~(\ref{logic:prop:FOPL:singlevar:support}).
\end{proof}


Suppose $\tau:U\to V$ and $\sigma:V\to W$ are maps while
$\pi\in{\bf\Pi}(U)$.  If the substitution $\tau$ is capture-avoiding
when acting on $\pi$ and $\sigma$ is capture-avoiding when acting on
$\tau(\pi)$, then $\sigma\circ\tau$ avoids variable capture when
acting on $\pi$. Conversely, if the substitution $\sigma\circ\tau$
is valid for $\pi$ then both $\tau$ and $\sigma$ are
capture-avoiding when acting on $\pi$ and $\tau(\pi)$ respectively.
The following proposition is the counterpart of
proposition~(\ref{logic:prop:FOPL:valid:composition}) which was
established for formulas:
\begin{prop}\label{logic:prop:FUAP:validsubproof:composition}
Let $U$, $V$, $W$ be sets and $\tau:U\to V$ and $\sigma:V\to W$ be
maps. Then for all $\pi\in{\bf\Pi}(U)$ we have the equivalence:
\[
    (\mbox{$\tau$ valid for $\pi$})\land(\mbox{$\sigma$ valid for
    $\tau(\pi)$})\ \Leftrightarrow\ (\mbox{$\sigma\circ\tau$ valid for
    $\pi$})
\]
where $\tau:{\bf\Pi}(U)\to{\bf\Pi}(V)$ also denotes the associated
proof substitution mapping.
\end{prop}
\begin{proof}
First we show $\Rightarrow$\,: so we assume that $\tau$ is valid for
$\pi$ and $\sigma$ is valid for $\tau(\pi)$. We need to show that
$\sigma\circ\tau$ is valid for $\pi$. So let
$\phi\in\hyp(\pi)\cup\ax(\pi)$. We need to show that
$\sigma\circ\tau$ is valid for $\phi$. Using
proposition~(\ref{logic:prop:FOPL:valid:composition}), it is
sufficient to prove that $\tau$ is valid for $\phi$ and that
$\sigma$ is valid for $\tau(\phi)$. The fact that $\tau$ is valid
for $\phi$ follows immediately from the validity of $\tau$ for
$\pi$. So it remains to show that $\sigma$ is valid for
$\tau(\phi)$. However, from
proposition~(\ref{logic:prop:FUAP:substitution:hypothesis}) we have
$\tau(\hyp(\pi))=\hyp(\tau(\pi))$ and from
proposition~(\ref{logic:prop:FUAP:axiomset:substitution}) we have
$\tau(\ax(\pi))=\ax(\tau(\pi))$. Hence we see that
$\tau(\phi)\in\hyp(\tau(\pi))\cup\ax(\tau(\pi))$, and having assumed
that $\sigma$ is valid for $\tau(\pi)$ we conclude that $\sigma$ is
valid for $\tau(\phi)$ as requested. We now consider $\gen
x\pi_{1}\preceq\pi$. From
proposition~(\ref{logic:prop:FUAP:validsubproof:criterion}), in
order to show that $\sigma\circ\tau$ is valid for $\pi$, we need:
    \begin{equation}\label{logic:eqn:FUAP:validsub:composition:1}
    u\in\free(\gen x\pi_{1})\ \Rightarrow\
    \sigma\circ\tau(u)\neq\sigma\circ\tau(x)
    \end{equation}
However, since $\tau$ is valid for $\pi$, we know the following
implication is true:
    \begin{equation}\label{logic:eqn:FUAP:validsub:composition:2}
    u\in\free(\gen x\pi_{1})\ \Rightarrow\
    \tau(u)\neq\tau(x)
    \end{equation}
So we assume that $u\in\free(\gen x\pi_{1})$. We need to show that
$\sigma\circ\tau(u)\neq\sigma\circ\tau(x)$. Defining $v=\tau(u)\in
V$ and $y=\tau(x)\in V$, we need to show that
$\sigma(v)\neq\sigma(y)$. However, from $\gen x\pi_{1}\preceq\pi$
and proposition~(\ref{logic:prop:FUAP:substitution:subformula}) we
obtain $\gen y\tau(\pi_{1})\preceq\tau(\pi)$. Having assumed that
$\sigma$ is valid for $\tau(\pi)$, the following implication holds:
    \begin{equation}\label{logic:eqn:FUAP:validsub:composition:3}
    v\in\free(\gen y\tau(\pi_{1}))\ \Rightarrow\
    \sigma(v)\neq\sigma(y)
    \end{equation}
Thus, in order to show $\sigma(v)\neq\sigma(y)$ it is sufficient to
prove $v\in\free(\gen y\tau(\pi_{1}))$. We already know from the
implication~(\ref{logic:eqn:FUAP:validsub:composition:2}) that
$v\neq y$. So it remains to show that $v\in\free(\tau(\pi_{1}))$.
However we have $v=\tau(u)\in\tau(\free(\pi_{1}))$. It is therefore
sufficient to prove that
$\tau(\free(\pi_{1}))=\free(\tau(\pi_{1}))$. Using
proposition~(\ref{logic:prop:FUAP:validsubproof:freevar}) we simply
need to show that $\tau$ is valid for $\pi_{1}$ which follows from
proposition~(\ref{logic:prop:FUAP:validsubproof:subformula}) and the
fact that $\pi_{1}\preceq\gen x\pi_{1}\preceq\pi$, i.e. that
$\pi_{1}$ is a sub-proof of $\pi$. We now show $\Leftarrow$\,: so we
assume that $\sigma\circ\tau$ is valid for $\pi$. We need to show
that $\tau$ is valid for $\pi$ and $\sigma$ is valid for
$\tau(\pi)$. First we show that $\tau$ is valid for $\pi$: so let
$\phi\in\hyp(\pi)\cup\ax(\pi)$. We need to show that $\tau$ is valid
for $\phi$. However we know by assumption that $\sigma\circ\tau$ is
valid for $\phi$. Using
proposition~(\ref{logic:prop:FOPL:valid:composition}) it follows
that $\tau$ is valid for $\phi$ as requested. We now consider $\gen
x\pi_{1}\preceq\pi$. In order to show that $\tau$ is valid for $\pi$
we need to show the implication $u\in\free(\gen x\pi_{1})\
\Rightarrow\ \tau(u)\neq\tau(x)$. So let $u\in\free(\gen x\pi_{1})$.
We need to show that $\tau(u)\neq\tau(x)$. However, we know by
assumption that $\sigma\circ\tau(u)\neq\sigma\circ\tau(x)$. So
$\tau(u)\neq\tau(x)$ must follow. We now show that $\sigma$ is valid
for $\tau(\pi)$. So let $\psi\in\hyp(\tau(\pi))\cup\ax(\tau(\pi))$.
We need to show that $\sigma$ is valid for $\psi$. From
proposition~(\ref{logic:prop:FUAP:substitution:hypothesis}),
$\hyp(\tau(\pi))=\tau(\hyp(\pi))$ and from
proposition~(\ref{logic:prop:FUAP:axiomset:substitution})
$\ax(\tau(\pi))=\tau(\ax(\pi))$. It follows that $\psi=\tau(\phi)$
for some $\phi\in\hyp(\pi)\cup\ax(\pi)$. Having assumed that
$\sigma\circ\tau$ is valid for $\pi$, it follows that
$\sigma\circ\tau$ is valid for $\phi$. Using
proposition~(\ref{logic:prop:FOPL:valid:composition}) we see that
$\sigma$ is valid for $\tau(\phi)$. In other words, $\sigma$ is
valid for $\psi$ as requested. We now consider $\gen
y\rho_{1}\preceq\tau(\pi)$. In order to show that $\sigma$ is valid
for $\tau(\pi)$ we need to prove the implication:
    \begin{equation}\label{logic:eqn:FUAP:validsub:composition:4}
    v\in\free(\gen y\rho_{1})\ \Rightarrow\
    \sigma(v)\neq\sigma(y)
    \end{equation}
However since $\gen y\rho_{1}$ is a sub-proof of $\tau(\pi)$, from
proposition~(\ref{logic:prop:FUAP:substitution:subformula}) we have
$\gen y\rho_{1}=\tau(\rho)$ for some $\rho\preceq\pi$. Furthermore,
from theorem~(\ref{logic:the:unique:representation}) of
page~\pageref{logic:the:unique:representation} the proof
$\rho\in{\bf\Pi}(U)$ can only be of four types, namely $\rho=\phi$
for some $\phi\in{\bf P}(U)$ or $\rho=\axi\phi$ for some
$\phi\in{\bf P}(U)$ or $\rho=\pi_{1}\pon\pi_{2}$ or $\rho=\gen
x\pi_{1}$. From the equation $\gen y\rho_{1}=\tau(\rho)$ and the
uniqueness of representation stated in
theorem~(\ref{logic:the:unique:representation}) it is clear the only
possibility is $\rho=\gen x\pi_{1}$ for some $x\in U$ and
$\pi_{1}\in{\bf\Pi}(U)$. Hence, we have found $x$ and $\pi_{1}$ such
that $\gen x\pi_{1}\preceq\pi$ and $\gen y\rho_{1}=\tau(\gen
x\pi_{1})=\gen\tau(x)\tau(\pi_{1})$, i.e. $y=\tau(x)$ and
$\rho_{1}=\tau(\pi_{1})$. So the
implication~(\ref{logic:eqn:FUAP:validsub:composition:4}) can be
stated as:
    \begin{equation}\label{logic:eqn:FUAP:validsub:composition:5}
    v\in\free(\,\gen\tau(x)\tau(\pi_{1})\,)\ \Rightarrow\
    \sigma(v)\neq\sigma\circ\tau(x)
    \end{equation}
So let $v\in\free(\,\gen\tau(x)\tau(\pi_{1})\,)$. We need to show
that $\sigma(v)\neq\sigma\circ\tau(x)$. However, from
proposition~(\ref{logic:prop:FUAP:freevarproof:substitution:inclusion})
we have $\free(\tau(\pi_{1}))\subseteq\tau(\free(\pi_{1}))$. It
follows that we have $v=\tau(u)$ for some $u\in\free(\pi_{1})$.
Furthermore, since $v\neq\tau(x)$ we have $u\neq x$ and consequently
$u\in\free(\gen x\pi_{1})$. Furthermore, recall that $\gen
x\pi_{1}\preceq\pi$. Having assumed that $\sigma\circ\tau$ is valid
for $\pi$ the following implication holds:
    \begin{equation}\label{logic:eqn:FUAP:validsub:composition:6}
    u\in\free(\gen x\pi_{1})\ \Rightarrow\
    \sigma\circ\tau(u)\neq\sigma\circ\tau(x)
    \end{equation}
Hence, we conclude that $\sigma\circ\tau(u)\neq\sigma\circ\tau(x)$
which is $\sigma(v)\neq\sigma\circ\tau(x)$ as requested.
\end{proof}

Let $\sigma:V\to W$ be a map and $\pi\in\pvs$. When attempting to
prove the validity of $\sigma$ for $\pi$, a very useful shortcut is
to argue that $\sigma(\pi)$ is in fact the same proof as $\tau(\pi)$
where $\tau:V\to W$ is a map which is known to be valid for~$\pi$.
The following proposition is the counterpart of
proposition~(\ref{logic:prop:FOPL:validsub:image}).
\begin{prop}\label{logic:prop:FUAP:validsubproof:equal:image}
Let $V,W$ be sets and $\sigma,\tau:V\to W$ be maps. Let $\pi\in\pvs$
such that the equality $\sigma(\pi)=\tau(\pi)$ holds. Then we have
the equivalence:
    \[
    (\mbox{$\sigma$ valid for $\pi$})\ \Leftrightarrow\
    (\mbox{$\tau$ valid for $\pi$})
    \]
\end{prop}
\begin{proof}
It is sufficient to prove $\Rightarrow$\,: so we assume that
$\sigma(\pi)=\tau(\pi)$ and that $\sigma$ is valid for $\pi$. We
need to show that $\tau$ is valid for $\pi$. Using
proposition~(\ref{logic:prop:FUAP:variable:support}) we see that
$\sigma$ and $\tau$ coincide on $\var(\pi)$. So let
$\phi\in\hyp(\pi)\cup\ax(\pi)$. We need to show that $\tau$ is valid
for $\phi$. However since $\sigma$ is valid for $\pi$, we know that
$\sigma$ is itself valid for $\phi$. Using
proposition~(\ref{logic:prop:FOPL:validsub:image}), in order to show
that $\tau$ is also valid for $\phi$ it is sufficient to prove that
$\sigma(\phi)=\tau(\phi)$. From
proposition~(\ref{logic:prop:substitution:support}) it is therefore
sufficient to show that $\sigma$ and $\tau$ coincide on
$\var(\phi)$. Having established that $\sigma$ and $\tau$ coincide
on $\var(\pi)$, we simply need to prove that
$\var(\phi)\subseteq\var(\pi)$. We shall distinguish two cases:
first we assume that $\phi\in\hyp(\pi)$. Then from
proposition~(\ref{logic:prop:FUAP:hypothesis:charac}) we have
$\phi\preceq\pi$ and using
proposition~(\ref{logic:prop:FUAP:variable:subformula}) it follows
that $\var(\phi)\subseteq\var(\pi)$ as requested. Next we assume
that $\phi\in\ax(\pi)$. Then from
proposition~(\ref{logic:prop:FUAP:axiomset:charac}) we see that
$\axi\phi\preceq\pi$ and again from
proposition~(\ref{logic:prop:FUAP:variable:subformula}) we have
$\var(\phi)=\var(\axi\phi)\subseteq\var(\pi)$. So we now consider
$\gen x\pi_{1}\preceq\pi$. From
proposition~(\ref{logic:prop:FUAP:validsubproof:criterion}), in
order to show that $\tau$ is valid for $\pi$, for all $u\in V$ we
need to show the implication:
    \begin{equation}\label{logic:eqn:FUAP:validsub:equal:image:1}
    u\in\free(\gen x\pi_{1})\ \Rightarrow\
    \tau(u)\neq\tau(x)
    \end{equation}
However, from the validity of $\sigma$ for $\pi$ we know the
following is true:
    \begin{equation}\label{logic:eqn:FUAP:validsub:equal:image:2}
    u\in\free(\gen x\pi_{1})\ \Rightarrow\
    \sigma(u)\neq\sigma(x)
    \end{equation}
In order to establish~(\ref{logic:eqn:FUAP:validsub:equal:image:1}),
since $\sigma$ and $\tau$ coincide on $\var(\pi)$ it is therefore
sufficient to prove that $\{u,x\}\subseteq\var(\pi)$. From $\gen
x\pi_{1}\preceq\pi$ and
proposition~(\ref{logic:prop:FUAP:variable:subformula})\,:
    \[
    x\in\var(\gen x\pi_{1})\subseteq\var(\pi)
    \]
So it remains to show that $u\in\var(\pi)$. Using
propositions~(\ref{logic:prop:FUAP:boundvarproof:var:free:bound})
and~(\ref{logic:prop:FUAP:variable:subformula})\,:
    \[
    u\in\free(\gen x\pi_{1})\subseteq \var(\gen x\pi_{1})\subseteq\var(\pi)
    \]
\end{proof}

Just as we did for formulas in
definition~(\ref{logic:def:FOPL:mintransform:transform}), we shall
define the notion of minimal transform for proofs in
definition~(\ref{logic:def:FUAP:mintransproof:transform}). The prime
motivation of minimal transforms is to replace all bound variables
of a formula or proof, with elements of a copy of \N\ which is
disjoint from $V$. In doing so, we ensure that any map $\sigma:V\to
W$ is always capture-avoiding when acting on the minimal transform
of a formula or proof. More precisely the map
$\bar{\sigma}:\bar{V}\to\bar{W}$ which is the minimal extension of
$\sigma$ as per
definition~(\ref{logic:def:FOPL:commute:minextensioon:map}), is
always valid for minimal transforms. The following proposition will
allow us to easily prove this validity and is the counterpart of
proposition~(\ref{logic:prop:FOPL:validsub:minimalextension}) which
was established for formulas.

\begin{prop}\label{logic:prop:FUAP:validsubproof:minimalextension}
Let $V$, $W$ be sets and $\sigma:V\to W$ be a map. Let $\pi\in\pvs$.
We assume that there exists a subset $V_{0}\subseteq V$ with the
following properties:
    \begin{eqnarray*}
    (i)&&\bound(\pi)\subseteq V_{0}\\
    (ii)&&\mbox{$\sigma_{|V_{0}}$ is injective}\\
    (iii)&&\sigma(V_{0})\cap\sigma(\var(\pi)\setminus
    V_{0})=\emptyset
    \end{eqnarray*}
Then the map $\sigma:V\to W$ is valid for the proof $\pi\in\pvs$.
\end{prop}
\begin{proof}
It is sufficient to show that properties $(i)$, $(ii)$ and $(iii)$
of proposition~(\ref{logic:prop:FUAP:validsubproof:criterion}) are
satisfied. First we show property $(i)$ and $(ii)$\,: so let
$\phi\in\hyp(\pi)\cup\ax(\pi)$. We need to show that $\sigma$ is
valid for $\phi$. Applying
proposition~(\ref{logic:prop:FOPL:validsub:minimalextension}) it is
sufficient to show that $\bound(\phi)\subseteq V_{0}$ and
$\sigma(V_{0})\cap\sigma(\var(\phi)\setminus V_{0})=\emptyset$.
Using~$(i)$ and~$(iii)$ it is therefore sufficient to prove that
$\var(\phi)\subseteq\var(\pi)$. We shall distinguish two cases:
first we assume that $\phi\in\hyp(\pi)$. Then from
proposition~(\ref{logic:prop:FUAP:hypothesis:charac}) we have
$\phi\preceq\pi$ and $\var(\phi)\subseteq\var(\pi)$ follows from
proposition~(\ref{logic:prop:FUAP:variable:subformula}). Next we
assume that $\phi\in\ax(\pi)$. Then from
proposition~(\ref{logic:prop:FUAP:axiomset:charac}) we have
$\axi\phi\preceq\pi$ and consequently
$\var(\phi)=\var(\axi\phi)\subseteq\var(\pi)$ as requested. We now
show property $(iii)$\,: Let $x\in V$ and $\pi_{1}\in\pvs$ be such
that $\gen x\pi_{1}\preceq\pi$. Given $u\in\free(\gen x\pi_{1})$ we
need to show that $\sigma(u)\neq\sigma(x)$. However, from
proposition~(\ref{logic:prop:FUAP:boundvarproof:subformula}) we have
$\bound(\gen x\pi_{1})\subseteq\bound(\pi)$ and consequently
$x\in\bound(\pi)$. From the assumption~$(i)$ it follows that $x\in
V_{0}$. We shall now distinguish two cases: first we assume that
$u\in V_{0}$. Then from assumption~$(ii)$, in order to show
$\sigma(u)\neq\sigma(x)$ it is sufficient to prove that $u\neq x$
which follows from $u\in\free(\gen x\pi_{1})$. We now assume that
$u\not\in V_{0}$. From
proposition~(\ref{logic:prop:FUAP:variable:subformula}) we have
$\var(\gen x\pi_{1})\subseteq\var(\pi)$ while from
proposition~(\ref{logic:prop:FUAP:boundvarproof:var:free:bound}),
$\free(\gen x\pi_{1})\subseteq\var(\gen x\pi_{1})$. Hence,
$u\in\var(\pi)$. It follows that $u\in\var(\pi)\setminus V_{0}$ and
$\sigma(u)\neq\sigma(x)$ is a consequence of~$(iii)$ and $x\in
V_{0}$.
\end{proof}
