We are now introducing the axioms which we intend to use as part of
our deductive system. These axioms come in five different groups
which we shall describe individually. The first three groups of
axioms are often referred to as {\em propositional axioms} while the
other two are specific to first order logic, dealing with
quantification. We have no axiom relating to equality of course,
having dismissed the equality predicate from our language. As
already explained prior to definition~(\ref{logic:def:FOPL:type}) of
page~\pageref{logic:def:FOPL:type}, we are hoping to define the
notion of equality in terms of the primitive symbol '$\in$', and
make equality axioms simple consequences of our specific coding. So
no equality for now. In choosing these five groups of axioms, we
have tried to keep it lean and simple, while reaching the greatest
possible consensus between the references available to us. The
axioms below are pretty much those of Donald W. Barnes and John M.
Mack~\cite{AlgLog}, Mikl\'os Ferenczi and Mikl\'os Sz\H
ots~\cite{Ferenczi}, P.T. Johnstone~\cite{Johnstone}, Elliot
Mendelson~\cite{Mendelson} and G. Takeuti and W.M.
Zaring~\cite{Takeuti}. They are referred to as {\em traditional
textbook axioms of predicate calculus} by Norman Megill's Metamath
web site~\cite{Metamath} which we find reassuring. However, these
references disagree on some minor details so we had to choose
between them. For example, contrary to~\cite{Ferenczi} we do not
wish to speak of {\em formula scheme}, which we feel do not bring
much mathematical value. We have also decided against~\cite{AlgLog}
and avoided as much congruence as possible when defining axioms, to
keep the material as down to earth as possible. Finally, we have
ignored a few oddities in~\cite{Johnstone} which were seemingly
motivated by P.T. Johnstone's desire to allow the {\em empty} model.

Our first group of axioms corresponds to 'Axiom ax-1' of
Metamath~\cite{Metamath}. We have decided to follow this reference
in calling an axiom within this group a {\em simplification axiom}.
It appears the terminology originates from the link between these
axioms and the formula $\phi_{1}\land\phi_{2}\to\phi_{1}$. Some
historical background can be found on the web page
\texttt{http://us.metamath.org/mpegif/ax-1.html}.
\index{axiom@Simplification axioms}\index{a@${\bf A}_{1}(V)$ :
simplification axioms}
\begin{defin}\label{logic:def:FOPL:simplification:axiom}
Let $V$ be a set. We call {\em simplification axiom} on \pv\ any
formula $\phi\in\pv$ for which there exist $\phi_{1},\phi_{2}\in\pv$
such that:
    \[
    \phi = \phi_{1}\to(\phi_{2}\to\phi_{1})
    \]
The set of simplification axioms on \pv\ is denoted ${\bf
A}_{1}(V)$.
\end{defin}

Our second group of axioms corresponds to 'Axiom ax-2' of
Metamath~\cite{Metamath}. We have again decided to follow this
reference in calling an axiom within this group a {\em Frege axiom}.
Further details and historical background can be found on the web
page \texttt{http://us.metamath.org/mpegif/ax-2.html}.
\index{axiom@Frege axioms}\index{a@${\bf A}_{2}(V)$ : Frege axioms}
\begin{defin}\label{logic:def:FOPL:frege:axiom}
Let $V$ be a set. We call {\em Frege axiom} on \pv\ any formula
$\phi\in\pv$ for which there exist
$\phi_{1},\phi_{2},\phi_{3}\in\pv$ such that:
    \[
    \phi =
    [\phi_{1}\to(\phi_{2}\to\phi_{3})]\to[(\phi_{1}\to\phi_{2})\to(\phi_{1}\to\phi_{3})]
    \]
The set of Frege axioms on \pv\ is denoted ${\bf A}_{2}(V)$.
\end{defin}

Our third group of axioms corresponds to 'Axiom ax-3' of
Metamath~\cite{Metamath} which can be found
on~\texttt{http://us.metamath.org/mpegif/ax-3.html}. However, we
chose a different form for these axioms, namely
$[(\phi_{1}\to\bot)\to\bot]\to\phi_{1}$ following~\cite{AlgLog}
and~\cite{Johnstone} rather than
$(\lnot\phi_{1}\to\lnot\phi_{2})\to(\phi_{2}\to\phi_{1})$ which is
also the form encountered in~\cite{Ferenczi}. It is easy to believe
that both choices lead to the same notion of provability. Our
terminology is {\em transposition axiom} following~\cite{Metamath}.
\index{axiom@Transposition axioms}\index{a@${\bf A}_{3}(V)$ :
transposition axioms}
\begin{defin}\label{logic:def:FOPL:transposition:axiom}
Let $V$ be a set. We call {\em transposition axiom} on \pv\ any
formula $\phi\in\pv$ for which there exists $\phi_{1}\in\pv$ such
that:
    \[
    \phi =[(\phi_{1}\to\bot)\to\bot]\to\phi_{1}
    \]
The set of transposition axioms on \pv\ is denoted ${\bf A}_{3}(V)$.
\end{defin}

Our fourth group of axioms is that of 'Theorem stdpc5' in
Metamath~\cite{Metamath} which can be found
on~\texttt{http://us.metamath.org/mpegif/stdpc5.html}. Norman Megill
has this as a theorem rather than an axiom but it is still mentioned
on \texttt{http://us.metamath.org/mpegif/mmset.html\#traditional} as
a {\em traditional textbook axiom}. It is indeed a chosen axiom
of~\cite{AlgLog}, \cite{Ferenczi} and~\cite{Johnstone}. As we were
not able to find an obvious name, we chose {\em quantification
axiom}.
\index{axiom@Quantification axioms}\index{a@${\bf A}_{4}(V)$
: quantification axioms}
\begin{defin}\label{logic:def:FOPL:quantification:axiom}
Let $V$ be a set. We call {\em quantification axiom} on \pv\ any
formula $\phi\in\pv$ for which there exist $\phi_{1},\phi_{2}\in\pv$
and $x\in V$ such that:
    \[
    x\not\in\free(\phi_{1})\ ,\ \phi =\forall x(\phi_{1}\to\phi_{2})\to(\phi_{1}
    \to\forall x \phi_{2})
    \]
The set of quantification axioms on \pv\ is denoted ${\bf
A}_{4}(V)$.
\end{defin}
Our fifth and last group of axioms is 'Theorem stdpc4' of
Metamath~\cite{Metamath} which can be found
on~\texttt{http://us.metamath.org/mpegif/stdpc4.html}. Following
this reference once again, we have decided to call axioms within
this group a {\em specialization axiom}. It is a chosen group of
axioms for~\cite{AlgLog}, \cite{Ferenczi} and~\cite{Johnstone} with
varying wordings in relation to the highly delicate problem of
making sure variable substitutions are valid. Indeed, a
specialization axiom is of the form:
    \begin{equation}\label{logic:eqn:FOPL:axioms:specialization:eqn1}
    \forall x\phi_{1}\to\phi_{1}[y/x]
    \end{equation}
Strictly speaking, this formula will have a different meaning, and
require different verbal qualifications, depending on the specifics
of how the substitution $[y/x]:\pv\to\pv$ is defined. For example,
most references in the literature will define $[y/x]$ with a
recursive formula in which {\em bound occurrences of} $x$ are not
replaced by $y$. This is contrary to our own
definition~(\ref{logic:def:substitution}) of
page~\pageref{logic:def:substitution} where all variables are
systematically substituted regardless of whether they are free or
bound occurrences. As it turns out, we did not rely on
definition~(\ref{logic:def:substitution}) to define specialization
axioms, but defined them instead in terms of an {\em essential
substitution} $[y/x]:\pv\to\pv$ as per
definition~(\ref{logic:def:FOPL:esssubstprop:essential}). In other
words, if $[y/x]:\pv\to\pv$ denotes an essential substitution
associated with the map $[y/x]:V\to V$ of
definition~(\ref{logic:def:single:var:substitution}), then the
formula~(\ref{logic:eqn:FOPL:axioms:specialization:eqn1}) is a
legitimate {\em specialization axiom}. In a more condensed form,
$\forall x\phi_{1}\to\phi_{1}[y/x]$ is an axiom whenever $[y/x]$ is
an {\em essential substitution of $y$ in place of $x$}. This is
beautiful, this is short and does not require any further
qualification such as '{\em provided $y$ is free for $x$ in
$\phi_{1}$}', i.e. provided $[y/x]$ is valid for $\phi_{1}$ as per
definition~(\ref{logic:def:FOPL:valid:substitution}). There is
however a huge disadvantage in using essential substitutions when
defining specialization axioms: the level of mathematical
sophistication required to follow the analysis is a lot higher and
most people will stop reading. Surely we do not want that. Another
drawback is that essential substitutions are intertwined with the
notion of alpha equivalence. If $\sigma:\pv\to \pv$ is an
essential substitution and $\phi\in\pv$ then $\sigma(\phi)$ is of
course a well defined formula of \pv\ but the specifics of
$\sigma(\phi)$ are usually unknown. We typically only know about the
class of $\sigma(\phi)$ modulo alpha equivalence. Luckily
this is often all we care about, but it does mean we are not able to
provide an axiomatization of first order logic without dependencies
to the alpha equivalence. In the light of these
considerations, our preferred option would have been to define the
specialization axioms in line with the existing literature:
    \begin{equation}\label{logic:eqn:FOPL:axioms:specialization:eqn2}
    \forall x\phi_{1}\to\phi_{1}[y/x]\ ,\ \mbox{where $[y/x]$ is valid for $\phi_{1}$}
    \end{equation}
where the substitution $[y/x]:\pv\to\pv$ is now simply the
substitution associated with $[y/x]:V\to V$ as per
definition~(\ref{logic:def:substitution}). This would have the
advantage of being simple, consistent with the usual practice and
devoid of any reference to alpha equivalence. The fact
that our substitutions of definition~(\ref{logic:def:substitution})
have an impact on bound occurrences of variables is contrary to
standard practice, but of little significance. The notion of {\em
valid substitution} of
definition~(\ref{logic:def:FOPL:valid:substitution}) is not commonly
found in textbooks, but is a nice and simple generalization of the
idea that {\em $y$ is free for $x$ in $\phi_{1}$}. So why not simply
use~(\ref{logic:eqn:FOPL:axioms:specialization:eqn2}) to define
specialization axioms? Why introduce the elaborate notion of {\em
essential substitutions} and seemingly make everything more
complicated with no visible benefit? The answer is this: we do not
have a choice. This document is devoted to the study of the algebra
\pv\ where $V$ is a set of arbitrary cardinality. In particular $V$
can be a finite set. By contrast, the existing literature focusses
on first order languages where the number of variables is typically
countably infinite. This makes a big difference. We believe our
insistence on allowing $V$ to be finite will be rewarded. We cannot
be sure at this stage, but we are quite certain the theory of vector
spaces was not solely developed in the infinite dimensional case.
There must be value in the study of first order languages which are
{\em finite dimensional}. This is of course no more than a hunch,
but we are not quite ready to give it up. So let us understand why
allowing $V$ to be finite changes everything: consider $V=\{x,y\}$
and $\phi=\forall x\phi_{1}=\forall x\forall y (x\in y)$ where
$x\neq y$. We know from
theorem~(\ref{logic:the:FOPL:esssubst:existence}) of
page~\pageref{logic:the:FOPL:esssubst:existence} that there exists
an essential substitution $[y/x]:\pv\to\pv$ associated with
$[y/x]:V\to V$. It is not difficult to prove that the only possible
value for $[y/x](\phi_{1})$ is $\forall x(y\in x)$. So because we
have chosen to define specialization axioms in terms of essential
substitutions, from $\forall x\phi_{1}\to\phi_{1}[x/x]$ and $\forall
x\phi_{1}\to\phi_{1}[y/x]$ we obtain:
    \begin{eqnarray*}
    (i)&&\forall x\forall y(x\in y)\to\forall y(x\in y)\\
    (ii)&&\forall x\forall y(x\in y)\to\forall x(y\in x)
    \end{eqnarray*}
as the two possible specialization axioms associated with
$\phi=\forall x\phi_{1}$. However, suppose we had opted
for~(\ref{logic:eqn:FOPL:axioms:specialization:eqn2}) to define
specialization axioms. Then the formula $\forall
x\phi_{1}\to\phi_{1}[y/x]$ would need to be excluded as an axiom
because $[y/x]$ is clearly not a valid substitution for
$\phi_{1}=\forall y(x\in y)$. So the formula~$(ii)$ above would
seemingly not be an axiom for us. Does it matter? Well in general
no: for in general we have a spare variable $z$ to work with. So
suppose $\{x,y,z\}\subseteq V$ with $x,y,z$ distinct. Then from
$\forall x\forall y(x\in y)$ we can prove $(z\in x)$ by successive
specialization. By successive generalization we obtain $\forall
z\forall x(z\in x)$ and finally by specializing once more we
conclude that $\forall x(y\in x)$. It follows that the
formula~$(ii)$ above would fail to be an axiom, but it would still
be a theorem of our deductive system, keeping the resulting notion
of provability unchanged. However, in the case when $V=\{x,y\}$ we
have a problem. There doesn't seem to be a way to turn the
formula~$(ii)$ into a theorem. Does it matter? Yes it does. It is
clear that $\forall x\forall y(x\in y)\to\forall x(y\in x)$ is going
to be true in any model, under any variables assignment. Unless it
is also a theorem, G\"odel's completeness theorem will fail, which
we do not want in a successful axiomatization of first order logic.
We are now in a position to state our chosen definition:
\index{axiom@Specialization axioms}\index{a@${\bf A}_{5}(V)$ :
specialization axioms}\index{essential@$[y/x]$ : essential $y$ in
place of $x$}
\begin{defin}\label{logic:def:FOPL:specialization:axiom}
Let $V$ be a set. We call {\em specialization axiom} on \pv\ any
formula $\phi\in\pv$ for which there exist $\phi_{1}\in\pv$ and
$x,y\in V$ such that:
    \[
    \phi =\forall x\phi_{1}\to\phi_{1}[y/x]
    \]
where $[y/x]:\pv\to\pv$ is an essential substitution of $y$ in place
of $x$. The set of specialization axioms on \pv\ is denoted ${\bf
A}_{5}(V)$.
\end{defin}
We know from
proposition~(\ref{logic:prop:FOPL:esssubstprop:redefine}) that an
essential substitution $\sigma:\pv\to\pv$ can be redefined
arbitrarily without changing its associated map $\sigma:V\to V$, as
long as such re-definition preserves classes modulo alpha equivalence. 
It follows that if $[y/x]:\pv\to\pv$ is an essential
substitution of $y$ in place of $x$, it remains an essential
substitution of $y$ in place of $x$ after re-definition modulo
substitution. It follows that any formula $\forall
x\phi_{1}\to\phi_{1}^{*}$ is a specialization axiom, as long as
$\phi_{1}^{*}\sim\phi_{1}[y/x]$, where $\sim$ denotes the
alpha equivalence:
\begin{prop}\label{logic:prop:FOPL:specialization:axiom:2}
Let $V$ be a set. Let $\phi_{1}\in\pv$ and $x,y\in V$. Let
$\phi_{1}^{*}\in\pv$ such that $\phi_{1}^{*}\sim\phi_{1}[y/x]$ where
$[y/x]:\pv\to\pv$ is an essential substitution of $y$ in place of
$x$. Then the formula $\phi=\forall x\phi_{1}\to\phi_{1}^{*}$ is a
specialization axiom.
\end{prop}
\begin{proof}
Suppose $\phi_{1}^{*}\sim\phi_{1}[y/x]$ where $[y/x]:\pv\to\pv$ is
an essential substitution associated with the map $[y/x]:V\to V$.
Define $\sigma:\pv\to\pv$ by setting $\sigma(\psi)=[y/x](\psi)$ if
$\psi\neq\phi_{1}$ and $\sigma(\phi_{1})=\phi_{1}^{*}$. Having
assumed that $\phi_{1}^{*}\sim\phi_{1}[y/x]$ we see that
$\sigma(\psi)\sim[y/x](\psi)$ for all $\psi\in\pv$. It follows from
proposition~(\ref{logic:prop:FOPL:esssubstprop:redefine}) that
$\sigma$ is also an essential substitution associated with
$[y/x]:V\to V$. In other words, $\sigma$ is also an essential
substitution of $y$ in place of $x$. Using
definition~(\ref{logic:def:FOPL:specialization:axiom}) it follows
that $\phi=\forall x\phi_{1}\to\sigma(\phi_{1})$ is a specialization
axiom. From $\sigma(\phi_{1})=\phi_{1}^{*}$ we conclude that
$\phi=\forall x\phi_{1}\to\phi_{1}^{*}$ is a specialization axiom as
requested.
\end{proof}
\index{axiom@Axioms of first order logic}\index{a@${\bf A}(V)$ :
axioms of FOL}
\begin{defin}\label{logic:def:FOPL:first:order:axiom}
Let $V$ be a set. We call {\em axiom of first order logic} on \pv\
any formula $\phi\in\pv$ which belongs to the set \av\ defined by:
    \[
    \av={\bf A}_{1}(V)\cup{\bf A}_{2}(V)\cup{\bf A}_{3}(V)\cup{\bf A}_{4}(V)
    \cup{\bf A}_{5}(V)
    \]
\end{defin}
