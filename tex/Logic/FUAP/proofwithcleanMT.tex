As you may recall, we defined a minimal transform for proofs which
naturally led to the notion of substitution congruence on \pvs. We
gave a result to characterize this $\alpha$-equivalence in
theorem~(\ref{logic:the:FUAP:charsubcong:charac}) of
page~\pageref{logic:the:FUAP:charsubcong:charac}. We also
established the equivalence between $\pi\sim\rho$ and ${\cal
M}(\pi)={\cal M}(\rho)$ in
theorem~(\ref{logic:the:FUAP:mintransfsubcong:kernel}) of
page~\pageref{logic:the:FUAP:mintransfsubcong:kernel} which serves
as a neat validation of our definition choices so far. One other
thing to check is that two equivalent proofs are indeed {\em the
same}. Specifically, if $\sim$ denotes the substitution congruence
on \pvs\ and $\pi\sim\rho$, we want to make sure the {\em
fundamental properties} of $\pi$ are also those of $\rho$. So if
$\pi$ is a clean proof, we should expect $\rho$ to be clean as well.
We should also expect $\pi$ and $\rho$ to have equivalent valuation
modulo, i.e. $\vals(\pi)\sim\vals(\rho)$ where $\sim$ now denotes
the substitution congruence on \pv. Finally, we would want to check
we have the equality $\hyp(\pi)\sim\hyp(\rho)$ modulo the
substitution congruence, as per
definition~(\ref{logic:def:FUAP:valuationmod:equality:modulo}). In
short, two equivalent proofs should prove the same thing from the
same hypothesis, modulo $\alpha$-equivalence, and they should both
be clean if one of them is. These results will be established in the
next section. For now, we shall focus on proving the following
proposition from which every thing else will flow. From
proposition~(\ref{logic:prop:FUAP:mintransproof:valuation:commute})
we already know that  the minimal transform of a clean proof is
clean. The following proposition establishes the reverse implication
$({\cal M}(\pi)\mbox{ is clean })\ \Rightarrow\ (\pi\mbox{ is
clean})$. This type of result requires proofs which are more
elaborate than usual. It is very similar to
lemma~(\ref{logic:lemma:FUAP:valsubaxmodulo:m:phi:phi}) where we
proved the implication $({\cal M}(\phi)\mbox{ is an axiom })\
\Rightarrow\ (\phi\mbox{ is an axiom modulo})$. As we shall see, the
proof of
proposition~(\ref{logic:prop:FUAP:proofwithcleanMT:mintrans:clean:equivalence})
below will rely on the local inversion theorem for proofs, which is
theorem~(\ref{logic:the:FUAP:localinversion:inversion}) of
page~\pageref{logic:the:FUAP:localinversion:inversion}. Until now,
the local inversion theorem was only ever used to prove the
implication ${\cal M}(\pi)={\cal M}(\rho)\ \Rightarrow\
\pi\sim\rho$.

\index{clean@Proof with clean min. transform}
\begin{prop}\label{logic:prop:FUAP:proofwithcleanMT:mintrans:clean:equivalence}
Let $V$ be a set and $\pi\in\pvs$. Then we have the equivalence:
    \[
    (\,\mbox{$\pi$ is clean}\,)\ \Leftrightarrow\ (\,\mbox{${\cal M}(\pi)$ is clean}\,)
    \]
where ${\cal M}(\pi)$ is the minimal transform of $\pi$ as per {\em
definition~(\ref{logic:def:FUAP:mintransproof:transform}).}
\end{prop}
\begin{proof}
The implication $\Rightarrow$ is already known from
proposition~(\ref{logic:prop:FUAP:mintransproof:valuation:commute}).
We shall prove $\Leftarrow$ with an induction argument, using
theorem~(\ref{logic:the:proof:induction}) of
page~\pageref{logic:the:proof:induction}. First we assume that
$\pi=\phi$ for some $\phi\in\pv$. Then from
definition~(\ref{logic:def:FUAP:almostclean:definition}) we see that
$\pi$ is always clean in this case, so our implication is true. Next
we assume that $\pi=\axi\phi$ for some $\phi\in\pv$. We need to show
our implication is true for $\pi$. So we assume that ${\cal M}(\pi)$
is clean. We need to show that $\pi$ is also clean. However, we have
${\cal M}(\pi)=\axi{\cal M}(\phi)$ and it follows that ${\cal
M}(\phi)$ is an axiom modulo, i.e. ${\cal M}(\phi)\in{\bf
A}^{+}(\bar{V})$. Using
proposition~(\ref{logic:prop:FUAP:valsubaxmodulo:min:transform}) we
see that $\phi\in\avs$, i.e. $\phi$ is itself an axiom modulo. So
$\pi$ is clean as requested. We now assume that
$\pi=\pi_{1}\pon\pi_{2}$, where $\pi_{1},\pi_{2}\in\pvs$ are proofs
satisfying our implication. We need to show the same is true of
$\pi$. So we assume that ${\cal M}(\pi)={\cal M}(\pi_{1})\pon\,{\cal
M}(\pi_{2})$ is clean. We need to show that $\pi$ is itself clean.
However, from
proposition~(\ref{logic:prop:FUAP:almostclean:modus:ponens}) we see
that both ${\cal M}(\pi_{1})$ and ${\cal M}(\pi_{2})$ are clean, and
furthermore $\vals\circ{\cal M}(\pi_{2})=\chi_{1}\to\chi_{2}$ for
some $\chi_{1},\chi_{2}\in\pvb$ such that
$\chi_{1}\sim\vals\circ{\cal M}(\pi_{1})$, where $\sim$ refers to
the substitution congruence on \pvb. Having assumed our implication
is true for $\pi_{1}$ and $\pi_{2}$, it follows that both $\pi_{1}$
and $\pi_{2}$ are clean. Using
proposition~(\ref{logic:prop:FUAP:almostclean:modus:ponens}), in
order to show that $\pi$ is clean it remains to prove that
$\vals(\pi_{2})=\psi_{1}\to\psi_{2}$ for some
$\psi_{1},\psi_{2}\in\pv$ such that $\psi_{1}\sim\vals(\pi_{1})$,
where $\sim$ is now the substitution congruence on \pv. However,
using
proposition~(\ref{logic:prop:FUAP:mintransproof:valuation:commute})
and the fact that $\pi_{2}$ is clean, we obtain ${\cal
M}\circ\vals(\pi_{2})\sim\vals\circ{\cal
M}(\pi_{2})=\chi_{1}\to\chi_{2}$. From
theorem~(\ref{logic:the:sub:congruence:charac}) of
page~\pageref{logic:the:sub:congruence:charac} it follows that
${\cal M}\circ\vals(\pi_{2})=\chi_{1}'\to\chi_{2}'$ for some
$\chi_{1}',\chi_{2}'\in\pvb$ such that $\chi_{1}\sim\chi_{1}'$ and
$\chi_{2}\sim\chi_{2}'$. Looking at
definition~(\ref{logic:def:FOPL:mintransform:transform}) of the
minimal transform and arguing from the uniqueness property of
theorem~(\ref{logic:the:unique:representation}) or
page~\pageref{logic:the:unique:representation}, we see that
$\vals(\pi_{2})$ must be of the form
$\vals(\pi_{2})=\psi_{1}\to\psi_{2}$ for some
$\psi_{1},\psi_{2}\in\pv$ such that $\chi_{1}'={\cal M}(\psi_{1})$.
In order to show that $\pi$ is clean, it remains to prove that
$\psi_{1}\sim\vals(\pi_{1})$. From
theorem~(\ref{logic:the:FOPL:mintransfsubcong:kernel}) of
page~\pageref{logic:the:FOPL:mintransfsubcong:kernel}, this amounts
to showing the equality ${\cal M}(\psi_{1})={\cal
M}\circ\vals(\pi_{1})$, for which it is in fact sufficient to show
the equivalence ${\cal M}(\psi_{1})\sim{\cal M}\circ\vals(\pi_{1})$
by virtue of
proposition~(\ref{logic:prop:FOPL:esssubst:mintransform:equiv:imp:equal}).
Since $\pi_{1}$ is clean, we can apply
proposition~(\ref{logic:prop:FUAP:mintransproof:valuation:commute})
and obtain:
    \begin{eqnarray*}
    {\cal M}(\psi_{1})&=&\chi_{1}'\\
    &\sim&\chi_{1}\\
    &\sim&\vals\circ{\cal M}(\pi_{1})\\
    \mbox{prop.~(\ref{logic:prop:FUAP:mintransproof:valuation:commute})}\ \rightarrow
    &\sim&{\cal M}\circ\vals(\pi_{1})\\
    \end{eqnarray*}
So we have proved that $\pi$ is clean as requested. We now assume
that $\pi=\gen x\pi_{1}$ for some $x\in V$ and $\pi_{1}\in\pvs$
satisfying our implication. We need to show the same is true of
$\pi$. So we assume that ${\cal M}(\pi)$ is clean. We need to show
that $\pi$ is itself clean. Using
proposition~(\ref{logic:prop:FUAP:almostclean:generalization}), it
is sufficient to show that $\pi_{1}$ is clean and furthermore that
$x\not\in\spec(\pi_{1})$. However from
definition~(\ref{logic:def:FUAP:mintransproof:transform}) we obtain
the equality ${\cal M}(\pi)=\gen n{\cal M}(\pi_{1})[n/x]$ where
$n\in\N$ is the smallest natural number for which $[n/x]$ is valid
for ${\cal M}(\pi_{1})$. Having assumed that ${\cal M}(\pi)$ is
clean, it follows from
proposition~(\ref{logic:prop:FUAP:almostclean:generalization}) that
${\cal M}(\pi_{1})[n/x]$ is clean and furthermore
$n\not\in\spec({\cal M}(\pi_{1})[n/x])$. So let us show that
$\pi_{1}$ is clean: Consider the substitution
$\sigma:\bar{V}\to\bar{V}$ defined by $\sigma=[n/x]$. We shall first
show that ${\cal M}(\pi_{1})$ is clean using the local inversion
theorem~(\ref{logic:the:FUAP:localinversion:inversion}) of
page~\pageref{logic:the:FUAP:localinversion:inversion}. So consider
the sets $V_{0}=V$ and $V_{1}=\N$. It is clear that both
$\sigma_{|V_{0}}$ and $\sigma_{|V_{1}}$ are injective maps. Define:
    \[
    \Pi=\{\kappa\in{\bf\Pi}(\bar{V}):(\free(\kappa)\subseteq
    V_{0})\land(\bound(\kappa)\subseteq V_{1})\land(\mbox{$\sigma$
    valid for $\kappa$})\}
    \]
Then using theorem~(\ref{logic:the:FUAP:localinversion:inversion})
there exits $\tau:{\bf\Pi}(\bar{V})\to{\bf\Pi}(\bar{V})$ such that
$\tau\circ\sigma(\kappa)=\kappa$ for all $\kappa\in\Pi$. Using
propositions~(\ref{logic:prop:FUAP:mintransformproof:freevar}),
(\ref{logic:prop:FUAP:mintransformproof:boundvar}) and the fact that
$[n/x]$ is valid for ${\cal M}(\pi_{1})$, it follows that ${\cal
M}(\pi_{1})\in\Pi$. Thus we obtain $\tau({\cal
M}(\pi_{1})[n/x])={\cal M}(\pi_{1})$. Since ${\cal M}(\pi_{1})[n/x]$
is clean, in order to show that ${\cal M}(\pi_{1})$ is itself clean
it is sufficient to prove that $\tau$ is valid for ${\cal
M}(\pi_{1})[n/x]$, by virtue for
proposition~(\ref{logic:prop:FUAP:strongvalsubalmostclean:valuation:commute}).
Using proposition~(\ref{logic:prop:FUAP:validsubproof:composition}),
it is therefore sufficient to prove that $\tau\circ[n/x]$ is valid
for ${\cal M}(\pi_{1})$, which follows from
proposition~(\ref{logic:prop:FUAP:validsubproof:equal:image}) and
$\tau({\cal M}(\pi_{1})[n/x])={\cal M}(\pi_{1})$. So we have proved
that ${\cal M}(\pi_{1})$ is a clean proof and consequently $\pi_{1}$
itself, as follows from the induction hypothesis. We now show that
$x\not\in\spec(\pi_{1})$: since $[n/x]$ is valid for ${\cal
M}(\pi_{1})$, from
proposition~(\ref{logic:prop:FUAP:validsubproof:specvar}) we have
the equality $\spec({\cal M}(\pi_{1})[n/x])=[n/x](\,\spec({\cal
M}(\pi_{1}))\,)$. Hence we have $n\not\in[n/x](\,\spec({\cal
M}(\pi_{1}))\,)$ and consequently $x\not\in\spec({\cal
M}(\pi_{1}))$. However having established that $\pi_{1}$ is a clean
proof, from
proposition~(\ref{logic:prop:FUAP:mintransproof:variable}) we have
$\spec({\cal M}(\pi_{1}))=\spec(\pi_{1})$. We conclude that
$x\not\in\spec(\pi_{1})$ as requested, which completes our induction
argument.
\end{proof}
