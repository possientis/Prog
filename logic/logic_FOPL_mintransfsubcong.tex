Let $\phi=\forall x\forall y(x\in y)$ and $\psi=\forall y\forall
x(y\in x)$, where $x\neq y$. We know that $\phi\sim\psi$ where
$\sim$ denotes the substitution congruence on \pv. The quickest way
to see this is to define $\phi_{1}=\forall y(x\in y)$ and argue
using definition~(\ref{logic:def:sub:congruence}) of
page~\pageref{logic:def:sub:congruence} that $\phi=\forall
x\phi_{1}$ while $\psi=\forall y\phi_{1}[y\!:\!x]$ with
$y\not\in\free(\phi_{1})$. If we now look at minimal transforms,
since ${\cal M}(\phi_{1})=\forall\,0\,(x\in 0)$ and ${\cal
M}(\phi_{1}[y\!:\!x]) = \forall\,0\,(y\in 0)$ we obtain:
    \[
    {\cal M}(\phi) = \forall\,1\forall\,0\,(1\in 0)={\cal M}(\psi)
    \]
So here is a case of two equivalent formulas with identical minimal
transforms. Of course, this is hardly surprising. We designed the
minimal transform so as to replace the bound variables with a unique
and sensible choice. In fact, we should expect the equality ${\cal
M}(\phi)={\cal M}(\psi)$ to hold whenever $\phi\sim\psi$.
Conversely, suppose $\chi\in\pv$ is a formula with minimal
transform:
    \[
    {\cal M}(\chi)=\forall\,1\forall\,0\,(1\in 0)
    \]
It is hard to imagine a formula $\chi$ which is not equivalent to
$\phi$ and $\psi$. It seems pretty obvious that $\chi$ would need to
be of the form $\chi=\forall u\forall v(u\in v)$ and it would not be
possible to have $u=v$. Hence:
    \[
    \chi=\forall u\forall v(u\in v)\sim\forall u\forall y(u\in
    y)\sim\forall x\forall y(x\in y)=\phi
    \]
The main theorem of this section will show that $\phi\sim\psi$ is
indeed equivalent to ${\cal M}(\phi)={\cal M}(\psi)$. As a preamble,
we shall prove:
\begin{prop}\label{logic:prop:FOPL:mintransfsubcong:congruence}
Let $V$ be a set and $\equiv$ be the relation on \pv\ defined by:
    \[
    \phi\equiv\psi\ \Leftrightarrow\ {\cal M}(\phi)={\cal M}(\psi)
    \]
for all $\phi,\psi\in\pv$. Then $\equiv$ is a congruence on \pv.
\end{prop}
\begin{proof}
The relation $\equiv$ is clearly reflexive, symmetric and
transitive. So it is an equivalence relation on \pv\ and we simply
need to prove that it is a congruent relation, as per
definition~(\ref{logic:def:congruent:relation}) of
page~\pageref{logic:def:congruent:relation}. We already know that
$\bot\equiv\bot$. So let $\phi=\phi_{1}\to\phi_{2}$ and
$\psi=\psi_{1}\to\psi_{2}$ where $\phi_{1}\equiv\psi_{1}$ and
$\phi_{2}\equiv\psi_{2}$. We need to show that $\phi\equiv\psi$.
This goes as follows:
    \begin{eqnarray*}
    {\cal M}(\phi)&=&{\cal M}(\phi_{1}\to\phi_{2})\\
    &=&{\cal M}(\phi_{1})\to{\cal M}(\phi_{2})\\
    (\phi_{1}\equiv\psi_{1})\land(\phi_{2}\equiv\psi_{2})\ \rightarrow
    &=&{\cal M}(\psi_{1})\to{\cal M}(\psi_{2})\\
    &=&{\cal M}(\psi_{1}\to\psi_{2})\\
    &=&{\cal M}(\psi)
    \end{eqnarray*}
Next we assume that $\phi=\forall x\phi_{1}$ and $\psi=\forall
x\psi_{1}$ where $x\in V$ and $\phi_{1}\equiv\psi_{1}$. We need to
show that $\phi\equiv\psi$. This goes as follows:
    \begin{eqnarray*}
    {\cal M}(\phi)&=&{\cal M}(\forall x\phi_{1})\\
    \mbox{$n=\min\{k:[k/x]\mbox{ valid for }{\cal M}(\phi_{1})\}$}\
    \rightarrow&=&\forall n{\cal M}(\phi_{1})[n/x]\\
    \phi_{1}\equiv\psi_{1}\ \rightarrow&=&\forall n{\cal M}(\psi_{1})[n/x]\\
    \phi_{1}\equiv\psi_{1}\ \rightarrow&=&\forall m{\cal M}(\psi_{1})[m/x]\\
    \mbox{$m=\min\{k:[k/x]\mbox{ valid for }{\cal M}(\psi_{1})\}$}\
    \rightarrow&=&{\cal M}(\forall x\psi_{1})\\
    &=&{\cal M}(\psi)
    \end{eqnarray*}
\end{proof}
\index{minimal@Minimal transform and $\sim$}
\begin{theorem}\label{logic:the:FOPL:mintransfsubcong:kernel}
Let $\sim$ be the substitution congruence on \pv\ where $V$ is a
set. Then for all $\phi,\psi\in\pv$ we have the equivalence:
    \[
    \phi\sim\psi\ \Leftrightarrow\ {\cal M}(\phi)={\cal M}(\psi)
    \]
where ${\cal M}(\phi)$ and ${\cal M}(\psi)$ are the minimal
transforms as per {\em
definition~(\ref{logic:def:FOPL:mintransform:transform})}.
\end{theorem}
\begin{proof}
First we show $\Rightarrow$\,: consider the relation $\equiv$ on
\pv\ defined by $\phi\equiv\psi$ \ifand\ ${\cal M}(\phi)={\cal
M}(\psi)$. We need to show the inclusion
$\sim\,\subseteq\,\equiv$\,. However, we know from
proposition~(\ref{logic:prop:FOPL:mintransfsubcong:congruence}) that
$\equiv$ is a congruence on \pv\ and furthermore from
proposition~(\ref{logic:prop:sub:congruence:from:admissible}) that
$\sim$ is generated by the set:
    \[
    R_{1}=\left\{\,(\,\phi\,,\,\sigma(\phi)\,):\phi\in\pv\ ,\
    \mbox{$\sigma:V\to V$ admissible for $\phi$} \right\}
    \]
It is therefore sufficient to prove the inclusion
$R_{1}\subseteq\,\equiv$\,. So let $\sigma:V\to V$ be an admissible
substitution for $\phi$ as per
definition~(\ref{logic:def:admissible:substitution}) of
page~\pageref{logic:def:admissible:substitution}. We need to show
that $\phi\equiv\sigma(\phi)$, i.e. that ${\cal M}(\phi)={\cal
M}\circ\sigma(\phi)$. Let $\bar{\sigma}:\bar{V}\to\bar{V}$ be the
minimal extension of $\sigma$ as per
definition~(\ref{logic:def:FOPL:commute:minextensioon:map}) of
page~\pageref{logic:def:FOPL:commute:minextensioon:map}. Since
$\sigma$ is admissible for $\phi$, in particular $\sigma$ is valid
for $\phi$. So we can apply
theorem~(\ref{logic:the:FOPL:commute:mintransform:validsub}) of
page~\pageref{logic:the:FOPL:commute:mintransform:validsub} from
which we obtain ${\cal M}\circ\sigma(\phi)=\bar{\sigma}\circ{\cal
M}(\phi)$. It is therefore sufficient to prove that ${\cal
M}(\phi)=\bar{\sigma}\circ{\cal M}(\phi)$. Let $i:\bar{V}\to\bar{V}$
be the identity mapping. We need to show that $i\circ {\cal
M}(\phi)=\bar{\sigma}\circ{\cal M}(\phi)$ and from
proposition~(\ref{logic:prop:substitution:support}) it is sufficient
to prove that $i$ and $\bar{\sigma}$ coincide on $\var({\cal
M}(\phi))$. So let $u\in\var({\cal M}(\phi))$. We need to show that
$\bar{\sigma}(u)=u$. Since $\bar{V}$ is the disjoint union of $V$
and \N, we shall distinguish two cases: first we assume that
$u\in\N$. Then $\bar{\sigma}(u)=u$ is immediate from
definition~(\ref{logic:def:FOPL:commute:minextensioon:map}). Next we
assume that $u\in V$. Then from $u\in\var({\cal M}(\phi))$ and
proposition~(\ref{logic:prop:FOPL:mintransform:variables}) it
follows that $u\in\free(\phi)$. Having assumed that $\sigma$ is
admissible for $\phi$ we conclude that $\bar{\sigma}(u)=\sigma(u)=u$
as requested. We now prove~$\Leftarrow$\,: we need to show that
every $\phi\in\pv$ satisfies the property:
    \[
    \forall\psi\in\pv\ ,\ [\,{\cal M}(\phi)={\cal M}(\psi)\
    \Rightarrow\ \phi\sim\psi\,]
    \]
We shall do so by a structural induction argument using
theorem~(\ref{logic:the:proof:induction}) of
page~\pageref{logic:the:proof:induction}. First we assume that
$\phi=(x\in y)$ for some $x,v\in V$. Let $\psi\in\pv$ such that
${\cal M}(\phi)=(x\in y)={\cal M}(\psi)$. We need to show that
$\phi\sim\psi$. So it is sufficient to prove that $\phi=\psi$. From
theorem~(\ref{logic:the:unique:representation}) of
page~\pageref{logic:the:unique:representation} the formula $\psi$
can be of one and only one of four types: first it can be of the
form $\psi=(u\in v)$ for some $u,v\in V$. Next, it can be of the
form $\psi=\bot$, and possibly of the form
$\psi=\psi_{1}\to\psi_{2}$ with $\psi_{1},\psi_{2}\in\pv$. Finally,
it can be of the form $\psi=\forall u\psi_{1}$ for some $u\in V$ and
$\psi_{1}\in\pv$. Looking at
definition~(\ref{logic:def:FOPL:mintransform:transform}) we see that
the minimal transform ${\cal M}(\psi)$ has the same basic structure
as $\psi$. Thus, from the equality ${\cal M}(\psi)=(x\in y)$ it
follows that $\psi$ can only be of the form $\psi=(u\in v)$. So we
obtain ${\cal M}(\psi)=(u\in v)=(x\in y)$ and consequently $u=x$ and
$v=y$ which implies that $\psi=(x\in y)=\phi$ as requested. We now
assume that $\phi=\bot$. Let $\psi\in\pv$ such that ${\cal
M}(\phi)=\bot={\cal M}(\psi)$. We need to show that $\phi\sim\psi$.
Once again it is sufficient to prove that $\phi=\psi$. From ${\cal
M}(\psi)=\bot$ we see that $\psi$ can only be of the form
$\psi=\bot$. So $\phi=\psi$ as requested. We now assume that
$\phi=\phi_{1}\to\phi_{2}$ where $\phi_{1},\phi_{2}\in\pv$ satisfy
our property. We need to show the same if true of $\phi$. So let
$\psi\in\pv$ such that ${\cal M}(\phi)={\cal M}(\phi_{1})\to{\cal
M}(\phi_{2})={\cal M}(\psi)$. We need to show that $\phi\sim\psi$.
From ${\cal M}(\psi) = {\cal M}(\phi_{1})\to{\cal M}(\phi_{2})$ we
see that $\psi$ can only be of the form $\psi=\psi_{1}\to\psi_{2}$.
It follows that ${\cal M}(\psi)={\cal M}(\psi_{1})\to{\cal
M}(\psi_{2})$ and consequently we obtain ${\cal M}(\phi_{1})\to{\cal
M}(\phi_{2})={\cal M}(\psi_{1})\to{\cal M}(\psi_{2})$. Thus, using
theorem~(\ref{logic:the:unique:representation}) of
page~\pageref{logic:the:unique:representation} we have ${\cal
M}(\phi_{1})={\cal M}(\psi_{1})$ and ${\cal M}(\phi_{2})={\cal
M}(\psi_{2})$. Having assumed $\phi_{1}$ and $\phi_{2}$ satisfy our
induction property, it follows that $\phi_{1}\sim\psi_{1}$ and
$\phi_{2}\sim\psi_{2}$ and consequently
$\phi_{1}\to\phi_{2}\sim\psi_{1}\to\psi_{2}$. So we have proved that
$\phi\sim\psi$ as requested. We now assume that $\phi=\forall
x\phi_{1}$ where $x\in V$ and $\phi_{1}\in\pv$ satisfy our property.
We need to show the same is true of $\phi$. So let $\psi\in\pv$ such
that ${\cal M}(\phi)={\cal M}(\psi)$. We need to show that
$\phi\sim\psi$. We have:
    \begin{equation}\label{logic:eqn:FOPL:mintransfsubcong:the:1}
    {\cal M}(\phi)=\forall n{\cal M}(\phi_{1})[n/x]
    \end{equation}
where $n=\min\{k\in\N:\mbox{$[k/x]$ valid for ${\cal
M}(\phi_{1})$}\}$. Hence from the equality ${\cal M}(\phi)={\cal
M}(\psi)$ we see that $\psi$ can only be of the form $\psi=\forall
y\psi_{1}$ for some $y\in V$ and $\psi_{1}\in\pv$. We shall
distinguish two cases: first we assume that $y=x$. Then we need to
show that $\forall x\phi_{1}\sim\forall x\psi_{1}$ and it is
therefore sufficient to prove that $\phi_{1}\sim\psi_{1}$. Having
assumed $\phi_{1}$ satisfy our property, we simply need to show that
${\cal M}(\phi_{1})={\cal M}(\psi_{1})$. However we have the
equality:
    \begin{equation}\label{logic:eqn:FOPL:mintransfsubcong:the:2}
    {\cal M}(\psi)=\forall m{\cal M}(\psi_{1})[m/x]
    \end{equation}
where $m=\min\{k\in\N:\mbox{$[k/x]$ valid for ${\cal
M}(\psi_{1})$}\}$.
Comparing~(\ref{logic:eqn:FOPL:mintransfsubcong:the:1})
and~(\ref{logic:eqn:FOPL:mintransfsubcong:the:2}), from ${\cal
M}(\phi)={\cal M}(\psi)$ and
theorem~(\ref{logic:the:unique:representation}) of
page~\pageref{logic:the:unique:representation} we obtain $n=m$ and
thus:
    \begin{equation}\label{logic:eqn:FOPL:mintransfsubcong:the:3}
    {\cal M}(\phi_{1})[n/x]={\cal M}(\psi_{1})[n/x]
    \end{equation}
Consider the substitution $\sigma:\bar{V}\to\bar{V}$ defined by
$\sigma=[n/x]$. We shall conclude that ${\cal M}(\phi_{1})={\cal
M}(\psi_{1})$ by inverting
equation~(\ref{logic:eqn:FOPL:mintransfsubcong:the:3}) using the
local inversion theorem~(\ref{logic:the:FOPL:localinv:main}) of
page~\pageref{logic:the:FOPL:localinv:main} on the substitution
$\sigma$. So consider the sets $V_{0}=V$ and $V_{1}=\N$. It is clear
that both $\sigma_{|V_{0}}$ and $\sigma_{|V_{1}}$ are injective
maps. Define:
    \[
    \Gamma=\{\chi\in\pvb:(\free(\chi)\subseteq
    V_{0})\land(\bound(\chi)\subseteq V_{1})\land(\mbox{$\sigma$
    valid for $\chi$})\}
    \]
Then using theorem~(\ref{logic:the:FOPL:localinv:main}) there exits
$\tau:\pvb\to\pvb$ such that $\tau\circ\sigma(\chi)=\chi$ for all
$\chi\in\Gamma$. Hence from
equation~(\ref{logic:eqn:FOPL:mintransfsubcong:the:3}) it is
sufficient to prove that ${\cal M}(\phi_{1})\in\Gamma$ and ${\cal
M}(\psi_{1})\in\Gamma$. First we show that ${\cal
M}(\phi_{1})\in\Gamma$. We already know that $\sigma=[n/x]$ is a
valid substitution for ${\cal M}(\phi_{1})$. The fact that
$\free({\cal M}(\phi_{1}))\subseteq V_{0}$ follows from
proposition~(\ref{logic:prop:FOPL:mintransform:variables}). The fact
that $\bound({\cal M}(\phi_{1}))\subseteq V_{1}$ follows from
proposition~(\ref{logic:prop:FOPL:mintransform:variables:bound}). So
we have proved that ${\cal M}(\phi_{1})\in\Gamma$ as requested. The
proof of ${\cal M}(\psi_{1})\in\Gamma$ is identical, which completes
our proof of $\phi\sim\psi$ in the case when $\psi=\forall
y\psi_{1}$ and $y=x$. We now assume that $y\neq x$. Consider the
formula $\psi^{*}=\forall x\psi_{1}[x\!:\!y]$ where $[x\!:\!y]$ is
the permutation mapping as per
definition~(\ref{logic:def:single:var:permutation}) of
page~\pageref{logic:def:single:var:permutation}. Suppose we have
proved the equivalence $\psi\sim\psi^{*}$. Then in order to prove
$\phi\sim\psi$ it is sufficient by transitivity to show that
$\phi\sim\psi^{*}$. However, having already proved the implication
$\Rightarrow$ of this theorem, we know that $\psi\sim\psi^{*}$
implies ${\cal M}(\psi)={\cal M}(\psi^{*})$. Hence, if we have
$\psi\sim\psi^{*}$, it is sufficient to prove $\phi\sim\psi^{*}$
knowing that ${\cal M}(\phi)={\cal M}(\psi^{*})$ and
$\psi^{*}=\forall x\psi_{1}^{*}$ where
$\psi_{1}^{*}=\psi_{1}[x\!:\!y]$. So we are back to the case when
$y=x$, a case we have already dealt with. It follows that we can
complete our induction argument simply by showing
$\psi\sim\psi^{*}$. Since $\psi=\forall y\psi_{1}$ and
$\psi^{*}=\forall x\psi_{1}[x\!:\!y]$ with $x\neq y$, from
definition~(\ref{logic:def:sub:congruence}) of
page~\pageref{logic:def:sub:congruence} we simply need to check that
$x\not\in\free(\psi_{1})$. So suppose to the contrary that
$x\in\free(\psi_{1})$. Since $x\neq y$ we obtain $x\in\free(\psi)$.
From proposition~(\ref{logic:prop:FOPL:mintransform:variables}) it
follows that $x\in\free({\cal M}(\psi))$. Having assumed that ${\cal
M}(\phi)={\cal M}(\psi)$ we obtain $x\in\free({\cal M}(\phi))$ and
finally using
proposition~(\ref{logic:prop:FOPL:mintransform:variables}) once
more, we obtain $x\in\free(\phi)$. This is our desired contradiction
since $\phi=\forall x\phi_{1}$.
\end{proof}

We conclude this section with an immediate consequence of
theorem~(\ref{logic:the:FOPL:commute:mintransform:validsub}) of
page~\pageref{logic:the:FOPL:commute:mintransform:validsub} and
theorem~(\ref{logic:the:FOPL:mintransfsubcong:kernel}). From
proposition~(\ref{logic:prop:sub:congruence:injective:substitution})
we know that $\sigma(\phi)\sim\sigma(\psi)$ follows from
$\phi\sim\psi$ whenever $\sigma:V\to W$ is injective. We can now
loosen the hypothesis by requiring that $\sigma$ be simply valid for
$\phi$ and $\psi$. Note that this result is not true if $\sim$
denotes the strong substitution congruence rather than the
substitution congruence. Take $V=\{x,y,z\}$ and $W=\{x,y\}$ with
$x,y,z$ distinct and $\phi=\forall x\forall y(x\in y)$ while
$\psi=\forall y\forall x(y\in x)$ with $\sigma(x)=x$ and
$\sigma(y)=y$. Then $\sigma$ is valid for $\phi$ and $\psi$ but
$\sigma(\phi)\sim\sigma(\psi)$ fails to be true.
\begin{theorem}\label{logic:the:FOPL:mintransfsubcong:valid}
Let $V$ and $W$ be sets and $\sigma:V\to W$ be a map. Let~$\sim$ be
the substitution congruence on \pv\ and ${\bf P}(W)$. Then if
$\sigma$ is valid for $\phi$ and $\psi$:
    \[
    \phi\sim\psi\ \Rightarrow\ \sigma(\phi)\sim\sigma(\psi)
    \]
for all $\phi,\psi\in\pv$, where $\sigma:\pv\to{\bf P}(W)$ is also
the substitution mapping.
\end{theorem}
\begin{proof}
We assume that $\phi\sim\psi$ and $\sigma:V\to W$ is valid for
$\phi$ and $\psi$. We need to show that
$\sigma(\phi)\sim\sigma(\psi)$. Using
theorem~(\ref{logic:the:FOPL:mintransfsubcong:kernel}) it is
sufficient to prove that ${\cal M}\circ\sigma(\phi)={\cal
M}\circ\sigma(\psi)$. Since $\sigma$ is valid for $\phi$ and $\psi$,
using theorem~(\ref{logic:the:FOPL:commute:mintransform:validsub})
of page~\pageref{logic:the:FOPL:commute:mintransform:validsub} it is
therefore sufficient to prove that $\bar{\sigma}\circ{\cal
M}(\phi)=\bar{\sigma}\circ{\cal M}(\psi)$, which follows immediately
from ${\cal M}(\phi)={\cal M}(\psi)$, itself a consequence of
$\phi\sim\psi$.
\end{proof}
