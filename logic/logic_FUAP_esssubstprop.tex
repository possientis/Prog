Just as we did for formulas, we shall now investigate some of the
basic properties of essential substitutions for proofs. The
following definition is very similar to
definition~(\ref{logic:def:FUAP:esssubst:main}). In effect, we are
simply defining an {\em essential substitution} as any map
$\sigma^{*}:\pvs\to{\bf\Pi}(W)$ for which there exists a map
$\sigma:V\to W$ such that $\sigma^{*}$ is associated to $\sigma$ as
per definition~(\ref{logic:def:FUAP:esssubst:main}). As soon as we
show that such $\sigma$ is unique, we shall drop the '$*$' and refer
to $\sigma$ and $\sigma^{*}$ with the same symbol. Indeed, if we
start with a map $\sigma:\pvs\to{\bf\Pi}(W)$, we can safely and
unambiguously refer to the map $\sigma:V\to W$. However, one should
be a little cautious about notations: if we start from $\sigma:V\to
W$, then there are potentially more than one map
$\sigma:\pvs\to{\bf\Pi}(W)$ which qualifies as an essential
substitution associated to $\sigma$. Furthermore, the notation
$\sigma:\pvs\to{\bf\Pi}(W)$ could also refer to the usual proof
substitution as per
definition~(\ref{logic:def:FUAP:substitution:substitution}). There
will be situations when we shall want to speak about both the
non-essential $\sigma:\pvs\to{\bf\Pi}(W)$ of
definition~(\ref{logic:def:FUAP:substitution:substitution}), and
some essential representative $\sigma^{*}:\pvs\to{\bf\Pi}(W)$. When
this happens, we shall use a different symbol $\sigma^{*}$ to single
out one of them.

\index{essential@Essential substitution for proof}
\begin{defin}\label{logic:def:FUAP:esssubstprop:essential}
Let $V,W$ be sets. We say that a map $\sigma^{*}:\pvs\to{\bf\Pi}(W)$
is an {\em essential proof substitution} \ifand\ there exists
$\sigma:V\to W$ such that:
    \[
    {\cal M}\circ\sigma^{*}=\bar{\sigma}\circ{\cal M}
    \]
where $\bar{\sigma}:\bar{V}\to\bar{W}$ is the minimal extension and
${\cal M}$ is the minimal transform.
\end{defin}

\begin{prop}\label{logic:prop:FUAP:esssubstprop:uniqueness}
Let $V$, $W$ be sets and $\sigma^{*}:\pvs\to{\bf\Pi}(W)$ be an
essential proof substitution. Then the map $\sigma:V\to W$
associated with $\sigma^{*}$ is unique.
\end{prop}
\begin{proof}
Suppose $\sigma_{0},\sigma_{1}:V\to W$ are two maps which are
associated with an essential proof substitution
$\sigma^{*}:\pvs\to{\bf\Pi}(W)$. Then using
proposition~(\ref{logic:prop:FUAP:esssubst:proof:to:formula}), the
restriction $\sigma^{*}_{|{\bf P}(V)}:\pv\to{\bf P}(W)$ is an
essential substitution associated with $\sigma_{0}$ and
$\sigma_{1}$. From the uniqueness property of
proposition~(\ref{logic:prop:FOPL:esssubstprop:uniqueness}), we have
$\sigma_{0}=\sigma_{1}$.
\end{proof}

An essential proof substitution $\sigma:\pvs\to{\bf\Pi}(W)$ can be
redefined arbitrarily modulo $\alpha$-equivalence, without affecting
it status of {\em essential substitution associated to $\sigma$}.
The following is the counterpart of
proposition~(\ref{logic:prop:FOPL:esssubstprop:redefine}).

\begin{prop}\label{logic:prop:FUAP:esssubstprop:redefine}
Let $V,W$ be sets and $\sigma:\pvs\to{\bf\Pi}(W)$ be an essential
proof substitution. Let $\tau:\pvs\to{\bf\Pi}(W)$ be a map such that
$\sigma(\pi)\sim\tau(\pi)$ for all $\pi\in\pvs$ where $\sim$ is the
substitution congruence on ${\bf\Pi}(W)$. Then $\tau$ is itself an
essential substitution with associated map $\tau:V\to W$ identical
to $\sigma$.
\end{prop}
\begin{proof}
Since $\sigma$ is essential we have ${\cal
M}\circ\sigma(\pi)=\bar{\sigma}\circ{\cal M}(\pi)$ for all
$\pi\in\pvs$. Having assumed that $\sigma(\pi)\sim\tau(\pi)$ for all
$\pi\in\pvs$, from
theorem~(\ref{logic:the:FUAP:mintransfsubcong:kernel}) of
page~\pageref{logic:the:FUAP:mintransfsubcong:kernel} we have ${\cal
M}\circ\sigma(\pi)={\cal M}\circ\tau(\pi)$. It follows that ${\cal
M}\circ\tau(\pi)=\bar{\sigma}\circ{\cal M}(\pi)$ and we conclude
that $\tau$ is itself essential with associated map $\sigma:V\to W$.
\end{proof}

The following proposition has no counterpart for formulas. We know
from proposition~(\ref{logic:prop:FUAP:esssubst:proof:to:formula})
that restricting an essential substitution
$\sigma:\pvs\to{\bf\Pi}(W)$ to \pv\ leads to an essential
substitution $\sigma_{|{\bf P}(V)}:\pv\to{\bf P}(W)$. In fact, every
essential substitution $\sigma:\pv\to{\bf P}(W)$ is obtained that
way. In other words, every essential substitution for formulas is
the restriction of an essential substitution for proofs, associated
with the same underlying map $\sigma:V\to W$. This proposition will
prove very useful when attempting to prove the substitution
theorem~(\ref{logic:the:FUAP:substitutiontheorem:main}) of
page~\pageref{logic:the:FUAP:substitutiontheorem:main}. Starting
from an essential substitution $\sigma:\pv\to{\bf P}(W)$ and a true
sequent $\Gamma\vdash\phi$ with underlying proof $\pi\in\pvs$, we
shall extend $\sigma$ to an essential substitution
$\sigma:\pvs\to{\bf\Pi}(W)$ which will give us the proof
$\sigma(\pi)$, allowing us to establish the sequent
$\sigma(\Gamma)\vdash\sigma(\phi)$. This is true magic.

\begin{prop}\label{logic:prop:FUAP:esssubstprop:extension}
Let $V,W$ be sets and $\sigma:\pv\to{\bf P}(W)$ be an essential
substitution. Then the map $\sigma$ can be extended to an essential
proof substitution $\sigma:\pvs\to{\bf\Pi}(W)$ associated with the
same underlying $\sigma:V\to W$.
\end{prop}
\begin{proof}
We assume that $\sigma:\pv\to{\bf P}(W)$ is an essential
substitution associated with $\sigma:V\to W$ as per
definition~(\ref{logic:def:FOPL:esssubstprop:essential}). We need to
show the existence of an essential proof substitution
$\sigma^{*}:\pvs\to{\bf\Pi}(W)$ associated with $\sigma:V\to W$ such
that $\sigma^{*}_{|{\bf P}(V)}=\sigma$. However, from the existence
of $\sigma:\pv\to{\bf P}(W)$ using
theorem~(\ref{logic:the:FOPL:esssubst:existence}) of
page~\pageref{logic:the:FOPL:esssubst:existence} it follows that
$|W|$ is an infinite cardinal, or that it is finite with
$|V|\leq|W|$. We can therefore apply
theorem~(\ref{logic:the:FUAP:esssubst:existence}) of
page~\pageref{logic:the:FUAP:esssubst:existence} from which we
deduce the existence of an essential substitution
$\sigma^{*}:\pvs\to{\bf\Pi}(W)$ associated with $\sigma:V\to W$.
Using proposition~(\ref{logic:prop:FUAP:esssubst:proof:to:formula}),
the restriction $\sigma^{*}:\pv\to{\bf P}(W)$ is an essential
substitution associated with $\sigma:V\to W$. Hence we must have
$\sigma(\phi)\sim\sigma^{*}(\phi)$ for all $\phi\in\pv$, where
$\sim$ is the substitution congruence on ${\bf P}(W)$. In fact, from
proposition~(\ref{logic:prop:FUAP:charsubsong:equivalence:formula})
this equivalence remains true if $\sim$ is regarded as the
substitution congruence on ${\bf\Pi}(W)$. From
proposition~(\ref{logic:prop:FUAP:esssubstprop:redefine}) we can
therefore redefine $\sigma^{*}$ on \pv\ by setting
$\sigma^{*}(\phi)=\sigma(\phi)$ so as to obtain an essential proof
substitution $\sigma^{*}:\pvs\to{\bf\Pi}(W)$ associated with
$\sigma:V\to W$ which coincides with $\sigma$ on \pv. This completes
our proof.
\end{proof}

Our initial idea of variable substitution for proofs is to blindly
substitute variables and create $\sigma^{*}:\pvs\to{\bf\Pi}(W)$ as
per definition~(\ref{logic:def:FUAP:substitution:substitution}).
This does not work very well, unless the substitution $\sigma:V\to
W$ is valid for the proof $\pi$. So we introduced essential
substitutions $\sigma:\pvs\to{\bf\Pi}(W)$ to eliminate the problem
of variable capture. We should check that the solution provided is a
good solution, namely that $\sigma(\pi)$ and $\sigma^{*}(\pi)$ are
in fact the same proofs modulo $\alpha$-equivalence, whenever
$\sigma$ is valid for $\pi$. See also
proposition~(\ref{logic:prop:FOPL:esssubstprop:validity}).

\begin{prop}\label{logic:prop:FUAP:esssubstprop:validity}
Let $V,W$ be sets and $\sigma:\pvs\to{\bf\Pi}(W)$ be an essential
substitution. Then if $\sigma$ is valid for $\pi\in\pvs$, we have
the substitution equivalence:
    \[
    \sigma(\pi)\sim\sigma^{*}(\pi)
    \]
where $\sigma^{*}:\pvs\to{\bf\Pi}(W)$ is the associated substitution
as per {\em
definition~(\ref{logic:def:FUAP:substitution:substitution})}.
\end{prop}
\begin{proof}
We assume that $\sigma$ is valid for $\pi$. We need to show that
$\sigma(\pi)\sim\sigma^{*}(\pi)$, that is ${\cal
M}\circ\sigma(\pi)={\cal M}\circ\sigma^{*}(\pi)$. However, having
assumed $\sigma$ is valid for $\pi$, from
theorem~(\ref{logic:the:FUAP:mintransvalidsub:commute}) of
page~\pageref{logic:the:FUAP:mintransvalidsub:commute} we have
${\cal M}\circ\sigma^{*}(\pi)=\bar{\sigma}\circ{\cal M}(\pi)$. Since
$\sigma$ is an essential proof substitution we also have ${\cal
M}\circ\sigma(\pi)=\bar{\sigma}\circ{\cal M}(\pi)$. So the result
follows.
\end{proof}

The following proposition is the counterpart of
proposition~(\ref{logic:prop:FOPL:esssubstprop:injective})\,:

\begin{prop}\label{logic:prop:FUAP:esssubstprop:injective}
Let $V,W$ be sets and $\sigma:V\to W$ be an injective map. Then the
associated map $\sigma:\pvs\to{\bf\Pi}(W)$ is essential with
associated map $\sigma$ itself.
\end{prop}
\begin{proof}
Let $\sigma:V\to W$ be an injective map. Let $\sigma:\pvs\to{\bf
\Pi}(W)$ be the associated substitution mapping as per
definition~(\ref{logic:def:FUAP:substitution:substitution}). The
fact that both mappings are called '$\sigma$' is standard practice
at this stage for us. We need to prove that
$\sigma:\pvs\to{\bf\Pi}(W)$ is an essential substitution mapping,
with associated map $\sigma:V\to W$. In other words, we need to
prove that ${\cal M}\circ\sigma(\pi)=\bar{\sigma}\circ{\cal M}(\pi)$
for all $\pi\in\pvs$. Using
theorem~(\ref{logic:the:FUAP:mintransvalidsub:commute}) of
page~\pageref{logic:the:FUAP:mintransvalidsub:commute} it is
sufficient to show that $\sigma$ is valid for $\pi$ which follows
from proposition~(\ref{logic:prop:FUAP:validsubproof:injective}) and
the injectivity of $\sigma:V\to W$.
\end{proof}

If we consider the inclusion map $i:V\to\bar{V}$, there are two ways
to {\em essentially} substitute variables in accordance to $i$. One
way to is to consider the associated $i:\pvs\to\pvsb$. The other and
possibly unexpected way is to consider the minimal transform ${\cal
M}:\pvs\to\pvsb$. See also
proposition~(\ref{logic:prop:FOPL:esssubstprop:mintransform}).

\begin{prop}\label{logic:prop:FUAP:esssubstprop:mintransform}
Let $V$ be a set. The minimal transform ${\cal M}:\pvs\to\pvsb$ is
an essential substitution with associated map the inclusion
$i:V\to\bar{V}$.
\end{prop}
\begin{proof}
Let $\bar{\cal M}:\pvsb\to{\bf\Pi}(\bar{\bar{V}})$ denote the
minimal transform on \pvsb. We need to show that $\bar{\cal
M}\circ{\cal M}(\pi)=\bar{i}\circ{\cal M}(\pi)$ for all
$\pi\in\pvs$. However, since $i$ is injective, from
proposition~(\ref{logic:prop:FUAP:validsubproof:injective}) it is
valid for $\pi$ and it follows from
theorem~(\ref{logic:the:FUAP:mintransvalidsub:commute}) of
page~\pageref{logic:the:FUAP:mintransvalidsub:commute} that
$\bar{\cal M}\circ i(\pi)=\bar{i}\circ{\cal M}(\pi)$. Hence we need
to show that $\bar{\cal M}\circ{\cal M}(\pi)=\bar{\cal M}\circ
i(\pi)$. Using
theorem~(\ref{logic:the:FUAP:mintransfsubcong:kernel}) of
page~\pageref{logic:the:FUAP:mintransfsubcong:kernel} we need to
show ${\cal M}(\pi)\sim i(\pi)$ where $\sim$ is the substitution
congruence on \pvsb, which we know is true from
proposition~(\ref{logic:prop:FUAP:mintransfsubcong:equivalence}) and
which completes our proof.
\end{proof}

The following proposition is the counterpart of
proposition~(\ref{logic:prop:FOPL:esssubstprop:subcong})\,:

\begin{prop}\label{logic:prop:FUAP:esssubstprop:subcong}
Let $V$ be a set and $\sim$ be the substitution congruence on \pvs.
Then for all $\pi,\rho\in\pvs$ we have $\pi\sim\rho$ \ifand\
$\rho=\sigma(\pi)$ for some essential substitution
$\sigma:\pvs\to\pvs$ such that $\sigma(u)=u$ for all
$u\in\free(\pi)$.
\end{prop}
\begin{proof}
First we show the 'if' part: so suppose $\rho=\sigma(\pi)$ for some
essential substitution $\sigma:\pvs\to\pvs$ such that $\sigma(u)=u$
for all $u\in\free(\pi)$. We need to show that $\pi\sim\rho$. From
theorem~(\ref{logic:the:FUAP:mintransfsubcong:kernel}) of
page~\pageref{logic:the:FUAP:mintransfsubcong:kernel} it is
sufficient to prove that ${\cal M}(\pi)={\cal M}(\rho)$. Having
assumed that $\rho=\sigma(\pi)$ we need to show that ${\cal
M}(\pi)={\cal M}\circ\sigma(\pi)$. However, since $\sigma$ is
essential we have ${\cal M}\circ\sigma(\pi)=\bar{\sigma}\circ{\cal
M}(\pi)$. Hence we need to show that ${\cal
M}(\pi)=\bar{\sigma}\circ{\cal M}(\pi)$. Using
proposition~(\ref{logic:prop:FUAP:variable:support}) it is
sufficient to prove that $\bar{\sigma}(u)=u$ for all $u\in\var({\cal
M}(\pi))$. So let $u\in\var({\cal M}(\pi))$. Since
$\bar{V}=V\uplus\N$ we shall distinguish two cases: first we assume
that $u\in\N$. Then $\bar{\sigma}(u)=u$ is clear from
definition~(\ref{logic:def:FOPL:commute:minextensioon:map}). Next we
assume that $u\in V$. Then from
proposition~(\ref{logic:prop:FUAP:mintransformproof:freevar}) we
obtain $u\in\free(\pi)$ and it follows that
$\bar{\sigma}(u)=\sigma(u)=u$. We now prove the 'only if' part: so
suppose $\pi\sim\rho$. We need to show that $\rho=\sigma(\pi)$ for
some essential substitution $\sigma:\pvs\to\pvs$ such that
$\sigma(u)=u$ for all $u\in\free(\pi)$. Let $i:\pvs\to\pvs$ be the
identity mapping. From
proposition~(\ref{logic:prop:FUAP:esssubstprop:injective}), $i$ is
an essential substitution associated with the identity $i:V\to V$.
Let $\sigma:\pvs\to\pvs$ be defined by $\sigma(\kappa)=i(\kappa)$
whenever $\kappa\neq\pi$ and $\sigma(\pi)=\rho$. Having assumed that
$\pi\sim\rho$ we have $\sigma(\kappa)\sim i(\kappa)$ for all
$\kappa\in\pvs$. It follows from
proposition~(\ref{logic:prop:FUAP:esssubstprop:redefine}) that
$\sigma$ is an essential substitution whose associated map
$\sigma:V\to V$ is the identity. In particular, we have
$\sigma(u)=u$ for all $u\in\free(\pi)$.
\end{proof}

The following proposition is the counterpart of
proposition~(\ref{logic:prop:FOPL:esssubstprop:composition})\,:

\begin{prop}\label{logic:prop:FUAP:esssubstprop:composition}
Let $U$, $V$ and $W$ be sets while the maps
$\tau:{\bf\Pi}(U)\to\pvs$ and $\sigma:\pvs\to{\bf\Pi}(W)$ are
essential substitutions. Then $\sigma\circ\tau:{\bf\Pi}(U)\to{\bf
\Pi}(W)$ is itself an essential proof substitution with associated
map $\sigma\circ\tau:U\to W$.
\end{prop}
\begin{proof}
We need to show that ${\cal
M}\circ(\sigma\circ\tau)=\overline{(\sigma\circ\tau)}\circ{\cal M}$.
However from
definition~(\ref{logic:def:FOPL:commute:minextensioon:map}), it is
clear that the minimal extension
$\overline{(\sigma\circ\tau)}:\bar{U}\to\bar{W}$ is equal to the
composition of the minimal extensions $\bar{\sigma}\circ\bar{\tau}$.
Hence we have:
    \begin{eqnarray*}
    {\cal M}\circ(\sigma\circ\tau)
    &=&\bar{\sigma}\circ{\cal
    M}\circ\tau\\
    &=&\bar{\sigma}\circ\bar{\tau}\circ{\cal
    M}\\
    &=&\overline{(\sigma\circ\tau)}\circ{\cal
    M}\\
    \end{eqnarray*}
\end{proof}


The following proposition is the counterpart of
proposition~(\ref{logic:prop:FOPL:esssubstprop:free:commute})\,:

\begin{prop}\label{logic:prop:FUAP:esssubstprop:free:commute}
Let $V,W$ be sets and $\sigma:\pvs\to{\bf\Pi}(W)$ be an essential
proof substitution. Then for all $\pi\in\pvs$ we have the equality:
    \[
    \free(\sigma(\pi))=\sigma(\free(\pi))
    \]
\end{prop}
\begin{proof}
Using proposition~(\ref{logic:prop:FUAP:mintransformproof:freevar})
we obtain the following:
    \begin{eqnarray*}
    \free(\sigma(\pi))&=&\free(\,{\cal M}\circ\sigma(\pi)\,)\\
    &=&\free(\,\bar{\sigma}\circ{\cal M}(\pi)\,)\\
    \mbox{$\bar{\sigma}$ valid for ${\cal M}(\pi)$}\ \rightarrow
    &=&\bar{\sigma}(\,\free({\cal M}(\pi))\,)\\
    \mbox{prop.~(\ref{logic:prop:FUAP:mintransformproof:freevar})}\ \rightarrow
    &=&\bar{\sigma}(\free(\pi))\\
    \free(\pi)\subseteq V\ \rightarrow&=&\sigma(\free(\pi))
    \end{eqnarray*}
\end{proof}

The following proposition is the counterpart of
proposition~(\ref{logic:prop:FOPL:esssubstprop:support})\,:

\begin{prop}\label{logic:prop:FUAP:esssubstprop:support}
Let $V,W$ be sets and $\sigma,\tau:\pvs\to{\bf\Pi}(W)$ be two
essential proof substitutions. Then for all $\pi\in\pvs$ we have the
equivalence:
    \[
    \sigma_{|\free(\pi)}=\tau_{|\free(\pi)}\ \ \Leftrightarrow\
    \ \sigma(\pi)\sim\tau(\pi)
    \]
where $\sim$ denotes the substitution congruence on the algebra
${\bf\Pi}(W)$.
\end{prop}
\begin{proof}
From theorem~(\ref{logic:the:FUAP:mintransfsubcong:kernel}) of
page~\pageref{logic:the:FUAP:mintransfsubcong:kernel},
$\sigma(\pi)\sim\tau(\pi)$ is equivalent to the equality ${\cal
M}\circ\sigma(\pi)={\cal M}\circ\tau(\pi)$. Having assumed $\sigma$
and $\tau$ are essential, this is in turn equivalent to
$\bar{\sigma}\circ{\cal M}(\pi)=\bar{\tau}\circ{\cal M}(\pi)$. Using
proposition~(\ref{logic:prop:FUAP:variable:support}), this last
equality is equivalent to $\bar{\sigma}(u)=\bar{\tau}(u)$ for all
$u\in\var({\cal M}(\pi))$. Hence we need to show that this last
statement is equivalent to $\sigma(u)=\tau(u)$ for all
$u\in\free(\pi)$. First we show $\Rightarrow$\,: so suppose
$\bar{\sigma}(u)=\bar{\tau}(u)$ for all $u\in\var({\cal M}(\pi))$
and let $u\in\free(\pi)$. We need to show that $\sigma(u)=\tau(u)$.
From proposition~(\ref{logic:prop:FUAP:mintransformproof:freevar})
we have $\var({\cal M}(\pi))\cap V=\free(\pi)$. It follows that
$u\in\var({\cal M}(\pi))\cap V$ and consequently we have
$\sigma(u)=\bar{\sigma}(u)=\bar{\tau}(u)=\tau(u)$ as requested. So
we now prove $\Leftarrow$\,: So we assume that $\sigma(u)=\tau(u)$
for all $u\in\free(\pi)$ and consider $u\in\var({\cal M}(\pi))$. We
need to show that $\bar{\sigma}(u)=\bar{\tau}(u)$. Since
$\bar{V}=V\uplus\N$ we shall distinguish two cases: first we assume
that $u\in\N$. Then $\bar{\sigma}(u)=u=\bar{\tau}(u)$. Next we
assume that $u\in V$. Then $u\in \var({\cal M}(\pi))\cap
V=\free(\pi)$ and $\bar{\sigma}(u)=\sigma(u)=\tau(u)=\bar{\tau}(u)$.
\end{proof}

The fact that $\alpha$-equivalence is preserved by certain maps
$\sigma:\pvs\to{\bf\Pi}(W)$ has already been seen on a few
occasions. We have the case of $\sigma:V\to W$ injective of
proposition~(\ref{logic:prop:FUAP:charsubcong:injective:substitution}).
We have the case of $\sigma:V\to W$ valid for both $\pi$ and $\rho$
of theorem~(\ref{logic:the:FUAP:mintransfsubcong:valid}) of
page~\pageref{logic:the:FUAP:mintransfsubcong:valid}. We now have
the case of $\sigma:\pvs\to{\bf\Pi}(W)$ essential. The following is
the counterpart of
proposition~(\ref{logic:prop:FOPL:esssubstprop:equivalence})\,:

\begin{prop}\label{logic:prop:FUAP:esssubstprop:equivalence}
Let $V,W$ be sets and $\sigma:\pvs\to {\bf\Pi}(W)$ be an essential
proof substitution. Then for all proofs $\pi,\rho\in\pvs$ we have
the implication:
    \[
    \pi\sim\rho\ \Rightarrow\ \sigma(\pi)\sim\sigma(\rho)
    \]
where~$\sim$ denotes the substitution congruence on \pvs\ and ${\bf
\Pi}(W)$.
\end{prop}
\begin{proof}
So we assume that $\pi\sim\rho$. We need to show that
$\sigma(\pi)\sim\sigma(\rho)$. Using
theorem~(\ref{logic:the:FUAP:mintransfsubcong:kernel}) of
page~\pageref{logic:the:FUAP:mintransfsubcong:kernel} it is
sufficient to show that ${\cal M}\circ\sigma(\pi)={\cal
M}\circ\sigma(\rho)$. Having assumed that $\sigma$ is essential, it
is sufficient to show that $\bar{\sigma}\circ{\cal
M}(\pi)=\bar{\sigma}\circ{\cal M}(\rho)$ which follows immediately
from ${\cal M}(\pi)={\cal M}(\rho)$, itself a consequence of
$\pi\sim\rho$.
\end{proof}

The following proposition is the counterpart of
proposition~(\ref{logic:prop:FOPL:esssubstprop:rank:injective})\,:

\begin{prop}\label{logic:prop:FUAP:esssubstprop:rank:injective}
Let $V,W$ be sets and $\sigma:\pvs\to{\bf\Pi}(W)$ be an essential
proof substitution. Let $\pi\in\pvs$ such that
$\sigma_{|\free(\pi)}$ is an injective map. Then:
    \[
    \rnk(\sigma(\pi))=\rnk(\pi)
    \]
where $\rnk(\,\,)$ refers to the substitution rank as per {\em
definition~(\ref{logic:def:FUAP:substrank:main})}.
\end{prop}
\begin{proof}
Using proposition~(\ref{logic:prop:FUAP:substrank:minrank}) we
obtain the following:
    \begin{eqnarray*}
    \rnk(\sigma(\pi))&=&\rnk({\cal M}\circ\sigma(\pi))\\
    \mbox{$\sigma$ essential}\ \rightarrow
    &=&\rnk(\bar{\sigma}\circ{\cal M}(\pi))\\
    \mbox{A: to be proved}\ \rightarrow
    &=&\rnk({\cal M}(\pi))\\
    \mbox{prop.~(\ref{logic:prop:FUAP:substrank:minrank})}\ \rightarrow
    &=&\rnk(\pi)
    \end{eqnarray*}
So it remains to show that $\rnk(\bar{\sigma}\circ{\cal
M}(\pi))=\rnk({\cal M}(\pi))$. Using
proposition~(\ref{logic:prop:FUAP:substrank:invariant:rank}) it is
sufficient to show that $\bar{\sigma}$ is valid for ${\cal M}(\pi)$
and furthermore that it is injective on $\free({\cal M}(\pi))$. We
know that $\bar{\sigma}$ is valid for ${\cal M}(\pi)$ from
proposition~(\ref{logic:prop:FUAP:mintransformproof:minextension:valid}).
We also know that $\free({\cal M}(\pi))=\free(\pi)$ from
proposition~(\ref{logic:prop:FUAP:mintransformproof:freevar}). So it
remains to show that $\bar{\sigma}$ is injective on
$\free(\pi)\subseteq V$ which is clearly the case since
$\bar{\sigma}$ coincide with $\sigma$ on $V$ and $\sigma$ is by
assumption injective on $\free(\pi)$.
\end{proof}

The following proposition is the counterpart of
proposition~(\ref{logic:prop:FOPL:esssubstprop:rank:less})\,:

\begin{prop}\label{logic:prop:FUAP:esssubstprop:rank:less}
Let $V,W$ be sets and $\sigma:\pvs\to{\bf\Pi}(W)$ be an essential
proof substitution. Then for all proof $\pi\in\pv$ we have the
inequality:
    \[
    \rnk(\sigma(\pi))\leq\rnk(\pi)
    \]
where $\rnk(\,\,)$ refers to the substitution rank as per {\em
definition~(\ref{logic:def:FUAP:substrank:main})}.
\end{prop}
\begin{proof}
Using proposition~(\ref{logic:prop:FUAP:substrank:minrank}) we
obtain the following:
    \begin{eqnarray*}
    \rnk(\sigma(\pi))&=&\rnk({\cal M}\circ\sigma(\pi))\\
    \mbox{$\sigma$ essential}\ \rightarrow
    &=&\rnk(\bar{\sigma}\circ{\cal M}(\pi))\\
    \mbox{prop.~(\ref{logic:prop:FUAP:substrank:substitution})}\ \rightarrow
    &\leq&\rnk({\cal M}(\pi))\\
    \mbox{prop.~(\ref{logic:prop:FUAP:substrank:minrank})}\ \rightarrow
    &=&\rnk(\pi)
    \end{eqnarray*}
\end{proof}

Essential substitutions provide a huge benefit as they avoid
variable capture. However, they come with a price. It is no longer
possible to write something like
$\sigma(\pi_{1}\pon\pi_{1})=\sigma(\pi_{1})\pon\,\sigma(\pi_{2})$
which we could do if $\sigma$ was a naive substitution as per
definition~(\ref{logic:def:FUAP:substitution:substitution}).
Instead, we have to settle for the substitution equivalence
$\sigma(\pi_{1}\pon\pi_{1})\sim\sigma(\pi_{1})\pon\,\sigma(\pi_{2})$.
Note that we cannot even write $\sigma(\gen
x\pi_{1})\sim\gen\sigma(x)\sigma(\pi_{1})$ in general. Indeed, if
the $\alpha$-equivalence was always true, then $\sigma(x)$ could
never be a free variable of $\sigma(\pi)$ where $\pi=\gen x\pi_{1}$.
This is the whole point of essential substitutions: we want to allow
some $u\in\free(\pi)$ to be such that $\sigma(u)=\sigma(x)$, while
making sure $\sigma(\pi)$ is still meaningful. So we certainly want
$\sigma(x)$ to be a free variable of $\sigma(\pi)$ on occasions.
When this happens, the essential substitution $\sigma$ will
automatically redefine the bound variable so to speak, and avoid
capture. This is what most texts in mathematical logic implicitly
assume. Essential substitutions provide the tool to do this
formally. The following proposition is the counterpart of
proposition~(\ref{logic:prop:FOPL:esssubstprop:charac})\,:

\begin{prop}\label{logic:prop:FUAP:esssubstprop:charac}
Let $V,W$ be sets and $\sigma:\pvs\to{\bf\Pi}(W)$ be an essential
proof substitution. Let $\sim$ be the substitution congruence on
${\bf\Pi}(W)$. Then:
    \[
    \forall\pi\in\pvs\ ,\ \sigma(\pi)\sim\left\{
                    \begin{array}{lcl}
                    \sigma(\phi)&\mbox{\ if\ }&\pi=\phi\in\pv\\
                    \axi\sigma(\phi)&\mbox{\ if\ }&\pi=\axi\phi\\
                    \sigma(\pi_{1})\pon\,\sigma(\pi_{2})&\mbox{\ if\ }&
                    \pi=\pi_{1}\pon\pi_{2}\\
                    \gen\sigma(x)\sigma(\pi_{1})
                    &\mbox{\ if\ }&\pi=\gen x\pi_{1}\ ,\
                    \sigma(x)\not\in\free(\sigma(\pi))\\
                    \end{array}\right.
    \]
\end{prop}
\begin{proof}
First we assume that $\pi=\phi$ for some $\phi\in\pv$. Then
$\sigma(\pi)=\sigma(\phi)$ and we are not saying much by claiming
$\sigma(\pi)\sim\sigma(\phi)$ which is clear. Next we assume that
$\pi=\axi\phi$ for some $\phi\in\pv$. We need to show that
$\sigma(\pi)\sim\axi\sigma(\phi)$. Using
theorem~(\ref{logic:the:FUAP:mintransfsubcong:kernel}) of
page~\pageref{logic:the:FUAP:mintransfsubcong:kernel}, we simply
compute the minimal transforms:
    \begin{eqnarray*}
    {\cal M}\circ\sigma(\pi)&=&\bar{\sigma}\circ{\cal M}(\pi)\\
    &=&\bar{\sigma}\circ{\cal M}(\axi\phi)\\
    &=&\bar{\sigma}(\,\axi{\cal M}(\phi)\,)\\
    &=&\axi\,\bar{\sigma}\circ{\cal M}(\phi)\\
    &=&\axi\,{\cal M}\circ\sigma(\phi)\\
    &=&{\cal M}(\,\axi\sigma(\phi)\,)\\
    \end{eqnarray*}
So we now assume that $\pi=\pi_{1}\pon\pi_{2}$ for some
$\pi_{1},\pi_{2}\in\pvs$. We need to show
$\sigma(\pi)\sim\sigma(\pi_{1})\pon\,\sigma(\pi_{2})$. Once again,
we simply compute the minimal transforms:
    \begin{eqnarray*}
    {\cal M}\circ\sigma(\pi)&=&\bar{\sigma}\circ{\cal M}(\pi)\\
    &=&\bar{\sigma}\circ{\cal M}(\pi_{1}\pon\pi_{2})\\
    &=&\bar{\sigma}(\,{\cal M}(\pi_{1})\pon\,{\cal M}(\pi_{2})\,)\\
    &=&\bar{\sigma}\circ{\cal M}(\pi_{1})\pon\,\bar{\sigma}\circ{\cal M}(\pi_{2})\\
    &=&{\cal M}\circ\sigma(\pi_{1})\pon\,{\cal M}\circ\sigma(\pi_{2})\\
    &=&{\cal M}(\,\sigma(\pi_{1})\pon\,\sigma(\pi_{2})\,)\\
    \end{eqnarray*}
So we now assume that $\pi=\gen x\pi_{1}$ and
$\sigma(x)\not\in\free(\sigma(\pi))$. We need to show that
$\sigma(\pi)\sim\gen\sigma(x)\sigma(\pi_{1})$. Likewise, we shall
compute minimal transforms:
    \begin{eqnarray*}
    {\cal M}\circ\sigma(\pi)&=&\bar{\sigma}\circ{\cal M}(\pi)\\
    &=&\bar{\sigma}\circ{\cal M}(\gen x\pi_{1})\\
    \mbox{$n=\min\{k:[k/x]\mbox{ valid for }{\cal M}(\pi_{1})\}$}\
    \rightarrow
    &=&\bar{\sigma}(\,\gen n{\cal M}(\pi_{1})[n/x]\,)\\
    &=&\gen \bar{\sigma}(n)\bar{\sigma}(\,{\cal
    M}(\pi_{1})[n/x]\,)\\
    &=&\gen n\,\bar{\sigma}\circ [n/x]\circ{\cal
    M}(\pi_{1})\\
    \mbox{A: to be proved}\ \rightarrow
    &=&\gen n\,[n/\sigma(x)]\circ\bar{\sigma}\circ{\cal M}(\pi_{1})\\
    &=&\gen n\,[n/\sigma(x)]\circ{\cal M}\circ\sigma(\pi_{1})\\
    &=&\gen n\,{\cal M}[\sigma(\pi_{1})][n/\sigma(x)]\\
    \mbox{B: to be proved}\ \rightarrow
    &=&\gen m\,{\cal M}[\sigma(\pi_{1})][m/\sigma(x)]\\
    \mbox{$m=\min\{k:[k/\sigma(x)]\mbox{ valid for }{\cal M}[\sigma(\pi_{1})]\}$}\
    \rightarrow
    &=&{\cal M}(\gen\sigma(x)\sigma(\pi_{1}))\\
    \end{eqnarray*}
So it remains to justify point A and B. First we deal with point A:
it is sufficient to prove the equality
$\bar{\sigma}\circ[n/x]\circ{\cal M}(\pi_{1})=
[n/\sigma(x)]\circ\bar{\sigma}\circ{\cal M}(\pi_{1})$, which follows
from lemma~(\ref{logic:lemma:FUAP:mintransvalidsub:mpi1}) and
$\sigma(x)\not\in\sigma(\free(\pi))$, itself a consequence of
$\sigma(x)\not\in\free(\sigma(\pi))$ and
proposition~(\ref{logic:prop:FUAP:esssubstprop:free:commute}). We
now deal with point B: it is sufficient to prove the equivalence
$\mbox{$[k/x]$ valid for ${\cal M}(\pi_{1})$}\ \Leftrightarrow\
\mbox{$[k/\sigma(x)]$ valid for $\bar{\sigma}\circ{\cal
M}(\pi_{1})$}$ which follows from
lemma~(\ref{logic:lemma:FUAP:mintransvalidsub:n:equivalence}) and
the fact that $\sigma(x)\not\in\sigma(\free(\pi))$.
\end{proof}

As we have discussed prior to
proposition~(\ref{logic:prop:FUAP:esssubstprop:charac}) we cannot
write the equivalence $\sigma(\pi)\sim\gen\sigma(x)\sigma(\pi_{1})$
in general when $\pi=\gen x\pi_{1}$ and $\sigma$ is an essential
substitution. However, we can write
$\sigma(\pi)\sim\gen\tau(x)\tau(\pi_{1})$ whenever $\tau$ is an
essential substitution which coincides with $\sigma$ on
$V\setminus\{x\}$ and is such that
$\tau(x)\not\in\free(\sigma(\pi))$. This is very useful, and even
more so knowing that such an essential substitution $\tau$ always
exists. The following is the counterpart of
proposition~(\ref{logic:prop:FOPL:esssubstprop:tau})\,:

\begin{prop}\label{logic:prop:FUAP:esssubstprop:tau}
Let $V,W$ be sets and $\sigma:\pvs\to{\bf\Pi}(W)$ be an essential
proof substitution. Let $\pi=\gen x\pi_{1}$ where $x\in V$,
$\pi_{1}\in\pvs$. There exists an essential substitution
$\tau:\pvs\to{\bf\Pi}(W)$ such that $\tau=\sigma$ on
$V\setminus\{x\}$ and $\tau(x)\not\in\free(\sigma(\pi))$.
Furthermore, for any such $\tau$ we have the substitution
equivalence:
    \[
    \sigma(\pi)\sim\gen\tau(x)\tau(\pi_{1})
    \]
\end{prop}
\begin{proof}
We shall first prove the substitution equivalence. So let
$\tau:\pvs\to{\bf\Pi}(W)$ be an essential proof substitution which
coincide with $\sigma$ on $V\setminus\{x\}$ and such that
$\tau(x)\not\in\free(\sigma(\pi))$. We need to show that
$\sigma(\pi)\sim\gen\tau(x)\tau(\pi_{1})$ where $\sim$ denotes the
substitution congruence on ${\bf\Pi}(W)$. However, since
$\free(\pi)\subseteq V\setminus\{x\}$ we see that $\sigma$ and
$\tau$ are essential substitutions which coincide on $\free(\pi)$.
It follows from
proposition~(\ref{logic:prop:FUAP:esssubstprop:support}) that
$\sigma(\pi)\sim\tau(\pi)$. Hence it is sufficient to prove that
$\tau(\pi)\sim\gen\tau(x)\tau(\pi_{1})$. Applying
proposition~(\ref{logic:prop:FUAP:esssubstprop:charac}) it is
sufficient to show that $\tau(x)\not\in\free(\tau(\pi))$. However,
by assumption we have $\tau(x)\not\in\free(\sigma(\pi))$ and so:
    \[
    \tau(x)\not\in\free(\sigma(\pi))=\sigma(\free(\pi))=\tau(\free(\pi))=\free(\tau(\pi))
    \]
where we have used
proposition~(\ref{logic:prop:FUAP:esssubstprop:free:commute}). It
remains to show that such an essential substitution
$\tau:\pvs\to{\bf\Pi}(W)$ exists. Suppose we have proved that
$\free(\sigma(\pi))$ is a proper subset of $W$. Then there exists
$y^{*}\in W$ such that $y^{*}\not\in\free(\sigma(\pi))$. Consider
the map $\tau:V\to W$ defined by:
    \[
    \forall u\in V\ ,\ \tau(u)=\left\{
        \begin{array}{lcl}
        \sigma(u)&\mbox{\ if\ }&u\in V\setminus\{x\}\\
        y^{*}&\mbox{\ if\ }&u=x
        \end{array}
    \right.
    \]
Then it is clear that $\tau$ coincides with $\sigma$ on
$V\setminus\{x\}$ and $\tau(x)\not\in\free(\sigma(\pi))$. In order
to show the existence of $\tau:\pvs\to{\bf\Pi}(W)$ it is sufficient
to show the existence of an essential substitution associated with
$\tau:V\to W$. From
theorem~(\ref{logic:the:FUAP:esssubst:existence}) of
page~\pageref{logic:the:FUAP:esssubst:existence} it is sufficient to
show that $|W|$ is an infinite cardinal, or that it is finite with
$|V|\leq|W|$. However, this follows immediately from the existence
of the essential substitution $\sigma:\pvs\to{\bf\Pi}(W)$ and
theorem~(\ref{logic:the:FUAP:esssubst:existence}). So it remains to
show that $\free(\sigma(\pi))$ is a proper subset of $W$. This is
clearly true if $|W|$ is an infinite cardinal. So we may assume that
$|W|$ if finite, in which case we have $|V|\leq |W|$. In particular,
$|V|$ is also a finite cardinal and since $x\not\in\free(\pi)$ we
have $|\free(\pi)|<|V|$. Hence we have the following inequalities:
    \[
    |\free(\sigma(\pi))|=|\sigma(\free(\pi))|\leq|\free(\pi)|<|V|\leq|W|
    \]
So $\free(\sigma(\pi))$ is indeed a proper subset of $W$, as
requested.
\end{proof}
