Having introduced the valuation modulo $\vals:\pvs\to\pv$ in
definition~(\ref{logic:def:FUAP:valuationmod:valuation:modulo}) and
clarified its domain of clean proofs in
definition~(\ref{logic:def:FUAP:almostclean:definition}), one of the
first task facing us is to extend
proposition~(\ref{logic:prop:FUAP:validsubtotclean:valuation:commute})
and establish a result of the form
$\vals\circ\sigma(\pi)=\sigma\circ\vals(\pi)$ whenever $\pi$ is a
clean proof and $\sigma:V\to W$ is valid for $\pi$. As it turns out,
the result is true and will be proved in
proposition~(\ref{logic:prop:FUAP:strongvalsubalmostclean:valuation:commute}).
What may be seen as somewhat surprising is the fact that the
equality is true, rather than the mere substitution equivalence
$\vals\circ\sigma(\pi)\sim\sigma\circ\vals(\pi)$. So we shall have a
stronger result than expected. Now if
proposition~(\ref{logic:prop:FUAP:strongvalsubalmostclean:valuation:commute})
is going to be true, then it must be true when $\pi$ is of the form
$\pi=\axi\phi$. We should be used to this idea by now: the proof
$\sigma(\pi)$ cannot be meaningful unless the substitution $\sigma$
behaves adequately on the {\em axioms} of the proof. So suppose
$\pi=\axi\phi$ is a clean proof. Then $\phi$ must be an axiom modulo
and consequently $\vals(\pi)=\phi$. Since
$\sigma(\pi)=\axi\sigma(\phi)$, we cannot hope to have the equality
$\vals\circ\sigma(\pi)=\sigma\circ\vals(\pi)$ unless $\sigma(\phi)$
is itself an axiom modulo. Hence, we need to establish that if
$\sigma$ is valid for an axiom modulo $\phi$, then $\sigma(\phi)$ is
itself an axiom modulo. This will be done in
proposition~(\ref{logic:prop:FUAP:valsubaxmodulo:axiom:modulo})
below. Now let $\phi\in\avs$ be an arbitrary axiom modulo, From
definition~(\ref{logic:def:FUAP:valuationmod:axiom:modulo}), the
formula $\phi$ is substitution equivalent to an axiom of first order
logic $\psi$. We want to show that $\sigma(\phi)$ is also an axiom
modulo, and the first thing which comes to mind is to attempt
proving the equivalence $\sigma(\phi)\sim\sigma(\psi)$. From
proposition~(\ref{logic:lemma:FUAP:substitution:axiom}) we know that
$\sigma(\psi)$ is itself an axiom of first order logic, provided
$\sigma$ is valid for $\psi$. Unfortunately, this approach is not
going to work. We are assuming that $\sigma$ is valid for $\phi$. We
have no control on whether $\sigma$ is also valid for $\psi$. So we
need to find another route and think in terms of minimal transforms
which have proved very handy in the past. From the equivalence
$\phi\sim\psi$ and
theorem~(\ref{logic:the:FOPL:mintransfsubcong:kernel}) of
page~\pageref{logic:the:FOPL:mintransfsubcong:kernel} we obtain
${\cal M}(\phi)={\cal M}(\psi)$ and consequently
$\bar{\sigma}\circ{\cal M}(\phi)=\bar{\sigma}\circ{\cal M}(\psi)$.
Having assumed that $\sigma$ is valid for $\phi$ we can apply
theorem~(\ref{logic:the:FOPL:commute:mintransform:validsub}) of
page~\pageref{logic:the:FOPL:commute:mintransform:validsub} and
obtain ${\cal M}\circ\sigma(\phi)=\bar{\sigma}\circ{\cal M}(\psi)$.
Since $\psi$ is an axiom of first order logic, it is not
unreasonable to hope that ${\cal M}(\psi)$ is also an axiom of first
order logic. We also know from
proposition~(\ref{logic:def:FOPL:commute:minextension:valid}) that
$\bar{\sigma}$ is always valid for ${\cal M}(\psi)$ and it follows
from proposition~(\ref{logic:lemma:FUAP:substitution:axiom}) that
$\bar{\sigma}\circ{\cal M}(\psi)$ is an axiom of first order logic.
Hence we see that the minimal transform of $\sigma(\phi)$ is an
axiom of first order logic. This may be enough for us to conclude
that $\sigma(\phi)$ is an axiom modulo. So here is the plan: on the
on hand, we need to show that ${\cal M}(\psi)$ is an axiom, i.e. we
need to establish that the minimal transform of an axiom is an
axiom. This will be done in
lemma~(\ref{logic:lemma:FUAP:valsubaxmodulo:phi:m:phi}) below. On
the other hand, we need to prove that if the minimal transform of a
formula is an axiom of first order logic, then the formula is an
axiom modulo. This will be the object of
lemma~(\ref{logic:lemma:FUAP:valsubaxmodulo:m:phi:phi}).

\index{axiom@Minimal transform of axiom}
\begin{lemma}\label{logic:lemma:FUAP:valsubaxmodulo:phi:m:phi}
Let $V$ be a set. Then for all $\phi\in\pv$ we have the implication:
    \[
    \phi\in\av\ \Rightarrow\ {\cal M}(\phi)\in{\bf A}(\bar{V})
    \]
In other words, the minimal transform of an axiom is an axiom.
\end{lemma}
\begin{proof}
We shall prove this implication by considering the five possible
types of axioms of
definition~(\ref{logic:def:FOPL:first:order:axiom}): first we assume
that $\phi$ is a simplification axiom as per
definition~(\ref{logic:def:FOPL:simplification:axiom}). Then
$\phi=\phi_{1}\to(\phi_{2}\to\phi_{1})$ for some
$\phi_{1},\phi_{2}\in\pv$ and from
definition~(\ref{logic:def:FOPL:mintransform:transform}) we obtain
${\cal M}(\phi)=\psi_{1}\to(\psi_{2}\to\psi_{1})$ where
$\psi_{1}={\cal M}(\phi_{1})$ and $\psi_{2}={\cal M}(\phi_{2})$. So
${\cal M}(\phi)$ is itself a simplification axiom on \pvb, and we
have ${\cal M}(\phi)\in{\bf A}(\bar{V})$. We now assume that $\phi$
is a Frege axiom as per
definition~(\ref{logic:def:FOPL:frege:axiom}). Then
$\phi=[\phi_{1}\to(\phi_{2}\to\phi_{3})]\to[(\phi_{1}\to\phi_{2})\to(\phi_{1}\to\phi_{3})]$
and consequently:
    \[
    {\cal M}(\phi)=[\psi_{1}\to(\psi_{2}\to\psi_{3})]\to[(\psi_{1}\to\psi_{2})\to(\psi_{1}\to\psi_{3})]
    \]
where $\psi_{1}={\cal M}(\phi_{1})$, $\psi_{2}={\cal M}(\phi_{2})$
and $\psi_{3}={\cal M}(\phi_{3})$. So ${\cal M}(\phi)$ is itself a
Frege axiom. Likewise, if $\phi$ is a transposition axiom as per
definition~(\ref{logic:def:FOPL:transposition:axiom}), then
$\phi=[(\phi_{1}\to\bot)\to\bot]\to\phi_{1}$ for some
$\phi_{1}\in\pv$ and it is clear that ${\cal M}(\phi)$ is itself a
transposition axiom. So we now assume that $\phi$ is a
quantification axiom as per
definition~(\ref{logic:def:FOPL:quantification:axiom}). Then
$\phi=\forall x(\phi_{1}\to\phi_{2})\to(\phi_{1}
    \to\forall x \phi_{2})$ for some $\phi_{1},\phi_{2}\in\pv$ such
that $x\not\in\free(\phi_{1})$. Using
definition~(\ref{logic:def:FOPL:mintransform:transform}) once more:
    \[
    {\cal M}(\phi)=\forall n(\,{\cal M}(\phi_{1})[n/x]\to{\cal
    M}(\phi_{2})[n/x]\,)\to(\,{\cal M}(\phi_{1})\to\forall m{\cal
    M}(\phi_{2})[m/x]\,)
    \]
where $n$ is the smallest integer $k$ such that $[k/x]$ is valid for
${\cal M}(\phi_{1})\to{\cal M}(\phi_{2})$, and $m$ is the smallest
integer $k\in\N$ such that $[k/x]$ is valid for ${\cal
M}(\phi_{2})$. Let us accept for now that $n=m$ and furthermore that
${\cal M}(\phi_{1})[n/x]={\cal M}(\phi_{1})$. Then:
    \[
    {\cal M}(\phi)=\forall n(\,{\cal M}(\phi_{1})\to{\cal
    M}(\phi_{2})[n/x]\,)\to(\,{\cal M}(\phi_{1})\to\forall n{\cal
    M}(\phi_{2})[n/x]\,)
    \]
which can be expressed as:
    \[
    {\cal M}(\phi)=\forall
    n(\psi_{1}\to\psi_{2})\to(\psi_{1}\to\forall n\psi_{2})
    \]
where $\psi_{1}={\cal M}(\phi_{1})$ and $\psi_{2}={\cal
M}(\phi_{2})[n/x]$. So we see that ${\cal M}(\phi)$ is itself a
quantification axiom on \pvb, provided we show
$n\not\in\free(\psi_{1})$. So a few things remain to be proved.
First we show that ${\cal M}(\phi_{1})[n/x]={\cal M}(\phi_{1})$.
Using proposition~(\ref{logic:prop:substitution:support}) it is
sufficient to show that $[n/x](u)=u$ for all $u\in\var({\cal
M}(\phi_{1}))$. It is therefore sufficient to prove that
$x\not\in\var({\cal M}(\phi_{1}))$. So suppose to the contrary that
$x\in\var({\cal M}(\phi_{1}))$. Since $x\in V$ we obtain
$x\in\var({\cal M}(\phi_{1}))\cap V$ and it follows from
proposition~(\ref{logic:prop:FOPL:mintransform:variables}) that
$x\in\free(\phi_{1})$ which contradicts our assumption. We now show
that $n\not\in\free(\psi_{1})$. This follows once again from
proposition~(\ref{logic:prop:FOPL:mintransform:variables}), since
$n\in\free(\psi_{1})=\free({\cal M}(\phi_{1}))$ implies
$n\in\free(\phi_{1})$ and consequently $n\in V$ which contradicts
$V\cap\N=\emptyset$. So it remains to show that $n=m$, for which it
is sufficient to show the equivalence:
    \[
    \mbox{$[k/x]$ valid for ${\cal M}(\phi_{1})\to{\cal M}(\phi_{2})$}\ \Leftrightarrow\
    \mbox{$[k/x]$ valid for ${\cal M}(\phi_{2})$}
    \]
The implication $\Rightarrow$ follows immediately from
proposition~(\ref{logic:prop:FOPL:valid:recursion:imp}) and it
remains to show $\Leftarrow$\,: so we assume that $k\in\N$ is such
that $[k/x]$ is valid for ${\cal M}(\phi_{2})$. We need to show that
it is also valid for ${\cal M}(\phi_{1})\to{\cal M}(\phi_{2})$.
Using proposition~(\ref{logic:prop:FOPL:valid:recursion:imp}), we
simply need to show that $[k/x]$ is valid for ${\cal M}(\phi_{1})$.
Using proposition~(\ref{logic:prop:FOPL:validsub:image}), the
identity mapping $i:\bar{V}\to\bar{V}$ being valid for ${\cal
M}(\phi_{1})$, it is sufficient to prove that ${\cal
M}(\phi_{1})[k/x]={\cal M}(\phi_{1})$, which follows from the
established fact that $x\not\in\var({\cal M}(\phi_{1}))$. This
completes our proof in the case when $\phi$ is a quantification
axiom on \pv. So we now assume that $\phi$ is a specialization axiom
as per definition~(\ref{logic:def:FOPL:specialization:axiom}). Then
$\phi=\forall x\phi_{1}\to\phi_{1}[y/x]$ where $\phi_{1}\in\pv$,
$x,y\in V$ and $[y/x]:\pv\to\pv$ is an essential substitution of $y$
in place of $x$. We have:
    \[
    {\cal M}(\phi)=\forall n\psi_{1}\to\psi_{1}^{*}
    \]
where $\psi_{1}={\cal M}(\phi_{1})[n/x]$, $\psi_{1}^{*}={\cal
M}\circ[y/x](\phi_{1})$ and $n$ is the smallest integer $k\in\N$
such that $[k/x]$ is valid for ${\cal M}(\phi_{1})$. Using
proposition~(\ref{logic:prop:FOPL:specialization:axiom:2}), in order
to show that ${\cal M}(\phi)$ is itself a specialization axiom, it
is sufficient to prove that $\psi_{1}^{*}\sim\psi_{1}[y/n]$ where
$[y/n]:\pvb\to\pvb$ is an essential substitution of $y$ in place for
$n$, and $\sim$ is the substitution congruence on~\pvb. Using the
minimal transform mapping $\bar{\cal M}:\pvb\to{\bf
P}(\bar{\bar{V}})$, by virtue of
theorem~(\ref{logic:the:FOPL:mintransfsubcong:kernel}) of
page~\pageref{logic:the:FOPL:mintransfsubcong:kernel} we simply need
to prove that $\bar{\cal M}(\psi_{1}^{*})=\bar{\cal
M}\circ[y/n](\psi_{1})$. Before we do so, let us say a few words on
the single variable substitutions involved in the proof so as to
avoid any possible confusion: first we have the essential
substitution $[y/x]:\pv\to\pv$ which is associated to the map
$[y/x]:V\to V$. We also have the minimal extension
$\overline{[y/x]}:\bar{V}\to\bar{V}$ which turns out to be exactly
the substitution of $y$ in place of $x$, so $\overline{[y/x]}=[y/x]$
this time on $\bar{V}$ rather than $V$. From
$[y/x]:\bar{V}\to\bar{V}$ we also obtain the minimal extension
$[y/x]:\bar{\bar{V}}\to\bar{\bar{V}}$, and of course we also have
the associated substitutions $[y/x]:\pvb\to\pvb$ and $[y/x]:{\bf
P}(\bar{\bar{V}})\to{\bf P}(\bar{\bar{V}})$ as per
definition~(\ref{logic:def:substitution}) (i.e. non-essential).
Hence the notation $[y/x]$ has several possible meanings depending
on the context. We have $[y/x]:A\to A$ where $A=V,\bar{V}$ and
$\bar{\bar{V}}$. We have $[y/x]:\pv\to\pv$ (essential) and
$[y/x]:{\bf P}(A)\to{\bf P}(A)$ (non-essential) with $A=\bar{V}$ and
$A=\bar{\bar{V}}$. Likewise, we have $[n/x]:\bar{V}\to\bar{V}$ with
minimal extension $[n/x]:\bar{\bar{V}}\to\bar{\bar{V}}$ and the
(non-essential) substitutions $[n/x]:\pvb\to\pvb$ and $[n/x]:{\bf
P}(\bar{\bar{V}})\to{\bf P}(\bar{\bar{V}})$. We also have the
essential substitution $[y/n]:\pvb\to\pvb$ and its associated
$[y/n]:\bar{V}\to\bar{V}$ with minimal extension
$[y/n]:\bar{\bar{V}}\to\bar{\bar{V}}$ and its associated
(non-essential) substitution $[y/n]:{\bf P}(\bar{\bar{V}})\to{\bf
P}(\bar{\bar{V}})$. From $[y/x]:\pv\to\pv$:
    \begin{eqnarray*}
    \bar{\cal M}(\psi_{1}^{*})&=&\bar{\cal M}\circ{\cal
    M}\circ[y/x](\phi_{1})\\
    \mbox{$[y/x]:\pvb\to\pvb$}\ \rightarrow
    &=&\bar{\cal M}\circ[y/x]\circ{\cal M}(\phi_{1})\\
    \mbox{Th.~(\ref{logic:the:FOPL:commute:mintransform:validsub})
    p.~\pageref{logic:the:FOPL:commute:mintransform:validsub}, $[y/x]$ valid for ${\cal M}(\phi_{1})$}
    \ \rightarrow&=&[y/x]\circ\bar{\cal M}\circ{\cal M}(\phi_{1})\\
    \mbox{A: to be proved}\ \rightarrow
    &=&[y/n]\circ[n/x]\circ\bar{\cal M}\circ{\cal M}(\phi_{1})\\
    \mbox{$[n/x]$ valid for ${\cal M}(\phi_{1})$}
    \ \rightarrow&=&[y/n]\circ\bar{\cal M}\circ[n/x]\circ{\cal M}(\phi_{1})\\
    \mbox{$[y/n]:{\bf P}(\bar{\bar{V}})\to{\bf P}(\bar{\bar{V}})$}\ \rightarrow
    &=&[y/n]\circ\bar{\cal M}(\psi_{1})\\
    \mbox{$[y/n]:\pvb\to\pvb$ essential}\ \rightarrow
    &=&\bar{\cal M}\circ[y/n](\psi_{1})
    \end{eqnarray*}
So it remains to show point A: using
proposition~(\ref{logic:prop:substitution:support}) it is sufficient
to show that the maps $[y/x]:\bar{\bar{V}}\to\bar{\bar{V}}$ and
$[y/n]\circ[n/x]:\bar{\bar{V}}\to\bar{\bar{V}}$ coincide on
$\var(\bar{\cal M}\circ{\cal M}(\phi_{1}))$. So let
$u\in\var(\bar{\cal M}\circ{\cal M}(\phi_{1}))$. We need to show
that $[y/x](u)=[y/n]\circ[n/x](u)$. Since
$\bar{\bar{V}}=\bar{V}\uplus\bar{\N}$, we shall distinguish two
cases: first we assume that $u\in\bar{\N}$. Since
$\bar{V}\cap\bar{\N}=\emptyset$, in particular we have
$u\not\in\{x,n\}$ and the equality is clear. Next we assume that
$u\in\bar{V}$. Then we have $u\in\var(\bar{\cal M}\circ{\cal
M}(\phi_{1}))\cap\bar{V}$ and it follows from
proposition~(\ref{logic:prop:FOPL:mintransform:variables}) that
$u\in\free({\cal M}(\phi_{1}))=\free(\phi_{1})\subseteq V$. Since
$V\cap\N=\emptyset$, in particular we have $u\neq n$. We shall
distinguish two further cases: first we assume that $u=x$. Then the
equality is clear. Next we assume that $u\neq x$. Then we have
$u\not\in\{x,n\}$ and the equality is again clear.
\end{proof}

The following lemma is hard. By this we mean the amount of small
technical details to check is substantial. The probability of
getting it wrong is a lot higher than in any other results. Most of
the things we do in these notes are easy and can be followed without
any pain. In fact, many of the proofs are so obvious that they
hardly need to be checked. They are proofs which would be omitted in
a typical mathematical textbook. The following proof should not be
omitted. It is boring and tedious but we are very keen to make sure
the lemma is true.

\begin{lemma}\label{logic:lemma:FUAP:valsubaxmodulo:m:phi:phi}
Let $V$ be a set. Then for all $\phi\in\pv$ we have the implication:
    \[
    {\cal M}(\phi)\in{\bf A}(\bar{V})\ \Rightarrow\ \phi\in\avs
    \]
i.e. if the minimal transform of $\phi$ is an axiom then $\phi$ is
an axiom modulo.
\end{lemma}
\begin{proof}
We shall prove the implication by considering the five possible
cases of axioms as per
definition~(\ref{logic:def:FOPL:first:order:axiom}). First we assume
that ${\cal M}(\phi)$ is a simplification axiom as per
definition~(\ref{logic:def:FOPL:simplification:axiom}). Then ${\cal
M}(\phi)=\psi_{1}\to(\psi_{2}\to\psi_{1})$ where
$\psi_{1},\psi_{2}\in\pvb$. We need to show that $\phi$ is an axiom
modulo. Using theorem~(\ref{logic:the:unique:representation}) of
page~\pageref{logic:the:unique:representation}, the formula $\phi$
can be of four possible types: we can have $\phi=(x\in y)$ for some
$x,y\in V$ or $\phi=\bot$ or $\phi=\phi_{1}\to\phi_{2}$ for some
$\phi_{1},\phi_{2}\in\pv$ or $\phi=\forall x\phi_{1}$ for some
$\phi_{1}\in\pv$ and $x\in V$. Looking at
definition~(\ref{logic:def:FOPL:mintransform:transform}), the
corresponding values for ${\cal M}(\phi)$ are ${\cal M}(\phi)=(x\in
y)$ or ${\cal M}(\phi)=\bot$ or ${\cal M}(\phi)={\cal
M}(\phi_{1})\to{\cal M}(\phi_{2})$ or ${\cal M}(\phi)=\forall n{\cal
M}(\phi_{1})[n/x]$ for some $n\in\N$. Since ${\cal
M}(\phi)=\psi_{1}\to(\psi_{2}\to\psi_{1})$, the uniqueness property
of theorem~(\ref{logic:the:unique:representation}) tells us the only
possible case is $\phi=\phi_{1}\to\phi_{2}$ and ${\cal
M}(\phi)={\cal M}(\phi_{1})\to{\cal M}(\phi_{2})$, and furthermore
we have ${\cal M}(\phi_{1})=\psi_{1}$ and ${\cal
M}(\phi_{2})=\psi_{2}\to\psi_{1}$. Pushing this argument further, we
see that $\phi_{2}$ must be of the form
$\phi_{2}=\phi_{3}\to\phi_{4}$ and we have ${\cal
M}(\phi_{3})=\psi_{2}$ and ${\cal M}(\phi_{4})=\psi_{1}$. So we have
proved that $\phi=\phi_{1}\to(\phi_{3}\to\phi_{4})$ where ${\cal
M}(\phi_{1})={\cal M}(\phi_{4})$. If we define the formula
$\phi^{*}=\phi_{1}\to(\phi_{3}\to\phi_{1})$, then $\phi^{*}$ is a
simplification axiom which satisfies ${\cal M}(\phi)={\cal
M}(\phi^{*})$. From
theorem~(\ref{logic:the:FOPL:mintransfsubcong:kernel}) of
page~\pageref{logic:the:FOPL:mintransfsubcong:kernel}, it follows
that $\phi\sim\phi^{*}$ where $\sim$ is the substitution congruence
on \pv, and we have proved that $\phi$ is an axiom modulo, i.e.
$\phi\in\avs$ as requested. So we now assume that ${\cal M}(\phi)$
is a Frege axiom as per
definition~(\ref{logic:def:FOPL:frege:axiom}). Then we have the
equality:
    \[
    {\cal M}(\phi)=[\psi_{1}\to(\psi_{2}\to\psi_{3})]\to[(\psi_{1}\to\psi_{2})\to(\psi_{1}\to\psi_{3})]
    \]
for some $\psi_{1},\psi_{2},\psi_{3}\in\pvb$. From our previous
argument we see that $\phi$ must be of the form
$\phi=\phi_{1}\to\phi_{2}$ with ${\cal
M}(\phi_{1})=\psi_{1}\to(\psi_{2}\to\psi_{3})$ together with ${\cal
M}(\phi_{2})=(\psi_{1}\to\psi_{2})\to(\psi_{1}\to\psi_{3})$. So
$\phi_{1}$ must be of the form $\phi_{1}=\phi_{3}\to\phi_{4}$ with
${\cal M}(\phi_{3})=\psi_{1}$ and ${\cal
M}(\phi_{4})=\psi_{2}\to\psi_{3}$. Pushing this argument further,
and renaming the formulas involved, we see that $\phi$ must be of
the form:
    \[
    \phi=[\phi_{1}\to(\phi_{2}\to\phi_{3})]\to[(\phi_{4}\to\phi_{5})\to(\phi_{6}\to\phi_{7})]
    \]
with ${\cal M}(\phi_{1})={\cal M}(\phi_{4})={\cal M}(\phi_{6})$,
${\cal M}(\phi_{2})={\cal M}(\phi_{5})$ and ${\cal
M}(\phi_{3})={\cal M}(\phi_{7})$. Let:
    \[
    \phi^{*}=[\phi_{1}\to(\phi_{2}\to\phi_{3})]\to[(\phi_{1}\to\phi_{2})\to(\phi_{1}\to\phi_{3})]
    \]
Then ${\cal M}(\phi)={\cal M}(\phi^{*})$ and consequently
$\phi\sim\phi^{*}$, so $\phi$ is an axiom modulo. We now assume that
${\cal M}(\phi)$ is a transposition axiom as per
definition~(\ref{logic:def:FOPL:transposition:axiom}). Then ${\cal
M}(\phi)=[(\psi_{1}\to\bot)\to\bot]\to\psi_{1}$ for some
$\psi\in\pvb$, and it follows that $\phi$ must be of the form
$\phi=[(\phi_{1}\to\bot)\to\bot]\to\phi_{2}$ where ${\cal
M}(\phi_{1})={\cal M}(\phi_{2})$. Defining
$\phi^{*}=[(\phi_{1}\to\bot)\to\bot]\to\phi_{1}$ we obtain ${\cal
M}(\phi)={\cal M}(\phi^{*})$ which is $\phi\sim\phi^{*}$ and $\phi$
is an axiom modulo as requested. So we now assume that ${\cal
M}(\phi)$ is a quantification axiom as per
definition~(\ref{logic:def:FOPL:quantification:axiom}). Then we have
the equality:
    \[
    {\cal M}(\phi)=\forall z(\psi_{1}\to\psi_{2})\to(\psi_{1}
    \to\forall z \psi_{2})
    \]
for some $\psi_{1},\psi_{2}\in\pvb$ and $z\in\bar{V}$ such that
$z\not\in\free(\psi_{1})$. So $\phi$ must be of the form
$\phi=\phi_{7}\to\phi_{8}$ with ${\cal M}(\phi_{7})=\forall
z(\psi_{1}\to\psi_{2})$ and ${\cal M}(\phi_{8})=\psi_{1}\to\forall
z\psi_{2}$. So $\phi_{7}$ must be of the form $\phi_{7}=\forall
x\phi_{5}$ for some $x\in V$ and $\phi_{5}\in\pv$, where ${\cal
M}(\phi_{5})[n/x]=\psi_{1}\to\psi_{2}$ and $n=z$, and $n$ is also
the smallest integer $k$ such that $[k/x]$ is valid for ${\cal
M}(\phi_{5})$. Furthermore, $\phi_{8}$ must be of the form
$\phi_{8}=\phi_{3}\to\phi_{6}$ where ${\cal M}(\phi_{3})=\psi_{1}$
and ${\cal M}(\phi_{6})=\forall n\psi_{2}$. Continuing in this way,
we see that $\phi_{5}$ must be of the form
$\phi_{5}=\phi_{1}\to\phi_{2}$ where ${\cal
M}(\phi_{1})[n/x]=\psi_{1}$ together with ${\cal
M}(\phi_{2})[n/x]=\psi_{2}$, while $\phi_{6}$ must be of the form
$\phi_{6}=\forall y\phi_{4}$ for some $y\in V$ and $\phi_{4}\in\pv$,
with ${\cal M}(\phi_{4})[n/y]=\psi_{2}$. Note that $n$ is also the
smallest integer such that $[k/y]$ is valid for ${\cal
M}(\phi_{4})$. So we have proved that $\phi$ is of the form:
    \[
    \phi=\forall x(\phi_{1}\to\phi_{2})\to(\phi_{3}\to\forall
    y\phi_{4})
    \]
where ${\cal M}(\phi_{1})[n/x]={\cal M}(\phi_{3})$, ${\cal
M}(\phi_{2})[n/x]={\cal M}(\phi_{4})[n/y]$ and $n=z\in\N$. We also
proved that $n=z$ is the smallest integer $k$ such that $[k/x]$ is
valid for ${\cal M}(\phi_{1})\to{\cal M}(\phi_{2})$ and it is also
the smallest integer $k$ such that $[k/y]$ is valid for ${\cal
M}(\phi_{4})$. We now consider the formula $\phi^{*}$ defined as:
    \[
    \phi^{*}=\forall x(\phi_{1}\to\phi_{2})\to(\phi_{1}\to\forall
    x\phi_{2})
    \]
Let us accept for now that $x\not\in\free(\phi_{1})$. Then
$\phi^{*}$ is a quantification axiom, and in order to show that
$\phi$ is an axiom modulo it is sufficient to prove that
$\phi\sim\phi^{*}$. So it is sufficient to show that
$\phi_{1}\sim\phi_{3}$ and $\forall x\phi_{2}\sim\forall y\phi_{4}$.
First we show that $\phi_{1}\sim\phi_{3}$. We need to show that
${\cal M}(\phi_{1})={\cal M}(\phi_{3})$. Having established that
${\cal M}(\phi_{1})[n/x]={\cal M}(\phi_{3})$, we simply need to
prove that ${\cal M}(\phi_{1})={\cal M}(\phi_{1})[n/x]$. Using
proposition~(\ref{logic:prop:substitution:support}), it is
sufficient to prove that $u=[n/x](u)$ for every $u\in\var({\cal
M}(\phi_{1}))$. So it is sufficient to show $x\not\in\var({\cal
M}(\phi_{1}))$. So suppose to the contrary that $x\in\var({\cal
M}(\phi_{1}))$. Since $x\in V$ we obtain $x\in\var({\cal
M}(\phi_{1}))\cap V$ and it follows from
proposition~(\ref{logic:prop:FOPL:mintransform:variables}) that
$x\in\free(\phi_{1})$ which is a contradiction. We now prove that
$\forall x\phi_{2}\sim\forall y\phi_{4}$. We need to show ${\cal
M}(\forall x\phi_{2})={\cal M}(\forall y\phi_{4})$. We already know
that $n$ is the smallest integer $k$ such that $[k/y]$ is valid for
${\cal M}(\phi_{4})$. So ${\cal M}(\forall y\phi_{4})=\forall n{\cal
M}(\phi_{4})[n/y]$. Let us accept for now that $n$ is also the
smallest integer $k$ such that $[k/x]$ is valid for ${\cal
M}(\phi_{2})$. Then ${\cal M}(\forall x\phi_{2})=\forall n{\cal
M}(\phi_{2})[n/x]$ and the equality ${\cal M}(\forall
x\phi_{2})={\cal M}(\forall y\phi_{4})$ follows from the established
fact that ${\cal M}(\phi_{2})[n/x]={\cal M}(\phi_{4})[n/y]$. So it
remains to show that $n$ is indeed the smallest integer $k$ such
that $[k/x]$ is valid for ${\cal M}(\phi_{2})$. However, we have
established the fact it is the smallest integer $k$ such that
$[k/x]$ is valid for ${\cal M}(\phi_{1})\to{\cal M}(\phi_{2})$. It
is therefore sufficient to prove the equivalence:
    \[
    \mbox{$[k/x]$ valid for ${\cal M}(\phi_{1})\to{\cal M}(\phi_{2})$}
    \ \Leftrightarrow\ \mbox{$[k/x]$ valid for ${\cal M}(\phi_{2})$}
    \]
The implication $\Rightarrow$ follows immediately from
proposition~(\ref{logic:prop:FOPL:valid:recursion:imp}). So it
remains to show $\Leftarrow$\,: so we assume that $k\in\N$ is such
that $[k/x]$ is valid for ${\cal M}(\phi_{2})$. We need to show that
$[k/x]$ is also valid for ${\cal M}(\phi_{1})\to{\cal M}(\phi_{2})$.
Using proposition~(\ref{logic:prop:FOPL:valid:recursion:imp}) it is
sufficient to show that $[k/x]$ is valid for ${\cal M}(\phi_{1})$.
The identity mapping $i:\bar{V}\to\bar{V}$ being valid for ${\cal
M}(\phi_{1})$, using
proposition~(\ref{logic:prop:FOPL:validsub:image}), it is sufficient
to prove that ${\cal M}(\phi_{1})={\cal M}(\phi_{1})[k/x]$ which
follows from the established fact that $x\not\in\var({\cal
M}(\phi_{1}))$. So we have proved that $\phi\sim\phi^{*}$ and $\phi$
is indeed an axiom modulo, provided we show
$x\not\in\free(\phi_{1})$. So suppose to the contrary that
$x\in\free(\phi_{1})$. Using
proposition~(\ref{logic:prop:FOPL:mintransform:variables}) we obtain
$x\in\free({\cal M}(\phi_{1}))$. However, recall that ${\cal
M}(\phi_{1})[n/x]=\psi_{1}$ and furthermore $[n/x]$ is valid for
${\cal M}(\phi_{1})$. Using
proposition~(\ref{logic:prop:FOPL:valid:free:commute}) we obtain
$n=[n/x](x)\in[n/x](\,\free({\cal M}(\phi_{1}))\,)=\free(\psi_{1})$.
So we see that $n\in\free(\psi_{1})$, i.e. $z\in\free(\psi_{1})$
which contradicts our initial assumption. This completes our proof
in the case when ${\cal M}(\phi)$ is a quantification axiom. We now
assume that ${\cal M}(\phi)$ is a specialization axiom as per
definition~(\ref{logic:def:FOPL:specialization:axiom}). Then:
    \[
    {\cal M}(\phi) =\forall z\psi_{1}\to\psi_{1}[y/z]
    \]
for some $\psi_{1}\in\pvb$ and $y,z\in\bar{V}$, where
$[y/z]:\pvb\to\pvb$ is an essential substitution of $y$ in place of
$z$. So $\phi$ must be of the form $\phi=\phi_{3}\to\phi_{2}$ where
${\cal M}(\phi_{3})=\forall z\psi_{1}$ and ${\cal
M}(\phi_{2})=\psi_{1}[y/z]$. It follows that $\phi_{3}$ must be of
the form $\phi_{3}=\forall x\phi_{1}$ for some $\phi_{1}\in\pv$ and
$x\in V$ such that ${\cal M}(\phi_{1})[n/x]=\psi_{1}$ where
$n=z\in\N$ and $n$ is also the smallest integer $k$ such that
$[k/x]$ is valid for ${\cal M}(\phi_{1})$. So we have proved that
$\phi$ is of the form $\phi=\forall x\phi_{1}\to\phi_{2}$. In order
to show that $\phi$ is an axiom modulo, it is sufficient to show
that $\phi$ is actually a specialization axiom. in order to do so,
we shall crucially distinguish two cases: we first consider the case
when $y\in\bar{V}$ is in fact an element of $V$. Using
proposition~(\ref{logic:prop:FOPL:specialization:axiom:2}), it is
sufficient to prove that $\phi_{2}\sim\phi_{1}[y/x]$ where
$[y/x]:\pv\to\pv$ is an essential substitution of $y$ in place of
$x$. Note that we cannot apply
proposition~(\ref{logic:prop:FOPL:specialization:axiom:2}) unless
$y$ is indeed an element of $V$. So it is sufficient to prove that
${\cal M}(\phi_{2})={\cal M}\circ[y/x](\phi_{1})$. Using
proposition~(\ref{logic:prop:FOPL:esssubst:mintransform:equiv:imp:equal})
it is in fact sufficient to prove the equivalence ${\cal
M}(\phi_{2})\sim{\cal M}\circ[y/x](\phi_{1})$ where $\sim$ also
denotes the substitution congruence on \pvb. It is therefore
sufficient to prove the equality $\bar{\cal M}\circ{\cal
M}(\phi_{2})=\bar{\cal M}\circ{\cal M}\circ[y/x](\phi_{1})$. It may
seem a bit odd that we do not attempt to prove the equality ${\cal
M}(\phi_{2})={\cal M}\circ[y/x](\phi_{1})$ directly, and choose
instead to make use of the minimal transform $\bar{\cal
M}:\pvb\to{\bf P}(\bar{\bar{V}})$. However, remember that ${\cal
M}(\phi_{2})=\psi_{1}[y/z]$ and $[y/z]:\pvb\to\pvb$ is an essential
substitution. From
proposition~(\ref{logic:prop:FOPL:esssubstprop:redefine}), an
essential substitution can always be redefined modulo the
substitution congruence. So it is very hard to say anything about
the exact value of $[y/z](\psi_{1})$. However, in this case we know
that $[y/z](\psi_{1})$ is the minimal transform of $\phi_{2}$. So
this allows us to conclude, and the easy way to do so is to use
$\bar{\cal M}$. So we need to show that $\bar{\cal M}\circ{\cal
M}(\phi_{2})=\bar{\cal M}\circ{\cal M}\circ[y/x](\phi_{1})$\,:

    \begin{eqnarray*}
    \bar{\cal M}\circ{\cal M}(\phi_{2})&=&\bar{\cal
    M}\circ[y/z](\psi_{1})\\
    z=n\ \rightarrow&=&\bar{\cal M}\circ[y/n](\psi_{1})\\
    \mbox{$[y/n]$ essential}\ \rightarrow
    &=&[y/n]\circ\bar{\cal M}(\psi_{1})\\
    {\cal M}(\phi_{1})[n/x]=\psi_{1}\ \rightarrow
    &=&[y/n]\circ\bar{\cal M}\circ[n/x]\circ{\cal M}(\phi_{1})\\
    \mbox{Th.~(\ref{logic:the:FOPL:commute:mintransform:validsub})
    p.~\pageref{logic:the:FOPL:commute:mintransform:validsub}, $[n/x]$ valid for ${\cal M}(\phi_{1})$}
    \ \rightarrow&=&[y/n]\circ[n/x]\circ\bar{\cal M}\circ{\cal M}(\phi_{1})\\
    \mbox{A: to be proved}\ \rightarrow&=&[y/x]\circ\bar{\cal M}\circ{\cal M}(\phi_{1})\\
    \mbox{B: to be proved}\ \rightarrow&=&\bar{\cal M}\circ[y/x]\circ{\cal M}(\phi_{1})\\
    \mbox{$[y/x]$ essential}\ \rightarrow&=&\bar{\cal M}\circ{\cal M}\circ[y/x](\phi_{1})\\\
    \end{eqnarray*}
So it remains to show point A and B. We shall first deal with point
A, for which it is sufficient to prove the equality
$[y/n]\circ[n/x](u)=[y/x](u)$ for all $u\in\var(\,\bar{\cal
M}\circ{\cal M}(\phi_{1})\,)$. Since
$\bar{\bar{V}}=\bar{V}\uplus\bar{\N}$ we shall distinguish two
cases: first we assume that $u\in\bar{\N}$. Then $u\not\in\{x,n\}$
and the equality is clear. Next we assume that $u\in\bar{V}$. Then
using proposition~(\ref{logic:prop:FOPL:mintransform:variables}) we
obtain the equality $u\in\free({\cal
M}(\phi_{1}))=\free(\phi_{1})\subseteq V$ and in particular $u\neq
n\in\N$. We shall distinguish two further cases: if $u=x$ then the
equality is clear. If $u\neq x$ then $u\not\in\{x,n\}$ and the
equality is again clear. We now deal with point B: here we crucially
need our assumption that $y\in V$. So  the substitution $[y/x]$ of
$y$ in place of $x$ is meaningful as a map $[y/x]:V\to V$, whose
minimal extension is $[y/x]:\bar{V}\to\bar{V}$. Using
proposition~(\ref{logic:def:FOPL:commute:minextension:valid}), this
minimal extension is valid for ${\cal M}(\phi_{1})$ and point B
follows from
theorem~(\ref{logic:the:FOPL:commute:mintransform:validsub}) of
page~\pageref{logic:the:FOPL:commute:mintransform:validsub}. This
completes our proof in the case when $y\in\bar{V}$ is in fact an
element of $V$. It remains to show that $\phi$ is a specialization
axiom in the case when $y\not\in V$. In this case, we shall see that
$x\not\in\free(\phi_{1})$. So let us accept this is true for now.
Applying proposition~(\ref{logic:prop:FOPL:specialization:axiom:2})
with $y=x$ and the essential substitution $[x/x]:\pv\to\pv$ being
the identity mapping, in order to show that $\phi$ is a
specialization axiom it is sufficient to prove that
$\phi_{2}\sim\phi_{1}$. Once again, we need to show that $\bar{\cal
M}\circ{\cal M}(\phi_{2})=\bar{\cal M}\circ{\cal M}(\phi_{1})$, and
our previous computation is still perfectly valid up to the point:
    \[
    \bar{\cal M}\circ{\cal M}(\phi_{2})=[y/x]\circ\bar{\cal M}\circ{\cal M}(\phi_{1})
    \]
Hence, we need to show that $[y/x]\circ\bar{\cal M}\circ{\cal
M}(\phi_{1})=\bar{\cal M}\circ{\cal M}(\phi_{1})$. Using
proposition~(\ref{logic:prop:substitution:support}), we need to show
that $[y/x](u)=u$ for all $u\in\var(\bar{\cal M}\circ{\cal
M}(\phi_{1}))$. In other words we need to prove that
$x\not\in\var(\bar{\cal M}\circ{\cal M}(\phi_{1}))$. So suppose to
the contrary that $x\in\var(\bar{\cal M}\circ{\cal M}(\phi_{1}))$.
Since $x\in V\subseteq\bar{V}$ we obtain:
    \[
    x\in\var(\bar{\cal M}\circ{\cal
    M}(\phi_{1}))\cap\bar{V}=\free({\cal
    M}(\phi_{1}))=\free(\phi_{1})
    \]
where we have made use of
proposition~(\ref{logic:prop:FOPL:mintransform:variables}). So we
obtain $x\in\free(\phi_{1})$, contradicting our accepted fact that
$x\not\in\free(\phi_{1})$. So it remains to show that
$x\not\in\free(\phi_{1})$. So suppose to the contrary that
$x\in\free(\phi_{1})$. Then we have $x\in\free({\cal M}(\phi_{1}))$.
Since ${\cal M}(\phi_{1})[n/x]=\psi_{1}$ and $[n/x]$ is valid for
${\cal M}(\phi_{1})$, using
proposition~(\ref{logic:prop:FOPL:valid:free:commute}) we obtain
$\free(\psi_{1})=[n/x](\free({\cal M}(\phi_{1}))$ and it follows
that $n\in\free(\psi_{1})$. Since ${\cal M}(\phi_{2})=\psi_{1}[y/n]$
and $[y/n]:\pvb\to\pvb$ is an essential substitution, from
proposition~(\ref{logic:prop:FOPL:esssubstprop:free:commute}) we
have $\free({\cal M}(\phi_{2}))=[y/n](\free(\psi_{1}))$ and it
follows that $y\in\free({\cal M}(\phi_{2}))=\free(\phi_{2})$, which
contradicts $y\not\in V$.
\end{proof}

The following proposition summarizes the results obtained in
lemma~(\ref{logic:lemma:FUAP:valsubaxmodulo:phi:m:phi}) and
lemma~(\ref{logic:lemma:FUAP:valsubaxmodulo:m:phi:phi}) with the
additional fact that if a minimal transform ${\cal M}(\phi)$ is an
axiom modulo, then it is in fact an axiom of first order logic.

\index{axiom@Minimal transform of axiom}
\begin{prop}\label{logic:prop:FUAP:valsubaxmodulo:min:transform}
Let $V$ be a set and $\phi\in\pv$. The following are equivalent:
    \begin{eqnarray*}
    (i)&&\phi\in\avs\mbox{\ , i.e. $\phi$ is an axiom modulo}\\
    (ii)&&{\cal M}(\phi)\in{\bf A}(\bar{V})\mbox{\ , i.e. ${\cal M}(\phi)$ is an axiom}\\
    (iii)&&{\cal M}(\phi)\in{\bf A}^{+}(\bar{V})\mbox{\ , i.e. ${\cal M}(\phi)$ is an axiom modulo}
    \end{eqnarray*}
where ${\cal M}:\pv\to\pvb$ is the minimal transform mapping of {\em
definition~(\ref{logic:def:FOPL:mintransform:transform})}.
\end{prop}
\begin{proof}
First we assume the equivalence $(i)\ \Leftrightarrow\ (ii)$ has
been proved, and we shall show $(ii)\ \Leftrightarrow\ (iii)$. From
definition~(\ref{logic:def:FUAP:valuationmod:axiom:modulo}) we have
${\bf A}(\bar{V})\subseteq{\bf A}^{+}(\bar{V})$ and the implication
$(ii)\ \Rightarrow\ (iii)$ is clear. We now show $(iii)\
\Rightarrow\ (ii)$\,: so we assume that ${\cal M}(\phi)$ is an axiom
modulo, i.e. ${\cal M}(\phi)\in{\bf A}^{+}(\bar{V})$. We need to
show that ${\cal M}(\phi)$ is in fact an axiom, i.e. ${\cal
M}(\phi)\in{\bf A}(\bar{V})$. However, from ${\cal M}(\phi)\in{\bf
A}^{+}(\bar{V})$ and the implication $(i)\ \Rightarrow\ (ii)$ we see
that ${\bar{\cal M}}\circ{\cal M}(\phi)\in{\bf A}(\bar{\bar{V}})$
where ${\bar{\cal M}}:\pvb\to{\bf P}(\bar{\bar{V}})$ is the minimal
transform mapping. Consider the map $p:\bar{\bar{V}}\to\bar{V}$ of
definition~(\ref{logic:def:FOPL:esssubst:weak:transform}). Then from
lemma~(\ref{logic:lemma:FOPL:esssubst:p:valid}) we know that $p$ is
valid for ${\bar{\cal M}}\circ{\cal M}(\phi)$. Hence using
lemma~(\ref{logic:lemma:FUAP:substitution:axiom}) we obtain
$p\circ{\bar{\cal M}}\circ{\cal M}(\phi)\in{\bf A}(\bar{V})$.
However, ${\cal N}=p\circ{\bar{\cal M}}$ is the weak transform of
definition~(\ref{logic:def:FOPL:esssubst:weak:transform}), and
applying proposition~(\ref{logic:lemma:FOPL:esssubst:NsM}) to the
identity mapping $\sigma:V\to V$ we obtain ${\cal N}\circ{\cal
M}(\phi)={\cal M}(\phi)$. Having proved that ${\cal N}\circ{\cal
M}(\phi)\in{\bf A}(\bar{V})$ we conclude that ${\cal M}(\phi)\in{\bf
A}(\bar{V})$ as requested. So we now prove the equivalence $(i)\
\Leftrightarrow\ (ii)$ starting with $(i)\ \Rightarrow\ (ii)$\,: we
assume that $\phi$ is an axiom modulo, i.e. $\phi\in\avs$. We need
to show that ${\cal M}(\phi)$ is an axiom, i.e. ${\cal
M}(\phi)\in{\bf A}(\bar{V})$. However, from
definition~(\ref{logic:def:FUAP:valuationmod:axiom:modulo}), there
exists an axiom $\psi\in\av$ such that $\phi\sim\psi$, where $\sim$
is the substitution congruence. Using
theorem~(\ref{logic:the:FOPL:mintransfsubcong:kernel}) of
page~\pageref{logic:the:FOPL:mintransfsubcong:kernel} we have ${\cal
M}(\phi)={\cal M}(\psi)$. It is therefore sufficient to prove that
${\cal M}(\psi)\in{\bf A}(\bar{V})$ which follows immediately from
$\psi\in\av$ and
lemma~(\ref{logic:lemma:FUAP:valsubaxmodulo:phi:m:phi}). It remains
to show the implication $(ii)\ \Rightarrow\ (i)$ which follows from
lemma~(\ref{logic:lemma:FUAP:valsubaxmodulo:m:phi:phi}).
\end{proof}

As discussed in the beginning of this section, we need to show that
the image of an axiom modulo by a valid substitution is an axiom
modulo. Having proved
proposition~(\ref{logic:prop:FUAP:valsubaxmodulo:min:transform}) we
can safely rely on minimal transforms and prove the result:

\begin{prop}\label{logic:prop:FUAP:valsubaxmodulo:axiom:modulo}
Let $V$, $W$ be sets and $\sigma:V\to W$ be a map. Let
$\phi\in\avs$\,:
    \[
    (\mbox{$\sigma$ valid for $\phi$})\ \Rightarrow\
    \sigma(\phi)\in{\bf A}^{+}(W)
    \]
i.e. the image of an axiom modulo by a valid substitution is an
axiom modulo.
\end{prop}
\begin{proof}
We assume that $\phi$ is an axiom modulo i.e. $\phi\in\avs$. We also
assume that $\sigma$ is valid for $\phi$. We need to show that
$\sigma(\phi)$ is an axiom modulo, i.e. $\sigma(\phi)\in{\bf
A}^{+}(W)$. From $\phi\in\avs$ and
proposition~(\ref{logic:prop:FUAP:valsubaxmodulo:min:transform}) we
obtain ${\cal M}(\phi)\in{\bf A}(\bar{V})$, i.e. the minimal
transform ${\cal M}(\phi)$ is an axiom. Using
proposition~(\ref{logic:def:FOPL:commute:minextension:valid}) the
minimal extension ${\bar \sigma}:\bar{V}\to\bar{W}$ is valid for
${\cal M}(\phi)$. Applying
lemma~(\ref{logic:lemma:FUAP:substitution:axiom}) we obtain
$\bar{\sigma}\circ{\cal M}(\phi)\in{\bf A}(\bar{W})$, i.e.
$\bar{\sigma}\circ{\cal M}(\phi)$ is an axiom. However, having
assumed that $\sigma$ is valid for $\phi$, from
theorem~(\ref{logic:the:FOPL:commute:mintransform:validsub}) of
page~\pageref{logic:the:FOPL:commute:mintransform:validsub} we have
$\bar{\sigma}\circ{\cal M}(\phi)={\cal M}\circ\sigma(\phi)$. It
follows that ${\cal M}\circ\sigma(\phi)\in{\bf A}(\bar{W})$ and from
proposition~(\ref{logic:prop:FUAP:valsubaxmodulo:min:transform}),
$\sigma(\phi)\in{\bf A}^{+}(W)$.
\end{proof}

The proof of
proposition~(\ref{logic:prop:FUAP:valsubaxmodulo:axiom:modulo})
actually works unchanged for essential substitutions
$\sigma:\pv\to{\bf P}(W)$. Hence, without any additional work, we
see that the image by an essential substitution of an axiom modulo
is an axiom modulo.

\begin{prop}\label{logic:prop:FUAP:valsubaxmodulo:essential}
Let $V$, $W$ be sets and $\sigma:\pv\to{\bf P}(W)$ be an essential
substitution. Then for all $\phi\in\pv$ we have the following
implication:
    \[
    \phi\in\avs\ \Rightarrow\ \sigma(\phi)\in{\bf A}^{+}(W)
    \]
i.e. the essential substitution image of an axiom modulo is an axiom
modulo.
\end{prop}
\begin{proof}
We assume that $\phi\in\avs$. We need to show that
$\sigma(\phi)\in{\bf A}^{+}(W)$. From $\phi\in\avs$ and
proposition~(\ref{logic:prop:FUAP:valsubaxmodulo:min:transform}) we
obtain ${\cal M}(\phi)\in{\bf A}(\bar{V})$. Recall that from
definition~(\ref{logic:def:FOPL:esssubstprop:essential}) the
essential substitution $\sigma:\pv\to{\bf P}(W)$ is associated to a
unique map $\sigma:V\to W$. Using
proposition~(\ref{logic:def:FOPL:commute:minextension:valid}) the
minimal extension ${\bar \sigma}:\bar{V}\to\bar{W}$ is valid for
${\cal M}(\phi)$. Applying
lemma~(\ref{logic:lemma:FUAP:substitution:axiom}) we obtain
$\bar{\sigma}\circ{\cal M}(\phi)\in{\bf A}(\bar{W})$. However from
definition~(\ref{logic:def:FOPL:esssubstprop:essential}) we have
$\bar{\sigma}\circ{\cal M}(\phi)={\cal M}\circ\sigma(\phi)$. It
follows that ${\cal M}\circ\sigma(\phi)\in{\bf A}(\bar{W})$ and from
proposition~(\ref{logic:prop:FUAP:valsubaxmodulo:min:transform}),
$\sigma(\phi)\in{\bf A}^{+}(W)$.
\end{proof}

This completes our section where we have carried out the study of
axioms modulo, and their images by minimal transforms, valid
substitutions and essential substitutions. We are now well-equipped
to establish the equality
$\vals\circ\sigma(\pi)=\sigma\circ\vals(\pi)$ in the case when $\pi$
is a clean proof and $\sigma$ is valid for $\pi$. This is the object
of
proposition~(\ref{logic:prop:FUAP:strongvalsubalmostclean:valuation:commute})
in the next section.
