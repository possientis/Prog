Let  $h:X\to Y$ be a homomorphism between two universal algebras $X$
and $Y$ of type $\alpha$. We know from
proposition~(\ref{logic:prop:UA:congruence:kernel}) that $\ker(h)$
is a congruence on $X$. From theorem~(\ref{logic:the:quotient}) we
obtain the quotient universal algebra $[X]$ of type $\alpha$ derived
from $X$ and $\ker(h)$. Furthermore from
proposition~(\ref{logic:prop:UA:subalg:homomorphic:image}) the
homomorphic image $h(X)$ is a sub-algebra of $Y$ and in particular,
it is also a universal algebra of type $\alpha$. As it turns out,
the algebras $[X]$ and $h(X)$ are isomorphic. This is of course
hardly surprising for anyone who has done a little bit of algebra
before. \index{morphism@First isomorphism theorem}
\begin{theorem}\label{logic:the:UA:firstiso}
Let $h:X\to Y$ be a homomorphism between two universal algebras $X$
and $Y$ of type $\alpha$. Let $[X]$ be the quotient universal
algebra of type $\alpha$ derived from $X$ and the congruence
$\ker(h)$. Let $h^{*}:[X]\to h(X)$ be defined as:
    \[
    \forall x\in X\ ,\ h^{*}([x])=h(x)
    \]
Then $h^{*}$ is an isomorphism between $[X]$ and $h(X)$.
\end{theorem}
\begin{proof}
Before we start, we should point out that the map $h^{*}:[X]\to
h(X)$ is well defined by the formula $h^{*}([x])=h(x)$ since $h(x)$
is independent of the particular choice of $x\in [x]$. Indeed, if
$x'\in [x]$, then $x'\sim x$ which is $(x',x)\in\ker(h)$ and
consequently $h(x')=h(x)$. Now we need to show that $h^{*}$ is a
bijective morphism. First we show that it is bijective. It is
clearly surjective, so we shall prove that $h^{*}$ is injective. Let
$x,x'\in X$ such that $h^{*}([x])=h^{*}([x'])$. We need to show that
$[x]=[x']$. However, our assumption can equally be written as
$h(x)=h(x')$ which is the same as $(x,x')\in\ker(h)$ or $x\sim x'$.
Hence we obtain $[x]=[x']$ as requested. It remains to show that
$h^{*}$ is a morphism. So let $f\in\alpha$ and
$x^{*}\in[X]^{\alpha(f)}$. We need to show that $h^{*}\circ
f(x^{*})=f\circ h^{*}(x^{*})$. However, for all $i\in\alpha(f)$ we
have $x^{*}(i)\in[X]$. So there exists $x_{i}\in X$ such that
$x^{*}(i)=[x_{i}]$. Let $x\in X^{\alpha(f)}$ be defined by
$x(i)=x_{i}$ for all $i\in\alpha(f)$. Then we have
$x^{*}(i)=[x_{i}]=[x(i)]=[x](i)$ and consequently $x^{*}=[x]$.
Hence:
    \begin{eqnarray*}
    h^{*}\circ f(x^{*})&=&h^{*}\circ f([x])\\
    \mbox{theorem~(\ref{logic:the:quotient})}\ \rightarrow
    &=&h^{*}([f(x)])\\
    &=&h(f(x))\\
    \mbox{$h$ is a morphism}\ \rightarrow&=&f\circ h(x)\\
    \mbox{A: to be proved}\ \rightarrow&=&f\circ h^{*}([x])\\
    &=&f\circ h^{*}(x^{*})
    \end{eqnarray*}
So it remains to show that $h(x)=h^{*}([x])$. For all
$i\in\alpha(f)$, we have:
    \[
    h(x)(i)=h(x(i))=h^{*}([x(i)])=h^{*}([x](i))=h^{*}([x])(i)
    \]
Note that this proof works very well when $\alpha(f)=0$.
\end{proof}


As an application of the first isomorphism
theorem~(\ref{logic:the:UA:firstiso}),  we now present a result
which we are unlikely to use again but which may be viewed as a
vindication of the effort we have put into the analysis of free
universal algebras. In theorem~(\ref{logic:the:main:existence}) of
page~\pageref{logic:the:main:existence}, given an arbitrary set
$X_{0}$ we showed the existence of a free universal algebra $X$ of
type $\alpha$ with free generator $X_{0}$. In
theorem~(\ref{logic:the:quotient}) of
page~\pageref{logic:the:quotient}, given a congruence $\sim$ on $X$
we showed how to construct the {\em quotient} universal algebra
$[X]$ of type $\alpha$. The following theorem shows that the
construction mechanism $X_{0}\to X\to [X]$ is as general as it gets:
{\em every} universal algebra of type $\alpha$ is in fact some
quotient universal algebra $[X]$ derived from a free universal
algebra $X$ and a congruence $\sim$ on $X$. Consequently, if we ever
wish to {\em construct} a universal algebra of a certain type and
with certain properties, there is absolutely no loss in generality
in first considering a free universal algebra, and subsequently
finding the appropriate congruence on it so as to fit the required
properties.

For instance, suppose we have a free universal algebra $X$ with a
binary operator $\otimes$. We would like this operator to be
commutative. We should only make sure our congruence $\sim$ on $X$
contains the set:
    \[
    A=\{(x\otimes y,y\otimes x):x,y\in X\}
    \]
If this is the case, the corresponding operator $\otimes$ on the
quotient universal algebra $[X]$ will indeed be commutative. For if
$x^{*},y^{*}\in [X]$ and $x,y\in X$ are such that $x^{*}=[x]$ and
$y^{*}=[y]$, then from theorem~(\ref{logic:the:quotient}) of
page~\pageref{logic:the:quotient} we have:
    \[
    x^{*}\otimes y^{*}=[x]\otimes [y]=[x\otimes y]=[y\otimes
    x]=[y]\otimes[x]=y^{*}\otimes x^{*}
    \]

\begin{theorem}\label{logic:the:quotient:free:algebra}
Any universal algebra of type $\alpha$ is isomorphic to the quotient
$[X]$ of some free universal algebra $X$ of type $\alpha$, relative
to some congruence on $X$.
\end{theorem}
\begin{proof}
Let $Y$ be a universal algebra of type $\alpha$ and $X_{0}=Y$. From
theorem~(\ref{logic:the:main:existence}) of
page~\pageref{logic:the:main:existence} there exists a free
universal algebra $X$ of type $\alpha$ with free generator $X_{0}$.
In particular $X_{0}\subseteq X$. Consider the identity mapping
$j:X_{0}\to Y$. Since $X_{0}$ is a free generator of $X$, there
exists a unique morphism $g:X\to Y$ such that $g_{|X_{0}}=j$. Let
$\sim\,=\ker(g)$ be the kernel of $g$. From
proposition~(\ref{logic:prop:UA:congruence:kernel}) we know that
$\sim$ is a congruence on $X$. We shall complete the proof of this
theorem by showing the corresponding quotient universal algebra
$[X]$ is isomorphic to~$Y$. From the first isomorphism
theorem~(\ref{logic:the:UA:firstiso}) we know that the map
$g^{*}:[X]\to g(X)$ defined by $g^{*}([x])=g(x)$ is an isomorphism.
So it is sufficient to show that $g(X)=Y$, i.e. that $g$ is a
surjective morphism. So let $y\in Y$. We need to show the existence
of $x\in X$ such that $g(x)=y$. But $Y=X_{0}\subseteq X$ and:
    \[
    g(y)=g_{|X_{0}}(y)=j(y)=y
    \]
\end{proof}
