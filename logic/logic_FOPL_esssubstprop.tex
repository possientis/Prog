Until now we have always started from a map $\sigma:V\to W$ and
investigated the existence of an associated essential substitution
$\sigma^{*}:\pv\to{\bf P}(W)$. Thanks to
theorem~(\ref{logic:the:FOPL:esssubst:existence}) of
page~\pageref{logic:the:FOPL:esssubst:existence} we have a simple
rule based on the cardinals $|V|$ and $|W|$ allowing us to determine
when such existence occurs. In this section, we wish to investigate
essential substitutions in their own right. We shall say that a map
$\sigma^{*}:\pv\to{\bf P}(W)$ is an {\em essential substitution}
whenever there exists a map $\sigma:V\to W$ such that $\sigma^{*}$
is an essential substitution associated with $\sigma$, as per
definition~(\ref{logic:def:FOPL:esssubst:esssubst}). We believe
essential substitutions should play an important role in
mathematical logic. Our study of substitutions started with the (not
essential) substitutions mappings $\sigma^{*}:\pv\to{\bf P}(W)$
associated with a map $\sigma:V\to W$ as per
definition~(\ref{logic:def:substitution}). These substitution
mappings allowed us to characterize the substitution congruence but
are somewhat limited by the fact that $\sigma$ is generally not a
valid substitution for $\phi\in\pv$. The introduction of valid
substitutions as per
definition~(\ref{logic:def:FOPL:valid:substitution}) allowed us to
move forward somehow, but the issue with valid substitutions
fundamentally remains: if a substitution $\sigma:V\to W$ is valid
for $\phi\in\pv$, then the associated $\sigma^{*}:\pv\to{\bf P}(W)$
behaves {\em properly} on $\phi$. In other words, it is meaningful
to speak of $\sigma^{*}(\phi)$. However, this validity of $\sigma$
for $\phi$ is a {\em local} property so to speak, and we are still
short of a map $\sigma^{*}:\pv\to{\bf P}(W)$ with the right global
property. This is what essential substitutions allow us to achieve:
if $\sigma^{*}:\pv\to{\bf P}(W)$  is an essential substitution then
$\sigma^{*}(\phi)$ is meaningful for all $\phi\in\pv$. It is as if
the associated map $\sigma:V\to W$ had the global property of being
valid for all $\phi\in\pv$.
\index{essential@Essential substitution
for formula}
\begin{defin}\label{logic:def:FOPL:esssubstprop:essential}
Let $V,W$ be sets. We say that a map $\sigma^{*}:\pv\to{\bf P}(W)$
is an {\em essential substitution} \ifand\ there exists $\sigma:V\to
W$ such that:
    \[
    {\cal M}\circ\sigma^{*}=\bar{\sigma}\circ{\cal M}
    \]
where $\bar{\sigma}:\bar{V}\to\bar{W}$ is the minimal extension and
${\cal M}$ is the minimal transform.
\end{defin}

The following proposition shows that any map $\sigma:V\to W$
associated with an essential substitution $\sigma^{*}:\pv\to{\bf
P}(W)$ is unique. This means an essential substitution $\sigma^{*}$
can safely be regarded as a map $\sigma^{*}:V\to W$ without any risk
of confusion. In fact, there is no need to have separate notations
$\sigma^{*}$ and $\sigma$. If $\sigma:\pv\to{\bf P}(W)$ is an
essential substitution, it is meaningful to speak of $\sigma(x)$.

\begin{prop}\label{logic:prop:FOPL:esssubstprop:uniqueness}
Let $V$, $W$ be sets and $\sigma^{*}:\pv\to{\bf P}(W)$ be an
essential substitution. Then the map $\sigma:V\to W$ associated with
$\sigma^{*}$ is unique.
\end{prop}
\begin{proof}
Let $\sigma_{0},\sigma_{1}:V\to W$ such that ${\cal
M}\circ\sigma^{*}=\bar{\sigma}_{i}\circ{\cal M}$ for all $i\in 2$.
We need to show that $\sigma_{0}(x)=\sigma_{1}(x)$ for all $x\in V$.
So let $x\in V$. Define $\phi=(x\in x)$. Then\,:
    \begin{eqnarray*}
    (\sigma_{0}(x)\in\sigma_{0}(x))&=&(\bar{\sigma}_{0}(x)\in\bar{\sigma}_{0}(x))\\
    &=&\bar{\sigma}_{0}(x\in x)\\
    &=&\bar{\sigma}_{0}\circ{\cal
    M}(\phi)\\
    &=&{\cal M}\circ\sigma^{*}(\phi)\\
    &=&\bar{\sigma}_{1}\circ{\cal
    M}(\phi)\\
    &=&(\sigma_{1}(x)\in\sigma_{1}(x))
    \end{eqnarray*}
From
$(\sigma_{0}(x)\in\sigma_{0}(x))=(\sigma_{1}(x)\in\sigma_{1}(x))$ we
conclude that $\sigma_{0}(x)=\sigma_{1}(x)$.
\end{proof}

The notion of essential substitution is tailor made for the
substitution congruence which is arguably the most important
congruence in \pv. If an essential substitution $\sigma:\pv\to{\bf
P}(W)$ is redefined without changing the class of $\sigma(\phi)$
modulo substitution, we still obtain an essential substitution which
is in fact identical to $\sigma$ when viewed as a map $\sigma:V\to
W$.
\begin{prop}\label{logic:prop:FOPL:esssubstprop:redefine}
Let $V,W$ be sets and $\sigma:\pv\to{\bf P}(W)$ be an essential
substitution. Let $\tau:\pv\to{\bf P}(W)$ be a map such that
$\sigma(\phi)\sim\tau(\phi)$ for all $\phi\in\pv$ where $\sim$ is
the substitution congruence on ${\bf P}(W)$. Then $\tau$ is itself
an essential substitution with associated map $\tau:V\to W$
identical to $\sigma$.
\end{prop}
\begin{proof}
Since $\sigma$ is essential we have ${\cal
M}\circ\sigma(\phi)=\bar{\sigma}\circ{\cal M}(\phi)$ for all
$\phi\in\pv$. Having assumed that $\sigma(\phi)\sim\tau(\phi)$ for
all $\phi\in\pv$, from
theorem~(\ref{logic:the:FOPL:mintransfsubcong:kernel}) of
page~\pageref{logic:the:FOPL:mintransfsubcong:kernel} we have ${\cal
M}\circ\sigma(\phi)={\cal M}\circ\tau(\phi)$. It follows that ${\cal
M}\circ\tau(\phi)=\bar{\sigma}\circ{\cal M}(\phi)$ and we conclude
that $\tau$ is itself essential with associated map $\sigma:V\to W$.
\end{proof}

Suppose $\sigma:\pv\to{\bf P}(W)$ is an essential substitution. We
have an associated map $\sigma:V\to W$ which we have decided to call
'$\sigma$' to keep our notations light and natural. But to every map
$\sigma:V\to W$ there is an associated variable substitution mapping
$\sigma^{*}:\pv\to{\bf P}(W)$ as per
definition~(\ref{logic:def:substitution}). It has been our practice
so far to denote the variable substitution $\sigma^{*}$ simply by
'$\sigma$'. Obviously we cannot keep calling everything '$\sigma$'
without creating great confusion. Whenever we are dealing with a map
$\sigma:V\to W$ and another map $\sigma:\pv\to{\bf P}(W)$, the fact
that these two maps have the same name is fine, as we can tell from
the context which map is being referred to. We are now confronted
with a situation where we have two maps $\sigma:\pv\to{\bf P}(W)$
which cannot be distinguished from the data type of their argument,
so to speak. Hence our notation $\sigma^{*}:\pv\to{\bf P}(W)$ for
the usual variable substitution as per
definition~(\ref{logic:def:substitution}). Now given $\phi\in\pv$,
we may be wondering what relationship there is between
$\sigma(\phi)$ and $\sigma^{*}(\phi)$. We cannot hope to have an
equality in general since from
proposition~(\ref{logic:prop:FOPL:esssubstprop:redefine}) we know
that $\sigma$ can be redefined modulo the substitution congruence
without affecting its associated map $\sigma:V\to W$. However, if
$\sim$ is the substitution congruence on ${\bf P}(W)$ we may be able
to claim that $\sigma(\phi)\sim\sigma^{*}(\phi)$. Indeed, this is of
course the case when $\sigma$ is valid for $\phi$:
\begin{prop}\label{logic:prop:FOPL:esssubstprop:validity}
Let $V,W$ be sets and $\sigma:\pv\to{\bf P}(W)$ be an essential
substitution. Then if $\sigma$ is valid for $\phi\in\pv$, we have
the substitution equivalence:
    \[
    \sigma(\phi)\sim\sigma^{*}(\phi)
    \]
where $\sigma^{*}:\pv\to{\bf P}(W)$ is the associated substitution
as per {\em definition~(\ref{logic:def:substitution})}.
\end{prop}
\begin{proof}
We assume that $\sigma$ is valid for $\phi$. We need to show that
$\sigma(\phi)\sim\sigma^{*}(\phi)$, that is ${\cal
M}\circ\sigma(\phi)={\cal M}\circ\sigma^{*}(\phi)$. However, having
assumed $\sigma$ is valid for $\phi$, from
theorem~(\ref{logic:the:FOPL:commute:mintransform:validsub}) of
page~\pageref{logic:the:FOPL:commute:mintransform:validsub} we have
${\cal M}\circ\sigma^{*}(\phi)=\bar{\sigma}\circ{\cal M}(\phi)$.
Since $\sigma$ is an essential substitution we also have ${\cal
M}\circ\sigma(\phi)=\bar{\sigma}\circ{\cal M}(\phi)$. So the result
follows.
\end{proof}

We have spent a lot of time proving the existence of essential
substitutions without providing natural examples. The most obvious
are the variable substitution mappings associated with injective
maps $\sigma:V\to W$ as per
definition~(\ref{logic:def:substitution}). A more interesting and
somewhat deeper result, is the fact that the minimal transform
mapping ${\cal M}:\pv\to\pvb$ is itself an essential substitution.

\begin{prop}\label{logic:prop:FOPL:esssubstprop:injective}
Let $V,W$ be sets and $\sigma:V\to W$ be an injective map. Then the
associated map $\sigma:\pv\to{\bf P}(W)$ is essential with
associated map $\sigma$ itself.
\end{prop}
\begin{proof}
Let $\sigma:V\to W$ be an injective map. Let $\sigma:\pv\to{\bf
P}(W)$ be the associated substitution mapping as per
definition~(\ref{logic:def:substitution}). The fact that both
mappings are called '$\sigma$' is standard practice at this stage
for us. We need to prove that $\sigma:\pv\to{\bf P}(W)$ is an
essential substitution mapping, with associated map $\sigma:V\to W$.
In other words, we need to prove that ${\cal
M}\circ\sigma(\phi)=\bar{\sigma}\circ{\cal M}(\phi)$ for all
$\phi\in\pv$. Using
theorem~(\ref{logic:the:FOPL:commute:mintransform:validsub}) of
page~\pageref{logic:the:FOPL:commute:mintransform:validsub} it is
sufficient to show that $\sigma$ is valid for $\phi$ which follows
from proposition~(\ref{logic:prop:FOPL:valid:injective}) and the
injectivity of $\sigma:V\to W$.
\end{proof}

The minimal transform ${\cal M}$ is an essential substitution with
associated map the inclusion $i:V\to\bar{V}$. In this case, we shall
break with our own tradition and refrain from using the same
notation for ${\cal M}$ and its associated map $i:V\to\bar{V}$.
\begin{prop}\label{logic:prop:FOPL:esssubstprop:mintransform}
Let $V$ be a set. The minimal transform ${\cal M}:\pv\to\pvb$ is an
essential substitution with associated map the inclusion
$i:V\to\bar{V}$.
\end{prop}
\begin{proof}
Let $\bar{\cal M}:\pvb\to{\bf P}(\bar{\bar{V}})$ denote the minimal
transform on \pvb. We need to show that $\bar{\cal M}\circ{\cal
M}(\phi)=\bar{i}\circ{\cal M}(\phi)$ for all $\phi\in\pv$. However,
since $i$ is injective, from
proposition~(\ref{logic:prop:FOPL:valid:injective}) it is valid for
$\phi$ and it follows from
theorem~(\ref{logic:the:FOPL:commute:mintransform:validsub}) of
page~\pageref{logic:the:FOPL:commute:mintransform:validsub} that
$\bar{\cal M}\circ i(\phi)=\bar{i}\circ{\cal M}(\phi)$. Hence we
need to show that $\bar{\cal M}\circ{\cal M}(\phi)=\bar{\cal M}\circ
i(\phi)$. Using
theorem~(\ref{logic:the:FOPL:mintransfsubcong:kernel}) of
page~\pageref{logic:the:FOPL:mintransfsubcong:kernel} we need to
show ${\cal M}(\phi)\sim i(\phi)$ where $\sim$ is the substitution
congruence on \pvb, which we know is true from
proposition~(\ref{logic:prop:FOPL:mintransform:eqivalence}).
\end{proof}

Given $\phi\in\pv$ and a map $\sigma:V\to V$ which is admissible for
$\phi$, we know from
proposition~(\ref{logic:prop:admissible:sub:congruence}) that
$\sigma(\phi)\sim\phi$ where $\sim$ denotes the substitution
congruence on \pv. In other words, if $\sigma$ is valid for $\phi$
and $\sigma(u)=u$ for all $u\in\free(\phi)$ then we have
$\phi\sim\psi$ where $\psi=\sigma(\phi)$. The following proposition
is an extension of this result for essential substitutions. In fact,
the converse is now also true. If $\phi\sim\psi$ then we must have
$\psi=\sigma(\phi)$ for some essential substitution $\sigma$ such
that $\sigma(u)=u$ for all $u\in\free(\phi)$. This is less deep than
it looks:

\begin{prop}\label{logic:prop:FOPL:esssubstprop:subcong}
Let $V$ be a set and $\sim$ be the substitution congruence on \pv.
Then for all $\phi,\psi\in\pv$ we have $\phi\sim\psi$ \ifand\
$\psi=\sigma(\phi)$ for some essential substitution
$\sigma:\pv\to\pv$ such that $\sigma(u)=u$ for all
$u\in\free(\phi)$.
\end{prop}
\begin{proof}
First we show the 'if' part: so suppose $\psi=\sigma(\phi)$ for some
essential substitution $\sigma:\pv\to\pv$ such that $\sigma(u)=u$
for all $u\in\free(\phi)$. We need to show that $\phi\sim\psi$. From
theorem~(\ref{logic:the:FOPL:mintransfsubcong:kernel}) of
page~\pageref{logic:the:FOPL:mintransfsubcong:kernel} it is
sufficient to prove that ${\cal M}(\phi)={\cal M}(\psi)$. Having
assumed that $\psi=\sigma(\phi)$ we need to show that ${\cal
M}(\phi)={\cal M}\circ\sigma(\phi)$. However, since $\sigma$ is
essential we have ${\cal M}\circ\sigma(\phi)=\bar{\sigma}\circ{\cal
M}(\phi)$. Hence we need to show that ${\cal
M}(\phi)=\bar{\sigma}\circ{\cal M}(\phi)$. Using
proposition~(\ref{logic:prop:substitution:support}) it is sufficient
to prove that $\bar{\sigma}(u)=u$ for all $u\in\var({\cal
M}(\phi))$. So let $u\in\var({\cal M}(\phi))$. Since
$\bar{V}=V\uplus\N$ we shall distinguish two cases: first we assume
that $u\in\N$. Then $\bar{\sigma}(u)=u$ is clear from
definition~(\ref{logic:def:FOPL:commute:minextensioon:map}). Next we
assume that $u\in V$. Then from
proposition~(\ref{logic:prop:FOPL:mintransform:variables}) we obtain
$u\in\free(\phi)$ and it follows that $\bar{\sigma}(u)=\sigma(u)=u$.
We now prove the 'only if' part: so suppose $\phi\sim\psi$. We need
to show that $\psi=\sigma(\phi)$ for some essential substitution
$\sigma:\pv\to\pv$ such that $\sigma(u)=u$ for all
$u\in\free(\phi)$. Let $i:\pv\to\pv$ be the identity mapping. From
proposition~(\ref{logic:prop:FOPL:esssubstprop:injective}), $i$ is
an essential substitution associated with the identity $i:V\to V$.
Let $\sigma:\pv\to\pv$ be defined by $\sigma(\chi)=i(\chi)$ whenever
$\chi\neq\phi$ and $\sigma(\phi)=\psi$. Having assumed that
$\phi\sim\psi$ we have $\sigma(\chi)\sim i(\chi)$ for all
$\chi\in\pv$. It follows from
proposition~(\ref{logic:prop:FOPL:esssubstprop:redefine}) that
$\sigma$ is an essential substitution whose associated map
$\sigma:V\to V$ is the identity. In particular, we have
$\sigma(u)=u$ for all $u\in\free(\phi)$.
\end{proof}

The composition of two essential substitutions is itself essential,
while the associated map is the obvious composition of the
respective associated maps.
\begin{prop}\label{logic:prop:FOPL:esssubstprop:composition}
Let $U$, $V$ and $W$ be sets while the maps $\tau:{\bf P}(U)\to\pv$
and $\sigma:\pv\to{\bf P}(W)$ are essential substitutions. Then
$\sigma\circ\tau:{\bf P}(U)\to{\bf P}(W)$ is itself an essential
substitution with associated map $\sigma\circ\tau:U\to W$.
\end{prop}
\begin{proof}
We need to show that ${\cal
M}\circ(\sigma\circ\tau)=\overline{(\sigma\circ\tau)}\circ{\cal M}$.
However from
definition~(\ref{logic:def:FOPL:commute:minextensioon:map}), it is
clear that the minimal extension
$\overline{(\sigma\circ\tau)}:\bar{U}\to\bar{W}$ is equal to the
composition of the minimal extensions $\bar{\sigma}\circ\bar{\tau}$.
Hence we have:
    \begin{eqnarray*}
    {\cal M}\circ(\sigma\circ\tau)
    &=&\bar{\sigma}\circ{\cal
    M}\circ\tau\\
    &=&\bar{\sigma}\circ\bar{\tau}\circ{\cal
    M}\\
    &=&\overline{(\sigma\circ\tau)}\circ{\cal
    M}\\
    \end{eqnarray*}
\end{proof}

Suppose $\sigma:V\to W$ is a map which is valid for $\phi\in\pv$.
From proposition~(\ref{logic:prop:FOPL:valid:free:commute}) we know
that $\free(\sigma(\phi))=\sigma(\free(\phi))$. In fact this
equality holds for every sub-formula $\psi\preceq\phi$. The
following proposition shows that the equality
$\free(\sigma(\phi))=\sigma(\free(\phi))$ holds whenever
$\sigma:\pv\to{\bf P}(W)$ is an essential substitution. As already
pointed out, this is {\em as if} $\sigma$ was valid for all
$\phi\in\pv$. But of course, nothing about sub-formulas in this
case. Consider the formula $\phi=\forall y(x\in y)$ where
$V=\{x,y\}$ and $x\neq y$. Let $\sigma:\pv\to\pv$ be an essential
substitution associated with $\sigma=[y/x]$. Then $\sigma:V\to V$ is
clearly not valid for $\phi$ and the equality
$\free(\sigma(\phi))=\sigma(\free(\phi))$ may look very surprising.
Indeed, we have $\free(\phi)=\{x\}$ and consequently
$\sigma(\free(\phi))=\{y\}$. It is very tempting to argue that
$\sigma(\phi)=\forall y(y\in y)$ has no free variable and that the
equality therefore cannot be true. But remember that
$\sigma:\pv\to\pv$ does not refer to the substitution mapping
associated with $\sigma:V\to V$. The context of our discussion makes
it very clear that we are starting with $\sigma:\pv\to\pv$ as a
given essential substitution. Hence $\sigma(\phi)$ is not equal to
$\forall y(y\in y)$. In fact, we must have ${\cal
M}\circ\sigma(\phi)=\forall\,0(y\in 0)$ and since $V=\{x,y\}$ it is
not difficult to prove that the only possible value is
$\sigma(\phi)=\forall x(y\in x)$, and $\free(\sigma(\phi))=\{y\}$.

\begin{prop}\label{logic:prop:FOPL:esssubstprop:free:commute}
Let $V,W$ be sets and $\sigma:\pv\to{\bf P}(W)$ be an essential
substitution. Then for all $\phi\in\pv$ we have the equality:
    \[
    \free(\sigma(\phi))=\sigma(\free(\phi))
    \]
\end{prop}
\begin{proof}
Using proposition~(\ref{logic:prop:FOPL:mintransform:variables}) we
obtain the following:
    \begin{eqnarray*}
    \free(\sigma(\phi))&=&\free(\,{\cal M}\circ\sigma(\phi)\,)\\
    &=&\free(\,\bar{\sigma}\circ{\cal M}(\phi)\,)\\
    \mbox{$\bar{\sigma}$ valid for ${\cal M}(\phi)$}\ \rightarrow
    &=&\bar{\sigma}(\,\free({\cal M}(\phi))\,)\\
    \mbox{prop.~(\ref{logic:prop:FOPL:mintransform:variables})}\ \rightarrow
    &=&\bar{\sigma}(\free(\phi))\\
    \free(\phi)\subseteq V\ \rightarrow&=&\sigma(\free(\phi))
    \end{eqnarray*}
\end{proof}

If $\sigma,\tau:V\to W$ are two maps and $\phi\in\pv$, we know from
proposition~(\ref{logic:prop:substitution:support}) that
$\sigma(\phi)=\tau(\phi)$ \ifand\ $\sigma$ and $\tau$ coincide on
$\var(\phi)$. The following proposition provides the counterpart
result for essential substitutions. So we start from
$\sigma,\tau:\pv\to{\bf P}(W)$ two essential substitutions and we
consider the equivalence $\sigma(\phi)\sim\tau(\phi)$ where $\sim$
is the substitution congruence. Note that there is not much point
considering the equality $\sigma(\phi)=\tau(\phi)$ in the case of
essential substitutions, since we know from
proposition~(\ref{logic:prop:FOPL:esssubstprop:redefine}) that both
$\sigma$ and $\tau$ can be arbitrarily redefined modulo
substitution, without affecting their associated maps
$\sigma,\tau:V\to W$. So we can only focus on
$\sigma(\phi)\sim\tau(\phi)$ which holds \ifand\ the associated maps
$\sigma,\tau:V\to W$ coincide on $\free(\phi)$\,:

\begin{prop}\label{logic:prop:FOPL:esssubstprop:support}
Let $V,W$ be sets and $\sigma,\tau:\pv\to{\bf P}(W)$ be two
essential substitutions. Then for all $\phi\in\pv$ we have the
equivalence:
    \[
    \sigma_{|\free(\phi)}=\tau_{|\free(\phi)}\ \ \Leftrightarrow\
    \ \sigma(\phi)\sim\tau(\phi)
    \]
where $\sim$ denotes the substitution congruence on the algebra
${\bf P}(W)$.
\end{prop}
\begin{proof}
From theorem~(\ref{logic:the:FOPL:mintransfsubcong:kernel}) of
page~\pageref{logic:the:FOPL:mintransfsubcong:kernel},
$\sigma(\phi)\sim\tau(\phi)$ is equivalent to the equality ${\cal
M}\circ\sigma(\phi)={\cal M}\circ\tau(\phi)$. Having assumed
$\sigma$ and $\tau$ are essential, this is in turn equivalent to
$\bar{\sigma}\circ{\cal M}(\phi)=\bar{\tau}\circ{\cal M}(\phi)$.
Using proposition~(\ref{logic:prop:substitution:support}), this last
equality is equivalent to $\bar{\sigma}(u)=\bar{\tau}(u)$ for all
$u\in\var({\cal M}(\phi))$. Hence we need to show that this last
statement is equivalent to $\sigma(u)=\tau(u)$ for all
$u\in\free(\phi)$. First we show $\Rightarrow$\,: so suppose
$\bar{\sigma}(u)=\bar{\tau}(u)$ for all $u\in\var({\cal M}(\phi))$
and let $u\in\free(\phi)$. We need to show that $\sigma(u)=\tau(u)$.
From proposition~(\ref{logic:prop:FOPL:mintransform:variables}) we
have $\var({\cal M}(\phi))\cap V=\free(\phi)$. It follows that
$u\in\var({\cal M}(\phi))\cap V$ and consequently we have
$\sigma(u)=\bar{\sigma}(u)=\bar{\tau}(u)=\tau(u)$ as requested. So
we now prove $\Leftarrow$\,: So we assume that $\sigma(u)=\tau(u)$
for all $u\in\free(\phi)$ and consider $u\in\var({\cal M}(\phi))$.
We need to show that $\bar{\sigma}(u)=\bar{\tau}(u)$. Since
$\bar{V}=V\uplus\N$ we shall distinguish two cases: first we assume
that $u\in\N$. Then $\bar{\sigma}(u)=u=\bar{\tau}(u)$. Next we
assume that $u\in V$. Then $u\in \var({\cal M}(\phi))\cap
V=\free(\phi)$ and
$\bar{\sigma}(u)=\sigma(u)=\tau(u)=\bar{\tau}(u)$.
\end{proof}

In theorem~(\ref{logic:the:FOPL:mintransfsubcong:valid}) of
page~\pageref{logic:the:FOPL:mintransfsubcong:valid} we proved that
$\sigma(\phi)\sim\sigma(\psi)$ whenever $\phi\sim\psi$ and
$\sigma:V\to W$ is a substitution which is valid for both $\phi$ and
$\psi$. We now provide an extension of this result for an essential
substitution $\sigma:\pv\to{\bf P}(W)$. We should not confuse the
following proposition with
proposition~(\ref{logic:prop:FOPL:esssubstprop:subcong}) relative to
essential substitutions $\sigma:\pv\to\pv$. If we want to claim that
$\sigma(\phi)\sim\phi$ then
proposition~(\ref{logic:prop:FOPL:esssubstprop:subcong}) requires
that $\sigma(u)=u$ for all $u\in\free(\phi)$. This is not the same
thing as claiming $\sigma(\phi)\sim\sigma(\psi)$ for which no
special conditions on the set $\free(\phi)=\free(\psi)$ is attached
(recall that $\free(\phi)=\free(\psi)$ whenever $\phi\sim\psi$). If
you have $\phi\sim\psi$, then you can move the free variables around
as much as you want with an essential substitution. It will not
change the fact that $\sigma(\phi)\sim\sigma(\psi)$.

\begin{prop}\label{logic:prop:FOPL:esssubstprop:equivalence}
Let $V,W$ be sets and $\sigma:\pv\to {\bf P}(W)$ be an essential
substitution. Then for all formulas $\phi,\psi\in\pv$ we have the
implication:
    \[
    \phi\sim\psi\ \Rightarrow\ \sigma(\phi)\sim\sigma(\psi)
    \]
where~$\sim$ denotes the substitution congruence on \pv\ and ${\bf
P}(W)$.
\end{prop}
\begin{proof}
So we assume that $\phi\sim\psi$. We need to show that
$\sigma(\phi)\sim\sigma(\psi)$. Using
theorem~(\ref{logic:the:FOPL:mintransfsubcong:kernel}) of
page~\pageref{logic:the:FOPL:mintransfsubcong:kernel} it is
sufficient to show that ${\cal M}\circ\sigma(\phi)={\cal
M}\circ\sigma(\psi)$. Having assumed that $\sigma$ is essential, it
is sufficient to show that $\bar{\sigma}\circ{\cal
M}(\phi)=\bar{\sigma}\circ{\cal M}(\psi)$ which follows immediately
from ${\cal M}(\phi)={\cal M}(\psi)$, itself a consequence of
$\phi\sim\psi$.
\end{proof}

In proposition~(\ref{logic:prop:FOPL:substrank:invariant:rank}) we
showed that a substitution $\sigma:V\to W$ which is valid for a
formula $\phi\in\pv$ will preserve its substitution rank, provided
$\sigma_{|\free(\phi)}$ is an injective map. The same is true of an
essential substitution $\sigma:\pv\to{\bf P}(W)$.
\begin{prop}\label{logic:prop:FOPL:esssubstprop:rank:injective}
Let $V,W$ be sets and $\sigma:\pv\to{\bf P}(W)$ be an essential
substitution. Let $\phi\in\pv$ such that $\sigma_{|\free(\phi)}$ is
an injective map. Then:
    \[
    \rnk(\sigma(\phi))=\rnk(\phi)
    \]
where $\rnk(\,\,)$ refers to the substitution rank as per {\em
definition~(\ref{logic:def:FOPL:substrank:substrank})}.
\end{prop}
\begin{proof}
Using proposition~(\ref{logic:prop:FOPL:substrank:minrank}) we
obtain the following:
    \begin{eqnarray*}
    \rnk(\sigma(\phi))&=&\rnk({\cal M}\circ\sigma(\phi))\\
    \mbox{$\sigma$ essential}\ \rightarrow
    &=&\rnk(\bar{\sigma}\circ{\cal M}(\phi))\\
    \mbox{A: to be proved}\ \rightarrow
    &=&\rnk({\cal M}(\phi))\\
    \mbox{prop.~(\ref{logic:prop:FOPL:substrank:minrank})}\ \rightarrow
    &=&\rnk(\phi)
    \end{eqnarray*}
So it remains to show that $\rnk(\bar{\sigma}\circ{\cal
M}(\phi))=\rnk({\cal M}(\phi))$. Using
proposition~(\ref{logic:prop:FOPL:substrank:invariant:rank}) it is
sufficient to show that $\bar{\sigma}$ is valid for ${\cal M}(\phi)$
and furthermore that it is injective on $\free({\cal M}(\phi))$. We
know that $\bar{\sigma}$ is valid for ${\cal M}(\phi)$ from
proposition~(\ref{logic:def:FOPL:commute:minextension:valid}). We
also know that $\free({\cal M}(\phi))=\free(\phi)$ from
proposition~(\ref{logic:prop:FOPL:mintransform:variables}). So it
remains to show that $\bar{\sigma}$ is injective on
$\free(\phi)\subseteq V$ which is clearly the case since
$\bar{\sigma}$ coincide with $\sigma$ on $V$ and $\sigma$ is by
assumption injective on $\free(\phi)$.
\end{proof}

Without the injectivity of $\sigma$ on $\free(\phi)$, the
substitution rank is generally not preserved. However as can be seen
below, it can never increase. The following proposition extends
proposition~(\ref{logic:prop:FOPL:subst:rank:substitution}) to
essential substitutions.
\begin{prop}\label{logic:prop:FOPL:esssubstprop:rank:less}
Let $V,W$ be sets and $\sigma:\pv\to{\bf P}(W)$ be an essential
substitution. Then for all formula $\phi\in\pv$ we have the
inequality:
    \[
    \rnk(\sigma(\phi))\leq\rnk(\phi)
    \]
where $\rnk(\,\,)$ refers to the substitution rank as per {\em
definition~(\ref{logic:def:FOPL:substrank:substrank})}.
\end{prop}
\begin{proof}
Using proposition~(\ref{logic:prop:FOPL:substrank:minrank}) we
obtain the following:
    \begin{eqnarray*}
    \rnk(\sigma(\phi))&=&\rnk({\cal M}\circ\sigma(\phi))\\
    \mbox{$\sigma$ essential}\ \rightarrow
    &=&\rnk(\bar{\sigma}\circ{\cal M}(\phi))\\
    \mbox{prop.~(\ref{logic:prop:FOPL:subst:rank:substitution})}\ \rightarrow
    &\leq&\rnk({\cal M}(\phi))\\
    \mbox{prop.~(\ref{logic:prop:FOPL:substrank:minrank})}\ \rightarrow
    &=&\rnk(\phi)
    \end{eqnarray*}
\end{proof}

Given an essential substitution $\sigma:\pv\to{\bf P}(W)$ and
$\phi\in\pv$, there is only so much we can say about $\sigma(\phi)$,
even if we know about the structure of $\phi$. For example, if
$\phi=\phi_{1}\to\phi_{2}$ we cannot claim that
$\sigma(\phi)=\sigma(\phi_{1})\to\sigma(\phi_{2})$. However, as the
following proposition will show, we have
$\sigma(\phi)\sim\sigma(\phi_{1})\to\sigma(\phi_{2})$ where $\sim$
is the substitution congruence on ${\bf P}(W)$. A more interesting
example is the case when $\phi=\forall x\phi_{1}$. We would like to
claim that $\sigma(\phi)\sim\forall\sigma(x)\sigma(\phi_{1})$.
However, there is little chance of this equivalence being true when
$\sigma(x)$ happens to be a free variable of $\sigma(\phi)$. So the
condition $\sigma(x)\not\in\free(\sigma(\phi))$ should be a minimal
requirement. Remember that the map $\sigma:\pv\to{\bf P}(W)$ is an
{\em essential} substitution. It is not simply the substitution
mapping associated with a map $\sigma:V\to W$ as per
definition~(\ref{logic:def:substitution}). If this was the case then
the equality $\sigma(\phi)=\forall\sigma(x)\sigma(\phi_{1})$ would
hold, and the condition $\sigma(x)\not\in\free(\sigma(\phi))$ would
be automatically satisfied. When $\sigma:\pv\to{\bf P}(W)$ is an
essential substitution, it is very well possible that
$\sigma(x)\in\free(\sigma(\phi))$. Let $\phi=\forall x(x\in y)$ with
$x\neq y$ and $\sigma:\pv\to\pv$ an essential substitution
associated with $[y/x]$. Then $\sigma(x)=y$ while
$\sigma(\phi)=\forall z(z\in y)$ for some $z\neq y$. So
$\sigma(x)\in\free(\sigma(\phi))$. The following proposition will
show that the condition $\sigma(x)\not\in\free(\sigma(\phi))$ is in
fact sufficient to obtain the equivalence
$\sigma(\phi)\sim\forall\sigma(x)\sigma(\phi_{1})$.

\begin{prop}\label{logic:prop:FOPL:esssubstprop:charac}
Let $V,W$ be sets and $\sigma:\pv\to{\bf P}(W)$ be an essential
substitution. Let $\sim$ be the substitution congruence on ${\bf
P}(W)$. Then we have:
    \[
    \forall\phi\in\pv\ ,\ \sigma(\phi):\left\{
                    \begin{array}{lcl}
                    =(\sigma(x)\in \sigma(y))&\mbox{\ if\ }&\phi=(x\in y)\\
                    =\bot&\mbox{\ if\ }&\phi=\bot\\
                    \sim\sigma(\phi_{1})\to\sigma(\phi_{2})&\mbox{\ if\ }&
                    \phi=\phi_{1}\to\phi_{2}\\
                    \sim\forall\sigma(x)\sigma(\phi_{1})
                    &\mbox{\ if\ }&\phi=\forall x\phi_{1}\ ,\
                    \sigma(x)\not\in\free(\sigma(\phi))\\
                    \end{array}\right.
    \]
\end{prop}
\begin{proof}
First we assume that $\phi=(x\in y)$ for some $x,y\in V$. We need to
show that $\sigma(\phi)=\sigma(x)\in\sigma(y)$. However, any map
$\sigma:V\to W$ is valid for $\phi$. It follows from
proposition~(\ref{logic:prop:FOPL:esssubstprop:validity}) that
$\sigma(\phi)\sim\sigma^{*}(\phi)$, where $\sigma^{*}:\pv\to{\bf
P}(W)$ is the associated substitution mapping as per
definition~(\ref{logic:def:substitution}). From
$\sigma^{*}(\phi)=\sigma(x)\in\sigma(y)$ and the equivalence
$\sigma(\phi)\sim\sigma^{*}(\phi)$, using
theorem~(\ref{logic:the:sub:congruence:charac}) of
page~\pageref{logic:the:sub:congruence:charac} we conclude that
$\sigma(\phi)=\sigma(x)\in\sigma(y)$ as requested. Next we assume
that $\phi=\bot$. Then once again $\sigma$ is valid for $\phi$ and
we obtain $\sigma(\phi)\sim\sigma^{*}(\phi)=\bot$ and consequently
from theorem~(\ref{logic:the:sub:congruence:charac}) we have
$\sigma(\phi)=\bot$. So we now assume that
$\phi=\phi_{1}\to\phi_{2}$ for some $\phi_{1},\phi_{2}\in\pv$. We
need to show $\sigma(\phi)\sim\sigma(\phi_{1})\to\sigma(\phi_{2})$.
Using theorem~(\ref{logic:the:FOPL:mintransfsubcong:kernel}) of
page~\pageref{logic:the:FOPL:mintransfsubcong:kernel}, we simply
compute the minimal transforms:
    \begin{eqnarray*}
    {\cal M}\circ\sigma(\phi)&=&\bar{\sigma}\circ{\cal M}(\phi)\\
    &=&\bar{\sigma}\circ{\cal M}(\phi_{1}\to\phi_{2})\\
    &=&\bar{\sigma}(\,{\cal M}(\phi_{1})\to{\cal M}(\phi_{2})\,)\\
    &=&\bar{\sigma}\circ{\cal M}(\phi_{1})\to\bar{\sigma}\circ{\cal M}(\phi_{2})\\
    &=&{\cal M}\circ\sigma(\phi_{1})\to{\cal M}\circ\sigma(\phi_{2})\\
    &=&{\cal M}(\,\sigma(\phi_{1})\to\sigma(\phi_{2})\,)\\
    \end{eqnarray*}
So we now assume that $\phi=\forall x\phi_{1}$ and
$\sigma(x)\not\in\free(\sigma(\phi))$. We need to show that
$\sigma(\phi)\sim\forall\sigma(x)\sigma(\phi_{1})$. Likewise, we
shall compute minimal transforms:
    \begin{eqnarray*}
    {\cal M}\circ\sigma(\phi)&=&\bar{\sigma}\circ{\cal M}(\phi)\\
    &=&\bar{\sigma}\circ{\cal M}(\forall x\phi_{1})\\
    \mbox{$n=\min\{k:[k/x]\mbox{ valid for }{\cal M}(\phi_{1})\}$}\
    \rightarrow
    &=&\bar{\sigma}(\,\forall n{\cal M}(\phi_{1})[n/x]\,)\\
    &=&\forall \bar{\sigma}(n)\bar{\sigma}(\,{\cal
    M}(\phi_{1})[n/x]\,)\\
    &=&\forall n\,\bar{\sigma}\circ [n/x]\circ{\cal
    M}(\phi_{1})\\
    \mbox{A: to be proved}\ \rightarrow
    &=&\forall n\,[n/\sigma(x)]\circ\bar{\sigma}\circ{\cal M}(\phi_{1})\\
    &=&\forall n\,[n/\sigma(x)]\circ{\cal M}\circ\sigma(\phi_{1})\\
    &=&\forall n\,{\cal M}[\sigma(\phi_{1})][n/\sigma(x)]\\
    \mbox{B: to be proved}\ \rightarrow
    &=&\forall m\,{\cal M}[\sigma(\phi_{1})][m/\sigma(x)]\\
    \mbox{$m=\min\{k:[k/\sigma(x)]\mbox{ valid for }{\cal M}[\sigma(\phi_{1})]\}$}\
    \rightarrow
    &=&{\cal M}(\forall\sigma(x)\sigma(\phi_{1}))\\
    \end{eqnarray*}
So it remains to justify point A and B. First we deal with point A:
it is sufficient to prove the equality
$\bar{\sigma}\circ[n/x]\circ{\cal M}(\phi_{1})=
[n/\sigma(x)]\circ\bar{\sigma}\circ{\cal M}(\phi_{1})$, which
follows from lemma~(\ref{logic:lemma:FOPL:commute:mphi1}) and
$\sigma(x)\not\in\sigma(\free(\phi))$, itself a consequence of
$\sigma(x)\not\in\free(\sigma(\phi))$ and
proposition~(\ref{logic:prop:FOPL:esssubstprop:free:commute}). We
now deal with point B: it is sufficient to prove the equivalence
$\mbox{$[k/x]$ valid for ${\cal M}(\phi_{1})$}\ \Leftrightarrow\
\mbox{$[k/\sigma(x)]$ valid for $\bar{\sigma}\circ{\cal
M}(\phi_{1})$}$ which follows from
lemma~(\ref{logic:lemma:FOPL:commute:n:equivalence}) and the fact
that $\sigma(x)\not\in\sigma(\free(\phi))$.
\end{proof}


This last proposition seems to indicate there is nothing we can say
about $\sigma(\phi)$ if it happens to have $\sigma(x)$ as a free
variable. Luckily this is not the case:
\begin{prop}\label{logic:prop:FOPL:esssubstprop:tau}
Let $V,W$ be sets and $\sigma:\pv\to{\bf P}(W)$ be an essential
substitution. Let $\phi=\forall x\phi_{1}$ where $x\in V$,
$\phi_{1}\in\pv$. There exists an essential substitution
$\tau:\pv\to{\bf P}(W)$ such that $\tau=\sigma$ on $V\setminus\{x\}$
and $\tau(x)\not\in\free(\sigma(\phi))$. Furthermore, for any such
$\tau$ we have the substitution equivalence:
    \[
    \sigma(\phi)\sim\forall\tau(x)\tau(\phi_{1})
    \]
\end{prop}
\begin{proof}
We shall first prove the substitution equivalence. So let
$\tau:\pv\to{\bf P}(W)$ be an essential substitution which coincide
with $\sigma$ on $V\setminus\{x\}$ and such that
$\tau(x)\not\in\free(\sigma(\phi))$. We need to show that
$\sigma(\phi)\sim\forall\tau(x)\tau(\phi_{1})$ where $\sim$ denotes
the substitution congruence on ${\bf P}(W)$. However, since
$\free(\phi)\subseteq V\setminus\{x\}$ we see that $\sigma$ and
$\tau$ are essential substitutions which coincide on $\free(\phi)$.
It follows from
proposition~(\ref{logic:prop:FOPL:esssubstprop:support}) that
$\sigma(\phi)\sim\tau(\phi)$. Hence it is sufficient to prove that
$\tau(\phi)\sim\forall\tau(x)\tau(\phi_{1})$. Applying
proposition~(\ref{logic:prop:FOPL:esssubstprop:charac}) it is
sufficient to show that $\tau(x)\not\in\free(\tau(\phi))$. However,
by assumption we have $\tau(x)\not\in\free(\sigma(\phi))$ and so:
    \[
    \tau(x)\not\in\free(\sigma(\phi))=\sigma(\free(\phi))=\tau(\free(\phi))=\free(\tau(\phi))
    \]
where we have used
proposition~(\ref{logic:prop:FOPL:esssubstprop:free:commute}). It
remains to show that such an essential substitution $\tau:\pv\to{\bf
P}(W)$ exists. Suppose we have proved that $\free(\sigma(\phi))$ is
a proper subset of $W$. Then there exists $y^{*}\in W$ such that
$y^{*}\not\in\free(\sigma(\phi))$. Consider the map $\tau:V\to W$
defined by:
    \[
    \forall u\in V\ ,\ \tau(u)=\left\{
        \begin{array}{lcl}
        \sigma(u)&\mbox{\ if\ }&u\in V\setminus\{x\}\\
        y^{*}&\mbox{\ if\ }&u=x
        \end{array}
    \right.
    \]
Then it is clear that $\tau$ coincides with $\sigma$ on
$V\setminus\{x\}$ and $\tau(x)\not\in\free(\sigma(\phi))$. In order
to show the existence of $\tau:\pv\to{\bf P}(W)$ it is sufficient to
show the existence of an essential substitution associated with
$\tau:V\to W$. From
theorem~(\ref{logic:the:FOPL:esssubst:existence}) of
page~\pageref{logic:the:FOPL:esssubst:existence} it is sufficient to
show that $|W|$ is an infinite cardinal, or that it is finite with
$|V|\leq|W|$. However, this follows immediately from the existence
of the essential substitution $\sigma:\pv\to{\bf P}(W)$ and
theorem~(\ref{logic:the:FOPL:esssubst:existence}). So it remains to
show that $\free(\sigma(\phi))$ is a proper subset of $W$. This is
clearly true if $|W|$ is an infinite cardinal. So we may assume that
$|W|$ if finite, in which case we have $|V|\leq |W|$. In particular,
$|V|$ is also a finite cardinal and since $x\not\in\free(\phi)$ we
have $|\free(\phi)|<|V|$. Hence we have the following inequalities:
    \[
    |\free(\sigma(\phi))|=|\sigma(\free(\phi))|\leq|\free(\phi)|<|V|\leq|W|
    \]
So $\free(\sigma(\phi))$ is indeed a proper subset of $W$, as
requested.
\end{proof}
