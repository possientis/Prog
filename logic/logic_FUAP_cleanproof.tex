In the previous section, we introduced a new semantics on our free
universal algebra of proofs \pvs. The new valuation
$\vals:\pvs\to\pv$ was designed to be more flexible and specifically
to cope with the substitution congruence, commonly known as
$\alpha$-equivalence. We made sure that $\vals$ would be an
extension of $\val:\pvs\to\pv$ from the domain of totally clean
proofs to a larger domain on which we shall now focus. It would be
wrong to think that the new flexibility embedded in $\vals$ allows
it it to return a meaningful conclusion $\vals(\pi)$ for every
$\pi\in\pvs$. Some elements of \pvs\ are regarded as {\em flawed},
as they contain steps which are not legitimate. For example, the
proof $\pi=\axi\phi$ is not a legitimate axiom invocation unless
$\phi$ is an axiom modulo, namely $\phi\in\avs$ as per
definition~(\ref{logic:def:FUAP:valuationmod:axiom:modulo}). The
proof $\pi=\pi_{1}\pon\pi_{2}$ is not a legitimate use of the modus
ponens rule of inference unless the conclusion modulo of $\pi_{2}$
is of the form $\vals(\pi_{2})=\psi_{1}\to\psi_{2}$ for some
$\psi_{1},\psi_{2}\in\pv$ where $\psi_{1}$ is substitution
equivalent to the conclusion modulo of $\pi_{1}$. The proof
$\pi=\gen x\pi_{1}$ is not a legitimate use of the generalization
rule of inference unless the variable $x$ is truly arbitrary, i.e.
is not a specific variable of the proof $\pi_{1}$. In this section,
we intend to study those proof $\pi\in\pvs$ which do not contain any
flawed steps in relation to the valuation modulo $\vals$. These
proofs constitute the true domain of $\vals$ and will be called {\em
clean proofs}. Although $\vals:\pvs\to\pv$ is a total function and
$\vals(\pi)$ is provable for all $\pi\in\pvs$, the clean proofs play
an important role for us as their behavior is predictable in
relation to $\vals$, just like totally clean proofs were seen to be
predictable in relation to $\val$. In particular, we shall see that
the crucial equality $\vals\circ\sigma(\pi)=\sigma\circ\vals(\pi)$
does hold for clean proofs in
proposition~(\ref{logic:prop:FUAP:strongvalsubalmostclean:valuation:commute}),
provided $\sigma$ is valid for $\pi$. We shall also prove the
$\alpha$-equivalence $\vals\circ{\cal M}(\pi)\sim{\cal
M}\circ\vals(\pi)$ in
proposition~(\ref{logic:prop:FUAP:mintransproof:valuation:commute}),
which will vindicate our chosen definition of minimal transform for
proofs and our design of $\vals$. In
definition~(\ref{logic:def:FUAP:clean:clean:proof}) we defined
totally clean proofs in terms of a {\em strength mapping} $s:\pvs\to
2$ and showed in
proposition~(\ref{logic:prop:FUAP:clean:characterization}) that
totally clean proofs were simply proofs without flawed steps. In
this section, we shall define clean proofs directly in the spirit of
proposition~(\ref{logic:prop:FUAP:clean:characterization}). A clean
proof is simply a proof without {\em flawed steps}, after {\em
flawlessness} has been redefined. Note that the mathematical
statement $(ii)$ of the following definition is a notational
shortcut, where '{\em there exists $\psi_{1},\psi_{2}\in\pv$ such
that\ldots}' has been omitted:

\index{clean@Clean proof}
\begin{defin}\label{logic:def:FUAP:almostclean:definition}
Let $V$ be a set and $\sim$ be the substitution congruence. A proof
$\pi\in\pvs$ is {\em clean} \ifand\ for all
$\pi_{1},\pi_{2}\in\pvs$, $\phi\in\pv$ and $x\in V$\,:
    \begin{eqnarray*}
    (i)&&\axi\phi\preceq\pi\ \Rightarrow\ \phi\in\avs\\
    (ii)&&\pi_{1}\pon\pi_{2}\preceq\pi\ \Rightarrow\
    \vals(\pi_{2})=\psi_{1}\to\psi_{2}\mbox{ where }\psi_{1}\sim\vals(\pi_{1})\\
    (iii)&&\gen x\pi_{1}\preceq\pi\ \Rightarrow\
    x\not\in\spec(\pi_{1})
    \end{eqnarray*}
\end{defin}

We claimed that $\vals$ was an extension of $\val$ and indeed we
proved in
proposition~(\ref{logic:prop:FUAP:valuationmod:clean:proof}) that
$\vals(\pi)=\val(\pi)$ whenever $\pi$ is totally clean. Strictly
speaking, in order for $\vals$ to qualify as an {\em extension} of
$\val$, it should have a wider domain than $\val$. A totally clean
proof should also be clean:
\begin{prop}\label{logic:def:FUAP:almostclean:clean}
Let $V$ be a set. If $\pi\in\pvs$ is totally clean, then it is
clean.
\end{prop}
\begin{proof}
We assume that $\pi$ is totally clean. We need to show that $\pi$ is
clean, namely that $(i)$, $(ii)$ and $(iii)$ of
definition~(\ref{logic:def:FUAP:almostclean:definition}) hold. First
we show $(i)$\,: so we assume that $\axi\phi$ is a sub-proof of
$\pi$. We need to show that $\phi$ is an axiom modulo. However, from
proposition~(\ref{logic:prop:FUAP:clean:characterization}) we have
$\phi\in\av$, i.e. $\phi$ is an axiom. In particular, from
definition~(\ref{logic:def:FUAP:valuationmod:axiom:modulo}) we
obtain $\phi\in\avs$ as requested. Next we show $(ii)$\,: so we
assume that $\pi_{1}\pon\pi_{2}$ is a sub-proof of $\pi$. We need to
show that $\vals(\pi_{2})=\psi_{1}\to\psi_{2}$ for some
$\psi_{1},\psi_{2}\in\pv$ such that $\psi_{1}\sim\vals(\pi_{1})$.
However, since $\pi_{1}\preceq\pi$ and $\pi_{2}\preceq\pi$ from
proposition~(\ref{logic:prop:FUAP:clean:sub:proof}) both $\pi_{1}$
and $\pi_{2}$ are totally clean proofs. It follows from
proposition~(\ref{logic:prop:FUAP:valuationmod:clean:proof}) that
$\vals(\pi_{1})=\val(\pi_{1})$ and $\vals(\pi_{2})=\val(\pi_{2})$.
Hence we need to show that $\val(\pi_{2})=\psi_{1}\to\psi_{2}$ for
some $\psi_{1},\psi_{2}\in\pv$ such that
$\psi_{1}\sim\val(\pi_{1})$. However since
$\pi_{1}\pon\pi_{2}\preceq\pi$ and $\pi$ is totally clean, from
proposition~(\ref{logic:prop:FUAP:clean:characterization}) we obtain
$\val(\pi_{2})=\psi_{1}\to\psi_{2}$ as requested where
$\psi_{1}=\val(\pi_{1})$ and $\psi_{2}=\val(\pi_{1}\pon\pi_{2})$. So
we now prove $(iii)$\,: we assume that $\gen x\pi_{1}\preceq\pi$.
From proposition~(\ref{logic:prop:FUAP:clean:characterization}) we
obtain $x\not\in\spec(\pi_{1})$.
\end{proof}

A proof is clean if it is {\em flawless}. So every sub-proof should
also be {\em flawless}. Conversely if every sub-proof is clean, we
should expect the proof to be clean. This is similar to
proposition~(\ref{logic:prop:FUAP:clean:sub:proof}) which was
established for totally clean proofs:
\begin{prop}\label{logic:prop:FUAP:almostclean:sub:proof}
Let $V$ be a set and $\pi\in\pvs$. Then the proof $\pi$ is clean
\ifand\ every sub-proof $\rho\preceq\pi$ of $\pi$ is itself clean.
\end{prop}
\begin{proof}
Since $\pi$ is a sub-proof of itself, the 'if' part is clear. So we
assume that $\pi$ is clean and $\rho\preceq\pi$. We need to show
that $\rho$ is clean. This follows immediately from the transitivity
of the sub-proof partial order on \pvs\,: if $\axi\phi\preceq\rho$
then $\axi\phi\preceq\pi$ and consequently $\phi\in\avs$. If
$\pi_{1}\pon\pi_{2}\preceq\rho$, then $\pi_{1}\pon\pi_{2}\preceq\pi$
and consequently $\vals(\pi_{2})=\psi_{1}\to\psi_{2}$ for some
$\psi_{1},\psi_{2}\in\pv$ such that $\psi_{1}\sim\vals(\pi_{1})$. If
$\gen x\pi_{1}\preceq\rho$ then $\gen x\pi_{1}\preceq\pi$ and
consequently we have $x\not\in\spec(\pi_{1})$.
\end{proof}

Just like in the case of totally clean proofs, we need the
appropriate tools to perform structural induction arguments on clean
proofs. Thus we need to say something on clean proofs of the form
$\pi_{1}\pon\pi_{2}$ and $\gen x\pi_{1}$ in relation to $\pi_{1}$
and $\pi_{2}$. The following proposition is the counterpart of
proposition~(\ref{logic:prop:FUAP:clean:modus:ponens}):
\begin{prop}\label{logic:prop:FUAP:almostclean:modus:ponens}
Let $V$ be a set and $\pi$ be a proof of the form
$\pi=\pi_{1}\pon\pi_{2}$ where $\pi_{1},\pi_{2}\in\pvs$. Let $\sim$
be the substitution congruence on \pv. Then $\pi$ is a clean proof
\ifand\ both $\pi_{1},\pi_{2}$ are clean and we have the equality:
    \[
    \vals(\pi_{2})=\psi_{1}\to\psi_{2}
    \]
for some $\psi_{1},\psi_{2}\in\pv$ such that
$\psi_{1}\sim\vals(\pi_{1})$ in which case $\psi_{2}=\vals(\pi)$.
\end{prop}
\begin{proof}
If $\vals(\pi_{2})=\psi_{1}\to\psi_{2}$ where
$\psi_{1}\sim\vals(\pi_{1})$ then from
definition~(\ref{logic:def:FUAP:valuationmod:valuation:modulo})\,:
    \begin{eqnarray*}
    \vals(\pi)&=&\vals(\pi_{1}\pon\pi_{2})\\
    &=&M^{+}\,(\vals(\pi_{1}), \vals(\pi_{2}))\\
    &=&M^{+}\,(\vals(\pi_{1}), \psi_{1}\to\psi_{2})\\
    \mbox{def.~(\ref{logic:def:FUAP:valuationmod:mp:modulo})},
    \ \psi_{1}\sim\vals(\pi_{1})\ \rightarrow&=&\psi_{2}
    \end{eqnarray*}
So it remains to show the equivalence. First we show the 'only if'
part: so we assume that $\pi=\pi_{1}\pon\pi_{2}$ is a clean proof.
Since $\pi_{1}\preceq\pi$ and $\pi_{2}\preceq\pi$, from
proposition~(\ref{logic:prop:FUAP:almostclean:sub:proof}) we see
that both $\pi_{1}$ and $\pi_{2}$ are clean. Furthermore, from
$\pi_{1}\pon\pi_{2}\preceq\pi$ and $(ii)$ of
definition~(\ref{logic:def:FUAP:almostclean:definition}) we see that
$\vals(\pi_{2})=\psi_{1}\to\psi_{2}$ as requested for some
$\psi_{1},\psi_{2}\in\pv$ such that $\psi_{1}\sim\vals(\pi_{1})$. We
now show the 'if' part: so we assume that both $\pi_{1},\pi_{2}$ are
clean and furthermore that $\vals(\pi_{2})=\psi_{1}\to\psi_{2}$
where $\psi_{1}\sim\vals(\pi_{1})$. We need to show that
$\pi=\pi_{1}\pon\pi_{2}$ is clean, namely that $(i)$, $(ii)$ and
$(iii)$ of definition~(\ref{logic:def:FUAP:almostclean:definition})
hold. First we show $(i)$. So we assume that
$\axi\phi\preceq\pi=\pi_{1}\pon\pi_{2}$. From
theorem~(\ref{logic:the:unique:representation}) of
page~\pageref{logic:the:unique:representation}, we cannot possibly
have $\axi\phi=\pi_{1}\pon\pi_{2}$. It follows that
$\axi\phi\preceq\pi_{1}$ or $\axi\phi\preceq\pi_{2}$, and
$\phi\in\avs$ follows from the fact that both $\pi_{1}$ and
$\pi_{2}$ are clean. We now prove $(iii)$. So we assume that $\gen
x\rho_{1}\preceq\pi=\pi_{1}\pon\pi_{2}$. Once again we cannot
possibly have $\gen x\rho_{1}=\pi_{1}\pon\pi_{2}$ and it follows
that $\gen x\rho_{1}\preceq\pi_{1}$ or $\gen
x\rho_{1}\preceq\pi_{2}$. So $x\not\in\spec(\rho_{1})$ follows from
the fact that $\pi_{1}$ and $\pi_{2}$ are clean proofs. We now prove
$(ii)$. So we assume that
$\rho_{1}\pon\rho_{2}\preceq\pi=\pi_{1}\pon\pi_{2}$. We need to show
that $\vals(\rho_{2})=\psi_{1}\to\psi_{2}$ for some
$\psi_{1},\psi_{2}$ with $\psi_{1}\sim\vals(\rho_{1})$. If
$\rho_{1}\pon\rho_{2}=\pi$ then from
theorem~(\ref{logic:the:unique:representation}) of
page~\pageref{logic:the:unique:representation} we have
$\rho_{1}=\pi_{1}$ and $\rho_{2}=\pi_{2}$ and the conclusion is true
by assumption. Otherwise, if $\rho_{1}\pon\rho_{2}\neq\pi$ then
$\rho_{1}\pon\rho_{2}\preceq\pi_{1}$ or
$\rho_{1}\pon\rho_{2}\preceq\pi_{2}$, and the conclusion follows
from the fact that both $\pi_{1}$ and $\pi_{2}$ are clean proofs.
\end{proof}

The following is the counterpart of
proposition~(\ref{logic:prop:FUAP:clean:generalization}) for clean
proofs:
\begin{prop}\label{logic:prop:FUAP:almostclean:generalization}
Let $V$ be a set and $\pi=\gen x\pi_{1}$ where $\pi_{1}\in\pvs$ and
$x\in V$. Then $\pi$ is a clean proof \ifand\ $\pi_{1}$ is itself
clean and $x\not\in\spec(\pi_{1})$.
\end{prop}
\begin{proof}
First we show the 'only if' part: so we assume that $\pi=\gen
x\pi_{1}$ is a clean proof. Since $\pi_{1}\preceq\pi$ we see that
$\pi_{1}$ is itself clean from
proposition~(\ref{logic:prop:FUAP:almostclean:sub:proof}). Since
$\gen x\pi_{1}\preceq\pi$, from $(iii)$ of
definition~(\ref{logic:def:FUAP:almostclean:definition}) we obtain
$x\not\in\spec(\pi_{1})$. We now show the 'if' part: so we assume
that $\pi_{1}$ is a clean proof and $x\not\in\spec(\pi_{1})$. We
need to show that $\pi=\gen x\pi_{1}$ is itself clean. So we need to
prove $(i)$, $(ii)$ and $(iii)$ of
definition~(\ref{logic:def:FUAP:almostclean:definition}). First we
show $(i)$\,: so we assume that $\axi\phi\preceq\pi=\gen x\pi_{1}$.
From theorem~(\ref{logic:the:unique:representation}) of
page~\pageref{logic:the:unique:representation} we cannot possibly
have $\axi\phi=\gen x\pi_{1}$. Hence $\axi\phi\preceq\pi_{1}$ and
$\phi\in\avs$ follows from the fact that $\pi_{1}$ is a clean proof.
We now show $(ii)$\,: so we assume that
$\rho_{1}\pon\rho_{2}\preceq\pi=\gen x\pi_{1}$. Once again we cannot
have the equality $\rho_{1}\pon\rho_{2}=\gen x\pi_{1}$ and
consequently $\rho_{1}\pon\rho_{2}\preceq\pi_{1}$. Having assumed
that $\pi_{1}$ is a clean proof, we obtain
$\vals(\rho_{2})=\psi_{1}\to\psi_{2}$ as requested where
$\psi_{1}\sim\vals(\rho_{1})$ and $\sim$ is the substitution
congruence. We now show $(iii)$\,: so we assume that $\gen
y\rho_{1}\preceq\pi=\gen x\pi_{1}$. We need to show that
$y\not\in\spec(\rho_{1})$. We shall distinguish two cases: first we
assume that $\gen y\rho_{1}=\pi$. Then from
theorem~(\ref{logic:the:unique:representation}) of
page~\pageref{logic:the:unique:representation} we obtain $y=x$ and
$\rho_{1}=\pi_{1}$ and the conclusion $y\not\in\spec(\rho_{1})$ is
true by assumption. Next we assume that $\gen y\rho_{1}\neq\pi$.
Then we must have $\gen y\rho_{1}\preceq\pi_{1}$ and the conclusion
$y\not\in\spec(\rho_{1})$ follows from the fact that $\pi_{1}$ is a
clean proof.
\end{proof}

In
proposition~(\ref{logic:prop:FUAP:varvalmod:conclusions:sub:proof})
we showed that the variables of the conclusion modulo of any
sub-proof were variables of the proof itself, a statement which is
best summarized as $\var(\vals(\rho))\subseteq\var(\pi)$ for all
$\rho\preceq\pi$. In general, the converse is not true. A variable
of the proof may not appear at all in any conclusion modulo of any
sub-proof.  We know from $\pi=(x\in x)\pon\,((x\in x)\to(y\in y))$
with $x\neq y$ that a variable of $\pi$ has no reason to be a
variable for $\vals(\pi)$. However, we should expect it to appear
somewhere in one of the conclusion modulo of a sub-proof of $\pi$.
This is not the case. Simply consider $\pi=\axi\phi$ where $\phi$
fails to be an axiom modulo and $\var(\phi)\neq\emptyset$. Then
$\vals(\pi)=\bot\to\bot$. This is what happens with proofs which are
not clean. They behave unpredictably. Fortunately, our expectations
are met when dealing with clean proofs:
\begin{prop}\label{logic:prop:FUAP:varvalmod:conclusions:sub:proof:almost:clean}
Let $V$ be a set and $\pi\in\pvs$ be a clean proof. Then:
    \begin{equation}\label{logic:eqn:FUAP:varvalmod:conclusions:almost:clean:1}
    \cup\{\,\var(\vals(\rho))\ :\
    \rho\preceq\pi\,\}=\var(\pi)
    \end{equation}
i.e. the variables of conclusions modulo of sub-proofs of $\pi$ are
the variables of~$\pi$.
\end{prop}
\begin{proof}
By virtue of
proposition~(\ref{logic:prop:FUAP:varvalmod:conclusions:sub:proof}),
we only need to show the implication:
    \[
    (\mbox{$\pi$ clean})\ \Rightarrow\ \var(\pi)
    \subseteq\cup\{\,\var(\vals(\rho))\ :\
    \rho\preceq\pi\,\}
    \]
We shall do so with a structural induction argument, using
theorem~(\ref{logic:the:proof:induction}) of
page~\pageref{logic:the:proof:induction}. First we assume that
$\pi=\phi$ for some $\phi\in\pv$. Then $\pi$ is always a clean
proof. From definition~(\ref{logic:def:subformula}), the only
sub-proof of $\pi$ is $\pi$ itself. Hence we need to show the
inclusion $\var(\pi)\subseteq\var(\vals(\pi))$ which follows from
the equalities $\var(\pi)=\var(\phi)$ and $\vals(\pi)=\phi$. Next we
assume that $\pi=\axi\phi$ for some $\phi\in\pv$. Furthermore we
assume that $\pi$ is a clean proof. Since $\axi\phi\preceq\pi$, from
definition~(\ref{logic:def:FUAP:almostclean:definition}) we have
$\phi\in\avs$. Once again the only sub-proof of $\pi$ is $\pi$
itself and we need to show that
$\var(\pi)\subseteq\var(\vals(\pi))$. However since $\phi\in\avs$ we
obtain $\vals(\pi)=\phi$ while $\var(\pi)=\var(\phi)$ so the
inclusion is again clear. So we now assume that
$\pi=\pi_{1}\pon\pi_{2}$ where $\pi_{1},\pi_{2}\in\pvs$ are proofs
for which the implication is true. We need to show the same is true
of $\pi$. So we assume that $\pi$ is a clean proof and we need to
show the inclusion is true for $\pi$. However, from
proposition~(\ref{logic:prop:FUAP:almostclean:modus:ponens}) we see
that both $\pi_{1}$ and $\pi_{2}$ are clean proofs. So $\pi_{1}$ and
$\pi_{2}$ satisfy the inclusion and consequently we have:
    \begin{eqnarray*}
    \var(\pi)&=&\var(\pi_{1}\pon\pi_{2})\\
    &=&\var(\pi_{1})\cup\var(\pi_{2})\\
    &\subseteq&(\cup\{\var(\vals(\rho)):\rho\preceq\pi_{1}\})
    \cup(\cup\{\var(\vals(\rho)):\rho\preceq\pi_{2}\})\\
    &=&\cup\{\var(\vals(\rho)):\rho\in\subf(\pi_{1})\cup\subf(\pi_{2})\}\\
    &\subseteq&\cup\{\var(\vals(\rho)):\rho\in\subf(\pi_{1})\cup\subf(\pi_{2})
    \cup\{\pi_{1}\pon\pi_{2}\}\}\\
    \mbox{def.~(\ref{logic:def:subformula})}\ \rightarrow
    &=&\cup\{\var(\vals(\rho)):\rho\in\subf(\pi_{1}\pon\pi_{2})\}\\
    &=&\cup\{\var(\vals(\rho)):\rho\preceq\pi\}\\
    \end{eqnarray*}
So we now assume that $\pi=\gen x\pi_{1}$ where $x\in V$ and
$\pi_{1}\in\pvs$ is a proof for which the implication is true. We
need to show the same is true of $\pi$. So we assume that $\pi$ is a
clean proof and we need to show the inclusion is true for $\pi$.
However, from
proposition~(\ref{logic:prop:FUAP:almostclean:generalization}) we
see that $\pi_{1}$ is itself clean. Having assumed the implication
is true for $\pi_{1}$,  it satisfies the inclusion and consequently:
    \begin{eqnarray*}
    \var(\pi)&=&\var(\gen x\pi_{1})\\
    &=&\{x\}\cup\var(\pi_{1})\\
    &\subseteq&\{x\}\cup(\,\cup\{\var(\val(\rho)):\rho\in\subf(\pi_{1})\}\,)\\
    \mbox{A: to be proved}\ \rightarrow&\subseteq&
    \cup\{\var(\val(\rho)):\rho\in\subf(\pi_{1})\cup\{\gen x\pi_{1}\}\}\\
    \mbox{def.~(\ref{logic:def:subformula})}\ \rightarrow
    &=&\cup\{\var(\val(\rho)):\rho\in\subf(\gen x\pi_{1})\}\\
    &=&\cup\{\var(\val(\rho)):\rho\preceq\pi\}\\
    \end{eqnarray*}
So it remains to show point A, for which it is sufficient to prove
$x\in\var(\vals(\rho))$ for $\rho=\gen x\pi_{1}$. This follows from
$\pi$ being clean and $\vals(\rho)=\forall x\vals(\pi_{1})$.
\end{proof}

Since $\vals$ is an extension of $\val$ from the domain of totally
clean proofs to that of clean proofs, the previous
proposition~(\ref{logic:prop:FUAP:varvalmod:conclusions:sub:proof:almost:clean})
can be restricted to the case of totally clean proofs and the
valuation  $\val:\pvs\to\pv$:
\begin{prop}\label{logic:prop:FUAP:varvalmod:conclusions:sub:proof:clean}
Let $V$ be a set and $\pi\in\pvs$ be a totally clean proof. Then:
    \begin{equation}\label{logic:eqn:FUAP:varvalmod:conclusions:clean:1}
    \cup\{\,\var(\val(\rho))\ :\
    \rho\preceq\pi\,\}=\var(\pi)
    \end{equation}
i.e. the variables of conclusions of sub-proofs of $\pi$ are the
variables of $\pi$.
\end{prop}
\begin{proof}
We assume that $\pi$ is totally clean. Then in particular from
proposition~(\ref{logic:def:FUAP:almostclean:clean}) it is a clean
proof. Applying
proposition~(\ref{logic:prop:FUAP:varvalmod:conclusions:sub:proof:almost:clean})
we obtain:
    \[
    \cup\{\,\var(\vals(\rho))\ :\
    \rho\preceq\pi\,\}=\var(\pi)
    \]
However, every $\rho\preceq\pi$ is totally clean by virtue of
proposition~(\ref{logic:prop:FUAP:clean:sub:proof}). It follows from
proposition~(\ref{logic:prop:FUAP:valuationmod:clean:proof}) that
$\vals(\rho)=\val(\rho)$ and
equation~(\ref{logic:eqn:FUAP:varvalmod:conclusions:clean:1})
follows.
\end{proof}

Given $\pi\in\pvs$, the specific variables of $\pi$ are the free
variables of the elements of $\hyp(\pi)$. If $\pi$ is a clean proof,
it contains no flawed application of the generalization rule of
inference. In other words, no generalization occurs with respect to
a variable which is not truly  arbitrary. So no specific variable
should get bound and the elements of $\spec(\pi)$ should remain free
variables:

\begin{prop}\label{logic:prop:FUAP:cleanproof:spec:free}
Let $V$ be a set and $\pi\in\pvs$ be a clean proof. Then:
    \[
    \spec(\pi)\subseteq\free(\pi)
    \]
\end{prop}
\begin{proof}
For every proof $\pi\in\pvs$ we need to show the following
implication:
    \[
    (\mbox{$\pi$ clean})\ \Rightarrow\ \spec(\pi)\subseteq\free(\pi)
    \]
We shall do so with a structural induction, using
theorem~(\ref{logic:the:proof:induction}) of
page~\pageref{logic:the:proof:induction}. First we assume that
$\pi=\phi$ for some $\phi\in\pv$. Then $\pi$ is always clean in this
case and we simply need to prove the inclusion which follows from:
    \[
    \spec(\phi)=\free(\hyp(\phi))=\free(\{\phi\})=\free(\phi)
    \]
We now assume that $\pi=\axi\phi$ for some $\phi\in\pv$. Then
$\spec(\pi)=\emptyset$ and the inclusion is always true, and so is
the implication. So we now assume that $\pi=\pi_{1}\pon\pi_{2}$
where $\pi_{1},\pi_{2}\in\pvs$ are proofs satisfying our
implication. We need to show the same is true for $\pi$. So we
assume that $\pi$ is clean. We need to show the inclusion is true
for $\pi$. However, from
proposition~(\ref{logic:prop:FUAP:almostclean:modus:ponens}) both
$\pi_{1}$ and $\pi_{2}$ are clean and it follows that the inclusion
is true for $\pi_{1}$ and $\pi_{2}$. Hence:
    \begin{eqnarray*}
    \spec(\pi)&=&\spec(\pi_{1}\pon\pi_{2})\\
        \mbox{prop.~(\ref{logic:prop:FUAP:freevar:recursive:def})}\ \rightarrow
        &=&\spec(\pi_{1})\cup\spec(\pi_{2})\\
        &\subseteq&\free(\pi_{1})\cup\free(\pi_{2})\\
        &=&\free(\pi_{1}\pon\pi_{2})\\
        &=&\free(\pi)
    \end{eqnarray*}
We now assume that $\pi=\gen x\pi_{1}$ where $x\in V$ and
$\pi_{1}\in\pvs$ is a proof satisfying our implication. We need to
show the same is true of $\pi$. So we assume that $\pi$ is clean. We
need to show the inclusion is true for $\pi$. However, from
proposition~(\ref{logic:prop:FUAP:almostclean:generalization}) we
see that $\pi_{1}$ is clean and $x\not\in\spec(\pi_{1})$. Hence:
    \begin{eqnarray*}
        \spec(\pi)&=&\spec(\gen x\pi_{1})\\
        \mbox{prop.~(\ref{logic:prop:FUAP:freevar:recursive:def})}\ \rightarrow
        &=&\spec(\pi_{1})\\
        x\not\in\spec(\pi_{1})\ \rightarrow
        &=&\spec(\pi_{1})\setminus\{x\}\\
        &\subseteq&\free(\pi_{1})\setminus\{x\}\\
        &=&\free(\gen x\pi_{1})\\
        &=&\free(\pi)
    \end{eqnarray*}
\end{proof}

This completes our section on clean proofs which brings the final
touches to the study of the new valuation $\vals:\pvs\to\pv$ which
we hope will fulfill its promise. We were driven to define $\vals$
after realizing that $\val$ would not allow us to write
$\val\circ{\cal M}(\pi)\sim{\cal M}\circ\val(\pi)$ despite opting
for a very natural definition of minimal transform for proofs. So we
now hope to have the equivalence $\vals\circ{\cal M}(\pi)\sim{\cal
M}\circ\vals(\pi)$. However, we should also extend
proposition~(\ref{logic:prop:FUAP:validsubtotclean:valuation:commute})
and obtain $\vals\circ\sigma(\pi)=\sigma\circ\vals(\pi)$ for clean
proofs when $\sigma$ is valid for $\pi$. Unless we can prove this
equality (or possibly equivalence modulo substitution), the
introduction of $\vals$ would hardly be a step forward.
