We have two compelling reasons to deal with model theory: one of
them is to establish G\"odel's completeness theorem which we need to
do in order to validate our deductive system. Another reason is that
model theory is the simplest way to prove that valuations $v:\pv\to
2$ actually do exist. So the dual space \pvd\ is not empty, the
sequent $\vdash\bot$ is false, the empty set is consistent and
satisfiable. This is not much but it has to be done: a deductive
system for which the sequent $\vdash\bot$ is true is pretty
worthless. A semantic entailment $\vDash$ defined in terms of an
empty dual space \pvd\ as per
definition~(\ref{logic:def:FOPL:semantics:entailments}) is also
pointless.

So we shall start by defining the notion of {\em model}, also known
as {\em structure}. Since our language \pv\ has no equality, no
constant or function symbol, and is effectively limited to a single
binary relation symbol '$\in$', the corresponding notion of {\em
model} is restricted to an ordered pair $(M,r)$ where $M$ is a set
and $r$ is a binary relation on $M$. This is obviously a significant
restriction compared to the general setting, for which the reader
should consult David Marker~\cite{Marker}. At this point in time,
our aim is simply to study the free universal algebra \pv, describe
various natural congruences on it, and see whether axiomatic set
theory can be coded with it. The language \pv\ is similar to untyped
$\lambda$-calculus: it is limited to the bare minimum, and we aim to
show this minimum is enough.

\index{model@Model of \pv}\index{m@$(M,r)$ : model of
\pv}\index{r@$r$ : binary relation on model $M$}
\begin{defin}\label{logic:def:FOPL:model:model}
Let $V$ be a set. We call {\em model} of\, \pv\ any ordered pair
$(M,r)$ where $M$ is a set and $r$ is a binary relation on $M$, i.e.
a subset of $M\times M$.
\end{defin}
Note that $(M,r)=(\emptyset,\emptyset)$ is a model of \pv\ for any
set $V$. Furthermore, if $(M,r)$ is a model of \pv, then it is also
a model of ${\bf P}(W)$ for any set $W$. Whenever the context is
clear, we shall refer to a model $(M,r)$ simply by '$M$'. Most
textbook references will exclude the empty set as a possible model.
We see no reason to do that, and will follow P.T.
Johnstone~\cite{Johnstone}  and Wilfrid Hodges~\cite{Hodges} in
allowing the empty structure. As already pointed out, there is
nothing deep or interesting about the empty set. It is often no more
than a limiting case which feels like an annoying glitch. It is
however a good discipline to confront our fear of the empty set, and
we would hate to exclude it unless we find compelling reasons to do
so. Thus we shall accept the empty set as a model, and carefully
describe the implications of this choice as we go along. The next
definition introduces the notion of {\em variables assignment} which
are simply maps $a:V\to M$ where $M$ is a model. If $M$ is the empty
model, then such variable assignments do not exist, unless of course
$V$ is itself empty, in which case there is a unique variables
assignment, namely the empty map. Let us assume for now that
$V\neq\emptyset$. In definition~(\ref{logic:def:FOPL:model:truth})
below, we introduce the standard notion of {\em truth} of a formula
$\phi\in\pv$ relative to a model $M$ and assignment $a:V\to M$. The
property of being {\em true} is denoted $M\vDash\phi[a]$. It is a
standard practice to say that a formula $\phi$ is {\em true} in the
model $M$ which we denote $M\vDash\phi$, \ifand\ it is true relative
to every assignment $a:V\to M$. Whenever $M$ is the empty model and
$V\neq\emptyset$, there exists no variables assignment $a:V\to M$
and the property $M\vDash\phi$ is vacuously true for every
$\phi\in\pv$. This is probably why most authors wish to exclude the
empty set as a possible structure: it is highly uncomfortable to
claim that {\em everything is true in the empty model}. This is even
more problematic when an author wishes to define a {\em satisfiable
set} $\Gamma\subseteq\pv$ as a set {\em which has a model}, i.e. for
which there exists a model $M$ such that $M\vDash\phi$ for all
$\phi\in\Gamma$. If we allow the empty model, then every set
$\Gamma\subseteq\pv$ is {\em satisfiable} which is a huge problem.
However, this problem is easily resolved: we just need to be careful
about what it means for a formula $\phi$ to {\em have a model}, or a
subset $\Gamma\subseteq\pv$ to be {\em satisfiable}: there should
exist a model $M$ {\em as well as } an assignment $a:V\to M$ such
that $M\vDash\phi[a]$, or such that $M\vDash\phi[a]$ for all
$\phi\in\Gamma$. With this in mind, it is clear the empty model is
no longer a problem, as there is no pair $(M,a)$ with $M=\emptyset$
which {\em satisfies} anything in the case when $V\neq\emptyset$. As
it turns out, we defined a {\em satisfiable} subset
$\Gamma\subseteq\pv$ in terms of valuations $v:\pv\to 2$ as per
definition~(\ref{logic:def:FOPL:semantics:valuation:truth}). As we
shall see, every valuation $v\in\pvd$ {\em has a model}, which is
the essence of G\"odel's completeness theorem. Once again we should
be clear as to what $v$ {\em having a model} means: there should
exist a model $M$ {\em as well as} an assignment $a:V\to M$ such
that $v=\beta(\,\cdot\,)(a)$, where $\beta$ is the model valuation
function of $M$, as per
definition~(\ref{logic:def:FOPL:model:valuation:function}) below.

\index{variable@Variables assignment}\index{assignment@Assignment of
variables}\index{a@$a,b:V\to M$ : var assignment}
\begin{defin}\label{logic:def:FOPL:model:assignment}
Let $V$ be a set and $M$ be a model of\, \pv. Any arbitrary map
$a:V\to M$ is called an {\em assignment of variables} relative to
the model $M$.
\end{defin}
