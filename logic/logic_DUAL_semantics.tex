As explained in the previous section, we shall define the notion of
semantic entailment in terms of valuation rather than model. In line
with this convention, the following definition introduces the notion
of {\em truth} and {\em satisfaction} also in terms of valuation.
Hence we shall introduce obvious notations such as $v\vDash\phi$ and
$v\vDash\Gamma$, which should be read as ``\,{\em $v$ satisfies
$\phi$}\,'' and ``\,{\em $v$ satisfies $\Gamma$}\,''.

\index{truth@Truth of formula}\index{satisfy@Satisfiability of
formula $\phi$}\index{satisfy@Satisfiability of subset
$\Gamma$}\index{v@$v\vDash\phi$ : $v$ satisfies
$\phi$}\index{v@$v\vDash\Gamma$ : $v$ satisfies $\Gamma$}
\begin{defin}\label{logic:def:FOPL:semantics:valuation:truth}
Let $V$ be a set and $v:\pv\to 2$ be a valuation on \pv. We say that
$\phi\in\pv$ is {\em true} for $v$  or that $v$ {\em satisfies}
$\phi$ and we write:
    \[
    v\vDash\phi
    \]
\ifand\ $v(\phi)=1$. Given $\Gamma\subseteq\pv$, we say that $v$
{\em satisfies} $\Gamma$ and we write:
    \[
    v\vDash\Gamma
    \]
\ifand\ $v\vDash\psi$ for all $\psi\in\Gamma$. We say $\Gamma$ is
{\em satisfiable} if $v\vDash\Gamma$ for some $v$.
\end{defin}

Note that the notations $v\vDash\phi$ and $v\vDash\{\phi\}$ have
equivalent meaning. It is also clear that if $v$ satisfies $\Delta$
and $\Delta\supseteq\Gamma$ then $v$ also satisfies $\Gamma$.
Finally, any valuation $v$ vacuously satisfies the empty set
$\emptyset$. This does not mean the empty set is satisfiable, until
we know for a fact that \pvd\ is not empty. In
definition~(\ref{logic:def:FOPL:proof:of:formula}) we introduced a
relation $\vdash\,\subseteq{\cal P}(\pv)\times\pv$, which is often
referred to as the relation of {\em syntactic entailment}. We shall
now define a similar relation $\vDash$ which is that of {\em
semantic entailment}. Given $\Gamma\subseteq\pv$ and $\phi\in\pv$,
the statement $\Gamma\vDash\phi$ expresses the fact that for every
valuation $v\in\pvd$\,:
    \[
    v\mbox{\ satisfies\ }\Gamma\ \Rightarrow\ v\mbox{\ satisfies\
    }\phi
    \]
At this stage of the discussion, we still cannot be sure that the
dual space \pvd\ is not empty. If this was the case, then
$\Gamma\vDash\phi$ would be vacuously true.

\index{semantic@Semantic consequence}\index{entailment@Semantic
entailment}\index{semantic@Semantic
entailment}\index{gamma@$\Gamma\vDash\phi$ : $\Gamma$ semantically
entails $\phi$} \index{phi@$\vDash\phi$ : $\phi$ is
valid}\index{valid@Valid formula}
\begin{defin}\label{logic:def:FOPL:semantics:entailments}
Let $V$ be a set and $\Gamma\subseteq\pv$. Let $\phi\in\pv$. We say
that $\phi$ is a semantic consequence of $\Gamma$, or that $\Gamma$
{\em semantically entails} $\phi$ and we write:
    \[
    \Gamma\vDash\phi
    \]
\ifand\ for every valuation $v:\pv\to 2$ we have the implication:
    \[
    v\vDash\Gamma\ \Rightarrow\ v\vDash\phi
    \]
Furthermore, we say that $\phi$ is {\em valid} and we write
$\vDash\phi$ \ifand\ $\emptyset\vDash\phi$.
\end{defin}

We defined a valid formula as any formula $\phi$ which satisfies the
semantic entailment $\emptyset\vDash\phi$. However, a valuation
$v:\pv\to 2$ always satisfies the empty set. Hence, a valid formula
is simply a formula which is true for every valuation.

\index{truth@Truth and validity}
\begin{prop}\label{logic:prop:FOPL:semantics:validity:charac}
Let $V$ be a set and $\phi\in\pv$. Then we have the equivalence:
    \[
    \vDash\phi\ \Leftrightarrow\ (\,\forall v\in\pvd\ ,\ v\vDash\phi\,)
    \]
In other words, a formula is valid \ifand\ it is true for every
valuation.
\end{prop}
\begin{proof}
First we show $\Rightarrow$\,: so we assume that $\phi$ is valid,
i.e. $\vDash\phi$. Let $v\in\pvd$ be a valuation. We need to show
that $v\vDash\phi$. However from $\vDash\phi$ which is
$\emptyset\vDash\phi$, we have the implication $v\vDash\emptyset\
\Rightarrow\ v\vDash\phi$. It is therefore sufficient to prove that
$v\vDash\emptyset$, i.e. that $v$ satisfies the empty set
$\emptyset$. But this last statement is equivalent to $v\vDash\psi$
for all $\psi\in\emptyset$ which is vacuously true. So we now prove
$\Leftarrow$\,: we assume that $v\vDash\phi$ for all $v\in\pvd$. We
need to show that $\vDash\phi$. So let $v$ be a valuation. We need
to show the implication $v\vDash\emptyset\ \Rightarrow\
v\vDash\phi$. However, $v\vDash\emptyset$ is vacuously true. So we
simply need to show that $v\vDash\phi$ which is true by assumption.
\end{proof}

The relation $\vDash\,\subseteq{\cal P}(\pv)\times\pv$ will be seen
to coincide with $\vdash$ after we prove
theorem~(\ref{logic:the:FOPL:semantics:syn:equiv:sem}) of
page~\pageref{logic:the:FOPL:semantics:syn:equiv:sem}. In
particular, both relations satisfy the same properties. So rather
than spend too much time investigating the relation $\vDash$, we
shall focus on establishing the equality $\vDash\,=\,\vdash$.
However, we now need to show:
\begin{prop}\label{logic:def:FOPL:semantics:monotonicity}
Let $V$ be a set and $\Gamma,\Delta\subseteq\pv$. Then for all
$\phi\in\pv$\,:
    \[
    (\Gamma\supseteq\Delta)\land(\Delta\vDash\phi)\ \Rightarrow\
    \Gamma\vDash\phi
    \]
\end{prop}
\begin{proof}
We assume that $\Gamma\supseteq\Delta$ and $\Delta\vDash\phi$. We
need to show that $\Gamma\vDash\phi$. So let $v$ be a valuation
which satisfies $\Gamma$, i.e. such that $v\vDash\Gamma$. We need to
show that $\phi$ is true for $v$, i.e. that $v\vDash\phi$. However,
from $v\vDash\Gamma$ we have $v\vDash\psi$ for all $\psi\in\Gamma$.
Having assumed that $\Gamma\supseteq\Delta$, it follows that
$v\vDash\psi$ for all $\psi\in\Delta$. Hence we see that $v$
satisfies $\Delta$ i.e. that $v\vDash\Delta$. From
$\Delta\vDash\phi$, we conclude that $v\vDash\phi$.
\end{proof}

We shall now establish one side of the equivalence between syntactic
and semantic entailments. The full result will be obtained in
theorem~(\ref{logic:the:FOPL:semantics:syn:equiv:sem}) of
page~\pageref{logic:the:FOPL:semantics:syn:equiv:sem}. The proof of
the following proposition is similar in spirit to that of the
transitivity of $\vdash$ in
proposition~(\ref{logic:prop:FOPL:deduction:transitivity}) which
relies on a induction on the cardinal $|\Gamma|$. The deduction
theorem~(\ref{logic:the:FOPL:deduction}) of
page~\pageref{logic:the:FOPL:deduction} effectively allows us to
focus on proving the implication $\vdash\phi\ \Rightarrow\
\vDash\phi$ which is true from the very definition of a valuation.

\begin{prop}\label{logic:prop:FOPL:semantics:syn:imp:sem}
Let $V$ be a set and $\Gamma\subseteq\pv$. Then for all
$\phi\in\pv$\,:
    \begin{equation}\label{logic:eqn:FOPL:semantics:syn:imp:sem:1}
    \Gamma\vdash\phi\ \Rightarrow\ \Gamma\vDash\phi
    \end{equation}
\end{prop}
\begin{proof}
Without loss of generality, we may assume that $\Gamma$ is a finite
set. Indeed, suppose the proposition has been proved for $\Gamma$
finite. We need to show it is also true in the general case. So
suppose $\Gamma\vdash\phi$. We need to show that $\Gamma\vDash\phi$.
However, there exists $\Gamma_{0}$ finite such that
$\Gamma_{0}\subseteq\Gamma$ and $\Gamma_{0}\vdash\phi$. Having
assumed the proposition is true in the finite case, we obtain
$\Gamma_{0}\vDash\phi$ and $\Gamma\vDash\phi$ follows from
$\Gamma_{0}\subseteq\Gamma$ and
proposition~(\ref{logic:def:FOPL:semantics:monotonicity}). So
without loss of generality, we assume that $\Gamma$ is a finite set.
We shall prove the
implication~(\ref{logic:eqn:FOPL:semantics:syn:imp:sem:1}) for all
$\phi\in\pv$ using an induction argument on the cardinal $|\Gamma|$
of the set $\Gamma$. So first we assume that $|\Gamma|=0$. Then
$\Gamma=\emptyset$ and given $\phi\in\pv$ we need to prove that
$\vdash\phi\ \Rightarrow\ \vDash\phi$. So we assume that
$\vdash\phi$ and we need to show that $\vDash\phi$. So let
$v\in\pvd$ be a valuation. Using
proposition~(\ref{logic:prop:FOPL:semantics:validity:charac}) we
need to show that $v\vDash\phi$, i.e. $v(\phi)=1$ which is true
since $v$ is a valuation and $\vdash\phi$. So given $n\in\N$, we now
assume that~(\ref{logic:eqn:FOPL:semantics:syn:imp:sem:1}) is true
for all $\phi\in\pv$ whenever we have $|\Gamma|=n$. We need to show
the same applies when $|\Gamma|=n+1$. So we assume that
$|\Gamma|=n+1$, and given $\phi\in\pv$ we assume that
$\Gamma\vdash\phi$. We need to show that $\Gamma\vDash\phi$. So let
$v\in\pvd$ be a valuation satisfying $\Gamma$ i.e. such that
$v\vDash\Gamma$. We need to show that $\phi$ is true for $v$ which
is $v\vDash\phi$. However, from $|\Gamma|=n+1$ we see that $\Gamma$
is not empty. So let $\psi^{*}\in\Gamma$ and define
$\Gamma^{*}=\Gamma\setminus\{\psi^{*}\}$. Then $|\Gamma^{*}|=n$ and
$\Gamma=\Gamma^{*}\cup\{\psi^{*}\}$. From our assumption
$\Gamma\vdash\phi$ we obtain $\Gamma^{*}\cup\{\psi^{*}\}\vdash\phi$.
Using the deduction theorem~(\ref{logic:the:FOPL:deduction}) of
page~\pageref{logic:the:FOPL:deduction} it follows that
$\Gamma^{*}\vdash(\psi^{*}\to\phi)$. Having assumed
that~(\ref{logic:eqn:FOPL:semantics:syn:imp:sem:1}) is true for all
$\phi\in\pv$ whenever $|\Gamma|=n$, in particular it is true for
$\psi^{*}\to\phi$ and $\Gamma^{*}$. Hence from
$\Gamma^{*}\vdash(\psi^{*}\to\phi)$ we obtain
$\Gamma^{*}\vDash(\psi^{*}\to\phi)$. Furthermore, from the inclusion
$\Gamma^{*}\subseteq\Gamma$ and the assumption that $v$ satisfies
$\Gamma$, we see that $v$ also satisfies $\Gamma^{*}$ i.e.
$v\vDash\Gamma^{*}$. So from $\Gamma^{*}\vDash(\psi^{*}\to\phi)$ we
see that $v\vDash(\psi^{*}\to\phi)$, which is
$1=v(\psi^{*}\to\phi)=v(\psi^{*})\to v(\phi)$. So we see that
$v(\psi^{*})\leq v(\phi)$. In order to show that $v\vDash\phi$ which
is $v(\phi)=1$, it is therefore sufficient to prove that
$v(\psi^{*})=1$. However, $\psi^{*}\in\Gamma$ and by assumption $v$
satisfies $\Gamma$. It follows that $\psi^{*}$ is true for $v$, i.e.
$v(\psi^{*})=1$, which completes our induction argument.
\end{proof}
