In definition~(\ref{logic:def:FUAP:mintransproof:transform}) we
extended the notion of minimal transform to proofs. In
definition~(\ref{logic:def:FUAP:subcong:substitution:congruence}) we
introduced a substitution congruence on the algebra \pvs. In light
of theorem~(\ref{logic:the:FOPL:mintransfsubcong:kernel}) of
page~\pageref{logic:the:FOPL:mintransfsubcong:kernel}, we expect
that ${\cal M}(\pi)={\cal M}(\rho)$ be equivalent to $\pi\sim\rho$.
In this section we prove that it is indeed the case. This will be
the object of theorem~(\ref{logic:the:FUAP:mintransfsubcong:kernel})
below. We shall also provide an immediate consequence of
theorem~(\ref{logic:the:FUAP:mintransfsubcong:kernel}) showing that
$\alpha$-equivalence is preserved by valid substitutions. This will
be the object of
theorem~(\ref{logic:the:FUAP:mintransfsubcong:valid}). We start by
showing that the equality  ${\cal M}(\pi)={\cal M}(\rho)$ defines a
congruence on the free universal algebra \pvs.

\begin{prop}\label{logic:prop:FUAP:mintransfsubcong:congruence}
Let $V$ be a set and $\equiv$ be the relation on \pvs\ defined by:
    \[
    \pi\equiv\rho\ \Leftrightarrow\ {\cal M}(\pi)={\cal M}(\rho)
    \]
for all $\pi,\rho\in\pvs$. Then $\equiv$ is a congruence on \pvs.
\end{prop}
\begin{proof}
The relation $\equiv$ is clearly reflexive, symmetric and
transitive. So it is an equivalence relation on \pvs\ and we simply
need to prove that it is a congruent relation, as per
definition~(\ref{logic:def:congruent:relation}) of
page~\pageref{logic:def:congruent:relation}. We already know that
$\axi\phi\equiv\axi\phi$ for all $\phi\in\pv$. So let
$\pi=\pi_{1}\pon\pi_{2}$ and $\rho=\rho_{1}\pon\rho_{2}$ where
$\pi_{1}\equiv\rho_{1}$ and $\pi_{2}\equiv\rho_{2}$. We need to show
that $\pi\equiv\rho$, which goes as follows:
    \begin{eqnarray*}
    {\cal M}(\pi)&=&{\cal M}(\pi_{1}\pon\pi_{2})\\
    &=&{\cal M}(\pi_{1})\pon\,{\cal M}(\pi_{2})\\
    (\pi_{1}\equiv\rho_{1})\land(\pi_{2}\equiv\rho_{2})\ \rightarrow
    &=&{\cal M}(\rho_{1})\pon\,{\cal M}(\rho_{2})\\
    &=&{\cal M}(\rho_{1}\pon\rho_{2})\\
    &=&{\cal M}(\rho)
    \end{eqnarray*}
Next we assume that $\pi=\gen x\pi_{1}$ and $\rho=\gen x\rho_{1}$
where $x\in V$ and $\pi_{1}\equiv\rho_{1}$. We need to show that
$\pi\equiv\rho$, which goes as follows:
    \begin{eqnarray*}
    {\cal M}(\pi)&=&{\cal M}(\gen x\pi_{1})\\
    \mbox{$n=\min\{k:[k/x]\mbox{ valid for }{\cal M}(\pi_{1})\}$}\
    \rightarrow&=&\gen n{\cal M}(\pi_{1})[n/x]\\
    \pi_{1}\equiv\rho_{1}\ \rightarrow&=&\gen n{\cal M}(\rho_{1})[n/x]\\
    \pi_{1}\equiv\rho_{1}\Rightarrow n=m\ \rightarrow&=&\gen m{\cal M}(\rho_{1})[m/x]\\
    \mbox{$m=\min\{k:[k/x]\mbox{ valid for }{\cal M}(\rho_{1})\}$}\
    \rightarrow&=&{\cal M}(\gen x\rho_{1})\\
    &=&{\cal M}(\rho)
    \end{eqnarray*}
\end{proof}

The following theorem is the counterpart of
theorem~(\ref{logic:the:FOPL:mintransfsubcong:kernel}) of
page~\pageref{logic:the:FOPL:mintransfsubcong:kernel}. We provide an
identical proof which relies on the local inversion
theorem~(\ref{logic:the:FUAP:localinversion:inversion}).

\index{minimal@Minimal transform and $\sim$}
\begin{theorem}\label{logic:the:FUAP:mintransfsubcong:kernel}
Let $\sim$ be the substitution congruence on \pvs\ where $V$ is a
set. Then for all $\pi,\rho\in\pvs$ we have the equivalence:
    \[
    \pi\sim\rho\ \Leftrightarrow\ {\cal M}(\pi)={\cal M}(\rho)
    \]
where ${\cal M}(\pi)$ and ${\cal M}(\rho)$ are the minimal
transforms as per {\em
definition~(\ref{logic:def:FUAP:mintransproof:transform})}.
\end{theorem}
\begin{proof}
First we show $\Rightarrow$\,: consider the relation $\equiv$ on
\pvs\ defined by $\pi\equiv\rho$ \ifand\ ${\cal M}(\pi)={\cal
M}(\rho)$. We need to show the inclusion
$\sim\,\subseteq\,\equiv$\,. However, we know from
proposition~(\ref{logic:prop:FUAP:mintransfsubcong:congruence}) that
$\equiv$ is a congruence on \pvs. Since $\sim$ is the smallest
congruence on \pvs\ which contains the sets $R_{0}$, $R_{1}$ and
$R_{2}$ of
definition~(\ref{logic:def:FUAP:subcong:substitution:congruence}) we
simply need to show that $R_{i}\subseteq\,\equiv$ for $i\in 3$.
First we show that $R_{0}\subseteq\,\equiv$\,: so let $\pi=\phi$ and
$\rho=\psi$ for some $\phi,\psi\in\pv$ such that $\phi\sim\psi$,
where $\sim$ also denotes the substitution congruence on \pv. We
need to show that $\pi\equiv\rho$ which is ${\cal M}(\phi)={\cal
M}(\psi)$ which follows immediately from
theorem~(\ref{logic:the:FOPL:mintransfsubcong:kernel}) of
page~\pageref{logic:the:FOPL:mintransfsubcong:kernel}. Next we show
that $R_{1}\subseteq\,\equiv$\,: so let $\pi=\axi\phi$ and
$\rho=\axi\psi$ for some $\phi,\psi\in\pv$ such that $\phi\sim\psi$.
We need to show that $\pi\equiv\rho$ which is ${\cal M}(\pi)={\cal
M}(\rho)$. Since ${\cal M}(\pi)=\axi{\cal M}(\phi)$ and ${\cal
M}(\rho)=\axi{\cal M}(\psi)$, we simply need to show that ${\cal
M}(\phi)={\cal M}(\psi)$ which also follows from
theorem~(\ref{logic:the:FOPL:mintransfsubcong:kernel}). We now show
that $R_{2}\subseteq\,\equiv$\,: so let $\pi=\gen x\pi_{1}$ and
$\rho=\gen y\pi_{1}[y\!:\!x]$ where $x\neq y$ and
$y\not\in\free(\pi_{1})$. We need to show that $\pi\equiv\rho$ which
is ${\cal M}(\pi)={\cal M}(\rho)$. However, we have
$\rho=\sigma(\pi)$ where $\sigma:V\to V$ is the single variable
permutation $\sigma=[y\!:\!x]$. So we need to show that ${\cal
M}(\pi)={\cal M}\circ\sigma(\pi)$. Let
$\bar{\sigma}:\bar{V}\to\bar{V}$ be the minimal extension of
$\sigma$ as per
definition~(\ref{logic:def:FOPL:commute:minextensioon:map}) of
page~\pageref{logic:def:FOPL:commute:minextensioon:map}. Since
$\sigma$ is an injective map, in particular from
proposition~(\ref{logic:prop:FUAP:validsubproof:injective}) $\sigma$
is valid for $\pi$. So we can apply
theorem~(\ref{logic:the:FUAP:mintransvalidsub:commute}) of
page~\pageref{logic:the:FUAP:mintransvalidsub:commute} from which we
obtain ${\cal M}\circ\sigma(\pi)=\bar{\sigma}\circ{\cal M}(\pi)$. It
is therefore sufficient to prove that ${\cal
M}(\pi)=\bar{\sigma}\circ{\cal M}(\pi)$. Let $i:\bar{V}\to\bar{V}$
be the identity mapping. We need to show that $i\circ {\cal
M}(\pi)=\bar{\sigma}\circ{\cal M}(\pi)$ and from
proposition~(\ref{logic:prop:FUAP:variable:support}) it is
sufficient to prove that $i$ and $\bar{\sigma}$ coincide on
$\var({\cal M}(\pi))$. So let $u\in\var({\cal M}(\pi))$. We need to
show that $\bar{\sigma}(u)=u$. Since $\bar{V}$ is the disjoint union
of $V$ and \N, we shall distinguish two cases: first we assume that
$u\in\N$. Then $\bar{\sigma}(u)=u$ is immediate from
definition~(\ref{logic:def:FOPL:commute:minextensioon:map}). Next we
assume that $u\in V$. Then from $u\in\var({\cal M}(\pi))$ and
proposition~(\ref{logic:prop:FUAP:mintransformproof:freevar}) it
follows that $u\in\free(\pi)=\free(\pi_{1})\setminus\{x\}$. In
particular we obtain $u\neq x$. Furthermore, having assumed
$y\not\in\free(\pi_{1})$, we must have $u\neq y$. Hence we see that
$u\not\in\{x,y\}$ and consequently since $u\in V$ we obtain the
equality $\bar{\sigma}(u)=\sigma(u)=[y\!:\!x](u)=u$ as requested. We
now prove~$\Leftarrow$\,: we need to show that every $\pi\in\pvs$
satisfies the property:
    \[
    \forall\rho\in\pvs\ ,\ [\,{\cal M}(\pi)={\cal M}(\rho)\
    \Rightarrow\ \pi\sim\rho\,]
    \]
We shall do so by a structural induction argument using
theorem~(\ref{logic:the:proof:induction}) of
page~\pageref{logic:the:proof:induction}. First we assume that
$\pi=\phi$ for some $\phi\in\pv$. Let $\rho\in\pvs$ with the
property ${\cal M}(\pi)={\cal M}(\phi)={\cal M}(\rho)$. We need to
show that $\pi\sim\rho$. From
theorem~(\ref{logic:the:unique:representation}) of
page~\pageref{logic:the:unique:representation} the proof $\rho$ can
be of one and only one of four types: first it can be of the form
$\rho=\psi$ for some $\psi\in\pv$. Next, it can be of the form
$\rho=\axi\psi$, and possibly of the form
$\rho=\rho_{1}\pon\rho_{2}$ with $\rho_{1},\rho_{2}\in\pvs$.
Finally, it can be $\rho=\gen u\rho_{1}$ for some $u\in V$ and
$\rho_{1}\in\pvs$. Looking at
definition~(\ref{logic:def:FUAP:mintransproof:transform}) we see
that the minimal transform ${\cal M}(\rho)$ has the same basic
structure as $\rho$. Thus, from the equality ${\cal M}(\rho)={\cal
M}(\phi)\in{\bf P}(\bar{V})$ it follows that $\rho$ can only be of
the form $\rho=\psi$ for some $\psi\in\pv$. So we obtain ${\cal
M}(\rho)={\cal M}(\psi)={\cal M}(\phi)$ and consequently from
theorem~(\ref{logic:the:FOPL:mintransfsubcong:kernel}) of
page~\pageref{logic:the:FOPL:mintransfsubcong:kernel} we obtain
$\phi\sim\psi$ which is $\pi\sim\rho$ as requested. So we now assume
that $\pi=\axi\phi$ for some $\phi\in\pv$. Let $\rho\in\pvs$ be such
that ${\cal M}(\pi)={\cal M}(\rho)$. We need to show that
$\pi\sim\rho$. However, since ${\cal M}(\pi)=\axi{\cal M}(\phi)$ we
see that $\rho$ must be of the form $\rho=\axi\psi$ for some
$\psi\in\pv$. It follows that $\axi{\cal M}(\phi)=\axi{\cal
M}(\psi)$ and thus ${\cal M}(\phi)={\cal M}(\psi)$ which implies
that $\phi\sim\psi$ by virtue of
theorem~(\ref{logic:the:FOPL:mintransfsubcong:kernel}). Hence we
have $\axi\phi\sim\axi\psi$ which is $\pi\sim\rho$ as requested. We
now assume that $\pi=\pi_{1}\pon\pi_{2}$ where $\pi_{1},\pi_{2}$
satisfy our property. We need to show the same if true of $\pi$. So
let $\rho\in\pvs$ such that ${\cal M}(\pi)={\cal
M}(\pi_{1})\pon\,{\cal M}(\pi_{2})={\cal M}(\rho)$. We need to show
that $\pi\sim\rho$. From ${\cal M}(\rho) = {\cal
M}(\pi_{1})\pon\,{\cal M}(\pi_{2})$ we see that $\rho$ can only be
of the form $\rho=\rho_{1}\pon\rho_{2}$. It follows that ${\cal
M}(\rho)={\cal M}(\rho_{1})\pon\,{\cal M}(\rho_{2})$ and
consequently we obtain ${\cal M}(\pi_{1})\pon\,{\cal
M}(\pi_{2})={\cal M}(\rho_{1})\pon{\cal M}(\rho_{2})$. Hence, using
theorem~(\ref{logic:the:unique:representation}) of
page~\pageref{logic:the:unique:representation} we have ${\cal
M}(\pi_{1})={\cal M}(\rho_{1})$ and ${\cal M}(\pi_{2})={\cal
M}(\rho_{2})$. Having assumed $\pi_{1}$ and $\pi_{2}$ satisfy our
induction property, it follows that $\pi_{1}\sim\rho_{1}$ and
$\pi_{2}\sim\rho_{2}$ and consequently
$\pi_{1}\pon\pi_{2}\sim\rho_{1}\pon\rho_{2}$. So we have proved that
$\pi\sim\rho$ as requested. We now assume that $\pi=\gen x\pi_{1}$
where $x\in V$ and $\pi_{1}\in\pvs$ satisfies our property. We need
to show the same is true of $\pi$. So let $\rho\in\pvs$ such that
${\cal M}(\pi)={\cal M}(\rho)$. We need to show that $\pi\sim\rho$.
We have:
    \begin{equation}\label{logic:eqn:FUAP:mintransfsubcong:kernel:1}
    {\cal M}(\pi)=\gen n{\cal M}(\pi_{1})[n/x]
    \end{equation}
where $n=\min\{k\in\N:\mbox{$[k/x]$ valid for ${\cal
M}(\pi_{1})$}\}$. Hence from the equality ${\cal M}(\pi)={\cal
M}(\rho)$ we see that $\rho$ can only be of the form $\rho=\gen
y\rho_{1}$ for some $y\in V$ and $\rho_{1}\in\pvs$. We shall
distinguish two cases: first we assume that $y=x$. Then we need to
show that $\gen x\pi_{1}\sim\gen x\rho_{1}$ and it is therefore
sufficient to prove that $\pi_{1}\sim\rho_{1}$. Having assumed
$\pi_{1}$ satisfy our property, we simply need to show that ${\cal
M}(\pi_{1})={\cal M}(\rho_{1})$. However we have the equality:
    \begin{equation}\label{logic:eqn:FUAP:mintransfsubcong:kernel:2}
    {\cal M}(\rho)=\gen m{\cal M}(\rho_{1})[m/x]
    \end{equation}
where $m=\min\{k\in\N:\mbox{$[k/x]$ valid for ${\cal
M}(\rho_{1})$}\}$.
Comparing~(\ref{logic:eqn:FUAP:mintransfsubcong:kernel:1})
and~(\ref{logic:eqn:FUAP:mintransfsubcong:kernel:2}), from ${\cal
M}(\pi)={\cal M}(\rho)$ and
theorem~(\ref{logic:the:unique:representation}) of
page~\pageref{logic:the:unique:representation} we obtain $n=m$ and
thus:
    \begin{equation}\label{logic:eqn:FUAP:mintransfsubcong:kernel:3}
    {\cal M}(\pi_{1})[n/x]={\cal M}(\rho_{1})[n/x]
    \end{equation}
Consider the substitution $\sigma:\bar{V}\to\bar{V}$ defined by
$\sigma=[n/x]$. We shall conclude that ${\cal M}(\pi_{1})={\cal
M}(\rho_{1})$ by inverting
equation~(\ref{logic:eqn:FUAP:mintransfsubcong:kernel:3}) using the
local inversion
theorem~(\ref{logic:the:FUAP:localinversion:inversion}) of
page~\pageref{logic:the:FUAP:localinversion:inversion} on the
substitution $\sigma$. So consider the sets $V_{0}=V$ and
$V_{1}=\N$. It is clear that both $\sigma_{|V_{0}}$ and
$\sigma_{|V_{1}}$ are injective maps. Define:
    \[
    \Pi=\{\kappa\in{\bf\Pi}(\bar{V}):(\free(\kappa)\subseteq
    V_{0})\land(\bound(\kappa)\subseteq V_{1})\land(\mbox{$\sigma$
    valid for $\kappa$})\}
    \]
Then using theorem~(\ref{logic:the:FUAP:localinversion:inversion})
there exits $\tau:{\bf\Pi}(\bar{V})\to{\bf\Pi}(\bar{V})$ such that
$\tau\circ\sigma(\kappa)=\kappa$ for all $\kappa\in\Pi$. Hence from
equation~(\ref{logic:eqn:FUAP:mintransfsubcong:kernel:3}) it is
sufficient to prove that ${\cal M}(\pi_{1})\in\Pi$ and ${\cal
M}(\rho_{1})\in\Pi$. First we show that ${\cal M}(\pi_{1})\in\Pi$.
We already know that $\sigma=[n/x]$ is a valid substitution for
${\cal M}(\pi_{1})$. The fact that $\free({\cal
M}(\pi_{1}))\subseteq V_{0}$ follows from
proposition~(\ref{logic:prop:FUAP:mintransformproof:freevar}). The
fact that $\bound({\cal M}(\pi_{1}))\subseteq V_{1}$ follows from
proposition~(\ref{logic:prop:FUAP:mintransformproof:boundvar}). So
we have proved that ${\cal M}(\pi_{1})\in\Pi$ as requested. The
proof of ${\cal M}(\rho_{1})\in\Pi$ is identical, which completes
our proof of $\pi\sim\rho$ in the case when $\rho=\forall y\rho_{1}$
and $y=x$. We now assume that $y\neq x$. Consider the proof
$\rho^{*}=\gen x\rho_{1}[x\!:\!y]$ where $[x\!:\!y]$ is the
permutation mapping as per
definition~(\ref{logic:def:single:var:permutation}) of
page~\pageref{logic:def:single:var:permutation}. Suppose we have
proved the equivalence $\rho\sim\rho^{*}$. Then in order to prove
$\pi\sim\rho$ it is sufficient by transitivity to show that
$\pi\sim\rho^{*}$. However, having already proved the implication
$\Rightarrow$ of this theorem, we know that $\rho\sim\rho^{*}$
implies ${\cal M}(\rho)={\cal M}(\rho^{*})$. Hence, if we have
$\rho\sim\rho^{*}$, it is sufficient to prove $\pi\sim\rho^{*}$
knowing that ${\cal M}(\pi)={\cal M}(\rho^{*})$ and $\rho^{*}=\gen
x\rho_{1}^{*}$ where $\rho_{1}^{*}=\rho_{1}[x\!:\!y]$. So we are
back to the case when $y=x$, a case we have already dealt with. It
follows that we can complete our induction argument simply by
showing $\rho\sim\rho^{*}$. Since $\rho=\gen y\rho_{1}$ and
$\rho^{*}=\gen x\rho_{1}[x\!:\!y]$ with $x\neq y$, from
definition~(\ref{logic:def:FUAP:subcong:substitution:congruence}) of
page~\pageref{logic:def:FUAP:subcong:substitution:congruence} we
simply need to check that $x\not\in\free(\rho_{1})$. So suppose to
the contrary that $x\in\free(\rho_{1})$. Since $x\neq y$ we obtain
$x\in\free(\rho)$. From
proposition~(\ref{logic:prop:FUAP:mintransformproof:freevar}) it
follows that $x\in\free({\cal M}(\rho))$. Having assumed that ${\cal
M}(\pi)={\cal M}(\rho)$ we obtain $x\in\free({\cal M}(\pi))$ and
finally using
proposition~(\ref{logic:prop:FUAP:mintransformproof:freevar}) once
more, we obtain $x\in\free(\pi)$. This is our desired contradiction
since $\pi=\gen x\pi_{1}$. This completes our induction argument.
\end{proof}

As an immediate consequence of
theorem~(\ref{logic:the:FUAP:mintransfsubcong:kernel}), we now
provide the following theorem which is the counterpart of
theorem~(\ref{logic:the:FOPL:mintransfsubcong:valid}) of
page~\pageref{logic:the:FOPL:mintransfsubcong:valid}. We know from
proposition~(\ref{logic:prop:FUAP:charsubcong:injective:substitution})
that $\alpha$-equivalence is preserved by injective maps. From
proposition~(\ref{logic:prop:FUAP:validsubproof:injective})
injective maps are valid substitutions. Hence, the following theorem
greatly improves on
proposition~(\ref{logic:prop:FUAP:charsubcong:injective:substitution})
by showing that $\alpha$-equivalence is in fact preserved by valid
substitutions, as we would expect.

\begin{theorem}\label{logic:the:FUAP:mintransfsubcong:valid}
Let $V$ and $W$ be sets and $\sigma:V\to W$ be a map. Let~$\sim$ be
the substitution congruence on \pvs\ and ${\bf\Pi}(W)$. Then if
$\sigma$ is valid for $\pi$ and $\rho$:
    \[
    \pi\sim\rho\ \Rightarrow\ \sigma(\pi)\sim\sigma(\rho)
    \]
for all $\pi,\rho\in\pvs$, where $\sigma:\pvs\to{\bf\Pi}(W)$ is also
the substitution mapping.
\end{theorem}
\begin{proof}
We assume that $\pi\sim\rho$ and $\sigma:V\to W$ is valid for $\pi$
and $\rho$. We need to show that $\sigma(\pi)\sim\sigma(\rho)$.
Using theorem~(\ref{logic:the:FUAP:mintransfsubcong:kernel}) it is
sufficient to prove that ${\cal M}\circ\sigma(\pi)={\cal
M}\circ\sigma(\rho)$. Since $\sigma$ is valid for $\pi$ and $\rho$,
using theorem~(\ref{logic:the:FUAP:mintransvalidsub:commute}) of
page~\pageref{logic:the:FUAP:mintransvalidsub:commute} it is
therefore sufficient to prove that $\bar{\sigma}\circ{\cal
M}(\pi)=\bar{\sigma}\circ{\cal M}(\rho)$, which follows immediately
from ${\cal M}(\pi)={\cal M}(\rho)$, itself a consequence of
$\pi\sim\rho$.
\end{proof}

We shall complete this section by providing a counterpart of
proposition~(\ref{logic:prop:FOPL:mintransform:eqivalence}). This
result establishes the equivalence ${\cal M}(\pi)\sim i(\pi)$ where
$i:V\to\bar{V}$ is the inclusion mapping. It is an elementary result
which is not a consequence of
theorem~(\ref{logic:the:FUAP:mintransfsubcong:kernel}) or
theorem~(\ref{logic:the:FUAP:mintransfsubcong:valid}). In fact, its
corresponding
proposition~(\ref{logic:prop:FOPL:mintransform:eqivalence}) is
provided shortly after minimal transforms have been defined. In the
case of proofs, we defined minimal transforms before we had any idea
of substitution congruence on \pvs. So we had to wait. We suspect
the following proposition could be used to design an alternative
proof of theorem~(\ref{logic:the:FUAP:mintransfsubcong:kernel})
which does not involve the local inversion
theorem~(\ref{logic:the:FUAP:localinversion:inversion}) of
page~\pageref{logic:the:FUAP:localinversion:inversion}. From the
equality ${\cal M}(\pi)={\cal M}(\rho)$ we would infer $i(\pi)\sim
i(\rho)$. We would still need to find a way to conclude
$\pi\sim\rho$ from the equivalence $i(\pi)\sim i(\rho)$. Assuming
$V\neq\emptyset$ for the purpose of this discussion, we could define
a left inverse $j:\bar{V}\to V$ of $i:V\to\bar{V}$. We would then
need to extend
proposition~(\ref{logic:prop:FUAP:charsubcong:injective:substitution})
by weakening its assumption and showing that the equivalence
$\pi\sim\rho$ is preserved simply by assuming the map $\sigma:V\to
W$ is injective on $\var(\pi)\cup\var(\rho)$, rather than injective
on $V$.

\begin{prop}\label{logic:prop:FUAP:mintransfsubcong:equivalence}
Let $V$ be a set and $i:V\to\bar{V}$ be the inclusion map. We
denote~$\sim$ the substitution congruence on ${\bf\Pi}(\bar{V})$.
Then for all $\pi\in\pvs$ \,:
    \[
    {\cal M}(\pi)\sim\, i(\pi)
    \]
\end{prop}
\begin{proof}
We shall prove the equivalence ${\cal M}(\pi)\sim\, i(\pi)$ by a
structural induction argument, using
theorem~(\ref{logic:the:proof:induction}) of
page~\pageref{logic:the:proof:induction}. First we assume that
$\pi=\phi$ for some $\phi\in\pv$. Then the equivalence follows from
proposition~(\ref{logic:prop:FOPL:mintransform:eqivalence}). Next we
assume that $\pi=\axi\phi$ for some $\phi\in\pv$. Using
proposition~(\ref{logic:prop:FOPL:mintransform:eqivalence}) once
more:
    \[
    {\cal M}(\pi)=\axi{\cal M}(\phi)\sim\axi i(\phi)=i(\pi)
    \]
We now assume that $\pi=\pi_{1}\pon\pi_{2}$ where
$\pi_{1},\pi_{2}\in\pvs$ satisfy the equivalence:
    \begin{eqnarray*}
    {\cal M}(\pi)&=&{\cal M}(\pi_{1}\pon\pi_{2})\\
    &=&{\cal M}(\pi_{1})\pon\,{\cal M}(\pi_{2})\\
    &\sim&i(\pi_{1})\pon\,i(\pi_{2})\\
    &=&i(\pi_{1}\pon\pi_{2})\\
    &=&i(\pi)
    \end{eqnarray*}
Finally we assume that $\pi=\gen x\pi_{1}$ where $x\in V$ and
$\pi_{1}\in\pvs$ satisfies the equivalence. In this case we have the
following equalities:
    \begin{eqnarray*}
    {\cal M}(\pi)&=&{\cal M}(\gen x\pi_{1})\\
    n=\min\{k:[k/x]\mbox{ valid for }{\cal M}(\pi_{1})\}\
    \rightarrow
    &=&\gen n{\cal M}(\pi_{1})[n/x]\\
    &=&[n/x](\gen x{\cal M}(\pi_{1}))\\
    \mbox{A: to be proved}\ \rightarrow
    &\sim&\gen x{\cal M}(\pi_{1})\\
    &\sim&\gen x\, i(\pi_{1})\\
    &=&\gen i(x)\, i(\pi_{1})\\
    &=&i(\gen x\pi_{1})\\
    &=&i(\pi)
    \end{eqnarray*}
So it remains to show that $[n/x](\gen x{\cal M}(\pi_{1}))\sim\gen
x{\cal M}(\pi_{1})$. Hence, from
proposition~(\ref{logic:prop:FUAP:subcong:admissible:subcong}) it is
sufficient to prove that $[n/x]$ is an admissible substitution for
$\gen x{\cal M}(\pi_{1})$. We already know that $[n/x]$ is valid for
${\cal M}(\pi_{1})$. Using
proposition~(\ref{logic:prop:FUAP:validsubproof:recursion:gen}),
given $u\in\free(\gen x{\cal M}(\pi_{1}))$, in order to prove that
$[n/x]$ is valid for $\gen x{\cal M}(\pi_{1})$ we need to show that
$[n/x](u)\neq[n/x](x)$. So it is sufficient to show that
$[n/x](u)\neq n$. Since $u\in\free(\gen x{\cal M}(\pi_{1}))$, in
particular we have $u\neq x$. Hence, we have to show that $u\neq n$.
Using proposition~(\ref{logic:prop:FUAP:mintransformproof:freevar})
we have:
    \[
    u\in\free(\gen x{\cal M}(\pi_{1}))\subseteq\free({\cal M}(\pi_{1}))\subseteq V
    \]
and consequently from $V\cap\N=\emptyset$ we conclude that $u\neq
n$. In order to show that  $[n/x]$ is an admissible substitution for
$\gen x{\cal M}(\pi_{1})$ it remains to prove that $[n/x](u)=u$ for
all $u\in\free(\gen x{\cal M}(\pi_{1}))$, which follows from $u\neq
x$.
\end{proof}
