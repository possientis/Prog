It feels somewhat artificial to define the set of variables
$\var(\pi)$ of a proof $\pi$. All variables involved are very
different in nature and it seems very little benefit can be derived
from aggregating them in one big set. Among these, there are the
free or bound variables coming from the set of hypothesis
$\hyp(\pi)$, and those coming from the set of axioms $\ax(\pi)$.
There are also the variables involved in the use of generalization.
These five groups of variables are not mutually exclusive, unless
when a proof is totally clean and no specific variable of
$\spec(\pi)$ can be used for generalization. So why define the set
$\var(\pi)$? Experience has shown us that given a formula
$\phi\in\pv$, the set $\var(\phi)$ was very useful in allowing us to
strengthen a few results on substitutions. If $\sigma:V\to W$ is an
injective map, then many things can be said about $\sigma:\pv\to{\bf
P}(W)$ or the formula $\sigma(\phi)$. However, we usually do not
need to know that $\sigma$ is injective to derive our conclusion. It
is usually sufficient to assume the injectivity of the restriction
$\sigma_{|\var(\phi)}$. This distinction is in fact crucial: given
an injective map $\tau:V\to W$, we often need to consider a
left-inverse of $\tau$, namely a map $\sigma:W\to V$ such that
$\sigma\circ\tau(x)=x$ for all $x\in V$. Such a left-inverse is
rarely injective, but the restriction $\sigma_{|\tau(V)}$ is an
injective map. The set $\var(\phi)$ together with $\free(\phi)$ and
$\bound(\phi)$ were also crucial notions when dealing with valid
substitutions and minimal transforms. As we shall soon discover, the
same sort of analysis can be carried out for proofs, leading up to
essential substitutions and their existence in
theorem~(\ref{logic:the:FUAP:esssubst:existence}) of
page~\pageref{logic:the:FUAP:esssubst:existence}. So $\var(\phi)$ is
a useful notion and so is $\var(\pi)$. \index{variable@Variables of
proof}\index{variable@$\var(\pi)$ : set of variables of $\pi$}
\begin{defin}\label{logic:def:FUAP:variable:variable}
Let $V$ be a set. The map $\var:\pvs\to {\cal P}(V)$ defined by the
following structural recursion is called {\em variable mapping on
\pvs}:
 \begin{equation}\label{logic:eqn:FUAP:variable:variable:1}
    \forall\pi\in\pvs\ ,\ \var(\pi)=\left\{
                    \begin{array}{lcl}
                    \var(\phi)&\mbox{\ if\ }&\pi=\phi\in\pv\\
                    \var(\phi)&\mbox{\ if\ }&\pi=\axi\phi\\
                    \var(\pi_{1})\cup\var(\pi_{2}) &\mbox{\ if\ }&\pi=\pi_{1}\pon\pi_{2}\\
                    \{x\}\cup\var(\pi_{1})&\mbox{\ if\ }&\pi=\gen x\pi_{1}
                    \end{array}\right.
    \end{equation}
We say that $x\in V$ is a {\em variable} of $\pi\in\pvs$ \ifand\
$x\in\var(\pi)$.
\end{defin}
Given a formula $\phi\in\pv$ the notation $\var(\phi)$ is
potentially ambiguous. Since $\pv\subseteq\pvs$, it may refer to the
usual $\var(\phi)$ of definition~(\ref{logic:def:variable}), or to
the set $\var(\pi)$ where $\pi=\phi$ of
definition~(\ref{logic:def:FUAP:variable:variable}). Luckily, the
two notions coincide.

\begin{prop}\label{logic:prop:FUAP:variable:recursion}
The structural recursion of {\em
definition~(\ref{logic:def:FUAP:variable:variable})} is legitimate.
\end{prop}
\begin{proof}
We need to show the existence and uniqueness of the map
$\var:\pvs\to{\cal P}(V)$ satisfying the four conditions of
equation~(\ref{logic:eqn:FUAP:variable:variable:1}). This follows
from an application of
theorem~(\ref{logic:the:structural:recursion}) of
page~\pageref{logic:the:structural:recursion} with $X=\pvs$,
$X_{0}=\pv$ and $A={\cal P}(V)$ where $g_{0}:X_{0}\to A$ is defined
as $g_{0}(\phi)=\var(\phi)$. Furthermore, given $\phi\in\pv$ we take
$h(\axi\phi):A^{0}\to A$ defined $h(\axi\phi)(0)=\var(\phi)$. We
take $h(\pon):A^{2}\to A$ defined by $h(\pon)(A_{0},A_{1})=A_{0}\cup
A_{1}$ and $h(\gen x):A^{1}\to A$ defined by $h(\gen
x)(A_{0})=\{x\}\cup A_{0}$.
\end{proof}

Many elementary properties of $\var(\pi)$ resemble those of
$\var(\phi)$. We have not attempted to design the right level of
abstraction to avoid what may appear as tedious repetition. The
following is the counterpart of
proposition~(\ref{logic:prop:var:is:finite})\,:

\begin{prop}\label{logic:prop:FUAP:variable:finite}
Let $V$ be a set and $\pi\in\pvs$. Then $\var(\pi)$ is finite.
\end{prop}
\begin{proof}
This follows from a structural induction argument using
theorem~(\ref{logic:the:proof:induction}) of
page~\pageref{logic:the:proof:induction}. If $\pi=\phi\in\pv$ then
$\var(\pi)=\var(\phi)$ is finite by virtue of
proposition~(\ref{logic:prop:var:is:finite}). If $\pi=\axi\phi$ then
$\var(\pi)=\var(\phi)$ is also finite. If $\pi=\pi_{1}\pon\pi_{2}$
then we have $\var(\pi)=\var(\pi_{1})\cup\var(\pi_{2})$ which is
finite. Finally  if $\pi=\gen x\pi_{1}$ then we have
$\var(\pi)=\{x\}\cup\var(\pi_{1})$ which is also finite if
$\var(\pi_{1})$ is itself finite.
\end{proof}

The map $\var:\pvs\to{\cal P}(V)$ is increasing with respect to the
standard inclusion on ${\cal P}(V)$. In other words, the variables
of a sub-proof are also variables of the proof itself. The following
is the counterpart of
proposition~(\ref{logic:prop:FOBL:variable:subformula})\,:
\index{subformula@$\pi\preceq\rho$ : $\pi$ is a sub-proof of $\rho$}
\begin{prop}\label{logic:prop:FUAP:variable:subformula}
Let $V$ be a set and $\rho,\pi\in\pvs$. Then we have:
    \[
    \rho\preceq\pi\ \Rightarrow\ \var(\rho)\subseteq\var(\pi)
    \]
\end{prop}
\begin{proof}
This is a simple application of
proposition~(\ref{logic:prop:UA:subformula:non:decreasing}) to
$\var:X\to A$ where $X=\pvs$ and $A={\cal P}(V)$ where the preorder
$\leq$ on $A$ is the usual inclusion $\subseteq$. We simply need to
check that given $\pi_{1},\pi_{2}\in\pvs$ and $x\in V$ we have the
inclusions $\var(\pi_{1})\subseteq\var(\pi_{1}\pon\pi_{2})$,
$\var(\pi_{2})\subseteq\var(\pi_{1}\pon\pi_{2})$ and
$\var(\pi_{1})\subseteq\var(\gen x\pi_{1})$ which follow immediately
from the recursive
definition~(\ref{logic:def:FUAP:variable:variable}).
\end{proof}

Given a map $\sigma:V\to W$ and $\pi\in\pvs$, the variables of the
proof $\sigma(\pi)$ are the images by $\sigma$ of the variables of
$\pi$. This is the counterpart of
proposition~(\ref{logic:prop:var:of:substitution})\,:

\begin{prop}\label{logic:prop:FUAP:variable:substitution}
Let $V$ and $W$ be sets and $\sigma:V\to W$ be a map. Then:
    \[
    \forall\pi\in\pvs\ ,\ \var(\sigma(\pi))=\sigma(\var(\pi))
    \]
where $\sigma:\pvs\to{\bf\Pi}(W)$ also denotes the associated proof
substitution mapping.
\end{prop}
\begin{proof}
Given $\pi\in\pvs$, we need to show that
$\var(\sigma(\pi))=\sigma(\var(\pi))$. We shall do so by structural
induction using theorem~(\ref{logic:the:proof:induction}) of
page~\pageref{logic:the:proof:induction}. First we assume that
$\pi=\phi$ for some $\phi\in\pv$. Then we have the equalities:
    \begin{eqnarray*}
    \var(\sigma(\pi))&=&\var(\sigma(\phi))\\
    \mbox{prop.~(\ref{logic:prop:var:of:substitution})}\ \rightarrow
    &=&\sigma(\var(\phi))\\
    &=&\sigma(\var(\pi))\\
    \end{eqnarray*}
Next we assume that $\pi=\axi\phi$ for some $\phi\in\pv$. Then we
have:
    \begin{eqnarray*}
    \var(\sigma(\pi))&=&\var(\sigma(\axi\phi))\\
    &=&\var(\axi\sigma(\phi))\\
    &=&\var(\sigma(\phi))\\
    \mbox{prop.~(\ref{logic:prop:var:of:substitution})}\ \rightarrow
    &=&\sigma(\var(\phi))\\
    &=&\sigma(\var(\axi\phi))\\
    &=&\sigma(\var(\pi))\\
    \end{eqnarray*}
Next we check that the property is true for $\pi=\pi_{1}\pon\pi_{2}$
if it is true for $\pi_{1},\pi_{2}$:
    \begin{eqnarray*}
    \var(\sigma(\pi))&=&\var(\sigma(\pi_{1}\pon\pi_{2}))\\
    &=&\var(\sigma(\pi_{1})\pon\,\sigma(\pi_{2}))\\
    &=&\var(\sigma(\pi_{1}))\cup\var(\sigma(\pi_{2}))\\
    &=&\sigma(\var(\pi_{1}))\cup\sigma(\var(\pi_{2}))\\
    &=&\sigma(\,\var(\pi_{1})\cup\var(\pi_{2})\,)\\
    &=&\sigma(\var(\pi_{1}\pon\pi_{2}))\\
    &=&\sigma(\var(\pi))\\
    \end{eqnarray*}
Finally we check that the property is true for $\pi=\gen x\pi_{1}$
if it is true for $\pi_{1}$:
    \begin{eqnarray*}
    \var(\sigma(\pi))&=&\var(\sigma(\gen x\pi_{1}))\\
    &=&\var(\,\gen\sigma(x)\sigma(\pi_{1})\,)\\
    &=&\{\sigma(x)\}\cup\var(\sigma(\pi_{1}))\\
    &=&\{\sigma(x)\}\cup\sigma(\var(\pi_{1}))\\
    &=&\sigma(\,\{x\}\cup\var(\pi_{1})\,)\\
    &=&\sigma(\var(\gen x\pi_{1}))\\
    &=&\sigma(\var(\pi))\\
    \end{eqnarray*}
\end{proof}


In order to have the equality $\sigma(\pi)=\tau(\pi)$ is it
sufficient that $\sigma,\tau:V\to W$ coincide on $\var(\pi)$, which
is pretty natural. What is less obvious is the fact that the
converse is also true. The following is the counterpart of
proposition~(\ref{logic:prop:substitution:support})\,:

\begin{prop}\label{logic:prop:FUAP:variable:support}
Let $V$ and $W$ be sets and $\sigma,\tau:V\to W$ be two maps. Then:
    \[
    \sigma_{|\var(\pi)}=\tau_{|\var(\pi)}\ \ \Leftrightarrow\
    \ \sigma(\pi)=\tau(\pi)
    \]
for all $\pi\in\pvs$, where $\sigma,\tau:\pvs\to{\bf\Pi}(W)$ are
also the proof substitutions.
\end{prop}
\begin{proof}
First we prove $\Rightarrow$\,: We shall do so with an induction
argument, using theorem~(\ref{logic:the:proof:induction}) of
page~\pageref{logic:the:proof:induction}. First we assume that
$\pi=\phi$ for some $\phi\in\pv$. Then we have
$\var(\pi)=\var(\phi)$ so we assume that $\sigma$ and $\tau$
coincide on $\var(\phi)$. Using
proposition~(\ref{logic:prop:substitution:support}) we obtain
$\sigma(\phi)=\tau(\phi)$ which is $\sigma(\pi)=\tau(\pi)$. Next we
assume that $\pi=\axi\phi$ for some $\phi\in\pv$. Then again we have
$\var(\pi)=\var(\phi)$ so we assume that $\sigma$ and $\tau$
coincide on $\var(\phi)$. Using
proposition~(\ref{logic:prop:substitution:support}) once more we
obtain $\sigma(\phi)=\tau(\phi)$. It follows that
$\sigma(\pi)=\axi\sigma(\phi)=\axi\tau(\phi)=\tau(\pi)$. So we now
assume that $\pi=\pi_{1}\pon\pi_{2}$ where $\pi_{1},\pi_{2}\in\pvs$
satisfy our property. We need to show the same is true of $\pi$. So
we assume that $\sigma$ and $\tau$ coincide on $\var(\pi)$. We need
to show that $\sigma(\pi)=\tau(\pi)$. However since
$\var(\pi)=\var(\pi_{1})\cup\var(\pi_{2})$ we see that $\sigma$ and
$\tau$ coincide on $\var(\pi_{1})$ and $\var(\pi_{2})$, and it
follows from our induction hypothesis that
$\sigma(\pi_{1})=\tau(\pi_{1})$ and $\sigma(\pi_{2})=\tau(\pi_{2})$.
Hence:
    \begin{eqnarray*}
    \sigma(\pi)&=&\sigma(\pi_{1}\pon\pi_{2})\\
    &=&\sigma(\pi_{1})\pon\,\sigma(\pi_{2})\\
    &=&\tau(\pi_{1})\pon\,\tau(\pi_{2})\\
    &=&\tau(\pi_{1}\pon\pi_{2})\\
    &=&\tau(\pi)
    \end{eqnarray*}
Finally, we assume that $\pi=\gen x\pi_{1}$ where $x\in V$ and
$\pi_{1}\in\pvs$ satisfies our property. We need to show the same is
true of $\pi$. So we assume that $\sigma$ and $\tau$ coincide on
$\var(\pi)$. We need to show that $\sigma(\pi)=\tau(\pi)$. However
since $\var(\pi)=\{x\}\cup\var(\pi_{1})$ we see that $\sigma$ and
$\tau$ coincide on $\var(\pi_{1})$, and it follows from our
induction hypothesis that $\sigma(\pi_{1})=\tau(\pi_{1})$. Hence:
    \begin{eqnarray*}
    \sigma(\pi)&=&\sigma(\gen x\pi_{1})\\
    &=&\gen\sigma(x)\sigma(\pi_{1})\\
    &=&\gen\sigma(x)\tau(\pi_{1})\\
    x\in\var(\pi)\ \rightarrow&=&\gen\tau(x)\tau(\pi_{1})\\
    &=&\tau(\gen x\pi_{1})\\
    &=&\tau(\pi)
    \end{eqnarray*}
We now show $\Leftarrow$\,: we shall do so with a structural
induction argument, using theorem~(\ref{logic:the:proof:induction})
of page~\pageref{logic:the:proof:induction}. First we assume that
$\pi=\phi$ for some $\phi\in\pv$. Then from $\sigma(\pi)=\tau(\pi)$
we obtain $\sigma(\phi)=\tau(\phi)$ and from
proposition~(\ref{logic:prop:substitution:support}) it follows that
$\sigma$ and $\tau$ coincide on $\var(\phi)=\var(\pi)$. Next we
assume that $\pi=\axi\phi$ for some $\phi\in\pv$. Suppose
$\sigma(\pi)=\tau(\pi)$. Then $\axi\sigma(\phi)=\axi\tau(\phi)$ and
consequently from
proposition~(\ref{logic:prop:FUAP:proof:axi:injective}) we have
$\sigma(\phi)=\tau(\phi)$. Using
proposition~(\ref{logic:prop:substitution:support}) once more it
follows that $\sigma$ and $\tau$ coincide on $\var(\phi)=\var(\pi)$.
We now assume that $\pi=\pi_{1}\pon\pi_{2}$ where $\pi_{1},\pi_{2}$
are proofs which satisfy our implication. We need to show the same
is true of $\pi$. So we assume that $\sigma(\pi)=\tau(\pi)$. We need
to show that $\sigma$ and $\tau$ coincide on
$\var(\pi)=\var(\pi_{1})\cup\var(\pi_{2})$. However, from
$\sigma(\pi)=\tau(\pi)$ we obtain
$\sigma(\pi_{1})\pon\,\sigma(\pi_{2})=\tau(\pi_{1})\pon\,\tau(\pi_{2})$
and consequently using
theorem~(\ref{logic:the:unique:representation}) of
page~\pageref{logic:the:unique:representation} we have
$\sigma(\pi_{1})=\tau(\pi_{1})$ and $\sigma(\pi_{2})=\tau(\pi_{2})$.
Having assumed that $\pi_{1}$ and $\pi_{2}$ satisfy our implication,
it follows that $\sigma$ and $\tau$ coincide on $\var(\pi_{1})$ and
also on $\var(\pi_{2})$. So they coincide on $\var(\pi)$ as
requested. So we now assume that $\pi=\gen x\pi_{1}$ where $x\in V$
and $\pi_{1}\in\pvs$ is a proof which satisfies our implication. We
need to show the same is true of $\pi$. So we assume that
$\sigma(\pi)=\tau(\pi)$. We need to show that $\sigma$ and $\tau$
coincide on $\var(\pi)=\{x\}\cup\var(\pi_{1})$. However, from
$\sigma(\pi)=\tau(\pi)$ we obtain
$\gen\sigma(x)\sigma(\pi_{1})=\gen\tau(x)\tau(\pi_{1})$ and
consequently using theorem~(\ref{logic:the:unique:representation})
of page~\pageref{logic:the:unique:representation} we have
$\sigma(\pi_{1})=\tau(\pi_{1})$ and $\sigma(x)=\tau(x)$. Having
assumed that $\pi_{1}$ satisfies our implication, it follows that
$\sigma$ and $\tau$ coincide on $\var(\pi_{1})$. Since
$\sigma(x)=\tau(x)$, they also coincide on
$\var(\pi)=\{x\}\cup\var(\pi_{1})$ as requested.
\end{proof}

Given a proof $\pi\in\pvs$, any variable which belongs to the
conclusion $\val(\rho)$ of a sub-proof $\rho\preceq\pi$ is of course
a variable of $\pi$. Note that the converse is not true in general,
unless $\pi$ is totally clean: if $\pi=\axi\phi$ and
$\phi\not\in\av$ then we have $\val(\pi)=\bot\to\bot$ and any
sub-proof of $\pi$ is $\pi$ itself. Hence we have
$\cup\{\,\var(\val(\rho))\ :\
    \rho\preceq\pi\,\}=\emptyset$, while $\var(\pi)=\var(\phi)$ may not be
empty. We shall deal with the case when $\pi$ is totally clean later
in
proposition~(\ref{logic:prop:FUAP:varvalmod:conclusions:sub:proof:clean}).

\begin{prop}\label{logic:prop:FUAP:variable:conclusions:sub:proof}
Let $V$ be a set and $\pi\in\pvs$. Then we have:
    \begin{equation}\label{logic:eqn:FUAP:variable:conclusions:1}
    \cup\{\,\var(\val(\rho))\ :\
    \rho\preceq\pi\,\}\subseteq\var(\pi)
    \end{equation}
i.e. the variables of conclusions of sub-proofs of $\pi$ are
variables of $\pi$.
\end{prop}
\begin{proof}
Using proposition~(\ref{logic:prop:FUAP:variable:subformula}) we
have $\var(\rho)\subseteq\var(\pi)$ for all $\rho\preceq\pi$. It is
therefore sufficient prove the inclusion
$\var(\val(\pi))\subseteq\var(\pi)$ for all $\pi\in\pvs$ which we
shall do with a structural induction argument, using
theorem~(\ref{logic:the:proof:induction}) of
page~\pageref{logic:the:proof:induction}. First we assume that
$\pi=\phi$ for some $\phi\in\pv$. Then the inclusion follows
immediately from $\val(\phi)=\phi$. Next we assume that
$\pi=\axi\phi$ for some $\phi\in\pv$. The inclusion is clear in the
case when $\val(\pi)=\bot\to\bot$. So we may assume that
$\val(\pi)\neq\bot\to\bot$ in which case $\phi$ must be an axiom of
first order logic, i.e. $\phi\in\av$ and $\val(\pi)=\phi$. The
inclusion follows from $\var(\pi)=\var(\phi)$. So we now assume that
$\pi=\pi_{1}\pon\pi_{2}$ where $\pi_{1},\pi_{2}\in\pvs$ are proofs
satisfying our inclusion. We need to show the same is true of $\pi$.
This is clearly the case if $\val(\pi)=\bot\to\bot$. So we may
assume $\val(\pi)\neq\bot\to\bot$ in which case we must have
$\val(\pi_{2})=\val(\pi_{1})\to\val(\pi)$. Hence we see that:
    \begin{eqnarray*}
    \var(\val(\pi))&\subseteq&\var(\val(\pi_{1}))\cup\var(\val(\pi))\\
    &=&\var(\,\val(\pi_{1})\to\val(\pi)\,)\\
    &=&\var(\val(\pi_{2}))\\
    &\subseteq&\var(\pi_{2})\\
    &\subseteq&\var(\pi_{1})\cup\var(\pi_{2})\\
    &=&\var(\pi_{1}\pon\pi_{2})\\
    &=&\var(\pi)\\
    \end{eqnarray*}
So we now assume that $\pi=\gen x\pi_{1}$ where $x\in V$ and
$\pi_{1}\in\pvs$ satisfies our inclusion. We need to show the same
is true of $\pi$. This is clearly the case if
$\val(\pi)=\bot\to\bot$. So we may assume that
$\val(\pi)\neq\bot\to\bot$ in which case we must have
$x\not\in\spec(\pi_{1})$ and $\val(\pi)=\forall x\val(\pi_{1})$.
Hence we see that:
    \begin{eqnarray*}
    \var(\val(\pi))&=&\var(\,\forall x\val(\pi_{1})\,)\\
    &=&\{x\}\cup\var(\val(\pi_{1}))\\
    &\subseteq&\{x\}\cup\var(\pi_{1})\\
    &=&\var(\gen x\pi_{1})\\
    &=&\var(\pi)\\
    \end{eqnarray*}
\end{proof}
