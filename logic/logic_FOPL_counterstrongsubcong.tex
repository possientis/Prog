A fair amount of work has been devoted to the study of the strong
substitution congruence, culminating with its characterization in
the form of theorem~(\ref{logic:the:strong:sub:congruence:charac}).
Unfortunately, most of this work was done vain. As we shall now
discover, the strong substitution congruence is not the appropriate
notion to study. It may give us insight and will certainly help us
in the forthcoming developments, it may be commonly referred to in
the literature, but it is marred by a major flaw. The problem is as
follows: when $V=\{x,y\}$ with $x\neq y$, i.e. when $V$ has only two
elements, the formulas $\phi=\forall x\forall y (x\in y)$ and
$\psi=\forall y\forall x(y\in x)$ are not equivalent. This is highly
disappointing. If there is one thing we would expect of a {\em
substitution congruence}, it is certainly to be such that $\phi$ and
$\psi$ be equivalent. This leaves us with a painful alternative: we
either give up on the possibility of finite sets of variables $V$,
or we abandon the notion of strong substitution congruence and all
the work that has gone along with it. As already hinted, we are not
prepared to accept that nothing interesting can be said with finite
sets $V$. There must be an appropriate notion of {\em substitution
congruence} to be defined on \pv, and we shall therefore continue
our search for it. But before we resume our quest, we shall first
prove that the problem truly exists:

\begin{prop}\label{logic:prop:counter:strong:1}
Let $\sim$ be the strong substitution congruence on \pv\ where
$V=\{x,y\}$ and $x\neq y$. Then, we have:
    \[
    \forall x\forall y\,(x\in y)\not\sim\forall y\forall x\,(y\in x)
    \]
\end{prop}
\begin{proof}
Define $\phi_{1}=\forall y(x\in y)$ and $\psi_{1}=\forall x(y\in
x)$. We need to show that $\forall x\phi_{1}\not\sim\forall
y\psi_{1}$. Suppose to the contrary that $\forall
x\phi_{1}\sim\forall y\psi_{1}$. We shall derive a contradiction.
Since $x\neq y$, from
theorem~(\ref{logic:the:strong:sub:congruence:charac}) of
page~\pageref{logic:the:strong:sub:congruence:charac} there exists
$\theta\in\pv$ such that $\phi_{1}\sim\theta$,
$\psi_{1}\sim\theta[y/x]$ and $y\not\in\var(\theta)$. From
$\phi_{1}\sim\theta$ we see that $\forall y(x\in y)\sim\theta$, and
applying theorem~(\ref{logic:the:strong:sub:congruence:charac}) once
more, since $V=\{x,y\}$ we see that $\theta$ must be of the form
$\theta=\forall x\theta_{1}$ or $\theta=\forall y\theta_{1}$.
However the case $\theta=\forall y\theta_{1}$ is impossible since
$y\not\in\var(\theta)$. Suppose now that $\theta=\forall
x\theta_{1}$ for some $\theta_{1}\in\pv$. Then
$x\not\in\free(\theta)=\free(\theta_{1})\setminus\{x\}$. However we
have $\forall y(x\in y)\sim\theta$ from
proposition~(\ref{logic:prop:strong:freevar}) we have
$\{x\}=\free(\,\forall y(x\in y)\,)=\free(\theta)$. This is our
desired contradiction.
\end{proof}

As we have just seen, the strong substitution congruence is
inadequate when $V$ has two elements. In fact, it is also inadequate
when $V$ has three elements as the following proposition shows. We
do not intend to spend too much time proving negative results, but
we certainly believe the strong substitution congruence on \pv\ will
fail whenever $V$ is a finite set.

\begin{prop}\label{logic:prop:counter:strong:2}
Let $\sim$ be the strong substitution congruence on \pv\ where
$V=\{x,y,z\}$ and $x\neq y$, $y\neq z$ and $x\neq z$. Then, we have:
    \[
    \forall x\forall y\forall z\,[(x\in y)\to(y\in z)]\not\sim\forall
    y\forall z\forall x\,[(y\in z)\to(z\in x)]
    \]
\end{prop}
\begin{proof}
Define $\phi_{1}=\forall y\forall z\,[(x\in y)\to(y\in z)]$ and
$\psi_{1}=\forall z\forall x\,[(y\in z)\to(z\in x)]$. We need to
show that $\forall x\phi_{1}\not\sim\forall y\psi_{1}$. Suppose to
the contrary that $\forall x\phi_{1}\sim\forall y\psi_{1}$. We shall
derive a contradiction. Since $x\neq y$, from
theorem~(\ref{logic:the:strong:sub:congruence:charac}) of
page~\pageref{logic:the:strong:sub:congruence:charac} there exists
$\theta\in\pv$ such that $\phi_{1}\sim\theta$,
$\psi_{1}\sim\theta[y/x]$ and $y\not\in\var(\theta)$. From
$\phi_{1}\sim\theta$ we see that $\forall y\forall z\,[(x\in
y)\to(y\in z)]\sim\theta$, and applying
theorem~(\ref{logic:the:strong:sub:congruence:charac}) once more,
since $V=\{x,y,z\}$ we see that $\theta$ must be of the form
$\theta=\forall x\,\theta_{1}$, $\theta=\forall y\,\theta_{1}$ or
$\theta=\forall z\,\theta_{1}$. However the case $\theta=\forall
y\,\theta_{1}$ is impossible since $y\not\in\var(\theta)$.
Furthermore, the case $\theta=\forall x\,\theta_{1}$ is also
impossible since this would imply that $x\not\in\free(\theta)$,
while from $\forall y\forall z\,[(x\in y)\to(y\in z)]\sim\theta$ and
proposition~(\ref{logic:prop:strong:freevar}):
    \[
    \{x\}=\free(\,\forall y\forall z\,[(x\in y)\to(y\in z)]\,)=\free(\theta)
    \]
It follows that $\theta=\forall z\,\theta_{1}$ for some
$\theta_{1}\in\pv$. Hence we see that:
    \[
    \forall y\forall z\,[(x\in y)\to(y\in z)]\sim\forall z\,\theta_{1}
    \]
Applying theorem~(\ref{logic:the:strong:sub:congruence:charac}) once
more, there exists $\eta\in\pv$ such that  we have $\forall
z\,[(x\in y)\to(y\in z)]\sim\eta$, $\theta_{1}\sim\eta[z/y]$ and
$z\not\in\var(\eta)$. From the strong equivalence $\forall z\,[(x\in
y)\to(y\in z)]\sim\eta$ and yet another application of
theorem~(\ref{logic:the:strong:sub:congruence:charac}), since
$V=\{x,y,z\}$ we see that $\eta$ must be of the form $\eta=\forall
x\,\eta_{1}$, $\eta=\forall y\,\eta_{1}$ or $\eta=\forall
z\,\eta_{1}$. However, the case $\eta=\forall z\,\eta_{1}$ is
impossible since $z\not\in\var(\eta)$. Suppose now that
$\eta=\forall x\,\eta_{1}$. Then $x\not\in\free(\eta)$,
contradicting the equality:
    \[
    \{x,y\}=\free(\,\forall z\,[(x\in y)\to(y\in z)]\,)=\free(\eta)
    \]
obtained from $\forall z\,[(x\in y)\to(y\in z)]\sim\eta$ and
proposition~(\ref{logic:prop:strong:freevar}). Since
$y\in\free(\eta)$ we conclude similarly that $\eta=\forall
y\,\eta_{1}$ is equally impossible.
\end{proof}
