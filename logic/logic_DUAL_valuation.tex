As far as we can tell, every textbook in mathematical logic
introduces semantics via model theory. We shall not do so. We
believe there are many good reasons to postpone the study of model
theory to a later stage. So here they are: first of all, the
approach we shall take will turn out to be equivalent to the
standard approach anyway. This is hardly a reason to be different,
but it may alleviate some of the irritation from the reader, and
generate some degree of acceptance. We are interested in the concept
of truth: so we shall define the notion of {\em valuation} which are
maps $v:\pv\to 2$ with the obvious properties and in particular the
{\em soundness} property $\vdash\phi\ \Rightarrow\ v(\phi)=1$. As it
turns out, every model and variables assignment will give rise to a
{\em valuation}. This is hardly surprising and is commonly known
under the slogan {\em first order logic is sound}. What is less
obvious and somehow linked to the completeness theorem, is the fact
that every valuation arises from a model and variables assignment.
So although we shall consider the notion of semantic entailment in
terms of valuations and not models, since {\em every valuation has a
model} the two approaches are equivalent. This is our first point.
Another reason for postponing the introduction of model theory is
this: the set of valuations $v:\pv\to 2$ which we shall denote \pvd\
and call the {\em dual space} is of enormous interest in its own
right. Rushing into model theory is a wasted opportunity as we fail
to pay due attention to what we believe is a fundamental object. We
are pretty sure this object has connections with existing works such
as P.T. Johnstone~\cite{JohnstoneStone}, Henrik
Forssell~\cite{Forssell} and Murdoch J. Gabbay~\cite{GabbayDuality}
which we have yet to investigate. Furthermore, the dual space \pvd\
is a set and not a proper class. Defining semantic entailments in
terms of a class of models seems to be madness when we can simply
and equivalently do so in terms of a set. Another reason for
postponing model theory is {\em decoupling}\,: most of us have
experience in coding where our projects should adopt a modular
design with decoupled functionality. The same is true in
mathematics: the proof of G\"odel's completeness theorem involves a
fair amount of work on maximal consistent sets which have nothing to
do with models. In fact as will appear later, a maximal consistent
set is nothing but a valuation. In other words, the dual space \pvd\
is exactly the set of maximal consistent sets. The crunch of the
completeness theorem is to show that {\em every valuation has a
model}, and this is the way we intend to present it. Finally, we
feel there is another good reason to define {\em semantic
entailment} in terms of valuations rather than models: it offers a
scope for further generalization. As pointed out by W.J. Blok and D.
Pigozzi~\cite{BlokPigozzi}, the {\em Lindenbaum-Tarski process} is
somehow general: we start with a consequence relation $\vdash$. We
obtain a congruence from it, which in turn gives us a quotient
algebra. The consequence relation $\vdash$ also gives us a dual
space \pvd. In fact as will appear later, this dual space can be
defined directly in terms of  the congruence, without reference to
the consequence relation, and the quotient algebra is isomorphic to
an algebra of subsets of the dual. We feel we are only scratching
the surface and there is a lot more to the story. This story belongs
to {\em abstract algebraic logic} and has little to do with model
theory as far as we can tell. It is a story which is most likely
expounded in the references~\cite{JohnstoneStone}, \cite{Forssell}
and \cite{GabbayDuality} which we hope to investigate further in due
course.

\index{valuation@Valuation, dual space \pvd}\index{dual@Dual space
\pvd, valuation}\index{v@$v:\pv\to 2$ : valuation}\index{pvd@$\pvd$
: dual space}
\begin{defin}\label{logic:def:FOPL:semantics:valuation}
Let $V$ be a set. A map $v:\pv\to 2$ is called a {\em valuation} on
\pv\ \ifand\ it satisfies the following properties. Given
$\phi_{1},\phi_{2},\phi\in\pv$\,:
    \begin{eqnarray*}
    (i)&&v(\bot)=0\\
    (ii)&&v(\phi_{1}\to\phi_{2})=v(\phi_{1})\to v(\phi_{2})\\
    (iii)&&\vdash\phi\ \Rightarrow\ v(\phi)=1
    \end{eqnarray*}
The set of valuations on \pv\ is called the {\em dual space} and
denoted \pvd.
\end{defin}

Looking at
definition~(\ref{logic:def:FOPL:propcong:prop:valuation}), a {\em
valuation} is therefore a propositional valuation which satisfies
the implication $\vdash\phi\ \Rightarrow\ v(\phi)=1$. We could call
this the {\em soundness} property. So a {\em valuation} is a
propositional valuation which is {\em sound}. Note that we are not
yet in a position to say whether there exists any valuation at all.
We do not know whether our deductive system is consistent. If the
sequent $\vdash\bot$ were to be true, then the dual space \pvd\
would be the empty set. Luckily this will not be the case, but we
shall need to wait for some model theory and the soundness
theorem~(\ref{logic:the:FOPL:soundness:soundness}) of
page~\pageref{logic:the:FOPL:soundness:soundness} to be sure.

So the dual space \pvd\ is not empty. But how many elements does it
have? We have no idea at this stage. Let us assume for now that set
theory can be coded in the language of \pv. Does there exist a
valuation $v:\pv\to 2$ which satisfies {\bf ZFC}\,? Obviously we do
not have a precise understanding of this question at this point. But
we understand it well enough to be interested. So suppose we have
defined a subset $\Gamma\subseteq\pv$ representing all the axioms of
{\bf ZFC}. Does there exist a valuation $v\in\pvd$ such that
$v(\phi)=1$ for all $\phi\in\Gamma$\,? Is such valuation unique? If
it is not unique, can we add a few reasonable axioms to {\bf ZFC} so
we end up with a unique valuation? We remember the phrase of Albert
Einstein: ``{\em God doesn't play dice}''. So if there is some form
of {\em mathematical God} presiding over the realm of {\em absolute
truth}, we should expect a particular valuation $v^{*}:\pv\to 2$ to
stand out: the official word from above. Given a formula
$\phi\in\pv$, we may not be able to evaluate $v^{*}(\phi)$ ever. But
$v^{*}(\phi)$ is there: it is either $1$ or $0$. It is for us humans
to decipher the divine truth. As David Hilbert once said:\,``{\em We
must know - we will know!\,}''.\footnote{Wir m\"ussen wissen � wir
werden wissen!}

Unfortunately, the ideal of an immutable God-sent {\em mathematical
truth} has become very hard to justify: there is no unique valuation
$v^{*}:\pv\to 2$ which satisfies the axioms of {\bf ZFC}. In fact,
we cannot even be sure there is any valuation at all satisfying
these axioms. So let us see why this is: in 1940 Kurt G\"odel
established that the {\em continuum hypothesis} {\bf CH} could not
be disproved from the axioms of {\bf ZFC}. In 1963 Paul Cohen showed
that $\lnot{\bf CH}$ could not be disproved either. In other words,
if we assume {\bf ZFC} is consistent, then both ${\bf ZFC}+{\bf CH}$
and ${\bf ZFC}+\lnot{\bf CH}$ are also consistent. As we shall soon
discover in
theorem~(\ref{logic:the:FOPL:semantics:satis:equiv:consistent}) of
page~\pageref{logic:the:FOPL:semantics:satis:equiv:consistent},
consistent subsets are satisfiable and vice-versa. So if there
exists a valuation $v\in\pvd$ which satisfies {\bf ZFC}, then there
are at least two such valuations, one which satisfies {\bf CH} and
one which doesn't. So we can forget about uniqueness. The best we
can hope for is to design a reasonable extension of {\bf ZFC}. But
this will not work either: assuming we had a good reason to accept
or reject the continuum hypothesis, we are told from G\"odel's first
incompleteness theorem that no {\em reasonable} extension $\Delta$
of {\bf ZFC} will ever be {\em complete}\,: there will always be a
formula $\phi\in\pv$ which can neither be disproved nor be proved
from $\Delta$, i.e. for which the sequents
$\Delta\vdash(\phi\to\bot)$ and $\Delta\vdash\phi$ are false. As we
shall see from
proposition~(\ref{logic:prop:FOPL:semantics:consistent:sequent}),
this would mean that both $\Delta\cup\{\phi\}$ and
$\Delta\cup\{\phi\to\bot\}$ are consistent hence satisfiable, and we
cannot have a unique valuation satisfying $\Delta$. Of course we do
not know what a {\em reasonable} extension of {\bf ZFC} is. The set
of axioms $\Delta$ would need to be {\em recursively enumerable} (as
well as consistent) and we have not yet studied any computability
theory in these pages. This is for later: we certainly intend to
push these notes further, until we fully understand the work of Kurt
G\"odel, and indeed Paul Cohen. In the meantime, we shall accept
that the only way to obtain a unique valuation $v^{*}:\pv\to 2$
would be to adopt a set of axioms $\Delta\supseteq{\bf ZFC}$ which
no computer program could ever enumerate, let alone decide.

We now consider the question of existence: is there a valuation
$v:\pv\to 2$ satisfying {\bf ZFC}? By virtue of
theorem~(\ref{logic:the:FOPL:semantics:satis:equiv:consistent}) of
page~\pageref{logic:the:FOPL:semantics:satis:equiv:consistent}, this
question amounts to asking whether {\bf ZFC} is consistent. Once,
again the answer is highly disappointing: if {\bf ZFC} happens to be
consistent, it would seem from G\"odel's second incompleteness
theorem that we shall never be able to prove it. Broadly speaking,
G\"odel second incompleteness theorem states that no consistent and
{\em effectively generated} theory can prove its own consistency. In
particular, {\bf ZFC} cannot be consistent and prove its own
consistency. So assume once again that $\Gamma\subseteq\pv$
represents all the axioms of {\bf ZFC}. Then we cannot ever hope to
prove that $\Gamma$ is a consistent subset of \pv. Why not? Well
suppose we are able to prove the mathematical statement ``\,{\em
$\Gamma$ is consistent}\,''. Saying that {\bf ZFC} is {\em
effectively generated} is the same as saying that $\Gamma$ is a {\em
recursively enumerable} set. In other words, there exists some
computer program which generates all the elements of $\Gamma$. A
computer program is nothing but a big formula in some formal
language. So it is easy to believe that if $\Gamma$ is recursively
enumerable, then the mathematical statement ``\,{\em $\Gamma$ is
consistent}\,'' can be coded as some formula of first order logic
$\phi\in\pv$. For the purpose of the present discussion, we may
denote this formula $\phi$ as $\ulcorner\,\Gamma\mbox{\ is
consistent}\urcorner$. Now remember that everything we do in these
notes relies on {\bf ZFC}. The foundation of our meta-mathematics is
naive set theory based on the axioms of {\bf ZFC}. So if we are able
to prove the mathematical statement ``\,{\em $\Gamma$ is
consistent}\,'', then by carefully following every step of the proof
we should be able to design a formal proof $\pi\in\pvs$ whose
conclusion is the formula $\val(\pi)=\phi=\ulcorner\,\Gamma\mbox{\
is consistent}\urcorner$, using premises which are axioms of {\bf
ZFC}, i.e. such that $\hyp(\pi)\subseteq\Gamma$. In other words, if
we are able to prove ``\,{\em $\Gamma$ is consistent}\,'', then the
sequent $\Gamma\vdash\ulcorner\,\Gamma\mbox{\ is
consistent}\urcorner$ should be true. According to G\"odel's second
incompleteness theorem, this cannot be the case, unless $\Gamma$ is
inconsistent. Note that we are not claiming any part of the
preceding argument is anything but informal gibberish, which an
expert reader will scorn. There is only so much we understand at
this stage. We simply hope to convince the less informed reader that
the consistency of {\bf ZFC} is probably beyond our reach and
furthermore, that fully understanding G\"odel will require serious
work. A recent textbook on G\"odel is Peter Smith~\cite{SmithGodel}.

So we shall never prove that {\bf ZFC} is consistent. Even if we are
successful in using the algebra \pv\ as the basis of a formal
language for axiomatic set theory, even if we manage to properly
define a subset $\Gamma\subseteq\pv$ representing all the axioms of
{\bf ZFC}, we shall never be able to prove the existence of a
valuation $v\in\pvd$ which satisfies $\Gamma$. As long as our
meta-mathematics is based on {\bf ZFC}, we shall never know.
However, the history of human thoughts has seen many changes and
{\bf ZFC} is not cast in stone. There is nothing excluding the
possibility of a stronger system emerging at some time in the
future. This formal system may be represented as a wider set of
axioms $\Delta\supseteq{\bf ZFC}$. When this happens, the newly
accepted set of axioms $\Delta$ may be strong enough for us to show
that {\bf ZFC} is consistent. There is light at the end of the
tunnel. After all, we could argue this type of scenario has already
occurred: once upon a time, the human race believed in Peano
Arithmetic {\bf PA}. In those days, no one could prove that {\bf PA}
was consistent. Yet one day, {\bf ZFC} appeared and we now know that
{\bf PA} is consistent. Regardless of how frivolous the story is, it
is unfortunate that the consistency of {\bf PA} is no longer what
matters: what we care about is {\bf ZFC}. So when the time comes and
the mathematical community finally adopts a stronger axiomatic
system $\Delta\supseteq{\bf ZFC}$, the odds are we shall care about
$\Delta$. This is the curse of G\"odel's second incompleteness
theorem: $\Delta$ cannot prove the consistency of $\Delta$. So
whichever way we look at it, we shall never be content.

At this point most of us will cry. Yet, it is precisely at the
height of gloom and disappointment that some twisted miracle
happens: this miracle is the realization that knowing the
consistency of {\bf ZFC} is in fact worthless. As we shall see from
proposition~(\ref{logic:prop:FOPL:semantics:inconsistent:everything}),
if $\Gamma$ is an inconsistent subset of \pv\ then the sequent
$\Gamma\vdash\phi$ is true for all $\phi\in\pv$. In other words, it
is possible to prove anything from an inconsistent theory. So
suppose G\"odel was wrong and we could somehow prove the consistency
of {\bf ZFC} from {\bf ZFC}. This could be for two reasons: the
first reason is that {\bf ZFC} is indeed consistent. The second
reason is that {\bf ZFC} is simply inconsistent. Proving the
consistency of {\bf ZFC} wouldn't prove anything. In fact, in light
of G\"odel's second incompleteness theorem, proving the consistency
of {\bf ZFC} would actually prove something: it would constitute a
contradiction and show that {\bf ZFC} is inconsistent.


\begin{prop}\label{logic:prop:FOPL:semantics:quasi:order:valuation}
Let $V$ be a set and $\leq$ be the Hilbert deductive preorder on
\pv. Then for all $\phi,\psi\in\pv$ and valuation $v\in\pvd\,$ we
have:
    \[
    \phi\leq\psi\ \Rightarrow\ v(\phi)\leq v(\psi)
    \]
\end{prop}
\begin{proof}
Suppose $\phi\leq\psi$. Then we have $\vdash(\phi\to\psi)$ and since
$v:\pv\to 2$ is a valuation we obtain $1=v(\phi\to\psi)=v(\phi)\to
v(\psi)$. It follows that $v(\phi)\leq v(\psi)$.
\end{proof}

The next proposition is elementary but very important: let
$\sigma:V\to W$ be a map which has an associated essential
substitution $\sigma:\pv\to{\bf P}(W)$. Then we know that such
essential substitution is not unique. However it is uniquely
determined modulo the substitution congruence. So if $v:{\bf
P}(W)\to 2$ is a valuation and $\phi\in\pv$, then $v(\sigma(\phi))$
is uniquely determined. For example in the case when $V=W$, the
formula $\phi=\forall x\phi_{1}\to\phi_{1}[y/x]$ is an axiom
provided $[y/x]:\pv\to\pv$ refers to an essential substitution of
$y$ in place of $x$. So the details of the formula $\phi_{1}[y/x]$
may be unclear, but $v(\phi_{1}[y/x])$ is unambiguous.

\begin{prop}\label{logic:prop:FOPL:semantics:stronger:congruence}
Let $V$ be a set and $\sim$ be a congruence which is stronger than
the Hilbert deductive congruence.  Then for all $\phi,\psi\in\pv$
and $v\in\pvd\,$:
    \[
    \phi\sim\psi\ \Rightarrow\ v(\phi)=v(\psi)
    \]
\end{prop}
\begin{proof}
So we assume that $v:\pv\to 2$ is a valuation and $\phi\sim\psi$. We
need to show that $v(\phi)=v(\psi)$. Having assumed that $\sim$ is
stronger than the Hilbert deductive congruence $\equiv$\,, in
particular we have $\phi\equiv\psi$. Using $\phi\leq\psi$, from
proposition~(\ref{logic:prop:FOPL:semantics:quasi:order:valuation})
we obtain $v(\phi)\leq v(\psi)$. Likewise, from $\psi\leq\phi$ we
obtain $v(\psi)\leq v(\phi)$.
\end{proof}
