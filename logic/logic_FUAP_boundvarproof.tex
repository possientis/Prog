In this section we define and study the set $\bound(\pi)$ of bound
variables of a proof $\pi\in\pvs$. The notion of bound variables has
proved particularly useful when dealing with the local inversion
theorem~(\ref{logic:the:FOPL:localinv:main}) of
page~\pageref{logic:the:FOPL:localinv:main} for formulas.  A
counterpart of this theorem for proofs will be established as
theorem~(\ref{logic:the:FUAP:localinversion:inversion}) of
page~\pageref{logic:the:FUAP:localinversion:inversion}. The set of
bound variables is also important in giving us a useful test for
valid substitutions in the form of
proposition~(\ref{logic:prop:FOPL:validsub:minimalextension}) for
which a counterpart shall be established for proofs as
proposition~(\ref{logic:prop:FUAP:validsubproof:minimalextension}).
The bound variables of a proof $\pi$ are simply the bound variables
of the hypothesis and axioms of $\pi$, to which is added every
variable occurring in the use of generalization.

\index{bound@Bound variables of proof}\index{bound@$\bound(\pi)$ :
bound variables of $\pi$}
\begin{defin}\label{logic:def:FUAP:boundvarproof:bound:var}
Let $V$ be a set. The map $\bound:\pvs\to {\cal P}(V)$ defined by
the following structural recursion is called {\em bound variable
mapping on \pvs}:
 \begin{equation}\label{logic:eqn:FUAP:boundvarproof:bound:var:1}
    \forall\pi\in\pvs\ ,\ \bound(\pi)=\left\{
                    \begin{array}{lcl}
                    \bound(\phi)&\mbox{\ if\ }&\pi=\phi\in\pv\\
                    \bound(\phi)&\mbox{\ if\ }&\pi=\axi\phi\\
                    \bound(\pi_{1})\cup\bound(\pi_{2}) &\mbox{\ if\ }&\pi=\pi_{1}\pon\pi_{2}\\
                    \{x\}\cup\bound(\pi_{1})&\mbox{\ if\ }&\pi=\gen x\pi_{1}
                    \end{array}\right.
    \end{equation}
We say that $x\in V$ is a {\em bound variable} of $\pi\in\pvs$
\ifand\ $x\in\bound(\pi)$.
\end{defin}
Given a formula $\phi\in\pv$ the notation $\bound(\phi)$ is
potentially ambiguous. Since $\pv\subseteq\pvs$, it may refer to the
usual $\bound(\phi)$ of definition~(\ref{logic:def:bound:variable}),
or to the set $\bound(\pi)$ where $\pi=\phi$ of
definition~(\ref{logic:def:FUAP:boundvarproof:bound:var}). Luckily,
the two notions coincide.

\begin{prop}\label{logic:prop:FUAP:boundvarproof:recursion}
The structural recursion of {\em
definition~(\ref{logic:def:FUAP:boundvarproof:bound:var})} is
legitimate.
\end{prop}
\begin{proof}
We need to show the existence and uniqueness of the map
$\bound:\pvs\to{\cal P}(V)$ satisfying the four conditions of
equation~(\ref{logic:eqn:FUAP:boundvarproof:bound:var:1}). This
follows from an application of
theorem~(\ref{logic:the:structural:recursion}) of
page~\pageref{logic:the:structural:recursion} with $X=\pvs$,
$X_{0}=\pv$ and $A={\cal P}(V)$ where $g_{0}:X_{0}\to A$ is defined
as $g_{0}(\phi)=\bound(\phi)$. Furthermore, given $\phi\in\pv$ we
take $h(\axi\phi):A^{0}\to A$ defined $h(\axi\phi)(0)=\bound(\phi)$.
We take $h(\pon):A^{2}\to A$ defined by
$h(\pon)(A_{0},A_{1})=A_{0}\cup A_{1}$ and $h(\gen x):A^{1}\to A$
defined by $h(\gen x)(A_{0})=\{x\}\cup A_{0}$.
\end{proof}

The free and bound variables of a proof $\pi\in\pvs$ are variables
of $\pi$, as we would expect. In fact, every variable of $\pi$ is
either free or bound, or both free and bound. The following
proposition is the counterpart of
proposition~(\ref{logic:prop:FOPL:boundvar:free})\,:

\begin{prop}\label{logic:prop:FUAP:boundvarproof:var:free:bound}
Let $V$ be a set and $\pi\in\pvs$. Then we have:
    \begin{equation}\label{logic:eqn:FUAP:boundvarproof:var:free:bound:1}
    \var(\pi)=\free(\pi)\cup\bound(\pi)
    \end{equation}
\end{prop}
\begin{proof}
We shall prove the equality with a structural induction, using
theorem~(\ref{logic:the:proof:induction}) of
page~\pageref{logic:the:proof:induction}. First we assume that
$\pi=\phi$ for some $\phi\in\pv$. Then the equality follows
immediately from proposition~(\ref{logic:prop:FOPL:boundvar:free}).
We now assume that $\pi=\axi\phi$ for some $\phi\in\pv$.  Then using
proposition~(\ref{logic:prop:FOPL:boundvar:free}) once more, we
have:
    \begin{eqnarray*}
    \var(\pi)&=&\var(\axi\phi)\\
    &=&\var(\phi)\\
    \mbox{prop.~(\ref{logic:prop:FOPL:boundvar:free})}\ \rightarrow
    &=&\free(\phi)\cup\bound(\phi)\\
    &=&\free(\axi\phi)\cup\bound(\axi\phi)\\
    &=&\free(\pi)\cup\bound(\pi)\\
    \end{eqnarray*}
So we now assume that $\pi=\pi_{1}\pon\pi_{2}$ where
$\pi_{1},\pi_{2}\in\pvs$ are proofs satisfying the equality. We need
to show that same is true of $\pi$ which goes as follows:
    \begin{eqnarray*}
    \var(\pi)&=&\var(\pi_{1}\pon\pi_{2})\\
    &=&\var(\pi_{1})\cup\var(\pi_{2})\\
    &=&\free(\pi_{1})\cup\bound(\pi_{1})\cup\free(\pi_{2})\cup\bound(\pi_{2})\\
    &=&\free(\pi_{1}\pon\pi_{2})\cup\bound(\pi_{1}\pon\pi_{2})\\
    &=&\free(\pi)\cup\bound(\pi)\\
    \end{eqnarray*}
We now assume that $\pi=\gen x\pi_{1}$ where $x\in V$ and
$\pi_{1}\in\pvs$ is a proof satisfying our equality. We need to show
the same is true of $\pi$ which goes as follows:
    \begin{eqnarray*}
    \var(\pi)&=&\var(\gen x\pi_{1})\\
    &=&\{x\}\cup\var(\pi_{1})\\
    &=&\{x\}\cup\free(\pi_{1})\cup\bound(\pi_{1})\\
    &=&(\free(\pi_{1})\setminus\{x\})\cup\{x\}\cup\bound(\pi_{1})\\
    &=&\free(\gen x\pi_{1})\cup\bound(\gen x\pi_{1})\\
    &=&\free(\pi)\cup\bound(\pi)
    \end{eqnarray*}
\end{proof}

The map $\bound:\pvs\to{\cal P}(V)$ is increasing with respect to
the standard inclusion on ${\cal P}(V)$. In other words, the bound
variables of a sub-proof are also bound variables of the proof
itself, a property which does not hold for free variables. The
following is the counterpart of
proposition~(\ref{logic:prop:FOBL:boundvar:subformula})\,:
\begin{prop}\label{logic:prop:FUAP:boundvarproof:subformula}
Let $V$ be a set and $\pi,\rho\in\pvs$. Then we have:
    \[
    \rho\preceq\pi\ \Rightarrow\ \bound(\rho)\subseteq\bound(\pi)
    \]
\end{prop}
\begin{proof}
This follows from an application of
proposition~(\ref{logic:prop:UA:subformula:non:decreasing}) to
$\bound:X\to A$ where $X=\pvs$ and $A={\cal P}(V)$ where the
preorder $\leq$ on $A$ is the usual inclusion $\subseteq$. We simply
need to check that given $\pi_{1},\pi_{2}\in\pvs$ and $x\in V$ we
have the inclusions
$\bound(\pi_{1})\subseteq\bound(\pi_{1}\pon\pi_{2})$,
$\bound(\pi_{2})\subseteq\bound(\pi_{1}\pon\pi_{2})$ and
$\bound(\pi_{1})\subseteq\bound(\gen x\pi_{1})$ which follow from
the recursive
definition~(\ref{logic:def:FUAP:boundvarproof:bound:var}).
\end{proof}

Given a map $\sigma:V\to W$ and $\pi\in\pvs$, the bound variables of
the proof $\sigma(\pi)$ are the images by $\sigma$ of the bound
variables of $\pi$. A similar result for free variables cannot be
stated, unless the substitution $\sigma$ is valid for $\pi$, as will
be seen from
proposition~(\ref{logic:prop:FUAP:validsubproof:freevar}). In
general, only the inclusion
$\free(\sigma(\pi))\subseteq\sigma(\free(\pi))$ is true. The
following is the counterpart of
proposition~(\ref{logic:prop:boundvar:of:substitution})\,:

\begin{prop}\label{logic:prop:FUAP:boundvarproof:substitution}
Let $V$, $W$ be sets and $\sigma:V\to W$ be a map. Let
$\pi\in\pvs$\,:
    \[
    \bound(\sigma(\pi))=\sigma(\bound(\pi))
    \]
where $\sigma:\pvs\to{\bf\Pi}(W)$ also denotes the associated proof
substitution mapping.
\end{prop}
\begin{proof}
We shall prove this equality with a structural induction, using
theorem~(\ref{logic:the:proof:induction}) of
page~\pageref{logic:the:proof:induction}. First we assume that
$\pi=\phi$ for some $\phi\in\pv$. Then we need to show that
$\bound(\sigma(\phi))=\sigma(\bound(\phi))$ which follows from
proposition~(\ref{logic:prop:boundvar:of:substitution}). Next we
assume that $\pi=\axi\phi$ for some $\phi\in\pv$. Then we have:
    \begin{eqnarray*}
    \bound(\sigma(\pi))&=&\bound(\sigma(\axi\phi))\\
    &=&\bound(\axi\sigma(\phi))\\
    &=&\bound(\sigma(\phi))\\
    \mbox{prop.~(\ref{logic:prop:boundvar:of:substitution})}\ \rightarrow
    &=&\sigma(\bound(\phi))\\
    &=&\sigma(\bound(\axi\phi))\\
    &=&\sigma(\bound(\pi))\\
    \end{eqnarray*}
We now assume that $\pi=\pi_{1}\pon\pi_{2}$ where
$\pi_{1},\pi_{2}\in\pvs$ are proofs satisfying our equality. We need
to show the same is true of $\pi$ which goes as follows:
    \begin{eqnarray*}
    \bound(\sigma(\pi))&=&\bound(\sigma(\pi_{1}\pon\pi_{2}))\\
    &=&\bound(\sigma(\pi_{1})\pon\,\sigma(\pi_{2}))\\
    &=&\bound(\sigma(\pi_{1}))\cup\bound(\sigma(\pi_{2}))\\
    &=&\sigma(\bound(\pi_{1}))\cup\sigma(\bound(\pi_{2}))\\
    &=&\sigma(\,\bound(\pi_{1})\cup\bound(\pi_{2})\,)\\
    &=&\sigma(\bound(\pi_{1}\pon\pi_{2}))\\
    &=&\sigma(\bound(\pi))\\
    \end{eqnarray*}
Finally, we assume that $\pi=\gen x\pi_{1}$ where $x\in V$ and
$\pi_{1}\in\pvs$ is a proof satisfying our equality. We need to show
the same is true of $\pi$\,:
    \begin{eqnarray*}
    \bound(\sigma(\pi))&=&\bound(\sigma(\gen x\pi_{1}))\\
    &=&\bound(\,\gen\sigma(x)\sigma(\pi_{1})\,)\\
    &=&\{\sigma(x)\}\cup\bound(\sigma(\pi_{1}))\\
    &=&\{\sigma(x)\}\cup\sigma(\bound(\pi_{1}))\\
    &=&\sigma(\,\{x\}\cup\bound(\pi_{1})\,)\\
    &=&\sigma(\bound(\gen x\pi_{1}))\\
    &=&\sigma(\bound(\pi))\\
    \end{eqnarray*}
\end{proof}
