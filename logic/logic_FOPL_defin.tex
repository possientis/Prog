In this section, we shall limit our investigation to the free
generator $X_{0}$ and type of universal algebra $\alpha$, postponing
the issue of congruence to later parts of this document. So we are
looking to define the {\em free universal algebra of first order
logic}. First we shall deal with the free generator $X_{0}$. The
elements of $X_{0}$ should represent the most elementary
mathematical statements. Things like $(x\in y)$ or $(x=y)$ are
possible candidates for membership to $X_{0}$. We could also have
constant symbols and regard $(0\in y)$ or ($x=\pi)$ as elementary
statements. More generally, statements such as $R[x,y]$ or even
$S[s,0,u,\pi,w]$ could be viewed as elementary statements, where $R$
and $S$ are so-called {\em predicate symbols}. We may even consider
$R[f(1,y),z]$ where $f$ is a so-called {\em function symbol}. As far
as we are concerned, the universal algebra of first order logic
should be the mathematical language of the lowest possible level. In
computing terms, it should be akin to assembly language. There
should be no {\em constant}, no {\em function symbol} and no {\em
predicate symbol} beyond the customary '$\in$'. Even the equality
'$=$' should be banned. The universal algebra of first order logic
should contain the bare minimum required to express every standard
mathematical statement. This bare minimum we believe can be based
simply on elementary propositions of the form $(x\in y)$ where $x$
and $y$ belong to a set of variable $V$. So our free generator
$X_{0}$ should therefore be reduced to:
    \[
    X_{0}=\{(x\in y)\ :\ x,y\in V\}
    \]

Turning now to the type of universal algebra $\alpha$, we believe
the minimal set of operations (also known as {\em connectives})
required to express every standard mathematical statement consists
in the {\em contradiction constant} $\bot$, which is an operator of
arity $0$, the {\em implication operator} $\to$ which is a binary
operator, and for all $x\in V$, the {\em quantification operator}
$\forall x$ which is a unary operator. So the free universal algebra
of first order logic should consist in elements of the form:
    \[
    \bot\ ,\  (x\in y)\ ,\ \forall x\forall y\,(x\in y)\ ,\
    \forall z[(z\in x)\to(z\in y)]\ ,\ \mbox{etc.}
    \]
At this point in time, we have very little to justify this choice,
beyond its simplicity. Clearly, there would be no point in defining
and studying a {\em free universal algebra of first order logic}
which does not fulfill its role. At some stage, we will need to
consider higher level formal languages which are actually usable by
human beings to express interesting mathematics, and show that high
level statements can effectively be compiled into the low level.
This we believe is the only way to vindicate our choice. For now, we
hope simply to reassure the reader that our conception of {\em bare
minimum} is sensible: for example, the negation operator could be
compiled as $\lnot\phi=(\phi\to\bot)$, the disjunction and
conjunction operators as $\phi\lor\psi=\lnot\phi\to\psi$ and
$\phi\land\psi=\lnot(\lnot\phi\lor\lnot\psi)$ respectively, the
equivalence operator as
$(\phi\leftrightarrow\psi)=[(\phi\to\psi)\land(\psi\to\phi)]$ and
finally the equality predicate as:
    \[
    (x=y) = \forall z[(z\in x)\leftrightarrow(z\in y)]\land\forall z[(x\in z)\leftrightarrow(y\in z)]
    \]
This does not tell us how to introduce constant symbols and function
symbols, and we are still a very long way to ever compile a
statement such as:
    \[
    \frac{1}{\sqrt{2\pi}}\int_{-\infty}^{+\infty}e^{-x^{2}/2}dx=1
    \]
We hope to address some of these questions in later parts of this
document. For now, we shall rest on the belief that our {\em free
universal algebra of first order logic} is the right tool to
consider and study.

But assuming our faith is justified, assuming that interesting
mathematical statements can effectively be compiled as {\em formulas
of first order logic}, one important question still remains: why
should we care about a {\em low level} algebra whose {\em
mathematical statements} are so remote from day to day mathematics?
Why not consider a high level language directly? The answer is
twofold: From a purely mathematical point of view it is a lot easier
to deal with a simple algebra. For instance, if we are looking to
prove anything by structural induction, we only need to consider
$(x\in y)$, $\bot$, $\to$ and $\forall x$ for all $x\in V$ which
would not be the case with a more complex algebra. Furthermore from
a computing perspective, we are unlikely to write sensible software
unless we start from the very simple and move on to the high level.
A computer chip typically understands a very limited set of
instructions. High level languages are typically compiled into a
sequence of these elementary tasks. Without thinking very hard on
this, it is a natural thing to believe that a similar approach
should be used for meta-mathematics. At the end of the day, if it is
indeed the case that deep mathematical statements can be compiled as
formulas of first order logic, focussing on these formulas will
surely prove very useful.

\index{Type@First order logic type}\index{bot@$\bot\,$: the nullary
operator symbol}\index{bot@$\bot\,$: the operator
$\bot:\{0\}\to\pv$}\index{bot@$\bot\,$: the formula
$\bot(0)$}\index{imp@$\to\,$: the binary operator
symbol}\index{imp@$\to\,$: operator
$\to:\pv^{2}\to\pv$}\index{imp@$\phi_{1}\to\phi_{2}\,$: the formula
in \pv}\index{variable@$U,V,W\,$: sets of
variables}\index{forall@$\forall x\,$: the unary operator
symbol}\index{forall@$\forall x\,$: operator $\forall
x:\pv^{1}\to\pv$}\index{forall@$\forall x\phi_{1}\,$: the formula in
\pv}
\begin{defin}\label{logic:def:FOPL:type}
Let $V$ be a set. We call {\em First Order Logic Type associated
with $V$}, the type of universal algebra $\alpha$ defined by:
    \[
    \alpha=\{\bot, \to\}\cup\{\forall x\ :\ x\in V\}
    \]
where $\bot=((0,0),0)$, $\to = ((1,0),2)$ and $\forall x=((2,x),1)$ given $x\in V$.
\end{defin}

There is no particular conditions imposed on the set $V$, which
could be empty, finite, infinite, countable or uncountable. The set
$\alpha$ of definition~(\ref{logic:def:FOPL:type}) is a set of
ordered pairs which is functional. It is therefore a map with domain
$\dom(\alpha)=\{(0,0),(1,0)\}\cup\{(2,x):x\in V\}$ and range
$\rng(\alpha)=\{0,1,2\}$. Since $\rng(\alpha)\subseteq\N$, by virtue
of definition~(\ref{logic:def:type:universal:algebra}) $\alpha$ is
indeed a type of universal algebra. With our customary abuse of
notation described in
page~\pageref{logic:def:type:universal:algebra}, we have
$\alpha(\bot)=0$, $\alpha(\to)=2$ and $\alpha(\forall x)=1$ for all
$x\in V$. It follows that $\bot$ is understood to be an operator of
arity $0$, while $\to$ is binary and $\forall x$ is unary.

\index{free@Free universal algebra of FOL}\index{pv@$\pv\,$: free
algebra of FOL}\index{phi@$\phi,\psi,\chi\,$: formulas in
\pv}\index{pvo@$\pvo\,$: atomic formulas $(x\in y)$}\index{UA@Free
universal algebra of FOL}
\begin{defin}\label{logic:def:FOPL:free:algebra}
Let $V$ be a set with first order logic type $\alpha$. We call {\em
Free Universal Algebra of First Order Logic associated with $V$},
the free universal algebra \pv\ of type $\alpha$ with free generator
$\pvo=V\times V$.
\end{defin}

The free universal algebra \pv\ exists by virtue of
theorem~(\ref{logic:the:main:existence}) of
page~\pageref{logic:the:main:existence}. It is also unique up to
isomorphism and we have $\pvo\subseteq\pv$. For all $x,y\in V$ the
ordered pair $(x,y)$ will be denoted $(x\in y)$. It follows that the
free generator \pvo\ of \pv\ is exactly what we had promised:
    \[
    \pvo=\{(x\in y)\ :\ x,y\in V\}
    \]
Furthermore, we have the operators $\bot:\pv^{0}\to\pv$,
$\to:\pv^{2}\to\pv$ and $\forall x:\pv^{1}\to \pv$. We shall refer
to the constant $\bot(0)$ simply as $\bot$. Given $\phi,\psi\in\pv$,
we shall write $\phi\to\psi$ instead of $\to(\phi,\psi)$. Given
$x\in V$ and $\phi\in\pv$, we shall write $\forall x\phi$ instead of
$\forall x(\phi)$.
