
Recall that a pre-ordered set is an ordered pair $(A,\leq)$ where
$A$ is a set and $\leq$ is a preorder on $A$, namely a binary
relation on $A$ which is reflexive and transitive. In the following
chapters, we shall encounter many cases of maps $v:X\to A$ where $X$
is a free universal algebra and $A$ is pre-ordered set. For example,
if $X=\pv$ is the free universal algebra of first order logic of
definition~(\ref{logic:def:FOPL:free:algebra}) and $A={\cal P}(V)$
is the power set of $V$ (the set of variables), we shall encounter
the maps $\var:X\to A$ and $\bound:X\to A$ which respectively return
the set of variables and bound variables of a formula $\phi\in X$.
Another example is $X=\pvs$ of
definition~(\ref{logic:def:FOPL:proof:algebra}), i.e. the free
universal algebra of proofs on \pv\ for which we shall have a map
$\hyp:X\to A$ where $A={\cal P}(\pv)$ returning the set of all
hypothesis of a proof $\pi\in X$. We shall also have a map
$\var:\pvs\to A$ where $A={\cal P}(V)$ returning the set of
variables being used in a proof $\pi\in X$. All these cases are
examples of maps $v:X\to A$ which are {\em increasing} in the sense
that $v(x)\leq v(y)$ whenever $x$ is a sub-formula of $y$ i.e.
$x\preceq y$. For example if $\psi$ is a sub-formula of $\phi$ then
$\var(\psi)\subseteq\var(\phi)$, or if $\rho$ is a sub-proof of
$\pi$ then $\hyp(\rho)\subseteq\hyp(\pi)$. The following proposition
allows us to establish this type of result once and for all with the
right level of abstraction.

\begin{prop}\label{logic:prop:UA:subformula:non:decreasing}
Let $X$ be a free universal algebra of type $\alpha$ with free
generator $X_{0}\subseteq X$. Let $(A,\leq)$ be a pre-ordered set
and $v:X\to A$ be a map such that:
    \[
    \forall i\in\alpha(f)\ ,\ v(x(i))\leq v(f(x))
    \]
for all $f\in\alpha$ and $x\in X^{\alpha(f)}$. Then for all $x,y\in
X$ we have the implication:
    \[
    x\preceq y\ \Rightarrow\ v(x)\leq v(y)
    \]
\end{prop}
\begin{proof}
Given $y\in X$ we need to show the property: $\forall x\in X\ \
[\,x\preceq y\ \Rightarrow\ v(x)\leq v(y)\,]$. We shall do so by
structural induction using theorem~(\ref{logic:the:proof:induction})
of page~\pageref{logic:the:proof:induction}. First we assume that
$y\in X_{0}$. We need to show the property is true for $y$. So let
$x\preceq y$ be a sub-formula of $y$. We need to show that $v(x)\leq
v(y)$. However, since $y\in X_{0}$ we have $\subf(y)=\{y\}$. So from
$x\preceq y$ we obtain $x\in\subf(y)$ and consequently $x=y$. So
$v(x)\leq v(y)$ follows from the reflexivity of the preorder $\leq$.
Next we assume that $f\in\alpha$ and $y\in X^{\alpha(f)}$ is such
that $y(i)$ satisfies our property for all $i\in\alpha(f)$. We need
to show that same is true of $f(y)$. So let $x\preceq f(y)$ be a
sub-formula of $f(y)$. We need to show that $v(x)\leq v(f(y))$.
However we have:
    \[
    x\in \subf(f(y)) =
    \{f(y)\}\cup\bigcup_{i\in\alpha(f)}\subf(y(i))
    \]
So we shall distinguish two cases: first we assume that $x=f(y)$.
Then the inequality $v(x)\leq v(f(y))$ follows from the reflexivity
of $\leq$. Next we assume that $x\in\subf(y(i))$ for some
$i\in\alpha(f)$. Then we have $x\preceq y(i)$ and having assumed
$y(i)$ satisfies our induction property, we obtain $v(x)\leq
v(y(i))$. However, by assumption we have $v(y(i))\leq v(f(y))$ and
$v(x)\leq v(f(y))$ follows from the transitivity of the preorder
$\leq$ on $A$. This completes our induction argument.
\end{proof}
