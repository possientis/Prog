An essential substitution $\sigma:\pvs\to{\bf\Pi}(W)$ associated
with a map $\sigma:V\to W$ is the sharpest tool so far created in
these notes. First order logic is plagued with the annoying glitch
of variable capture. Most texts in mathematical logic will simply
ignore the problem and ask the reader to accept that every
mathematical argument involving variable substitution is sound and
preserves $\alpha$-equivalence. There are two issues with this
approach: on the one hand, the reader cannot formally check the
arguments presented to him and must instead resort to some form of
informal mathematical intuition to accept the proof presented to him
in good faith, thinking 'yes, this is probably correct, modulo
$\alpha$-equivalence'. On the other hand, as discussed prior to
definition~(\ref{logic:def:FOPL:specialization:axiom}), the informal
approach of implicitly renaming variables where needed, or simply
rejecting substitutions which are not capture-avoiding cannot work
in the context of first order logic with finitely many variables.
Our purpose is to study the language ${\cal L}=\{\in\}$ which is the
language of {\bf ZF} without equality, and to do so with a set of
variables $V$ of arbitrary cardinality. We are of course sceptical
that a cardinality beyond $\aleph_{0}$ will bring anything
interesting, but we certainly think that fragments of first order
logic with finitely many variables are worthy of attention. In any
case, we are looking for a unified approach in our study of the free
algebra \pv, that is an approach which makes no assumption on the
cardinality of $V$. In this context, the ability to substitute
variables while avoiding capture is tantamount and the introduction
of essential substitutions $\sigma:\pv\to{\bf P}(W)$ in
definition~(\ref{logic:def:FOPL:esssubst:esssubst}) has proved
invaluable. In particular, it has allowed us to present an
axiomatization of first order logic with specialization axioms
$\forall x\phi_{1}\to\phi_{1}[y/x]$ which have no caveats on
variable capture. As discussed prior to
definition~(\ref{logic:def:FOPL:specialization:axiom}) this is a key
fact to ensure completeness of the system with finitely many
variables.

The ability to substitute variables in proofs while avoiding capture
will also be very useful. In fact, a first application of essential
substitutions for proofs will be given in
theorem~(\ref{logic:the:FUAP:substitutiontheorem:main}) of
page~\pageref{logic:the:FUAP:substitutiontheorem:main} allowing us
to carry over sequents from $\Gamma\vdash\phi$ into
$\sigma(\Gamma)\vdash\sigma(\phi)$ for every essential substitution
$\sigma:\pv\to{\bf P}(W)$ and no other assumption. In this section,
we shall prove the existence of essential substitutions
$\sigma:\pvs\to{\bf\Pi}(W)$ associated with a map $\sigma:V\to W$
with the usual conditions on the cardinals $|V|$ and $|W|$ as can be
seen from theorem~(\ref{logic:the:FUAP:esssubst:existence}) of
page~\pageref{logic:the:FUAP:esssubst:existence} below. We shall
literally follow the steps which were used prior to
theorem~(\ref{logic:the:FOPL:esssubst:existence}) of
page~\pageref{logic:the:FOPL:esssubst:existence}. As we have said on
a few occasions before, it is rather shameful to {\em cut-and-paste}
existing proofs with no effort to bring about a unified abstract
approach. On the positive side, this allowed us to get the result
quickly and safely, and we can move on to worry about other things.

\index{essential@Essential proof substitution of map}
\begin{defin}\label{logic:def:FUAP:esssubst:main}
Let $V,W$ be sets and $\sigma:V\to W$ be a map. We call {\em
essential proof substitution mapping} associated with $\sigma$, a
map $\sigma^{*}:\pvs\to{\bf\Pi}(W)$ with:
    \begin{equation}\label{logic:def:FUAP:esssubst:main:key}
    {\cal M}\circ\sigma^{*}=\bar{\sigma}\circ{\cal M}
    \end{equation}
where $\bar{\sigma}:\bar{V}\to\bar{W}$ is the minimal extension and
${\cal M}$ is the minimal transform.
\end{defin}

Suppose $\sigma^{*}:\pvs\to{\bf\Pi}(W)$ is an essential substitution
associated to the map $\sigma:V\to W$. Since $\pv\subseteq\pvs$, it
is meaningful to consider the restriction $(\sigma^{*})_{|{\bf
P}(V)}$ with domain \pv. Since the equality ${\cal
M}\circ\sigma^{*}(\pi)=\bar{\sigma}\circ{\cal M}(\pi)$ holds for all
$\pi\in\pvs$, in particular we have ${\cal
M}\circ\sigma^{*}(\phi)=\bar{\sigma}\circ{\cal M}(\phi)$ for all
$\phi\in\pv$. So it is very tempting to conclude that the
restriction $(\sigma^{*})_{|{\bf P}(V)}$ is an essential
substitution associated with $\sigma:V\to W$ in the sense of
definition~(\ref{logic:def:FOPL:esssubst:esssubst}). This conclusion
is correct. However, it is not completely obvious that
$\sigma^{*}(\phi)$ should be an element of ${\bf P}(W)$ for all
$\phi\in\pv$. The following proposition checks this is indeed the
case, so $(\sigma^{*})_{|{\bf P}(V)}:\pv\to{\bf P}(W)$ is essential.

\begin{prop}\label{logic:prop:FUAP:esssubst:proof:to:formula}
Let $V,W$ be sets and $\sigma:V\to W$ be a map. We assume that
$\sigma^{*}:\pvs\to{\bf\Pi}(W)$ is an essential proof substitution
associated with $\sigma$. Then the restriction $(\sigma^{*})_{|{\bf
P}(V)}$ is an essential substitution associated with $\sigma$.
\end{prop}
\begin{proof}
We assume that $\sigma^{*}:\pvs\to{\bf\Pi}(W)$ is a map which
satisfies the equality ${\cal
M}\circ\sigma^{*}(\pi)=\bar{\sigma}\circ{\cal M}(\pi)$ for all
$\pi\in\pvs$. Then in particular for all $\phi\in\pv$ we have ${\cal
M}\circ\sigma^{*}(\phi)=\bar{\sigma}\circ{\cal M}(\phi)$. So it
remains to check that the restriction $(\sigma^{*})_{|{\bf P}(V)}$
is indeed a map $(\sigma^{*})_{|{\bf P}(V)}:\pv\to{\bf P}(W)$. So we
need to show that $\sigma^{*}(\phi)\in{\bf P}(W)$ for all
$\phi\in\pv$. However we have ${\cal M}\circ\sigma^{*}(\phi)\in{\bf
P}(\bar{W})$. Furthermore, we know from
theorem~(\ref{logic:the:unique:representation}) of
page~\pageref{logic:the:unique:representation} that the proof
$\sigma^{*}(\phi)$ can only be of four types, namely
$\sigma^{*}(\phi)=\psi$ for some $\psi\in{\bf P}(W)$ or
$\sigma^{*}(\phi)=\axi\psi$ or
$\sigma^{*}(\phi)=\rho_{1}\pon\rho_{2}$ or $\sigma^{*}(\phi)=\gen
u\rho_{1}$. Looking at
definition~(\ref{logic:def:FUAP:mintransproof:transform}) the
minimal transform ${\cal M}\circ\sigma^{*}(\phi)$ has essentially
the same type as $\sigma^{*}(\phi)$. Thus from ${\cal
M}\circ\sigma^{*}(\phi)\in{\bf P}(\bar{W})$ and the uniqueness
property of theorem~(\ref{logic:the:unique:representation}) we
conclude that $\sigma^{*}(\phi)$ must be of the form
$\sigma^{*}(\phi)=\psi$ for some $\psi\in{\bf P}(W)$.
\end{proof}

We shall now follow the trail of what was done for formulas. The
following definition is the counterpart of
definition~(\ref{logic:def:FOPL:esssubst:weak:transform}), prior to
which the reader may find some motivating comments and a description
of the proof strategy. Note that the maps
$p:\bar{\bar{V}}\to\bar{V}$ of
definition~(\ref{logic:def:FOPL:esssubst:weak:transform}) and
definition~(\ref{logic:def:FUAP:esssubst:weak:transform}) below are
actually the same. Furthermore given $\phi\in\pvb$, since
$\pvb\subseteq\pvsb$ the notation ${\cal N}(\phi)$ is potentially
ambiguous as it may refer to the weak transform of
definition~(\ref{logic:def:FOPL:esssubst:weak:transform}), or to
this new weak transform for proofs of
definition~(\ref{logic:def:FUAP:esssubst:weak:transform}) below.
Luckily, the two notions coincide as they do for minimal transforms.

\begin{defin}\label{logic:def:FUAP:esssubst:weak:transform}
Let $V$ be a set. We call {\em weak transform} on \pvsb\ the map
${\cal N}:\pvsb\to\pvsb$ defined by ${\cal N}=p\circ\bar{\cal M}$
where $\bar{\cal M}:\pvsb\to{\bf\Pi}(\bar{\bar{V}})$ is the minimal
transform mapping and $p:\bar{\bar{V}}\to\bar{V}$ is defined by:
    \[
    \forall u\in\bar{\bar{V}}\ ,\ p\,(u)=\left\{
        \begin{array}{lcl}
        u&\mbox{\ if\ }&u\in\bar{V}\\
        n&\mbox{\ if\ }&u=\bar{n}\in\bar{\N}
        \end{array}
    \right.
    \]
\end{defin}

The following lemma is the counterpart of
lemma~(\ref{logic:lemma:FOPL:esssubst:MNi}). It shows that the
minimal transform can be defined from the weak transform and the
inclusion map $i:V\to\bar{V}$ with ${\cal M}={\cal N}\circ
i=p\circ\bar{\cal M}\circ i$. For example, suppose $\pi$ is the
proof (formula) $\pi=\forall y(x\in y)$. Then $i(\pi)=\forall y(x\in
y)$ now viewed as an element of $\pvsb$. It follows that $\bar{\cal
M}\circ i(\pi)=\forall\,\bar{0}(x\in\bar{0})$ and $p\circ\bar{\cal
M}\circ i(\pi)=\forall\,0(x\in 0)$ which is indeed the minimal
transform of $\pi$. So let us quote:

\begin{lemma}\label{logic:lemma:FUAP:esssubst:MNi}
Let $V$ be a set and ${\cal N}:\pvsb\to\pvsb$ be the weak transform
on \pvsb. Let $i:V\to\bar{V}$ be the inclusion map. Then we have:
    \[
    {\cal M}={\cal N}\circ i
    \]
where ${\cal M}:\pvs\to\pvsb$ is the minimal transform mapping.
\end{lemma}
\begin{proof}
Let $\bar{\cal M}:\pvsb\to{\bf\Pi}(\bar{\bar{V}})$ be the minimal
transform mapping and $p:\bar{\bar{V}}\to\bar{V}$ be the map of
definition~(\ref{logic:def:FUAP:esssubst:weak:transform}). Given
$\pi\in\pvs$, since $i:V\to\bar{V}$ is an injective map, in
particular it is valid for $\pi$. Hence we have:
    \begin{eqnarray*}
    {\cal N}\circ i(\pi)&=&p\circ\bar{\cal M}\circ i(\pi)\\
    \mbox{theorem~(\ref{logic:the:FUAP:mintransvalidsub:commute})}\
    \rightarrow&=&p\circ\bar{i}\circ{\cal M}(\pi)\\
    \mbox{A: to be proved}\ \rightarrow&=&{\cal M}(\pi)
    \end{eqnarray*}
So it remains to show that $p\circ\bar{i}(u)=u$ for all
$u\in\bar{V}$, where $\bar{i}:\bar{V}\to\bar{\bar{V}}$ is the
minimal extension of $i$. So let $u\in\bar{V}$. We shall distinguish
two cases: first we assume that $u\in V$. Then $\bar{i}(u)=i(u)=u$
and consequently $p\circ\bar{i}(u)=p\,(u)=u$ as requested. Next we
assume that $u\in\N$. Then $\bar{i}(u)=\bar{u}$ and it follows that
$p\circ\bar{i}(u)=p\,(\bar{u})=u$, which completes our proof. Please
refer to the proof of lemma~(\ref{logic:lemma:FOPL:esssubst:MNi})
for a more detailed discussion on the step $\bar{i}(u)=\bar{u}$.
\end{proof}

The following lemma is the counterpart of
lemma~(\ref{logic:lemma:FOPL:esssubst:NsM}). It shows that the weak
transform ${\cal N}$ has no effect when acting on
$\bar{\sigma}\circ{\cal M}$. For example, suppose $\pi=\forall
y(x\in y)$ and $\sigma=[y/x]$. Then $\bar{\sigma}\circ{\cal
M}(\pi)=\forall\, 0(y\in 0)$. Taking the minimal transform
$\bar{\cal M}$ we obtain $\bar{\cal M}\circ\bar{\sigma}\circ{\cal
M}(\pi)=\forall\, \bar{0}(y\in \bar{0})$. Composing by $p$ we
conclude that ${\cal N}\circ\bar{\sigma}\circ{\cal M}(\pi)=\forall\,
0(y\in 0)=\bar{\sigma}\circ{\cal M}(\pi)$. More generally:

\begin{lemma}\label{logic:lemma:FUAP:esssubst:NsM}
Let $V,W$ be sets and $\sigma:V\to W$ be a map. Then:
    \[
    {\cal N}\circ\bar{\sigma}\circ{\cal M}=\bar{\sigma}\circ{\cal M}
    \]
where ${\cal N}:{\bf\Pi}(\bar{W})\to{\bf\Pi}(\bar{W})$ is the weak
transform on ${\bf\Pi}(\bar{W})$.
\end{lemma}
\begin{proof}
For once we shall be able to prove the formula without resorting to
a structural induction argument. In the interest of lighter
notations, we shall keep the same notations ${\cal M},\bar{\cal M},
{\cal N}$ and $p$ in relation to the sets $V$ and $W$. So let
$\pi\in\pvs$:
    \begin{eqnarray*}
    {\cal N}\circ\bar{\sigma}\circ{\cal M}(\pi)&=&p\circ\bar{\cal
    M}\circ\bar{\sigma}\circ{\cal M}(\pi)\\
    \mbox{theorem~(\ref{logic:the:FUAP:mintransvalidsub:commute})
    of p.~\pageref{logic:the:FUAP:mintransvalidsub:commute}}\ \rightarrow
    &=&p\circ\bar{\bar{\sigma}}\circ\bar{\cal
    M}\circ{\cal M}(\pi)\\
    \mbox{prop.~(\ref{logic:prop:FUAP:mintransfsubcong:equivalence})}\ \rightarrow
    &=&p\circ\bar{\bar{\sigma}}\circ\bar{\cal
    M}\circ i(\pi)\\
    \mbox{A: to be proved}\ \rightarrow&=&\bar{\sigma}\circ p\circ\bar{\cal
    M}\circ i(\pi)\\
    &=&\bar{\sigma}\circ{\cal
    N}\circ i(\pi)\\
    \mbox{lemma~(\ref{logic:lemma:FUAP:esssubst:MNi})}\ \rightarrow
    &=&\bar{\sigma}\circ{\cal
    M}(\pi)\\
    \end{eqnarray*}
So it remains to show that
$p\circ\bar{\bar{\sigma}}(u)=\bar{\sigma}\circ p\,(u)$ for all
$u\in\bar{\bar{V}}$. So let $u\in\bar{\bar{V}}$. Since
$\bar{\bar{V}}$ is the disjoint union of $\bar{V}$ and $\bar{\N}$,
we shall distinguish two cases: first we assume that $u\in\bar{V}$.
Then $\bar{\bar{\sigma}}(u)=\bar{\sigma}(u)\in\bar{W}$ and
consequently $p\circ\bar{\bar{\sigma}}(u)=\bar{\sigma}(u)$. So the
equality $p\circ\bar{\bar{\sigma}}(u)=\bar{\sigma}\circ p\,(u)$ is
satisfied since $p\,(u)=u$ for $u\in\bar{V}$. Next we assume that
$u\in\bar{\N}$ i.e. that $u=\bar{n}$ for some $n\in\N$. Then
$\bar{\bar{\sigma}}(u)=u=\bar{n}$ and consequently
$p\circ\bar{\bar{\sigma}}(u)=n=\bar{\sigma}\circ p\,(u)$.
\end{proof}

The following proposition is the counterpart of
proposition~(\ref{logic:prop:FOPL:esssubst:mintransform:equiv:imp:equal}).
It is not directly related to our objective of proving
theorem~(\ref{logic:the:FUAP:esssubst:key}) below, but it is a very
useful proposition allowing us to establish an equality ${\cal
M}(\pi)={\cal M}(\rho)$ simply from an $\alpha$-equivalence ${\cal
M}(\pi)\sim{\cal M}(\rho)$. Now is a good time to prove this:

\begin{prop}\label{logic:prop:FUAP:esssubst:mintransform:equiv:imp:equal}
Let $V$ be a set and $\pi,\rho\in\pvs$. Then we have:
    \[
    {\cal M}(\pi)\sim{\cal M}(\rho)\ \Rightarrow\ {\cal
    M}(\pi)={\cal M}(\rho)
    \]
where the relation $\sim$ denotes the substitution congruence on
\pvsb.
\end{prop}
\begin{proof}
We assume that ${\cal M}(\pi)\sim{\cal M}(\rho)$. We need to show
that ${\cal M}(\pi)={\cal M}(\rho)$. However, from
theorem~(\ref{logic:the:FUAP:mintransfsubcong:kernel}) of
page~\pageref{logic:the:FUAP:mintransfsubcong:kernel} we obtain
$\bar{\cal M}\circ{\cal M}(\pi)=\bar{\cal M}\circ{\cal M}(\rho)$,
where $\bar{\cal M}:\pvsb\to{\bf\Pi}(\bar{\bar{V}})$ is the minimal
transform mapping. Hence, from
definition~(\ref{logic:def:FUAP:esssubst:weak:transform}), we see
that ${\cal N}\circ{\cal M}(\pi)={\cal N}\circ{\cal M}(\rho)$.
Applying lemma~(\ref{logic:lemma:FUAP:esssubst:NsM}) to $W=V$ and
the identity mapping $\sigma:V\to V$ we conclude that ${\cal
M}(\pi)={\cal M}(\rho)$.
\end{proof}

The following lemma is the counterpart of
lemma~(\ref{logic:lemma:FOPL:esssubst:p:valid}). It is also not
related to theorem~(\ref{logic:the:FUAP:esssubst:key}) below, but
will be very useful in the future.

\begin{lemma}\label{logic:lemma:FUAP:esssubst:p:valid}
Let $V$ be a set and $p:\bar{\bar{V}}\to\bar{V}$ be the map of {\em
definition~(\ref{logic:def:FUAP:esssubst:weak:transform})}. Then for
all $\pi\in\pvs$, the substitution $p$ is valid for the proof
$\bar{\cal M}\circ{\cal M}(\pi)$.
\end{lemma}
\begin{proof}
Using proposition~(\ref{logic:prop:FUAP:validsubproof:injective}) it
is sufficient to show that $p$ is injective on the set
$\var(\bar{\cal M}\circ{\cal M}(\pi))$. First we shall show that $p$
is injective on $V\cup\bar{\N}$: we already did this in the course
of proving lemma~(\ref{logic:lemma:FOPL:esssubst:p:valid}). We
repeat the argument here for the reader's convenience: so suppose
$u,v\in V\cup\bar{\N}$ are such that $p\,(u)=p\,(v)$. We need to
show that $u=v$. We shall distinguish four cases: first we assume
that $u,v\in V\subseteq\bar{V}$. Then $u=v$ follows immediately from
definition~(\ref{logic:def:FUAP:esssubst:weak:transform}). Next we
assume that $u,v\in\bar{\N}$. Then $u=\bar{n}$ and $v=\bar{m}$ for
some $n,m\in\N$. From $p\,(u)=p\,(v)$ we obtain $n=m$ and
consequently $u=v$. Next we assume that $u\in V$ and $v\in\bar{\N}$.
Then $v=\bar{m}$ for some $m\in\N$ and from the equality
$p\,(u)=p\,(v)$ we obtain $u=n$ which contradicts the fact that
$V\cap\N=\emptyset$. So this case is in fact impossible. The case
$u\in\bar{\N}$ and $v\in V$ is likewise impossible and we have
proved that $p$ is injective on $V\cup\bar{\N}$. So it remains to
show that $\var(\bar{\cal M}\circ{\cal M}(\pi))\subseteq
V\cup\bar{\N}$. Let $u\in\var(\bar{\cal M}\circ{\cal M}(\pi))$. We
need to show that $u\in V\cup\bar{\N}$. Since
$\bar{\bar{V}}=\bar{V}\uplus\bar{\N}$, we shall distinguish two
cases: first we assume that $u\in\bar{V}$. Then using
proposition~(\ref{logic:prop:FUAP:mintransformproof:freevar}) we
obtain:
    \[
    u\in\var(\bar{\cal M}\circ{\cal
    M}(\pi))\cap\bar{V}=\free({\cal M}(\pi))=\free(\pi)\subseteq
    V\subseteq V\cup\bar{\N}
    \]
Next we assume that $u\in\bar{\N}$. Then it is clear that $u\in
V\cup\bar{\N}$.
\end{proof}

We are now ready to quote the main theorem of this section which is
the counterpart of theorem~(\ref{logic:the:FOPL:esssubst:key}) of
page~\pageref{logic:the:FOPL:esssubst:key}. This theorem shows that
the condition $\rnk(\,\bar{\sigma}\circ{\cal M}(\pi)\,)\leq|W|$ is
indeed sufficient for the proof $\bar{\sigma}\circ{\cal M}(\pi)$ to
be {\em squeezed} into the space ${\bf \Pi}(W)$. Specifically, it is
sufficient to ensure the existence of $\rho\in{\bf\Pi}(W)$ such that
$\bar{\sigma}\circ{\cal M}(\pi)={\cal M}(\rho)$. This will crucially
allow us to prove the existence of essential proof substitutions in
theorem~(\ref{logic:the:FUAP:esssubst:existence}) below. Note that
theorem~(\ref{logic:the:FUAP:esssubst:key}) has nothing magical and
fundamentally consists in a simple application of
proposition~(\ref{logic:prop:FUAP:substrank:changeofvar}), allowing
us to move variables around provided they are bound, and provided
the substitution rank meets the right condition.

\begin{theorem}\label{logic:the:FUAP:esssubst:key}
Let $V,W$ be sets and $\sigma:V\to W$ be a map. Let $\pi\in\pvs$
with:
    \[
    \rnk(\,\bar{\sigma}\circ{\cal M}(\pi)\,)\leq|W|
    \]
where $\bar{\sigma}:\bar{V}\to\bar{W}$ is the minimal extension of
$\sigma$ and ${\cal M}(\pi)$ is the minimal transform of $\pi$.
Then, there exists $\rho\in{\bf\Pi}(W)$ such that
$\bar{\sigma}\circ{\cal M}(\pi)={\cal M}(\rho)$.
\end{theorem}
\begin{proof}
Our first step is to find $\rho_{1}\in{\bf\Pi}(\bar{W})$ such that
$\bar{\sigma}\circ{\cal M}(\pi)\sim\rho_{1}$ and
$\var(\rho_{1})\subseteq W$, where $\sim$ denotes the substitution
congruence on ${\bf\Pi}(\bar{W})$. We shall do so using
proposition~(\ref{logic:prop:FUAP:substrank:changeofvar}) applied to
the set $\bar{W}$ and the proof $\bar{\sigma}\circ{\cal
M}(\pi)\in{\bf\Pi}(\bar{W})$. Suppose for now that we have proved
the inclusion $\free(\bar{\sigma}\circ{\cal M}(\pi))\subseteq W$.
Then from the assumption $\rnk(\,\bar{\sigma}\circ{\cal
M}(\pi)\,)\leq|W|$ and
proposition~(\ref{logic:prop:FUAP:substrank:changeofvar}) we obtain
the existence of $\rho_{1}\in{\bf\Pi}(\bar{W})$ such that
$\bar{\sigma}\circ{\cal M}(\pi)\sim\rho_{1}$ and
$\var(\rho_{1})\subseteq W$. In fact,
proposition~(\ref{logic:prop:FUAP:substrank:changeofvar}) allows to
assume that $|\var(\rho_{1})|=\rnk(\,\bar{\sigma}\circ{\cal
M}(\pi)\,)$ but we shall not be using this property. So we need to
prove that $\free(\bar{\sigma}\circ{\cal M}(\pi))\subseteq W$:
    \begin{eqnarray*}
    \free(\bar{\sigma}\circ{\cal M}(\pi))&=&\free(\,\bar{\sigma}({\cal
    M}(\pi))\,)\\
    \mbox{prop.~(\ref{logic:prop:FUAP:freevarproof:substitution:inclusion})}\ \rightarrow
    &\subseteq&\bar{\sigma}(\,\free({\cal M}(\pi))\,)\\
    \mbox{prop.~(\ref{logic:prop:FUAP:mintransformproof:freevar})}\ \rightarrow
    &=&\bar{\sigma}(\free(\pi))\\
    \pi\in\pvs\ \rightarrow&\subseteq&\bar{\sigma}(V)\\
    \mbox{def.~(\ref{logic:def:FOPL:commute:minextensioon:map})}\ \rightarrow
    &=&\sigma(V)\\
    &\subseteq&W
    \end{eqnarray*}
So the existence of $\rho_{1}\in{\bf\Pi}(\bar{W})$ such that
$\bar{\sigma}\circ{\cal M}(\pi)\sim\rho_{1}$ and
$\var(\rho_{1})\subseteq W$ is now established. Next we want to
project $\rho_{1}$ onto ${\bf\Pi}(W)$ by defining $\rho=q(\rho_{1})$
where $q:\bar{W}\to W$ is a substitution such that $q(u)=u$ for all
$u\in W$. To be rigorous, we need to define $q(n)$ for $n\in\N$
which we cannot do when $W=\emptyset$. So in the case when
$W\neq\emptyset$, let $u^{*}\in W$ and define $q:\bar{W}\to W$ as
follows:
    \[
    \forall u\in \bar{W}\ ,\ q(u)=\left\{
        \begin{array}{lcl}
        u&\mbox{\ if\ }&u\in W\\
        u^{*}&\mbox{\ if\ }&u\in\N
        \end{array}
    \right.
    \]
and consider the associated proof substitution mapping
$q:{\bf\Pi}(\bar{W})\to{\bf\Pi}(W)$. In the case when $W=\emptyset$,
there exists no substitution $q:\bar{W}\to W$ but we can define an
operator $q:{\bf P}(\bar{W})\to{\bf P}(W)$ with the following
structural recursion:
    \begin{equation}\label{logic:eqn:FUAP:esssubst:the:W:empty:formula}
                    q(\chi)=\left\{
                    \begin{array}{lcl}
                    \bot&\mbox{\ if\ }&\chi=(u\in v)\\
                    \bot&\mbox{\ if\ }&\chi=\bot\\
                    q(\chi_{1})\to q(\chi_{2})
                    &\mbox{\ if\ }&\chi=\chi_{1}\to\chi_{2}\\
                    \bot&
                    \mbox{\ if\ }&\chi=\forall u\chi_{1}
                    \end{array}\right.
    \end{equation}
and we can extend this operator as
$q:{\bf\Pi}(\bar{W})\to{\bf\Pi}(W)$ with the recursion:
    \begin{equation}\label{logic:eqn:FUAP:esssubst:the:W:empty:proof}
        q(\kappa)=\left\{
                    \begin{array}{lcl}
                    q(\chi)&\mbox{\ if\ }&\kappa=\chi\in{\bf P}(\bar{W})\\
                    \axi q(\chi)&\mbox{\ if\ }&\kappa=\axi\chi\\
                    q(\kappa_{1})\pon\,q(\kappa_{2})&\mbox{\ if\ }&\kappa=\kappa_{1}\pon\kappa_{2}\\
                    \bot&\mbox{\ if\ }&\kappa=\gen
                    u\kappa_{1}
                    \end{array}\right.
    \end{equation}
We do not really care how $q(\kappa)$ is defined when $\kappa=\gen
u\kappa_{1}$, so we are simply setting $q(\kappa)=\bot$ which is the
proof with hypothesis $\bot$ and conclusion $\bot$. Whether
$W=\emptyset$ or not, we have an operator
$q:{\bf\Pi}(\bar{W})\to{\bf\Pi}(W)$ and we set
$\rho=q(\rho_{1})\in{\bf\Pi}(W)$. We shall complete the proof of the
theorem by proving the equality $\bar{\sigma}\circ{\cal
M}(\pi)={\cal M}(\rho)$. Using
lemma~(\ref{logic:lemma:FUAP:esssubst:NsM}) we have:
    \begin{eqnarray*}
    \bar{\sigma}\circ{\cal M}(\pi)&=&{\cal N}\circ\bar{\sigma}\circ{\cal
    M}(\pi)\\
    \mbox{def.~(\ref{logic:def:FUAP:esssubst:weak:transform})}\ \rightarrow
    &=&p\circ\bar{\cal M}\circ\bar{\sigma}\circ{\cal
    M}(\pi)\\
    \mbox{theorem~(\ref{logic:the:FUAP:mintransfsubcong:kernel})
    and $\bar{\sigma}\circ{\cal M}(\pi)\sim\rho_{1}$}\ \rightarrow
    &=&p\circ\bar{\cal M}(\rho_{1})\\
    &=&{\cal N}(\rho_{1})\\
    \mbox{A: to be proved}\ \rightarrow
    &=&{\cal N}\circ i\circ q(\rho_{1})\\
    \rho=q(\rho_{1})\ \rightarrow
    &=&{\cal N}\circ i(\rho)\\
    \mbox{lemma~(\ref{logic:lemma:FUAP:esssubst:MNi})}\ \rightarrow
    &=&{\cal M}(\rho)\\
    \end{eqnarray*}
So it remains to show that $i\circ q(\rho_{1})=\rho_{1}$ where
$i:W\to\bar{W}$ is the inclusion map. As before, we shall
distinguish two cases: first we assume that $W\neq\emptyset$. Then
$q$ arises from the substitution $q:\bar{W}\to W$ and from
proposition~(\ref{logic:prop:FUAP:variable:support}) it is
sufficient to prove that $i\circ q(u)=u$ for all
$u\in\var(\rho_{1})$. Since $\var(\rho_{1})\subseteq W$ the equality
is clear. Next we assume that $W=\emptyset$. From the inclusion
$\var(\rho_{1})\subseteq W$ it follows that
$\var(\rho_{1})=\emptyset$ and it is therefore sufficient to prove
the property:
    \[
    \var(\kappa)=\emptyset\ \Rightarrow\ i\circ q(\kappa)=\kappa
    \]
for all $\kappa\in{\bf\Pi}(\bar{W})$. We shall first prove the
property is true for $\kappa=\chi\in{\bf P}(\bar{W})$ with a
structural induction argument on ${\bf P}(\bar{W})$. Note that when
$W=\emptyset$, the inclusion map $i:W\to\bar{W}$ is simply the map
with empty domain, namely the empty set. First we assume that
$\chi=(u\in v)$ for some $u,v\in\bar{W}$. Then the above implication
is vacuously true. Next we assume that $\chi=\bot$. Then $i\circ
q(\chi)=\chi$ is clear. So we now assume that
$\chi=\chi_{1}\to\chi_{2}$ where $\chi_{1},\chi_{2}$ satisfy the
above implication. We need to show the same is true of $\chi$. So we
assume that $\var(\chi)=\emptyset$. We need to show that $i\circ
q(\chi)=\chi$. However, from $\var(\chi)=\emptyset$ we obtain
$\var(\chi_{1})=\emptyset$ and $\var(\chi_{2})=\emptyset$. Having
assumed $\chi_{1}$ and $\chi_{2}$ satisfy the implication, we obtain
the equalities $i\circ q(\chi_{1})=\chi_{1}$ and $i\circ
q(\chi_{2})=\chi_{2}$ from which $i\circ q(\chi)=\chi$ follows
immediately. So it remains to check the case when $\chi=\forall
u\chi_{1}$ for which the above implication is also vacuously true.
We now show the above property with an induction argument on
${\bf\Pi}(\bar{W})$\,: first we assume that $\kappa=\chi\in{\bf
P}(\bar{W})$. Then the above implication has been established. Next
we assume that $\kappa=\axi\chi$ for some $\chi\in{\bf P}(\bar{W})$.
We need to show the implication is true for $\kappa$. So we assume
that $\var(\kappa)=\emptyset$ and we have:
    \begin{eqnarray*}
    i\circ q(\kappa)&=&i\circ q(\axi\chi)\\
    &=&\axi\, i\circ q(\chi)\\
    \var(\chi)=\emptyset\ \rightarrow&=&\axi\chi\\
    &=&\kappa
    \end{eqnarray*}
Next we assume that $\kappa=\kappa_{1}\pon\kappa_{2}$ for some
$\kappa_{1},\kappa_{2}\in{\bf\Pi}(\bar{W})$ satisfying our
implication. We need to show the same is true of $\kappa$. So we
assume that $\var(\kappa)=\emptyset$\,:
    \begin{eqnarray*}
    i\circ q(\kappa)&=&i\circ q(\kappa_{1}\pon\kappa_{2})\\
    &=&i (\,q(\kappa_{1})\pon\, q(\kappa_{2})\,)\\
    &=&i\circ q(\kappa_{1})\pon\, i\circ q(\kappa_{2})\\
    \var(\kappa_{1})=\emptyset,\ \var(\kappa_{2})=\emptyset\ \rightarrow
    &=&\kappa_{1}\pon\kappa_{2}\\
    &=&\kappa
    \end{eqnarray*}
Finally we assume that $\kappa=\gen u\kappa_{1}$ in which case the
property is vacuously true.
\end{proof}

The following theorem is the counterpart of
theorem~(\ref{logic:the:FOPL:esssubst:existence}) of
page~\pageref{logic:the:FOPL:esssubst:existence}.

\index{essential@Existence of essential sub}
\begin{theorem}\label{logic:the:FUAP:esssubst:existence}
Let $V,W$ be sets and $\sigma:V\to W$ be a map. Then, there exists
an essential proof substitution mapping $\sigma^{*}:\pvs\to{\bf
\Pi}(W)$ associated with $\sigma$, \ifand\  $|W|$ is an infinite
cardinal or the inequality  $|V|\leq|W|$ holds.
\end{theorem}
\begin{proof}
First we prove the 'if' part: so we assume that $|W|$ is an infinite
cardinal, or that it is finite with $|V|\leq |W|$. We need to prove
the existence of an essential substitution mapping
$\sigma^{*}:\pvs\to{\bf\Pi}(W)$ associated with $\sigma$. Let
$c:{\cal P}({\bf\Pi}(W))\setminus\{\emptyset\}\to{\bf\Pi}(W)$ be a
choice function whose existence follows from the axiom of choice.
Let us accept for now that for all $\pi\in\pvs$, there exists some
$\rho\in{\bf\Pi}(W)$ such that $\bar{\sigma}\circ{\cal M}(\pi)={\cal
M}(\rho)$. Given $\rho\in{\bf\Pi}(W)$, let $[\rho]$ denote the
congruence class of $\rho$ modulo the substitution congruence, which
is a non-empty subset of ${\bf\Pi}(W)$. Define
$\sigma^{*}:\pvs\to{\bf\Pi}(W)$ by setting
$\sigma^{*}(\pi)=c([\rho])$, where $\rho$ is an arbitrary proof of
${\bf\Pi}(W)$ such that $\bar{\sigma}\circ{\cal M}(\pi)={\cal
M}(\rho)$. We need to check that $\sigma^{*}$ is well defined,
namely that $\sigma^{*}(\pi)$ is independent of the particular
choice of $\rho$. But if $\rho'$ is such that
$\bar{\sigma}\circ{\cal M}(\pi)={\cal M}(\rho')$ then ${\cal
M}(\rho)={\cal M}(\rho')$ and it follows from
theorem~(\ref{logic:the:FUAP:mintransfsubcong:kernel}) of
page~\pageref{logic:the:FUAP:mintransfsubcong:kernel} that
$[\rho]=[\rho']$. So it remains to show the existence of $\rho$ such
that $\bar{\sigma}\circ{\cal M}(\pi)={\cal M}(\rho)$ for all
$\pi\in\pvs$. So let $\pi\in\pvs$. Using
theorem~(\ref{logic:the:FUAP:esssubst:key}) of
page~\pageref{logic:the:FUAP:esssubst:key} it is sufficient to prove
that $\rnk(\,\bar{\sigma}\circ{\cal M}(\pi)\,)\leq|W|$. The
substitution rank of a proof being always finite, this is clearly
true if $|W|$ is infinite. Otherwise, using
proposition~(\ref{logic:prop:FUAP:substrank:substitution})\,:
    \begin{eqnarray*}
    \rnk(\,\bar{\sigma}\circ{\cal M}(\pi)\,)&\leq&\rnk({\cal
    M}(\pi))\\
    \mbox{prop.~(\ref{logic:prop:FUAP:substrank:minrank})}\ \rightarrow
    &=&\rnk(\pi)\\
    &\leq&|\var(\pi)|\\
    &\leq&|V|\\
    |V|\leq|W|\ \rightarrow&\leq&|W|
    \end{eqnarray*}
We now prove the 'only if' part: So we assume there exists
$\sigma^{*}:\pvs\to{\bf\Pi}(W)$ essential substitution associated
with $\sigma$. We need to show that $|W|$ is an infinite cardinal,
or that it is finite with $|V|\leq |W|$. However, from
proposition~(\ref{logic:prop:FUAP:esssubst:proof:to:formula}) the
restriction $\sigma^{*}_{|{\bf P}(V)}:\pv\to{\bf P}(W)$ is an
essential substitution and the conclusion follows from
theorem~(\ref{logic:the:FOPL:esssubst:existence}) of
page~\pageref{logic:the:FOPL:esssubst:existence}.
\end{proof}
