In this section, we explicitly
construct a free universal algebra of type $\alpha$ whose free generator is a {\em
copy} of a given set $X_{0}$. In a later section, we shall prove the existence of
a free universal algebra whose generator is the set $X_{0}$ itself. Recall that if
$A$ and $B$ are sets, the {\em difference} $A\setminus B$ is defined as:
    \[
    A\setminus B =\{x: x\in A\mbox{\ and\ }x\not\in B\}
    \]
This notion will be used while proving the following proposition:
\begin{prop}\label{logic:prop:construction} Let $\alpha$ be a type of universal
algebra and $X_{0}$ be a set. Define:
    \[
    Y_{0}=\{(0,x):x\in X_{0}\}\ ,\ Y_{n+1}=Y_{n}\cup\bar{Y}_{n}\ ,\ n\in\N
    \]
with:
    \[
    \bar{Y}_{n}=\left\{(1,(f,x)):\ f\in\alpha\ ,\ x\in Y_{n}^{\alpha(f)}\right\}
    \]
Let:
    \[
    Y=\bigcup_{n=0}^{+\infty}Y_{n}
    \]
and $T$ be the map with domain $\alpha$ defined by setting $T(f):Y^{\alpha(f)}\to
Y$ as: \[
    T(f)(x)=(1,(f,x))
\] Then $(Y,T)$ is a free universal algebra of type $\alpha$ with free generator
$Y_{0}$.
\end{prop}
\begin{proof} First we need to check that we indeed have
$T(f):Y^{\alpha(f)}\to Y$ for all $f\in\alpha$. It is clear that $T(f)$ is a well
defined map with domain $Y^{\alpha(f)}$. However given $x\in Y^{\alpha(f)}$, we
need to check that $T(f)(x)\in Y$. In order to do so, it is sufficient to show the
existence of $n\in\N$ such that $x\in Y_{n}^{\alpha(f)}$ as this will imply that:
    \[
    T(f)(x)=(1,(f,x))\in\bar{Y}_{n}\subseteq Y_{n+1}\subseteq Y
    \]
If $\alpha(f)=0$ then $Y_{n}^{\alpha(f)}=Y^{\alpha(f)}=1$ for all $n\in\N$ and in
particular $x\in Y_{n}^{\alpha(f)}$. So we assume that $\alpha(f)\geq 1$. Since
$x:\alpha(f)\to Y$, given $i\in \alpha(f)$ we have $x(i)\in Y$. So there exists
$n_{i}\in\N$ such that $x(i)\in Y_{n_{i}}$. If we define:
    \[
    n = \max(n_{0},\ldots,n_{\alpha(f)-1})
    \]
since $Y_{n_{i}}\subseteq Y_{n}$ for all $i\in\alpha(f)$, we see that $x(i)\in
Y_{n}$ for all $i\in\alpha(f)$. It follows that $x:\alpha(f)\to Y_{n}$, i.e. $x\in
Y_{n}^{\alpha(f)}$ and we have proved that $T(f):Y^{\alpha(f)}\to Y$ for all
$f\in\alpha$ as required. It follows that $Y$ is a set and $T$ is a map with
domain $\alpha$ such that $T(f):Y^{\alpha(f)}\to Y$ for all $f\in\alpha$. So
$(Y,T)$ is a universal algebra of type $\alpha$. It remains to show that $(Y,T)$
is free with $Y_{0}$ as a free generator. So let $(Z,S)$ be a universal algebra of
type $\alpha$ and $g_{0}:Y_{0}\to Z$ be a map. We need to show that $g_{0}$ can be
uniquely extended into a morphism $g:Y\to Z$. We define $g$ by setting $g_{|Y_{0}}
= g_{0}$ and $g_{|(Y_{n+1})}=g_{n+1}$ where each $g_{n}$ is a map $g_{n}:Y_{n}\to
Z$ and the sequence $(g_{n})_{n\in\N}$ is defined by recursion with the property
that $(g_{n+1})_{|Y_{n}}=g_{n}$ for all $n\in\N$. Specifically, we define
$g_{n+1}(y)=g_{n}(y)$ for all $y\in Y_{n}$, and given $y\in Y_{n+1}\setminus Y_{n}
\subseteq \bar{Y}_{n}$, we consider $f\in\alpha$ and $x\in Y_{n}^{\alpha(f)}$ such
that $y=(1,(f,x))$. Note that such representation of $y$ exists and is clearly
unique. We then define $g_{n+1}(y)$ by setting:
    \begin{equation}\label{logic:eqn:existence:free}
    g_{n+1}(y) = S(f)(g_{n}^{\alpha(f)}(x))
    \end{equation}
Recall that $g_{n}^{\alpha(f)}:Y_{n}^{\alpha(f)}\to Z^{\alpha(f)}$ is the map
defined by $g_{n}^{\alpha(f)}(x)(i) = g_{n}(x(i))$ for all $i\in\alpha(f)$. Since
$S(f):Z^{\alpha(f)}\to Z$ it follows that $g_{n+1}(y)$ as given by
equation~(\ref{logic:eqn:existence:free}) is a well-defined element of $Z$. This
completes our recursion and we have a sequence of maps $g_{n}:Y_{n}\to Z$ such
that $(g_{n+1})_{|Y_{n}}=g_{n}$ for all $n\in\N$. From this last property, it
follows that $g:Y\to Z$ is itself well-defined by setting $g_{|Y_{n}}=g_{n}$ for
all $n\in\N$. So we have map $g:Y\to Z$ such that $g_{|Y_{0}}=g_{0}$. We shall now
check that $g$ is a morphism. So let $f\in\alpha$ and $x\in Y^{\alpha(f)}$. We
need to check that $g\circ T(f)(x)=S(f)\circ g^{\alpha(f)}(x)$. Since $x\in
Y^{\alpha(f)}$ we have already shown the existence of $n\in\N$ such that $x\in
Y_{n}^{\alpha(f)}$. In fact, let us pick $n$ to be the smallest of such integers.
By definition we have $T(f)(x) = (1,(f,x))$. From this equality and $x\in
Y_{n}^{\alpha(f)}$, it follows that $T(f)(x)$ is an element of
$\bar{Y}_{n}\subseteq Y_{n+1}$. We claim that $T(f)(x)$ is in fact an element of
$Y_{n+1}\setminus Y_{n}$. So we need to show that $T(f)(x)\not\in Y_{n}$. Suppose
to the contrary that $T(f)(x)\in Y_{n}$. Let $k\in\N$ denote the smallest integer
such that $T(f)(x)\in Y_{k}$. Note that $k\leq n$. If $k=0$ we obtain $(1,(f,x)) =
(0,x')$ for some $x'\in X_{0}$ which is a contradiction. So $k\geq 1$ and
$Y_{k}=Y_{k-1}\cup\bar{Y}_{k-1}$. From the minimality of $k$ we have
$T(f)(x)\not\in Y_{k-1}$. It follows that $T(f)(x)\in\bar{Y}_{k-1}$. So there
exist $f'\in\alpha$ and $x'\in Y_{k-1}^{\alpha(f)}$ such that:
    \[
    (1,(f,x))=T(f)(x)=(1,(f',x'))
    \]
From this we see that $x=x'\in Y_{k-1}^{\alpha(f)}$. Since $k\leq n$, we have
$k-1< n$ and $x\in Y_{k-1}^{\alpha(f)}$ contradicts the minimality of $n$. So we
have shown that $T(f)(x)\in Y_{n}$ leads to a contradiction, and it follows that
$T(f)(x)\in Y_{n+1}\setminus Y_{n}$. Thus:
    \[
    g\circ T(f)(x) = g_{n+1}(T(f)(x)) = S(f)(g_{n}^{\alpha(f)}(x))=S(f)\circ
    g^{\alpha(f)}(x)
    \]
where the last equality follows from $g_{n}=g_{|Y_{n}}$ and $x\in
Y_{n}^{\alpha(f)}$. This completes our proof of the fact that $g:Y\to Z$ is a
morphism. It remains to check that $g$ is unique. So let $g':Y\to Z$ be another
morphism such that $g'_{|Y_{0}}=g_{0}$. We shall prove by induction that
$g_{|Y_{n}}=g'_{|Y_{n}}$ for all $n\in\N$. Since $g_{|Y_{0}}=g_{0}=g'_{|Y_{0}}$,
this is clearly true for $n=0$. So we assume that $n\in\N$ and $y\in Y_{n+1}$. We
need to show that $g(y)=g'(y)$. From the induction hypothesis, the equality is
true for $y\in Y_{n}$. So we may assume that $y\in Y_{n+1}\setminus
Y_{n}\subseteq\bar{Y}_{n}$. In particular, there exist $f\in\alpha$ and $x\in
Y_{n}^{\alpha(f)}$ such that $y=(1,(f,x))$. It follows that $y=T(f)(x)$ and
finally:
    \[
    g(y)=g\circ T(f)(x)=S(f)\circ g^{\alpha(f)}(x)=S(f)\circ
    g'^{\alpha(f)}(x)=g'\circ T(f)(x)=g'(y)
    \]
where the third equality follows from the induction hypothesis and $x\in
Y_{n}^{\alpha(f)}$. \end{proof}
