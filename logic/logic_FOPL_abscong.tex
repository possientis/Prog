After the substitution and permutation congruence, following our
pursuit of the {\em right} congruence on \pv\ which should be
defined to identify mathematical statements which have the same {\em
meaning}, a third source of identity will be studied in this
section. As mathematicians, it is not uncommon for us to regard a
mathematical statement $\phi_{1}$ as being exactly the same as
$\forall x\phi_{1}$ whenever $x$ is not a free variable of
$\phi_{1}$. We shall define the {\em absorption congruence} on \pv\
as being generated by the ordered pairs $(\phi_{1},\forall
x\phi_{1})$ where $x\not\in\free(\phi_{1})$. Note that whichever
final congruence we wish to adopt to define the {\em universal
algebra of first order logic}, it is not completely obvious that
such congruence should contain the absorption congruence: on the one
hand it is appealing to say that $\bot$ and $\forall x\bot$ are the
same mathematical statements. On the other hand, a consequence of
doing so will be that $\top$ and $\exists x\top$ will also be
identical statements, after we include some propositional
equivalence, and define $\top$ and $\exists x$ in the obvious way.
This state of affairs is somewhat unsatisfactory: The statement
$\top$ does not say anything, contrary to the statement $\exists
x\top$ which expresses the existence of something. If we are to use
the {\em universal algebra of first order logic} to study axiomatic
set theory at a later stage, we would rather have the existence of
at least one set guaranteed by the axioms of {\bf ZF}, rather than
being embedded in the language being used. The universe being void
is a logical possibility.
\index{congruence@Absorption congruence}
\begin{defin}\label{logic:def:FOPL:abscong:absorption}
Let $V$ be a set. We call {\em absorption congruence} on \pv, the
congruence on \pv\ generated by the following set $R_{0}\subseteq
\pv\times\pv$:
    \[
    R_{0}=\left\{\,(\phi_{1},\forall x\phi_{1})\ :\ \
    x\not\in\free(\phi_{1})\ ,\ \phi_{1}\in\pv\ ,\ x\in V\
    \,\right\}
    \]
\end{defin}

Just like the substitution and the permutation congruence, the
absorption congruence preserves free variables. The following
proposition is similar to
propositions~(\ref{logic:prop:sub:congruence:freevar})
and~(\ref{logic:prop:FOPL:permcong:freevar}) and simply involves
checking the property on  $R_{0}$\,:
\begin{prop}\label{logic:prop:FOPL:abscong:freevar}
Let $\sim$ denote the absorption congruence on \pv\ where $V$ is a
set. Then for all $\phi,\psi\in\pv$ we have the implication:
    \[
    \phi\sim\psi\ \Rightarrow\ \free(\phi)=\free(\psi)
    \]
\end{prop}
\begin{proof}
Let $\equiv$ denote the relation on \pv\ defined by $\phi\equiv\psi$
\ifand\ we have $\free(\phi)=\free(\psi)$. Then we need to show that
$\sim\,\subseteq\,\equiv\,$. However, we know from
proposition~(\ref{logic:prop:congruence:freevar}) that $\equiv$ is a
congruence on \pv. Since $\sim$ is defined as the smallest
congruence on \pv\ which contains the set $R_{0}$ of
definition~(\ref{logic:def:FOPL:abscong:absorption}), we simply need
to show that $R_{0}\subseteq\,\equiv\,$. So let $\phi_{1}\in\pv$ and
$x\in V$ such that $x\not\in\free(\phi_{1})$. We need to show that
$\phi_{1}\equiv\forall x\phi_{1}$ which is
$\free(\phi_{1})=\free(\forall x\phi_{1})$. Since $\free(\forall
x\phi_{1})=\free(\phi_{1})\setminus\{x\}$, the result follows from
$x\not\in\free(\phi_{1})$.
\end{proof}

When $\sim$ is the substitution or permutation congruence and
$\phi,\psi\in\pv$ are such that $\phi\sim\psi$, from the structural
form of $\phi$ we are able to infer the structural form of $\psi$
thanks to the characterizations of
theorem~(\ref{logic:the:sub:congruence:charac}) of
page~\pageref{logic:the:sub:congruence:charac} and
theorem~(\ref{logic:the:perm:congruence:charac}) of
page~\pageref{logic:the:perm:congruence:charac}. For example, when
$\phi=\phi_{1}\to\phi_{2}$, we are able to tell that $\psi$ must be
of the form $\psi=\psi_{1}\to\psi_{2}$ which can be very useful when
dealing with structural induction arguments. So we would like to
obtain a similar result for the absorption congruence, and we shall
do so in theorem~(\ref{logic:the:characabscong:charac}) of
page~\pageref{logic:the:characabscong:charac}. However, the case of
the absorption congruence is more complicated than any other in this
respect, as it does not preserve the structure of formulas. So for
example if $\phi_{1}\sim\psi_{1}$ and $\phi_{2}\sim\psi_{2}$ then
the formula $\phi=\phi_{1}\to\phi_{2}$ and
$\psi=\psi_{1}\to\psi_{2}$ must be equivalent simply because $\sim$
is a congruent relation. But from the equivalence $\phi\sim\psi$ and
the knowledge that $\phi=\phi_{1}\to\phi_{2}$, it is impossible for
us to conclude that $\psi=\psi_{1}\to\psi_{2}$ with
$\phi_{1}\sim\psi_{1}$ and $\phi_{2}\sim\psi_{2}$. This is because
we can always add quantifiers $\forall x$, $\forall y\forall x$ and
$\forall z\forall y\forall x$ in front of the formula $\psi$ with
$x,y,z\not\in\free(\psi)$, while preserving the equivalence class of
$\psi$. So the situation appears hopeless, and one natural instinct
is to give up attempting to provide a characterization theorem for
the absorption congruence, in line with
theorem~(\ref{logic:the:sub:congruence:charac}) and
theorem~(\ref{logic:the:perm:congruence:charac}). However, it may be
that the equivalence $\phi\sim\psi$ allows us to say something about
$\phi$ and $\psi$ nonetheless, something which may be a weaker
statement than initially hoped for, but still better than nothing at
all. Consider the formula $\phi\in\pv$: It may have a few {\em
pointless quantifiers} coming at the front. In other words, it may
be of the form $\phi=\forall z\forall y\forall x\phi^{*}$ with
$x,y,z\not\in\free(\phi^{*})$. So let us assume that these {\em
pointless} quantifications at the front have been removed and we are
left with the formula $\phi^{*}$. In particular, we assume that any
quantifier coming at the front of $\phi^{*}$ is {\em meaningful}, in
the sense that if $\phi^{*}=\forall x\phi_{1}$ then
$x\in\free(\phi_{1})$. Now consider the formula $\psi$\,: we can
also remove the {\em first layer of meaningless} quantification and
retain the {\em core} $\psi^{*}$ of the formula $\psi$. It is clear
that we have $\phi\sim\phi^{*}$ and $\psi\sim\psi^{*}$. So whenever
the equivalence $\phi\sim\psi$ arises, we accept the inescapable
fact that nothing can be said about the relative structures of
$\phi$ and $\psi$ as these are being obfuscated by the presence of
{\em pointless quantifications} at the front. But it may be that
something can be said about the relative structures of $\phi^{*}$
and $\psi^{*}$. In other words, after we remove the {\em first
layer} of {\em meaningless quantifications}, the remaining {\em core
formulas} may have a common structure. This is indeed the case, as
will be seen from theorem~(\ref{logic:the:characabscong:charac}) of
page~\pageref{logic:the:characabscong:charac}. For now, we shall
concentrate on giving a precise definition of the {\em core formula}
$\phi^{*}$ obtained from $\phi$ after removing the {\em first layer
of pointless quantifications}. We shall prove in
proposition~(\ref{logic:prop:FOPL:abscong:representation}) below
that any formula~$\phi$ can be uniquely expressed as $\phi=\forall
u\phi^{*}$ with $u\in V^{n}$, $n\in\N$ and additional properties
formalizing the idea that the iterated quantification $\forall u$ of
definition~(\ref{logic:def:iterated:quant}) is {\em pointless} while
any remaining quantifier at the front of $\phi^{*}$ is meaningful.
This allows us to put forward the definition:


\index{core@Core decomposition of formula}\index{phi@$\phi^{*}$ :
the core of $\phi$}
\begin{defin}\label{logic:def:FOPL:abscong:core:decomposition}
Let $V$ be a set and $\phi\in\pv$. We call $\phi=\forall u\phi^{*}$
of {\em proposition~(\ref{logic:prop:FOPL:abscong:representation})}
{\em the core decomposition} of $\phi$, and we call $\phi^{*}$ the
{\em core} of $\phi$.
\end{defin}

Before we prove
proposition~(\ref{logic:prop:FOPL:abscong:representation}) we shall
establish the following lemma:

\begin{lemma}\label{logic:lemma:FOPL:abscong:equivdef}
Let $n\in\N$, $u\in V^{n}$ and $\phi^{*}\in\pv$. The following are
equivalent:
    \begin{eqnarray*}
    (i)&& u(k)\not\in\free(\phi^{*})\mbox{\ ,\ for all $k\in n$}\\
    (ii)&& u(k)\not\in\free(\,\forall u_{|k}\phi^{*}\,)\mbox{\ ,\ for all $k\in n$}\\
    \end{eqnarray*}
\end{lemma}
\begin{proof}
In order to show the equivalence between $(i)$ and $(ii)$, it is
sufficient to prove that if $(i)$ or $(ii)$ is satisfied, then the
property $(k\in n)\ \Rightarrow\ \free(\phi^{*})=\free(\,\forall
u_{|k}\phi^{*}\,)$ is true. Recall that given $u\in V^{n}$, $u$ is a
map $u:n\to V$ and $u_{|k}$ is the restriction of $u$ to $k\subseteq
n$. When $k=0$, from definition~(\ref{logic:def:iterated:quant}) of
the iterated quantification we have $\free(\,\forall
u_{|k}\phi^{*}\,)=\free(\forall\,0\phi^{*})=\free(\phi^{*})$ so the
property is true, regardless of whether $(i)$ or $(ii)$ holds. So
suppose the property is true for $k\in\N$. We need to show it is
also true for $k+1$. So we assume that $(k+1)\in n$. We need to show
the equality $\free(\phi^{*})=\free(\,\forall
u_{|(k+1)}\phi^{*}\,)$. Assuming $(i)$ is true:
    \begin{eqnarray*}
    \free(\,\forall u_{|(k+1)}\phi^{*}\,)&=&\free(\,\forall u(k)\forall
    u_{|k}\phi^{*}\,)\\
    &=&\free(\,\forall
    u_{|k}\phi^{*}\,)\setminus\{u(k)\}\\
    \mbox{Induction hypothesis}\ \rightarrow
    &=&\free(\phi^{*})\setminus\{u(k)\}\\
    \mbox{$(i)$}\ \rightarrow&=&\free(\phi^{*})\\
    \end{eqnarray*}
If we assume that $(ii)$ is true, then the argument goes as follows:
    \begin{eqnarray*}
    \free(\,\forall u_{|(k+1)}\phi^{*}\,)&=&\free(\,\forall u(k)\forall
    u_{|k}\phi^{*}\,)\\
    &=&\free(\,\forall
    u_{|k}\phi^{*}\,)\setminus\{u(k)\}\\
    \mbox{$(ii)$}\ \rightarrow&=&\free(\,\forall
    u_{|k}\phi^{*}\,)\\
    \mbox{Induction hypothesis}\ \rightarrow
    &=&\free(\phi^{*})\\
    \end{eqnarray*}
\end{proof}

In proposition~(\ref{logic:prop:FOPL:abscong:representation}) below,
we chose to write $u(k)\not\in\free(\,\forall u_{|k}\phi^{*}\,)$
in~$(i)$ rather than the simpler $u(k)\not\in\free(\phi^{*})$. This
will make some of the formal proofs smoother. By virtue of
lemma~(\ref{logic:lemma:FOPL:abscong:equivdef}) the two are
equivalent.

\begin{prop}\label{logic:prop:FOPL:abscong:representation}
Let $V$ be a set and $\phi\in\pv$. Then $\phi$ can be uniquely
represented as $\phi=\forall u\phi^{*}$ where $u\in V^{n}$,
$n\in\N$, $\phi^{*}\in\pv$ with the property:
    \begin{eqnarray*}
    (i)&& u(k)\not\in\free(\,\forall u_{|k}\phi^{*}\,)\mbox{\ ,\ for all $k\in n$}\\
    (ii)&& \phi^{*}=\forall z\psi\ \Rightarrow\
    z\in\free(\psi)
    \end{eqnarray*}
where it is understood that $(ii)$ holds for all $z\in V$ and
$\psi\in\pv$.
\end{prop}
\begin{proof}
We shall prove this property with a structural induction using
theorem~(\ref{logic:the:proof:induction}) of
page~\pageref{logic:the:proof:induction}. First we assume that
$\phi=(x\in y)$ for some $x,y\in V$. Take $\phi^{*}=\phi$, $n=0$ and
$u=0$. We claim that $\forall u\phi^{*}$ is a core decomposition of
$\phi$. From definition~(\ref{logic:def:iterated:quant}) of the
iterated quantification, it is clear that $\phi=\forall u\phi^{*}$.
Since $n=0$, property $(i)$ of
proposition~(\ref{logic:prop:FOPL:abscong:representation}) is
vacuously satisfied. Furthermore since $\phi^{*}=(x\in y)$, from
theorem~(\ref{logic:the:unique:representation}) of
page~\pageref{logic:the:unique:representation} we see that
$\phi^{*}$ cannot be expressed as $\phi^{*}=\forall z\psi$ where
$z\in V$ and $\psi\in\pv$. It follows that $(ii)$ is also vacuously
satisfied. So we have proved the existence of the core decomposition
$\phi=\forall u\phi^{*}$. We now show the uniqueness: so suppose
$\phi=\forall u\phi^{*}$ for some $u\in V^{n}$, $n\in\N$ and
$\phi^{*}\in\pv$. We need to show that $n=0$, $u=0$ and
$\phi^{*}=\phi$. So suppose to the contrary that $n>0$. Then we
obtain the following equality:
    \[
    (x \in y)=\phi=\forall u\phi^{*}=\forall u(n-1)\forall
    u_{|(n-1)}\phi^{*}
    \]
which contradict the uniqueness property of
theorem~(\ref{logic:the:unique:representation}). It follows that
$n=0$ and from $u\in V^{n}=\{0\}$ we obtain $u=0$ and finally
$\phi=\forall u\phi^{*}=\phi^{*}$. So we now assume that
$\phi=\bot$. Once again, taking $\phi^{*}=\phi$, $n=0$ and $u=0$ we
obtain a core decomposition $\forall u\phi^{*}$ of $\phi$.
Furthermore if we assume $\phi=\forall u\phi^{*}$ for some $u\in
V^{n}$, $n\in\N$ and $\phi^{*}\in\pv$, then $n=0$ follows
immediately from theorem~(\ref{logic:the:unique:representation}) and
consequently $u=0$ and $\phi^{*}=\phi$. So we now assume that
$\phi=\phi_{1}\to\phi_{2}$ for some $\phi_{1},\phi_{2}\in\pv$ which
satisfy our property. We need to show the same is true of $\phi$.
Take $\phi^{*}=\phi$, $n=0$ and $u=0$. We obtain a core
decomposition $\forall u\phi^{*}$ of $\phi$ which is unique by
virtue of theorem~(\ref{logic:the:unique:representation}). The fact
that $\phi_{1}$ and $\phi_{2}$ are themselves uniquely decomposable
is irrelevant here. So we now assume that $\phi=\forall x\phi_{1}$
for some $x\in V$ and $\phi_{1}\in\pv$ satisfying our property. We
need to show the same is true of $\phi$. We shall distinguish two
cases: first we assume that $x\in\free(\phi_{1})$. Take
$\phi^{*}=\phi$, $n=0$ and $u=0$. We claim that $\forall u\phi^{*}$
is a core decomposition of $\phi$. Since $\phi=\forall u\phi^{*}$
and $(i)$ is vacuously satisfied, we simply need to check that
$(ii)$ is true. So suppose $\phi^{*}=\forall z\psi$ for some $z\in
V$ and $\psi\in\pv$. We need to show that $z\in\free(\psi)$. However
the equality $\phi^{*}=\forall z\psi$ implies that $\forall
x\phi_{1}=\forall z\psi$ and consequently from
theorem~(\ref{logic:the:unique:representation}) we have $x=z$ and
$\phi_{1}=\psi$. It follows that $z\in\free(\psi)$ from the
assumption $x\in\free(\phi_{1})$. We now prove the uniqueness of the
core decomposition in the case when $x\in\free(\phi_{1})$. So
suppose $\phi=\forall u\phi^{*}$ for some $u\in V^{n}$, $n\in\N$ and
$\phi^{*}\in\pv$ satisfying $(i)$ and $(ii)$. We need to show that
$n=0$, $u=0$ and $\phi^{*}=\phi$. Suppose to the contrary that
$n>0$. Then we obtain the following equality:
    \[
    \forall x\phi_{1}=\phi=\forall u\phi^{*}=\forall u(n-1)\forall
    u_{|(n-1)}\phi^{*}
    \]
Using theorem~(\ref{logic:the:unique:representation}) we obtain
$x=u(n-1)$ and $\phi_{1}=\forall u_{|(n-1)}\phi^{*}$. Applying
property $(i)$ to $k=(n-1)\in n$ we obtain $x\not\in\free(\phi_{1})$
which contradicts our assumption. We now assume that
$x\not\in\free(\phi_{1})$. First we show that a core decomposition
exists: from our induction hypothesis there exists a core
decomposition $\phi_{1}=\forall u_{1}\phi_{1}^{*}$ of $\phi_{1}$,
where $u_{1}\in V^{n_{1}}$, $n_{1}\in\N$ and $\phi_{1}^{*}\in\pv$
satisfying $(i)$ and $(ii)$. Take $\phi^{*}=\phi_{1}^{*}$,
$n=n_{1}+1$ and $u\in V^{n}$ defined by $u_{|(n-1)}=u_{1}$ and
$u(n-1)=x$. We claim $\forall u\phi^{*}$ is a core decomposition
of~$\phi$\,:
    \begin{eqnarray*}
    \phi&=&\forall x\phi_{1}\\
    &=&\forall x\forall u_{1}\phi_{1}^{*}\\
    &=&\forall u(n-1)\forall u_{|(n-1)}\phi^{*}\\
    &=&\forall u\phi^{*}
    \end{eqnarray*}
So it remains to check that $(i)$ and $(ii)$ are satisfied. First we
show $(i)$. So let $k\in n$. We need to show that
$u(k)\not\in\free(\forall u_{|k}\phi^{*})$. We shall distinguish two
cases. First we assume that $k=n-1$. Then we need to show that
$x\not\in\free(\forall u_{1}\phi_{1}^{*})=\free(\phi_{1})$ which is
true by assumption. Next we assume that $k\in (n-1)$. Then we need
to show that $u_{1}(k)\not\in\free(\forall
(u_{1})_{|k}\phi_{1}^{*})$ which is true by virtue of property $(i)$
applied to the core decomposition $\phi_{1}=\forall
u_{1}\phi_{1}^{*}$. We now show property $(ii)$. So we assume that
$\phi^{*}=\forall z\psi$ for some $z\in V$ and $\psi\in\pv$. We need
to show that $z\in\free(\psi)$. Since $\phi^{*}=\phi_{1}^{*}$, this
follows from property $(ii)$ applied to the core decomposition
$\phi_{1}=\forall u_{1}\phi_{1}^{*}$. So we have proved that
$\forall u\phi^{*}$ is indeed a core decomposition of $\phi$ in the
case when $x\not\in\free(\phi_{1})$. It remains to show the
uniqueness. So we assume that $\phi=\forall u\phi^{*}$ for some
$u\in V^{n}$, $n\in\N$ and $\phi^{*}\in\pv$ satisfying $(i)$ and
$(ii)$. We need to show that $n=n_{1}+1$, $u(n-1)=x$,
$u_{|(n-1)}=u_{1}$ and $\phi^{*}=\phi_{1}^{*}$. First we show that
$n>0$. Indeed if $n=0$ then $u=0$ and consequently we obtain
$\phi^{*}=\forall u\phi^{*}=\phi=\forall x\phi_{1}$. It follows from
property $(ii)$ that $x\in\free(\phi_{1})$ which contradicts our
assumption. Having established that $n>0$ we obtain the equality:
    \[
    \forall u(n-1)\forall u_{|(n-1)}\phi^{*}=\forall
    u\phi^{*}=\phi=\forall x\phi_{1}=\forall x\forall
    u_{1}\phi_{1}^{*}
    \]
Using theorem~(\ref{logic:the:unique:representation}) we see that
$x=u(n-1)$ as requested and furthermore:
    \[
    \forall u_{|(n-1)}\phi^{*}=\phi_{1}=\forall
    u_{1}\phi_{1}^{*}
    \]
From our induction hypothesis and the uniqueness of the core
decomposition $\phi_{1}=\forall u_{1}\phi_{1}^{*}$, in order to show
that $u_{|(n-1)}=u_{1}$ and $\phi^{*}=\phi_{1}^{*}$ it is sufficient
to prove that $\forall u_{|(n-1)}\phi^{*}$ is a core decomposition
of $\phi_{1}$. So we simply need to check that $(i)$ and $(ii)$ are
satisfied, which is clearly the case since $(i)$ and $(ii)$ are
assumed to be true for the decomposition $\forall u\phi^{*}$. Having
established the equality between the maps $u_{|(n-1)}$ and $u_{1}$,
these must have identical domain and consequently $n=n_{1}+1$ as
requested, which completes our induction.
\end{proof}

We shall now formally check that the core of $\phi$ is equivalent to
$\phi$ modulo the absorption congruence, which is what we expect.
Removing the {\em first layer of pointless quantification} does not
affect the equivalence class of $\phi$\,:

\begin{prop}\label{logic:prop:FOPL:abscong:core:equivalent}
Let $\sim$ be the absorption congruence on \pv\ where $V$ is a set.
Let $\phi\in\pv$ with core $\phi^{*}\in\pv$. Then we have the
equivalence:
    \[
    \phi\sim\phi^{*}
    \]
\end{prop}
\begin{proof}
We shall prove the equivalence with an induction argument over
$n\in\N$, where $n$ is the integer underlying the core decomposition
$\phi=\forall u\phi^{*}$ where $u\in V^{n}$ and $\phi^{*}\in\pv$
satisfying $(i)$ and $(ii)$ of
proposition~(\ref{logic:prop:FOPL:abscong:representation}). First we
assume that $n=0$. Then $u=0$ and $\phi=\phi^{*}$ so the equivalence
is clear. Next we assume that $n>0$ and the equivalence is true for
$n-1$. We need to show that $\forall u\phi^{*}\sim\phi^{*}$.
However, since $\forall u\phi^{*}$ satisfy $(i)$ and $(ii)$ of
proposition~(\ref{logic:prop:FOPL:abscong:representation}), the same
can be said of $\forall u_{|(n-1)}\phi^{*}$ which is therefore a
core decomposition itself. From our induction hypothesis we obtain
$\forall u_{|(n-1)}\phi^{*}\sim\phi^{*}$. It is therefore sufficient
to prove that $\forall u(n-1)\forall u_{|(n-1)}\phi^{*}\sim\forall
u_{|(n-1)}\phi^{*}$ which follows immediately from
$u(n-1)\not\in\free(\forall u_{|(n-1)}\phi^{*})$ and
definition~(\ref{logic:def:FOPL:abscong:absorption}).
\end{proof}


\begin{lemma}\label{logic:lemma:FOPL:abscong:x:not:free}
Let $V$ be a set and $\phi\in\pv$ with core decomposition
$\phi=\forall u\phi^{*}$ where $u\in V^{n}$. Let $x\in V$ with
$x\not\in\free(\phi)$. Then $\forall x\phi$ has core decomposition:
    \[
    \forall x\phi=\forall v\phi^{*}
    \]
where $v\in V^{n+1}$ is defined by the equalities $v(n)=x$ and
$v_{|n}=u$.
\end{lemma}
\begin{proof}
We assume that $\forall u\phi^{*}$ is the core decomposition of
$\phi$ where $u\in V^{n}$. Let $x\in V$ with $x\not\in\free(\phi)$.
Let $v\in V^{n+1}$ be defined by $v(n)=x$ and $v_{|n}=u$. We need to
show that $\forall v\phi^{*}$ is the core decomposition of $\forall
x\phi$. We have:
    \begin{eqnarray*}
    \forall v\phi^{*}&=&\forall v(n)\forall v_{|n}\phi^{*}\\
    &=&\forall x\forall u\phi^{*}\\
    &=&\forall x\phi\\
    \end{eqnarray*}
So it remains to show that $(i)$ and $(ii)$ of
proposition~(\ref{logic:prop:FOPL:abscong:representation}) are
satisfied. First we show $(i)$. So let $k\in(n+1)$. We need to show
that $v(k)\not\in\free(\forall v_{|k}\phi^{*})$. We shall
distinguish two cases: first we assume that $k=n$. Then we need to
show that $x\not\in\free(\forall u\phi^{*})=\free(\phi)$ which is
true by assumption. Next we assume that $k\in n$. Then we need to
show that $u(k)\not\in\free(\forall u_{|k}\phi^{*})$ which follows
from $(i)$ applied to the core decomposition $\forall u\phi^{*}$. We
now prove $(ii)$. So we assume that $\phi^{*}=\forall z\psi$ for
some $z\in V$ and $\psi\in\pv$. We need to show that
$z\in\free(\psi)$. This follows immediately from $(ii)$ applied to
the core decomposition $\forall u\phi^{*}$.
\end{proof}
