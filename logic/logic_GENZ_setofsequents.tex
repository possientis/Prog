\index{sv@$\sv$ : set sequents}\index{gamma@$\Gamma\vdash\Delta$ :
sequent} \index{gamma@$\Gamma\vdash\phi$ :
sequent}\index{gamma@$\Gamma_{1},\Gamma_{2}\vdash\Delta_{1},\Delta_{2}$
: sequent}\index{gamma@$\Gamma,\psi\vdash\Delta,\phi$ : sequent}
\index{gamma@$\Gamma(s)$ : left part of sequent $s$}
\index{delta@$\Delta(s)$ : right part of sequent $s$}
\index{s@$s,s_{1},s_{2}$ sequents, elements of $\sv$}

\begin{defin}\label{logic:def:GENZ:UAsequents:sequent}
Let $V$ be a set. We call {\em sequent} on $V$ any ordered pair
$(\Gamma,\Delta)$ where $\Gamma,\Delta\subseteq\pv$. The set of all
sequents on $V$ is denoted \sv.
\end{defin}

Given $\Gamma,\Delta\subseteq\pv$, the sequent $(\Gamma,\Delta)$
will be denoted $\Gamma\vdash\Delta$ or $(\Gamma\vdash\Delta)$
whichever is more readable. Given $\phi,\psi\in\pv$, we shall write
$\Gamma\vdash\phi$, $\psi\vdash\Delta$ and $\psi\vdash\phi$ as a
shortcut for $\Gamma\vdash\{\phi\}$, $\{\psi\}\vdash\Delta$ and
$\{\psi\}\vdash\{\phi\}$ respectively. Given
$\Gamma_{1},\Gamma_{2},\Delta_{1},\Delta_{2}$, we may write
$\Gamma_{1},\Gamma_{2}\vdash\Delta_{1},\Delta_{2}$ as a shortcut for
$\Gamma_{1}\cup\Gamma_{2}\vdash\Delta_{1}\cup\Delta_{2}$. The
meaning of $\Gamma,\psi\vdash\Delta,\phi$ should also be clear. Note
that we are introducing notational overloading for the sake of
readability and tradition, but at the risk of some confusion. The
statement $\Gamma\vdash\phi$ says that $\phi$ is provable from
$\Gamma$ as per definition~(\ref{logic:def:FOPL:proof:of:formula}),
while $\Gamma\vdash\phi$ also refers to the ordered pair
$(\Gamma,\{\phi\})$ which is an element of \sv. Hopefully the
context will prevent any confusion. Given $s\in\sv$, we shall also
denote $\Gamma(s)$ and $\Delta(s)$ the left and right projections of
$s$ respectively, so that $s=(\Gamma(s)\vdash\Delta(s))$. Among
sequents, we shall have:

\index{sequent@Initial sequent}\index{phi@$\phi\vdash\phi$ : initial
sequent}\index{svo@$\svo$ : set of initial sequents}
\begin{defin}\label{logic:def:GENZ:UAsequents:elementary:sequent}
Let $V$ be a set. We call {\em initial sequent} on $V$ any sequent
$(\phi\vdash\phi)$ with $\phi\in\pv$. The set of initial sequents on
$V$ is denoted \svo.
\end{defin}

\index{free@Free variables of sequent}\index{free@$\free(s)$ : free
variables of sequent}
\begin{defin}\label{logic:def:GENZ:setofsequents:freevar}
Let $V$ be a set. Given a sequent $s\in\sv$, we say that $x\in V$ is
a free variable of $s$, \ifand\ it belongs to the set $\free(s)$
defined by:
    \[
    \free(s)=\free(\,\Gamma(s)\cup\Delta(s)\,)
    \]
\end{defin}
