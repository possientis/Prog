In this section we prove the soundness theorem, which can be
expressed as:
    \[
    \vdash\phi\ \Rightarrow\ M\vDash\phi[a]
    \]
In other words, any provable formula $\phi$ is true in any model $M$
under any assignment $a:V\to M$. If $\beta:\pv\to {\cal P}(M^{V})$
denotes the model valuation function of $M$, the soundness theorem
becomes $\vdash\phi\ \Rightarrow\ \beta(\phi)(a)=1$. This is hugely
important for us: we already know from
definition~(\ref{logic:def:FOPL:model:valuation:function}) that
$\beta(\bot)(a)=0$ and
$\beta(\phi_{1}\to\phi_{2})(a)=\beta(\phi_{1})(a)\to\beta(\phi_{2})(a)$.
So the map $\beta(\,\cdot\,)(a):\pv\to 2$ is a propositional
valuation as per
definition~(\ref{logic:def:FOPL:propcong:prop:valuation}). Knowing
furthermore that the implication $\vdash\phi\ \Rightarrow\
\beta(\phi)(a)=1$ holds allows us to claim that
$\beta(\,\cdot\,)(a)$ is in fact a valuation as per
definition~(\ref{logic:def:FOPL:semantics:valuation}). So the dual
space \pvd\ is not empty: take any set $M$ together with a binary
relation $r$, take any assignment $a:V\to M$ and you obtain a
corresponding element $\beta(\,\cdot\,)(a)$ of the dual space \pvd.
This means our relation $\vDash$ of semantic entailment as per
definition~(\ref{logic:def:FOPL:semantics:entailments}) is not
trivial. It also means the sequent $\vdash\bot$ is false as
otherwise the dual space \pvd\ would clearly be empty. Our deductive
system is therefore consistent which is an important validation. The
soundness theorem will be quoted as
theorem~(\ref{logic:the:FOPL:soundness:soundness}) of
page~\pageref{logic:the:FOPL:soundness:soundness} below. We start
with a small lemma which establishes the truth of any axiom of first
order logic under any model and assignment:

\index{truth@Truth of axioms under all models}
\begin{lemma}\label{logic:lemma:FOPL:soundness:axioms}
Let $V$ be a set. Let $M$ be a model of\, \pv\ and $a:V\to M$ be an
arbitrary variables assignment. Then for all  $\phi\in\av$ we have:
    \[
    M\vDash\phi[a]
    \]
i.e. an axiom of first order logic is true under any model and
assignment.
\end{lemma}
\begin{proof}
Let $\beta:\pv\to{\cal P}(M^{V})$ be the model valuation function
associated with the model $M$. Given an assignment $a:V\to M$, we
need to show that $\beta(\phi)(a)=1$ for every $\phi\in\av$. We
shall consider the five possible cases of axioms individually. First
we assume that $\phi$ is a simplification axiom. From
definition~(\ref{logic:def:FOPL:simplification:axiom}), there exist
$\phi_{1},\phi_{2}\in\pv$ such that $\phi =
\phi_{1}\to(\phi_{2}\to\phi_{1})$. We need to show that
$\beta(\phi)(a)=1$. However, defining $v_{1}=\beta(\phi_{1})(a)$ and
$v_{2}=\beta(\phi_{2})(a)$\,:
    \[
    \beta(\phi)(a)=v_{1}\to(v_{2}\to v_{1})
    \]
where we have used
definition~(\ref{logic:def:FOPL:model:valuation:function}). So
suppose $\beta(\phi)(a)=0$. Then $v_{1}=1$ while $v_{2}\to v_{1}=0$.
From this last equality we see that $v_{2}=1$ while $v_{1}=0$, which
is a contradiction. So $\beta(\phi)(a)=1$ is proved. We now assume
that $\phi$ is a Frege axiom. From
definition~(\ref{logic:def:FOPL:frege:axiom}) there exist
$\phi_{1},\phi_{2},\phi_{3}\in\pv$ such that $\phi =
    [\phi_{1}\to(\phi_{2}\to\phi_{3})]\to[(\phi_{1}\to\phi_{2})\to(\phi_{1}\to\phi_{3})]$.
Defining $v_{1}=\beta(\phi_{1})(a)$, $v_{2}=\beta(\phi_{2})(a)$ and
$v_{3}=\beta(\phi_{3})(a)$ we obtain from
definition~(\ref{logic:def:FOPL:model:valuation:function})\,:
    \[
    \beta(\phi)(a) =
    [v_{1}\to(v_{2}\to v_{3})]\to[(v_{1}\to v_{2})\to(v_{1}\to v_{3})]
    \]
We need to show that $\beta(\phi)(a)=1$. So suppose to the contrary
that $\beta(\phi)(a)=0$. Then we have $v_{1}\to(v_{2}\to v_{3})=1$
while $(v_{1}\to v_{2})\to(v_{1}\to v_{3})=0$. From this last
equality we see that $v_{1}\to v_{2}=1$ while $v_{1}\to v_{3}=0$,
which in turn implies that $v_{1}=1$ while $v_{3}=0$. However, from
$v_{1}=1$ and $v_{1}\to v_{2}=1$ we obtain $v_{2}=1$. So we have
proved that $(v_{1},v_{2},v_{3})=(1,1,0)$ and consequently:
    \[
    v_{1}\to(v_{2}\to v_{3})=1\to(1\to 0)=1\to 0=0
    \]
which is a contradiction. So we now assume that $\phi$ is a
transposition axiom. From
definition~(\ref{logic:def:FOPL:transposition:axiom}) we have $\phi
=[(\phi_{1}\to\bot)\to\bot]\to\phi_{1}$ for some $\phi_{1}\in\pv$.
Defining $v_{1}=\beta(\phi_{1})(a)$ from
definition~(\ref{logic:def:FOPL:model:valuation:function}) we obtain
the equality:
    \[
    \beta(\phi)(a)=[(v_{1}\to 0)\to 0]\to v_{1}
    \]
We need to show that $\beta(\phi)(a)=1$. So suppose to the contrary
that $\beta(\phi)(a)=0$. Then $(v_{1}\to 0)\to 0=1$ while $v_{1}=0$.
It follows that:
    \[
    (v_{1}\to 0)\to 0=(0\to 0)\to 0= 1\to 0 = 0
    \]
which is a contradiction. So we now assume that $\phi$ is a
quantification axiom. From
definition~(\ref{logic:def:FOPL:quantification:axiom}), there exist
$\phi_{1},\phi_{2}\in\pv$ and $x\in V$ with
$x\not\in\free(\phi_{1})$ and $\phi =\forall
x(\phi_{1}\to\phi_{2})\to(\phi_{1}\to\forall x \phi_{2})$. Defining
$v_{1}=\beta(\phi_{1})(a)$, $v_{2}=\beta(\forall x\phi_{2})(a)$ and
$v_{3}=\beta(\forall x(\phi_{1}\to\phi_{2}))(a)$ we obtain the
equality:
    \[
    \beta(\phi)(a) = v_{3}\to(v_{1}\to v_{2})
    \]
We need to show that $\beta(\phi)(a)=1$. So suppose to the contrary
that $\beta(\phi)(a)=0$. Then we have $v_{3}=1$ while $v_{1}\to
v_{2}=0$ which in turn implies that $v_{1}=1$ while $v_{2}=0$. Using
definition~(\ref{logic:def:FOPL:model:valuation:function}) we
therefore obtain:
    \[
    0=v_{2}=\beta(\forall x\phi_{2})(a)=\min\left\{\beta(\phi_{2})(b)\
    :\ b=a\mbox{\ on\ }V_{x}\right\}
    \]
Hence, we see that there exists an assignment $b^{*}:V\to M$ such
that $b^{*}=a$ on $V\setminus\{x\}$ and $\beta(\phi_{2})(b^{*})=0$.
It follows that we have:
    \begin{eqnarray*}
    v_{3}&=&\beta(\forall x(\phi_{1}\to\phi_{2}))(a)\\
    &=&\min\left\{\beta(\phi_{1}\to\phi_{2})(b)\
    :\ b=a\mbox{\ on\ }V_{x}\right\}\\
    &\leq&\beta(\phi_{1}\to\phi_{2})(b^{*})\\
    &=&\beta(\phi_{1})(b^{*})\to\beta(\phi_{2})(b^{*})\\
    &=&\beta(\phi_{1})(b^{*})\to 0\\
    \mbox{A: to be proved}\ \rightarrow&=&\beta(\phi_{1})(a)\to 0\\
    &=&v_{1}\to 0\\
    &=&1\to 0\\
    &=&0
    \end{eqnarray*}
which contradicts $v_{3}=1$. So it remains to show that
$\beta(\phi_{1})(b^{*})=\beta(\phi_{1})(a)$. Using
proposition~(\ref{logic:prop:FOPL:model:assignment:support}) it is
sufficient to show that $b^{*}=a$ on $\free(\phi_{1})$, which
follows from $\free(\phi_{1})\subseteq V\setminus\{x\}$, itself a
consequence of the assumption $x\not\in\free(\phi_{1})$. So we now
assume that $\phi$ is a specialization axiom. From
definition~(\ref{logic:def:FOPL:specialization:axiom}) there exist
$\phi_{1}\in\pv$ and $x,y\in V$ such that $\phi =\forall
x\phi_{1}\to\phi_{1}[y/x]$ where $[y/x]:\pv\to\pv$ denotes an
essential substitution of $y$ in place of $x$, i.e. an essential
substitution associated with the map $[y/x]:V\to V$. Defining
$v_{1}=\beta(\forall x\phi_{1})(a)$ and
$v_{2}=\beta(\phi_{1}[y/x])(a)$ we obtain $\beta(\phi)(a)=v_{1}\to
v_{2}$ and we need to show that $\beta(\phi)(a)=1$. So suppose to
the contrary that $\beta(\phi)(a)=0$. Then we have $v_{1}=1$ while
$v_{2}=0$ and consequently we obtain:
    \begin{eqnarray*}
    v_{1}&=&\beta(\forall x\phi_{1})(a)\\
    &=&\min\left\{\beta(\phi_{1})(b)\ :\ b=a\mbox{\ on\
    }V_{x}\right\}\\
    a\circ[y/x]=a\mbox{\ on\ }V_{x}\ \rightarrow
    &\leq&\beta(\phi_{1})(\,a\circ[y/x]\,)\\
    \mbox{theorem~(\ref{logic:the:FOPL:model:essential:substitution}),
    p.~\pageref{logic:the:FOPL:model:essential:substitution}}
    \ \rightarrow&=&\beta(\,[y/x](\phi_{1})\,)(a)\\
    &=&\beta(\phi_{1}[y/x])(a)\\
    &=&v_{2}\\
    &=&0
    \end{eqnarray*}
which contradicts the equality $v_{1}=1$ and completes our proof.
\end{proof}

The proof of the soundness theorem which follows relies on a
structural induction argument based on our free universal algebra of
proofs \pvs.

\index{soundness@Soundness theorem}
\begin{theorem}[Soundness]\label{logic:the:FOPL:soundness:soundness}
Let $V$ be a set and $M$ be a model of\, \pv. Then for every
assignment $a:V\to M$ and any formula $\phi\in\pv$ we have:
    \[
    \vdash\phi\ \Rightarrow\ M\vDash\phi[a]
    \]
In other words, a provable formula is true under any model and
assignment.
\end{theorem}
\begin{proof}
Let $\beta:\pv\to{\cal P}(M^{V})$ denote the model valuation
function of the model $M$. Given a provable formula $\phi\in\pv$ we
need to show that $\beta(\phi)(a)=1$ for every assignment $a:V\to
M$. It is therefore sufficient to prove that for every proof
$\pi\in\pvs$ and every assignment $a:V\to M$, we have the
implication:
    \begin{equation}\label{logic:eqn:FOPL:soundness:proof:1}
    \hyp(\pi)=\emptyset\ \Rightarrow\ \beta(\val(\pi))(a)=1
    \end{equation}
Indeed, suppose this property has been established and let
$\phi\in\pv$ be such that $\vdash\phi$. Then there exists a proof
$\pi\in\pvs$ such that $\val(\pi)=\phi$ and $\hyp(\pi)=\emptyset$.
Using~(\ref{logic:eqn:FOPL:soundness:proof:1}) we obtain
$\beta(\phi)(a)=1$ for all $a:V\to M$. So we shall now prove
that~(\ref{logic:eqn:FOPL:soundness:proof:1}) is true for every
assignment and every proof $\pi\in\pvs$. We shall do so by
structural induction, using
theorem~(\ref{logic:the:proof:induction}) of
page~\pageref{logic:the:proof:induction}. First we assume that
$\pi=\phi$ for some $\phi\in\pv$. Then
$\hyp(\pi)=\{\phi\}\neq\emptyset$
and~(\ref{logic:eqn:FOPL:soundness:proof:1}) is vacuously true. Next
we assume that $\pi=\axi\phi$ for some $\phi\in\pv$. We shall
distinguish two cases: first we assume that $\phi\not\in\av$. Then
$\hyp(\pi)=\emptyset$ and $\val(\pi)=\bot\to\bot$. So we need to
show that $\beta(\bot\to\bot)(a)=1$ for every assignment, which
follows immediately from
definition~(\ref{logic:def:FOPL:model:valuation:function}). Next we
assume that $\phi\in\av$. Then $\hyp(\pi)=\emptyset$ and
$\val(\pi)=\phi$. So we need to show that $\beta(\phi)(a)=1$ for
every assignment, which follows immediately from
lemma~(\ref{logic:lemma:FOPL:soundness:axioms}). So we now assume
that $\pi=\pi_{1}\pon\pi_{2}$ where $\pi_{1},\pi_{2}\in\pvs$ are
such that~(\ref{logic:eqn:FOPL:soundness:proof:1}) is true for every
assignment. We need to show the same is true of $\pi$. So let
$a:V\to M$ be an assignment and suppose $\hyp(\pi)=\emptyset$. We
need to show that $\beta(\val(\pi))(a)=1$. We shall distinguish two
cases: first we assume that $\val(\pi_{2})$ cannot be expressed as
$\val(\pi_{2})=\val(\pi_{1})\to\phi$ for any $\phi\in\pv$. Then we
have $\val(\pi)=M(\val(\pi_{1}),\val(\pi_{2}))=\bot\to\bot$ where
$M:\pv^{2}\rightarrow \pv$ is the modus ponens mapping of
definition~(\ref{logic:def:FOPL:modus:ponens}). Hence we have:
    \begin{eqnarray*}
    \beta(\val(\pi))(a)&=&\beta(\bot\to\bot)(a)\\
    &=&\beta(\bot)(a)\to\beta(\bot)(a)\\
    &=&0\to 0\\
    &=&1
    \end{eqnarray*}
So we now assume that $\val(\pi_{2})=\val(\pi_{1})\to\phi$ for some
$\phi\in\pv$. In this case we have
$\val(\pi)=M(\val(\pi_{1}),\val(\pi_{2}))=\phi$ and we need to show
that $\beta(\phi)(a)=1$. However, from
$\emptyset=\hyp(\pi)=\hyp(\pi_{1})\cup\hyp(\pi_{2})$ we see that
both $\hyp(\pi_{1})$ and $\hyp(\pi_{2})$ are the empty set. Having
assumed the implication~(\ref{logic:eqn:FOPL:soundness:proof:1}) is
true for $\pi_{1}$ and $\pi_{2}$ it follows that
$\beta(\val(\pi_{1}))(a)=1$ and furthermore:
    \begin{eqnarray*}
    1&=&\beta(\val(\pi_{2}))(a)\\
    &=&\beta(\,\val(\pi_{1})\to\phi\,)(a)\\
    &=&\beta(\val(\pi_{1}))(a)\to\beta(\phi)(a)\\
    &=&1\to\beta(\phi)(a)\\
    \end{eqnarray*}
So we conclude that $\beta(\phi)(a)=1$ as requested. We now assume
that $\pi=\gen x\pi_{1}$ where $x\in V$ and $\pi_{1}\in\pvs$ is such
that~(\ref{logic:eqn:FOPL:soundness:proof:1}) is true for every
assignment. We need to show the same if true of $\pi$. So let
$a:V\to M$ be an assignment and suppose $\hyp(\pi)=\emptyset$. We
need to show that $\beta(\val(\pi))(a)=1$. Since we have
$\hyp(\pi_{1})=\hyp(\pi)=\emptyset$ we see that
$x\not\in\free(\pi_{1})$. So $\val(\pi)=\forall x\phi_{1}$ with
$\phi_{1}=\val(\pi_{1})$ and we need to show that $\beta(\forall
x\phi_{1})(a)=1$. However from $\hyp(\pi_{1})=\emptyset$, having
assumed the implication~(\ref{logic:eqn:FOPL:soundness:proof:1}) is
true for $\pi_{1}$ and every assignment, we have
$\beta(\phi_{1})(b)=1$ for every assignment $b:V\to M$. It follows
that $\beta(\forall
x\phi_{1})(a)=\min\{\beta(\phi_{1})(b):b=a\mbox{\ on\
    }V_{x}\}=1$ as requested.
\end{proof}
\begin{theorem}\label{logic:the:FOPL:soundness:soundness:2}
Let $V$ be a set and $M$ be a model of\, \pv\ with model valuation
function $\beta$. For every $a:V\to M$ the map
$\beta(\,.\,)(a):\pv\to 2$ is a valuation.
\end{theorem}
\begin{proof}
Let $a:V\to M$ be an assignment and define $v_{a}:\pv\to 2$ by
setting $v_{a}(\phi)=\beta(\phi)(a)$. We need to show that $v_{a}$
is a valuation. The fact that $v_{a}(\bot)=0$ and
$v_{a}(\phi_{1}\to\phi_{2})=v_{a}(\phi_{1})\to v_{a}(\phi_{2})$ for
all $\phi_{1},\phi_{2}\in\pv$ follows immediately from
definition~(\ref{logic:def:FOPL:model:valuation:function}).
Furthermore, the implication $\vdash\phi\ \Rightarrow\
v_{a}(\phi)=1$ follows from
theorem~(\ref{logic:the:FOPL:soundness:soundness}). By virtue of
definition~(\ref{logic:def:FOPL:semantics:valuation}), $v_{a}$ is a
valuation.
\end{proof}
