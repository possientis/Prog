In this section we introduce yet another congruence on \pv\ which we
have chosen to call the {\em Hilbert deductive congruence}, hoping
this will make it obvious to the familiar reader. Another possible
name could have been the {\em Lindenbaum-Tarski congruence} since it
is the congruence giving rise to the Lindenbaum-Tarski algebra, a
seemingly well established term in mathematical logic. The Hilbert
deductive congruence is probably not what we are looking for to
define our {\em Universal Algebra of First Order Logic}. We are
looking to identify mathematical statements which have the same
meaning, and such identification should be decidable. We are not
looking for theorems. However, the study of the Hilbert deductive
congruence gives us a great opportunity to introduce fundamental
notions of mathematical logic which we shall no doubt require at a
later stage: these are the concept of proof and provability, the
deduction theorem, the notion of semantics and model, and of course
G\"odel's completeness theorem. These are the bread and butter of
any textbook on mathematical logic and it is about time we touch
upon them. So our first objective will be to define the notion of
proof and provability, which is the question of axiomatization of
first order logic. Having reviewed some of the existing textbooks
and references, it is astonishing to see how little consensus there
is on the subject. It is impossible to find two references which
will universally agree on which axioms should be used, or which
rules of inference. So we had to make our own mind and decide for
ourselves what constitutes a {\em good} axiomatization of first
order logic, and here it is: firstly, a good axiomatization of first
order logic should be {\em sound} and ensure G\"odel's completeness
theorem holds. This is rather uncontroversial so we shall not say
more about it. Secondly, The deduction theorem should hold {\em
without any form of restriction}. Most references will depart from
this principle and quote the deduction theorem imposing some form of
restriction, typically that a formula should be closed (e.g.
\cite{Ferenczi}, \cite{Hoyois}, \cite{Johnstone}, \cite{Kunen},
\cite{Mendelson}, \cite{Monk}, \cite{Metamath}, \cite{Tourlakis}).
The only exception known to us is~\cite{AlgLog} where the deduction
theorem is restored to its full glory. This we believe should be the
case. There is also a web page~\cite{Planet} on
\texttt{http://planetmath.org} indicating the awareness by some
living mathematicians  that the deduction theorem need not fail in
first order logic, provided the concept of proof is carefully
defined. This brings us to our third requirement which a good
axiomatization of first order logic ought to meet: the concept of
proof {\em should not} be presented in terms of finite sequences of
formulas. Instead, proofs should be defined as formal expressions in
a free universal algebra generated by the formulas of the language
itself (the set of possible hypothesis) and whose operators are the
chosen rules of inference: to every formula $\phi$ corresponds a
proof $\pi=\axi \phi$ arising from a constant operator, indicating
that $\phi$ is invoked as an axiom; to every proof $\pi$ corresponds
another proof $\gen x\pi$ arising from a unary operator indicating a
{\em generalization} with respect to the variable $x$; to every pair
of proof $(\pi_{1},\pi_{2})$ corresponds another proof $\pi_{1}\pon
\pi_{2}$ arising from a binary operator, that of {\em modus ponens}
for example\ldots On this algebra of proofs should be defined a key
semantics associating every proof to its conclusion. By imposing
this free algebraic structure on proofs, we are preserving their
natural tree structure, giving us the full power of structural
induction and structural recursion which we cannot enjoy with the
linear structure of finite sequences. These should be abandoned.
Some authors still want to define formulas as finite sequences of
characters and will go as far as introducing a concatenation
operator on their {\em strings}, explaining why the parenthesis $($
and $)$ should be used. At best, they are just creating a free
universal algebra of formulas which could have been defined in a
generic way, up to isomorphism. At worse, by insisting on the
details of their {\em string} implementation, they are making
everything very awkward to prove. Finite sequences are not suited to
mathematical logic.
