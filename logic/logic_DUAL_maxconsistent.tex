In this section we introduce the notion of {\em consistency} and
define {\em maximal consistent subsets} leading up to Lindenbaum's
lemma~(\ref{logic:lemma:FOPL:semantics:lindenbaum}) in the following
section.

\index{consistent@Consistent subset
$\Gamma$}\index{gamma@$\Gamma\vdash\bot$ : $\Gamma$ is inconsistent}
\begin{defin}\label{logic:def:FOPL:semantics:consistent:subset}
Let $V$ be a set and $\Gamma\subseteq\pv$. We say that $\Gamma$ is
{\em consistent} \ifand\ there exists no proof of $\bot$ from
$\Gamma$, i.e. the sequent $\Gamma\vdash\bot$ is false.
\end{defin}

It is clear that any subset $\Delta$ which is larger than an
inconsistent subset $\Gamma$ is itself inconsistent. At this point
in time, we do not know of any subset $\Gamma$ which is consistent.
Since we have not yet proved that the sequent $\vdash\bot$ is false,
it may be that the empty set itself is inconsistent. If this were
the case, then any sequent $\Gamma\vdash\phi$ would be true, as the
following proposition shows:

\index{inconsistent@Inconsistency proves everything}
\begin{prop}\label{logic:prop:FOPL:semantics:inconsistent:everything}
Let $V$ be a set and $\Gamma\subseteq\pv$ be an inconsistent
subset\,:
    \[
    \forall\phi\in\pv\ ,\ \Gamma\vdash\phi
    \]
In other words, every formula can be proved from an inconsistent
subset.
\end{prop}
\begin{proof}
We assume that $\Gamma$ is inconsistent, i.e. $\Gamma\vdash\bot$.
Given $\phi\in\pv$, we need to show that $\Gamma\vdash\phi$.
However, using the simplification property of
proposition~(\ref{logic:prop:FOPL:simplification}) we obtain
$\Gamma\vdash[(\phi\to\bot)\to\bot]$. Hence we see from the
transposition property of
proposition~(\ref{logic:prop:FOPL:transposition}) that
$\Gamma\vdash\phi$ as requested.
\end{proof}

If a statement $\phi$ cannot be proven from a set of premises
$\Gamma$, then in particular $\Gamma$ must be consistent by virtue
of
proposition~(\ref{logic:prop:FOPL:semantics:inconsistent:everything}).
In fact, we can add the negation of $\phi$ to the set $\Gamma$ and
still obtain a consistent set. In short, if something is not
provable, adding its negation to a set of premises is consistent.

\index{provability@Provability and
consistency}\index{consistentcy@Consistency and provability}
\begin{prop}\label{logic:prop:FOPL:semantics:consistent:sequent}
Let $V$ be a set. Then for all $\phi\in\pv$ and
$\Gamma\subseteq\pv$\,:
    \[
    \Gamma\vdash\phi\mbox{\ is false}\ \Leftrightarrow\
    \Gamma\cup\{\phi\to\bot\}\mbox{\ is consistent}
    \]
\end{prop}
\begin{proof}
First we show $\Rightarrow$\,: so suppose the sequent
$\Gamma\vdash\phi$ is false. We need to show that
$\Gamma\cup\{\phi\to\bot\}$ is consistent. Suppose to the contrary
that $\Gamma\cup\{\phi\to\bot\}$ is not consistent. Then we can
deduce a contradiction from it, i.e. we have
$\Gamma\cup\{\phi\to\bot\}\vdash\bot$. Using the deduction
theorem~(\ref{logic:the:FOPL:deduction}) of
page~\pageref{logic:the:FOPL:deduction} we obtain
$\Gamma\vdash(\phi\to\bot)\to\bot$ and consequently, from the
transposition property of
proposition~(\ref{logic:prop:FOPL:transposition}) we conclude that
$\Gamma\vdash\phi$ which is a contradiction. We now prove
$\Leftarrow$\,: so we assume that $\Gamma\cup\{\phi\to\bot\}$ is
consistent. We need to show that $\Gamma\vdash\phi$ is false. So
suppose to the contrary that $\Gamma\vdash\phi$ is true. Then in
particular we have $\Gamma\cup\{\phi\to\bot\}\vdash\phi$. However,
it is clear that $\Gamma\cup\{\phi\to\bot\}\vdash(\phi\to\bot)$.
Using the modus ponens property of
proposition~(\ref{logic:prop:FOPL:modus:ponens}) we obtain
$\Gamma\cup\{\phi\to\bot\}\vdash\bot$, which contradicts our
assumption that $\Gamma\cup\{\phi\to\bot\}$ is consistent.
\end{proof}

As we shall see in
theorem~(\ref{logic:the:FOPL:semantics:satis:equiv:consistent}) of
page~\pageref{logic:the:FOPL:semantics:satis:equiv:consistent}, the
notions of {\em consistent} and {\em satisfiable} subsets are in
fact equivalent. This will allow us to derive the compactness
theorem~(\ref{logic:the:FOPL:semantics:compactness}) of
page~\pageref{logic:the:FOPL:semantics:compactness}. We can now
prove one side of the equivalence:

\begin{prop}\label{logic:prop:FOPL:semantics:satis:imp:consistent}
Let $V$ be a set and $\Gamma\subseteq\pv$. Then we have the
implication:
    \[
    \Gamma\mbox{\ is satisfiable}\ \Rightarrow\ \Gamma\mbox{\
    is consistent}
    \]
\end{prop}
\begin{proof}
We assume that $\Gamma$ is satisfiable. So there exists a valuation
$v\in\pvd$ which satisfies $\Gamma$, i.e. such that $v\vDash\Gamma$.
We need to show that $\Gamma$ is consistent. So suppose to the
contrary that $\Gamma$ is not consistent. Then the sequent
$\Gamma\vdash\bot$ is true. Using
proposition~(\ref{logic:prop:FOPL:semantics:syn:imp:sem}) it follows
that $\Gamma\vDash\bot$. Since the valuation $v$ satisfies $\Gamma$
we conclude that it also satisfies $\bot$, i.e. $v\vDash\bot$. Hence
we obtain $v(\bot)=1$ which contradicts
definition~(\ref{logic:def:FOPL:semantics:valuation}) of $v$ being a
valuation on \pv.
\end{proof}

The subsets $\Gamma\subseteq\pv$ which are consistent form a subset
$X$ of the power set ${\cal P}(\pv)$. In other words, they form a
set of subsets of \pv. This set $X$ can be partially ordered by
inclusion. A maximal consistent set $\Gamma$ is an element of $X$
which is a maximal element with respect to this partial order.
Recall that given a partially ordered set $(X,\leq)$, an element
$a\in X$ is said to be {\em maximal} \ifand\ for all $b\in X$ we
have the implication:
    \[
    a\leq b\ \Rightarrow\ a=b
    \]
We shall need to remember this when stating Zorn's lemma in the next
section.

\index{maximal@Maximal consistent subset}\index{consistent@Maximal
consistent subset $\Gamma$}
\begin{defin}\label{logic:def:FOPL:semantics:maximal:consistent}
Let $V$ be a set and $\Gamma\subseteq\pv$. We say that $\Gamma$ is
{\em maximal consistent} \ifand\ it is consistent and for all
$\Delta\subseteq\pv$ we have:
    \[
    (\Gamma\subseteq\Delta)\land(\Delta\mbox{\ consistent})\
    \Rightarrow \Gamma=\Delta
    \]
\end{defin}

Maximal consistent subsets have many interesting properties. In
particular, they are {\em closed under syntactic entailment}. In
other words, anything which can be proven from a maximal consistent
subset, actually belongs to that subset.

\begin{prop}\label{logic:prop:FOPL:semantics:max:cons:deduction}
Let $V$ be a set and $\Gamma\subseteq\pv$ be maximal consistent.
Then:
    \[
    \forall\phi\in\pv\ ,\ (\,\Gamma\vdash\phi\ \Rightarrow\
    \phi\in\Gamma\,)
    \]
\end{prop}
\begin{proof}
We assume that $\Gamma\subseteq\pv$ is maximal consistent. Let
$\phi\in\pv$ be such that $\Gamma\vdash\phi$. We need to show that
$\phi\in\Gamma$. So suppose to the contrary that
$\phi\not\in\Gamma$. Then $\Delta=\Gamma\cup\{\phi\}$ is a set which
is strictly larger than $\Gamma$. Having assumed that $\Gamma$ is
maximal consistent, $\Delta$ cannot be consistent. It follows that
$\Delta\vdash\bot$ which is $\Gamma\cup\{\phi\}\vdash\bot$. Using
the deduction theorem~(\ref{logic:the:FOPL:deduction}) of
page~\pageref{logic:the:FOPL:deduction} we obtain
$\Gamma\vdash(\phi\to\bot)$. However, by assumption
$\Gamma\vdash\phi$. Using the modus ponens property of
proposition~(\ref{logic:prop:FOPL:modus:ponens}) we see that
$\Gamma\vdash\bot$, contradicting the consistency of $\Gamma$.
\end{proof}

Another interesting property of maximal consistent subsets is the
fact they always contain a formula $\phi$ or its negation
$\phi\to\bot$, for all $\phi\in\pv$. Obviously they cannot contain
both without being inconsistent. Hence we have:

\begin{prop}\label{logic:prop:FOPL:semantics:max:cons:dychotomy}
Let $V$ be a set and $\Gamma\subseteq\pv$ be maximal consistent.
Then for all $\phi\in\pv$ we have $\phi\in\Gamma$ or
$(\phi\to\bot)\in\Gamma$, and these are exclusive.
\end{prop}
\begin{proof}
If $\Gamma$ is consistent, it is clear we cannot have both
$\phi\in\Gamma$ and $(\phi\to\bot)\in\Gamma$, as otherwise the
sequents $\Gamma\vdash\phi$ and $\Gamma\vdash(\phi\to\bot)$ would
both be true, and using the modus ponens property of
proposition~(\ref{logic:prop:FOPL:modus:ponens}), this would imply
$\Gamma\vdash\bot$ contradicting the consistency of $\Gamma$. So we
assume that $\Gamma\subseteq\pv$ is maximal consistent and given
$\phi\in\pv$, we need to show that $\phi\in\Gamma$ or
$(\phi\to\bot)\in\Gamma$. So we assume that
$(\phi\to\bot)\not\in\Gamma$. We need to show that $\phi\in\Gamma$.
However, having assumed $(\phi\to\bot)\not\in\Gamma$, the set
$\Delta=\Gamma\cup\{\phi\to\bot\}$ is a set which is strictly larger
than $\Gamma$. Having assumed $\Gamma$ is maximal consistent, it
follows that $\Delta$ cannot be consistent. So
$\Gamma\cup\{\phi\to\bot\}$ is not consistent and consequently from
proposition~(\ref{logic:prop:FOPL:semantics:consistent:sequent}) we
see that the sequent $\Gamma\vdash\phi$ is true. Since $\Gamma$ is
maximal consistent, using
proposition~(\ref{logic:prop:FOPL:semantics:max:cons:deduction}) we
conclude that $\phi\in\Gamma$ as requested.
\end{proof}

A maximal consistent subset can be thought of as the set of {\em
true} formulas of a valuation $v\in\pvd$. This fact will be made
precise after we show the following proposition,  which expresses
the familiar idea that an implication $\phi_{1}\to\phi_{2}$ is {\em
true}, \ifand\ $\phi_{1}$ is {\em false} or $\phi_{2}$ is {\em
true}. More precisely, we have:

\begin{prop}\label{logic:prop:FOPL:max:cons:implication:charac}
Let $V$ be a set and $\Gamma\subseteq\pv$ be a maximal consistent
subset. Then for all formulas $\phi_{1},\phi_{2}\in\pv$ we have the
equivalence:
    \[
    (\phi_{1}\to\phi_{2})\in\Gamma\ \Leftrightarrow\
    (\phi_{1}\not\in\Gamma)\lor(\phi_{2}\in\Gamma)
    \]
\end{prop}
\begin{proof}
First we prove $\Rightarrow$\,: so we assume that
$(\phi_{1}\to\phi_{2})\in\Gamma$. We need to show that
$\phi_{1}\not\in\Gamma$ or $\phi_{2}\in\Gamma$. So suppose
$\phi_{1}\in\Gamma$. We need to show that $\phi_{2}\in\Gamma$.
However, we obviously have $\Gamma\vdash(\phi_{1}\to\phi_{2})$ and
$\Gamma\vdash\phi_{1}$. Using the modus ponens property of
proposition~(\ref{logic:prop:FOPL:modus:ponens}) we obtain
$\Gamma\vdash\phi_{2}$. Having assumed $\Gamma$ is maximal
consistent, it follows from
proposition~(\ref{logic:prop:FOPL:semantics:max:cons:deduction})
that $\phi_{2}\in\Gamma$ as requested. We now prove $\Leftarrow$\,:
so we assume that $\phi_{1}\not\in\Gamma$ or $\phi_{2}\in\Gamma$. We
need to show that $(\phi_{1}\to\phi_{2})\in\Gamma$. Using
proposition~(\ref{logic:prop:FOPL:semantics:max:cons:deduction})
once more, it is sufficient to prove that
$\Gamma\vdash(\phi_{1}\to\phi_{2})$. From the deduction
theorem~(\ref{logic:the:FOPL:deduction}) of
page~\pageref{logic:the:FOPL:deduction}, it is therefore sufficient
to show that $\Gamma\cup\{\phi_{1}\}\vdash\phi_{2}$. This is clearly
true if $\phi_{2}\in\Gamma$, so we may assume that
$\phi_{2}\not\in\Gamma$. However by assumption, this implies that
$\phi_{1}\not\in\Gamma$. Having assumed $\Gamma$ is maximal
consistent, from
proposition~(\ref{logic:prop:FOPL:semantics:max:cons:dychotomy}) it
follows that $(\phi_{1}\to\bot)\in\Gamma$ and in particular we have
$\Gamma\vdash(\phi_{1}\to\bot)$. Using the deduction theorem once
more, we see that $\Gamma\cup\{\phi_{1}\}\vdash\bot$. So the subset
$\Gamma\cup\{\phi_{1}\}$ is inconsistent and the required sequent
$\Gamma\cup\{\phi_{1}\}\vdash\phi_{2}$ follows from
proposition~(\ref{logic:prop:FOPL:semantics:inconsistent:everything}).
\end{proof}

The following proposition establishes a bijection between the dual
space \pvd\ and the set of maximal consistent subsets of \pv. This
gives us valuable insight, and allows us to view any valuation
$v:\pv\to 2$ as a maximal consistent subset, or to regard any
maximal consistent subset as a valuation. This result will be
particularly interesting once we show that $\pvd\neq\emptyset$.

\index{maximal@Maximal consistent and
valuation}\index{valuation@Valuation and maximal consistent}
\begin{prop}\label{logic:prop:FOPL:semantics:bijection:max:cons:val}
Let $V$ be a set and $v:\pv\to 2$ be a valuation. Consider:
    \[
    \Gamma_{v}=\{\phi\in\pv:v(\phi)=1\}
    \]
Then $\Gamma_{v}$ is maximal consistent. Conversely, let $\Gamma$ be
maximal consistent and:
     \[
     \forall \phi\in\pv\ ,\ 1_{\Gamma}(\phi)=\left\{
        \begin{array}{lcl}
        1&\mbox{\ if\ }&\phi\in\Gamma\\
        0&\mbox{\ if\ }&\phi\not\in\Gamma
        \end{array}
    \right.
    \]
Then the characteristic function $1_{\Gamma}:\pv\to 2$ is a
valuation. Furthermore, the maps $v\to\Gamma_{v}$ and $\Gamma\to
1_{\Gamma}$ are bijections which are inverse of each other, between
the dual space \pvd\ and the set of maximal consistent subsets of
\pv.
\end{prop}
\begin{proof}
We first consider a valuation $v:\pv\to 2$ and
$\Gamma_{v}=\{\phi\in\pv:v(\phi)=1\}$. We need to show that
$\Gamma_{v}$ is maximal consistent. First we show that $\Gamma_{v}$
is consistent. So suppose to the contrary that $\Gamma_{v}$ is not
consistent. Then we have $\Gamma_{v}\vdash\bot$. It follows from
proposition~(\ref{logic:prop:FOPL:semantics:syn:imp:sem}) that
$\Gamma_{v}\vDash\bot$. However, for all $\phi\in\Gamma_{v}$ we have
$v(\phi)=1$. So $v$ satisfies $\Gamma_{v}$, i.e. we have
$v\vDash\Gamma_{v}$. From $\Gamma_{v}\vDash\bot$ we therefore
conclude that $v\vDash\bot$ which is $v(\bot)=1$ and contradicts
definition~(\ref{logic:def:FOPL:semantics:valuation}) of $v$ being a
valuation on \pv. So we now show that $\Gamma_{v}$ is in fact
maximal consistent. So suppose $\Delta\subseteq\pv$ is a set which
is strictly larger than $\Gamma_{v}$. We need to show that $\Delta$
is not consistent, i.e. that $\Delta\vdash\bot$. However by
assumption the set $\Delta\setminus\Gamma_{v}$ is not empty. So let
$\phi\in\Delta\setminus\Gamma_{v}$. From $\phi\not\in\Gamma_{v}$ we
obtain $v(\phi)=0$ and consequently $v(\phi\to\bot)=v(\phi)\to
v(\bot)=v(\phi)\to 0=1$. It follows that
$(\phi\to\bot)\in\Gamma_{v}$. However, from
$\Gamma_{v}\subseteq\Delta$ we also have $(\phi\to\bot)\in\Delta$.
Furthermore, by assumption $\phi\in\Delta$. Hence we see that
$\Delta\vdash\phi$ and $\Delta\vdash(\phi\to\bot)$ are both true and
using the modus ponens property of
proposition~(\ref{logic:prop:FOPL:modus:ponens}) we obtain
$\Delta\vdash\bot$ as requested. So we have proved that $\Gamma_{v}$
is maximal consistent. We now assume that $\Gamma\subseteq\pv$ is
maximal consistent and we need to show that its characteristic
function $1_{\Gamma}:\pv\to 2$ is a valuation. First we show that
$1_{\Gamma}(\bot)=0$: we need to prove that $\bot\not\in\Gamma$.
This is clearly the case since $\bot\in\Gamma$ implies
$\Gamma\vdash\bot$ which would contradict the consistency of
$\Gamma$. Next we show that
$1_{\Gamma}(\phi_{1}\to\phi_{2})=1_{\Gamma}(\phi_{1})\to
1_{\Gamma}(\phi_{2})$ for all $\phi_{1},\phi_{2}\in\pv$. This goes
as follows:
    \begin{eqnarray*}
    1_{\Gamma}(\phi_{1}\to\phi_{2})=1&\Leftrightarrow&(\phi_{1}\to\phi_{2})\in\Gamma\\
    \mbox{prop.~(\ref{logic:prop:FOPL:max:cons:implication:charac})}
    \ \rightarrow&\Leftrightarrow&(\phi_{1}\not\in\Gamma)\lor(\phi_{2}\in\Gamma)\\
    &\Leftrightarrow&(1_{\Gamma}(\phi_{1})=0)\lor(1_{\Gamma}(\phi_{2})=1)\\
    &\Leftrightarrow&1_{\Gamma}(\phi_{1})\to1_{\Gamma}(\phi_{2})=1
    \end{eqnarray*}
Finally given $\phi\in\pv$, we need to show the implication
$\vdash\phi\ \Rightarrow\ 1_{\Gamma}(\phi)=1$. So we assume that
$\vdash\phi$. We need to show that $1_{\Gamma}(\phi)=1$, i.e. that
$\phi\in\Gamma$. However, from $\vdash\phi$ we have in particular
$\Gamma\vdash\phi$. Having assumed $\Gamma$ is maximal consistent,
using
proposition~(\ref{logic:prop:FOPL:semantics:max:cons:deduction}) we
obtain $\phi\in\Gamma$ as requested. So we have proved that
$1_{\Gamma}:\pv\to 2$ is a valuation on \pv. It remains to show that
the maps $v\to\Gamma_{v}$ and $\Gamma\to 1_{\Gamma}$ are bijections
which are inverse of one another, from $\pvd$ to the set of maximal
consistent subsets of \pv. Let us denote ${\cal C}(V)$ the set of
maximal consistent subsets of \pv, and define $f:\pvd\to{\cal C}(V)$
by $f(v)=\Gamma_{v}$ and $g:{\cal C}(V)\to\pvd$ by
$g(\Gamma)=1_{\Gamma}$. Until now we have proved that the range of
$f$ is indeed a subset of ${\cal C}(V)$ while the range of $g$ is
indeed a subset of \pvd. We need to show that $f$ and $g$ are
bijective and inverse of one another. First we show that $g\circ
f(v)=v$ for all $v\in\pvd$. This will show that $f$ is injective and
$g$ is surjective. We need to show that $1_{\Gamma_{v}}=v$. Let
$\phi\in\pv$\,:
    \[
    1_{\Gamma_{v}}(\phi)=1\ \Leftrightarrow\ \phi\in\Gamma_{v}\
    \Leftrightarrow\ v(\phi)=1
    \]
We now show that $f\circ g(\Gamma)=\Gamma$ for all $\Gamma\in{\cal
C}(V)$. This will show that $g$ is injective and $f$ is surjective.
We need to show that $\Gamma_{1_{\Gamma}}=\Gamma$. Let
$\phi\in\pv$\,:
    \[
    \phi\in\Gamma_{1_{\Gamma}}\ \Leftrightarrow\ 1_{\Gamma}(\phi)=1\
    \Leftrightarrow\ \phi\in\Gamma
    \]
\end{proof}
