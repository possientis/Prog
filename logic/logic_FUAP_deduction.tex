It is now time to deliver on our promise. Our chosen axiomatization
of first order logic allows us to state the deduction theorem in its
full generality. As already mentioned, a good number of references
(e.g. \cite{Ferenczi}, \cite{Hoyois}, \cite{Johnstone},
\cite{Kunen}, \cite{Mendelson}, \cite{Monk}, \cite{Metamath},
\cite{Tourlakis}) have chosen axiomatic systems in which the
deduction theorem fails in general, and can only be stated for
closed formulas. As far as we can tell, the most common source of
failure is the acceptance of $(\phi_{1},\forall x\phi_{1})$ as a
rule of inference. This is in contrast with the formalization
presented in these notes, where the conclusion $\forall x\phi_{1}$
is only reached when $\phi_{1}$ is the conclusion of a proof
$\pi_{1}$ such that $x\not\in\spec(\pi_{1})$, i.e. $x$ is not a {\em
specific variable} of $\pi_{1}$. This is in line with the common
mathematical practice of not claiming $\forall x\phi_{1}$ from
$\phi_{1}$ unless the variable $x$ is truly {\em arbitrary}. The
deduction theorem also appears in full generality as Theorem~4.9
page~34 in Donald W. Barnes, John M. Mack~\cite{AlgLog}. Their proof
relies on an induction argument based on the length of the proof
underlying the sequent $\Gamma\cup\{\phi\}\vdash\psi$. Having chosen
a free algebraic structure for our set \pvs, our own proof of the
deduction theorem will benefit from the clarity and power of
structural induction. Interestingly, the online course of Prof.
Arindama Singh of IIT Madras~\cite{Singh} also offers a proof of the
deduction theorem in full generality which implicitly makes use of
structural induction, despite having proofs defined as finite
sequences of formulas. This can be seen at the end of the lecture 39
available on YouTube via the link
\texttt{http://nptel.iitm.ac.in/courses/111106052/39}.
\index{deduction@Dedutction theorem}
\begin{theorem}\label{logic:the:FOPL:deduction}
Let $V$ be a set and $\Gamma\subseteq\pv$. For all $\phi,\psi\in\pv$
we have:
    \[
    \Gamma\cup\{\phi\}\vdash\psi\ \Leftrightarrow\ \Gamma\vdash
    (\phi\to\psi)
    \]
\end{theorem}
\begin{proof}
We shall first prove the implication $\Leftarrow$\,: so suppose $V$
is a set and $\Gamma\subseteq\pv$ is such that
$\Gamma\vdash(\phi\to\psi)$ where $\phi,\psi\in\pv$. In particular,
we have:
    \[
    \Gamma\cup\{\phi\}\vdash(\phi\to\psi)
    \]
and it is clear that $\Gamma\cup\{\phi\}\vdash\phi$. From the modus
ponens property of proposition~(\ref{logic:prop:FOPL:modus:ponens})
it follows that $\Gamma\cup\{\phi\}\vdash\psi$ as requested. We now
prove the reverse implication $\Rightarrow$\,: let $\Pi^{*}$ be the
set of all proofs $\pi\in\pvs$ with the property that for all
$\Gamma\subseteq\pv$ and $\phi,\psi\in\pv$ the following implication
holds:
    \[
    (\val(\pi)=\psi)\land(\hyp(\pi)\subseteq\Gamma\cup\{\phi\})\
    \Rightarrow\ \Gamma\vdash(\phi\to\psi)
    \]
First we shall prove that in order to complete the proof of this
theorem it is sufficient to show that $\Pi^{*}=\pvs$. So we assume
that $\Pi^{*}=\pvs$. Let $\Gamma\subseteq\pv$ and $\phi,\psi\in\pv$
be such that $\Gamma\cup\{\phi\}\vdash\psi$. We need to show that
$\Gamma\vdash(\phi\to\psi)$. From the assumption
$\Gamma\cup\{\phi\}\vdash\psi$ we obtain the existence of a proof
$\pi\in\pvs$ such that $\val(\pi) = \psi$ and
$\hyp(\pi)\subseteq\Gamma\cup\{\phi\}$. However, having assumed
$\Pi^{*} = \pvs$ it follows that $\pi$ is also an element of
$\Pi^{*}$. From $\val(\pi) = \psi$ and
$\hyp(\pi)\subseteq\Gamma\cup\{\phi\}$ we therefore conclude that
$\Gamma\vdash(\phi\to\psi)$ as requested. We shall now complete the
proof of this theorem by showing the equality $\Pi^{*}=\pvs$ using a
structural induction argument as per
theorem~(\ref{logic:the:proof:induction}) of
page~\pageref{logic:the:proof:induction}. Since \pv\ is a generator
of the algebra \pvs\ we first check that $\pv\subseteq\Pi^{*}$. So
suppose $\pi$ is a proof of the form $\pi=\phi_{1}$ for some
$\phi_{1}\in\pv$. We need to show that $\phi_{1}\in\Pi^{*}$. So let
$\Gamma\subseteq\pv$ and $\phi,\psi\in\pv$ be such that
$\phi_{1}=\val(\pi)=\psi$ and
$\{\phi_{1}\}=\hyp(\pi)\subseteq\Gamma\cup\{\phi\}$. We need to show
that $\Gamma\vdash(\phi\to\psi)$, that is to say
$\Gamma\vdash(\phi\to\phi_{1})$. We shall distinguish two cases:
first we assume that $\phi_{1}\in\Gamma$. In that case,
$\hyp(\pi)\subseteq\Gamma$ and $\pi$ is in fact a proof of
$\phi_{1}$ from $\Gamma$. So $\Gamma\vdash\phi_{1}$ and our desired
conclusion $\Gamma\vdash(\phi\to\phi_{1})$ follows immediately from
the simplification property of
proposition~(\ref{logic:prop:FOPL:simplification}). We now assume
that $\phi_{1}\not\in\Gamma$. From the inclusion
$\{\phi_{1}\}\subseteq\Gamma\cup\{\phi\}$ it follows that
$\phi_{1}=\phi$ and we therefore need to show that
$\Gamma\vdash(\phi_{1}\to\phi_{1})$ which follows immediately from
proposition~(\ref{logic:prop:FOPL:PimP}). This completes our proof
of $\pv\subseteq\Pi^{*}$. We shall now proceed with our induction
argument by showing that every proof $\pi\in\pvs$ of the form
$\pi=\axi\phi_{1}$ for some $\phi_{1}\in\pv$ is an element of
$\Pi^{*}$. So suppose $\pi=\axi\phi_{1}$ and let
$\Gamma\subseteq\pv$ and $\phi,\psi\in\pv$ be such that
$\val(\pi)=\psi$ and
$\emptyset=\hyp(\pi)\subseteq\Gamma\cup\{\phi\}$ (This inclusion of
course does not tell us anything). We need to show that
$\Gamma\vdash(\phi\to\psi)$. We shall distinguish two cases: first
we assume that $\phi_{1}\in\av$, i.e. that $\phi_{1}$ is an axiom.
Then we have $\phi_{1}=\val(\pi)=\psi$ and we need to show that
$\Gamma\vdash(\phi\to\phi_{1})$. However, having assumed
$\phi_{1}\in\av$ is an axiom of first order logic, we have
$\Gamma\vdash\phi_{1}$ and $\Gamma\vdash(\phi\to\phi_{1})$ follows
immediately from the simplification property of
proposition~(\ref{logic:prop:FOPL:simplification}). We now assume
that $\phi_{1}\not\in\av$. Then we have
$(\bot\to\bot)=\val(\pi)=\psi$ and we need to show that
$\Gamma\vdash(\phi\to(\bot\to\bot))$. However, we know that
$\Gamma\vdash(\bot\to\bot)$ is true by virtue of
proposition~(\ref{logic:prop:FOPL:PimP}) and
$\Gamma\vdash(\phi\to(\bot\to\bot))$ therefore follows from the
simplification property of
proposition~(\ref{logic:prop:FOPL:simplification}). We shall now
proceed with our induction argument by assuming $\pi\in\pvs$ is of
the form $\pi=\pi_{1}\pon\pi_{2}$ where both $\pi_{1}$ and $\pi_{2}$
are elements of $\Pi^{*}$. We need to show that $\pi$ is also an
element of $\Pi^{*}$. So let $\Gamma\subseteq\pv$ and
$\phi,\psi\in\pv$ be such that
$M(\val(\pi_{1}),\val(\pi_{2}))=\val(\pi)=\psi$ and
$\hyp(\pi_{1})\cup\hyp(\pi_{2})=\hyp(\pi)\subseteq\Gamma\cup\{\phi\}$,
where $M:\pv^{2}\to\pv$ is the modus ponens mapping of
definition~(\ref{logic:def:FOPL:modus:ponens}). We need to show that
$\Gamma\vdash(\phi\to\psi)$. We shall distinguish two cases: first
we assume that $\val(\pi_{2})$ is not of the form
$\val(\pi_{2})=\val(\pi_{1})\to\phi_{1}$ for any $\phi_{1}\in\pv$.
From definition~(\ref{logic:def:FOPL:modus:ponens}) it follows that
$M(\val(\pi_{1}),\val(\pi_{2}))= \bot\to\bot$ and consequently
$\psi=\bot\to\bot$. So we need to prove that
$\Gamma\vdash\phi\to(\bot\to\bot)$ which follows from the
simplification property of
proposition~(\ref{logic:prop:FOPL:simplification}) and the fact that
$\Gamma\vdash(\bot\to\bot)$, itself an outcome of
proposition~(\ref{logic:prop:FOPL:PimP}). We now assume that
$\val(\pi_{2})$ is of the form
$\val(\pi_{2})=\val(\pi_{1})\to\phi_{1}$ for some $\phi_{1}\in\pv$.
In this case we obtain $M(\val(\pi_{1}),\val(\pi_{2})) =
\phi_{1}=\psi$ so we need to prove that
$\Gamma\vdash(\phi\to\phi_{1})$. From the modus ponens property of
proposition~(\ref{logic:prop:FOPL:modus:ponens}) it is therefore
sufficient to show that $\Gamma\vdash(\phi\to\val(\pi_{1}))$ and
$\Gamma\vdash(\phi\to\val(\pi_{1}))\to(\phi\to\phi_{1})$. In fact,
from the Frege property of
proposition~(\ref{logic:prop:FOPL:Frege}),
$\Gamma\vdash(\phi\to\val(\pi_{1}))\to(\phi\to\phi_{1})$ is itself a
consequence of $\Gamma\vdash\phi\to(\val(\pi_{1})\to\phi_{1})$ so we
can concentrate on proving this last entailment together with
$\Gamma\vdash(\phi\to\val(\pi_{1}))$. First we show that
$\Gamma\vdash(\phi\to\val(\pi_{1}))$. Recall our assumption that the
proof $\pi_{1}$ is an element of $\Pi^{*}$, and from
$\hyp(\pi_{1})\cup\hyp(\pi_{2})\subseteq\Gamma\cup\{\phi\}$ we
obtain in particular $\hyp(\pi_{1})\subseteq\Gamma\cup\{\phi\}$.
Hence from the equality $\val(\pi_{1})=\val(\pi_{1})$ we conclude
immediately that $\Gamma\vdash(\phi\to\val(\pi_{1}))$ as requested.
We now show that $\Gamma\vdash\phi\to(\val(\pi_{1})\to\phi_{1})$.
Since $\val(\pi_{1})\to\phi_{1}=\val(\pi_{2})$ this is in fact the
same as showing $\Gamma\vdash(\phi\to\val(\pi_{2}))$ which is also a
simple consequence of the assumption $\pi_{2}\in\Pi^{*}$ and
$\hyp(\pi_{2})\subseteq\Gamma\cup\{\phi\}$. This completes our proof
in the case when $\pi\in\pvs$ is of the form
$\pi=\pi_{1}\pon\pi_{2}$. We now assume that $\pi$ is of the form
$\pi=\gen x\pi_{1}$ where $x\in V$ and $\pi_{1}$ is an element of
$\Pi^{*}$. We need to show that $\pi$ is also an element of
$\Pi^{*}$. So let $\Gamma\subseteq\pv$ and $\phi,\psi\in\pv$ be such
that $\val(\gen x\pi_{1})=\val(\pi)=\psi$ and
$\hyp(\pi_{1})=\hyp(\pi)\subseteq\Gamma\cup\{\phi\}$. We need to
show that $\Gamma\vdash(\phi\to\psi)$. We shall distinguish two
cases: first we assume that $x\in\spec(\pi_{1})$. In this case we
have $\val(\gen x\pi_{1})=\bot\to\bot$ and therefore
$\psi=\bot\to\bot$. So we need to prove that
$\Gamma\vdash\phi\to(\bot\to\bot)$ which we already know is true as
we have shown in this proof. We now assume that
$x\not\in\spec(\pi_{1})$. Once again, we need to show that
$\Gamma\vdash(\phi\to\psi)$ and we shall distinguish two further
cases: first we assume that $x\in\free(\phi)$. From
$x\not\in\spec(\pi_{1})=\free(\hyp(\pi_{1}))$ it follows that
$\phi\not\in\hyp(\pi_{1})$. Hence, from
$\hyp(\pi)=\hyp(\pi_{1})\subseteq\Gamma\cup\{\phi\}$ we obtain
$\hyp(\pi)\subseteq\Gamma$, which together with $\val(\pi)=\psi$
imply that $\Gamma\vdash\psi$. So $\Gamma\vdash(\phi\to\psi)$
follows immediately from the simplification property of
proposition~(\ref{logic:prop:FOPL:simplification}). We now assume
that $x\not\in\free(\phi)$. From $x\not\in\spec(\pi_{1})$ it follows
that $\val(\gen x\pi_{1})=\forall x\val(\pi_{1})$ and consequently
$\psi=\forall x\val(\pi_{1})$. Hence we need to show that
$\Gamma\vdash(\phi\to\forall x\val(\pi_{1}))$. However, since
$x\not\in\free(\phi)$, from the quantification property of
proposition~(\ref{logic:prop:FOPL:quantification}), it is sufficient
to prove that $\Gamma\vdash\forall x(\phi\to\val(\pi_{1}))$. From
$\hyp(\pi_{1})\subseteq\Gamma\cup\{\phi\}$, defining
$\Gamma^{*}=\hyp(\pi_{1})\setminus\{\phi\}$ we obtain
$\Gamma^{*}\subseteq\Gamma$. It is therefore sufficient to prove
that $\Gamma^{*}\vdash\forall x(\phi\to\val(\pi_{1}))$. From
$x\not\in\spec(\pi_{1})=\free(\hyp(\pi_{1}))$ it follows that
$x\not\in\free(\Gamma^{*})$. Using the generalization property of
proposition~(\ref{logic:prop:FOPL:generalization}), it is therefore
sufficient to prove that $\Gamma^{*}\vdash(\phi\to\val(\pi_{1}))$.
Now recall our assumption that the proof $\pi_{1}$ is an element of
$\Pi^{*}$. From $\hyp(\pi_{1})\subseteq\Gamma^{*}\cup\{\phi\}$ and
$\val(\pi_{1})=\val(\pi_{1})$ we obtain immediately
$\Gamma^{*}\vdash(\phi\to\val(\pi_{1}))$ as requested. This
completes our induction argument.
\end{proof}
