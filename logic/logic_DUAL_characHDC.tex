The theorem which follows is yet another indirect consequence of
Lindenbaum's lemma~(\ref{logic:lemma:FOPL:semantics:lindenbaum}). It
provides us with a characterization of the Hilbert deductive
congruence in terms of the dual space \pvd. The result has nothing
to do with model theory: We started from a consequence relation
$\vdash\,\subseteq{\cal P}(\pv)\times\pv$ from which both a
congruence and a dual space were defined. This congruence turns out
to be fully characterized by this dual space. Obviously the question
should be asked: is the result specific to the Hilbert deductive
congruence, or are there other congruences on \pv\ for which a {\em
natural consequence relation and dual space} can be defined, leading
up to the following characterization?

\index{congruence@Charact. of deductive congruence}
\begin{theorem}\label{logic:the:FOPL:semantics:HDC:charac}
Let $V$ be a set and $\equiv$ be the Hilbert deductive congruence on
\pv. Then for all formulas $\phi,\psi\in\pv$ we have the
equivalence:
    \[
    \phi\equiv\psi\ \Leftrightarrow\ \forall v\in\pvd\ ,\
    v(\phi)=v(\psi)
    \]
\end{theorem}
\begin{proof}
The implication $\Rightarrow$ follows from
proposition~(\ref{logic:prop:FOPL:semantics:stronger:congruence}).
So we now prove $\Leftarrow$\,: let $\phi,\psi\in\pv$ such that
$v(\phi)=v(\psi)$ for all $v\in\pvd$. We need to show that
$\phi\equiv\psi$. By symmetry, it is sufficient to show that
$\phi\leq\psi$ i.e. that we have $\vdash(\phi\to\psi)$. Using
theorem~(\ref{logic:the:FOPL:semantics:syn:equiv:sem}) it is
therefore sufficient to prove that $\vDash(\phi\to\psi)$. In other
words, from
proposition~(\ref{logic:prop:FOPL:semantics:validity:charac}) it is
sufficient to show that $v(\phi\to\psi)=1$ for all $v\in\pvd$.
However, given $v\in\pvd$ we have:
    \[
    v(\phi\to\psi)=v(\phi)\to v(\psi)=v(\phi)\to v(\phi)=1
    \]
\end{proof}

From theorem~(\ref{logic:the:FOPL:semantics:syn:equiv:sem}) of
page~\pageref{logic:the:FOPL:semantics:syn:equiv:sem} we have the
equivalence $\vdash\phi\ \Leftrightarrow\ \vDash\phi$. In other
words, a formula is provable \ifand\ it is valid. In fact as we
shall now see, a formula is provable \ifand\ it is logically
equivalent to $\bot\to\bot$.

\begin{prop}\label{logic:prop:FOPL:semantics:provable:charac}
Let $V$ be a set. For all $\phi\in\pv$ we have the equivalence:
    \[
    \vdash\phi\ \Leftrightarrow\ \phi\equiv(\bot\to\bot)
    \]
where $\equiv$ is the Hilbert deductive congruence on \pv.
\end{prop}
\begin{proof}
Using theorem~(\ref{logic:the:FOPL:semantics:HDC:charac}), the
statement $\phi\equiv(\bot\to\bot)$ is equivalent to $v(\phi)=1$ for
all $v\in\pvd$. Using
proposition~(\ref{logic:prop:FOPL:semantics:validity:charac}), this
is in turn equivalent to $\vDash\phi$, which is itself equivalent to
$\vdash\phi$ by virtue of
theorem~(\ref{logic:the:FOPL:semantics:syn:equiv:sem}).
\end{proof}

In definition~(\ref{logic:def:FOPL:semantics:valuation}), we defined
the dual space \pvd\ in terms of the consequence relation $\vdash$
by imposing the {\em soundness} property $\vdash\phi\ \Rightarrow\
v(\phi)=1$. The following theorem shows that \pvd\ could equally
have been defined directly in terms of the Hilbert deductive
congruence $\equiv$\,, induced by the consequence relation $\vdash$.
In other words, we do not need to have a consequence relation in
order to define a dual space: given a congruence $\sim$ on \pv, we
can define an associated dual space \pvd\ as the set of all
propositional valuations $v$ which are compatible with the
congruence $\sim$\,, i.e. for which given $\phi,\psi\in\pv$\,:
    \[
    \phi\sim\psi\ \Rightarrow\ v(\phi)=v(\psi)
    \]
We can then investigate whether this dual space \pvd\ gives rise to
a characterization of the congruence~$\sim$ which is similar to that
of theorem~(\ref{logic:the:FOPL:semantics:HDC:charac}).

\index{congruence@Deductive congruence and dual} \index{dual@Dual
and deductive congruence}
\begin{theorem}\label{logic:the:FOPL:semantics:valuation:charac}
Let $V$ be a set and $v:\pv\to 2$ be a map. Then $v$ is a valuation
\ifand\ it is a propositional valuation such that for all
$\phi,\psi\in\pv$\,:
    \[
    \phi\equiv\psi\ \Rightarrow\ v(\phi)=v(\psi)
    \]
i.e. such that $v$ is compatible with the Hilbert deductive
congruence $\equiv$ on \pv.
\end{theorem}
\begin{proof}
First we show the 'only if' part: so we assume that $v:\pv\to 2$ is
a valuation. From
definition~(\ref{logic:def:FOPL:semantics:valuation}) we see that
$v(\bot)=0$ and $v(\phi_{1}\to\phi_{2})=v(\phi_{1})\to v(\phi_{2})$
for all $\phi_{1},\phi_{2}\in\pv$. Comparing with
definition~(\ref{logic:def:FOPL:propcong:prop:valuation}) it follows
that $v$ is indeed a propositional valuation. Furthermore, if
$\phi,\psi\in\pv$ are such that $\phi\equiv\psi$, it is clear from
theorem~(\ref{logic:the:FOPL:semantics:HDC:charac}) that
$v(\phi)=v(\psi)$. So we now prove the 'if' part: we assume that
$v:\pv\to 2$ is a propositional valuation which is compatible with
the Hilbert deductive congruence, i.e. such that $v(\phi)=v(\psi)$
whenever we have $\phi\equiv\psi$. We need to show that $v$ is a
valuation. Being a propositional valuation, we already know that
$v(\bot)=0$ and $v(\phi_{1}\to\phi_{2})=v(\phi_{1})\to v(\phi_{2})$
for all $\phi_{1},\phi_{2}\in\pv$. It remains to show the
implication $\vdash\phi\ \Rightarrow\ v(\phi)=1$. So we assume that
$\phi\in\pv$ is such that $\vdash\phi$. We need to show that
$v(\phi)=1$. However, from $\vdash\phi$ and
proposition~(\ref{logic:prop:FOPL:semantics:provable:charac}) we
obtain $\phi\equiv(\bot\to\bot)$. By assumption, it follows that
$v(\phi)=v(\bot\to\bot)= v(\bot)\to v(\bot)=0\to 0=1$ as requested.
\end{proof}
