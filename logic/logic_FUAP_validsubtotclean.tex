There is not much point substituting variables in a proof
$\pi\in\pvs$ according to some map $\sigma:V\to W$, unless the proof
$\sigma(\pi)$ has the properties everyone would want it to have.
From proposition~(\ref{logic:prop:FUAP:substitution:hypothesis}) we
have $\hyp(\sigma(\pi))=\sigma(\hyp(\pi))$. So the set of hypothesis
of $\sigma(\pi)$ is nicely what we expect it to be, without any
assumption on the proof $\pi$ or substitution $\sigma$. However, we
also crucially want to control the conclusion of $\sigma(\pi)$ with
the equality $\val\circ\sigma(\pi)=\sigma\circ\val(\pi)$. As
previously discussed, we cannot hope this to be true unless axioms
of the proof $\pi$ remain axioms of first order logic after
substitution by $\sigma$. By virtue of
lemma~(\ref{logic:lemma:FUAP:substitution:axiom}), one way to
achieve this is to impose that $\sigma$ be valid for every axiom of
$\pi$. However, we also need to make sure any use of the
generalization rule of inference in the proof $\pi$ remains
legitimate after substitution by $\sigma$. Specifically, if $\gen
x\pi_{1}$ is a sub-proof of $\pi$, the variable $\sigma(x)$ should
remain an arbitrary variable of the proof $\sigma(\pi_{1})$ so that
$\gen\sigma(x)\sigma(\pi_{1})$ can be inferred legitimately. In
other words, no free variable of an hypothesis of $\pi_{1}$ should
be {\em captured} by the substitution $\sigma$. In
definition~(\ref{logic:def:FUAP:validsubproof:validsub}) we
introduced the notion of {\em valid substitution} for proofs with
two considerations in mind: on the one hand we wanted to impose
conditions on $\sigma$ which are strong enough to ensure the
conclusion of the proof $\sigma(\pi)$ is what we expect. On the
other hand, we wanted the theory of capture-avoiding substitutions
for proofs to be formally as close as possible if not identical, to
the theory of capture-avoiding substitutions for formulas, even if
this meant imposing conditions on $\sigma$ which are slightly
stronger than necessary. As we shall see, a huge benefit will be
derived from the formal proximity of the notions of validity for
proofs and formulas. We shall be able to define a minimal transform
and substitution congruence on \pvs, and prove the existence of
essential substitutions for proofs in
theorem~(\ref{logic:the:FUAP:esssubst:existence}) of
page~\pageref{logic:the:FUAP:esssubst:existence}. This will in turn
lead to the substitution
theorem~(\ref{logic:the:FUAP:substitutiontheorem:main}) of
page~\pageref{logic:the:FUAP:substitutiontheorem:main}. Now going
back to the discussion at hand, the hope is that provided $\sigma$
is valid for $\pi$, the conclusion of $\sigma(\pi)$ is the right
one. However, we already saw that imposing a condition on $\sigma$
was not enough. The example of $\pi=(x\in x)\pon[(y\in y)\to\bot]$
with $x\neq y$ and $\sigma=[y/x]$ shows that things can go wrong
despite $\sigma$ being valid for $\pi$. Since $\pi$ is derived from
a flawed application of modus ponens, its conclusion is
$\val(\pi)=\bot\to\bot$, while the conclusion of $\sigma(\pi)=(y\in
y)\pon[(y\in y)\to\bot]$ is $\val\circ\sigma(\pi)=\bot$. In
definition~(\ref{logic:def:FUAP:clean:clean:proof}) we introduced
the notion of {\em totally clean proof} so as to focus only on those
proofs of \pvs\ which do not contain any flawed application of rules
of inference or axiom invocation. Unless a proof is totally clean,
we cannot say anything sensible about it. Granted its conclusion is
provable from its hypothesis, but this is as far as it goes. So we
need $\pi$ to be totally clean and $\sigma$ to be valid for $\pi$,
in which case $\sigma(\pi)$ becomes interesting:

\index{clean@Image of totally clean proof}
\begin{prop}\label{logic:prop:FUAP:validsubtotclean:valuation:commute}
Let $V,W$ be sets and $\sigma:V\to W$ be a map. Let $\pi\in\pvs$ be
totally clean and $\sigma$ be valid for $\pi$. Then $\sigma(\pi)$ is
totally clean and:
    \begin{equation}\label{logic:eqn:FUAP:validsub:valuation:commute:1}
    \val\circ\sigma(\pi)=\sigma\circ\val(\pi)
    \end{equation}
\end{prop}
\begin{proof}
For all $\pi\in\pvs$ we need to show the following implication:
    \[
    (\mbox{$\pi$ t-clean})\land(\mbox{$\sigma$ valid for $\pi$})\ \Rightarrow
    \ (\mbox{$\sigma(\pi)$ t-clean})
    \land(\,\mbox{eq.~(\ref{logic:eqn:FUAP:validsub:valuation:commute:1})}\,)
    \]
We shall do so with a structural induction using
theorem~(\ref{logic:the:proof:induction}) of
page~\pageref{logic:the:proof:induction}. First we assume that
$\pi=\phi$ for some $\phi\in\pv$. From
definition~(\ref{logic:def:FUAP:clean:clean:proof}), $\pi$ is always
totally clean in this case. So we assume that $\sigma$ is valid for
$\pi$, and we need to show that $\sigma(\pi)$ is totally clean
together with
equation~(\ref{logic:eqn:FUAP:validsub:valuation:commute:1}). From
definition~(\ref{logic:def:FUAP:substitution:substitution}) we have
$\sigma(\pi)=\sigma(\phi)\in{\bf P}(W)$ and consequently
$\sigma(\pi)$ is totally clean. Furthermore, we have
$\val(\sigma(\pi))=\val(\sigma(\phi))=\sigma(\phi)=\sigma(\val(\pi))$
which shows that
equation~(\ref{logic:eqn:FUAP:validsub:valuation:commute:1}) is
true. This completes the case when $\pi=\phi$ for which the
assumption of validity of $\sigma$ for $\pi$ was unnecessary. We now
assume that $\pi=\axi\phi$ for some $\phi\in\pv$. We further assume
that $\pi$ is totally clean and $\sigma$ is valid for $\pi$. we need
to show that $\sigma(\pi)$ is totally clean together with
equation~(\ref{logic:eqn:FUAP:validsub:valuation:commute:1}). From
definition~(\ref{logic:def:FUAP:clean:clean:proof}), having assumed
that $\pi$ is totally clean we obtain $\phi\in\av$, i.e. $\phi$ is a
legitimate axiom. Having assumed that $\sigma$ is valid for $\pi$,
in particular from
proposition~(\ref{logic:prop:FUAP:validsubproof:recursion:axiom})
$\sigma$ is valid for $\phi$. It follows from
lemma~(\ref{logic:lemma:FUAP:substitution:axiom}) that
$\sigma(\phi)\in{\bf A}(W)$. Hence, using
definition~(\ref{logic:def:FUAP:clean:clean:proof}) once more we see
that $\sigma(\pi)=\axi\sigma(\phi)$ is totally clean. Furthermore
from definition~(\ref{logic:def:FOPL:proof:valuation}) we have
$\val(\sigma(\pi))=\sigma(\phi)=\sigma(\val(\pi))$ and it follows
that equation~(\ref{logic:eqn:FUAP:validsub:valuation:commute:1}) is
true. So we now assume that $\pi=\pi_{1}\pon\pi_{2}$ where
$\pi_{1},\pi_{2}\in\pvs$ are proofs which satisfy our implication.
We need to show the same is true of $\pi$. So we assume that $\pi$
is totally clean and furthermore that $\sigma$ is valid for $\pi$.
We need to show that $\sigma(\pi)$ is totally clean and
equation~(\ref{logic:eqn:FUAP:validsub:valuation:commute:1}) is
true. However, using
proposition~(\ref{logic:prop:FUAP:clean:modus:ponens}) we see that
both $\pi_{1}$ and $\pi_{2}$ are totally clean and the following
equality holds:
    \begin{equation}\label{logic:eqn:FUAP:validsub:valuation:commute:2}
    \val(\pi_{2})=\val(\pi_{1})\to\val(\pi)
    \end{equation}
Furthermore, using
proposition~(\ref{logic:prop:FUAP:validsubproof:recursion:pon}) we
see that $\sigma$ is valid for $\pi_{1}$ and $\pi_{2}$. Having
assumed our induction hypothesis holds for $\pi_{1},\pi_{2}$ it
follows that $\sigma(\pi_{1})$ and $\sigma(\pi_{2})$ are totally
clean and
equation~(\ref{logic:eqn:FUAP:validsub:valuation:commute:1}) is true
for $\pi_{1}$, $\pi_{2}$. So let us prove that $\sigma(\pi)$ is
totally clean: since
$\sigma(\pi)=\sigma(\pi_{1})\pon\,\sigma(\pi_{2})$, from
proposition~(\ref{logic:prop:FUAP:clean:modus:ponens}) it is
sufficient to prove that $\sigma(\pi_{1})$ and $\sigma(\pi_{2})$ are
totally clean and:
    \begin{equation}\label{logic:eqn:FUAP:validsub:valuation:commute:3}
    \val(\sigma(\pi_{2}))=\val(\sigma(\pi_{1}))\to\val(\sigma(\pi))
    \end{equation}
We already know that $\sigma(\pi_{1})$ and $\sigma(\pi_{2})$ are
totally clean so we only need to focus on
equation~(\ref{logic:eqn:FUAP:validsub:valuation:commute:3}).
However, applying $\sigma:\pv\to{\bf P}(W)$ on both sides of
equation~(\ref{logic:eqn:FUAP:validsub:valuation:commute:2}) we
obtain
$\sigma(\val(\pi_{2}))=\sigma(\val(\pi_{1}))\to\sigma(\val(\pi))$,
which is:
    \begin{equation}\label{logic:eqn:FUAP:validsub:valuation:commute:4}
    \val(\sigma(\pi_{2}))=\val(\sigma(\pi_{1}))\to\sigma(\val(\pi))
    \end{equation}
since $\pi_{1}$ and $\pi_{2}$ satisfy
equation~(\ref{logic:eqn:FUAP:validsub:valuation:commute:1}).
Comparing~(\ref{logic:eqn:FUAP:validsub:valuation:commute:3})
with~(\ref{logic:eqn:FUAP:validsub:valuation:commute:4}) we only
need to show that $\val(\sigma(\pi))=\sigma(\val(\pi))$, which is
showing that~(\ref{logic:eqn:FUAP:validsub:valuation:commute:1}) is
true and which we have to do anyway. In fact, this follows
immediately
from~(\ref{logic:eqn:FUAP:validsub:valuation:commute:4})\,:
    \begin{eqnarray*}
    \val(\sigma(\pi))&=&\val(\sigma(\pi_{1}\pon\pi_{2}))\\
    &=&\val(\sigma(\pi_{1})\pon\,\sigma(\pi_{2}))\\
    \mbox{def.~(\ref{logic:def:FOPL:proof:valuation})}\ \rightarrow
    &=&M(\,\val(\sigma(\pi_{1}))\,,\,\val(\sigma(\pi_{2}))\,)\\
    \mbox{(\ref{logic:eqn:FUAP:validsub:valuation:commute:4})}\ \rightarrow
    &=&M(\,\val(\sigma(\pi_{1}))\,,\,\val(\sigma(\pi_{1}))\to\sigma(\val(\pi))\,)\\
    \mbox{def.~(\ref{logic:def:FOPL:modus:ponens})}\ \rightarrow
    &=&\sigma(\val(\pi))
    \end{eqnarray*}
This completes the case when $\pi=\pi_{1}\pon\pi_{2}$. So we now
assume that $\pi=\gen x\pi_{1}$ where $x\in V$ and $\pi_{1}\in\pvs$
satisfies our implication. We need to show the same is true of
$\pi$. So we assume that $\pi$ is totally clean and $\sigma$ is
valid for $\pi$. We need to show that $\sigma(\pi)$ is totally clean
and equation~(\ref{logic:eqn:FUAP:validsub:valuation:commute:1}) is
true. However from
proposition~(\ref{logic:prop:FUAP:clean:generalization}), $\pi_{1}$
is totally clean and $x\not\in\spec(\pi_{1})$. Furthermore, using
proposition~(\ref{logic:prop:FUAP:validsubproof:recursion:gen}) we
see that $\sigma$ is valid for $\pi_{1}$ and for all $u\in V$\,:
    \begin{equation}\label{logic:eqn:FUAP:validsub:valuation:commute:5}
    u\in\free(\gen x\pi_{1})\ \Rightarrow\
    \sigma(u)\neq\sigma(x)
    \end{equation}
Having assumed our induction hypothesis holds for $\pi_{1}$ it
follows that $\sigma(\pi_{1})$ is totally clean and
equation~(\ref{logic:eqn:FUAP:validsub:valuation:commute:1}) is true
for $\pi_{1}$. So let us prove that $\sigma(\pi)$ is totally clean:
since $\sigma(\pi)=\gen\sigma(x)\sigma(\pi_{1})$, from
proposition~(\ref{logic:prop:FUAP:clean:generalization}) it is
sufficient to prove that $\sigma(\pi_{1})$ is totally clean and
$\sigma(x)\not\in\spec(\sigma(\pi_{1}))$. We already know that
$\sigma(\pi_{1})$ is totally clean so we only need to show that
$\sigma(x)\not\in\spec(\sigma(\pi_{1}))$. So suppose to the contrary
that $\sigma(x)\in\spec(\sigma(\pi_{1}))$. From
proposition~(\ref{logic:prop:FUAP:freevar:substitution}) we have
$\spec(\sigma(\pi_{1}))\subseteq\sigma(\spec(\pi_{1}))$ and it
follows that $\sigma(x)=\sigma(u)$ for some $u\in\spec(\pi_{1})$.
Having established that $x\not\in\spec(\pi_{1})$ we must have $u\neq
x$ and so $u\in\spec(\pi_{1})\setminus\{x\}$. Using
proposition~(\ref{logic:prop:FUAP:freevarproof:spec:free}), since
$\pi_{1}$ is totally clean we have
$\spec(\pi_{1})\subseteq\free(\pi_{1})$ and it follows that
$u\in\free(\gen x\pi_{1})$ while $\sigma(u)=\sigma(x)$. This
contradicts the
implication~(\ref{logic:eqn:FUAP:validsub:valuation:commute:5})
above. So we have proved that $\sigma(\pi)$ is totally clean and it
remains to show~(\ref{logic:eqn:FUAP:validsub:valuation:commute:1}),
which goes as follows:
    \begin{eqnarray*}
    \val(\sigma(\pi))&=&\val(\sigma(\gen x\pi_{1}))\\
    \mbox{def.~(\ref{logic:def:FUAP:substitution:substitution})}\ \rightarrow
    &=&\val(\gen\sigma(x)\sigma(\pi_{1}))\\
    \sigma(x)\not\in\spec(\sigma(\pi_{1}))\ \rightarrow
    &=&\forall\sigma(x)\val(\sigma(\pi_{1}))\\
    \mbox{(\ref{logic:eqn:FUAP:validsub:valuation:commute:1}) true for $\pi_{1}$}\ \rightarrow
    &=&\forall\sigma(x)\sigma(\val(\pi_{1}))\\
    \mbox{def.~(\ref{logic:def:substitution})}\ \rightarrow
    &=&\sigma(\forall x\val(\pi_{1}))\\
    x\not\in\spec(\pi_{1})\ \rightarrow
    &=&\sigma(\val(\gen x\pi_{1}))\\
    &=&\sigma(\val(\pi))\\
    \end{eqnarray*}
\end{proof}

So we have made some reasonable progress in our quest towards the
substitution theorem~(\ref{logic:the:FUAP:substitutiontheorem:main})
of page~\pageref{logic:the:FUAP:substitutiontheorem:main} and the
objective to carry over the sequent $\Gamma\vdash\phi$ into
$\sigma(\Gamma)\vdash\sigma(\phi)$. Given reasonable assumptions on
the proof $\pi$ and substitution $\sigma$,
proposition~(\ref{logic:prop:FUAP:validsubtotclean:valuation:commute})
shows that $\sigma(\pi)$ has the right property. Unfortunately, this
will not take us very far. Given a sequent $\Gamma\vdash\phi$ with
underlying proof $\pi$, even if we choose $\pi$ to be totally clean
we have no way to guarantee the validity of $\sigma$ for $\pi$. In
effect,
proposition~(\ref{logic:prop:FUAP:validsubtotclean:valuation:commute})
is pretty useless. However, we know from
proposition~(\ref{logic:prop:FUAP:validsubproof:injective}) that an
injective map $\sigma:V\to W$ will always be valid for any proof
$\pi\in\pvs$. So we can prove a version of the substitution theorem
in the injective case, which is the purpose of
proposition~(\ref{logic:prop:FUAP:validsubtotclean:sequent}) below.
This is of course a very weak and temporary result. Dull, boring and
expected. Is it ever the case that an injective map will fail to
{\em carry over} properties nicely from one space to the other?
Actually, coming to think of it, there is a case when injectivity is
not enough to make a result trivial: suppose $\Gamma\subseteq\pv$ is
a consistent subset, i.e. such that the sequent $\Gamma\vdash\bot$
is false. The fact that $\sigma(\Gamma)$ should also be consistent
when $\sigma:V\to W$ is injective we feel is not trivial at all. It
seems particularly hard when $V$ is a finite set, a point we shall
need to resolve later.

\index{substitution@Substitution theorem injective}
\begin{prop}\label{logic:prop:FUAP:validsubtotclean:sequent}
Let $V$, $W$ be sets and $\sigma:V\to W$ be an injective map. Then
for every subset $\Gamma\subseteq\pv$ and formula $\phi\in\pv$ we
have the implication:
    \[
    \Gamma\vdash\phi\ \Rightarrow\ \sigma(\Gamma)\vdash\sigma(\phi)
    \]
where $\sigma:\pv\to{\bf P}(W)$ also denotes the associated
substitution mapping.
\end{prop}
\begin{proof}
Let $\pi\in\pvs$ be a proof underlying the sequent
$\Gamma\vdash\phi$. Without loss of generality, from
proposition~(\ref{logic:prop:FUAP:clean:counterpart}) we may assume
that $\pi$ is totally clean. Since $\sigma$ is an injective map,
from proposition~(\ref{logic:prop:FUAP:validsubproof:injective}) it
is valid for $\pi$. Using
proposition~(\ref{logic:prop:FUAP:validsubtotclean:valuation:commute})
it follows that $\val\circ\sigma(\pi)=\sigma\circ\val(\pi)$. Hence
we see that $\val(\sigma(\pi))=\sigma(\phi)$. Furthermore from
proposition~(\ref{logic:prop:FUAP:substitution:hypothesis}) we
obtain $\hyp(\sigma(\pi))=\sigma(\hyp(\pi))\subseteq\sigma(\Gamma)$.
So we have found $\pi^{*}=\sigma(\pi)\in{\bf\Pi}(W)$ such that
$\val(\pi^{*})=\sigma(\phi)$ together with
$\hyp(\pi^{*})\subseteq\sigma(\Gamma)$. It follows that
$\sigma(\Gamma)\vdash\sigma(\phi)$ as requested.
\end{proof}
