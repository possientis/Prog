In an earlier section, we claimed that most universal algebras were
not free. The following theorem arguably provides the most
convincing evidence of this fact. Let $X$ be a group viewed as a
universal algebra of type $\alpha=\{(0,0),(1,1),(2,2)\}$, where the
product $T(2,2):X^{2}\to X$ is denoted $\otimes$. Then the following
theorem shows that $X$ cannot be a free universal algebra of type
$\alpha$ unless:
    \[
    x_{1}\otimes y_{1} = x_{2}\otimes y_{2}\ \Rightarrow\ x_{1}=x_{2}\mbox{\ and\ }y_{1}=y_{2}
    \]
for all $x_{1}$, $x_{2}$, $y_{1}$ and $y_{2}\in X$. This is a pretty
strong condition. In fact, we can easily show that $X$ can never be
a free universal algebra of type $\alpha$. Indeed, let $e=T(0,0)(0)$
be the identity element of $X$. Then $e = e\otimes e$, or
equivalently:
    \[
    T(0,0)(0) = T(2,2)(e,e)
    \]
The following theorem shows this cannot be true in a free universal
algebra.
\index{free@Unique readability in free algebra}
\begin{theorem}\label{logic:the:unique:representation}
Let $X$ be a free universal algebra of type $\alpha$ with free
generator $X_{0}\subseteq X$ . Then for all $y\in X$, one and only
one of the following is the case:
    \begin{eqnarray*}
    (i)&y\in X_{0}&\\
    (ii)&y=f(x)&\ ,\ f\in\alpha\ ,\ x\in X^{\alpha(f)}
    \end{eqnarray*}
Furthermore, the representation $(ii)$ is unique, that is:
    \[
    f(x)=f'(x')\ \Rightarrow\ f=f'\mbox{\ and\ }x=x'
    \]
for all $f,f'\in\alpha$, $x\in X^{\alpha(f)}$ and $x'\in X^{\alpha(f')}$.
\end{theorem}
\begin{proof}
We shall first prove the theorem for a particular case of free
universal algebra of type $\alpha$. Instead of looking at $X$
itself, we shall first consider the free universal algebra $Y$ with
free generator $Y_{0}=\{(0,x): x\in X_{0}\}$ as described in
proposition~(\ref{logic:prop:construction}). So let $y\in Y$ and
suppose that $y\not\in Y_{0}$. There exists $n\in\N$ such that $y\in
Y_{n+1}$. Let $n$ denote the smallest of such integers. Then $y\in
Y_{n+1}\setminus Y_{n}\subseteq\bar{Y}_{n}$. Hence, there exist
$f\in\alpha$ and $x\in Y_{n}^{\alpha(f)}$ such that $y=(1,(f,x))$.
However by definition, we have $f(x)=(1,(f,x))$. Since
$Y_{n}^{\alpha(f)}\subseteq Y^{\alpha(f)}$, we have found
$f\in\alpha$ and $x\in Y^{\alpha(f)}$ such that $y=f(x)$. This shows
that $(ii)$ is satisfied whenever $(i)$ is not. We shall now prove
that $(i)$ and $(ii)$ cannot occur simultaneously. Indeed if
$y=f(x)$ for some $f\in\alpha$ and $x\in Y^{\alpha(f)}$, then
$y=(1,(f,x))$. If $y$ is also an element of $Y_{0}$ then $y=(0,x')$
for some $x'\in X_{0}$, which is a contradiction as $0\neq 1$. So we
have proved that one and one only of $(i)$ and $(ii)$ must occur. It
remains to show that the representation $(ii)$ is unique. So suppose
$f(x)=f'(x')$ for some $f,f'\in\alpha$, $x\in Y^{\alpha(f)}$ and
$x'\in Y^{\alpha(f')}$. Then we have $(1,(f,x))=(1,(f',x'))$ and it
follows immediately that $f=f'$ and $x=x'$. So we have proved the
theorem in the case of the free universal algebra $Y$. We shall now
prove the theorem for $X$. Consider the bijection $j:X_{0}\to Y_{0}$
defined by $j(x)=(0,x)$. Then we also have $j:X_{0}\to Y$ and since
$X_{0}$ is a free generator of $X$, the map $j$ can be uniquely
extended into a morphism $g:X\to Y$. Similarly, the map
$j^{-1}:Y_{0}\to X$ can be uniquely extended into a morphism
$g':Y\to X$. Furthermore, since $g'\circ g:X\to X$ is a morphism
such that:
    \[
    (g'\circ g)_{|X_{0}}=g'\circ j=g'_{|Y_{0}}\circ j=j^{-1}\circ j = (id_{X})_{|X_{0}}
    \]
we conclude by uniqueness that $g'\circ g=id_{X}$ and similarly
$g\circ g'=id_{Y}$. This shows that $g:X\to Y$ and $g':Y\to X$ are
in fact isomorphisms which are inverse of each other. Of course we
already knew that $X$ and $Y$ were isomorphic from the existence of
the bijection $j:X_{0}\to Y_{0}$. But we now have a particular
isomorphism $g:X\to Y$ which is such that $g_{|X_{0}}=j$. This
should allow us to complete the proof of the theorem. So suppose
$y\in X$ and $y\not\in X_{0}$. Let $y'=g(y)\in Y$. Since the theorem
is true for $Y$, either $(i)$ or $(ii)$ is true for $y'$. We claim
that $y'\in Y_{0}$ is impossible, as otherwise:
    \[
    y=g'\circ g(y)=g'(y')=g'_{|Y_{0}}(y')=j^{-1}(y')\in X_{0}
    \]
which contradicts the assumption $y\not\in X_{0}$. It follows that
$(ii)$ is true for $y'$, i.e. $y'=f(x')$ for some $f\in\alpha$ and
$x'\in Y^{\alpha(f)}$. Taking $x=g'(x')\in X^{\alpha(f)}$, since
$g':Y\to X$ is a morphism we obtain:
    \[
    y = g'(y')=g'\circ f(x')=f\circ g'(x')=f(x)
    \]
which shows that $(ii)$ is true for $y$. We now prove that $(i)$ and
$(ii)$ cannot occur simultaneously. Suppose that $y\in X_{0}$ and
$y=f(x)$ for some $f\in\alpha$ and $x\in X^{\alpha(f)}$. Taking
$y'=g(y)\in Y$  and $x'=g(x)\in Y^{\alpha(f)}$, we obtain:
    \[
    y' = g(y)=g_{|X_{0}}(y)=j(y)\in Y_{0}
    \]
and furthermore:
    \[
    y' = g(y)=g\circ f (x)=f\circ g(x)=f(x')
    \]
which shows that $(i)$ and $(ii)$ are simultaneously true for $y'\in
Y$, which is a contradiction. To complete the proof of the theorem,
it remains to show that the representation $(ii)$ is unique. So
suppose $f(x)=f'(x')$ for some $f,f'\in\alpha$, $x\in X^{\alpha(f)}$
and $x'\in X^{\alpha(f')}$. We obtain:
    \[
    f(g(x)) = f\circ g(x)=g\circ f(x)=g\circ f'(x') = f'(g(x'))
    \]
From the uniqueness of the representation $(ii)$ in $Y$, it follows
that $f=f'$ and $g(x)=g(x')\in Y^{\alpha(f)}$. Hence, for all
$i\in\alpha(f)$:
    \[
    g(x(i))=g(x)(i)=g(x')(i)=g(x'(i))
    \]
and since $g:X\to Y$ is injective, we conclude that $x(i)=x'(i)$.
This being true for all $i\in\alpha(f)$, we have proved that $x=x'$.
\end{proof}
\begin{prop}
A free universal algebra of type $\alpha$ has a unique free generator.
\end{prop}
\begin{proof}
Let $X$ be a free universal algebra of type $\alpha$. Suppose
$X_{0}\subseteq X$ and $Y_{0}\subseteq X$ are both free generators
of $X$. We need to show that $X_{0}=Y_{0}$. First we show that
$X_{0}\subseteq Y_{0}$. So let $y\in X_{0}$. Applying
theorem~(\ref{logic:the:unique:representation}) of
page~\pageref{logic:the:unique:representation} to $X$ with free
generator $Y_{0}$ we see that if $y\not\in Y_{0}$ there exists
$f\in\alpha$ and $x\in X^{\alpha(f)}$ such that $y=f(x)$. But this
contradicts theorem~(\ref{logic:the:unique:representation}) applied
to $X$ with free generator $X_{0}$ since $y\in X_{0}$, and $(i)$ and
$(ii)$ cannot occur simultaneously. it follows that $y\in Y_{0}$ and
we have proved that $X_{0}\subseteq Y_{0}$. The reverse inclusion is
proved in a similar fashion.
\end{proof}
